\documentclass[11pt]{article}
\usepackage[margin=1in]{geometry}
\usepackage{graphicx}
\usepackage{booktabs}
\usepackage{longtable}
\usepackage{array}
\usepackage{fancyhdr}
\usepackage{titlesec}
\usepackage{hyperref}
\usepackage{enumitem}
\usepackage{multicol}

\pagestyle{fancy}
\fancyhf{}
\rhead{\thepage}
\lhead{Fate's Edge SRD}
\renewcommand{\headrulewidth}{0.4pt}

\titleformat{\section}{\large\bfseries}{\thesection}{1em}{}
\titleformat{\subsection}{\bfseries}{\thesubsection}{1em}{}

\title{Fate's Edge System Reference Document (SRD)}
\author{}
\date{}

\begin{document}

\maketitle

\tableofcontents
\newpage

\section{Core Principles}

\subsection{Identity of Fate's Edge}
Fate's Edge is a narrative-first tabletop roleplaying system where every action carries weight, every choice has consequence, and every spell risks backlash. Dice are not simply a measure of success or failure—they are instruments of fate, weaving opportunity with risk.

\subsection{A World of Consequences}

\subsubsection{Design Goals}
\begin{itemize}
    \item \textbf{Narrative Primacy}: Mechanics exist to serve the story.
    \item \textbf{Risk as Drama}: Every roll carries the potential for triumph and complication.
    \item \textbf{Meaningful Growth}: Advancement is more than improving statistics.
\end{itemize}

\subsubsection{The Central Question}
What are you willing to risk, and what are you willing to pay, to reshape the world around you?

\subsubsection{Tone of Play}
\begin{itemize}
    \item Cinematic, with pacing tied to narrative beats.
    \item Consequential, where even small choices ripple outward.
    \item Collaborative, empowering both GM and players.
\end{itemize}

\subsection{Key Concepts}

\subsubsection{Narrative Time}
Time is measured by story weight:
\begin{itemize}
    \item A Moment — A heartbeat, a glance, a single strike or word.
    \item Some Time — A few minutes, enough for a skirmish.
    \item Significant Time — Hours, long enough for travel or rituals.
    \item Days — Large-scale endeavors: marches, training, recovery.
\end{itemize}

\subsubsection{Complication Points}
Whenever a player rolls dice, each result of 1 generates a Complication Point (CP). These are narrative fuel. The GM spends them to introduce twists.

\subsubsection{Affinity}
Each culture provides an Affinity: a narrative edge or metaphysical bond. Affinities make certain Arts, skills, or actions more reliable.

\subsubsection{Prestige Abilities}
High-level talents unlocked by mastering cultural arts or philosophies. They are narrative milestones as much as mechanical ones.

\subsubsection{On-Screen vs. Off-Screen}
\begin{itemize}
    \item \textbf{On-Screen Resources}: Companions, hirelings, or allies who stand beside you in danger.
    \item \textbf{Off-Screen Resources}: Taverns, estates, titles, or networks of informants.
\end{itemize}

\section{Core Mechanic}

\subsection{The Art of Consequence}

\subsubsection{Procedure}
All significant actions follow a three-step process:
\begin{enumerate}
    \item \textbf{Approach}: The player describes both what their character wants and how they attempt it.
    \item \textbf{Execution}: Build a dice pool equal to Attribute + Skill and roll that many d10s. Each die of 6 or higher counts as a success. Each 1 rolled generates a Complication Point.
    \item \textbf{Outcome}: The GM interprets total successes against the difficulty of the task. Complication Points are then spent to weave narrative setbacks.
\end{enumerate}

\subsubsection{The Description Ladder}
\begin{itemize}
    \item \textbf{Basic Action}: Roll the pool as-is. All 1s remain as Complication Points.
    \item \textbf{Detailed Action}: A clear, descriptive flourish allows the player to re-roll one die showing 1.
    \item \textbf{Intricate Action}: A richly described, multi-sensory action allows the player to re-roll all dice showing 1, and add one positive narrative flourish to the scene if they succeed.
\end{itemize}

\subsubsection{Complication Points}
Complication Points (CP) are the engine of drama. They are not simple penalties, but narrative levers. The GM spends CP to introduce setbacks appropriate to the context:
\begin{itemize}
    \item Escalation — drawing more enemies, raising the stakes.
    \item Exhaustion — draining time, resources, or positioning.
    \item Exposure — revealing hidden actions, alerting foes.
    \item Collateral — harm or danger spilling over onto allies, innocents, or surroundings.
\end{itemize}

\subsubsection{Design Intent}
This mechanic ensures that every roll changes the story. Success without risk is rare, and even failure opens new narrative avenues.

\subsubsection{GM Quick Reference: Adjudicating Skill Checks}

\paragraph{Difficulty Ladder (Set Before the Roll)}
\begin{center}
\begin{tabular}{cll}
\toprule
\textbf{DV} & \textbf{Name} & \textbf{When to Use} \\
\midrule
1 & Routine & Clear intent, modest stakes, controlled environment. \\
2 & Pressured & Time pressure, mild resistance, partial info. \\
3 & Hard & Hostile conditions, active opposition, precise timing. \\
4+ & Extreme & Multiple constraints, high precision, dramatic failure. \\
\bottomrule
\end{tabular}
\end{center}

\paragraph{Outcome Matrix (After the Roll)}
Let $S$ be successes (≥ 6) and $C$ be Complication Points (number of 1s rolled).
\begin{center}
\begin{tabular}{lll}
\toprule
\textbf{Case} & \textbf{Name} & \textbf{Guidance} \\
\midrule
$S \geq DV$ and $C = 0$ & Clean Success & Deliver the intent crisply. \\
$S \geq DV$ and $C > 0$ & Success \& Cost & Grant the intent; spend/bank CP for complications. \\
$0 < S < DV$ & Partial & Progress with a fork. \\
$S = 0$ & Miss & No progress. Cash/bank CP or offer Devil's Bargain. \\
\bottomrule
\end{tabular}
\end{center}

\paragraph{Complication Point (CP) Spend Menu}
\begin{itemize}
    \item \textbf{1 CP}: Minor pressure: noise, trace, +1 Supply segment.
    \item \textbf{2 CP}: Moderate setback: alarm, lose position, lesser foe.
    \item \textbf{3 CP}: Serious trouble: reinforcements, key gear breaks, rail tick.
    \item \textbf{4+ CP}: Major turn: trap springs, authority arrives, scene shifts.
\end{itemize}

\paragraph{Assistance, Boons, \& Description}
\begin{itemize}
    \item \textbf{Assists}: One helper per action; up to +3 dice.
    \item \textbf{Boons}: A player may re-roll one die after seeing the pool. Between sessions, 2 Boons = 1 XP.
    \item \textbf{Description Ladder}: Basic (roll as-is), Detailed (re-roll one 1), Intricate (re-roll all 1s and add one flourish if successful).
\end{itemize}

\paragraph{Setting Stakes Fast (Cheat Prompts)}
\begin{itemize}
    \item If this goes right, what changes?
    \item If this goes wrong, what bites back?
\end{itemize}

\paragraph{Banking \& Cashing CP}
\begin{itemize}
    \item Banked CP should pay off within the same scene or arc.
    \item Avoid nickel-and-diming. Prefer one memorable complication over many petty penalties.
\end{itemize}

\subsection{Worked Micro-Examples}
\begin{itemize}
    \item \textbf{Lockpick Under Watch (DV 2)}: Player rolls 6 dice: 10, 8, 5, 4, 1, 1 ⇒ S=2, C=2. Success \& Cost. Door opens; GM spends 1 CP for a squeal (patrol starts moving) and banks 1 CP to bring that patrol around on the next beat.
    \item \textbf{Charm the Captain (DV 2)}: Player rolls 5 dice: 7, 6, 6, 2, 1 ⇒ S=3, C=1. Success \& Cost. Passage granted; GM spends 1 CP: “He expects a favor on the return leg—he’ll collect.”
    \item \textbf{Traverse the Pass (DV 3)}: Group roll pools to a net 3 successes but produces C=3. Success \& Cost. GM spends 2 CP to add Fatigue 1 to all from cold and exposure, banks 1 CP to crack a wagon axle next scene.
\end{itemize}

\section{Advancement \& XP}

\subsection{Awarding XP}
\begin{itemize}
    \item \textbf{Gritty}: 4–6 XP per session (slow burn).
    \item \textbf{Standard}: 6–10 XP per session (default pace).
    \item \textbf{Heroic}: 10–14 XP per session (fast growth).
\end{itemize}

\subsubsection{Session Awards}
\begin{itemize}
    \item Table Attendance: +2 XP
    \item Major Objective Reached: +2–4 XP
    \item Discovery or Lore Unlocked: +1–2 XP
    \item Hard Choice Embraced: +1–2 XP
    \item Complication Spotlight: +1–3 XP
    \item Bond/Flag Driven Play: +1–2 XP
    \item GM Curveball Award: +0–3 XP
\end{itemize}

\subsubsection{Milestones}
\begin{itemize}
    \item +8–12 XP to all players at the conclusion of a major story arc.
    \item +2 XP bonus to one player for a signature moment of the arc.
\end{itemize}

\subsubsection{Complication Dividend}
\begin{itemize}
    \item Face Card: +1 XP
    \item Ace: +2 XP
\end{itemize}

\subsection{Spending XP}
\begin{itemize}
    \item \textbf{Attributes}: Cost = new rating × 3. Downtime = new rating in days.
    \item \textbf{Skills}: Cost = new level × 2. Downtime = new level in days.
    \item \textbf{On-Screen Followers}: Cost = Cap². Downtime = 1–3 days to recruit and brief.
    \item \textbf{Off-Screen Assets}: Minor (4 XP, 1 day), Standard (8 XP, 1 week), Major (12 XP, 1 month).
\end{itemize}

\subsubsection{Rush Rule}
A player may skip downtime, but the GM creates a Haste clock of four segments. If the clock fills, the new ability or asset carries flaws or narrative complications.

\subsection{Tiers of Reputation}
\begin{itemize}
    \item \textbf{Tier I – Rookie} (0–40 XP): Local reputation; prestige locked.
    \item \textbf{Tier II – Seasoned} (41–90): Regional notice; prestige abilities may be unlocked.
    \item \textbf{Tier III – Veteran} (91–150): National influence; second follower slot suggested.
    \item \textbf{Tier IV – Paragon} (151–220): Movers and shakers; rivals emerge to challenge.
    \item \textbf{Tier V – Mythic} (221+): Legendary status; kingdoms and cults respond.
\end{itemize}

\section{Rules Clarifications}

\subsection{Follower Assist}
\begin{itemize}
    \item Assist dice come from the helper, not the leader.
    \item Total Assist on any roll (from any sources) remains hard-capped at +3.
\end{itemize}

\subsection{Boon Economy}
\begin{itemize}
    \item Holding cap: You can hold at most 5 Boons.
    \item Conversion: Once per session, in downtime, you may convert 2 Boons → 1 XP (max 1 XP via conversion per session).
\end{itemize}

\subsection{Asset Activations}
\begin{itemize}
    \item Off-Screen effects: Use each Asset's listed Off-Screen effect once per session for free.
    \item On-Screen activations: To reshape the current scene, spend 1 Boon.
    \item Plausibility test: The Asset must have scope and reach.
\end{itemize}

\subsection{Over-Stack}
\begin{itemize}
    \item Structural advantages: active buff/tag, favorable venue/pennant, Follower Initiative unused, on-screen Asset activation, opponent disadvantaged by fiction, ritual prep that applies now.
    \item Trigger: If the crew enters a scene with ≥ 3 structural advantages, apply Over-Stack once for that scene: either start one named rail at +1 or the GM banks +1 CP for the first ♠ Twist.
\end{itemize}

\subsection{Familiar Bond}
\begin{itemize}
    \item Familiars use the standard Follower Exposure/Harm tracks and require no upkeep.
    \item Each time a familiar acts on-screen in a high-risk beat, mark Exposure +1 on the familiar after the second such beat this scene.
\end{itemize}

\subsection{Ritual Casting}
\begin{itemize}
    \item Helper cap: Maximum simultaneous helpers = ⌈primary caster's Ritual/Arcana/2⌉, max 3.
    \item Relevant skills: Helpers may use different relevant skills if their procedure is fictionally distinct.
    \item CP distribution: CP from Channel resolves on that roller. CP from Weave is assigned to the primary caster.
\end{itemize}

\subsection{Deck of Consequences}
\begin{itemize}
    \item After a roll that generates CP, the GM chooses one method for that roll:
    \begin{enumerate}
        \item Direct Spend: Translate CP into consequences/rail ticks immediately.
        \item Deck Draw: Draw 1 card per CP and synthesize a single twist consistent with the suits/ranks.
    \end{enumerate}
\end{itemize}

\subsection{Casting Loop}
\begin{itemize}
    \item Order of operations: Channel (resolve effect + CP) → if Channel fails catastrophically, the spell may fizzle; otherwise Weave (resolve effect) → apply Backlash from Weave.
    \item Mitigation: Boons do not reduce CP unless a Talent/Asset explicitly says “Mitigate CP.”
\end{itemize}

\subsection{Initiative Action}
\begin{itemize}
    \item Cost choice: An Initiative Action by a follower costs either Exposure +1 or Harm 1.
    \item Cadence: By default, the crew has 1 Follower Initiative window per scene.
\end{itemize}

\subsection{Prestige Prerequisites}
\begin{itemize}
    \item Qualifying: Attribute/Skill prerequisites must be met with permanent ratings.
    \item After purchase: If you later lose the Asset prerequisite, you keep the Talent but cannot activate features that require that Asset until restored.
\end{itemize}

\section{Character Framework}

\subsection{Starting XP \& Complications}

\subsubsection{Starting Pool}
Characters begin with a baseline of 30 XP to allocate across Attributes, Skills, Talents, and Assets.

\subsubsection{Optional Trades}
\begin{itemize}
    \item \textbf{Obligation Deficit}: You may begin with a deficit of up to 2 XP.
    \item \textbf{Complication Trade}: You may accept up to 2 Complications during Session 0, each granting +1 XP.
\end{itemize}

\subsubsection{Starting Complications Table}
\begin{center}
\begin{tabular}{cl}
\toprule
\textbf{d10} & \textbf{Starting Complication} \\
\midrule
1 & Debt Mark: You owe coin or favor to a guild, syndicate, or noble family. \\
2 & Broken Heirloom: Begin play with one signature item already Compromised. \\
3 & Enemy at Large: Someone you wronged is alive and plotting. \\
4 & Dark Patron: A whispering spirit, godling, or demon has touched your life. \\
5 & Notoriety: You are famous (or infamous) for something unsavory. \\
6 & Haunted: Nightmares, visions, or a literal ghost troubles your sleep. \\
7 & Cursed Token: You carry an item that brings ill luck. \\
8 & Fractured Loyalty: You belong to two groups with conflicting demands. \\
9 & Lost Ally: A former follower, familiar, or friend is missing. \\
10 & Blood Feud: Your kin, clan, or culture is sworn against another. \\
\bottomrule
\end{tabular}
\end{center}

\subsection{Attributes \& Skills}

\subsubsection{Attributes}
\begin{itemize}
    \item \textbf{Body}: Strength, endurance, and physical action.
    \item \textbf{Wits}: Perception, cleverness, and reaction speed.
    \item \textbf{Spirit}: Willpower, intuition, and resilience.
    \item \textbf{Presence}: Charm, command, and social force.
\end{itemize}

Each Attribute is rated from 1–5 for most mortals.

\subsubsection{Skills}
\begin{itemize}
    \item \textbf{Athletics} — climbing, running, swimming.
    \item \textbf{Arcana} — magical theory, rituals, spellwork.
    \item \textbf{Brawl} — fists, grappling, improvised fighting.
    \item \textbf{Insight} — intuition, empathy, lie detection.
    \item \textbf{Melee} — blades, axes, polearms.
    \item \textbf{Ranged} — bows, crossbows, thrown arms.
    \item \textbf{Diplomacy} — negotiation, mediation, etiquette.
    \item \textbf{Stealth} — hiding, shadowing, evading.
    \item \textbf{Deception} — disguise, misdirection, bluffing.
    \item \textbf{Survival} — tracking, foraging, navigation.
    \item \textbf{Command} — leadership, intimidation, rallying.
    \item \textbf{Craft} — smithing, alchemy, tinkering.
    \item \textbf{Performance} — music, oratory, storytelling.
    \item \textbf{Lore} — history, cultures, languages.
\end{itemize}

\subsubsection{Skill Ratings}
\begin{itemize}
    \item 0: Untrained — You rely on raw attribute alone.
    \item 1: Familiar — Basic competence.
    \item 2: Skilled — Trained and reliable.
    \item 3: Expert — Professional mastery.
    \item 4: Master — Renowned.
    \item 5: Legendary — Almost supernatural.
\end{itemize}

\section{The Three Paths of XP}

\subsection{Enhance Self}
\begin{itemize}
    \item \textbf{Attributes}: Raising an Attribute costs XP equal to (new rating × 3).
    \item \textbf{Skills}: Raising a Skill costs XP equal to (new level × 2).
\end{itemize}

\subsection{Acquire Assets}
\begin{itemize}
    \item \textbf{Off-Screen Resources}: Purchased with XP.
    \item \textbf{On-Screen Followers}: Allies who act in the scene and lend their specialty.
    \begin{itemize}
        \item Cost: A follower with Specialty Cap C costs C² XP.
        \item Specialty: Define one narrow lane.
        \item Assist Dice: When applicable, the follower adds help dice equal to min(C, your relevant Skill), capped at +3 dice.
        \item Slot Limit: Only one follower may assist a given action.
        \item Upkeep: Each Downtime, pay Coin equal to C or spend a Scene tending the relationship.
        \item Risk: If the GM spends 2+ Complication Points on an action you take with assistance, they may endanger, injure, or separate the follower instead of you if fictionally appropriate.
    \end{itemize}
\end{itemize}

\subsection{Learn Talents}
\begin{itemize}
    \item \textbf{General Talents}:
    \begin{itemize}
        \item Battle Instincts (Cost: 6 XP): Once per scene, re-roll a failed defense roll.
        \item Silver Tongue (Cost: 4 XP): Gain +1 die when persuading or deceiving through speech.
        \item Iron Stomach (Cost: 3 XP): Immune to mundane poisons and spoiled food; halve Complications from toxic sources.
    \end{itemize}
    \item \textbf{Racial or Cultural Talents}:
    \begin{itemize}
        \item Stone-Sense (Dwarves, Cost: 5 XP): Detect flaws in stone or earth; gain +1 die on Engineering or Craft rolls underground.
        \item Backlash Soothing (Wood Elves, Cost: 6 XP): Once per session, reduce a magical Backlash Complication by 2 points when in natural terrain.
        \item Blood Memory (Ykrul, Cost: 5 XP): After a battle, meditate to gain one temporary Skill die reflecting a foe's tactics for the next scene.
    \end{itemize}
    \item \textbf{Prestige Abilities}:
    \begin{itemize}
        \item Echo-Walker (High Elf, Cost: 20 XP; Req: Wits 5, Arcana 4): Step briefly into Aerisahl; once per arc, turn any Complication into a boon.
        \item Warglord (Ykrul, Cost: 18 XP; Req: Body 5, Leadership 3): Rally scattered warbands into a single host; once per campaign, may unify tribes under one banner.
        \item Spirit-Shield (Dwarves, Cost: 15 XP; Req: Spirit 4, Resolve 3): Once per session, erase up to 3 Complications from an ally's roll, taking 1 Backlash yourself.
    \end{itemize}
\end{itemize}

\section{Followers \& Off-Screen Assets}

\subsection{Concept}
\begin{itemize}
    \item \textbf{On-Screen Followers}: Stat-light allies with a Skill Cap who add dice in their specialty when present.
    \item \textbf{Off-Screen Assets}: Holdings resolved between sessions (keeps, charters, titles, spy webs).
\end{itemize}

\subsection{Buying Followers \& Assets}
\begin{itemize}
    \item \textbf{Followers}: Cost = Cap² XP.
    \item \textbf{Off-Screen Assets}:
    \begin{itemize}
        \item Minor: 4 XP
        \item Standard: 8 XP
        \item Major: 12 XP
    \end{itemize}
\end{itemize}

\subsection{Condition Tracks \& Upkeep}
\begin{itemize}
    \item \textbf{Assets/FOLLOWERS}: Maintained → Neglected → Compromised
    \item \textbf{PARTY RESOURCES}: Supply (0-Full → 2-Low → 3-Dangerous → 4-Empty)
    \item \textbf{CHARACTER STATE}: Fatigue (1-4 levels, re-roll successes)
\end{itemize}

\subsection{Stress, Harm, \& Loss (GM Tools)}
\begin{itemize}
    \item \textbf{Pin}: The follower is separated/boxed out.
    \item \textbf{Wound}: The follower is Injured: until treated off-screen, their Cap counts as 1 lower.
    \item \textbf{Burn}: Mark Neglected immediately.
    \item \textbf{Seize}: Escalate to Compromised.
    \item \textbf{PC Choice Lever}: The GM should offer the player a save.
\end{itemize}

\subsection{Loyalty \& Bonds (Optional)}
\begin{itemize}
    \item Track a simple Loyalty tag per follower: Wary / Steady / Devoted.
    \item Devoted followers can once per arc convert one GM Complication targeting them into a lesser setback.
    \item Wary followers cost +1 XP to Maintain.
\end{itemize}

\section{Magic \& The Arts}

\subsection{Philosophy of Magic}
Magic in Fate's Edge is not a tool of convenience but a dangerous negotiation with the fabric of reality.

\subsection{The Nature of Magic}
\begin{itemize}
    \item \textbf{Volatile by Design}: Magic is not fully understood.
    \item \textbf{Risk Embodied}: Each spell generates Complication Points.
    \item \textbf{Narrative Weight}: Casting is always a story moment.
    \item \textbf{Thematic Consequence}: Backlash is not arbitrary; it aligns with the opposing or uncontrolled element of the Art invoked.
\end{itemize}

\subsection{The Caster's Burden}
Magicians are defined not by what they can do, but by what they are willing to risk.

\subsection{Casting Loop}
All spellcasting follows a structured sequence called the Casting Loop:
\begin{enumerate}
    \item \textbf{Channel}: The caster focuses, rolling Wits + Arcana to gather Potential.
    \item \textbf{Weave}: On the following turn, the caster rolls Wits + (Art) to shape Potential into a defined effect.
    \item \textbf{Backlash}: Complication Points spent by the GM manifest as uncontrolled consequences.
\end{enumerate}

\subsection{Example of Backlash}
\begin{itemize}
    \item \textbf{Fire}: Flames leap to unattended surfaces, smoke blinds allies, or the heat weakens structures.
    \item \textbf{Shadow}: Illusions persist too long, unseen things whisper truths best left hidden, morale crumbles.
    \item \textbf{Storm}: Winds scatter allies' plans, lightning arcs toward unintended targets, storms linger beyond the caster's will.
\end{itemize}

\subsection{Ritual Casting (Optional Rule)}
\begin{itemize}
    \item \textbf{Ritual Helper Cap}: You may draw on ceil(Arcana/2) helpers (max 3).
    \item \textbf{Procedure}:
    \begin{enumerate}
        \item Declare the Ritual.
        \item Channel Together.
        \item Weave.
        \item Backlash.
    \end{enumerate}
\end{itemize}

\section{GM Toolkit}

\subsection{Logistics as Drama}
In Fate's Edge, arrows, rations, and waterskins are tracked only in the fiction.

\subsection{The Supply Clock}
\begin{itemize}
    \item \textbf{Full Supply (0 filled)}: The party is well-equipped.
    \item \textbf{Low Supply (2 filled)}: Minor narrative complications.
    \item \textbf{Dangerously Low (3 filled)}: Each character gains Fatigue.
    \item \textbf{Out of Supply (4 filled)}: Severe penalties.
\end{itemize}

\subsection{Fatigue}
\begin{itemize}
    \item Effect: On their next roll, a character must reroll one success.
    \item Stacking: Each level adds another forced reroll.
    \item Recovery: A night's rest with adequate supply removes 1 Fatigue.
\end{itemize}

\subsection{Gear Damage}
\begin{itemize}
    \item \textbf{Compromised Items}: Introduced via Complication Points or narrative consequence. A Compromised item gives −1 die on relevant rolls.
    \item \textbf{Breaking Point}: If a Compromised item suffers another setback, it breaks entirely.
    \item \textbf{Repair}: Field Repair or Proper Repair.
\end{itemize}

\section{Deck of Consequences}

\subsection{Structure of the Deck}
\begin{itemize}
    \item \textbf{Suits} = Domains of Complications
    \begin{itemize}
        \item Hearts: Emotional, social, or relational fallout.
        \item Swords: Harm, danger, or escalation of conflict.
        \item Pentacles: Resource strain, economic or material cost.
        \item Wands: Magical, spiritual, or cosmic disturbances.
    \end{itemize}
    \item \textbf{Ranks} = Severity of Complications
    \begin{itemize}
        \item Ace–3: Minor inconvenience or flavor complication.
        \item 4–6: Moderate setback with some narrative teeth.
        \item 7–9: Significant consequence altering the course of action.
        \item 10–King: Major fallout, introducing new problems or lasting scars.
    \end{itemize}
\end{itemize}

\subsection{Using the Deck}
\begin{enumerate}
    \item Player rolls; each 1 generates a Complication Point.
    \item GM may draw a card for each Complication Point.
    \item The suit frames the type of complication; the rank determines severity.
    \item GM interprets and narrates based on context.
\end{enumerate}

\section{Player Archetypes at the Table}

\subsection{The Solo}
\begin{itemize}
    \item Invests XP primarily in Attributes and Skills.
    \item Strengths: always ready, iconic spotlight.
    \item Risks: narrow toolkit; may lag in social or resource scenes.
\end{itemize}

\subsection{The Mixed Player}
\begin{itemize}
    \item Balances XP between self and assets.
    \item Strengths: adaptable, bridges party gaps.
    \item Risks: upkeep spread thin.
\end{itemize}

\subsection{The Mastermind}
\begin{itemize}
    \item Builds networks, followers, and assets.
    \item Strengths: broad reach, drives strategies.
    \item Risks: Complication fallout; vulnerable allies.
\end{itemize}

\section{Campaign Frame / Finale: The Crown Spread}

\subsection{Session 0: The Crown Spread (Initial Draw)}
Draw 5 cards: Spade, Heart, Club, Diamond, and a Wild (any suit; reveal last).

\subsection{The Campaign Clock}
Track two dials over the campaign:
\begin{itemize}
    \item \textbf{Mandate (0–6)}: The table's public legitimacy and buy-in.
    \item \textbf{Crisis (0–6)}: The opposition engine (rivals, pressure rails, attrition).
\end{itemize}

\subsection{Finale Procedure (Crown Beat)}
Use the Session 0 Crown Spread to seed setup; then run the three-beat crown.

\subsection{Legacy Conversion (Epilogue)}
After the Finale, each PC draws 2 cards and answers epilogue prompts by suit.

\section{Travel Framework}

\subsection{Core Travel Procedure}
For each leg of a journey, draw 3–4 cards using the decks for your destination and controlling authority.
\begin{itemize}
    \item Spade from the destination deck: sets the scene (place).
    \item Heart from the destination deck: introduces the local actor or faction.
    \item Club from the Wilds (general hazards) or destination (if strongly policed): brings pressure.
    \item Diamond from the authority that gates the route: papers, escorts, rights, or exceptions.
\end{itemize}

Set a travel clock by the highest rank (2–5⇒4 • 6–10⇒6 • J/Q/K⇒8 • A⇒10). On success, advance to the next leg; on failure, mark delay, debt, or diversion and resolve a consequence in the fiction.

\subsection{Mode rules}
\begin{itemize}
    \item \textbf{Sea legs} (Amaranthine/Dolmis/Aberderrin): If Theona or Valewood 9s show up anywhere in the seed, add an omission or taboo to the leg.
    \item \textbf{Passes Underways} (Aeler): Any A may convert a surface route to an under-route.
    \item \textbf{Rivers}: Bridges, booms, and law in Ecktoria/Viterra; reed-mazes and bell-lines in Mistlands/Linn waters.
    \item \textbf{Frontier blends}: When origin and destination disagree on law, draw two Diamonds (one from each law) and choose which you will be judged by at the end of the leg.
\end{itemize}

\subsection{Route Modules}
\subsubsection{Amaranthine Coastway}
Kahfagia → Ecktoria → Acasia → Marcott (Vhasia) → Fairport (Viterra).

\subsubsection{Astroegro Straits}
Thepyrgos controls the hinge between seas.

\subsubsection{Dolmis Circuits}
Fairport (Viterra) → Theona (Three Greens) → Ubral fjords → Aelinnel west shore.

\subsubsection{Aelerian Passes Underways}
Vhasia/Viterra/Ubral south slopes → Aeler gates → Mistlands.

\subsubsection{Shadow Corridors}
Thin Shore (Valewood east coast): risky misted corridor north–south toward Zakov.

\subsubsection{River Roads}
Belworth: forms the boundary between Vhasia and Viterra.

\subsubsection{Steppe Frontiers (Violet Steppes Meadows)}
Ykrul ↔ Vilikari ↔ Ecktoria/Acasiaborders.

\section{Design Philosophy Guardrails}

\subsection{Core Principles}
\begin{enumerate}
    \item \textbf{Narrative Primacy}: Mechanics serve story, not replace it.
    \item \textbf{Risk as Drama}: Every roll carries potential for triumph + complication.
    \item \textbf{Meaningful Growth}: XP investment creates lasting character/world change.
    \item \textbf{Consequence Weight}: Choices ripple outward, nothing is free.
\end{enumerate}

\subsection{Mechanical Constraints}
\begin{itemize}
    \item \textbf{ASSIST MAX}: +3 dice total per roll, regardless of helpers.
    \item \textbf{BOON MAX}: 5 total, 2→1 XP conversion once/session.
    \item \textbf{INITIATIVE}: 1 Follower Action per scene crew-wide.
    \item \textbf{OVER-STACK}: 2+ structural advantages = start rails +1 OR GM banks +1 CP.
    \item \textbf{POSITION}: Controlled | Risky | Desperate (affects success/failure texture).
\end{itemize}

\end{document}
