\documentclass[12pt]{article}

% --- Packages ---
\usepackage{geometry}
\geometry{margin=1in}
\usepackage{titlesec}
\usepackage{hyperref}
\usepackage{enumitem}
\usepackage{multicol}
\usepackage{tcolorbox}

% --- Formatting ---
\titleformat{\section}{\Huge\bfseries}{\thesection}{1em}{}
\titleformat{\section}{\Large\bfseries}{\thesection}{1em}{}
\titleformat{\subsection}{\large\bfseries}{\thesubsection}{1em}{}

% --- Title Page ---
\title{\Huge Fate's Edge \\[6pt] \Large A World of Consequences}
\author{System Reference Document (SRD)}
\date{}

\begin{document}
\maketitle
\tableofcontents
\newpage

% ======================================================
\part{Core Principles}
% ======================================================

\section{Identity of Fate's Edge}

\section{A World of Consequences}
\textit{Fate’s Edge} is a narrative-first tabletop roleplaying system where every action carries weight, every choice has consequence, and every spell risks backlash. Dice are not simply a measure of success or failure—they are instruments of fate, weaving opportunity with risk.  

In this world, victory is rarely clean, and defeat is rarely final. Every roll pushes the story forward, not just by resolving what happens, but by revealing what it costs.  

\subsection{Design Goals}
The design philosophy of \textit{Fate’s Edge} rests on three core goals:  

\begin{description}[leftmargin=2cm, style=nextline]
  \item[Narrative Primacy] Mechanics exist to serve the story. Rules reward creativity and descriptive play, ensuring that characters are defined as much by their choices as by their numbers.  
  \item[Risk as Drama] Every roll carries the potential for triumph and the shadow of complication. Tension, not certainty, drives the narrative.  
  \item[Meaningful Growth] Advancement is more than improving statistics. Players invest experience not only in personal ability but also in influence, legacies, and the world around them.  
\end{description}

\subsection{The Central Question}
At its heart, \textit{Fate’s Edge} asks:  

\begin{quote}
\centering
What are you willing to risk, and what are you willing to pay, to reshape the world around you?
\end{quote}

This question is both philosophical and mechanical. Players gamble with fate every time they act, and the consequences—good or ill—become the foundation of their legend.  

\subsection{Tone of Play}
\textit{Fate’s Edge} is designed to produce stories that feel:  
\begin{itemize}
  \item \textbf{Cinematic}, with pacing tied to narrative beats rather than strict measures of time.  
  \item \textbf{Consequential}, where even small choices ripple outward in meaningful ways.  
  \item \textbf{Collaborative}, empowering both GM and players to co-create the unfolding drama.  
\end{itemize}

\subsection{Identity in Brief}
\begin{quote}
\textit{Fate’s Edge} is a world balanced on a knife:  
every decision sharpens or dulls the blade,  
every risk cuts both ways.  
\end{quote}

\section{Key Concepts}

\subsection{Narrative Time}
Time in \textit{Fate’s Edge} is measured by story weight, not by clocks. Actions are framed in four narrative scales:
\begin{itemize}
  \item \textbf{A Moment} — A heartbeat, a glance, a single strike or word.
  \item \textbf{Some Time} — A few minutes, enough for a skirmish, a careful lockpick, or a short negotiation.
  \item \textbf{Significant Time} — Hours, long enough to travel between locations, work a ritual, or endure a siege.
  \item \textbf{Days} — Large-scale endeavors: marches across a countryside, training a cadre, or recovering from wounds.
\end{itemize}
This framing gives players and the GM a shared sense of pacing and tension. The focus is always on narrative rhythm, not stopwatch precision.  

\subsection{Complication Points}
Whenever a player rolls dice, each result of \texttt{1} generates a \textbf{Complication Point}. These are not mere penalties—they are narrative fuel. The GM spends them to introduce twists: broken weapons, misplaced trust, sudden weather, collateral damage.  

Complications ensure that even a success carries texture. A hero may win the duel but lose the crowd’s favor; a spell may ignite, but it leaves behind smoke and whispers.  

\subsection{Affinity}
Races and cultures in \textit{Fate’s Edge} do not define characters through numbers alone. Instead, each provides an \textbf{Affinity}: a narrative edge or metaphysical bond. Affinities make certain Arts, skills, or actions more reliable, weaving identity into mechanics.  
\begin{quote}
Wood Elves find quiet in nature; Dwarves in stone; Humans in adaptability. Each Affinity is a way the world itself leans in your favor.
\end{quote}

\subsection{Prestige Abilities}
Prestige Abilities are high-level talents unlocked by mastering cultural arts or philosophies. They are narrative milestones as much as mechanical ones. A High Elf Echo-Walker, a Dwarven Spirit Shield, or a Human Bannerlord does not simply roll better—they reshape the story around them.  

\subsection{On-Screen vs. Off-Screen}
\textit{Fate’s Edge} distinguishes between resources you see at the table and those that shape the world in the background:
\begin{itemize}
  \item \textbf{On-Screen Resources} are companions, hirelings, or allies who stand beside you in danger. They add dice pools and flavor, but they can falter, be taken, or die.  
  \item \textbf{Off-Screen Resources} are taverns, estates, titles, or networks of informants. They never swing a blade in combat, but they shape the story between sessions, turning XP into narrative leverage.  
\end{itemize}
The choice between the two defines whether a character invests in independence, leadership, or legacy.  

\subsection{Why These Concepts Matter}
Together, these concepts shift the game away from strict measures of efficiency toward dramatic consequence. Time is elastic, outcomes ripple, and choices reach far beyond the battlefield. This keeps the table focused not only on \emph{what happens}, but also on \emph{why it matters}.

% ======================================================
\part{Core Mechanic}
% ======================================================
\section{The Art of Consequence}

At the heart of \textit{Fate’s Edge} lies a single design principle:  
\textbf{every action carries consequence}. Dice do not merely determine success or failure; they shape the unfolding narrative by introducing new problems, twists, or opportunities. The system thrives on tension between intent and outcome, where even the best successes may carry unexpected costs.

\subsection{Procedure}

All significant actions follow a three-step process:

\begin{enumerate}
  \item \textbf{Approach} — The player describes both \emph{what} their character wants and \emph{how} they attempt it. This defines the primary Skill and clarifies the fiction.  
  \item \textbf{Execution} — Build a dice pool equal to \emph{Attribute + Skill} and roll that many d10s. Each die of 6 or higher counts as a success. Each \texttt{1} rolled generates a \emph{Complication Point}.  
  \item \textbf{Outcome} — The GM interprets total successes against the difficulty of the task. Complication Points are then spent to weave narrative setbacks, collateral costs, or escalating danger.  
\end{enumerate}

\subsection{The Description Ladder}

The quality of the player’s description affects the resilience of their roll against complication.  
More detail means more control over how the dice fall:

\begin{description}[leftmargin=2cm]
  \item[Basic Action] Roll the pool as-is. All \texttt{1}s remain as Complication Points.  
  \item[Detailed Action] A clear, descriptive flourish allows the player to re-roll one die showing \texttt{1}.  
  \item[Intricate Action] A richly described, multi-sensory action allows the player to re-roll \emph{all} dice showing \texttt{1}, and add one positive narrative flourish to the scene if they succeed.  
\end{description}

\subsection{Complication Points}

Complication Points (CP) are the engine of drama. They are not simple penalties, but narrative levers.  
The GM spends CP to introduce setbacks appropriate to the context, such as:

\begin{itemize}
  \item Escalation — drawing more enemies, raising the stakes.  
  \item Exhaustion — draining time, resources, or positioning.  
  \item Exposure — revealing hidden actions, alerting foes.  
  \item Collateral — harm or danger spilling over onto allies, innocents, or surroundings.  
\end{itemize}

\subsection{Design Intent}

This mechanic ensures that every roll changes the story. Success without risk is rare, and even failure opens new narrative avenues. The dice shape the texture of events, but the story itself remains in the hands of the players and GM, always circling back to choice and consequence.

\subsection{GM Quick Reference: Adjudicating Skill Checks}

\subsubsection{Difficulty Ladder (Set Before the Roll)}
\begin{tabular}{@{}llp{8.5cm}@{}}
\toprule
\textbf{DV (successes)} & \textbf{Name} & \textbf{When to Use} \\
\midrule
1 & Routine & Clear intent, modest stakes, controlled environment. Most single-step tasks. \\
2 & Pressured & Time pressure, mild resistance, partial information, split attention. \\
3 & Hard & Hostile conditions, active opposition, precise timing, brittle outcomes. \\
4+ & Extreme & Multiple constraints at once, high precision or secrecy, \emph{and} failure is dramatic. Use sparingly. \\
\bottomrule
\end{tabular}

\smallskip
\noindent\emph{Note:} Opposed tests can be run as \emph{contests} (both sides roll; higher total successes wins; ties lean to status quo), or as a fixed DV informed by the opposition's pool.

\subsubsection{Outcome Matrix (After the Roll)}
Let $S$ be successes ($\ge 6$) and $C$ be Complication Points (number of \texttt{1}s rolled).

\begin{tabular}{@{}llp{9cm}@{}}
\toprule
\textbf{Case} & \textbf{Name} & \textbf{Guidance} \\
\midrule
$S \ge \mathrm{DV}$ and $C=0$ & Clean Success & Deliver the intent crisply. Offer a small positional or information edge if description was \emph{Intricate}. \\
$S \ge \mathrm{DV}$ and $C>0$ & Success \& Cost & Grant the intent; spend/bank CP to add friction (noise, time lost, resource wear, new eyes on the scene). \\
$0<S<\mathrm{DV}$ & Partial & Progress with a fork: \emph{Get it but…} (time/position/gear cost) or \emph{Leave it and…} (take safety, new intel). Optionally consume or award Boons per table tone. \\
$S=0$ & Miss & No progress. Cash some CP now or bank for a coming beat. Consider offering a Devil’s Bargain: succeed at DV-1 if you accept a named complication. \\
\bottomrule
\end{tabular}

\subsubsection{Complication Point (CP) Spend Menu}
\noindent Spend CP immediately, or bank them to escalate a future beat in the same scene/arc. Scale by fiction; the numbers below are defaults.

\paragraph{Universal CP Options}
\begin{tabular}{@{}lp{11cm}@{}}
\toprule
\textbf{1 CP} & Noise, tell, or trace left; \emph{+1 segment} on the party Supply clock; a tool or item becomes \emph{Compromised}; +1 round of time passes; a bystander notices something off. \\
\textbf{2 CP} & Alarmed attention (not full alarm); lose position/cover; add a lesser foe or lock; advance a Threat clock by 1; traveler gains \emph{Fatigue 1}. \\
\textbf{3 CP} & Reinforcements en route; \emph{Out of Supply}; key gear breaks now; split the party’s options (e.g., fire, flood, collapse); escalate a faction clock by 1. \\
\textbf{4+ CP} & Major turn: trap springs, rival claims the prize first, authority arrives with mandate; convert saved CP into a scene-defining twist (one big thing, not many small). \\
\bottomrule
\end{tabular}

\paragraph{By Pillar (Examples)}
\begin{tabular}{@{}lp{11cm}@{}}
\toprule
\textbf{Combat} & 1 CP: lose footing (next defense --1d). 2 CP: weapon or shield \emph{Compromised}. 3 CP: pinned, disarmed, or separated. 4 CP: battlefield shifts (fireline, cave-in, cavalry arrives). \\
\textbf{Stealth \& Intrusion} & 1 CP: footstep/squeak; shadow seen. 2 CP: patrol pattern changes; lock resists (extra test). 3 CP: partial alarm (search initiated). 4 CP: full alarm and lockdown protocol. \\
\textbf{Social} & 1 CP: rumor cost or faux pas (future --1d with this circle). 2 CP: a concession is now required (gift, favor). 3 CP: rival interjects with leverage. 4 CP: patron turns, audience ends, or oath invoked. \\
\textbf{Travel \& Survival} & 1 CP: lose time; minor injury; weather turns. 2 CP: Supply +1 segment; mount lamed. 3 CP: wrong valley or blocked pass; \emph{Fatigue 1} to all. 4 CP: storm, rockslide, flood—route rewritten. \\
\textbf{Arcana \& Ritual} & 1 CP: prickle of backlash; sensory bleed. 2 CP: unintended side-effect (cold from fire, echoes draw attention). 3 CP: residue anchors a foe/hex. 4 CP: backlash condition or entity manifests; ritual mark persists. \\
\bottomrule
\end{tabular}

\subsubsection{Assistance, Boons, \& Description}
\begin{itemize}
  \item \textbf{Assists:} One helper per action; up to +3 dice; helper’s contribution is capped by their relevant Skill rating.
  \item \textbf{Boons:} A player may re-roll one die after seeing the pool. Between sessions, \emph{2 Boons = 1 XP}. Hard cap on Boons per the core rules to prevent farming.
  \item \textbf{Description Ladder:} \emph{Basic} (roll as-is), \emph{Detailed} (re-roll one \texttt{1}), \emph{Intricate} (re-roll all \texttt{1}s and add one flourish if successful). Reward vivid framing; deny ladder benefits for vague or purely mechanical declarations.
\end{itemize}

\subsubsection{Setting Stakes Fast (Cheat Prompts)}
\begin{itemize}
  \item \textbf{If this goes right, what changes?} (Position, time, access, consent.)
  \item \textbf{If this goes wrong, what bites back?} (Noise, harm, debt, clocks.)
  1--2 quick answers here give you DV and a CP target to aim your spends.
\end{itemize}

\subsubsection{Banking \& Cashing CP}
\begin{itemize}
  \item Banked CP should \emph{pay off within the same scene or arc}. If an arc closes with banked CP, convert them into a \emph{front-loaded complication} at the start of the next arc (e.g., “Your forged papers triggered a quiet inquiry: the customs clerk is asking around.”).
  \item Avoid nickel-and-diming. Prefer \emph{one memorable complication} over many petty penalties.
\end{itemize}

\subsection{Worked Micro-Examples}
\paragraph{Lockpick Under Watch (DV 2).}
Player rolls 6 dice: \texttt{10, 8, 5, 4, 1, 1} $\Rightarrow$ $S{=}2$, $C{=}2$. \emph{Success \& Cost}. Door opens; GM spends 1 CP for a squeal (patrol starts moving) and banks 1 CP to bring that patrol around on the next beat.

\paragraph{Charm the Captain (DV 2).}
Player rolls 5 dice: \texttt{7, 6, 6, 2, 1} $\Rightarrow$ $S{=}3$, $C{=}1$. \emph{Success \& Cost}. Passage granted; GM spends 1 CP: “He expects a favor on the return leg—he’ll collect.”

\paragraph{Traverse the Pass (DV 3).}
Group roll pools to a net 3 successes but produces $C{=}3$. \emph{Success \& Cost}. GM spends 2 CP to add \emph{Fatigue 1} to all from cold and exposure, banks 1 CP to crack a wagon axle next scene.

\bigskip
\noindent This reference is meant to keep you fast and fair: set a DV, count successes, gather 1s into CP, and let complications push the fiction forward.

\section{Advancement \& XP}

In \textit{Fate’s Edge}, characters do not advance through levels. 
Instead, growth flows from \textbf{Experience Points (XP)} --- the universal currency of progression. 
XP may be spent to hone the self, acquire worldly assets, or unlock cultural prestige. 
This design ensures that every moment of advancement is the result of a meaningful choice.

\subsection{Awarding XP}

The GM distributes XP at the end of each session or arc. 
Choose a pacing mode appropriate to the campaign:

\begin{description}[leftmargin=2cm]
  \item[Gritty:] 4--6 XP per session (slow burn; each purchase is weighty).  
  \item[Standard:] 6--10 XP per session (default pace).  
  \item[Heroic:] 10--14 XP per session (fast growth; shorter arcs).  
\end{description}

\subsection*{Session Awards}
\begin{itemize}
  \item Table Attendance: +2 XP  
  \item Major Objective Reached: +2--4 XP  
  \item Discovery or Lore Unlocked: +1--2 XP  
  \item Hard Choice Embraced: +1--2 XP  
  \item Complication Spotlight (leaning into drawn Complications): +1--3 XP  
  \item Bond/Flag Driven Play: +1--2 XP  
  \item GM Curveball Award: +0--3 XP for standout creativity  
\end{itemize}

\subsection*{Milestones}
At the conclusion of a major story arc, grant:
\begin{itemize}
  \item +8--12 XP to all players  
  \item +2 XP bonus to one player for a signature moment of the arc  
\end{itemize}

\subsection*{Complication Dividend}
If a player accepts a high Complication card as drawn, without mitigation:
\begin{itemize}
  \item Face Card: +1 XP  
  \item Ace: +2 XP  
\end{itemize}

\subsection{Spending XP}

\begin{description}[leftmargin=2cm]
  \item[Attributes:] Cost = new rating $\times$ 3.  
  Downtime = new rating in days.  
  \item[Skills:] Cost = new level $\times$ 2.  
  Downtime = new level in days.  
  \item[On-Screen Followers:] Cost = Cap$^2$.  
  Downtime = 1--3 days to recruit and brief.  
  \item[Off-Screen Assets:] Minor (4 XP, 1 day), Standard (8 XP, 1 week), Major (12 XP, 1 month).  
\end{description}

\subsection*{Rush Rule}
A player may skip downtime, but the GM creates a \textbf{Haste clock} of four segments.  
If the clock fills, the new ability or asset carries flaws or narrative complications.

\subsection{Tiers of Reputation}

Though there are no levels, XP spent creates soft ``tiers'' that shape world response:

\begin{description}[leftmargin=2cm]
  \item[Tier I -- Rookie (0--40 XP):] Local reputation; prestige locked.  
  \item[Tier II -- Seasoned (41--90):] Regional notice; prestige abilities may be unlocked.  
  \item[Tier III -- Veteran (91--150):] National influence; second follower slot suggested.  
  \item[Tier IV -- Paragon (151--220):] Movers and shakers; rivals emerge to challenge.  
  \item[Tier V -- Mythic (221+):] Legendary status; kingdoms and cults respond.  
\end{description}

% ===== SRD PATCH: Core Clarifications & Edge-Case Rulings =====
\section*{Rules Clarifications (Ambiguities Resolved)}
\addcontentsline{toc}{section}{Rules Clarifications}

\subsection*{1) Follower Assist — What counts, how to cap}
\paragraph{Source of dice.} Assist dice come from the \emph{helper}, not the leader. A follower contributes:
\begin{itemize}
  \item \textbf{In-role:} \textbf{+Cap} (their specialty rating), \emph{or}
  \item \textbf{Off-role (intricate)}: \textbf{+1} if the assist description is \emph{intricate} (see Description Ladder).
\end{itemize}
\textbf{Total Assist} on any roll (from any sources) remains hard-capped at \textbf{+3}. The leader’s own Skill rating does not limit or boost follower Assist. If multiple leader skills could apply, the table agrees on the \emph{primary action skill} for fiction; followers must justify against \emph{that} procedure.

\subsection*{2) Boon Economy — Caps, conversion, timing}
\paragraph{Holding cap.} You can hold at most \textbf{5 Boons}. If you would gain Boons beyond 5, you may \emph{spend} during this beat; any remainder is \emph{lost}.
\paragraph{Conversion.} \textbf{Once per session, in downtime}, you may convert \textbf{2 Boons $\rightarrow$ 1 XP} (max 1 XP via conversion per session). Conversions never occur mid-scene.

\subsection*{3) Asset Activations — Cost, plausibility, frequency}
\paragraph{Off-Screen effects.} Use each Asset’s listed Off-Screen effect \textbf{once per session} for free (unless the Asset says otherwise).
\paragraph{On-Screen activations.} To reshape the current scene (set \emph{start Controlled}, trim a named rail $-1$, or grant \emph{+1 effect} where the Asset clearly applies), spend \textbf{1 Boon}. XP is \emph{not} spent mid-scene.
\paragraph{Plausibility test.} The Asset must have \emph{scope} (this is the kind of thing it does) and \emph{reach} (it can matter here, now). If one is missing, the GM may call for a \emph{Setup} roll or rule it unavailable.

\subsection*{4) Over-Stack — When it triggers}
\paragraph{Structural advantages (examples).} Count each of the following present at scene start: \emph{active buff/tag}, \emph{favorable venue/pennant}, \emph{Follower Initiative unused}, \emph{on-screen Asset activation}, \emph{opponent disadvantaged by fiction}, \emph{ritual prep that applies now}.
\paragraph{Trigger.} If the crew enters a scene with \textbf{$\geq$ 3} structural advantages, apply \textbf{Over-Stack once} for that scene: either start one named rail at \textbf{+1} \emph{or} the GM banks \textbf{+1 CP} for the first \spadesuit~Twist. (Check again only if the crew adds \textbf{2 new} advantages mid-scene; max \textbf{1} Over-Stack per scene.)

\subsection*{5) Familiar Bond — Upkeep, “hot beats,” tracking}
Familiars use the standard \emph{Follower} Exposure/Harm tracks and require \emph{no upkeep}. “Flagged” is shorthand, not a separate status:
\begin{itemize}
  \item Each time a familiar \emph{acts on-screen in a high-risk beat} (CP generated \emph{or} a danger/Hazard rail is in play), mark \textbf{Exposure +1} on the familiar after the \textbf{second} such beat this scene.
  \item Otherwise treat costs exactly as any follower: an Initiative Action may mark \emph{Exposure +1} or \emph{Harm 1} (player’s choice unless fiction dictates).
\end{itemize}

\subsection*{6) Ritual Casting — Helper cap, skills, CP sharing}
\paragraph{Helper cap.} Maximum simultaneous helpers = $\lceil \text{primary caster’s Ritual/Arcana} / 2 \rceil$, \textbf{max 3}.
\paragraph{Relevant skills.} Helpers may use \emph{different} relevant skills (Ritual, Lore, Choir, Craft, etc.) if their procedure is fictionally distinct and contributes.
\paragraph{CP distribution.} CP from \emph{Channel} resolves on that roller. CP from \emph{Weave} is assigned to the \emph{primary caster}. Each helper may voluntarily \emph{absorb 1 CP} from Weave as \emph{Fatigue 1} (max 1 CP per helper).

\subsection*{7) Deck of Consequences — When to draw vs spend CP}
After a roll that generates CP, the GM chooses one method for that roll:
\begin{enumerate}
  \item \textbf{Direct Spend:} Translate CP into consequences/rail ticks immediately, \emph{or}
  \item \textbf{Deck Draw:} Draw 1 card per CP and synthesize a single twist consistent with the suits/ranks.
\end{enumerate}
Do not mix both methods on the \emph{same} roll. If a card plainly contradicts established fiction, \emph{redraw} \emph{or} reinterpret to the nearest equivalent pressure (same suit).

\subsection*{8) Casting Loop — Backlash sequence \& mitigation}
\paragraph{Order of operations.} \emph{Channel} (resolve effect + CP) $\rightarrow$ if Channel fails catastrophically, the spell may fizzle; otherwise \emph{Weave} (resolve effect) $\rightarrow$ apply Backlash from Weave.
\paragraph{Mitigation.} Boons \emph{do not} reduce CP unless a Talent/Asset explicitly says “Mitigate CP.” Talents like \emph{Backlash Resistant}, \emph{Elemental Focus}, \emph{Backlash Soothing} apply at the moment CP is assigned. Resolve Channel CP before attempting Weave.

\subsection*{9) Initiative Action — Costs, tracks, max}
\paragraph{Cost choice.} An Initiative Action by a follower costs either \textbf{Exposure +1} \emph{or} \textbf{Harm 1} (not both) unless the fiction mandates a specific cost.
\paragraph{Tracks.} \emph{Exposure} = heat/attention; clears in downtime by laying low or narrative relief. At \textbf{Exposure 3}, the follower is \emph{Compromised} (GM may remove from scene or tilt a rail against you) until cleared. \emph{Harm} = injury; at \textbf{Harm 3} the follower is out until treated.
\paragraph{Cadence.} By default, the crew has \textbf{1 Follower Initiative} window per scene (some Prestige may add a second). Using two followers on the same roll consumes the scene’s window.

\subsection*{10) Prestige Prerequisites — Meeting and losing reqs}
\paragraph{Qualifying.} Attribute/Skill prerequisites must be met with \emph{permanent} ratings (no temporary buffs/items). Asset prerequisites must be owned.
\paragraph{After purchase.} If you later lose the Asset prerequisite, you keep the Talent but cannot \emph{activate} features that require that Asset until restored. If your ratings later drop, you keep the Talent.

\subsection*{Secondary Points}
\paragraph{Banking CP.} The GM may bank CP only when a rule says so (e.g., Over-Stack) or when the table agrees a consequence is better \emph{deferred to entry} of the next venue. Banked CP expires at \emph{end of scene}.
\paragraph{Description Ladder.} \emph{Simple} = a single concrete action (“I shove the door”). \emph{Detailed} = adds a tool/angle (“I wedge the crowbar and lever”). \emph{Intricate} = two or more linked, specific procedures that change position/effect (“I chalk the hinge line, wedge the bar, and lift on count so the ward-line doesn’t scrape”). Off-role follower assists require \emph{intricate}.
\paragraph{Supply Clock.} Fill 1 segment when: (a) extended travel leg without provisioning, (b) failed logistics/procurement, or (c) invoking an Asset that explicitly consumes stores. Clear via Farm/Market/Logistics assets or downtime provisioning.
\paragraph{Fatigue.} Fatigue is cumulative. At \textbf{2+} you take \emph{-1 die} to strenuous physical actions; at \textbf{4} you are \emph{exhausted} (no sprints/climbs; most actions Risky at best). Clear 1–2 with rest, sanctuary, or assets as listed.
\paragraph{Asset “Significant Time.”} Means a \emph{downtime scene} (order of hours) at or in reach of the Asset; field-use may require a Setup roll.

% ===== End Patch =====

\section{Archetypes in Play}

\subsection*{The Solo}
Invests XP primarily in self.  
\emph{Strengths:} always ready, iconic spotlight.  
\emph{Risks:} narrow toolkit; may lag in social or resource scenes.  

\subsection*{The Mixed Player}
Balances XP between self and assets.  
\emph{Strengths:} adaptable, bridges party gaps.  
\emph{Risks:} upkeep spread thin.  

\subsection*{The Mastermind}
Builds networks, followers, and assets.  
\emph{Strengths:} broad reach, drives strategies.  
\emph{Risks:} Complication fallout; vulnerable allies.  

\subsection{Training \& Mentorship}

\begin{itemize}
  \item Training from a PC or NPC with Skill $\geq 3$ or Attribute $\geq$ new rating halves downtime.  
  \item A vivid training montage may convert 1 day of downtime into on-screen play (twice per arc).  
  \item Research or crafting may be handled as Significant Time with a simple project clock.  
\end{itemize}

\section{Optional Systems}

\subsection*{Patronage Track}
Spending 8+ XP on civic, guild, or temple projects grants a \emph{Patron tag}.  
Once per arc, a Patron may erase a resource-related Complication (Diamonds suit).

\subsection*{Respec \& Drift}
\begin{itemize}
  \item Light respec: Once per arc, redistribute up to 6 XP worth of Skills.  
  \item Major pivot: With narrative justification, reclaim one Talent or Prestige at 50\% refund.  
\end{itemize}

\section{End-of-Session Checklist}

\begin{enumerate}
  \item Did we reach a Major Objective? (+2--4 XP)  
  \item Who embraced Complications? (+1--3 XP)  
  \item Did Bonds or Flags drive scenes? (+1--2 XP)  
  \item Any discoveries or lore revealed? (+1--2 XP)  
  \item Attendance and GM Curveball awarded? (+2 +0--3 XP)  
  \item Any XP spending declared? Note downtime or begin Haste clocks.  
\end{enumerate}


% ======================================================
\part{Character Framework}
% ======================================================

\section{Starting XP \& Complications}

\subsection{Starting Pool}
Characters begin with a baseline of \textbf{30 XP} to allocate across Attributes, Skills, Talents, and Assets.

\subsection{Optional Trades}
At character creation, players may increase their starting XP by accepting story complications.  

\begin{itemize}
  \item \textbf{Obligation Deficit:} You may begin with a \emph{deficit of up to 2 XP}, representing debts, unfinished business, or personal weakness. This deficit must be paid off with your first earned XP before further advancement.  
  \item \textbf{Complication Trade:} You may accept up to \textbf{2 Complications} during Session 0, each granting +1 XP. Complications are drawn from the Deck of Consequences (or selected by the GM) and permanently attached to your backstory.  
\end{itemize}

\subsection{Starting Complications Table}
At Session 0, players may either draw randomly or roll 1d10 on this table for each Complication taken. The GM should weave these into the campaign’s opening act.

\begin{tabular}{|c|p{10cm}|}
\hline
\textbf{d10} & \textbf{Starting Complication} \\
\hline
1 & \textbf{Debt Mark:} You owe coin or favor to a guild, syndicate, or noble family. Collectors will come calling. \\
\hline
2 & \textbf{Broken Heirloom:} Begin play with one signature item already \emph{Compromised}. Repairing it is costly or politically fraught. \\
\hline
3 & \textbf{Enemy at Large:} Someone you wronged (or their heir) is alive and plotting. They will cross your path again. \\
\hline
4 & \textbf{Dark Patron:} A whispering spirit, godling, or demon has touched your life. It may offer help, but always demands a price. \\
\hline
5 & \textbf{Notoriety:} You are famous (or infamous) for something unsavory. People recognize you, and not always kindly. \\
\hline
6 & \textbf{Haunted:} Nightmares, visions, or a literal ghost troubles your sleep. Occasionally, this interferes with your actions. \\
\hline
7 & \textbf{Cursed Token:} You carry an item that brings ill luck. The GM may spend 1 Complication Point per session to invoke its curse. \\
\hline
8 & \textbf{Fractured Loyalty:} You belong to two groups, guilds, or families with conflicting demands. They will not tolerate neutrality forever. \\
\hline
9 & \textbf{Lost Ally:} A former follower, familiar, or friend is missing. You may one day recover them, but until then you carry the burden. \\
\hline
10 & \textbf{Blood Feud:} Your kin, clan, or culture is sworn against another. The feud may be dormant, but never gone. \\
\hline
\end{tabular}

\subsection{Example}
A player wants to begin with 32 XP.  
They accept 2 Complications: they roll a 2 (\emph{Broken Heirloom}) and a 7 (\emph{Cursed Token}).  
Their character begins with a bent Viterran longspear and an ominous carved stone.  
The GM now has two strong narrative hooks to introduce in play.

\subsection*{Backstory Bonds \& Initial Boons}

At character creation, players are encouraged to tie their backstories together.  
Each bond represents a shared history, debt, rivalry, or mutual respect that grounds the party in the world.  

\begin{itemize}
  \item \textbf{Initial Boon:} Two characters may gain \emph{one Boon each} if they establish a mutual backstory tie.  
  \item \textbf{Consent Required:} Both players must agree to the bond; it cannot be imposed.  
  \item \textbf{Maximum:} Each player may earn up to 4 Boons this way during character creation. They can also be added or removed between sessions, again with consent all around. 
  \item \textbf{In-Play Rewards:} A GM may award additional Boons or 1 XP when players call upon these ties in-session, provided the bond meaningfully shapes the scene.  
\end{itemize}

\paragraph{Examples of Bonds:}
\begin{itemize}
  \item You and another character once fought in the same doomed levy and survived together.  
  \item You saved another character’s kin from debt-slavery; they owe you, and you both know it.  
  \item You and another character were rivals in the same guild, and your rivalry sharpened you both.  
  \item You and another character share a forbidden secret that neither dares speak aloud.  
\end{itemize}

\noindent
\textbf{Design Note:} Boons gained from backstory are a narrative resource.  
They reinforce the social fabric of the party, and they encourage players to invoke personal history as part of ongoing play.

\subsection{Attributes \& Skills}

Characters in \textit{Fate’s Edge} are defined not by long columns of modifiers, but by the interplay of four \textbf{core attributes} and a lean, narrative-focused \textbf{skill list}. Together they shape dice pools, guide player expression, and reinforce the philosophy that \emph{who you are} matters as much as \emph{what you do}.

\section{Attributes}
Attributes represent broad aspects of a character’s being. They are archetypal, not granular, and apply across many situations.  

\begin{description}[leftmargin=2cm]
  \item[Body] Strength, endurance, and physical action. Lifting a gate, wrestling an ogre, or running across rooftops all call on Body.  
  \item[Wits] Perception, cleverness, and reaction speed. Spotting an ambush, solving a puzzle, or drawing first in a duel depend on Wits.  
  \item[Spirit] Willpower, intuition, and resilience. Resisting fear, sensing lies, or pressing on through pain come from Spirit.  
  \item[Presence] Charm, command, and social force. Inspiring allies, bending wills, or commanding a hall all rely on Presence.  
\end{description}

Each Attribute is rated from 1–5 for most mortals. Exceptional beings (angels, demons, dragons) may reach higher.  

\section{Skills}
Skills are focused expressions of talent. They sharpen attributes into action and give players narrative handles for describing their characters. Each Skill is tied loosely to one or more attributes, but creative players may justify alternative pairings if the narrative fits.

\subsection*{Skill List}
\begin{multicols}{2}
\begin{itemize}
  \item Athletics — climbing, running, swimming.
  \item Brawl — fists, grappling, improvised fighting.
  \item Melee — blades, axes, polearms.
  \item Ranged — bows, crossbows, thrown arms.
  \item Stealth — hiding, shadowing, evading.
  \item Survival — tracking, foraging, navigation.
  \item Craft — smithing, alchemy, tinkering.
  \item Lore — history, cultures, languages.
  \item Arcana — magical theory, rituals, spellwork.
  \item Insight — intuition, empathy, lie detection.
  \item Diplomacy — negotiation, mediation, etiquette.
  \item Deception — disguise, misdirection, bluffing.
  \item Command — leadership, intimidation, rallying.
  \item Performance — music, oratory, storytelling.
\end{itemize}
\end{multicols}

\subsection*{Skill Ratings}
\begin{itemize}
  \item \textbf{0: Untrained} — You rely on raw attribute alone.  
  \item \textbf{1: Familiar} — Basic competence, a journeyman’s touch.  
  \item \textbf{2: Skilled} — Trained and reliable in most circumstances.  
  \item \textbf{3: Expert} — Professional mastery or long experience.  
  \item \textbf{4: Master} — Renowned, your work or ability is widely recognized.  
  \item \textbf{5: Legendary} — Almost supernatural; few mortals reach this level.  
\end{itemize}

\subsection{Attributes and Skills in Play}
When players attempt a significant action, they combine the relevant \textbf{Attribute + Skill} to determine their dice pool. Attributes set the foundation of capability, while Skills give the roll texture and identity.
\begin{quote}
\textbf{Body + Melee} is raw steel and sinew.\\
\textbf{Wits + Melee} is anticipation and precision.\\
\textbf{Presence + Melee} is flourish, feint, and intimidation.
\end{quote}
The same Skill, seen through different Attributes, tells a different story.

\subsection{XP as Currency}
In \textit{Fate’s Edge}, Experience Points (XP) are not a tally of victories. They are the \textbf{universal resource of progression}—a currency spent to shape your character, your influence, and your place in the world. Growth always comes from choice: hone the self, strengthen bonds, or weave deeper ties into the setting.

\section{The Three Paths of XP}
Players may allocate XP across three broad categories. Each represents a different philosophy of growth.

\subsection{Enhance Self}
XP buys personal mastery.
\begin{description}[leftmargin=2cm]
  \item[Attributes] Raising an Attribute costs XP equal to \emph{(new rating $\times$ 3)}.\\
  \emph{Example:} Body 2 $\rightarrow$ 3 costs $3\times 3=9$ XP.
  \item[Skills] Raising a Skill costs XP equal to \emph{(new level $\times$ 2)}.\\
  \emph{Example:} Stealth 1 $\rightarrow$ 2 costs $2\times 2=4$ XP.
\end{description}

\subsection{Acquire Assets}
XP can be invested into property, holdings, networks, and followers—worldly influence rather than personal prowess.

\paragraph{Off-Screen Resources.}
Purchased with XP, these assets provide leverage \emph{between sessions}: a tavern, a title, a mercantile charter, a safehouse network. They solve problems off-screen but cannot intervene in an adventure scene.

\paragraph{On-Screen Followers (Assistance).}
Allies who act \emph{in the scene} and lend their specialty.
\begin{itemize}
  \item \textbf{Cost:} A follower with Specialty Cap $C$ costs \emph{$C^2$ XP} (one-time).
  \item \textbf{Specialty:} Define one narrow lane (e.g., \emph{Bodyguard in melee}, \emph{Scout on overland navigation}, \emph{Archivist for ancient scripts}). The follower only assists in that lane.
  \item \textbf{Assist Dice:} When applicable, the follower adds \emph{help dice} equal to $\min(C,\ \text{your relevant Skill})$, \emph{capped at +3 dice}.
  \item \textbf{Slot Limit:} Only \emph{one} follower may assist a given action.
  \item \textbf{Upkeep:} Each Downtime, pay Coin equal to $C$ \emph{or} spend a Scene tending the relationship. Miss two upkeeps: the follower becomes \emph{Unreliable} until mended.
  \item \textbf{Risk:} If the GM spends \emph{2+ Complication Points} on an action you take with assistance, they may endanger, injure, or separate the follower instead of you if fictionally appropriate.
\end{itemize}

\begin{designnote}
\textbf{Why $C^2$ and a +3 cap?} A Cap~5 helper used to be the best XP deal in the game. Squaring the cost and capping assist dice keeps followers powerful but \emph{not} more efficient than raising your own Skills. Followers become scene-shaping \emph{tools} with obligations and risks, rather than a cheap dice battery.
\end{designnote}

\subsection{Learn Talents}
Talents are unique abilities that expand what a character can do. They are purchased with XP and often serve as stepping stones toward Prestige Abilities. They may be:

\begin{itemize}
  \item \textbf{General Talents} — broadly available to any character.  
  \begin{itemize}
    \item \emph{Battle Instincts} (Cost: 6 XP): Once per scene, re-roll a failed defense roll.  
    \item \emph{Silver Tongue} (Cost: 4 XP): Gain +1 die when persuading or deceiving through speech.  
    \item \emph{Iron Stomach} (Cost: 3 XP): Immune to mundane poisons and spoiled food; halve Complications from toxic sources.  
  \end{itemize}

  \item \textbf{Racial or Cultural Talents} — tied to Affinities and philosophies, reflecting the arts of a people.  
  \begin{itemize}
    \item \emph{Stone-Sense} (Dwarves, Cost: 5 XP): Detect flaws in stone or earth; gain +1 die on Engineering or Craft rolls underground.  
    \item \emph{Backlash Soothing} (Wood Elves, Cost: 6 XP): Once per session, reduce a magical Backlash Complication by 2 points when in natural terrain.  
    \item \emph{Blood Memory} (Ykrul, Cost: 5 XP): After a battle, meditate to gain one temporary Skill die reflecting a foe’s tactics for the next scene.  
  \end{itemize}

  \item \textbf{Prestige Abilities} — narrative-capstone powers available only at high investment. These are the pinnacles of a culture’s arts.  
  \begin{itemize}
    \item \emph{Echo-Walker} (High Elf, Cost: 20 XP; Req: Wits 5, Arcana 4): Step briefly into Aerisahl; once per arc, turn any Complication into a boon.  
    \item \emph{Warglord} (Ykrul, Cost: 18 XP; Req: Body 5, Leadership 3): Rally scattered warbands into a single host; once per campaign, may unify tribes under one banner.  
    \item \emph{Spirit-Shield} (Dwarves, Cost: 15 XP; Req: Spirit 4, Resolve 3): Once per session, erase up to 3 Complications from an ally’s roll, taking 1 Backlash yourself.  
  \end{itemize}
\end{itemize}

\section*{Familiars \& the Bond}
\addcontentsline{toc}{section}{Familiars \& the Bond}

\subsection*{Lore (table-facing)}
Familiars are \textbf{spirits wearing a shape} — bird, beast, or clever construct — drawn by vows, names, and roles. They learn by \emph{doing} and remember by \emph{ritual}; mirrors and still water may show their true nature for a breath.

\subsection*{Familiar Bond (Talent)}
You are bound to one or more \textbf{Familiars}, lightweight Followers tuned to a single \textbf{Role}.

\begin{itemize}
  \item \textbf{Requirement:} This Talent is required to have any Familiar.
  \item \textbf{Upkeep:} None. Familiars do not consume upkeep.
  \item \textbf{Cap (number):} You may have at most \textbf{2} Familiars bound at once.
  \item \textbf{Stat line:} \emph{Role} (e.g., Scout, Mimic, Distract, Fetch, Sentry) \textbullet{} \emph{Cap} 1--2 \textbullet{} \emph{Bond} 1/2 \textbullet{} \emph{Exposure} 0/2.
  \item \textbf{Restrictions:} Familiars cannot be Stewards/Deputies, cannot hold assets, and cannot trigger Diamonds on their own.
  \item \textbf{Swap:} Rebinding a different Familiar mid-arc requires a short downtime scene; the old bond goes dormant until the next arc.
\end{itemize}

\paragraph{Durability \& Flags}
If a Familiar appears in \textbf{two or more} hot beats in an arc, mark \textbf{Flagged}; clear with a quiet mentorship/care scene or \textbf{2 XP}. Familiar Harm follows follower harm (Harm 1--2). If it would be taken out, it withdraws until treated.

\subsection*{Prestige Talent: Spirit Keeper}
\begin{itemize}
  \item \textbf{Prerequisite:} Familiar Bond; Tier III+.
  \item \textbf{Effect:} Your bound Familiar \emph{cap (number)} increases to \textbf{3}.
  \item \textbf{Note:} Individual Familiar \emph{Cap} (quality) remains 1--2 unless advanced by other rules. This talent increases the \emph{number} you can bind, not their personal Cap.
\end{itemize}

\subsection*{Unified Assist (Followers \& Familiars)}
Any \textbf{Follower} (including Familiars) may Assist a PC’s action:

\paragraph{In-Role Assist (specialty)}
\begin{itemize}
  \item \textbf{Bonus:} \textbf{+Cap} (the helper’s Cap value).
  \item \textbf{Gate:} Plausible fiction in their \emph{Role} (no intricate requirement).
\end{itemize}

\paragraph{Off-Role Assist (any skill) --- \emph{intricate gate}}
\begin{itemize}
  \item \textbf{Bonus:} \textbf{+1}.
  \item \textbf{Gate:} The Assist narration must be \textbf{intricate}, including three distinct details:
  \begin{enumerate}
    \item a \textbf{Sense} the helper uses (sight/sound/current/scent),
    \item the \textbf{Method} (how it helps),
    \item a \textbf{Risk} accepted (what could go wrong for it).
  \end{enumerate}
  \item \textbf{Note:} This intricate gate \emph{permits} the +1 off-Role bonus; it does not grant re-rolls by itself. Any existing re-roll rules still apply.
\end{itemize}

\paragraph{Assist stacking limits}
\begin{itemize}
  \item \textbf{Per roll cap:} Total Assist bonus from all sources is capped at \textbf{+3}.
  \item \textbf{Per helper cap:} A single helper cannot exceed its mode’s bonus (either \textbf{+Cap} in-Role \emph{or} \textbf{+1} off-Role).
\end{itemize}

\paragraph{Rails \& Over-Stack}
Assists never directly move rails. If the crew enters a scene with \textbf{2+ structural advantages}, apply the Over-Stack rule (start rails at +1 \emph{or} GM banks +1 CP on the first \spadesuit{} Twist).

\subsection*{Familiar Solo Beat}
\textbf{Once per scene (total across your Familiars)}, one Familiar may take a small \textbf{Solo Beat} appropriate to its form:

\begin{description}
  \item[Scout \& Signal:] change an ally’s next action \textbf{position to Controlled} \emph{or} grant \textbf{+1 effect}.
  \item[Distract \& Draw:] reduce a \textbf{kinetic rail} (Hunt/Escape/Hazard) by \textbf{--1 tick}.
  \item[Fetch \& Carry:] move a small, scene-relevant object across danger; on the recipient’s next success, \textbf{advance +1 tick} on the target clock.
\end{description}

\noindent \textbf{Cost:} mark \textbf{Exposure +1} on the Familiar \emph{or} mark \textbf{Harm 1} (scrape/shock). The GM may spend \textbf{1 CP} to escalate the risk in tense scenes (e.g., force a Resist or tick a rail if you press).

\subsection*{Role quick-list (examples)}
\begin{tabular}{@{}l l@{}}
\toprule
\textbf{Role} & \textbf{Typical in-Role help} \\
\midrule
Scout & Routes/chases, timing the surge, spotting cutters \\
Mimic & Voices/calls, baiting impostors, echo tricks \\
Distract & Crowd/curfew manipulation, misdirection, flare \\
Fetch & Lines/keys/notes ferried under pressure \\
Sentry & Watch/warn, early hazard callouts \\
\bottomrule
\end{tabular}

\subsection*{Examples} 
\begin{itemize}
  \item \textbf{In-Role (Scout, Cap 2):} Raven spirals ahead, marks the counter-tide with two dips, risks a net near the mast \textrightarrow{} \textbf{+2} Assist on a chase.
  \item \textbf{Off-Role (Mimic):} Perched over the dais, copies a host’s lilt to bait an impostor; if they throw a shoe, it’s in range \textrightarrow{} \textbf{+1} Assist on a social roll (intricate met).
  \item \textbf{Solo Beat:} Distract \& Draw \textrightarrow{} \emph{Hunt --1 tick}, mark Exposure +1 on the Familiar.
\end{itemize}

% ===== SRD Addendum: Followers as Story Agents =====
\section*{Followers as Story Agents}
\addcontentsline{toc}{section}{Followers as Story Agents}

\subsection*{Design Principle}
\begin{quote}
\textbf{Followers Are Story Agents, Not Stat Blocks.}
Followers in Fate's Edge are flexible narrative tools that enhance the fiction. Their specialties give mechanical clarity, but their true value is how they create choices, complications, and heart.
\end{quote}

\subsection*{Assistance Tiers (Unified)}
\label{sec:assist-tiers}
Followers (including Familiars) assist under three flexible modes:

\paragraph{In-Role Excellence} When acting squarely within their \emph{Role}, a follower grants \textbf{+Cap} to the roll (helper’s Cap value). Plausible fiction is enough.

\paragraph{Narrative Flexibility (Off-Role)} With an \emph{intricate} description, a follower can help adjacent or non-core tasks for \textbf{+1}.
\emph{Intricate} means the narration includes:
\begin{enumerate}\setlength\itemsep{0pt}
  \item a \textbf{Sense} the follower uses,
  \item the \textbf{Method} of help,
  \item a concrete \textbf{Risk} they accept.
\end{enumerate}
This gate permits the +1; it does not grant re-rolls by itself.

\paragraph{Character Moments (Initiative Action)} A follower may take a small independent action (see \S\ref{sec:follower-initiative}) that fits their nature.

\paragraph{Assist caps} Per roll, total Assist bonuses are capped at \textbf{+3}. A single helper cannot exceed its mode’s bonus (either \textbf{+Cap} in-Role or \textbf{+1} Off-Role). Assists never move rails directly. Over-Stack still applies if the crew brings \textbf{2+} structural advantages into the scene.

\subsection*{Follower Initiative Actions}
\label{sec:follower-initiative}
Rename “Familiar Solo Beat” to \textbf{Follower Initiative Action} and make it core:

\begin{itemize}\setlength\itemsep{2pt}
  \item \textbf{Limit:} \textbf{Once per scene across the crew}, one on-screen follower may take an Initiative Action (does not require a player to “spend” their action).
  \item \textbf{Pick 1 effect (appropriate to form/Role):}
    \begin{description}\setlength\itemsep{0pt}
      \item[Scout \& Signal:] Change an ally’s next action \textbf{position to Controlled} \emph{or} grant \textbf{+1 effect}.
      \item[Distract \& Draw:] Reduce a \textbf{kinetic rail} (Hunt/Escape/Hazard) by \textbf{–1 tick}.
      \item[Fetch \& Carry:] Move a small, relevant object through danger; on the recipient’s next success, \textbf{advance +1 tick} on the target clock.
    \end{description}
  \item \textbf{Cost:} Mark \textbf{Exposure +1} on that follower \emph{or} mark \textbf{Harm 1} (scrape/shock).
  \item \textbf{Pressure:} The GM may spend \textbf{1 CP} to escalate risk (e.g., force a Resist or tick a rail if you press).
\end{itemize}

\noindent\textit{Heroic Variant:} Tables wanting more follower spotlight may allow \textbf{up to 2} Initiative Actions per scene; the second \emph{must} take a Cost and grants the GM \textbf{+1 CP}. Never more than 2.

\subsection*{Exposure, Harm, \& Flags (Standardized)}
All on-screen followers track \textbf{Bond (2–3)}, \textbf{Exposure (2–4)}, and \textbf{Harm (1–2)}.
If a follower appears in \textbf{2+ hot beats} in an arc, mark \textbf{Flagged}; clear with a mentorship/care scene or \textbf{2 XP}.

\subsection*{Narrative Upkeep States}
Treat follower condition as fiction-first states (mapped to tracks):

\begin{description}\setlength\itemsep{0pt}
  \item[Maintained:] engaged and reliable. (\textit{Bond} steady, \textit{Exposure} 0, no fresh Harm)
  \item[Neglected:] strain shows; needs time or a gesture. (\textit{Exposure} \(\ge\) 1 or \textit{Bond} at 0)
  \item[Compromised:] major story turn—capture, betrayal, departure. (\textit{Exposure} maxed \emph{or} a hard Complication hits)
\end{description}

\noindent\textbf{Repair:} Clear \emph{Neglected} with a short scene (time, gift, apology) or \textbf{2 XP}. Clear \emph{Compromised} with a dedicated scene plus a \textbf{public cost} or sacrifice named in fiction.

\subsection*{GM Principles for Followers}
\begin{itemize}\setlength\itemsep{0pt}
  \item Say “\textbf{yes}” to creative uses that fit the fiction; scale bonus by appropriateness (In-Role vs Off-Role).
  \item Use \textbf{complications to create tension}, not to punish creativity.
  \item Let followers \textbf{have arcs}: wants, doubts, and lines they won’t cross.
  \item When Over-Stack triggers, start rails at +1 \emph{or} bank +1 CP for the first \(\spadesuit\) Twist—followers count toward that advantage tally.
\end{itemize}

\subsection*{Examples in Play}
\paragraph{The Scout (Cap 3 — Navigation)}
\begin{itemize}\setlength\itemsep{0pt}
  \item \textbf{In-Role Assist:} calls counter-tide and handholds \(\to\) \textbf{+3} to a hazardous crossing.
  \item \textbf{Off-Role Assist (intricate):} reads crowd eddies to slip an escort through \(\to\) \textbf{+1}.
  \item \textbf{Initiative Action:} Distract \& Draw \(\to\) \emph{Hunt –1 tick}; marks \emph{Exposure +1}.
  \item \textbf{Complication:} gets separated at a switchback—start \emph{Rescue(4)} or spend a Boon.
\end{itemize}

\paragraph{The Mimic (Cap 2 — Voices)}
\begin{itemize}\setlength\itemsep{0pt}
  \item \textbf{In-Role Assist:} baits an impostor with a perfect lilt \(\to\) \textbf{+2}.
  \item \textbf{Off-Role Assist (intricate):} signals rope cadence with bird-calls \(\to\) \textbf{+1}.
  \item \textbf{Initiative Action:} Scout \& Signal \(\to\) next ally starts \textbf{Controlled}.
\end{itemize}

\subsection*{Advancement Paths (Story-First)}
Followers grow through \textbf{relationship and role}, not just numbers:
\begin{itemize}\setlength\itemsep{0pt}
  \item \textbf{Develop the bond:} specific scenes unlock one-use Boons or a once/arc tag.
  \item \textbf{New Role through story:} a Scout who leads drills might add \emph{Distract} as a secondary lane (Cap unchanged).
  \item \textbf{Promotion}: Apprentice \(\to\) Associate \(\to\) Steward/Deputy via milestones (once/arc benefits).
  \item \textbf{Become a Plot Anchor:} on a big beat, a follower can take a seat (venue tag, charter clause) or retire as a powerful Ally.
\end{itemize}

% ==== CORE: Action Economy (insert near top of Actions & Resolution) ====
\subsection*{Action Economy at a Glance}
\begin{itemize}
  \item \textbf{Player Action} — a PC attempts something; position/effect apply.
  \item \textbf{Assist} — an ally or follower adds dice: \emph{In-Role} \textbf{+Cap} or \emph{Off-Role (intricate)} \textbf{+1}, up to \textbf{+3} total.
  \item \textbf{Follower Initiative} — \emph{once per scene across the crew}, one on-screen follower takes a small independent action that changes position, trims a rail, or ferries an object (see p.~\pageref{sec:follower-initiative-core}).
\end{itemize}

\subsection{Clarifications}
% ==== Minimal Clarifications ====
\paragraph{XP Timing.} Advancement purchases occur in \textbf{downtime only}. The Boon-to-XP conversion (2 Boons → 1 XP) may be used \textbf{once per session}.

\paragraph{Follower Initiative vs Coordinated Assist.} If \textbf{two followers} assist the \textbf{same roll} (e.g., via a teamwork talent), it \textbf{consumes this scene’s Follower Initiative}. Total Assist from all sources still caps at \textbf{+3}.

\subsection*{Familiars (Talent) — Compatibility Note}
\emph{Familiar Bond} remains required to have any Familiars; they still have \textbf{no upkeep} and a \textbf{hard cap of 2} (or \textbf{3} with \emph{Spirit Keeper} prestige). Familiars now use these \textbf{same} Assist tiers and Initiative Action rules. Individual Familiar \emph{Cap (quality)} remains 1–2 unless advanced by other rules.

% ==== Example: Avyra's Sentinel Raven ====
\paragraph{Example — Sentinel Raven (Scout, Cap 2).}
On the rope-span, the raven spirals ahead, calls the counter-tide, and risks a net near the mast.
\emph{Initiative: Scout \& Signal} → next ally starts \textbf{Controlled}. Cost: \textbf{Exposure +1}.
Later, Avyra still gets \emph{Assist (In-Role)} \textbf{+2} from the same raven on a chase—Assist and Initiative are distinct.

% ==== CORE: Follower Initiative (promoted to Core) ====
\subsection*{Follower Initiative}
\label{sec:follower-initiative-core}
Followers are story agents. In addition to Assisting, they may take an \textbf{Initiative Action} as a small, independent beat.

\paragraph{Limit}
\textbf{Once per scene across the crew}, exactly one on-screen follower may take an Initiative Action. It does not consume a PC’s action.

\paragraph{Pick 1 effect (fiction must fit Role/form)}
\begin{description}
  \item[Scout \& Signal:] Change an ally’s next action \textbf{position to Controlled} \emph{or} grant \textbf{+1 effect}.
  \item[Distract \& Draw:] Reduce a \textbf{kinetic rail} (Hunt/Escape/Hazard) by \textbf{--1 tick}.
  \item[Fetch \& Carry:] Move a small, relevant object through danger; on the recipient’s next \emph{success}, \textbf{advance +1 tick} on the target clock.
\end{description}

\paragraph{Cost \& Pressure}
Mark \textbf{Exposure +1} on that follower \emph{or} mark \textbf{Harm 1} (scrape/shock). The GM may spend \textbf{1 CP} to escalate: force a Resist, or tick a rail if you press.

\paragraph{Guardrails}
Initiative never directly completes the \emph{Primary} clock. It can set up allies, trim rails, or move objects in danger. If the crew enters a scene with \textbf{2+ structural advantages}, apply \emph{Over-Stack}: start rails at +1 \emph{or} the GM banks +1 CP on the first \(\spadesuit\) Twist. Off-screen followers cannot take Initiative Actions.

\paragraph{Heroic Variant (optional)}
Tables wanting more follower spotlight may allow \textbf{up to 2} Initiative Actions per scene; the second \emph{must} pay a Cost and grants the GM \textbf{+1 CP}. Never more than 2.

\section{Economy of Choice}
Because XP is the sole currency, every expenditure forces tension:  

\begin{itemize}
  \item Invest in yourself to become unmatched in one arena.  
  \item Invest in the world to gain influence and resources.  
  \item Invest in talents to expand into new, unique possibilities.  
\end{itemize}

There is no “wrong” choice — only the question: \emph{Will you be remembered for what you mastered, what you owned, or what you became?}

\section{Archetypes \& Prestige Abilities}

Archetypes represent broad playstyles, while Prestige Abilities are the capstone talents tied to culture, philosophy, or exceptional mastery.  
This section offers examples across the tiers of advancement to demonstrate how XP investment and narrative identity converge.

\subsection{Archetype Examples}

\subsection*{The Blade-Seeker (Solo Archetype)}
\begin{description}[leftmargin=2cm]
  \item[Theme:] The lone duelist whose personal mastery becomes legend.  
  \item[XP Focus:] Attributes (Body, Wits) and direct combat skills.  
  \item[Play Impact:] Thrives in spotlight scenes; less reliable for logistics or diplomacy.  
  \item[Prestige Gateways:] Duelist’s Insight, Echo-Killer.  
\end{description}

\subsection*{The Bond-Keeper (Mixed Archetype)}
\begin{description}[leftmargin=2cm]
  \item[Theme:] A hero who balances self-growth with allies, networks, or family.  
  \item[XP Focus:] Skills and Off-Screen Assets, with some Attributes raised.  
  \item[Play Impact:] Strongest when weaving personal hooks into group play.  
  \item[Prestige Gateways:] Hearth-Banner, Oath-Bound Captain.  
\end{description}

\subsection*{The Spider (Mastermind Archetype)}
\begin{description}[leftmargin=2cm]
  \item[Theme:] The strategist who commands webs of allies, followers, and secrets.  
  \item[XP Focus:] On-Screen Followers, Off-Screen Assets, Presence.  
  \item[Play Impact:] Shapes campaigns through schemes, but risks backlash if resources collapse.  
  \item[Prestige Gateways:] Shadow Broker, Master of Coin.  
\end{description}

\section{Prestige Ability Examples by Tier}

\subsection*{Tier II (41--90 XP)}
\begin{description}[leftmargin=2cm]
  \item[Duelist’s Insight]  
  \emph{Req: Body 3, Melee 3.}  
  Once per duel, re-roll all failed dice if you describe a flourish tied to your rival’s weakness.  

  \item[Hearth-Banner]  
  \emph{Req: Presence 2, Leadership 2, Off-Screen Asset (Homestead).}  
  Allies defending your home or banner gain +1 die to all rolls within its borders.  
\end{description}

\subsection*{Tier III (91--150 XP)}
\begin{description}[leftmargin=2cm]
  \item[Oath-Bound Captain]  
  \emph{Req: Presence 3, Leadership 3, one On-Screen Follower.}  
  When leading a unit, you may convert one Complication per roll into a rallying boon: grant +1 die to all allies on their next action.  

  \item[Echo-Killer]  
  \emph{Req: Wits 4, Arcana 3, Affinity (Elven).}  
  Once per scene, negate an incoming magical effect by paying its Complication cost yourself.  
\end{description}

\subsection*{Tier IV (151--220 XP)}
\begin{description}[leftmargin=2cm]
  \item[Shadow Broker]  
  \emph{Req: Presence 4, Subterfuge 3, Off-Screen Asset (Network).}  
  Once per arc, declare that you ``already had an agent there.''  
  Spend 6 XP to permanently expand your Network’s reach into that region.  

  \item[Spirit-Shield]  
  \emph{Req: Spirit 4, Resolve 3.}  
  Once per session, you may erase up to 3 Complication Points from an ally’s roll, at the cost of taking 1 Backlash Complication yourself.  
\end{description}

\subsection*{Tier V (221+ XP)}
\begin{description}[leftmargin=2cm]
  \item[Master of Coin]  
  \emph{Req: Presence 5, Commerce 4, three Off-Screen Assets.}  
  You may treat wealth as narrative leverage: once per session, erase a Diamonds-suit Complication by spending from your fortune.  

  \item[Echo-Walker]  
  \emph{Req: Wits 5, Arcana 4, Affinity (High Elf).}  
  Once per arc, step briefly into the demi-plane of Aerisahl, gaining one automatic success on a roll \emph{and} reshaping one drawn Complication into a boon.  
\end{description}

\subsection{Talent Progression by Tier}

Talents are organized into tiers of investment. Early Talents broaden a character’s toolkit, Mid-Tier Talents deepen specialization, and Prestige Abilities represent cultural pinnacles and narrative milestones.  

\begin{table}[h]
\centering
\renewcommand{\arraystretch}{1.3}
\begin{tabular}{|p{3cm}|p{2cm}|p{3cm}|p{6cm}|}
\hline
\textbf{Name} & \textbf{Cost (XP)} & \textbf{Requirements} & \textbf{Effect} \\
\hline
\multicolumn{4}{|c|}{\textbf{Early Talents (3–5 XP)}} \\
\hline
Battle Instincts & 4 & None & Once per scene, re-roll a failed defense roll. \\
Silver Tongue & 3 & Presence 2+ & Gain +1 die on Persuasion/Deception rolls. \\
Stone-Sense & 5 & Dwarf only & Detect flaws in stone/earth; +1 die on underground Craft/Engineering rolls. \\
Iron Stomach & 3 & Body 2+ & Resist mundane poisons; halve Complications from toxins or spoiled food. \\
\hline
\multicolumn{4}{|c|}{\textbf{Mid-Tier Talents (6–10 XP)}} \\
\hline
Backlash Soothing & 6 & Wood Elf only, Spirit 2+ & Once per session, reduce a magical Backlash by 2 points in natural terrain. \\
Blood Memory & 7 & Ykrul only, Body 3+ & After a battle, gain one temporary Skill die reflecting a foe’s tactics in the next scene. \\
Commanding Presence & 8 & Presence 3+ & Followers gain +1 to morale rolls; allies re-roll one failed die when obeying your orders. \\
Familiar Bond & 9 & Spirit 3+ & Gain a magical familiar (Cap 3). Counts as an On-Screen follower. \\
\hline
\multicolumn{4}{|c|}{\textbf{Prestige Abilities (12+ XP)}} \\
\hline
Echo-Walker & 20 & High Elf only; Wits 5, Arcana 4 & Step briefly into Aerisahl. Once per arc, transform a Complication into a boon. \\
Warglord & 18 & Ykrul only; Body 5, Leadership 3 & Rally scattered warbands; once per campaign, unify tribes under one banner. \\
Spirit-Shield & 15 & Dwarf only; Spirit 4, Resolve 3 & Once per session, erase up to 3 Complications from an ally’s roll; take 1 Backlash yourself. \\
Shadowbinder & 16 & Tulkani or Dark Elf; Wits 4, Stealth 3 & Manipulate shadow as substance; once per session, negate visibility or tracking entirely. \\
\hline
\end{tabular}
\caption{Talent progression tiers with examples.}
\end{table}
\subsection{Cultural Talent Examples}

Each culture offers its own path of growth. Early Talents establish identity, Mid-Tier Talents deepen specialization, and Prestige Abilities embody cultural myths and pinnacles of mastery.

\begin{table}[h]
\centering
\renewcommand{\arraystretch}{1.3}
\begin{tabular}{|p{3cm}|p{2cm}|p{3cm}|p{6cm}|}
\hline
\textbf{Name} & \textbf{Cost (XP)} & \textbf{Requirements} & \textbf{Effect} \\
\hline
\multicolumn{4}{|c|}{\textbf{Humans}} \\
\hline
Versatile & 3 & None & Once per session, swap one Skill die for another you lack. \\
Guild Ties & 7 & Presence 2+, Membership & Call in favors from a guild or order once per session. \\
Banner-Bearer & 15 & Presence 4+, Leadership 3 & Inspire an allied unit: once per battle, erase 2 Complications for the group. \\
\hline
\multicolumn{4}{|c|}{\textbf{Dwarves (Aeler)}} \\
\hline
Stone-Sense & 5 & Dwarf only & Detect flaws in stone; +1 die to underground Craft/Engineering rolls. \\
Spirit Shield & 8 & Spirit 3+ & Commune with ancestors; once per session, block 1 Complication for an ally. \\
Forge-Patriarch & 18 & Body 4+, Craft 4 & Found a forge-citadel; gain loyal smiths and engineers who count as On-Screen specialists (Cap 5). \\
\hline
\multicolumn{4}{|c|}{\textbf{Wood Elves}} \\
\hline
Backlash Soothing & 5 & Wood Elf only & Once per session, cancel 1 Backlash die in natural terrain. \\
Ranger’s Step & 9 & Wits 3+, Stealth 2 & Move unseen in forests; treat terrain penalties as one step lower. \\
Wild Speaker & 18 & Spirit 5+, Nature 4 & Speak to beasts and trees; once per session, command local wildlife as allies. \\
\hline
\multicolumn{4}{|c|}{\textbf{High Elves}} \\
\hline
Lorekeeper & 4 & High Elf only & Recall obscure history or magic without rolling once per session. \\
Weave Anchor & 8 & Wits 3+, Arcana 3 & Reduce magical Backlash by 1 when casting Intricate spells. \\
Echo-Walker & 20 & High Elf only; Wits 5, Arcana 4 & Step briefly into Aerisahl; once per arc, turn a Complication into a boon. \\
\hline
\multicolumn{4}{|c|}{\textbf{Ykrul}} \\
\hline
Blood Frenzy & 4 & Body 2+ & When reduced to half health, gain +1 die on melee rolls. \\
Blood Memory & 7 & Body 3+ & After battle, gain 1 temporary Skill die reflecting a foe’s tactics in the next scene. \\
Warglord & 18 & Body 5, Leadership 3 & Rally scattered warbands; once per campaign, unify tribes under one banner. \\
\hline
\multicolumn{4}{|c|}{\textbf{Tulkani}} \\
\hline
Wanderer’s Luck & 3 & Tulkani only & Once per session, re-roll a failed travel or fortune roll. \\
Shadowbinder’s Touch & 8 & Wits 3+, Stealth 2 & Manipulate shadow as substance; gain +1 die in concealment rolls. \\
Shadowbinder & 16 & Wits 4+, Stealth 3 & Command shadows fully; once per session, negate visibility or tracking entirely. \\
\hline
\multicolumn{4}{|c|}{\textbf{Other Folk}} \\
\hline
Halfling Hearthbond & 4 & Halfling only & Once per session, restore an ally’s Spirit die pool by sharing food or song. \\
Gnomish Numeromancy & 7 & Gnome only, Wits 3+ & Once per session, apply “mathematical foresight” to re-roll up to 2 dice in a planning scene. \\
Dwarves’ Ally (Human) & 6 & Any, History 2 & Gain safe passage and trade credit in Dwarven protectorates. \\
\hline
\end{tabular}
\caption{Example Talents by culture and tier.}
\end{table}

\section{Followers \& Off-Screen Assets}
\label{chap:followers-assets}

\subsection{Concept}
Followers (on-screen allies) and Off-Screen Assets (titles, safehouses, networks) are powerful—but they require upkeep. Instead of coin ledgers, \emph{Fate’s Edge} uses narrative condition tracks, downtime, and XP to model drift, loyalty, and wear.

\subsection*{Design Aims}
\begin{itemize}
  \item \textbf{Agency First:} Assets amplify characters; they never replace on-screen presence (see \S\ref{core-contract} \emph{Unbreakable Contract}).
  \item \textbf{Fiction Is the Ledger:} Stress, scarcity, and politics degrade resources; players repair with \emph{time} or \emph{XP}.
  \item \textbf{Meaningful Tradeoffs:} Keep a network sharp or grow your stats—you usually can’t do both.
\end{itemize}

\subsection{Buying Followers \& Assets}
Followers use a simple \textbf{Cap} rating (their specialty dice). Assets use \textbf{Tiers}. XP costs below include creation/buy-in.

\subsection{Followers (On-Screen)}
\begin{description}[leftmargin=3.2cm]
  \item[Cap 1] 3 XP — Competent assistant (e.g., Porter, Squire).
  \item[Cap 2] 5 XP — Trained specialist (e.g., Scout, Bodyguard).
  \item[Cap 3] 8 XP — Veteran operative (e.g., Spymaster’s Agent).
  \item[Cap 4] 12 XP — Elite aide (e.g., Master-at-Arms).
  \item[Cap 5] 17 XP — Exceptional lieutenant (rare).
\end{description}
\textbf{Assist Dice:} When the follower’s specialty applies, add a bonus pool up to their Cap. The PC still rolls the action; followers don’t take turns away from players.

\paragraph{Limits \& Liabilities}
\begin{itemize}
  \item \textbf{Spotlight Cap:} A PC may benefit from at most \textbf{one} follower’s assist per roll.
  \item \textbf{Exposure:} Any roll that uses a follower exposes them to \emph{Complications}; the GM may spend CP to \emph{Harm} or \emph{Degrade} that follower (see \S\ref{upkeep}).
\end{itemize}

\subsection{Off-Screen Assets}
\begin{description}[leftmargin=3.2cm]
  \item[Minor] 4 XP — Safehouse, small shop charter, parish patron, petty title.
  \item[Standard] 8 XP — Noble title, guild section, spy ring, estated farm.
  \item[Major] 12 XP — City license/monopoly, regional network, fortress lease.
\end{description}
\textbf{Use:} Assets solve problems between scenes, seed clues, or grant \emph{+1--2 dice} when you \emph{personally} leverage them on-screen. Assets never play the scene for you.

\paragraph{Face Requirement}
To buy an Asset, you must show personal capability:
\begin{itemize}
  \item \textbf{Civic/Title:} Presence 3+ or Command 2+.
  \item \textbf{Spy Network:} Wits 3+ or Subterfuge 2+.
  \item \textbf{Mercantile Charter:} Presence 3+ or Diplomacy 2+.
  \item \textbf{Workshop/Hold:} Spirit 3+ or Craft 2+.
\end{itemize}

\subsection{Condition Tracks \& Upkeep}
\label{upkeep}
Each Follower or Asset has a \textbf{Condition} clock:
\[
\textit{Maintained} \rightarrow \textit{Neglected} \rightarrow \textit{Compromised}
\]

\subsubsection{Triggering Degradation}
At the end of an arc (or after heavy use), the GM flags any resource that was:
\begin{itemize}
  \item \textbf{Heavily Used} (front-line risk, repeated favors, overclocked),
  \item \textbf{Put at Risk} (nearly burned, hunted by creditors/guards),
  \item \textbf{Under Strain by Default} (titles, networks, and rents decay without tending).
\end{itemize}
Flagged resources degrade one step unless \emph{Maintained} by Time or XP (below).

\subsubsection{Maintaining a Resource}
Choose one per flagged resource:
\begin{description}[leftmargin=3.2cm]
  \item[Downtime] Spend \textbf{Significant Time} personally (≈ a week) training, smoothing politics, repairing. No parallel training/projects.
  \item[XP Injection] Pay \textbf{XP equal to Tier/Cap} (e.g., Cap~3 follower costs 3 XP; Standard Asset costs 8 XP).
\end{description}

\subsubsection{Effects by Condition}
\begin{description}[leftmargin=3.2cm]
  \item[Maintained] Full capability; no penalties.
  \item[Neglected] \textbf{-1 die} when used (assist or leverage). Narratively: slower, sullen, short-staffed.
  \item[Compromised] Unavailable. Narratively: captured, burned, seized, defected.
\end{description}

\subsubsection{Repair \& Recovery}
\begin{itemize}
  \item \textbf{Neglected} $\rightarrow$ \textbf{Maintained}: Downtime \emph{or} pay \textbf{$\lceil$Tier/2$\rceil$ XP}.
  \item \textbf{Compromised}: Requires a \textbf{Quest/Operation} to recover the person/asset \emph{then} Downtime \textbf{or} full \textbf{Tier/Cap XP} to restore to \textit{Maintained}.
\end{itemize}

\subsection{Stress, Harm, \& Loss (GM Tools)}
When a follower is used in a risky scene, the GM may spend Complication Points to:
\begin{description}[leftmargin=3.2cm]
  \item[Pin] The follower is separated/boxed out; no assist next roll/scene.
  \item[Wound] The follower is \textbf{Injured}: until treated off-screen, their Cap counts as 1 lower.
  \item[Burn] Mark \textbf{Neglected} immediately (blown cover, angry creditors).
  \item[Seize] Escalate to \textbf{Compromised} (capture, flight, betrayal) if dramatically earned.
\end{description}
\textbf{PC Choice Lever:} The GM should \emph{offer the player a save}: protect the follower (accept a harsher on-screen complication for the PC) or let the follower take the hit.

\subsection{Loyalty \& Bonds (Optional)}
Track a simple \textbf{Loyalty} tag per follower: \emph{Wary} / \emph{Steady} / \emph{Devoted}. 
\begin{itemize}
  \item \textbf{Raise:} Meaningful risk shared, fair pay, defended in public.
  \item \textbf{Lower:} Broken promises, scapegoating, repeated neglect.
\end{itemize}
\textit{Devoted} followers can \emph{once per arc} convert one GM Complication targeting them into a lesser setback; \textit{Wary} followers cost +1 XP to Maintain.

\subsection{Promotion \& Replacement}
\begin{itemize}
  \item \textbf{Promote:} Pay the \emph{difference in XP} to raise a follower’s Cap; requires a brief training or milestone scene.
  \item \textbf{Replace:} Buying a similar new follower costs full XP (they don’t know your ways; loyalty starts \emph{Wary}).
\end{itemize}

\subsection{Quick Reference Tables}

\subsubsection*{Follower Cost \& Upkeep}
\begin{center}
\begin{tabular}{lccc}
\toprule
\textbf{Follower} & \textbf{Buy (XP)} & \textbf{Maintain (XP)} & \textbf{Repair Neglected (XP)} \\
\midrule
Cap 1 & 3 & 1 & 1 \\
Cap 2 & 5 & 2 & 1 \\
Cap 3 & 8 & 3 & 2 \\
Cap 4 & 12 & 4 & 2 \\
Cap 5 & 17 & 5 & 3 \\
\bottomrule
\end{tabular}
\end{center}

\subsubsection*{Asset Cost \& Upkeep}
\begin{center}
\begin{tabular}{lccc}
\toprule
\textbf{Asset Tier} & \textbf{Buy (XP)} & \textbf{Maintain (XP)} & \textbf{Repair Neglected (XP)} \\
\midrule
Minor & 4 & 2 & 1 \\
Standard & 8 & 4 & 2 \\
Major & 12 & 6 & 3 \\
\bottomrule
\end{tabular}
\end{center}
\noindent
\emph{Compromised} always requires a mission \emph{plus} the Maintain cost or Downtime.

\subsection{Examples in Play}

\subsubsection*{The Mastermind’s Agent (Cap 3)}
After a heist, the GM flags the Cap~3 agent. The player can:
\begin{enumerate}
  \item Spend a week of Downtime to mentor and cover tracks; stays \emph{Maintained}, or
  \item Pay 3 XP to keep them sharp, or
  \item Do nothing: the agent becomes \emph{Neglected} (assist rolls at -1 die) and a creditor shows up next use.
\end{enumerate}

\subsubsection*{The Noble’s Title (Standard Asset)}
A Standard title constantly grinds. At arc end, it’s flagged. The player:
\begin{itemize}
  \item Pays 4 XP to keep retainers loyal, or
  \item Spends a week in court, delaying personal training, or
  \item Lets it slip to \emph{Neglected}: -1 die when leveraging the title; rivals whisper.
\end{itemize}

\subsection{GM Guidance}
\begin{itemize}
  \item \textbf{Flag, Don’t Flog:} Only mark resources that mattered this arc.
  \item \textbf{Offer the Fork:} Present a clear \emph{time vs. XP} choice every time.
  \item \textbf{Let It Bite:} If players neglect resources, let fallout drive new adventures.
  \item \textbf{Keep It Human:} Followers are people. Name them, give them a want, and let loyalty shift.
\end{itemize}

\subsection{Activating Off-Screen Assets}
\label{sec:asset-activation}

Off-Screen Assets (titles, safehouses, spy rings, charters, workshops, etc.) are levers you can pull to affect the world. They do not act on their own; you \emph{activate} them when the story demands.

\subsubsection*{The Cost}
\begin{itemize}
  \item \textbf{To activate any Off-Screen Asset}, the player must spend \textbf{1 Boon} \emph{or} \textbf{2 XP}.
  \item This cost is per \emph{distinct ask} or \emph{discrete scene effect} (see Scope below).
  \item Spending XP represents burning favors, cashing ledgers, or indebting the future; it is the ``break-glass'' option.
\end{itemize}

\subsubsection*{Scope \& Examples}
An activation accomplishes \emph{one clear outcome} that the asset is plausibly able to produce. Assets are spent off-screen, generally between sessions.
\begin{itemize}
  \item \textbf{Safehouse Network (Standard):} ``Hide us for the night and get us past the cordon at dawn.'' (1 Boon)
  \item \textbf{Spy Ring (Standard):} ``Pull the magistrate’s travel plan for tomorrow.'' (1 Boon)
  \item \textbf{Noble Title (Standard):} ``Secure an audience this afternoon.'' (1 Boon) \emph{OR} ``Quash these charges quietly.'' (2 XP, high heat)
  \item \textbf{Workshop (Minor):} ``Rush a field repair to remove the \emph{Compromised} tag on our mail.'' (1 Boon)
  \item \textbf{Mercantile Charter (Major):} ``Float a short-term convoy credit to bribe the garrison and open the gate at dusk.'' (1 Boon; 2 XP if no Boons remain)
\end{itemize}

\subsubsection*{Limits \& Stacking}
\begin{itemize}
  \item \textbf{One Asset, One Outcome:} Each activation buys one concrete effect. A second, different effect requires another activation.
  \item \textbf{No Off-Screen Auto-Wins:} Assets \emph{enable} outcomes; they do not bypass scenes wholesale. If an activation moves the fiction to a new scene, play that scene.
  \item \textbf{Plausibility Gate (GM):} If an ask is beyond the asset’s plausible reach, the GM may: (a) refuse, (b) offer a reduced effect for the same cost, or (c) set a \emph{Setup} requirement (a short scene or roll) and then allow activation.
\end{itemize}

\subsubsection*{Complications Still Apply}
Activation guarantees the \emph{asset cooperates}; it does not guarantee a frictionless world.
\begin{itemize}
  \item The GM may still spend \emph{Complication Points} (from rolls in the scene) to attach consequences to the activation’s outcome (e.g., ``Yes, the smuggler opens the culvert—\emph{and} the river swells at midnight'').
  \item If an activation is the only lever in the moment (no roll), the GM may draw from the \emph{Deck of Consequences} at low severity to lace in texture without nullifying the effect.
\end{itemize}

\subsection*{Interaction with Condition Tracks}
Activation is separate from upkeep (see \S\ref{upkeep}).
\begin{itemize}
  \item \textbf{Neglected:} The asset functions, but any roll \emph{leveraging} it suffers \textbf{-1 die}.
  \item \textbf{Compromised:} The asset cannot be activated until recovered (quest/operation) and restored (Downtime or XP).
  \item \textbf{Emergency Spend:} You may still pay 2 XP to attempt an activation of a \emph{Neglected} asset; the penalty applies. \emph{Compromised} assets cannot be activated.
\end{itemize}

\subsection*{Why Boons?}
Boons represent \emph{story credit} earned by embracing risk (see \S Boons). Tying assets to Boons:
\begin{itemize}
  \item Rewards players who play into complications.
  \item Keeps XP focused on character growth, making 2 XP activations rare, high-drama choices.
\end{itemize}

\subsection*{Quick Reference}
\begin{center}
\begin{tabular}{ll}
\toprule
\textbf{Action} & \textbf{Cost} \\
\midrule
Activate any Off-Screen Asset & 1 Boon \emph{or} 2 XP \\
Activate again for a second, distinct outcome & +1 Boon \emph{or} +2 XP \\
Activate while Asset is Neglected & Normal cost; related rolls at -1 die \\
Activate while Asset is Compromised & \textit{Not possible} (recover first) \\
\bottomrule
\end{tabular}
\end{center}

\subsection*{GM Guidance}
\begin{itemize}
  \item \textbf{Say what it costs, then ask:} ``Your spy ring can get that letter—\emph{1 Boon or 2 XP}?'' If they pay, it happens.
  \item \textbf{Keep the scene alive:} Activation changes position, not authorship. Use it to \emph{open doors}, then play what’s inside.
  \item \textbf{Reserve 2 XP moments:} When a player spends XP to fire an asset, spotlight it like a spell: show the favors called, the ledger smudged, the bridge burned.
\end{itemize}

\subsection{Activating Off-Screen Assets}
\label{sec:asset-activation}

Off-Screen Assets (titles, safehouses, spy rings, charters, workshops, etc.) are levers you can pull to affect the world. They do not act on their own; you \emph{activate} them when the story demands.

\subsubsection*{The Cost}
\begin{itemize}
  \item \textbf{To activate any Off-Screen Asset}, the player must spend \textbf{1 Boon} \emph{or} \textbf{2 XP}.
  \item This cost is per \emph{distinct ask} or \emph{discrete scene effect} (see Scope below).
  \item Spending XP represents burning favors, cashing ledgers, or indebting the future; it is the ``break-glass'' option.
\end{itemize}

\subsubsection*{Scope \& Examples}
An activation accomplishes \emph{one clear outcome} that the asset is plausibly able to produce.
\begin{itemize}
  \item \textbf{Safehouse Network (Standard):} ``Hide us for the night and get us past the cordon at dawn.'' (1 Boon)
  \item \textbf{Spy Ring (Standard):} ``Pull the magistrate’s travel plan for tomorrow.'' (1 Boon)
  \item \textbf{Noble Title (Standard):} ``Secure an audience this afternoon.'' (1 Boon) \emph{OR} ``Quash these charges quietly.'' (2 XP, high heat)
  \item \textbf{Workshop (Minor):} ``Rush a field repair to remove the \emph{Compromised} tag on our mail.'' (1 Boon)
  \item \textbf{Mercantile Charter (Major):} ``Float a short-term convoy credit to bribe the garrison and open the gate at dusk.'' (1 Boon; 2 XP if no Boons remain)
\end{itemize}

\subsubsection*{Limits \& Stacking}
\begin{itemize}
  \item \textbf{One Asset, One Outcome:} Each activation buys one concrete effect. A second, different effect requires another activation.
  \item \textbf{No Off-Screen Auto-Wins:} Assets \emph{enable} outcomes; they do not bypass scenes wholesale. If an activation moves the fiction to a new scene, play that scene. Between-session activations have between-session consequences. 
  \item \textbf{Plausibility Gate (GM):} If an ask is beyond the asset’s plausible reach, the GM may: (a) refuse, (b) offer a reduced effect for the same cost, or (c) set a \emph{Setup} requirement (a short scene or roll) and then allow activation.
\end{itemize}

\subsubsection*{Complications Still Apply}
Activation guarantees the \emph{asset cooperates}; it does not guarantee a frictionless world.
\begin{itemize}
  \item The GM may still spend \emph{Complication Points} (from rolls in the scene) to attach consequences to the activation’s outcome (e.g., ``Yes, the smuggler opens the culvert—\emph{and} the river swells at midnight'').
  \item If an activation is the only lever in the moment (no roll), the GM may draw from the \emph{Deck of Consequences} at low severity to lace in texture without nullifying the effect.
\end{itemize}

\subsubsection*{Interaction with Condition Tracks}
Activation is separate from upkeep (see \S\ref{upkeep}).
\begin{itemize}
  \item \textbf{Neglected:} The asset functions, but any roll \emph{leveraging} it suffers \textbf{-1 die}.
  \item \textbf{Compromised:} The asset cannot be activated until recovered (quest/operation) and restored (Downtime or XP).
  \item \textbf{Emergency Spend:} You may still pay 2x XP to attempt an activation of a \emph{Neglected} asset; the penalty applies. \emph{Compromised} assets cannot be activated.
\end{itemize}

\subsubsection*{Why Boons?}
Boons represent \emph{story credit} earned by embracing risk (see \S Boons). Tying assets to Boons:
\begin{itemize}
  \item Rewards players who play into complications.
  \item Keeps XP focused on character growth, making 2 XP activations rare, high-drama choices.
  \item Boons are capped at 5 total at any given time.
\end{itemize}

\subsection*{Quick Reference}
\begin{center}
\begin{tabular}{ll}
\toprule
\textbf{Action} & \textbf{Cost} \\
\midrule
Activate any Off-Screen Asset & 1 Boon per XP Cost \emph{or} 2 XP \\
Activate again for a second, distinct outcome & +1 Boon \emph{or} +2 XP \\
Activate while Asset is Neglected & Normal cost; related rolls at -1 die \\
Activate while Asset is Compromised & \textit{Not possible} (recover first) \\
A Player may only carry 5 Boons at a time, any extra boon is converted to XP, 2 to 1, up to 2 (so, up to 4 boons) instantly and added at the end of a session.
\bottomrule
\end{tabular}
\end{center}

\subsection*{GM Guidance}
\begin{itemize}
  \item \textbf{Say what it costs, then ask:} ``Your spy ring can get that letter—\emph{1 Boon or 2 XP}?'' If they pay, it happens.
  \item \textbf{Keep the scene alive:} Activation changes position, not authorship. Use it to \emph{open doors}, then play what’s inside.
  \item \textbf{Reserve 2 XP moments:} When a player spends XP to fire an asset, spotlight it like a spell: show the favors called, the ledger smudged, the bridge burned.
\end{itemize}

\section{Table Contract \& Player Presence}

\subsection{The Unbreakable Contract}
\textbf{Every player-character must have on-screen agency during play.}
\begin{itemize}
  \item \textbf{Agency:} Each player must have meaningful choices to make for their character every session.
  \item \textbf{Presence:} The character appears in scenes and participates directly. Off-screen assets never replace the character’s involvement.
\end{itemize}
\noindent
\textit{Design intent:} Investment in holdings, titles, and networks is powerful, but it augments a character who acts on-screen. “Pure patrons” who never leave their estate are not valid PCs.

\subsection{The Hands-On Patron}
\textbf{Archetype intent:} A PC who invests heavily in Off-Screen Assets but \emph{leads from the front}: traveling, negotiating, and facing danger personally.

\subsubsection{Mechanical Requirements}
\begin{description}[leftmargin=2cm]
  \item[Face Requirement] You cannot purchase an Off-Screen Asset unless you meet a relevant threshold:
    \begin{itemize}
      \item \textbf{Noble Title (Standard):} \textit{Presence 3+} or \textit{Command 2+}
      \item \textbf{Spy Network (Major):} \textit{Wits 3+} or \textit{Subterfuge 2+}
      \item \textbf{Mercantile Charter (Major):} \textit{Presence 3+} or \textit{Diplomacy 2+}
      \item \textbf{Stonehold (Minor):} \textit{Spirit 3+} or \textit{Craft 2+}
    \end{itemize}
  \item[No Empty Chair] A character sheet must include on-screen capabilities (Attributes, Skills, or Followers) sufficient to act in scenes. Assets are tools, not a proxy PC.
\end{description}

\subsubsection{Example Patron Build (31 XP)}
\begin{itemize}
  \item \textbf{Presence 3} (9 XP), \textbf{Command 2} (6 XP total)
  \item \textbf{Off-Screen:} Noble Title (Standard) (8 XP), Safehouse (Minor) (4 XP)
  \item \textbf{On-Screen Follower:} Bodyguard (Cap 2) (4 XP)
\end{itemize}
\noindent
Result: 5-die social/command pools, portable authority, and a modest on-screen retinue—always present, never abstracted away.

\subsection{Off-Screen Assets vs. On-Screen Followers}
\begin{description}[leftmargin=2cm]
  \item[Off-Screen Assets] Holdings resolved between sessions (keeps, charters, titles, spy webs). They \textbf{create opportunities} and \textbf{solve background problems} but do not intervene mid-scene.
  \item[On-Screen Followers] Stat-light allies with a \textit{Skill Cap} who add dice in their specialty when present. They are \textbf{narrative liabilities} and \textbf{can be threatened, swayed, or lost}.
\end{description}

\subsection{Death, Replacement \& Downtime}
When a PC is slain or removed from play, choose one:

\subsubsection{Promote a Follower}
\begin{itemize}
  \item The player takes control of a loyal follower already established on-screen.
  \item Grant a lump sum equal to \textbf{50\% of the deceased PC's total XP}.
  \item Transfer a \textit{reasonable} share of assets explicitly tied to that follower (retainers, a stipend, or a minor charter).
\end{itemize}

\subsubsection{Heir or Partner Steps In}
\begin{itemize}
  \item Introduce an heir, sibling, apprentice, or business partner foreshadowed by the fiction.
  \item Start with \textbf{20–30 XP} plus a \textit{fraction} of Off-Screen Assets that make sense (with GM sign-off).
\end{itemize}

\subsubsection{New Character}
\begin{itemize}
  \item Create a fresh PC by standard rules.
  \item GM integrates them swiftly: tie to an existing Asset, contract, or faction introduced by the party.
\end{itemize}

\subsection{Temporary Absences (Captured, Lost, Waylaid)}
\begin{itemize}
  \item Use a \textbf{Peril Clock}: each session absent ticks it forward; when full, a hard consequence lands (ransom due, injury, asset seized).
  \item Offer targeted rescue/escape scenes to restore agency quickly.
\end{itemize}

\subsection{GM Guidance: Keeping the Contract}
\begin{itemize}
  \item Frame scenes where \textbf{every} PC has a lever: talk, scout, strike, scheme, or safeguard.
  \item When Assets would solve a problem off-screen, \textbf{surface a new on-screen complication}: a guild veto, a rival’s counter-claim, a courier intercepted.
  \item Spotlight balance: rotate beats between Solo specialists, Mixed players, and Masterminds; ensure no one dictates outcomes without table consent.
\end{itemize}

\subsection{Designer Note}
This contract is the spine of play. XP economies and Assets broaden stories; they must never replace the character’s presence at the table. If a build trends toward absenteeism, redirect with the Face Requirement, on-screen costs, and scene framing that demands the player’s direct choices.

\section{Boons and Failure's Gift}

\subsection*{Core Idea}
In \textit{Fate’s Edge}, failure does not mean nothing happens — it always moves the story forward.  
To emphasize this, every time a player suffers a **Complication from failure** (not just a rolled \texttt{1}, but an actual failed action with narrative fallout), they also earn a **Boon**.

\subsection*{What is a Boon?}
A Boon is a small token of narrative resilience, representing how characters learn from mistakes, seize unexpected opportunities, or grow under pressure.

\subsection*{Spending Boons}
Boons are flexible, but limited. They can be spent in two ways:

\subsection*{Limits}
A character may carry up to five boons at any given time. Four can be gained from interlocking backstories between sessions for the next session. So, a player can gain 5 during play, spend them, and gain four... or gain four, or some combination.

\begin{description}[leftmargin=2cm]
  \item[Re-roll Opportunity:] During a session, a player may spend \textbf{1 Boon} to re-roll a single die (success or failure) after rolling. This must be declared immediately.
  \item[Asset Activation:] To activate an Off-Screen Asset for a specific purpose, a player must spend \textbf{1 Boon} (preferred) or \textbf{2 XP} (emergency use). This guarantees cooperation within the asset’s scope, though the GM may introduce fitting complications.
  \item[Experience Conversion:] Between sessions, a player may convert \textbf{2 Boons = 1 XP}. This reflects growth from adversity and encourages long-term investment.
\end{description}

\subsection*{Guidelines}
\begin{itemize}
  \item A Boon is only awarded when failure has narrative teeth — the GM must introduce a complication or setback.
  \item Boons are tracked individually, not as a group resource.
  \item The GM may offer additional narrative color when Boons are earned (“You stumble on the climb, but you notice a hidden sigil carved into the stone — mark a Boon.”).
  \item A character may only have 4 Boons max, overflow are converted: 2 boons to 1 XP.
\end{itemize}

\subsection*{Design Philosophy}
Boons ensure that \emph{failure still rewards play}. They turn setbacks into fuel for later triumphs, embodying the core philosophy of \textit{Fate’s Edge}: every consequence carries opportunity.  

However, Boons are earned only through \textbf{meaningful engagement}. The attempt must matter in the fiction—taking risks, advancing the story, or revealing character. Empty actions taken solely to trigger failure do not qualify. This prevents players from ``fishing'' for Boons, an ineffective and disruptive form of griefing.  

In short: \emph{fail forward, not sideways.} Every Boon represents a lesson paid for with genuine dramatic weight.

\begin{tcolorbox}[title=GM Callout: Awarding Boons,colback=black!2,colframe=black!40!white]
  \textbf{Ask yourself:} \emph{Did this action move the fiction forward?}  
  
  \begin{itemize}
    \item If \textbf{yes}, award a Boon—even if the roll failed.  
    \item If \textbf{no}, deny it—this was fishing, not play.  
  \end{itemize}
  
  Boons mark dramatic risks taken, not empty gestures.
  \end{tcolorbox}

% ======================================================
\section{Magic \& The Arts}
% ======================================================

\subsection{Philosophy of Magic}

Magic in \textit{Fate’s Edge} is not a tool of convenience but a dangerous negotiation with the fabric of reality.  
It is powerful, flexible, and transformative — yet every attempt to shape it carries risk. The dice never merely ask \emph{“does it work?”} but always whisper \emph{“what is the cost?”}

\subsection{The Nature of Magic}
\begin{itemize}
  \item \textbf{Volatile by Design:} Magic is not fully understood, even by its most adept practitioners. Every working pushes against boundaries that resist being bent.  
  \item \textbf{Risk Embodied:} Each spell generates Complication Points. These points do not vanish; they manifest as \emph{Backlash}, unpredictable consequences that ripple outward.  
  \item \textbf{Narrative Weight:} Casting is always a story moment. Even a “successful” spell alters the scene in ways the caster did not intend.  
  \item \textbf{Thematic Consequence:} Backlash is not arbitrary; it aligns with the opposing or uncontrolled element of the Art invoked (flame flares out of control, shadows linger too long, storms roll beyond command).  
\end{itemize}

\subsection{The Caster’s Burden}
Magicians are defined not by what they can do, but by what they are willing to risk.  
A cautious spellcaster describes carefully, invests in detailed actions, and may survive long. A reckless one courts power at great personal and narrative cost. Both choices shape the story.

\subsection{Casting Loop}

All spellcasting follows a structured sequence called the \emph{Casting Loop}.  
It unfolds across two phases of play: gathering strength, then weaving it into form.

\begin{enumerate}
  \item \textbf{Channel} — The caster focuses, rolling \textbf{Wits + Arcana} to gather \emph{Potential}.  
  Each success becomes fuel for shaping the spell. Each \texttt{1} adds Complication Points immediately.  
  \item \textbf{Weave} — On the following turn, the caster rolls \textbf{Wits + (Art)} to shape Potential into a defined effect.  
  The Description Ladder applies: Basic/Detailed/Intricate descriptions reduce or redirect Complication Points.  
  \item \textbf{Backlash} — Complication Points spent by the GM manifest as uncontrolled consequences.  
  These are thematic to the Art and scale with the number of points spent: minor nuisances at low levels, dangerous disasters at high levels.  
\end{enumerate}

\subsection{Example of Backlash}
\begin{description}[leftmargin=2cm]
  \item[Fire] Flames leap to unattended surfaces, smoke blinds allies, or the heat weakens structures.  
  \item[Shadow] Illusions persist too long, unseen things whisper truths best left hidden, morale crumbles.  
  \item[Storm] Winds scatter allies’ plans, lightning arcs toward unintended targets, storms linger beyond the caster’s will.  
\end{description}

\subsection{Design Intent}
The Casting Loop keeps magic tense and thematic: no spell is “free.”  
Every magical act alters not just the world, but the flow of narrative itself.

\subsection{Ritual Casting (Optional Rule)}

\subsubsection*{Concept}
Some workings of magic are too great for a single will. 
A ritual allows multiple characters to join forces, pooling their dice and narrative effort to achieve extraordinary effects—
but the risk of backlash rises with every participant.  

\paragraph{Ritual Helper Cap:} 
You may draw on ceil(Arcana/2) helpers (max 3).
Each helper contributes as per the ritual rules and adds their Complication risk.

\subsubsection*{Procedure}

\begin{enumerate}
  \item \textbf{Declare the Ritual:} The primary caster names the effect and how others can help. The GM confirms whether the scope is appropriate for a ritual. 
  \item \textbf{Channel Together:}
    \begin{itemize}
      \item Primary caster rolls \emph{Wits + Arcana}. 
      \item Each assistant rolls \emph{Spirit + Relevant Skill} (Lore, Prayer, Craft, etc.). 
      \item Each success adds +1 Potential. 
      \item Each \texttt{1} contributes a Complication Point (CP) to the \textbf{shared ritual pool}.
    \end{itemize}
  \item \textbf{Weave:} The primary caster rolls \emph{Wits + (Art)}. Assistants may add dice if their contributions are narrated in play. Each die rolled by an assistant can also generate \texttt{1}s.
  \item \textbf{Backlash:} Total all CPs. 
    \begin{itemize}
      \item Apply normal Complication spending rules. 
      \item Increase severity by +1 tier per assistant beyond the first.
    \end{itemize}
\end{enumerate}

\subsubsection*{Why Use Rituals?}
\begin{itemize}
  \item \textbf{Higher Ceiling:} Rituals can achieve effects impossible through normal spellcasting (summoning storms, collapsing fortresses, teleporting armies).
  \item \textbf{Shared Spotlight:} Every participant has narrative agency in the casting. 
  \item \textbf{Bigger Risk:} More dice mean more \texttt{1}s. Consequences can spread across the entire party or region.
\end{itemize}

\subsubsection*{Ritual Backlash Table}
When resolving Backlash from a ritual, the GM may draw from the Deck of Consequences or roll on this table.
Severity rises with the number of participants: each additional caster/assistant raises the effective tier by one. 

\begin{tabular}{|c|p{10cm}|}
\hline
\textbf{Tier} & \textbf{Example Consequences} \\
\hline
1 -- Minor & Fainting fits, minor burns, hair turns white, illusions ripple out briefly. \\
\hline
2 -- Moderate & A random assistant suffers Fatigue; uncontrolled elemental effects lash outward; a local animal or object becomes possessed. \\
\hline
3 -- Severe & The ritual area is scarred (blight, flooding, fire); one participant is marked by otherworldly attention; a spirit slips loose. \\
\hline
4 -- Major & A participant becomes a vessel for hostile forces; the effect echoes uncontrollably (affecting unintended targets); a divine or arcane rival takes notice. \\
\hline
5+ -- Catastrophic & The ritual succeeds but catastrophically overreaches: weather patterns shift for weeks; a planar breach opens; a godlike being demands recompense. \\
\hline
\end{tabular}

\subsubsection*{Example at the Table}
The party seeks to collapse a bridge to halt an advancing army.
\begin{itemize}
  \item \textbf{Caster:} Channels lightning through the stone. 
  \item \textbf{Fighter:} Engraves runes into the pillars (Body + Craft). 
  \item \textbf{Cleric:} Anchors the backlash with prayer (Spirit + Faith). 
  \item \textbf{Rogue:} Offers blood as a focus (Presence + Subterfuge). 
\end{itemize}

The bridge shatters, stopping the army—\emph{but five CPs mean severe fallout: the lightning spreads into nearby homes, the rogue’s blood offering attracts a spirit, and the cleric feels the weight of a god’s judgment}. 

The victory stands, but new story threads are born.


% ======================================================
\section{GM Toolkit}
% ======================================================
\subsection{Logistics as Drama}

\subsection{Core Principle: The Fiction Is the Ledger}
In \textit{Fate’s Edge}, arrows, rations, and waterskins are tracked only in the
fiction. Mechanics engage only when those resources become \textbf{scarce}.
The focus is always narrative tension, not bookkeeping.

\subsection{The Supply Clock}
A shared condition for the entire party, the Supply Clock represents food,
water, and basic gear.

\begin{description}[leftmargin=2cm]
  \item[Full Supply (0 filled)] The party is well-equipped. No penalties.
  \item[Low Supply (2 filled)] Minor narrative complications: bland food,
    damaged arrows, thinning waterskins.
  \item[Dangerously Low (3 filled)] Each character gains \textbf{Fatigue}.
  \item[Out of Supply (4 filled)] Severe penalties; starvation, dehydration,
    failing gear.
\end{description}

\subsubsection*{Filling the Clock}
\begin{itemize}
  \item Harsh travel or lost pack animals (GM fiat).
  \item GM spends \textbf{2+ Complication Points}.
  \item The party chooses to travel light for advantage.
\end{itemize}

\subsubsection*{Emptying the Clock}
\begin{itemize}
  \item Reaching civilization resets to \textbf{Full}.
  \item Foraging/hunting: group Survival check clears 1 segment.
  \item Downtime in safety removes 1 segment.
\end{itemize}

\subsection{Fatigue}
Fatigue represents exhaustion, hunger, and strain.

\begin{description}[leftmargin=2cm]
  \item[Effect:] On their next roll, a character must reroll one success.
  \item[Stacking:] Each level adds another forced reroll.
  \item[Recovery:] A night’s rest with adequate supply removes 1 Fatigue.
    Fatigue cannot be removed while the party is \textbf{Dangerously Low}.
\end{description}

\subsection{Gear Damage}
Gear does not have hit points. It suffers only when drama demands it.

\subsubsection*{Compromised Items}
\begin{itemize}
  \item Introduced via Complication Points or narrative consequence.
  \item A \textbf{Compromised} item gives \(-1\) die on relevant rolls.
\end{itemize}

\subsubsection*{Breaking Point}
If a Compromised item suffers another setback, it breaks entirely.

\subsubsection*{Repair}
\begin{description}[leftmargin=2cm]
  \item[Field Repair:] Temporary; Craft or Survival check removes penalty for
    one scene.
  \item[Proper Repair:] Permanent; requires tools, materials, and downtime.
\end{description}

\subsection{XP and Logistics Assets}
Upkeep between adventures is assumed trivial unless the party is destitute.
XP investments make logistics part of the character’s story.

\begin{description}[leftmargin=2cm]
  \item[4 XP: Signature Weapon] A named heirloom or crafted masterpiece.
    If lost or damaged, recovery becomes a personal story hook.
  \item[8 XP: Superior Workshop] Off-screen asset. Enables permanent repairs,
    masterwork crafting, and party-wide gear quality during downtime.
\end{description}

\subsubsection{Example in Play}
\begin{quote}
Three days into a desert crossing, the GM fills two Supply segments. The party
is now at \textbf{Low Supply}.  
A player scouts for water, rolling two \texttt{1}s. The GM spends those
Complications to fill another segment—\textbf{Dangerously Low}. Everyone gains
\textbf{Fatigue}.  
Later, in combat, the fatigued archer rerolls one success. The arrow flies wide,
the lack of water written into the dice themselves.
\end{quote}


\section{Deck of Consequences}

The Deck of Consequences is a shared tool that externalizes risk and narrative fallout. 
Whenever a check generates Complication Points, the GM may draw from the deck instead of (or in addition to) improvising outcomes. 
This provides consistent tone and escalating tension.

\subsection{Structure of the Deck}
\begin{itemize}
  \item \textbf{Suits = Domains of Complications}
    \begin{description}
      \item[\ding{171} Cups] Emotional, social, or relational fallout.
      \item[\ding{171} Swords] Harm, danger, or escalation of conflict.
      \item[\ding{171} Pentacles] Resource strain, economic or material cost.
      \item[\ding{171} Wands] Magical, spiritual, or cosmic disturbances.
    \end{description}
  \item \textbf{Ranks = Severity of Complications}  
  Higher ranks indicate more enduring or severe consequences:
  \begin{description}
    \item[Ace–3] Minor inconvenience or flavor complication.
    \item[4–6] Moderate setback with some narrative teeth.
    \item[7–9] Significant consequence altering the course of action.
    \item[10–King] Major fallout, introducing new problems or lasting scars.
  \end{description}
\end{itemize}

\subsection{Using the Deck}
\begin{enumerate}
  \item Player rolls; each \texttt{1} generates a Complication Point.
  \item GM may draw a card for each Complication Point.
  \item The suit frames the type of complication; the rank determines severity.
  \item GM interprets and narrates based on context.
\end{enumerate}

\subsection{Guidance for Mundane Skills}
Not every task should rely on freeform GM invention. The following table offers examples to keep complications consistent across play.

\begin{table}[h]
\centering
\begin{tabular}{|p{2cm}|p{3cm}|p{7cm}|}
\hline
\textbf{Skill} & \textbf{Complication Domain} & \textbf{Examples by Severity} \\
\hline
Athletics & Swords (harm) &  
Ace–3: Twisted ankle;  
4–6: Lose grip, drop equipment;  
7–9: Injury forces pause or retreat;  
10–King: Serious wound, out of action. \\
\hline
Stealth & Cups (social) or Swords (harm) &  
Ace–3: A creak alerts suspicion;  
4–6: Minor evidence left behind;  
7–9: Spotted and pursued;  
10–King: Ambushed or captured. \\
\hline
Crafting & Pentacles (resources) &  
Ace–3: Wasted materials;  
4–6: Minor flaw, item unreliable;  
7–9: Precious resource ruined;  
10–King: Catastrophic failure or danger from collapse/explosion. \\
\hline
Persuasion & Cups (emotional/social) &  
Ace–3: Offended someone mildly;  
4–6: Reputation takes a small hit;  
7–9: Relationship strained;  
10–King: Alliance broken, enmity gained. \\
\hline
Scholarship & Wands (cosmic) or Pentacles (resources) &  
Ace–3: Misremember detail;  
4–6: Hours wasted chasing false lead;  
7–9: Dangerous knowledge misapplied;  
10–King: Catastrophic misunderstanding with lasting fallout. \\
\hline
\end{tabular}
\caption{Mundane Skills and Example Complications}
\end{table}

\section{Player Archetypes at the Table}

Not every group plays the same way. These archetypes describe \emph{how} players spend XP and seek spotlight. They are styles, not classes; most characters drift between them across a campaign.

\subsection{Assumptions for Examples}
Unless noted, examples assume a 30~XP starting budget.
\begin{description}[leftmargin=2cm]
  \item[Costs:] Attributes = (new rating $\times$ 3), Skills = (new level $\times$ 2), Followers = $C^2$ (Cap $C$), Assist bonus = $\min(C,\text{your Skill})$ up to +3 dice.
  \item[Off-Screen Assets:] Minor 4~XP (single town, narrow help), Standard 8~XP (regional reach), Major 12~XP (multi-region, political teeth). Assets act between sessions; no direct scene dice.
\end{description}

\subsection{The Solo}
\begin{description}[leftmargin=2cm]
  \item[Definition:] Invests XP primarily in \textbf{Attributes and Skills}. Minimal followers, minimal holdings. All power is \emph{on the sheet}.
  \item[Typical XP Spread:] 70--90\% Self; 0--10\% On-screen help; 0--20\% Off-screen.
\end{description}

\subsubsection*{Solo: Example Build (30 XP)}
\begin{itemize}
  \item Raise \textbf{Body} 2$\rightarrow$3: $3\times 3=9$ XP
  \item Raise \textbf{Wits} 2$\rightarrow$3: $9$ XP \hfill (\emph{18 XP total})
  \item Skill \textbf{Melee} 0$\rightarrow$3: $2+4+6=12$ XP
  \item \emph{Remaining:} 0 XP
\end{itemize}

\textbf{Result:} Melee pool 6 (Body~3 + Melee~3) \emph{or} finesse pool 6 (Wits~3 + Melee~3).  
With \emph{Intricate} descriptions, the Solo rerolls all \texttt{1}s, minimizing Complications and staying reliable.

\paragraph{Strengths} Consistent scene impact; few moving parts; resilient to follower loss.  
\paragraph{Risks} Limited fiction reach between sessions; can stall when problems demand logistics or networks.  
\paragraph{GM Guidance} Feed high-ceiling tests (tight targets, layered stakes). Reward description---let \emph{Intricate} actions shine. Present \emph{hard} softlocks (codes, customs) that nudge teamwork or creative reframing.

% -------------------------------------------------------------

\subsection{The Mixed Player}
\begin{description}[leftmargin=2cm]
  \item[Definition:] Splits XP between self-growth and \textbf{one or two} meaningful assets (a small follower or a reliable holding).
  \item[Typical XP Spread:] 50--65\% Self; 15--25\% On-screen help; 15--25\% Off-screen.
\end{description}

\subsubsection*{Mixed: Example Build (30 XP)}
\begin{itemize}
  \item Raise \textbf{Presence} 2$\rightarrow$3: $9$ XP
  \item Skill \textbf{Sway} 0$\rightarrow$3: $2+4+6=12$ XP \hfill (\emph{21 XP total})
  \item \textbf{Follower} (Cap~3 \emph{Archivist}): $3^2=9$ XP
\end{itemize}

\textbf{Result:} Social pool 6 (Presence~3 + Sway~3).  
In scenes of texts, records, or lore, the Archivist can assist for up to \textbf{+3 dice} (one assistant total), provided the approach uses their specialty.

\paragraph{Strengths} Versatile: credible in scenes \emph{and} has a lever for special problems.  
\paragraph{Risks} Upkeep pressure; helper can be targeted when the GM spends 2+ Complication Points.  
\paragraph{GM Guidance} Present multi-key problems: one key favors the PC’s sheet, the other their follower/asset. Let off-screen prep create \emph{edges}, not free wins.

% -------------------------------------------------------------

\subsection{The Mastermind}
\begin{description}[leftmargin=2cm]
  \item[Definition:] Prioritizes \textbf{followers/cadres/familiars} and \textbf{off-screen networks}. The sheet is the hub of a larger apparatus.
  \item[Typical XP Spread:] 25--40\% Self; 35--55\% On-screen help; 20--40\% Off-screen.
\end{description}

\subsubsection*{Mastermind: Example Build (30 XP)}
\begin{itemize}
  \item Raise \textbf{Wits} 2$\rightarrow$3: $9$ XP
  \item Skill \textbf{Tactics} 0$\rightarrow$1: $2$ XP
  \item Skill \textbf{Insight} 0$\rightarrow$2: $2+4=6$ XP \hfill (\emph{17 XP total})
  \item \textbf{Follower} (Cap~3 \emph{Scout/Ranger}): $3^2=9$ XP
  \item \textbf{Off-Screen Asset} (Minor Safehouse Ring): $4$ XP
\end{itemize}

\textbf{Result:} In navigation/stealth approaches, the Scout can assist for up to \textbf{+3 dice} (one assistant total) when the method fits their lane.  
The Safehouse Ring solves travel/logistics in downtime (activation: \emph{1 Boon or 2 XP}) and seeds rumors/leads.

\paragraph{Strengths} Scene control via assistance; strategic reach between sessions; strong heist/social-planning play.  
\paragraph{Risks} Dependency on assist lanes; followers can be endangered on 2+ Complication Point spends; upkeep pressure (Resource Clock).  
\paragraph{GM Guidance} Make lanes matter. Enforce \textbf{one assistant max, +3 dice cap}. Target consequences fairly; endangering a follower should escalate stakes, not punish creativity.

\subsection{Comparative Guidance at a Glance}

\begin{center}
\begin{tabular}{p{3.2cm} p{3.2cm} p{3.2cm} p{3.2cm}}
\toprule
 & \textbf{Solo} & \textbf{Mixed} & \textbf{Mastermind} \\
\midrule
XP Focus & Attributes/Skills & Split & Followers + Assets \\
Core Dice & Highest personal pools & Good pools + situational +3 & Moderate pools + frequent +3 \\
Off-Screen & Light & Some leverage & Strong leverage \\
Fragility & Low (self-reliant) & Medium (one helper) & Higher (helper risk, upkeep) \\
GM Dials & Hard DCs, layered stakes & Multi-key scenes & Lanes, logistics, comp-targeting \\
\bottomrule
\end{tabular}
\end{center}

\subsection{Balancing the Spotlight}
\begin{itemize}
  \item \textbf{Solos} earn spotlight through Intricate action—reward rich description with rerolls and fictional position.
  \item \textbf{Mixed} characters should unlock doors others can’t, via their one specialty follower or asset.
  \item \textbf{Masterminds} shine in plans coming together; ensure each step needs a different PC so orchestration lifts everyone.
\end{itemize}

\subsection{Advancement Notes}
\begin{itemize}
  \item \textbf{Followers Scale Fast:} $C^2$ costs keep high-Cap help rare. A Cap~5 bodyguard is 25 XP—roughly the same as taking a Skill from 0 to 4 \emph{and} nudging an Attribute.
  \item \textbf{+3 Assist Cap:} Maintains parity—raising your Skill above 3 still matters; help doesn’t eclipse mastery.
  \item \textbf{Off-Screen Assets} don’t add dice; they \emph{change the fiction}. Use them to set position (entry, cover, rumors, writs) so they feel worth the XP without stealing scenes.
\end{itemize}

% ===== Fate's Edge — Campaign Crown & Clock (SRD Addendum) =====
% Drop-in LaTeX section; compiles standalone or inside the SRD.
\section\*{Campaign Frame / Finale: The Crown Spread}
\addcontentsline{toc}{section}{Campaign Frame / Finale: The Crown Spread}

\subsection\*{Overview}
A Fate's Edge campaign crowns with a single, public decision that the world remembers (a doctrine, charter, rite, bridge, or standard). This section formalizes: (1) a \emph{Crown Spread} you draw at Session 0, (2) a \emph{Campaign Clock} that advances across arcs, and (3) the \emph{Finale} procedure that turns those cards and ticks into fair, swingy play.

\subsection\*{Session 0: The Crown Spread (Initial Draw)}
Draw \textbf{5 cards}: Spade, Heart, Club, Diamond, and a \textbf{Wild} (any suit; reveal last).
\begin{enumerate}
\item \textbf{Spade = Crown Site} (where the monument is decided). Face/Ace implies a historic venue; Ace means the \emph{site itself is the stakes} later (sets Finale Primary to 10).
\item \textbf{Heart = Crown Rival} (who can still stop it). Rank sets initial influence; Face/Ace means a named, national actor. You will generate full motives (\heartsuit,\clubsuit,\diamondsuit,\spadesuit) for this NPC at the finale.
\item \textbf{Club = Crown Pressure} (the rail that will bite if the table turtles). Map by suit rank to your rails: low = \smallcaps{Crowd} or \smallcaps{Sanctity}; mid = \smallcaps{Curfew} or \smallcaps{Hazard}; high/face = \smallcaps{Escape Net} or \smallcaps{Hunt}. Pick one and name it now.
\item \textbf{Diamond = Crown Leverage} (the payoff that can be codified). Examples: seasonal endorsement, city license, doctrinal clause, oath-right. Diamonds never roll; they change \emph{position} or grant one clear \emph{outcome}.
\item \textbf{Wild (reveal last)}: Face = a hidden patron steps out; Ace = the site becomes a 10-clock regardless of Heart rank; Number = an unexpected constituency (guild, shrine, crowd) gets a voting voice at the finale.
\end{enumerate}
\textit{Reveal cadence:} Keep the Wild face-down until the Finale. You may reveal Heart or Club early as foreshadowing, but the Diamond's exact form should stay fluid and harden during Tier IV play.

\subsection\*{The Campaign Clock}
Track two dials over the campaign:
\begin{description}
\item\[Mandate (0--6)] the table's \emph{public legitimacy} and buy-in.
\item\[Crisis (0--6)] the \emph{opposition engine} (rivals, pressure rails, attrition).
\end{description}
When either dial reaches its threshold, the Crown can (or must) be attempted.

\paragraph{Advancing the dials (end of each Major scene)}
\begin{tabular}{@{}l l@{}}
\toprule
\textbf{Outcome} & \textbf{Dial changes} \\
\midrule
Clean Major win (Primary filled; no rail filled) & Mandate +2 \\
Messy Major win (Primary filled; a rail filled) & Mandate +1, Crisis +1 \\
Fail forward (Primary not filled; you keep leverage) & Crisis +1 \\
Clean loss (rival codifies or escapes with leverage) & Crisis +2 \\
Public cost paid (feast, free day, penance) & Mandate +1 (once per arc) \\
Asset neglect (flagged Major degrades) & Crisis +1 \\
Evidence \emph{Immaculate} defended this arc & Mandate +1 (once per arc) \\
Evidence turns \emph{Scorched} & Crisis +1 \\
\bottomrule
\end{tabular}

\paragraph{Calling or Forcing the Crown}
\begin{itemize}
\item \textbf{Player-called finale (favorable terrain):} when Mandate $\ge 6$ and Crisis $\le 3$, the table may schedule the Finale at the next opportune site. Start rails at 0; grant one free \textit{Boost} ("On Their Procedure" or equivalent) at scene start.
\item \textbf{Forced finale (hostile terrain):} when Crisis $\ge 6$ (regardless of Mandate), the Rival forces a decision next arc. Start both rails at +1 (or +2 if the Wild is an Ace); GM CP budget +1.
\item \textbf{Balanced finale:} if Mandate 4--5 and Crisis 4--5, start both rails at +1; CP budget as normal.
\end{itemize}
Mandate and Crisis do not decrease; they exist to \emph{time and tint} the finale, not to micro-score scenes.

\subsection\*{Finale Procedure (Crown Beat)}
Use the Session 0 Crown Spread to seed setup; then run the three-beat crown.
\begin{enumerate}
\item \textbf{Reckoning (procedure):} defend or sanctify the record. Draw the Rival's motives (\heartsuit,\clubsuit,\diamondsuit,\spadesuit) and place the Crown Pressure rail chosen at Session 0. Diamonds change position or grant one clear outcome; do not roll Diamonds.
\item \textbf{Crossing (the bite):} stage the kinetic rail (Escape/Hunt/Hazard) that threatens to end the scene if the table turtles.
\item \textbf{Coronation (codify):} use the Diamond Leverage to sign, seal, or oath the monument.
\end{enumerate}
\textbf{Twist Collision (Finale clause):} exactly once, when the Rival's \spadesuit{} Twist contradicts their \clubsuit{} Belief, the reveal \emph{must} swing a dial: either \textbf{GM +1 CP} \emph{or} \textbf{players reduce \underline{two} ticks total across the rails}. Table chooses the flow.

\paragraph{Finale clocks}
\begin{itemize}
\item \textbf{Primary:} 10 if any Ace is in the Spread (Spade or Wild), else 8.
\item \textbf{Rails:} Use the Club card chosen at Session 0 to define the social rail (\smallcaps{Sanctity}, \smallcaps{Crowd}, or \smallcaps{Curfew}). Choose a kinetic rail that best fits the Site (\smallcaps{Escape Net}, \smallcaps{Hunt}, or \smallcaps{Hazard}).
\item \textbf{CP budget:} 6--7 (declare aloud). Spend to weaponize the Twist, escalate hazards, and threaten \emph{adjournment}, not to hard-block play.
\end{itemize}

\subsection\*{Outcomes}
\begin{description}
\item\[Crowned] Primary fills before either rail. Your doctrine/standard/charter becomes \textbf{Doctrine}. Record any caveats from the Wild or rails as tags.
\item\[Tempered] Primary fills, but a rail hit forces a caveat (seasonal, regional, or review in a month). Record as a \textbf{Seasonal Endorsement} with the decay track (Fresh→Weathered→Compromised).
\item\[Shattered] A rail fills first. The world changes against you, but you keep one \textbf{Legacy} (an ally promoted, a rival unmasked, or a follower seated). Proceed to epilogues.
\end{description}

\subsection\*{Legacy Conversion (Epilogue)}
After the Finale, each PC draws 2 cards and answers epilogue prompts by suit. Then convert:
\begin{itemize}
\item \textbf{Major asset → Institution} (12 XP, or 8 with a Steward epilogue scene): permanent setting change the next table can invoke (hall, rite, standing charter).
\item \textbf{Seasonal endorsement → Doctrine rider} (4 XP): fold the season into the base Accord footprint.
\item \textbf{Follower (Cap 3+) → Stationed NPC} (0 XP): promote to Custodian/Deputy Chair; once per future arc they grant \textit{Controlled} on one filing/chase in their house.
\item \textbf{Rival → Fixture}: if they survive, crown them Opposition: they auto‑tick the relevant rail by +1 whenever your style of play shows in that region.
\end{itemize}

\subsection\*{Patch Notes (SRD edits)}
\begin{itemize}
\item Add \textbf{Crown Spread} to Session 0 procedure (draw S/H/C/D + Wild; record Site, Rival, Pressure, Leverage; keep Wild hidden).
\item Introduce \textbf{Mandate/Crisis} as the \emph{Campaign Clock} (two 0--6 dials). Provide the advancement table above and the \emph{Calling/Forcing} thresholds.
\item In Finale rules, codify \textbf{Twist Collision (Finale clause)} and \textbf{CP budget 6--7} with declared spends and sensory cues.
\item Cross‑reference \textbf{Diamonds as outcomes} (no dice; position/outcome only) and \textbf{Evidence Tags} (Immaculate/Scorched) from the evidence section.
\item Epilogue: add the \textbf{Legacy Conversion} table and \textbf{2‑card prompts} by suit.
\end{itemize}

\subsection\*{Pacing Profiles / Variable Campaign Clock (Optional)}
To keep the core game open for one-shots/endless play, the Campaign Clock is \emph{optional}. Choose a profile at Session 0 to set how "epic" the season feels. The core rules remain unchanged.

\begin{center}
\begin{tabular}{@{}l c c c c@{}}
\toprule
\textbf{Profile} & \textbf{Cap (ticks)} & \textbf{Player-Called Finale} & \textbf{Forced Finale} & \textbf{Finale CP Budget} \\
\midrule
Skirmish (Tier II target) & 6 & Mandate \$\ge 5\$, Crisis \$\le 3\$ & Crisis \$\ge 5\$ & 5--6 \\
Chronicle (Tier III target) & 8 & Mandate \$\ge 6\$, Crisis \$\le 3\$ & Crisis \$\ge 6\$ & 6--7 \\
Saga (Tier IV target) & 10 & Mandate \$\ge 8\$, Crisis \$\le 4\$ & Crisis \$\ge 8\$ & 7--8 \n\bottomrule
\end{tabular}
\end{center}

\noindent \textit{Balanced Finale:} when both dials sit in the mid-band (Skirmish: 4/4; Chronicle: 4--5/4--5; Saga: 6--7/6--7), start both rails at +1; CP budget as listed.

\paragraph{Dial advances scale naturally.} Use the same advancement table; a longer cap simply asks for more Major scenes to accumulate ticks. Expect rough arcs per profile: Skirmish 4--6 Majors; Chronicle 6--8; Saga 8--12.

\paragraph{Over-stack check by profile.} If the crew begins a finale with 2+ structural advantages (e.g., Proof Set + On-Their-Procedure + Public Witness): Skirmish start each rail at +1; Chronicle start rails at +1 \emph{or} grant GM +1 CP on first \spadesuit{} reveal; Saga start rails at +1 (or +2 if \emph{three} structural advantages).

\paragraph{Downtime cadence (guideline).} Skirmish: 1 light downtime between arcs; Chronicle: 1--2; Saga: 2+ with follower promotion (Steward/Deputy) to keep networks alive.

\paragraph{Mini-Crown (optional mid-season capstone).} When Mandate reaches half the cap (Skirmish 3, Chronicle 4, Saga 5), you may stage a local \emph{Mini-Crown}: Primary 6, CP 4--5, produces a \textbf{Public Notice} or \textbf{Seasonal Endorsement} (not full Doctrine). This lets long campaigns milestone without ending.

\paragraph{One-shots / endless play.} Ignore Mandate/Crisis entirely; run the Deck as normal. You can still draw a Crown Spread at Session 0 as \emph{theme and foreshadowing} without ever cashing it.

% End of Campaign Crown & Clock

% ======================================================
\appendix
% ======================================================

\section{Sample Characters}

This section presents three ready-to-play exemplars of the archetypes defined earlier. Each is system-legal using the costs defined in this book (Attributes: new$\times$3; Skills: new$\times$2; Followers: $C^2$; Assist cap +3; Off-screen assets by tier). Use them as pregens, benchmarks, or templates.

\section{The Solo — Sable Kestrel, Road-Worn Duelist}

\subsection*{Concept}
A blade for hire who trusts steel, footwork, and a cool read of the room more than favors or retainers.

\subsection*{Build Summary (30 XP)}
\begin{itemize}
  \item \textbf{Attributes}: Body 3 (9 XP), Wits 3 (9 XP), Spirit 2, Presence 2 \hfill \emph{18 XP}
  \item \textbf{Skills}: Melee 3 (2+4+6 = 12 XP), Observation 1 (2 XP) \hfill \emph{14 XP}
  \item \textbf{Total:} 32 XP $\rightarrow$ drop Observation 1 (2 XP) or start at 32 XP tables. \textbf{Baseline 30 XP}: Body 3, Wits 3, Melee 3.
\end{itemize}

\subsection*{Dice Pools (Common)}
\begin{itemize}
  \item \textbf{Body + Melee}: 6 (brutal, direct)
  \item \textbf{Wits + Melee}: 6 (feints, tempo control)
  \item \textbf{Wits + Observation}: 3 (if taken)
\end{itemize}

\subsection*{Signature Move}
\textbf{Intricate Riposte:} Declare a narrow objective (\emph{``disarm without blood''}), describe footwork and angle; roll Wits+Melee (6d10). On 1s, Intricate rerolls minimize Complications; on a clean success, add a flourish (\emph{opponent’s blade clatters to your boot}).

\subsection*{Complication Hooks (GM)}
\begin{itemize}
  \item \emph{Overreach:} Fatigue clock; armor strap loosens; duel attracts unwanted gambler attention.
  \item \emph{Collateral:} Stray blow nicks a banner; a house guard demands reparations.
\end{itemize}

\subsection*{Advancement Path}
\begin{itemize}
  \item Short term: Melee 3$\rightarrow$4 (8 XP); Observation 0$\rightarrow$2 (6 XP).
  \item Long term: Presence 2$\rightarrow$3 (9 XP) to open a credible social line.
\end{itemize}

\bigskip
\hrule
\bigskip

\section{The Mixed — Bryn of the Ledger, Field Archivist}

\subsection*{Concept}
A scholar-adventurer who solves problems with keen recall and a single trusted specialist.

\subsection*{Build Summary (30 XP)}
\begin{itemize}
  \item \textbf{Attributes}: Presence 3 (9 XP), Wits 3 (9 XP), Body 2, Spirit 2 \hfill \emph{18 XP}
  \item \textbf{Skills}: Sway 3 (2+4+6 = 12 XP) \hfill \emph{12 XP}
  \item \textbf{Follower}: Cap 3 \emph{Archivist} (Lore/Research specialist) (9 XP)
  \item \textbf{Adjust}: To stay at 30 XP, delay Wits 3 until first advance \emph{or} start at 33 XP tables. \textbf{Baseline 30 XP}: Presence 3, Sway 3, Follower Cap 3.
\end{itemize}

\subsection*{Dice \& Assist}
\begin{itemize}
  \item \textbf{Presence + Sway}: 6 (negotiations, permissions, de-escalation)
  \item \textbf{Archivist Assist}: Up to +3 dice on \emph{Lore/Research/Occult} actions, capped by Bryn’s own relevant Skill (encourages training to 3).
\end{itemize}

\subsection*{Signature Move}
\textbf{Stamped and Countersigned:} Start a parley with a citation (\emph{charter, doctrine, ledger mark}); roll Presence+Sway (6d10). If Complications, the GM may introduce red tape (\emph{a missing counter-seal}); spend a minute with the Archivist to produce the cross-reference and earn a \emph{position} advantage on the next roll.

\subsection*{Complication Hooks (GM)}
\begin{itemize}
  \item \emph{Academic Rivalry:} A jealous clerk disputes your citation.
  \item \emph{Paper Trail:} Your documents help you now but later implicate an ally.
\end{itemize}

\subsection*{Advancement Path}
\begin{itemize}
  \item Short term: Wits 2$\rightarrow$3 (9 XP); Research 0$\rightarrow$3 (12 XP).
  \item Long term: Off-screen Asset (Minor Scriptorium Cell, 4 XP) to seed rumors and requisitions.
\end{itemize}

\bigskip
\hrule
\bigskip

\section{The Mastermind — Ash Thorne, Quiet Conductor}

\subsection*{Concept}
A calm planner whose small crew and safehouses convert problems into steps.

\subsection*{Build Summary (30–34 XP)}
\begin{itemize}
  \item \textbf{Attributes}: Wits 3 (9 XP), Presence 3 (9 XP), Body 2, Spirit 2 \hfill \emph{18 XP}
  \item \textbf{Skills}: Tactics 2 (2+4 = 6 XP), Skullduggery 1 (2 XP) \hfill \emph{8 XP}
  \item \textbf{Follower}: Cap 4 \emph{Scout/Ranger} (16 XP)
  \item \textbf{Off-Screen Asset}: Minor Safehouse (4 XP)
  \item \textbf{Trim Options}: Drop Skullduggery 1 and the Safehouse to hit 30 XP; or accept 34 XP at generous tables.
\end{itemize}

\subsection*{Dice \& Assist}
\begin{itemize}
  \item \textbf{Wits + Tactics}: 5 (planning, route selection, timing)
  \item \textbf{Scout Assist}: Up to +3 dice on movement/stealth/survival lanes, capped by Ash’s Skill.
\end{itemize}

\subsection*{Signature Move}
\textbf{Three Doors Ahead:} Ash frames the approach with two decoys and one true lane (Tactics roll). On success, pick one benefit: \emph{split heat between decoys}, \emph{advance a clock two ticks}, or \emph{bank +1 forward} for the next ally action. On 2+ Complication Points, the GM may \emph{target the Scout} or compromise a decoy.

\subsection*{Complication Hooks (GM)}
\begin{itemize}
  \item \emph{Crew Liability:} The Scout’s old feud surfaces.
  \item \emph{Burned Safehouse:} A rival pays a porter to misdirect deliveries.
\end{itemize}

\subsection*{Advancement Path}
\begin{itemize}
  \item Short term: Skullduggery 1$\rightarrow$3 (6+8 = 14 XP) to self-cap assists at +3.
  \item Long term: Follower Cap 5 bodyguard (25 XP) \emph{or} Standard Safehouse Ring (8 XP) for regional reach.
\end{itemize}

\bigskip
\hrule
\bigskip

\section{At-the-Table Play Examples}

\subsection*{Heist Entry (Intricate Actions)}
\textbf{Solo} cases the guard’s stance (Wits+Melee as feint-reading): success, banks a disarm detail for later.  
\textbf{Mixed} flashes a merchant writ (Presence+Sway), Archivist supplies a ledger note: +2 dice assist.  
\textbf{Mastermind} runs the decoy wagon (Wits+Tactics): success splits heat; one Complication Point: a junior guard memorizes faces.

\subsection*{Duel in the Courtyard}
\textbf{Solo} declares \emph{``first blood, no scars''}, wins on clean 6+, rerolls 1s via Intricate.  
\textbf{Mixed} keeps the crowd aligned: reduces fallout with Sway; on Complication, a heckler changes the mood.  
\textbf{Mastermind} positions allies: next turn the Scout provides climbing lines (+3 capped assist) for a swift exit.

\subsection*{Debrief \& Downtime}
\textbf{Solo} buys Skill 3$\rightarrow$4 to keep personal ceiling high.  
\textbf{Mixed} trains Research 0$\rightarrow$2 and deepens the Archivist’s narrative ties.  
\textbf{Mastermind} invests XP into a Standard Safehouse Ring to convert future \emph{position} into on-screen advantage.

\bigskip
\hrule
\bigskip

\section{Conversion Notes \& Reskinning}
These statlines are intentionally lean. To reskin:
\begin{itemize}
  \item Swap \textbf{Melee} for \textbf{Marksmanship} on the Solo to make a sniper.
  \item Change the \textbf{Archivist} to a \textbf{Quartermaster} (assist lanes: Logistics, Forgery, Procurement).
  \item Rebuild the \textbf{Scout} as a \textbf{Whisper-Familiar} for casters; same Cap rules, different fiction.
\end{itemize}

\section{Quick Reference Sheets}

These condensed sheets summarize the core procedures, costs, and tables for use at the table.  
They are designed for GMs and players to keep the flow of play without flipping through sections.

\section*{Core Mechanic: The Art of Consequence}
\begin{enumerate}
  \item \textbf{Approach:} Player states intent and method (Attribute + Skill).
  \item \textbf{Execution:} Roll dice pool of d10s. Each 6+ is a success; each 1 is a Complication Point.
  \item \textbf{Outcome:}  
  \begin{description}[leftmargin=2cm]
    \item[Basic] Roll as-is; tally Complications.  
    \item[Detailed] Re-roll one \texttt{1}.  
    \item[Intricate] Re-roll all \texttt{1}s; add one positive flourish if successful.  
  \end{description}
\end{enumerate}

\section*{Attributes \& Skills}
\begin{description}[leftmargin=2cm]
  \item[Body] Strength, endurance, physical force.  
  \item[Wits] Perception, cleverness, reflexes.  
  \item[Spirit] Willpower, intuition, resilience.  
  \item[Presence] Charm, command, social force.  
\end{description}

\textbf{Skill Ratings (0–5)}  
\begin{description}[leftmargin=2cm]
  \item[0] Untrained — rely on raw Attribute.  
  \item[1] Familiar — basic competence.  
  \item[2] Skilled — reliable training.  
  \item[3] Expert — professional mastery.  
  \item[4] Master — renowned in your field.  
  \item[5] Legendary — near-mythic talent.  
\end{description}

\section*{XP Costs}
\begin{description}[leftmargin=2cm]
  \item[Attributes] New rating × 3.  
  \item[Skills] New level × 2.  
  \item[Followers (On-Screen)] Cap$^2$ XP (e.g., Cap 3 = 9 XP).  
  \item[Off-Screen Assets] Tiered cost (Minor 4 XP, Standard 8 XP, Major 12 XP).  
\end{description}

\section*{Deck of Consequences}
Draw from the deck (or roll a d52 equivalence) when Complication Points are spent.  
Suits = type of complication. Rank = severity.

\begin{tabular}{|c|c|}
\hline
\textbf{Suit} & \textbf{Complication Domain} \\
\hline
Hearts & Emotional / Social fallout (fear, anger, betrayal) \\
\hline
Diamonds & Resource / Wealth loss (gear breaks, expenses rise) \\
\hline
Clubs & Physical harm / Obstacles (injuries, blockades, fatigue) \\
\hline
Spades & Mystical / Narrative twists (omens, curses, chance) \\
\hline
\end{tabular}

\medskip
\noindent
\textbf{Ranks (1–10, J–K–A):}  
\begin{itemize}
  \item 2–5: Minor setback, scene continues smoothly.  
  \item 6–9: Moderate complication; new obstacle or clock starts.  
  \item 10–King: Severe twist; alters stakes of the scene.  
  \item Ace: Catastrophic turn; reshapes narrative or mission goal.  
\end{itemize}

\section*{Magic Casting Loop}
\begin{enumerate}
  \item \textbf{Channel:} Wits + Arcana roll to gather Potential.  
  \item \textbf{Weave:} Wits + Art roll to shape spell.  
  \item \textbf{Backlash:} Complication Points spent through Deck of Consequences, themed to the opposing element.  
\end{enumerate}

\section*{Player Archetypes}
\begin{itemize}
  \item \textbf{Solo:} Invests in Attributes + Skills. Strong spotlight.  
  \item \textbf{Mixed:} Balances self with one follower or off-screen assets.  
  \item \textbf{Mastermind:} Multiple followers + networks, but more narrative liabilities.  
\end{itemize}

\section*{Narrative Time}
\begin{description}[leftmargin=2cm]
  \item[A Moment] A heartbeat; single action.  
  \item[Some Time] A few minutes; quick exchanges.  
  \item[Significant Time] An hour or more; downtime actions.  
  \item[Days] Extended travel or projects.  
\end{description}

\section*{GM Guidance at a Glance}
\begin{itemize}
  \item Spend Complication Points to \textbf{add story problems}, not punish.  
  \item Always tie consequences back to \textbf{thematic domains} (Hearts, Diamonds, Clubs, Spades).  
  \item Encourage Intricate actions: reward description with rerolls and narrative control.  
  \item Let Off-Screen assets resolve downtime problems but keep adventures on the table.  
\end{itemize}

\section{GM Toolkit: Gear, Kits, and Artifacts}

This section provides guidance and ready-to-use gear for \textit{Fate’s Edge}.  
Unlike the player-facing primer, this section is designed for GMs.  
It contains condensed tables of common kits, weapons, armor, rare items, and magical artifacts.  
It also offers rules of thumb for creating new gear without tracking individual numbers.

\section{Design Philosophy}
\begin{itemize}
  \item \textbf{Narrative, not accounting.} Gear matters when it changes the story, not as a spreadsheet.
  \item \textbf{Complications drive gear drama.} Items break, spoil, or splinter when the dice say so.
  \item \textbf{XP defines weight.} Items that are always-on or arc-defining should cost XP.  
  Flavorful, narrow, or once-per-session items may be free at creation or as treasure.
\end{itemize}

\section{Standard Kits}
Characters begin with one free kit. Upgraded versions may cost XP, but mundane versions are assumed to be coin-purchased or handwaved.

\begin{tabularx}{\textwidth}{lX}
\textbf{Kit} & \textbf{Narrative Effect} \\
\hline
Traveler’s Pack & Baseline supplies. Without it, a PC suffers Fatigue one step sooner than the party. \\
Scout’s Rig & +1 die to a Survival/Stealth roll per session. Loses bonus if Supply is Dangerously Low. \\
Scholar’s Satchel & +1 die to a Lore/Arcana roll per session. Requires light. \\
Burglar’s Tools & +1 die to Skulduggery/Stealth vs. mundane locks or traps. May become Compromised. \\
Healer’s Bag & Once per session: Spirit+Craft roll to remove 1 Fatigue from an ally or stabilize the dying. \\
\end{tabularx}

\section{Weapons and Armor}
Coin buys mundane versions. XP upgrades represent cultural edges or masterwork reliability.

\begin{tabularx}{\textwidth}{lX}
\textbf{Item} & \textbf{Narrative Effect} \\
\hline
Ykrul Recurve Bow & Powerful composite bow. If Compromised, requires rare materials to repair. \\
Viterran Longspear & Once per scene: +1 die against a charging mounted foe. \\
Duelist’s Rapier & On Intricate attack, reroll one failed die. Bends (Compromised) on failed roll with Complications. \\
Knight’s Heater Shield & May forgo attack to give ally +2 dice to next defense. Compromise is a story event. \\
Aeler-forged Hauberk & Counts as armor. When gaining Fatigue, roll a die; on 6+, avoid it. \\
\end{tabularx}

\section{Rare Gear (Story Rewards)}
These are not purchased; they are discovered, inherited, or earned.  
A GM may attach XP “Attunement” costs if players want to keep and rely on them.

\begin{tabularx}{\textwidth}{lX}
\textbf{Item} & \textbf{Narrative Effect} \\
\hline
Rothari War-Kit & Once per session: cancel 1 Complication from a stealth/ambush roll. \\
Whisper-thin Elven Cloak & Stealth in natural terrain: reroll one \texttt{1}. At Full Supply, reroll all \texttt{1}s. \\
Lence Dueling Gauntlet & +2 dice to Presence when issuing challenges or demanding satisfaction. \\
Oshiiran Abacus & On trip-planning, add +1 free Supply segment if Lore roll succeeds. \\
Sihai Warrior-Monk Gi & If unarmored, may use Wits instead of Body for defense rolls. \\
\end{tabularx}

\section{Artifacts and Relics}
Artifacts reshape arcs. Always introduce them with strong narrative hooks.

\begin{tabularx}{\textwidth}{lX}
\textbf{Artifact} & \textbf{Narrative Effect} \\
\hline
Sundered King’s Coin & Payment always creates ripples. GM draws 2 Consequence cards, applies one. \\
Echo-Wrought Blade & May strike spirits. On hit, can inflict Fatigue instead of damage. \\
Khemesh-Scale Cloak & Wearer breathes underwater for Significant Time; failed Spirit roll causes Fatigue. \\
Unkwa Ward-Post & Wards camp from natural Supply loss. Failed Spirit roll attracts hungry things. \\
Silent Bell & Once per arc, erase all Complication Points from a roll. Debt returns later as larger problem. \\
\end{tabularx}

\section{Creating New Gear}
GMs should keep new items simple and narrative-facing. Use these guidelines:
\begin{itemize}
  \item \textbf{+1 die once per session} — Minor, 0–2 XP if made permanent.  
  \item \textbf{Cancel 1 Complication} — Mid-tier, 3–4 XP.  
  \item \textbf{Always-on swap or defense} — Major, 6+ XP.  
  \item \textbf{Arc-shaping artifact} — 8–10 XP. Found, not bought.  
\end{itemize}

Tie items to culture, history, or gods of the setting to deepen immersion.  
A dwarven hammer is not “just” a hammer — it is a relic of a guild feud or an heirloom lost in a kingdom’s fall.

\section{Equipment Costs}

\subsection{Cost Tiers}
\begin{description}[leftmargin=2cm]
  \item[Basic (1 XP)] Common gear that gives small situational edges. May be lost or compromised easily.
  \item[Standard (2 XP)] Reliable, quality equipment with repeatable scene impact. Uses the same math as Cap~2 follower.
  \item[Superior (6 XP)] Specialized or rare items with strong hooks. Equivalent to Cap~3+ follower, but with no upkeep.
  \item[Artifact (12 XP or GM Award)] Legendary or magical gear. Powerful, quest-worthy, and permanent unless destroyed in play.
\end{description}

\subsection*{Weapons (Melee)}
\begin{tabular}{@{}llll@{}}
\toprule
\textbf{Name} & \textbf{Tags} & \textbf{Notes} & \textbf{Cost} \\
\midrule
Dagger & Concealable, Light & +1 die to surprise attacks. & 1 XP \\
Longsword & Versatile & Standard dueling blade. & 2 XP \\
Spear & Reach & May attack from back rank. & 2 XP \\
Great Axe & Heavy, Brutal & -1 die if used indoors. & 6 XP \\
Whip & Nonlethal, Reach & May inflict \emph{Restrained}. & 2 XP \\
\bottomrule
\end{tabular}

\subsection*{Weapons (Ranged)}
\begin{tabular}{@{}llll@{}}
\toprule
\textbf{Name} & \textbf{Tags} & \textbf{Notes} & \textbf{Cost} \\
\midrule
Shortbow & Silent, Light & Favored by scouts. & 2 XP \\
Longbow & Powerful, Two-Handed & -1 die in cramped spaces. & 2 XP \\
Crossbow & Reliable, Slow & Reload costs 1 action. & 2 XP \\
Throwing Knives & Concealable, Quick & +1 die at close range. & 1 XP \\
Hand Cannon & Loud, Devastating & Creates 1 CP on use. & 6 XP \\
\bottomrule
\end{tabular}

\subsection*{Armor \& Protection}
\begin{tabular}{@{}llll@{}}
\toprule
\textbf{Name} & \textbf{Tags} & \textbf{Notes} & \textbf{Cost} \\
\midrule
Leather Jerkin & Light, Flexible & No penalty; +1 die vs. glancing blows. & 1 XP \\
Chain Shirt & Medium, Noticeable & -1 die on stealth. & 2 XP \\
Plate Harness & Heavy, Bulky & +2 dice resisting harm; -2 stealth. & 6 XP \\
Shield (Buckler) & Light, One-Handed & +1 die on parries. & 1 XP \\
Shield (Tower) & Cover, Heavy & +2 dice vs. ranged; -1 init. & 2 XP \\
\bottomrule
\end{tabular}

\subsection*{Tools \& Kits}
\begin{tabular}{@{}llll@{}}
\toprule
\textbf{Name} & \textbf{Tags} & \textbf{Notes} & \textbf{Cost} \\
\midrule
Lockpicks & Fragile, Precise & Needed for most lockpicking rolls. & 1 XP \\
Navigator’s Tools & Durable & +1 die on travel/sea checks. & 2 XP \\
Forgery Set & Subtle & Enables document creation. & 2 XP \\
Healer’s Pouch & Limited Uses & May reduce harm by 1 tier. & 2 XP \\
Alchemist’s Kit & Volatile & Brew poisons/potions in downtime. & 6 XP \\
\bottomrule
\end{tabular}

\subsection*{Upkeep \& Degradation}
Standard gear (2 XP tier) follows the same upkeep rules as followers:
\begin{itemize}
  \item After a story arc or 2--3 sessions of heavy use, the GM may flag the item as at risk.
  \item To keep it \textbf{Maintained}, the player must either:
  \begin{itemize}
    \item Spend downtime repairing/servicing it, or
    \item Pay 1 XP immediately (representing replacement parts, masterwork upkeep).
  \end{itemize}
  \item If neglected, the gear becomes \textbf{Compromised} (imposes a $-1$ die penalty) and may eventually break.
  \item Superior and Artifact gear do \emph{not} require upkeep, but if Compromised through Complications, only narrative quests can repair them.
\end{itemize}

\section{Skills \& Archetypes}

\subsection{Skill Basics}
\begin{itemize}
  \item Skills cost $2 + 2 \times (\text{new rating})$ XP to advance.
  \item No skill may exceed its linked Attribute (cap).
  \item Certain skills require Attribute minimums to unlock.
  \item Each die added from a skill stacks with the Attribute for pools.
\end{itemize}

\subsection{Class Archetypes}
The following collections of skills represent broad playstyles. Players are not locked to them, but each archetype provides a clear path for specialization.

\subsubsection*{Warrior}
\begin{tabular}{@{}llll@{}}
\toprule
\textbf{Skill} & \textbf{Linked Attribute} & \textbf{Prerequisites} & \textbf{Notes / Cost} \\
\midrule
Melee & Body & None & Weapon combat, duels, brawls. Cost 2/4/6 XP. \\
Athletics & Body & None & Running, climbing, grappling. Cost 2/4/6 XP. \\
Endurance & Spirit & None & Resist fatigue, poison, pain. Cost 2/4/6 XP. \\
Command & Presence & None & Rally allies, issue battlefield orders. Cost 2/4/6 XP. \\
\bottomrule
\end{tabular}

\subsubsection*{Rogue}
\begin{tabular}{@{}llll@{}}
\toprule
\textbf{Skill} & \textbf{Linked Attribute} & \textbf{Prerequisites} & \textbf{Notes / Cost} \\
\midrule
Stealth & Wits & None & Hide, sneak, vanish in crowds. Cost 2/4/6 XP. \\
Skulduggery & Wits & None & Locks, traps, dirty tricks. Cost 2/4/6 XP. \\
Deception & Presence & None & Lies, disguises, feints. Cost 2/4/6 XP. \\
Subterfuge & Presence & None & Intrigue, shadow networks. Cost 2/4/6 XP. \\
\bottomrule
\end{tabular}

\subsubsection*{Scholar}
\begin{tabular}{@{}llll@{}}
\toprule
\textbf{Skill} & \textbf{Linked Attribute} & \textbf{Prerequisites} & \textbf{Notes / Cost} \\
\midrule
Lore & Wits & None & History, theology, occult. Cost 2/4/6 XP. \\
Craft & Wits & None & Forging, tinkering, building. Cost 2/4/6 XP. \\
Research & Wits & None & Libraries, archives, data. Cost 2/4/6 XP. \\
Arcana & Spirit & Spirit 3+ & Spell theory, wards, recognition. Cost 2/4/6 XP. \\
\bottomrule
\end{tabular}

\subsubsection*{Face}
\begin{tabular}{@{}llll@{}}
\toprule
\textbf{Skill} & \textbf{Linked Attribute} & \textbf{Prerequisites} & \textbf{Notes / Cost} \\
\midrule
Diplomacy & Presence & None & Persuasion, negotiation. Cost 2/4/6 XP. \\
Insight & Wits & None & Read motives, detect lies. Cost 2/4/6 XP. \\
Performance & Presence & None & Oratory, music, stagecraft. Cost 2/4/6 XP. \\
Sway & Presence & None & Street charisma, bargains, leverage. Cost 2/4/6 XP. \\
\bottomrule
\end{tabular}

\subsubsection*{Mystic}
\begin{tabular}{@{}llll@{}}
\toprule
\textbf{Skill} & \textbf{Linked Attribute} & \textbf{Prerequisites} & \textbf{Notes / Cost} \\
\midrule
Ritual & Spirit & Spirit 3+ & Circle magic, shared casting. Cost 2/4/6 XP. \\
Occult & Spirit & None & Recognize curses, sigils, traditions. Cost 2/4/6 XP. \\
Meditation & Spirit & None & Resist Backlash, channel inner force. Cost 2/4/6 XP. \\
Tactics & Wits & None & Military planning, ambush prep. Cost 2/4/6 XP. \\
\bottomrule
\end{tabular}

\subsubsection*{Explorer}
\begin{tabular}{@{}llll@{}}
\toprule
\textbf{Skill} & \textbf{Linked Attribute} & \textbf{Prerequisites} & \textbf{Notes / Cost} \\
\midrule
Survival & Wits & None & Forage, navigate wilds, endure climates. Cost 2/4/6 XP. \\
Navigation & Wits & None & Sea routes, maps, stars. Cost 2/4/6 XP. \\
Animal Handling & Spirit & None & Ride, train, soothe beasts. Cost 2/4/6 XP. \\
Scouting & Wits & None & Spot danger, track, recon. Cost 2/4/6 XP. \\
\bottomrule
\end{tabular}

\section{Prestige Skills \& Roles}

Prestige Skills represent specialized arts tied to cultures, philosophies, or advanced training. 
They cannot be purchased at character creation. 
Each requires meeting Attribute and Skill minimums and costs XP to acquire. 
They also come with cultural or narrative obligations.

\subsection*{Mid-Tier Prestige Skills (unlockable at XP 40--60)}

\begin{tabular}{@{}llll@{}}
\toprule
\textbf{Skill} & \textbf{Prerequisites} & \textbf{XP Cost} & \textbf{Setting Role} \\
\midrule
\textbf{Stone-Sense} & Spirit 3, Lore 2 & 6 XP & Dwarves attune to stone; detect voids, stress, hidden chambers. \\
\textbf{Backlash Soothing} & Spirit 3, Occult 2 & 6 XP & Wood Elves dissipate magical backlash; allies reroll one failed die vs. Backlash. \\
\textbf{Battle Chant} & Presence 3, Performance 2 & 6 XP & Ubral and Viterran war-singers; allies ignore first Fatigue penalty in battle. \\
\textbf{Guild Ledgercraft} & Wits 3, Craft 2, Research 2 & 6 XP & Oshiiran merchants; manage trade networks. Gain +1 Supply clock segment once per arc. \\
\textbf{Trial Duelist} & Body 3, Melee 3 & 6 XP & Vhasian aristocrats; declare formal duels to resolve conflicts socially through combat. \\
\textbf{Blood-Ties Oath} & Spirit 3, Diplomacy 2 & 6 XP & Fhara tribes; forge binding oaths. Once/session, seal an alliance—breaking it inflicts Fatigue or worse. \\
\textbf{Storm-Caller} & Wits 3, Survival 2 & 6 XP & Seafolk navigators; manipulate weather. Once/session, adjust environmental stakes (favorable wind, sudden gale). \\
\textbf{Rune-Keeper} & Wits 3, Craft 2, Occult 2 & 6 XP & Ashaani scribes; inscribe wards. Once/session, place a rune that grants +1 die to a single roll later. \\
\textbf{Beast-Tongue} & Spirit 3, Survival 2 & 6 XP & Ykrul shamans; communicate with beasts. Once/session, command or calm an animal beyond normal skill. \\
\textbf{Echo-Courtier} & Presence 3, Subterfuge 2 & 6 XP & Thepyrgos elves; manipulate memory. Once/session, plant a false but plausible detail in an NPC’s mind. \\
\bottomrule
\end{tabular}

\subsection*{High-Tier Prestige Skills (unlockable at XP 60+)}

\begin{tabular}{@{}llll@{}}
\toprule
\textbf{Skill} & \textbf{Prerequisites} & \textbf{XP Cost} & \textbf{Setting Role} \\
\midrule
\textbf{Echo-Walker} & Spirit 4, Ritual 3, Lore 3 & 10 XP & High Elves; step into echo-realms. Once/arc, vanish from a scene and reappear later with 1 Boon. \\
\textbf{Warglord} & Body 4, Command 3, Tactics 3 & 10 XP & Ykrul war chiefs; command beyond action caps. Once/combat, direct 2 followers to act together. \\
\textbf{Sandseer} & Spirit 4, Survival 3, Insight 3 & 10 XP & Fhara prophets; roll to glimpse next arc’s event. GM must reveal a true omen. \\
\textbf{Star-Reader} & Wits 4, Navigation 3, Lore 3 & 10 XP & Sihai or Ashaani astrologers; grant +2 dice to a Ritual once/arc. \\
\textbf{Everflame Keeper} & Spirit 4, Diplomacy 3, Occult 3 & 10 XP & Ecktoria’s priests; call cleansing fire. Once/arc, inflict 1 Fatigue on all foes (take 1 Fatigue yourself). \\
\textbf{Resonant Architect} & Spirit 4, Craft 3, Lore 3 & 10 XP & Dwarves; awaken stonework. Once/arc, reshape terrain (collapse tunnel, raise wall) at risk of Backlash. \\
\textbf{Whisper-Lord} & Presence 4, Subterfuge 3, Insight 3 & 10 XP & Sekogo or Oshiiran spymasters; once/arc, rewrite one fact about NPC loyalties in the fiction. \\
\textbf{River-Sovereign} & Spirit 4, Diplomacy 3, Survival 3 & 10 XP & Dhaharan mystics; once/arc, call a river to flood, cleanse, or protect. Always alters the map. \\
\textbf{Mask-Bearer} & Presence 4, Performance 3, Subterfuge 3 & 10 XP & Nihon or Ayohkhan traditions; assume ritual masks. Once/arc, adopt a persona with +2 dice to its cadence. \\
\textbf{Shadow-Binder} & Wits 4, Occult 3, Subterfuge 3 & 10 XP & Cultist paths; bind shadows as allies. Once/scene, summon shadow-minions (narrative-only) to distract or mislead. \\
\bottomrule
\end{tabular}

\subsection*{Notes}
\begin{itemize}
  \item Mid-tier prestige skills enhance niche play and tie PCs into their culture’s fabric.  
  \item High-tier prestige skills warp play at the arc level—rewriting fiction, bending the GM’s hand, or shaping battles.  
  \item All are intentionally “once per session/arc” to avoid spam and keep focus on narrative consequences.  
  \item Some mid-tier skills can grow into high-tier forms (e.g., Stone-Sense → Resonant Architect).  
\end{itemize}

\subsection*{Leadership Talents}

Some characters aspire not to be the sharpest blade or the keenest wit, but to lead others with uncanny precision. To keep this option balanced, Leadership is expressed as \textbf{Talents}, not skills. These Talents expand the flexibility of followers without undermining the core dice economy.

\paragraph{Talent: Coordinated Assault}
\begin{description}[leftmargin=2cm]
  \item[Cost:] 8 XP
  \item[Requirements:] Presence 3, Command 2
  \item[Effect:] You may direct two On-Screen Followers to assist with the same action, provided their specialties both apply. The total dice bonus remains capped at \textbf{+3}. This talent increases flexibility, not raw numbers.
\end{description}

\paragraph{Talent: Master Coordinator (Prestige)}
\begin{description}[leftmargin=2cm]
  \item[Cost:] 15 XP
  \item[Requirements:] Presence 4, Command 3, \emph{Coordinated Assault}
  \item[Effect:] You may raise the maximum dice from On-Screen Followers on a single action to \textbf{+4}. This represents a rare mastery of battlefield coordination.
\end{description}

\paragraph{Design Notes}
\begin{itemize}
  \item These Talents emphasize \emph{narrative identity}: the character is a commander, not just someone with friends.
  \item More helpers on-screen means more vectors for GM Complication Points. Followers can be harmed, swayed, or compromised—greater power comes with greater risk.
  \item Investment is steep. A Mastermind Commander sacrifices self-focused growth to enable these broader tactical options.
\end{itemize}

\paragraph{Example in Play}
\begin{quote}
Ryn the Duelist (Body 4 + Melee 4) rolls 8 dice solo.
Althene the Commander (Body 3 + Melee 2 = 5 dice) directs her Cap 4 Bodyguard (+3 dice). With \emph{Master Coordinator}, she can add her Scout as well, pushing to 9 dice—but if the GM spends Complication Points, those allies can be endangered.
\end{quote}

\section{GM Scaling Guidance for Mixed-Tier Parties}
\label{sec:gm-scaling}

This section gives the GM a compact toolkit to keep scenes tense as player characters (PCs) level up, especially when parties mix different Tiers.

\subsection{Core Resolution Recap}
\begin{itemize}
  \item \textbf{Dice:} Roll d8; \textbf{hits on 5--8}.
  \item \textbf{Ticks from hits:} 0$\to$+0;\;1$\to$+1;\;2--3$\to$+2;\;4+$\to$+3.
  \item \textbf{Tier vs.\ Obstacle:} After counting ticks from hits, apply: Higher Tier $+1$ tick;\; Equal $+0$;\; Lower $-1$ tick (min 0).
  \item \textbf{Approach (\emph{Intricate}):} Reroll all \texttt{1}s once; \texttt{1}s still generate GM Complication Points (CP).
  \item \textbf{Complication Points (CP):} Each player-rolled \texttt{1} $\Rightarrow$ +1 GM CP. Spend 1 CP to flip 1 Consequence card; if multiple are flipped, apply only the \emph{nastiest one}.
  \item \textbf{Epic (Mains only, once/scene):} On any success, add \textbf{+1 extra tick} (after Tier math) and broaden scope (affect more targets/area). For each \texttt{1} in the \emph{initial} Epic roll, flip +1 extra Consequence and resolve the nastiest among all flips that beat.
\end{itemize}

\subsection{Consequence Suits (Cheat Sheet)}
\begin{description}
  \item[Spades] Position, surveillance, chokepoints (raise \emph{Alarm/Heat}, spawn tails).
  \item[Hearts] Morale, social fallout (Fatigue, crowd turns, friction).
  \item[Diamonds] Time, money, legal (tick \emph{Supply/Time}, fees, audits).
  \item[Clubs] Harm, gear strain (Harm boxes, \emph{impaired} items).
\end{description}
\textit{Guidance:} Aim Spades/Hearts at lower-Tier PCs; reserve big Clubs for anchors or Epic blowback.

\subsection{Scaling Dials (choose 1--3 per scene)}
\paragraph{CP Floor (Pressure Budget)}
Each beat, set GM CP to at least $\Delta$Tier, where $\Delta$Tier $=\max(0,\ \text{Obstacle Tier} - \text{Highest PC Tier})$.
\begin{itemize}
  \item $\Delta=0$: no free CP; rely on player \texttt{1}s.
  \item $\Delta=1$: you can always flip once per beat.
  \item $\Delta=2$: you flip once per beat \emph{and} bank quickly.
\end{itemize}

\paragraph{Free Flip Cadence}
When \emph{Danger} $\ge 4/6$, the GM gains \textbf{1 free flip/beat} (no CP cost).

\paragraph{Severity Ceiling}
\begin{itemize}
  \item \textbf{Standard:} use ranks 2--10 only (no Faces/Aces).
  \item \textbf{Hard:} 2--10 plus J/Q.
  \item \textbf{Boss:} full deck (Faces/Aces allowed).
\end{itemize}

\paragraph{Position Floor}
Lower-Tier PCs default to \emph{Risky}; anchors operate \emph{Controlled/Risky}.

\paragraph{Boss Clocks \& Gates}
Add \textbf{Gates} the crew must clear (e.g., ``Credentials 4/4''); until cleared, Progress ticks \emph{do not apply}. For elite foes, consider \emph{Job 10} with one or two mandatory gates and \emph{Danger 8}.

\paragraph{Opposition Talents (pick one light tag)}
\begin{itemize}
  \item \textbf{Counterwatch:} The first Spade flip each scene also +1 \emph{Danger}.
  \item \textbf{Hard Cover:} The first Clubs 2--6 becomes \emph{impaired gear} instead of Harm.
  \item \textbf{Paper Teeth:} The first Diamonds 6--10 also +1 \emph{Supply/Time}.
\end{itemize}

\paragraph{Domain Pressure (ambient clocks)}
Locations may start a \textbf{Domain} clock (4--6): \emph{Witnesses}, \emph{Bureaucracy}, \emph{Terrain}, etc. These tick when relevant suit flips occur.

\subsection{Scene Bands (Drop-In Matrix)}
\begin{center}
\begin{tabular}{l l c c c c l}
\toprule
\textbf{Band} & \textbf{Use When\ldots} & \textbf{Job} & \textbf{Danger} & \textbf{Prefill} & \textbf{CP Floor} & \textbf{Severity} \\
\midrule
Balanced & PCs $\approx$ Obstacle & 6--8 & 6 & 1--2 & 0 & 2--10 \\
Hard     & Obstacle = Highest PC  & 8--10 + 1 Gate & 6 & 2--3 & 1 & 2--10 + J/Q \\
Boss     & Obstacle = Highest+1   & 10 + 2 Gates & 8 & 3--4 & 2 & Full Deck \\
\bottomrule
\end{tabular}
\end{center}

\subsection{Mixed-Tier Spot Rulings}
\begin{itemize}
  \item \textbf{Anchor the danger:} The highest-Tier PC should usually target \emph{Danger} each beat; others advance \emph{Progress} or side clocks (4--6).
  \item \textbf{Assist vs.\ Act:} If a lower-Tier PC expects only 1 hit, their net is often 0 ticks after $-1$ Tier mod $\Rightarrow$ \emph{assist} (+1 die) instead. If they can reliably reach 2--3 hits, act on a side clock.
  \item \textbf{Exit Tails:} If \emph{Danger} $\ge 4/6$ when \emph{Job} completes, start a \textbf{Tail 2/4} in the next scene.
\end{itemize}

\subsection{60-Second Prep Recipe}
\begin{enumerate}
  \item Pick a \emph{Scene Band} (Balanced/Hard/Boss) and set \emph{Job}, \emph{Danger}, \emph{Prefill}, CP Floor, and Severity.
  \item Add 1--2 \emph{Gates} (Hard/Boss only).
  \item Optionally add 1 \emph{Domain} clock and 1 \emph{Opposition Talent}.
  \item Note a \emph{Reward} (below). Roll.
\end{enumerate}

\subsection{Beat Procedure at the Table}
\begin{enumerate}
  \item All declare actions; roll the anchor first.
  \item Convert hits $\to$ ticks (apply Tier mods; add Epic tick if used).
  \item Spend CP/free flips; if multiple cards are flipped this beat, apply only the \emph{nastiest} single fallout.
  \item Tick \emph{Job}/\emph{Danger}/\emph{Gates}/\emph{Domain}.
  \item If \emph{Job} completes while \emph{Danger} is hot, run an \emph{exit beat}, then apply \emph{Tail} if triggered.
\end{enumerate}

\subsection{Reward Scaling}
\begin{itemize}
  \item \textbf{Balanced:} Single-use chit (waive a fee; erase 1 \emph{Supply/Time} tick).
  \item \textbf{Hard:} \emph{Level-1 Asset} (persistent, 1 use/session) or district Heat $-1$.
  \item \textbf{Boss:} \emph{Level-2 Asset}, a faction shift, or a 3-charge boon.
\end{itemize}

\subsection{Micro-Examples (Same Roll, Different Tier)}
Versus a Tier IV checkpoint:
\begin{itemize}
  \item \textbf{Tier I} rolls 2--3 hits $\Rightarrow$ base +2, then $-1$ Tier mod $=$ \textbf{+1 tick}. Great for 4-segment side clocks or assisting.
  \item \textbf{Tier II} rolls 2--3 hits $\Rightarrow$ \textbf{+1 tick}, but Talents/position keep them acting each beat instead of only assisting.
  \item \textbf{Tier IV} rolls 2--3 hits $\Rightarrow$ \textbf{+2 ticks} (Equal). With Epic, often \textbf{+3} and widened scope (covering allies or multiple targets).
\end{itemize}

\section*{GM Guidance: Creating Items \& Skills}

\subsection*{Philosophy}
Items and skills in \textit{Fate’s Edge} are not about +1 modifiers.  
They exist to:
\begin{itemize}
  \item Reinforce the setting’s cultures and philosophies.
  \item Give players new narrative levers and spotlight moments.
  \item Introduce risks and obligations that generate drama.
\end{itemize}

\subsection*{Creating Items}
When making a new piece of gear, follow these steps:

\begin{enumerate}
  \item \textbf{Anchor in Culture:} Tie the item to a region, people, or tradition. (e.g., a Viterran dueling rapier, a Sekogo ledger-abacus).
  \item \textbf{Define its Lane:} Decide if it boosts Combat, Social, Exploration, or Survival.
  \item \textbf{Give a Trigger:} One clear effect, usually:
    \begin{itemize}
      \item \emph{+1 die} on a narrow cadence (e.g., “once/session on Lore to decipher scripts”).
      \item \emph{Mitigate Fatigue or Complications} in specific contexts.
      \item \emph{Open narrative doors} (e.g., bypass a lock, read a map).
    \end{itemize}
  \item \textbf{Set Cost:}
    \begin{itemize}
      \item Normal Gear = Follower math (Cap$^2$ XP) + upkeep (downtime or XP).
      \item Superior Gear = Normal cost +2 XP (no upkeep).
      \item Artifact Gear = Normal cost +4 XP (no upkeep; may carry risks/Backlash).
    \end{itemize}
  \item \textbf{Add a Hook:} What story comes with it? (heirloom, guild claim, fae attention).
\end{enumerate}

\paragraph{GM Shortcut Table}
\begin{tabular}{@{}lll@{}}
\toprule
\textbf{Item Type} & \textbf{XP Cost} & \textbf{Example Effect} \\
\midrule
Kit (basic tools) & 1--2 XP & +1 die once/session in niche context \\
Superior Craft & +2 XP & Avoids upkeep; sturdier or rarer \\
Artifact / Relic & +4 XP & Narrative-warping; may trigger Complications \\
\bottomrule
\end{tabular}

\subsection*{Creating Skills}
Skills are story cadences, not technical feats.  
To design one:

\begin{enumerate}
  \item \textbf{Name the Cadence:} A one-word or two-word handle (e.g., Diplomacy, Skulduggery, Survival).
  \item \textbf{Define the Questions It Answers:} “What does this let a PC do that no other skill covers?”
  \item \textbf{Attribute Pairings:} Ensure the skill can flex across multiple Attributes. (e.g., Melee + Body = raw strength; Melee + Wits = finesse).
  \item \textbf{Cost to Raise:} Always (new rating × 2 XP).
  \item \textbf{Prestige Gateways:} Decide if this skill is required for a Prestige path (e.g., Lore 3 to become a Rune-Keeper).
\end{enumerate}

\paragraph{GM Shortcut Table}
\begin{tabular}{@{}lll@{}}
\toprule
\textbf{Skill Level} & \textbf{Cost} & \textbf{Narrative Reach} \\
\midrule
1 (Familiar) & 2 XP & Basic tasks, small spotlight \\
2 (Skilled) & 4 XP & Consistent success, broad coverage \\
3 (Expert) & 6 XP & Respected authority, unlocks mid-tier Prestige \\
4 (Master) & 8 XP & Cultural exemplar, broad story influence \\
5 (Legendary) & 10 XP & Setting-shaping, unlocks high-tier Prestige \\
\bottomrule
\end{tabular}

\subsection*{Final Reminder}
\begin{quote}
Every item and skill is a \emph{story hook first, mechanic second}.  
If it doesn’t suggest complications, obligations, or spotlight scenes, rework it until it does.
\end{quote}

\section{Killing Without Hit Points (GM Toolkit)}

\section{Principle}
\emph{Fate's Edge} uses no hit points. Threat accrues through three instruments:
\textbf{Complication Points (CP)} as damage, \textbf{Fatigue} as attrition, and \textbf{Clocks} as death timers.
Death is rare but possible; wipes are common when clocks fill and options run out.

\section{Complications as Damage}
When players roll, each \texttt{1} generates a CP for the GM. Spend boldly.

\subsection*{CP Spend Ladder}
\begin{tabular}{@{}p{1.2cm}p{12cm}@{}}
\toprule
\textbf{1 CP} & Compromise a tool (\emph{shield splinters} $\rightarrow$ \textbf{Compromised}); impose \textbf{+1 Fatigue} on one PC; add nuisance pressure (\emph{distant horns, watchlights}).\\
\textbf{2 CP} & Separate PCs; restrain or knock a PC down; endanger a follower/asset; advance a relevant \textbf{Clock} by 1.\\
\textbf{3+ CP} & Permanent injury; scene flips (\emph{floor collapses, fire spreads}); instant clock surge (+2); irrevocably burn an asset.\\
\bottomrule
\end{tabular}

\paragraph{Example (CP $\Rightarrow$ damage)}
The duelist wins the exchange but rolled two \texttt{1}s. GM spends \textbf{2 CP}:
\emph{``You drive him back, but your footing skids on oil. Gain 1 Fatigue and you're separated from the healer behind falling crates.''}

\section{Fatigue Is Your Soft HP}
Fatigue weakens rolls by forcing the player to re-roll successes.

\begin{tabular}{@{}p{3cm}p{10.5cm}@{}}
\toprule
\textbf{1 Fatigue} & Minor drain. On your next roll, re-roll \emph{one} success (player's choice).\\
\textbf{2 Fatigue} & Worn down. On each roll, re-roll \emph{one} success.\\
\textbf{3 Fatigue} & Failing fast. On each roll, re-roll \emph{two} successes.\\
\textbf{4 Fatigue} & Collapse/KO/Spiritual break. You fall out of the scene until treated or rescued.\\
\bottomrule
\end{tabular}

\paragraph{Clearing Fatigue}
A safe rest with adequate \emph{Supply} removes 1 level. You cannot clear Fatigue if the party's Supply clock is \textbf{Dangerously Low} or \textbf{Empty}.

\section{Clocks Are Death Timers}
Clocks track looming threats. Each has a name, size, and triggers that fill segments.

\subsection*{Clock Template}
\begin{description}[leftmargin=2cm]
  \item[Name:] What is ticking?
  \item[Size:] 4, 6, 8 segments (4 = fast, 8 = long arc).
  \item[Fill Triggers:] Fictional events, CP spends, time.
  \item[Consequences at Full:] What happens when it fills?
  \item[Pressure Moves:] How do you show it escalating?
\end{description}

\subsection*{Example 1: Supply (4-segment)}
\begin{description}[leftmargin=2cm]
  \item[Name:] Desert Crossing Supply
  \item[Size:] 4
  \item[Fill Triggers:] Harsh travel day; GM spends 2 CP on foraging mishap; pack loss; choice to \emph{move fast}.
  \item[Full:] Party collapses (all gain +2 Fatigue immediately) and must stop.
  \item[Pressure:] Stale water, ration fights, cracked lips, fraying bootlace.
\end{description}

\subsection*{Example 2: Peril (6-segment)}
\begin{description}[leftmargin=2cm]
  \item[Name:] City Watch Escalation
  \item[Size:] 6
  \item[Fill Triggers:] Alarm bell; 1 CP on a stealth miss; time passage; captured thug talks.
  \item[Full:] Full cordon and rooftop archers; rooftop extraction or capture.
  \item[Pressure:] Patrol frequency rises; checkpoints sprout; wanted posters.
\end{description}

\subsection*{Example 3: Doom (8-segment)}
\begin{description}[leftmargin=2cm]
  \item[Name:] Blood Moon Invocation
  \item[Size:] 8
  \item[Fill Triggers:] Each ritual step completed; 3 CP on arcane mishaps; each day at midnight.
  \item[Full:] The ritual completes---summoning, quake, or curse.
  \item[Pressure:] Cold drafts; red motes; animals fleeing; cultists hum.
\end{description}

\subsection*{How to Start a Clock (Step-By-Step)}
\begin{enumerate}
  \item \textbf{Name the danger} in plain fiction (\emph{``The mine floods''}).
  \item \textbf{Pick size} by pace: 4 (tense scene), 6 (session), 8 (arc).
  \item \textbf{Write two easy and two hard triggers} you can hit fairly.
  \item \textbf{Define the payoff} when it fills (no ambiguity).
  \item \textbf{Show pressure} every time you tick it (sensory tells).
\end{enumerate}

\paragraph{Worked Example: The Sluice Gate}
\begin{quote}\small
Name: \emph{Sluice Fails} (4). Triggers: failed \emph{Wits+Craft} to brace (GM can also spend 2 CP); \emph{Significant Time} passes with pressure unchecked; tremor; sabotage discovered too late. Full: \emph{Tunnel floods}---all PCs test \emph{Body+Survival} or gain +2 Fatigue and are separated downstream. Pressure: dripping mortar, rising water line, groaning timbers.
\end{quote}

\section{Off-Screen Fallout (Hurt Without Body Count)}
When things sour, harm the network.
\begin{itemize}
  \item Followers quit, flip, or die (2+ CP spend).
  \item Assets \emph{Neglected} $\rightarrow$ \emph{Compromised}. Repair costs downtime or XP (per your upkeep rules).
  \item Debts tighten: add or advance a \emph{Debt Clock}.
\end{itemize}

\section{When to Kill (Hard Move)}
Escalate to outright death only when:
\begin{enumerate}
  \item The fiction demands it (\emph{beheading, collapsing inferno, failed abyss leap}).
  \item The player consents (\emph{heroic sacrifice} or \emph{tragic end}).
  \item No credible bargain or asset burn remains (\emph{``Spend a follower/asset to live?''}---if not, death lands).
\end{enumerate}

\section{How to Wipe Without HP}
A party wipe occurs when:
\begin{enumerate}
  \item Most PCs hit \textbf{3--4 Fatigue} simultaneously, \emph{and}
  \item Relevant \textbf{Clocks fill} (Peril, Doom, Supply), \emph{and}
  \item No assets/Boons remain to blunt the blow.
\end{enumerate}
Outcomes: capture, exile, or extinction of their cause.

\section{Procedural Cheat Sheet (At the Table)}
\begin{enumerate}
  \item \textbf{On a PC roll:} Count \texttt{1}s $\rightarrow$ gain CP.
  \item \textbf{Spend CP immediately or bank} (declare what the table can see).
  \item \textbf{Tick a clock} when fiction or CP spend says so. Describe pressure.
  \item \textbf{Apply Fatigue} for exertion, deprivation, or CP spend.
  \item \textbf{Offer bargains} before death (\emph{burn a follower/asset/boon}).
\end{enumerate}

\section{Sample Play (Clocks + CP + Fatigue)}
\subsection*{Scene Frame}
Smuggling a witness over the palace wall at night. Active clocks:
\emph{Patrol Sweep (6)} at 3/6, \emph{Supply (4)} at 1/4.

\subsection*{Beats}
\paragraph{1) Scout the south wall}
Wits+Stealth (5d10): \texttt{10, 8, 6, 3, 1} $\Rightarrow$ 3 hits, \textbf{1 CP}.
GM spends \textbf{1 CP}: \emph{``You succeed, but your old rope is \textbf{Compromised}.''}
Patrol Sweep stays at 3/6.

\paragraph{2) Climb with the witness}
Body+Athletics (4d10): \texttt{7, 6, 5, 1} $\Rightarrow$ 2 hits, \textbf{1 CP}.
GM spends \textbf{1 CP} to tick \emph{Patrol Sweep} $\Rightarrow$ 4/6 and adds pressure:
\emph{``Lanterns brighten; bootsteps quicken along the inner walk.''}

\paragraph{3) Last push, running low}
GM fills \emph{Supply} to 2/4 due to \emph{Significant Time} climbing and cold.
Face tries to soothe the witness (Presence+Sway, 5d10): \texttt{9, 8, 2, 1, 1} $\Rightarrow$ 2 hits, \textbf{2 CP}.
GM spends \textbf{2 CP}: +1 \textbf{Fatigue} to the Face (shivering, breathless) and tick \emph{Patrol Sweep} to 5/6.

\paragraph{4) Extraction or collapse}
They have a choice: burn a Boon to activate an off-screen contact’s garden gate, \emph{or} risk one more roll with Fatigue penalties. If they roll and give the GM \textbf{1 CP}, the GM can fill \emph{Patrol Sweep} to 6/6 $\Rightarrow$ \emph{``Cordon slams shut; horns blare. Capture is on the table.''}

\section{Design Guardrails (So It Feels Fair)}
\begin{itemize}
  \item \textbf{Telegraph clocks.} Name them aloud or show them in fiction.
  \item \textbf{Spend CP transparently.} ``I'm using 2 CP to tick Patrol Sweep.''
  \item \textbf{Always offer a way out.} Boon, asset burn, or devil’s bargain.
  \item \textbf{Let Fatigue bite.} It's your attrition dial; don't be shy.
\end{itemize}

\section{Clock Starters (d6 Prompts)}
\begin{tabular}{@{}r p{11.5cm}@{}}
1 & \textbf{Peril (6):} ``\emph{They triangulate your hideout.}'' Triggers: noise, bribed neighbor, CP spend.\\
2 & \textbf{Supply (4):} ``\emph{The cold eats your prep.}'' Triggers: storm, lost mule, CP spend.\\
3 & \textbf{Doom (8):} ``\emph{Blood moon eats the sky.}'' Triggers: daily tick, ritual step, CP spend.\\
4 & \textbf{Debt (6):} ``\emph{Collector circles.}'' Triggers: downtime passes, flash spend, CP spend.\\
5 & \textbf{Injury (4):} ``\emph{Cracking ice underfoot.}'' Triggers: sprint, fall, CP spend.\\
6 & \textbf{Hunt (6):} ``\emph{Rival fixer stalks your routes.}'' Triggers: contact flips, trace, CP spend.\\
\end{tabular}

\bigskip
\noindent\emph{Use these tools together: CP for sharp blows, Fatigue for steady bleed, Clocks for the guillotine. Death is a choice at the edge of the blade; wipes happen when the clocks say ``time.''}

\section*{GM Toolkit: The Hunger Below}

\subsection*{Core Clocks}
Use these clocks to track the campaign’s tension. Add or subtract ticks as PCs succeed, fail, or make narrative choices.

\begin{description}[leftmargin=2cm]
  \item[Waystone Cracks (6-step):] Advances when PCs delay, fail an investigation, or rival factions tamper with wards. When full, protective mists fail entirely.
  \item[Whisper Spread (8-step):] Advances when NPCs succumb to temptation, when the PCs ignore faction crises, or when they spend downtime in corrupted regions. When full, at least one major faction falls under the Wyrm’s sway.
  \item[Chains Breaking (10-step):] Advances on major defeats, ritual mishaps, or if both other clocks fill. When full, the Primordial Wyrm rises.
\end{description}

\subsection*{Faction Clocks (Optional)}
Track how each group responds to the crisis:
\begin{itemize}
  \item Vigil’s Plan (6-step): Harness the Wyrm as a weapon.
  \item True Masons’ Faith (4-step): Re-bind with sacred sacrifice.
  \item Mistland Revolt (6-step): Humans abandon trust in outsiders.
  \item Valewood Split (4-step): Fey courts fracture between war and retreat.
\end{itemize}

\subsection*{Scene Templates}

\paragraph{Scene 1: The Cracked Waystone}
\textbf{Setup:} PCs encounter a waystone buzzing with unstable energy. Fey creatures stalk the thinning mists.
\textbf{Clock Impact:} +1 Waystone Crack if not stabilized.
\textbf{Twist:} Stabilization reveals runes etched from inside the stone, not outside.

\paragraph{Scene 2: Whispers in the Market}
\textbf{Setup:} A crowd gathers around a preacher claiming the mines hold wealth for all. The voice is too persuasive.
\textbf{Clock Impact:} +1 Whisper Spread if not contained.
\textbf{Twist:} The preacher has no agenda—he is an ordinary farmer channeling the Wyrm’s influence.

\paragraph{Scene 3: Khaz Vurim’s Gate}
\textbf{Setup:} PCs find the sealed Aeler mine. The gate is chained with runes that pulse like a heartbeat.
\textbf{Clock Impact:} +1 Chains Breaking if tampered with.
\textbf{Twist:} A rival faction arrives mid-scene, claiming rights to the mine.

\paragraph{Scene 4: The Fey Conclave}
\textbf{Setup:} Valewood emissaries demand the PCs choose a side in their court split.
\textbf{Clock Impact:} +1 Whisper Spread if they hesitate or alienate both sides.
\textbf{Twist:} Both sides insist the other has already been compromised by whispers.

\paragraph{Scene 5: The Vigil’s Bargain}
\textbf{Setup:} Agents of the Vigil offer coin, protection, or rare lore if the PCs hand over key findings.
\textbf{Clock Impact:} Advance Vigil’s Plan if PCs cooperate.
\textbf{Twist:} The Vigil doesn’t just want data—they want to bind the Wyrm into a weapon.

\subsection*{Sample NPCs}

\paragraph{Eldric Stonewarden (Aeler Mason)}
\emph{Grim, cautious, wracked by guilt.} Believes the Wyrm should never have been discovered. Will help PCs with stonecraft and rituals, but hides his clan’s culpability.

\paragraph{Liraen Leafveil (Valewood Envoy)}
\emph{Charming, pragmatic, sly.} Wants peace between courts, but secretly tempted to use the Wyrm’s whispers to force unity.

\paragraph{Captain Veyra (Vigil Agent)}
\emph{Cold, calculating, relentless.} Frames herself as an ally, but ultimately serves the Vigil’s larger agenda: conquest by control.

\paragraph{The Voice Beneath (Whisper Projection)}
\emph{Smooth, alien, tempting.} Appears in dreams or reflective stone. Offers PCs exactly what they most want—at a price.

\subsection*{Endgame Tools}
When finale approaches, choose 2--3 faction clocks that are closest to full. Use those to define the battlefield:
\begin{itemize}
  \item \textbf{High Vigil Clocks:} PCs may face militarized zealots trying to chain the Wyrm mid-battle.
  \item \textbf{High True Mason Clocks:} PCs must choose whether to sacrifice something dear to re-bind the Wyrm.
  \item \textbf{High Whisper Spread:} At least one PC may receive a direct temptation—personal power in exchange for betrayal.
\end{itemize}

\section{Adventure Generation with a Standard Deck}

This subsystem allows a GM to generate locations, factions, complications, and rewards using a standard 52--card deck. It is designed as a companion to the \emph{Deck of Consequences} but works independently for fast prep and improvisation.

\subsection{Suit Meanings}
\begin{description}[leftmargin=2cm]
  \item[Spades:] Places (where events occur).
  \item[Hearts:] People and Factions (who is involved).
  \item[Clubs:] Complications and Threats (what makes it messy).
  \item[Diamonds:] Rewards and Leverage (why it matters).
\end{description}

\subsection{Rank Severity and Clock Size}
The card rank determines the size of the primary Clock for the scene or mission:
\begin{itemize}
  \item 2--5 (Minor): 4--segment Clock
  \item 6--10 (Standard): 6--segment Clock
  \item J, Q, K (Major): 8--segment Clock
  \item Ace (Pivotal): 10--segment Clock
\end{itemize}

Color influences tone:
\begin{itemize}
  \item Black suits (♠♣): travel hazards, tangible threats, fatigue.
  \item Red suits (♥♦): social intrigue, reputational pressure.
\end{itemize}

\subsection{Draw Procedures}
\paragraph{Quick Hook (2 cards).} Draw one Spade and one Heart. The Spade provides the place, the Heart the faction. Use the higher rank to set the Clock.

\paragraph{Full Seed (4 cards).} Draw until one card of each suit appears.
\begin{enumerate}
  \item Spade = location
  \item Heart = main actor/faction
  \item Club = complication
  \item Diamond = reward/leverage
\end{enumerate}
The highest rank sets the main Clock. If multiple face cards or Aces appear, begin parallel Clocks.

\paragraph{Act Builder.} For each act or session, draw three cards: setting, actor, complication. Save Diamonds to foreshadow leverage or as act payoffs.

\subsection{Combo Rules}
\begin{itemize}
  \item Pair (same rank): recurring motif with a twist.
  \item Run (3+ sequential ranks): momentum—reduce the main Clock by 1 segment.
  \item Flush (3+ same suit): strongly theme the act toward that axis.
  \item Face + Ace: reveal a hidden patron or power behind the drawn element.
  \item All one color: GM gains 1 free Complication Point in that scene.
\end{itemize}

\subsection{Suit Tables (Full 52)}

\paragraph{Spades (Places).}
\begin{itemize}
  \item 2: Toll bridge at a foggy ford
  \item 3: Abandoned location, squatters within
  \item 4: Wilderness path marked by strange cairns
  \item 5: Lonely inn with too few patrons
  \item 6: Old quarry turned smugglers’ pit
  \item 7: Overgrown orchard or vineyard hiding a shrine
  \item 8: Border outpost, undermanned and tense
  \item 9: Collapsed mine, faint chanting below
  \item 10: Fortress granary with suspicious activity
  \item J: City undercroft used by syndicates
  \item Q: Island cloister where tides conceal secrets
  \item K: High keep overlooking three borders
  \item A: Sealed under-vault that hums at night
\end{itemize}

\paragraph{Hearts (People/Factions).}
\begin{itemize}
  \item 2: Wary farmer seeking protection
  \item 3: Young patrol captain seeking a break
  \item 4: Ambitious scribe angling for promotion
  \item 5: Traveling merchant juggling debts
  \item 6: Sectarian monk with hidden doubts
  \item 7: Temple advocate torn between creeds
  \item 8: Caravan mistress guarding a secret cargo
  \item 9: Disgraced noble clinging to relevance
  \item 10: Guild factor balancing ledgers and lies
  \item J: Street gang leader with local sway
  \item Q: Exiled courtier with dangerous friends
  \item K: Kahfagian commodore off the books
  \item A: ``Benefactor'' whose money moves borders
\end{itemize}

\paragraph{Clubs (Complications/Threats).}
\begin{itemize}
  \item 2: Slippery mudslide blocks the way
  \item 3: Bad weather ruins supplies
  \item 4: Hidden snare line across the approach
  \item 5: Distracted allies miss a cue
  \item 6: A rival party races you to the goal
  \item 7: Local law arrives at the worst time
  \item 8: Beast attack; mundane but dangerous
  \item 9: Your plan leaks; ambush in the wings
  \item 10: Magical hazard: wards failing, curses flaring
  \item J: Insider betrayal during the action
  \item Q: Ally-of-convenience turns on you
  \item K: Entire faction shifts allegiance mid-scene
  \item A: The past returns to claim someone
\end{itemize}

\paragraph{Diamonds (Rewards/Leverage).}
\begin{itemize}
  \item 2: Trinket or token worth 1 Boon
  \item 3: Favor marker from a minor official
  \item 4: Small cache of supplies (1 free Supply segment)
  \item 5: Local renown; 1 Boon usable in this region
  \item 6: Pouch of coin; enough to offset upkeep once
  \item 7: Access to training: reduce XP cost of 1 skill by 1
  \item 8: A Cap~2 follower offers service
  \item 9: A minor holding (workshop, stall, shack)
  \item 10: Letters of credit with a guild (worth 2 Boons)
  \item J: Blackmail dossier (usable as leverage)
  \item Q: Land deed or minor title (Asset)
  \item K: Noble charter or mercantile license (Standard Asset)
  \item A: Artifact-grade leverage (4 XP to keep)
\end{itemize}

\subsection{Mechanical Links}
\begin{itemize}
  \item Clock size: determined by rank severity.
  \item Supply/Fatigue: if the highest card is black, begin Low Supply or add a Fatigue tick.
  \item Free CP: if all cards are the same color, the GM banks 1 Complication Point.
  \item Diamonds: represent Boons, XP-gated Assets, or one-shot favors. A one-shot favor costs 1 Boon (or 2 XP in desperation) to activate.
\end{itemize}

\subsection{Example Draw}
\begin{quote}
\textbf{Cards:} ♠10, ♥7, ♣9, ♦Q

\textbf{Result:}
\begin{itemize}
  \item Place: Fortress granary with suspicious ledgers (6--segment Clock).
  \item People: Temple advocate torn between creeds.
  \item Complication: Plan leaks; ambush imminent.
  \item Reward: Land deed or minor title (Asset).
\end{itemize}
\textbf{Hook:} Grain is being skimmed to fund a private militia. A wavering temple advocate may testify—if the party can extract them before the ambush springs.
\end{quote}

\section{Absent PCs and Shared Resources}

Characters remain part of the world even when their players are absent. The group may reference their influence, but never treat them as “free actions.” Two separate options handle this cleanly:

\subsection*{1. Off-Screen Asset Activation (Preparation Only)}
\begin{itemize}
  \item The party may activate an absent PC’s \textbf{off-screen assets} (followers, safehouses, trade links, etc.).
  \item These uses remain \emph{off-screen}: they prepare conditions, set clocks, or provide leverage, but do not manifest as sudden on-screen rescues.
  \item \textbf{Cost:} 2 Boons \emph{or} 1 Boon + 1 XP (paid by the table).
  \item \textbf{Limit:} Once per scene.
  \item \textbf{Fiction First:} The use must be established before or between scenes. Example: “Danny’s network smuggled a wagon inside yesterday” is valid; “Danny’s friend appears out of nowhere to save us now” is not.
\end{itemize}

\subsection*{2. Borrowed Action (On-Screen, Once per Session)}
With the absent player’s consent, the group may ask the GM to perform one visible, decisive action as that character:
\begin{itemize}
  \item This may include a single roll, an intervention in dialogue, or a crucial gesture during a scene.
  \item \textbf{Cost:} The absent PC gains a new \textbf{Complication}, assigned by the GM and grounded in the fiction.
  \item \textbf{Limit:} Once per session per absent character.
\end{itemize}

\subsection*{Example}
\begin{quote}
\textbf{Asset Activation:} “Before the heist, Danny’s smuggler contact hid a wagon by the docks. Let’s spend 2 Boons to have it ready.”  

\textbf{Borrowed Action:} Cornered in negotiations, the party calls on Danny’s title. The GM rolls Presence + Command as Danny—but adds the Complication \emph{‘Overplayed His Title’} to Danny’s sheet.
\end{quote}

\subsection*{GM Guidance}
\begin{itemize}
  \item Scale the Complication to the gravity of the borrowed action:
    \begin{description}
      \item[Minor Action:] A single roll or small assist (e.g., pulling a lever, offering brief testimony). Apply a light Complication such as \emph{“Fatigued”} or \emph{“In Someone’s Debt”}.
      \item[Major Action:] A scene-defining choice or bold intervention (e.g., intimidating a magistrate, striking a killing blow). Apply a weighty Complication such as \emph{“Enemy at Large”} or \emph{“Notoriety”}.
    \end{description}
  \item Ensure the absent PC remains playable when they return. Do not stack multiple heavy Complications unless agreed with the player.
  \item The GM should remind the table that these options are emergency levers, not default strategies.
\end{itemize}

\section{NPC Motive Generating Deck}
\addcontentsline{toc}{section}{Appendix E: NPC Motive Generating Deck}

\subsection*{How to Use This Deck}

\noindent To quickly generate an NPC’s motive and flavor at the table:

\begin{enumerate}
  \item \textbf{Shuffle a standard 52-card deck.}
  \item \textbf{Draw four cards, one for each table:}
    \begin{enumerate}
      \item Hearts = Ambition (what the NPC wants).
      \item Clubs = Beliefs (their inner philosophy).
      \item Diamonds = Personality (how they present outwardly).
      \item Spades = Twist (the hidden complication).
    \end{enumerate}
  \item \textbf{Interpret the results.} Combine the four draws into a coherent picture. If contradictions appear, use them—illogical motives often make the most interesting characters.
  \item \textbf{Add hooks.} Each entry comes with a suggested complication or expression. Use this to tie the NPC back into the current adventure or to foreshadow future trouble.
  \item \textbf{Optional:} For minor NPCs, draw only one or two cards (e.g., Personality + Twist) to keep them simple. For major NPCs, use all four.
\end{enumerate}

\noindent \textbf{GM Tip:} If a result clashes with the established story, adjust by shifting the rank up or down by one, or simply swap the suit for a better fit. Treat the tables as creative prompts, not strict rules.

\noindent NPC motives are generated by drawing four cards from a standard 52-card deck.  
\begin{enumerate}
  \item \textbf{Ambition} (Hearts): what the NPC wants.  
  \item \textbf{Beliefs} (Clubs): internal worldview, values, or philosophy.  
  \item \textbf{Personality} (Diamonds): how the NPC presents outwardly.  
  \item \textbf{Twist} (Spades): complication, hidden agenda, or destabilizer.  
\end{enumerate}

Each table maps card rank to an entry. For major NPCs, draw all four; for minor NPCs, draw only one or two.

\subsection*{Ambition (Hearts)}
\begin{longtable}{|p{0.1\linewidth}|p{0.25\linewidth}|p{0.55\linewidth}|}
\hline
\textbf{Card} & \textbf{Ambition} & \textbf{Hook Suggestion} \\
\hline
2 & Acquire wealth & Desperate gambler, hoarder, thief \\
3 & Gain prestige & Climbing hierarchy, faking noble blood \\
4 & Secure power & Aims to control guild, militia, shrine \\
5 & Discover truth & Hunts ancient ruins, pries into secrets \\
6 & Protect someone/something & Bodyguard, sacred relic keeper \\
7 & Destroy rival & Scheming, poisoning, hiring assassins \\
8 & Escape obligation & Draft-dodger, runaway betrothed \\
9 & Build legacy & Starting dynasty, monument, or school \\
10 & Serve cause & Religious zealot, political revolutionary \\
Jack & Pursue love & Entangled romance, doomed affair \\
Queen & Seek revenge & Cold calculation, sudden violence \\
King & Master a craft & Genius artisan, perfectionist \\
Ace & Transcend mortality & Cultist, lich-aspirant, philosopher \\
\hline
\end{longtable}

\subsection*{Beliefs (Clubs)}
\begin{longtable}{|p{0.1\linewidth}|p{0.25\linewidth}|p{0.55\linewidth}|}
\hline
\textbf{Card} & \textbf{Belief} & \textbf{Expression} \\
\hline
2 & Pragmatism & Ends justify means \\
3 & Honor & Bound by oath or tradition \\
4 & Faith & Trust in divine, prophecy, or fate \\
5 & Rationalism & Demands proof, loves logic \\
6 & Hedonism & Pleasure first, consequences later \\
7 & Communalism & Group over individual \\
8 & Libertarianism & Freedom above all \\
9 & Fatalism & ``It is already written'' \\
10 & Idealism & Believes world can be better \\
Jack & Cynicism & Distrusts everyone and everything \\
Queen & Self-determinism & Only I decide my path \\
King & Supremacism & One group or idea is superior \\
Ace & Paradoxical & Holds contradictory beliefs at once \\
\hline
\end{longtable}

\subsection*{Personality (Diamonds)}
\begin{longtable}{|p{0.1\linewidth}|p{0.25\linewidth}|p{0.55\linewidth}|}
\hline
\textbf{Card} & \textbf{Personality} & \textbf{Expression} \\
\hline
2 & Cheerful & Disarms with optimism \\
3 & Brooding & Always in shadow \\
4 & Pompous & Full of grandeur \\
5 & Humble & Soft-spoken, self-effacing \\
6 & Cunning & Always scheming \\
7 & Naïve & Earnest, easily duped \\
8 & Ruthless & Cold efficiency \\
9 & Charming & Social magnet, honeyed words \\
10 & Stoic & Unreadable, unflinching \\
Jack & Erratic & Wild mood swings \\
Queen & Nurturing & Maternal, protective \\
King & Commanding & Projects authority \\
Ace & Masked & Persona hides true face \\
\hline
\end{longtable}

\subsection*{Twist (Spades)}
\begin{longtable}{|p{0.1\linewidth}|p{0.25\linewidth}|p{0.55\linewidth}|}
\hline
\textbf{Card} & \textbf{Twist} & \textbf{Complication Hook} \\
\hline
2 & Debt & Owes someone dangerous \\
3 & Betrayer & Will eventually turn on allies \\
4 & Doomed & Prophecy or sickness awaits \\
5 & Impostor & Not who they say they are \\
6 & Haunted & Spirit, trauma, or guilt follows them \\
7 & Pawn & Controlled by stronger hand \\
8 & Double-life & Upright citizen + criminal underworld \\
9 & Cursed & Literal or metaphorical affliction \\
10 & Obsessed & Consumed by one idea \\
Jack & Incompetent & Overestimates own ability \\
Queen & Sympathetic flaw & Their vice endears them \\
King & Secret noble/bloodline & Raises stakes \\
Ace & Wild card & GM improvises unpredictable element \\
\hline
\end{longtable}

\subsection*{Example Draw}
\noindent Ambition: 7♥ Destroy rival. \\
Beliefs: 5♣ Rationalism. \\
Personality: Jack♦ Erratic. \\
Twist: 9♠ Cursed. \\
\medskip

\noindent \textbf{Result:} A cursed scholar convinced only logic can free them, but their unstable moods and obsessive hunt to destroy a rival sorcerer make them as much a liability as an ally.

\section{Geography of the Known World}
\label{sec:geography}

The \textbf{Amaranthine Sea} is the inland heart of civilization. Lands are named by their bearing from this sea:
\begin{itemize}
  \item \textbf{Vililan} --- the West (root of the demonym \emph{Vilikari} for its Ikari peoples);
  \item \textbf{Akilan} --- the South;
  \item \textbf{Ostrilan} --- the East (root of \emph{Ostrikari} for eastern Ikari).
\end{itemize}

\subsection{The Aelerian Mountain Range}
\textbf{Course:} An east--west spine along southern Vililan. \\
\textbf{Moisture engine:} Moist air from the Amaranthine rises against the Aelerians, producing heavy orographic rain on the \emph{southern faces} and the adjacent \emph{northern coasts of Akilan}. \\
\textbf{Climatic results:}
\begin{itemize}
  \item Akilan's north coast: unexpectedly lush, densely settled belts and river mouths.
  \item Vililan's Aelerian north slopes: fertile foothills and valleys.
  \item Vililan interior: rain shadow yields temperate/continental bands.
  \item Akilan interior and far south: increasingly arid with distance from sea and range.
\end{itemize}
\textit{Designer touchstone (non-diegetic):} Amaranthine $\approx$ Mediterranean scale; Aelerians $\approx$ Alps placed farther north.

%--------------------------------------------------------------------
\subsection{Vililan --- The Western Continent}
\textbf{Scope:} Northern \& western shores of the Amaranthine and the great lands beyond; former core of the Utaran Empire and its successor states.

\subsubsection{Principal Polities \& Regions (west-to-east)}
\begin{itemize}
  \item \textbf{Kahfagia}: Maritime oligarchy controlling the \emph{Titan's Throat} strait and dominating \emph{Dolmis Sea} routes.
  \item \textbf{Ameria}: Partitioned buffer along the northern Amaranthine coast, balancing Kahfagia and Oshiira.
  \item \textbf{Ecktoria}: Wealthy remnant clinging to imperial forms in the west.
  \item \textbf{Acasia}: Fractured province of petty crowns; exception: cosmopolitan port of \emph{Silkstrand}.
  \item \textbf{Vhasia}: Frontier crown under ducal power; cities \emph{Lence} (former capital) and \emph{Vhaston}.
  \item \textbf{Viterra}: Consolidated successor; \emph{Fens} and \emph{Riverlands} with \emph{Tarlington} and \emph{Valora}.
  \item \textbf{Thepyrgos}: Scholarly city-state on the \emph{Astroegro} peninsula; marble archives; notable high-elf population.
  \item \textbf{Ubral}: Clan hill-forts; oaths, cairns, and iron.
  \item \textbf{The Mistlands}: Fertile, fogged valleys under a \emph{Dwarven protectorate}; \emph{Payden's Port} is the trade hinge.
  \item \textbf{Ikari Lands (Vilikari)}: Agrarian heartlands along coasts and river valleys.
  \item \textbf{Midh Adkaz}: Neutral frontier city from the Ykrul Wars; parley ground for Vililan, Ykrul, and Ikari.
  \item \textbf{Ykrul Steppes}: Vast nomad grasslands to far NW and NE (western and eastern confederations).
  \item \textbf{Valewood}: Immense, uncanny forest; mutable geography in the far east.
  \item \textbf{Haayr Peninsula}: Rugged, contested range jutting between the \emph{Dolmis} and \emph{Amaranthine}.
\end{itemize}

\subsubsection{Seas \& Straits}
\begin{itemize}
  \item \textbf{Amaranthine Sea} (core basin).
  \item \textbf{Dolmis Sea} (northeast), linked inland to the \textbf{Alberriden Sea} by the \textbf{Yrolka River} delta.
  \item \textbf{Alberriden Sea} (inland).
  \item \textbf{Shoreless Bay} (outer ocean) reached through \textbf{Titan's Throat}.
\end{itemize}

%--------------------------------------------------------------------
\subsection{Akilan --- The Southern Continent}
\textbf{Scope:} Southern shores of the Amaranthine extending deep south; strongly shaped by Aelerian rainfall on the north rim and desiccation inland.

\subsubsection{North Rim \& Rain-Favored Belts}
\begin{itemize}
  \item \textbf{Oshiira}: Logistic--bureaucratic empire of canals and granaries; martial orders (e.g.\ the Long Sorrow); holds terrain above the \emph{Crimson Basin}.
  \item \textbf{Sekogo}: Druidic governance across the \emph{Crimson Basin} western jungles; lagoon ports and river management.
  \item \textbf{Taharka}: Terrace highlands and hydraulic works astride the \textbf{Khesai River} through the \emph{Mkusaro Highlands}.
  \item \textbf{The Crimson Basin}: Immense rainforest drained by the \textbf{Enjwe River}; contested between Wood Elves and Oshiiran settlers.
\end{itemize}

\subsubsection{Eastern Littoral \& Interior}
\begin{itemize}
  \item \textbf{Galanina}: Sidhi coastal realm along the eastern Amaranthine.
  \item \textbf{Ashaan}: Fractured successor of a slaver-empire along the eastern Amaranthine and the \textbf{Sea of Tears}; centered on the \textbf{Khesai} delta.
  \item \textbf{Cultural Complex}: 
  \begin{itemize}
    \item \emph{Sidhi} (Amazigh/Persian analogs) in ports and trade cities,
    \item \emph{Perishi} (Iranian analogs) in inland kingdom belts,
    \item \emph{Fhara} (Arabian analogs) along desert/oasis routes,
    \item \emph{Kuva} (Turkic/Mongol analogs) in steppe marches---especially NE toward \textbf{Dhahara},
    \item \emph{Ashaani} legacies scattered through the former empire.
  \end{itemize}
  \item \textbf{The Great Desert}: Dominates the interior around the Khesai's northern reaches.
\end{itemize}

\subsubsection{Farther South}
Harsher heat and aridity with scattered kingdoms and nomad circuits separated by long stretches of unpeopled waste.

\subsubsection{Seas}
\begin{itemize}
  \item \textbf{Amaranthine Sea};
  \item \textbf{Sea of Tears} (gateway to the Dhaharan Sea);
  \item \textbf{Dhaharan Sea} (outer ocean along the peninsula).
\end{itemize}

%--------------------------------------------------------------------
\subsection{Ostrilan --- The Eastern Continent}
\textbf{Scope:} Eastern Amaranthine shores outward across vast landmasses and archipelagos; monsoons, old empires, and steppe power.

\subsubsection{Eastern Shores \& Transitional Fringe}
\begin{itemize}
  \item \textbf{Galanina} and \textbf{Ashaan} continue along this rim, blending into eastern trade circuits with Sidhi, Perishi, Fhara, Kuva, and Ashaani diasporas.
\end{itemize}

\subsubsection{The Dhaharan Peninsula}
\begin{itemize}
  \item \textbf{Dhahara}: A crossroads peninsula with:
  \begin{itemize}
    \item \emph{Himdal Aeler} marches in the north, homeland of the \textbf{Ostrikari} (eastern Ikari);
    \item Desert oases and incense routes across the center;
    \item A \emph{monsoon coast} to the south with blue-water harbors.
  \end{itemize}
\end{itemize}

\subsubsection{Far Eastern Landmasses \& Seas}
\begin{itemize}
  \item \textbf{Sihai}: Vast centralized empire.
  \item \textbf{Nihon}: Storm-wracked clan archipelago; perennial Sihai rival.
  \item \textbf{Ayokha}: Monsoon-fed temple--trade kingdom anchoring Hintara routes.
\end{itemize}

\subsubsection{Steppes \& Northern Reaches}
Great steppe highways east and north of Dhahara reach toward Sihai; power shifts among riding confederations. Harsher northern bands host Ostrikari enclaves with ties echoing across the Ykrul worlds to Vililan.

\subsubsection{Ikari of the East}
\textbf{Ostrikari} ancestral zones lie in the \emph{Himdal Marches} and adjacent northern steppes of Ostrilan/Dhahara.

\subsubsection{Seas}
\begin{itemize}
  \item \textbf{Amaranthine Sea} (to the west);
  \item \textbf{Sea of Tears} (southwest link to Akilan);
  \item \textbf{Dhaharan Sea} (peninsula ocean);
  \item \textbf{Hintara Ocean} (far east);
  \item \textbf{Nasan Sea} (between Sihai and Nihon).
\end{itemize}

%--------------------------------------------------------------------
\subsection{Major Waterways \& Chokepoints (Cross-Region)}
\begin{itemize}
  \item \textbf{Titan's Throat}: Gate from the Amaranthine to the \emph{Shoreless Bay}; controlled by Kahfagia.
  \item \textbf{Sea of Tears} $\rightarrow$ \textbf{Dhaharan Sea}: Southern trade artery and pilgrimage route.
  \item \textbf{Yrolka River}: Dolmis--Alberriden conduit; barge/portage economy.
  \item \textbf{Khesai River}: Ashaan heartland; floodplain states and delta control.
  \item \textbf{Enjwe River}: Spine of the \emph{Crimson Basin} jungle trade.
\end{itemize}

%--------------------------------------------------------------------
\subsection{Demonyms \& Usage (Style Guide)}
\begin{itemize}
  \item \textbf{Vililan} (land), \textbf{Vilikari} (Ikari peoples of Vililan).
  \item \textbf{Ostrilan} (land), \textbf{Ostrikari} (Ikari peoples of the east).
  \item \textbf{Akilan} (land); avoid broad umbrellas---use local peoples (\emph{Sidhi, Perishi, Fhara, Kuva}, etc.).
  \item Keep designer analogs (e.g., ``Black Sea analog'') out of diegetic text; reserve for GM notes.
\end{itemize}

%--------------------------------------------------------------------
\subsection{Map Notes (SRD Conventions)}
\begin{itemize}
  \item \textbf{Climate bands}: show orographic wet zones on Aelerian south faces and Akilan north coast; shade the Vililan rain shadow.
  \item \textbf{Trade lanes}: mark Titan's Throat, Sea of Tears, Yrolka, Khesai, Enjwe.
  \item \textbf{Protectorates \& Marches}: hatched overlays (e.g., Dwarven Protectorate in the Mistlands; Himdal Aeler marches).
  \item \textbf{Unstable terrain}: stipple fill for the Valewood.
  \item \textbf{Disputed zones}: cross-hatch Haayr Peninsula and Ameria partitions.
\end{itemize}

%--------------------------------------------------------------------
\subsection{Play Hooks (System-Agnostic)}
\begin{itemize}
  \item \textbf{Strait Rent}: Tariffs at Titan's Throat spike---seek escorts or alternate Dolmis$\rightarrow$Alberriden routes.
  \item \textbf{Rain-Shadow Famine}: Weak Aelerian snowpack stresses Vililan granaries; Oshiira grain fleets become political weapons.
  \item \textbf{Steppe Thunder}: Kuva and Ykrul emissaries convene at Midh Adkaz to redraw grazing rights.
  \item \textbf{Fog and Stone}: Mistlands dwarf-wardens summon aid: something moves beneath the cairns.
  \item \textbf{Monsoon Pact}: Ayokha brokers a blue-water convoy between Sihai and Nihon---unless Dhahara undercuts them.
\end{itemize}

%--------------------------------------------------------------------
\subsection{Pronunciation (Optional)}
\begin{itemize}
  \item \emph{Aelerian} (AY-leer-ee-an), \emph{Amaranthine} (am-uh-RAN-theen), \emph{Akilan} (ah-KEE-lan).
  \item \emph{Vililan} (vih-LEE-lan), \emph{Ostrilan} (OSS-trih-lan), \emph{Oshiira} (oh-SHEER-ah).
  \item \emph{Galanina} (gah-lah-NEE-nah), \emph{Ashaan} (ah-SHAHN), \emph{Dhahara} (dha-HAH-rah).
  \item \emph{Ykrul} (EE-krool), \emph{Vhasia} (VAH-see-uh), \emph{Thepyrgos} (theh-PEER-goss).
\end{itemize}

\medskip
\noindent\textit{SRD intent:} This section fixes canonical geography, climate logic, and naming. Keep mechanical modules (travel times, encounter tables) elsewhere; this section is system-neutral and cartography-ready.

% ===== SRD Section: Magic — Spells, Backlash, and Generation =====
\section*{Magic: Spells, Backlash, and Generation}
\addcontentsline{toc}{section}{Magic: Spells, Backlash, and Generation}

\subsection*{Design Goals (table-facing)}
\begin{itemize}
  \item \textbf{Fiction first.} Spells change \emph{position, effect, or rails}—they do not create Diamonds and they do not exist to stack raw damage dice.
  \item \textbf{Cost with teeth.} Backlash scales with Complication Points (CP) and should drive story: noise, attention, injuries, hazards, or new problems.
  \item \textbf{Short, sharp, seen.} Default durations are brief (\emph{3 beats for buffs, 1 beat for area control}). Sustaining beyond defaults costs \textbf{1 Fatigue per beat}.
  \item \textbf{Visible power.} Flashy magic is \textbf{Obvious}: the GM may tick \textbf{Crowd +1} or offer a hard bargain upon entry.
  \item \textbf{No snowplow.} Active spell advantages \textbf{count toward Over-Stack} when the crew enters a scene with multiple structural edges.
\end{itemize}

\subsection*{Casting Procedure (Weave Roll)}
\begin{enumerate}
  \item \textbf{Choose Art \& Intent.} Name the Art and the concrete narrative effect.
  \item \textbf{Set DV.} Use the DV guidance below (typical DV 1--3).
  \item \textbf{Roll the Weave.} Resolve per core rules (position/effect apply).
  \item \textbf{Apply Outcome.} On success, apply a clear outcome: start \emph{Controlled}, grant \emph{+1 effect}, or tick a rail/clock as specified by the spell.
  \item \textbf{Backlash.} Tally CP from the roll; consult the Backlash entry. A caster may once per spell \emph{shunt} \textbf{1--2 CP} into \textbf{Fatigue 1} or \textbf{Harm 1} (their choice).
\end{enumerate}

\subsection*{Backlash Severity Table (guidance)}
\begin{tabular}{@{}l l@{}}
\toprule
\textbf{CP} & \textbf{Typical Consequence} \\
\midrule
1--2 & Minor nuisance or tell; short-lived cost, noise, or reveal. \\
3--4 & Noticeable setback: a real hazard, condition, or new pressure clock. \\
5+  & Major turn: scene shifts, a new foe/clock enters, or severe condition. \\
\bottomrule
\end{tabular}

\subsection*{Global Guardrails (mechanical)}
\begin{itemize}
  \item \textbf{Duration defaults:} buffs \emph{3 beats}; areas \emph{1 beat}. Sustaining costs \textbf{1 Fatigue/beat}.
  \item \textbf{Stacking:} same-Art buffs do not stack; take the best one.
  \item \textbf{Diamond line:} spells cannot create Diamonds; only position/effect/rails/clock movement.
  \item \textbf{Over-Stack:} any standing spell advantage counts toward Over-Stack at scene start.
\end{itemize}

% -------------------- SPELL LIST --------------------
\subsection*{Spell List (ready to use)}
\paragraph{Cinder-Fist (Pyromancy, DV 2).}
\emph{Effect:} Your hand ignites; for up to \textbf{3 beats}, unarmed actions gain \textbf{+1 effect}. On a strong hit vs flammables, you may \textbf{Hazard --1} once. Requires a free hand; \textbf{Obvious}.\\
\emph{Backlash:} \textbf{1 CP} flame gutters after one use; \textbf{2 CP} caster takes \textbf{Fatigue 1} and scorches sleeve; \textbf{3 CP} unintended item ignites (start a small Hazard clock); \textbf{4+ CP} fire wreathes arm (\textbf{Harm 2}) and draws attention.

\paragraph{Stone-Sense (Geomancy, DV 1).}
\emph{Effect:} Sense through contiguous stone (\textbf{30 ft}); learn flaws/layout. Grant \textbf{Controlled} to a single move/breach using this read.\\
\emph{Backlash:} \textbf{1 CP} echo of ancient pain (distraction); \textbf{2 CP} muddled by a competing presence; \textbf{3 CP} partial petrification (\textbf{-1 die} physical for the next scene); \textbf{4+ CP} something attuned in the stone is alerted to you.

\paragraph{Still the Currents (Hydromancy, DV 2).}
\emph{Effect:} Calm a \textbf{10 ft} water square for \textbf{1 beat}. Either \textbf{start Controlled} for one crossing \emph{or} \textbf{Hazard --1} in that zone. Sustaining costs \textbf{1 Fatigue/beat}.\\
\emph{Backlash:} \textbf{1 CP} water becomes too mirror-flat (stealth tell); \textbf{2 CP} chop intensifies outside the zone; \textbf{3 CP} area stays eerily still until disturbed (curiosity magnet); \textbf{4+ CP} water turns foul/stagnant, inviting pests.

\paragraph{Cloak of Shadows (Umbramancy, DV 2).}
\emph{Effect:} In \textbf{dim or darker} light, target starts \textbf{Controlled vs sight-based detection}; bright light ends the effect. Does not help vs sound/scent.\\
\emph{Backlash:} \textbf{1 CP} slight self-blindness (\textbf{-1 die} sight checks); \textbf{2 CP} whispering shadows create a faint tell; \textbf{3 CP} you borrow light from elsewhere, leaving a conspicuously bright patch; \textbf{4+ CP} a shadow-being takes interest.

\paragraph{Zealot's Blade (Thaumaturgy, DV 3).}
\emph{Effect:} For this scene, the weapon is \textbf{holy} and gains \textbf{+1 effect} vs undead/fiends; sheds dim light (\textbf{Obvious}).\\
\emph{Backlash:} \textbf{1 CP} flickering light (telegraphs position); \textbf{2 CP} \emph{Compromised}—shatters on an overcommit/crit; \textbf{3 CP} a beacon—nearby unnatural beings feel it; \textbf{4+ CP} caster is drained (\textbf{Fatigue 2}); the sanctity \emph{persists for this scene} with strict tenets.

\paragraph{Mend Flesh (Vitalism, DV 3).}
\emph{Effect:} Choose one: clear \textbf{2 Fatigue} \emph{or} \textbf{step down 1 Injury level} (max to Moderate). Requires stillness (no sprint/fight beat).\\
\emph{Backlash:} \textbf{1 CP} patient must rest soon or gain \textbf{Fatigue 1}; \textbf{2 CP} neglected minor injury festers; \textbf{3 CP} clumsy overgrowth (\textbf{-1 die} related physical until treated); \textbf{4+ CP} life drawn from surroundings (plants wither, small life dies).

% -------------------- GENERATION GUIDANCE --------------------
\subsection*{Spell Creation Guidance (table or prep)}
\paragraph{DV guidance.} DV \textbf{1} = subtle sense or tiny edge; DV \textbf{2} = scene-shaping buff/patch in a small area; DV \textbf{3} = potent, loud, or multi-target edge. Ace-level miracles should demand a price (Fatigue/Harm, costly component, or \emph{Obvious} in a hostile place).

\paragraph{Write the spell in 3 lines.}
\begin{enumerate}
  \item \textbf{Name \& Art:} pick an evocative verb+noun (\emph{Cinder-Fist}, \emph{Stone-Sense}).
  \item \textbf{Effect:} one clear board change: \emph{start Controlled}, \emph{+1 effect}, \emph{Hazard --1}, \emph{Hunt --1}, or grant a one-shot capability.
  \item \textbf{Backlash ladder:} 1–2 CP = tell/minor cost; 3–4 CP = new pressure (rail tick, condition, hazard clock); 5+ CP = scene-altering twist.
\end{enumerate}

\paragraph{Common outcome verbs (pick one).}
\emph{Start Controlled} • \emph{+1 effect} • \emph{Reduce (Rail) by --1} • \emph{Advance (Primary/side) by +1} • \emph{Grant 1-beat access} • \emph{Silence/Obscure one sense}.

\subsection*{Deck-Based Spell Seed Generator (optional)}
\noindent When improvising, draw \textbf{2–3 cards}. Use the suit to pick an \textbf{Art}, rank to set \textbf{DV} and a \textbf{scope tweak}. Face cards add a \textbf{quirk}; Aces add a \textbf{price}.

\begin{tabular}{@{}l l l@{}}
\toprule
\textbf{Suit} & \textbf{Art} & \textbf{Themes} \\
\midrule
\heartsuit & Vitalism / Hydromancy & life, vigor, calm, flow, restoration \\
\clubsuit  & Geomancy / Discipline  & stone, structure, weight, binding \\
\diamondsuit & Pyromancy / Transformation & heat, light, change, urgency \\
\spadesuit & Umbramancy / Veil & shadow, silence, misdirection, fear \\
\bottomrule
\end{tabular}

\medskip
\noindent \textbf{Rank $\to$ DV \& scope:}
\begin{itemize}
  \item \textbf{2–4:} DV 1 (self or tiny area; 1 target; 1 beat sense/edge).
  \item \textbf{5–9:} DV 2 (small zone \textasciitilde10 ft; team-sized buff; 1 rail tick).
  \item \textbf{10, J, Q, K:} DV 3 (loud, group-facing, or multi-rail influence).
  \item \textbf{Ace:} DV 3 + \emph{price} (component, \textbf{Fatigue 1}, \textbf{Harm 1}, or an \textbf{Obvious} flare).
\end{itemize}

\noindent \textbf{Face-card quirk (pick if drawn):}
\begin{itemize}
  \item \textbf{J}: brief mobility or reach boost (leap, glide, slip).
  \item \textbf{Q}: social resonance (hush, awe, dread).
  \item \textbf{K}: durability/impact bump (+1 effect on one hard task).
\end{itemize}

\paragraph{Backlash prompts by suit (to speed prep).}
\begin{itemize}
  \item \heartsuit~(Vitalism/Water): overgrowth, exhaustion, stagnation, sympathetic drain.
  \item \clubsuit~(Stone/Discipline): rigidity, slow, echoing calls, guardians notice.
  \item \diamondsuit~(Fire/Change): flare, scorch, noise, uncontrolled spread.
  \item \spadesuit~(Shadow/Veil): whispers, self-blindness, cold spots, entities take interest.
\end{itemize}

\subsubsection*{Generated Example (drawn: \diamondsuit~10, \spadesuit~J)}
\textbf{Name:} Ember-Stride (Pyromancy). \textbf{DV:} 3.\\
\textbf{Effect:} For \textbf{1 beat}, you dash through a chokepoint with \textbf{start Controlled} and \textbf{+1 effect} on clears; leaves a hot shimmer that \emph{Hazard --1} as pursuers hesitate. \textbf{Obvious}.\\
\textbf{Backlash:} \textbf{1–2 CP} singe/heat-haze reveals route; \textbf{3–4 CP} smolder opens a \emph{Fire(4)} clock; \textbf{5+ CP} flare alarms watchers (start \emph{Crowd +1} and \emph{Hunt +1}).

\subsection*{Authoring Checklist (fast)}
\begin{itemize}
  \item One clear \textbf{outcome} verb (position/effect/rail/clock), not three.
  \item DV matches ambition (1 subtle, 2 local scene-shaper, 3 loud/potent).
  \item Backlash ladders from \emph{tell} $\to$ \emph{pressure} $\to$ \emph{problem}.
  \item Mark \textbf{Obvious} if it would draw eyes or ears.
  \item Note if sustaining beyond defaults costs \textbf{Fatigue}.
\end{itemize}
% ===== End Section =====

% ===== SRD PATCH ADDENDUM: Micro-Clarifications =====
\subsection*{Addenda (GM-facing Micro-Clarifications)}
\addcontentsline{toc}{subsection}{Addenda (GM-facing Micro-Clarifications)}

% — ties into §9 Initiative Action
\paragraph{Follower Actions.}
One follower may \emph{Assist} normally (\,+Cap in-role or +1 off-role with an \emph{intricate} description\,) and a \emph{different} follower may take an \emph{Initiative Action} in the same scene. Both count against the crew’s \textbf{single Initiative window per scene}. (If a Talent opens a second window, you still cannot use the \emph{same} follower for both in one scene unless stated.)

% — extends §6 Ritual Casting
\paragraph{Ritual CP Absorption.}
Each helper may absorb \textbf{1 CP per ritual procedure} as \emph{Fatigue 1}: up to 1 from a \emph{Channel} they personally rolled, and up to 1 from the \emph{primary caster’s Weave}. Helpers cannot absorb CP that was not generated by their own roll or the primary’s Weave. This stacks with the helper cap and other mitigation limits.

% — refines §7 Deck of Consequences
\paragraph{Deck Contradictions.}
If a drawn consequence \emph{breaks established facts} (e.g., “ally betrays you” when that ally died last scene), the GM must either \emph{redraw} or narrate a \emph{related pressure in the same suit} (e.g., \heartsuit\ a confidant’s letter surfaces; \spadesuit\ a hired spy flips; \diamondsuit\ a debt is called; \clubsuit\ a warrant is issued). Do not mix direct CP spend and deck resolution on the \emph{same} roll.

% — extends §3 Asset Activations
\paragraph{Asset Stretch Uses.}
If an activation \emph{stretches plausibility} but does not break fiction (scope is right, reach is shaky), the GM may allow it with an additional \textbf{+1 CP} cost on the current roll \emph{or} require a \emph{Setup} test (Wits+Lore or Presence+Sway, DV 2). Success clears the stretch; failure denies or downgrades the activation.

% — extends §4 Over-Stack
\paragraph{Over-Stack Refresh.}
“New advantages” are \emph{additions not present at scene start}: assets activated, buffs gained, or tactical shifts that create a \emph{different} structural benefit. Re-check Over-Stack only when \textbf{two} such new advantages accrue. The \emph{same asset} cannot be counted twice unless it provides a \emph{distinct} advantage (e.g., first grants \emph{start Controlled}, later retuned to impose \emph{Hazard $-1$} on a different lane).

\end{document}
