\documentclass[11pt]{article}
\usepackage[margin=1in]{geometry}
\usepackage{graphicx}
\usepackage{booktabs}
\usepackage{longtable}
\usepackage{array}
\usepackage{fancyhdr}
\usepackage{titlesec}
\usepackage{hyperref}
\usepackage{enumitem}
\usepackage{multicol}

\pagestyle{fancy}
\fancyhf{}
\rhead{\thepage}
\lhead{Fate's Edge SRD}
\renewcommand{\headrulewidth}{0.4pt}

\titleformat{\section}{\large\bfseries}{\thesection}{1em}{}
\titleformat{\subsection}{\bfseries}{\thesubsection}{1em}{}

\title{Fate's Edge System Reference Document (SRD)}
\author{}
\date{}

\begin{document}

\maketitle
\tableofcontents
\newpage

\section{Core Principles}

\subsection{Identity of Fate's Edge}
Fate's Edge is a narrative-first tabletop roleplaying system where every action carries weight, every choice has consequence, and every spell risks backlash. Dice are not simply a measure of success or failure—they are instruments of fate, weaving opportunity with risk.

\subsection{A World of Consequences}

\subsubsection{Design Goals}
\begin{itemize}
    \item \textbf{Narrative Primacy}: Mechanics exist to serve the story.
    \item \textbf{Risk as Drama}: Every roll carries the potential for triumph and complication.
    \item \textbf{Meaningful Growth}: Advancement is more than improving statistics.
\end{itemize}

\subsubsection{The Central Question}
What are you willing to risk, and what are you willing to pay, to reshape the world around you?

\subsubsection{Tone of Play}
\begin{itemize}
    \item Cinematic, with pacing tied to narrative beats.
    \item Consequential, where even small choices ripple outward.
    \item Collaborative, empowering both GM and players.
\end{itemize}

\subsection{Key Concepts}

\subsubsection{Narrative Time}
Time is measured by story weight:
\begin{itemize}
    \item \textbf{A Moment} — A heartbeat, a glance, a single strike or word.
    \item \textbf{Some Time} — A few minutes, enough for a skirmish.
    \item \textbf{Significant Time} — Hours, long enough for travel or rituals.
    \item \textbf{Days} — Large-scale endeavors: marches, training, recovery.
\end{itemize}

\subsubsection{Complication Points}
Whenever a player rolls dice, each result of 1 generates a Complication Point (CP). These are narrative fuel. The GM spends them to introduce twists.

\subsubsection{Affinity}
Each culture provides an Affinity: a narrative edge or metaphysical bond. Affinities make certain Arts, skills, or actions more reliable.

\subsubsection{Prestige Abilities}
High-level talents unlocked by mastering cultural arts or philosophies. They are narrative milestones as much as mechanical ones.

\subsubsection{On-Screen vs. Off-Screen}
\begin{itemize}
    \item \textbf{On-Screen Resources}: Companions, hirelings, or allies who stand beside you in danger.
    \item \textbf{Off-Screen Resources}: Taverns, estates, titles, or networks of informants.
\end{itemize}

\section{Core Mechanic}

\subsection{The Art of Consequence}

\subsubsection{Procedure}
All significant actions follow a three-step process:
\begin{enumerate}
    \item \textbf{Approach}: The player describes both what their character wants and how they attempt it.
    \item \textbf{Execution}: Build a dice pool equal to Attribute + Skill and roll that many d10s. Each die of 6 or higher counts as a success. Each 1 rolled generates a Complication Point.
    \item \textbf{Outcome}: The GM interprets total successes against the difficulty of the task. Complication Points are then spent to weave narrative setbacks.
\end{enumerate}

\subsubsection{The Description Ladder}
\begin{itemize}
    \item \textbf{Basic Action}: Roll the pool as-is. All 1s remain as Complication Points.
    \item \textbf{Detailed Action}: A clear, descriptive flourish allows the player to re-roll one die showing 1.
    \item \textbf{Intricate Action}: A richly described, multi-sensory action allows the player to re-roll all dice showing 1, and add one positive narrative flourish to the scene if they succeed.
\end{itemize}

\paragraph{Rule — Re-rolling 1s and CP}
Re-rolling 1s does \emph{not} remove the Complication Point already generated by those dice. If any re-rolled dice show 1 again, they generate additional CP as normal.
\[
\text{Let } C_{0}=\text{initial 1s,\quad } C_{r}=\text{1s on re-rolls}\;\Rightarrow\; \text{Total CP}=C_{0}+C_{r}.
\]

\textit{Example:} You roll 7d10: \(\{9,8,5,4,3,\;1,\;1\}\Rightarrow C_{0}=2\).
You re-roll both 1s (Intricate): \(\{6,2\}\Rightarrow C_{r}=0\).
Final: successes = 3, \(\text{CP}=2\) (those initial CP remain).

\subsubsection{Complication Points}
Complication Points (CP) are the engine of drama. They are not simple penalties, but narrative levers. The GM spends CP to introduce setbacks appropriate to the context:
\begin{itemize}
    \item \textbf{Escalation} — drawing more enemies, raising the stakes.
    \item \textbf{Exhaustion} — draining time, resources, or positioning.
    \item \textbf{Exposure} — revealing hidden actions, alerting foes.
    \item \textbf{Collateral} — harm or danger spilling over onto allies, innocents, or surroundings.
\end{itemize}

\subsubsection{Design Intent}
This mechanic ensures that every roll changes the story. Success without risk is rare, and even failure opens new narrative avenues.

\subsubsection{GM Quick Reference: Adjudicating Skill Checks}

\paragraph{Difficulty Ladder (Set Before the Roll)}
\begin{center}
\begin{tabular}{cll}
\toprule
\textbf{DV} & \textbf{Name} & \textbf{When to Use} \\
\midrule
2 & Routine & Clear intent, modest stakes, controlled environment. \\
3 & Pressured & Time pressure, mild resistance, partial info. \\
4 & Hard & Hostile conditions, active opposition, precise timing. \\
5+ & Extreme & Multiple constraints, high precision, dramatic failure. \\
\bottomrule
\end{tabular}
\end{center}

\begin{table}[htbp]
\centering
\caption{Boon Generation by Dice Pool and Difficulty}
\begin{tabular}{|c|c|c|c|c|c|}
\hline
\textbf{Pool Size} & \textbf{DV 2} & \textbf{DV 3} & \textbf{DV 4} & \textbf{DV 5} & \textbf{Avg Boons/Roll} \\
\hline
2d10 & 22\% CS, 35\% Miss & 33\% Miss & 42\% Miss & 49\% Miss & 0.64-0.91 \\
3d10 & 12\% CS, 45\% Miss & 45\% Miss & 51\% Miss & 55\% Miss & 0.78-0.97 \\
4d10 & 6\% CS, 53\% Miss & 53\% Miss & 57\% Miss & 58.5\% Miss & 0.87-0.995 \\
\hline
\end{tabular}
\end{table}

\paragraph{Outcome Matrix (After the Roll)}
Let $S$ be successes (≥ 6) and $C$ be Complication Points (number of 1s rolled).
\begin{center}
\begin{tabular}{lll}
\toprule
\textbf{Case} & \textbf{Name} & \textbf{Guidance} \\
\midrule
$S \geq DV$ and $C = 0$ & Clean Success & Deliver the intent crisply. \\
$S \geq DV$ and $C > 0$ & Success \& Cost & Grant the intent; spend/bank CP for complications. \\
$0 < S < DV$ & Partial & Progress with a fork. \\
$S = 0$ & Miss & No progress. Cash/bank CP. \\
\bottomrule
\end{tabular}
\end{center}

\paragraph{Complication Point (CP) Spend Menu}
\begin{itemize}
    \item \textbf{1 CP}: Minor pressure: noise, trace, +1 Supply segment.
    \item \textbf{2 CP}: Moderate setback: alarm raised, lose position/cover, lesser foe or lock.
    \item \textbf{3 CP}: Serious trouble: reinforcements, key gear breaks, rail tick.
    \item \textbf{4+ CP}: Major turn: trap springs, authority arrives, scene shifts.
\end{itemize}

\paragraph{Assistance, Boons, \& Description}
\begin{itemize}
    \item \textbf{Assists}: One helper per action; up to +3 dice.
    \item \textbf{Boons}: A player may re-roll one die after seeing the pool. Once per session, in downtime, you may convert 2 Boons → 1 XP (max 2 XP via conversion per session).
    \item \textbf{Description Ladder}: Basic (roll as-is), Detailed (re-roll one 1), Intricate (re-roll all 1s and add one flourish if successful).
\end{itemize}

\subsection{Time Guidance Framework}

\subsubsection{Narrative Time Scales}
Time in Fate's Edge is measured by story weight, not by clocks:
\begin{itemize}
    \item \textbf{A Moment} — A heartbeat, a glance, a single strike or word.
    \item \textbf{Some Time} — A few minutes, enough for a skirmish, a careful lockpick, or a short negotiation.
    \item \textbf{Significant Time} — Hours, long enough to travel between locations, work a ritual, or endure a siege.
    \item \textbf{Days} — Large-scale endeavors: marches across a countryside, training a cadre, or recovering from wounds.
\end{itemize}

\subsubsection{Game Structure Definitions}

\paragraph{Scene}
The basic unit of narrative gameplay, typically covering "Some Time" to "Significant Time". Contains multiple player turns, resolves a specific narrative question or conflict.

\paragraph{Player Turn (Beat)}
Individual player agency within the scene flow. Player declares action → GM sets position → Player rolls → Resolve outcome → Manage consequences.

\paragraph{Round}
Simultaneous or near-simultaneous actions within a scene (primarily for combat), representing a few seconds of real time.

\paragraph{Session}
One complete game session (typically 3-6 hours of real time), containing 2-4 major scenes and resolving significant narrative progress.

\paragraph{Campaign}
Entire story arc (multiple sessions, typically 6-20+ sessions) with major character development and lasting consequences.

\subsubsection{Magic and Ritual Time}

\paragraph{Standard Spell Casting}
Channel and Weave phases each take 1 Player Turn, resolving within a single scene.

\paragraph{Ritual Casting (Optional Rule)}
Channel and Weave phases each require 1 Scene (Significant Time).

\paragraph{Rites Invocation}
Invoke takes 1 Player Turn; Weave takes 1 Player Turn. High-Power rites may require Extended time.

\paragraph{Extended Rituals}
When a ritual \emph{must} take a long time, attach it to a clock:
\begin{itemize}
    \item 4-segment clock: Significant Time (hours)
    \item 6-segment clock: Extended Time (days)
    \item 8+ segment clock: Campaign Time (weeks/months)
\end{itemize}
Advance the clock through player actions, scenes, or set time intervals.

\subsection*{Fail Forward: Every Roll Matters}\index{Fail Forward}\index{Boons}

When you \textbf{MISS} on a \emph{significant action}, you gain \textbf{1 Boon}. Boons can be spent immediately for re-rolls, Asset activations, Rites, and other abilities. You can hold up to \textbf{5} Boons.

\paragraph{Significant Action (Meaningful Failure)}
A miss only awards a Boon if \textbf{all three} are true:
\begin{enumerate}
  \item \textbf{Procedure followed:} intent and approach declared; DV set; roll resolved.
  \item \textbf{Stakes stated:} what changes on success; what bites on failure.
  \item \textbf{Consequence lands now:} the GM spends or banks CP, applies a condition, or advances a thread.
\end{enumerate}
\noindent Rolling a \textbf{1} always creates CP for the GM. Re-rolling \textbf{1}s never removes CP already generated.

\paragraph{No Boon For}
Rehearsal or null-risk probes, and repeated identical attempts in the same scene without a new approach, position, or stakes.

\paragraph{Other Ways to Gain Boons}
Strong bond-driven play and scene prompts can also award Boons at the GM's discretion. Boons remain capped by the limits below.

\paragraph{Boon Carryover (Scene-Based)}\index{Boon Carryover}
\textbf{Hold Cap:} You can hold up to \textbf{5} Boons.

\textbf{Carryover Limit:} At the \emph{end of each scene}, reduce your held Boons to a maximum of \textbf{2}. Excess Boons are lost.

\textbf{Spend As You Earn:} You may spend Boons at any time during the scene (re-rolls, Asset activations, Rites, abilities, etc.).

\textbf{Multi-Phase Set Pieces:} If the GM declares a multi-phase scene (e.g., chase $\rightarrow$ duel), trim to 2 only when the entire set piece ends.

\paragraph{Rite \& Asset Notes}
High-Power Rites that require \textbf{2 Boons} remain viable—you can start a scene at 2 and must earn more in-scene to chain further Invokes. On-screen Asset activations still cost 1 Boon as normal.

\paragraph{Anti-Fishing Dials}\index{Anti-Fishing}
These optional limits help keep flow healthy:
\begin{itemize}
  \item \textbf{Once/Scene (Failures):} At most \textbf{2 Boons from failures per character per scene}. Further misses still generate CP but no Boon.
  \item \textbf{Repetition Rule:} Same approach \emph{and} same stakes in the same scene cannot award another Boon.
  \item \textbf{Position Gate:} \textbf{Controlled} tests with trivial fallout do not award Boons; they're for information or positioning, not currency.
\end{itemize}

\paragraph{Optional: Partial-with-Cost Safety Valve}
By default, \textbf{Partial (Success \& Cost)} does \emph{not} grant a Boon. If your table wants more flow, you may award \textbf{1 Boon} on a Partial when the GM spends \textbf{3+ CP} on that outcome (\emph{max once per scene per character}). \emph{Tip:} Use this only when the cost meaningfully changes the situation.

\paragraph{Examples}
\begin{itemize}
  \item \textbf{Boon awarded:} Picking a lock under watch (\emph{Risky}, DV 3). Stakes set: success opens; miss triggers the alarm. The roll \textbf{MISS}es; the GM spends 2 CP to start ``Guards Incoming.'' The player gains \textbf{1 Boon}.
  \item \textbf{No Boon:} Tapping flagstones ``just in case'' (Controlled, no stated stakes). Info only; no CP spent/banked. No Boon.
  \item \textbf{Carryover:} End of scene, a character holds 4 Boons. They trim to \textbf{2} for the next scene. During the next scene, they earn and spend Boons freely, never exceeding the \textbf{5} hold cap in-scene; trim back to 2 when that scene ends.
\end{itemize}

\paragraph{Bond-Driven Resource Generation}
When a player takes a significant action to aid an ally with whom they share a bond, and explicitly references that bond in an intricate description, they may mark that bond to gain 1 boon after the action resolves.

\textbf{Requirements:}
\begin{itemize}
    \item \textbf{Mutual Bond:} Player shares a bond with the ally they're aiding
    \item \textbf{Intricate Description:} Player describes how their bond motivates their action
    \item \textbf{Significant Aid:} Meaningful assistance (not just +1 die)
    \item \textbf{Fiction First:} The bond must genuinely drive the choice to help
\end{itemize}

\textbf{Limitations:}
\begin{itemize}
    \item Once per bond per session
    \item Must be a meaningful sacrifice or risk
    \item GM approval for "significant action"
    \item Cannot be used for basic assistance rolls
\end{itemize}

\paragraph{Setting Stakes Fast (Cheat Prompts)}
\begin{itemize}
    \item If this goes right, what changes?
    \item If this goes wrong, what bites back?
\end{itemize}

\paragraph{Banking \& Cashing CP}
\begin{itemize}
    \item Banked CP should pay off within the same scene or arc.
    \item Avoid nickel-and-diming. Prefer one memorable complication over many petty penalties.
\end{itemize}

\subsection{Worked Micro-Examples}
\begin{itemize}
    \item \textbf{Lockpick Under Watch (DV 2)}: Player rolls 6 dice: 10, 8, 5, 4, 1, 1 ⇒ S=2, C=2. Success \& Cost. Door opens; GM spends 1 CP for a squeal (patrol starts moving) and banks 1 CP to bring that patrol around on the next beat.
    \item \textbf{Charm the Captain (DV 2)}: Player rolls 5 dice: 7, 6, 6, 2, 1 ⇒ S=3, C=1. Success \& Cost. Passage granted; GM spends 1 CP: "He expects a favor on the return leg—he'll collect."
    \item \textbf{Traverse the Pass (DV 3)}: Group roll pools to a net 3 successes but produces C=3. Success \& Cost. GM spends 2 CP to add Fatigue 1 to all from cold and exposure, banks 1 CP to crack a wagon axle next scene.
\end{itemize}

\section{Integrated Combat System}

\subsection{Core Philosophy}
Combat is violent conflict resolved through the standard consequence mechanics. Every combat action generates potential for both triumph and complication, with consequences that cascade through the same economy as all other challenges.

\subsection{Resolution Procedure}
\begin{enumerate}
    \item \textbf{Declare Action}: Player states intent and approach (Attribute + Skill)
    \item \textbf{Set Stance}: GM sets Controlled, Risky, or Desperate based on tactical situation
    \item \textbf{Roll Dice}: Roll pool = Attribute + Skill (takes 1 Player Turn)
    \item \textbf{Count Results}: 6+ = Success, 1 = Complication Point (CP)
    \item \textbf{Apply Outcome}: Use standard Outcome Matrix
    \item \textbf{Manage Consequences}: GM spends CP or draws from Consequences Deck
\end{enumerate}

\section{Range \& Position}

\subsection*{Bands + Melee Flag}
\begin{description}
  \item[\textbf{Near}] Same room/street segment/skirmish space.
  \item[\textbf{Far}] Same site/district but not in immediate reach.
  \item[\textbf{Absent}] Off-screen / away; requires a cut or travel.
\end{description}
\textbf{Melee (flag):} At any moment, the GM may mark two parties \emph{in Melee} if they are in \textbf{Near} and directly engaged. This is a flag, not a separate band.

\paragraph{Enter/leave Melee.}
\begin{itemize}
  \item \textbf{Enter:} 1 Move to engage from \textbf{Near}. If under fire, treat as \textbf{Risky}.
  \item \textbf{Leave:} 1 Move to break off; under threat, test to Disengage at \textbf{Risky}. On a Partial/Miss, suffer a soft consequence (shove/nick/stumble) and remain in Melee.
\end{itemize}

\subsection*{Position (Controlled / Risky / Desperate)}
Position sets \emph{consequence severity} on a Partial/Miss; it \emph{does not} change DV.

\subsection*{Position Dynamics (Hybrid)}
\begin{itemize}
  \item \textbf{GM Spend (1 CP):} Shift Position \emph{one step worse} for the current action \emph{or} apply a brief environmental shove (smoke, crowd surge, slippery ledge) that justifies the shift.
  \item \textbf{Player Spend (1 Boon):} Shift Position \emph{one step better} for your current action \emph{or} cancel a \emph{single} 1-step GM shift just declared.
  \item \textbf{Narrative Triggers (free):} Flanking, reinforcements, collapsing cover, superior leverage, or proof can move Position one step (GM call). Say why.
\end{itemize}
\emph{Limits:} One Boon shift per action. Multiple forces can't stack beyond one step each way—resolve in order declared.

\subsection*{Targeting \& Tools (quick defaults)}
\begin{itemize}
  \item \textbf{Melee/Touch:} requires \textbf{Melee} flag (or long haft from \textbf{Near} at \textbf{Limited Effect}).
  \item \textbf{Thrown/short magic:} \textbf{Near}; at \textbf{Far} expect Limited Effect or +1 DV unless the source says otherwise.
  \item \textbf{Firearms (not in the core setting) /line magic:} \textbf{Near} and \textbf{Far} if stated; off-band \Rightarrow lower Effect or +1 DV.
\end{itemize}

\subsection*{Interactions (alignment with SRD)}
\begin{itemize}
  \item \texttt{[WARD]} \& \texttt{[BANISH]} use unified Outsider DV/impact; Position still colors consequences if the attempt goes sideways.
  \item Summoned spirits take orders best at \textbf{Near}; rushing a spirit from \textbf{Far} into Melee under pressure is typically \textbf{Risky}.
\end{itemize}

\subsection*{GM Quick Cues}
\begin{itemize}
  \item Start \textbf{Risky/Standard}. Move one step at a time as fiction or spends justify.
  \item When distance is fuzzy, ask: "Do you need a beat to get there?" If yes, spend a Move; if no, you're \textbf{Near}.
  \item Call \textbf{Melee} when knives are actually in play—don't track hexes.
\end{itemize}

\subsection{Combat-Specific Consequence Types}
\begin{itemize}
    \item \textbf{Hearts}: Morale, fear, command/control breakdown
    \item \textbf{Spades}: Physical harm, positioning changes, weapon status
    \item \textbf{Clubs}: Resource depletion, gear damage, fatigue
    \item \textbf{Diamonds}: Environmental hazards, reinforcements, tactical setbacks
\end{itemize}

\subsection{Harm Integration}
Harm tracks directly tie to CP economy:
\begin{itemize}
    \item \textbf{Minor (-)}: Generate 1 CP on next 2 rolls
    \item \textbf{Moderate (=)}: Generate 1 CP on next roll, -1 die to relevant actions
    \item \textbf{Severe (≈)}: Generate 2 CP on next roll, -2 dice to relevant actions  
    \item \textbf{Critical (†)}: Generate 3 CP on next roll, out of action until treated
\end{itemize}

\subsection{Tactical Clocks}
Persistent combat conditions tracked through clocks:
\begin{itemize}
    \item \textbf{Mob Overwhelm} (6): Enemy numbers become advantage
    \item \textbf{Fatigue Spiral} (4): Exhaustion affects performance
    \item \textbf{Morale Collapse} (6): Fear undermines effectiveness
    \item \textbf{Environmental Collapse} (8): Terrain/fire/building failure
\end{itemize}

\subsection{Position Dynamics}
Position can shift during combat based on CP spending:
\begin{itemize}
    \item \textbf{1 CP}: Shift position one step (GM choice)
    \item \textbf{Player Spending}: 1 CP to improve position one step
    \item \textbf{Narrative Triggers}: Flanking, reinforcement arrival, environmental changes
\end{itemize}

\section{Range Bands} 

\textit{Keep distance simple. Use bands to answer: can you reach them, affect them, or see them right now?}

\subsection*{Bands}
\begin{description}
  \item[\textbf{Close}] Arm's length, grapples, knives. You can touch, shove, or clinch.
  \item[\textbf{Near}] Same room/street segment/skirmish space; a quick step or two away.
  \item[\textbf{Far}] Same site/area but not in immediate reach; you need time, route, or a long implement.
  \item[\textbf{Absent}] Off-screen / away; outside the current scene frame (requires a cut or travel to interact).
\end{description}
\emph{For brevity's sake, assume \textbf{Near} unless otherwise specified.}

\subsection*{Movement (beats, not meters)}
\begin{itemize}
  \item \textbf{1 Move} shifts \textbf{one band}: Close$\leftrightarrow$Near \emph{or} Near$\leftrightarrow$Far.
  \item \textbf{Dash (your action)} shifts \textbf{two bands}: Close$\rightarrow$Far or Far$\rightarrow$Close in one go.
  \item \textbf{Terrain/obstacles} may add \textbf{+1 Move} (crowds, rubble, locks) or demand a test first.
  \item \textbf{Break contact:} Leaving \textbf{Close} cleanly can be a \textbf{Disengage} test; on success you step to \textbf{Near} without a complication.
  \item \textbf{Absent $\leftrightarrow$ Far/Near:} use a \textbf{Travel clock [2–4]} or a quick cut; under pressure, treat as \textbf{Risky}.
\end{itemize}

\subsection*{Targeting \& Tools}
\begin{itemize}
  \item \textbf{Melee/Touch}: \textbf{Close} only (or \textbf{Near} on long polearms).
  \item \textbf{Thrown / Short bows / Small spells}: \textbf{Near}. At \textbf{Far}, expect \textbf{Limited Effect} or +1 DV unless the source says otherwise.
  \item \textbf{Firearms / Long bows / Line spells}: \textbf{Near} and \textbf{Far} if stated; at off-band, decrease \textbf{Effect} or increase \textbf{DV} by +1.
  \item \textbf{Social}: \textbf{Near} by default; at \textbf{Far} you need voice, signal, or proxy; at \textbf{Absent} you need message, oath, or setup.
  \item \textbf{No line of sight}: either raise \textbf{DV} by +1, drop \textbf{Effect} a step, or require \texttt{[REVEAL]}/\texttt{[MARK]} to proceed.
\end{itemize}

\subsection*{Perception, Stealth, Cover}
\begin{itemize}
  \item \textbf{Notice at Near}: normal. \textbf{Far}: harder—raise \textbf{DV} by +1 or impose Limited \textbf{Effect}, especially in noise/cover.
  \item \textbf{Hide}: break line of sight or move to \textbf{Far}; at \textbf{Absent} you're off the current frame unless someone tracks you.
  \item \textbf{Cover}: improves \textbf{Position} or reduces consequence severity; quality/angle sets the bump.
\end{itemize}

\subsection*{Summoning \& Bands (alignment)}
\begin{itemize}
  \item \textbf{Commanding a spirit:} assume \textbf{Near} to issue precise orders. Barked orders across \textbf{Far} are possible but often \textbf{Limited Effect} or +1 \textbf{DV}.
  \item \textbf{Leash ticks (from Summoning rules):} if a called spirit \textbf{moves from Close to Far from you} in a hurry, that fiction can justify a tick (GM call). Crossing a \texttt{[WARD]} uses the unified Ward rule.
\end{itemize}

\subsection*{Tags \& Ranges (defaults)}
\begin{itemize}
  \item \texttt{[WARD]}: placed on a line/edge you can touch; crossing is resolved at that edge.
  \item \texttt{[BANISH]}: target a visible Outsider within the source's range (usually \textbf{Near}; some rites allow \textbf{Far}).
  \item \texttt{[VEIL]/[REVEAL]/[MARK]}: usually \textbf{Near}; specify \textbf{Far} reach in the source if intended.
  \item \texttt{[TRANSPORT]/[PASSAGE]}: bridge \textbf{Far} or pull \textbf{Absent} into the scene if the text allows.
\end{itemize}

\subsection*{GM Quick Cues}
\begin{itemize}
  \item Start at \textbf{Risky/Standard} in \textbf{Near}. Shift \textbf{Position} for rushes or retreats; shift \textbf{Effect} for reach/quality.
  \item If torn on distance minutiae, ask one question: \emph{"Do you need a beat to get there?"} If yes, spend a \textbf{Move}; if no, you're \textbf{Near}.
  \item Keep bands elastic: one cramped factory floor can be all \textbf{Near}; a fortress courtyard may make each wall its own \textbf{Far} hop.
\end{itemize}

\subsection{Design Note:} Positioning, Movement, and Range are all \textbf{Theater of the Mind} and should flow in a naturalistic way. The rules are to add a framework to this. You can also adapt these rules for play with minatures on a grid or hex map. 

\subsection{Magic Combat Integration}
Spellcasting in combat feeds the same consequence economy:
\begin{itemize}
    \item Channel/Weave Backlash CP applies to tactical situation
    \item Spells can shift position, create tactical clocks, or generate combat consequences
    \item Magic consequences cascade through existing combat systems
\end{itemize}

\subsection{Asset/Follower Combat Integration}
\begin{itemize}
    \item \textbf{Follower Risk}: 2+ CP spent in combat can endanger assisting followers
    \item \textbf{Asset Compromise}: Combat in certain locations can damage relevant assets  
    \item \textbf{Offensive Activation}: 1 Boon activates asset for combat advantage
    \item \textbf{Initiative Actions}: Followers can take combat-relevant independent actions
\end{itemize}

\subsection{Outcome Matrix Application}
Same as standard resolution, but consequences are combat-specific:
\begin{itemize}
    \item \textbf{Clean Success}: Intent achieved with no tactical complications
    \item \textbf{Success \& Cost}: Intent achieved, but GM spends CP for combat consequences
    \item \textbf{Partial}: Progress with tactical fork (accept cost OR concede ground)
    \item \textbf{Miss}: No progress; GM spends CP for combat consequences OR offers tactical bargain
\end{itemize}

\section{Advancement \& XP}

\subsection{Awarding XP}
\begin{itemize}
    \item \textbf{Gritty}: 4–6 XP per session (slow burn).
    \item \textbf{Standard}: 6–10 XP per session (default pace).
    \item \textbf{Heroic}: 10–14 XP per session (fast growth).
\end{itemize}

\subsubsection{Session Awards}
\begin{itemize}
    \item Table Attendance: +2 XP
    \item Major Objective Reached: +2–4 XP
    \item Discovery or Lore Unlocked: +1–2 XP
    \item Hard Choice Embraced: +1–2 XP
    \item Complication Spotlight: +1–3 XP
    \item Bond/Flag Driven Play: +1–2 XP
    \item GM Curveball Award: +0–3 XP
\end{itemize}

\subsubsection{Milestones}
\begin{itemize}
    \item +8–12 XP to all players at the conclusion of a major story arc.
    \item +2 XP bonus to one player for a signature moment of the arc.
\end{itemize}

\subsubsection{Complication Dividend}
\begin{itemize}
    \item Face Card (J/Q/K): +1 XP
    \item Ace: +2 XP
\end{itemize}

\subsection{Spending XP}
\begin{itemize}
    \item \textbf{Attributes}: Cost = new rating × 3. Downtime = new rating in days.
    \item \textbf{Skills}: Cost = new level × 2. Downtime = new level in days.
    \item \textbf{On-Screen Followers}: Cost = Cap². Downtime = 1–3 days to recruit and brief.
    \item \textbf{Off-Screen Assets}: Minor (4 XP, 1 day), Standard (8 XP, 1 week), Major (12 XP, 1 month).
\end{itemize}

\subsection{Rush Rule}
A player may skip downtime, but the GM creates a Haste clock of four segments. If the clock fills, the new ability or asset carries flaws or narrative complications.

\subsection{Tiers of Reputation}
\begin{itemize}
    \item \textbf{Tier I – Rookie} (0–40 XP): Local reputation; prestige locked.
    \item \textbf{Tier II – Seasoned} (41–90): Regional notice; prestige abilities may be unlocked.
    \item \textbf{Tier III – Veteran} (91–150): National influence; second follower slot suggested.
    \item \textbf{Tier IV – Paragon} (151–220): Movers and shakers; rivals emerge to challenge.
    \item \textbf{Tier V – Mythic} (221+): Legendary status; kingdoms and cults respond.
\end{itemize}

\section{Character Creation}
\subsection{Starting Build Points}
Players begin with 30 Experience Points (XP) to allocate during initial character creation. This represents a balanced baseline for competent starting characters.

\subsection{Enhanced Starting Builds}
Players may exceed the standard 30 XP build through narrative engagement:

\begin{itemize}
    \item \textbf{Bonds:} Up to two player-defined mutual bonds may be taken for +2 XP total (Section \ref{sessionawards})
    \item \textbf{Complications:} Up to two initial complications may be accepted for +4 XP total (Section \ref{sessionawards}) \textit{NOTE: Scenes start with +CP per complication per character until cleared.}
\end{itemize}

This allows for a maximum starting build of 34 XP, though players are encouraged to aim for 30 XP and use bonds/complications to mitigate slight overages while maintaining narrative balance.

\subsection{Recommended Approach}
The GM should encourage players to:
\begin{itemize}
    \item Target 30 XP for balanced starting characters
    \item Use bonds and complications to enhance characterization rather than pure mechanical optimization
    \item Consider the narrative implications of any starting advantages
\end{itemize}

\subsection*{Initial Complications}
\textbf{Per Complication:}
\begin{itemize}
    \item Start each scene  with +1 banked CP per character with  initial complications until those complications have cleared.
\end{itemize}

\section{Complication Point Management}

The GM should manage Complication Point (CP) spending to maintain dramatic tension while preserving player agency and game flow. CP spending scales with character tier but is subject to hard limits to ensure playability.

\subsection{Core Principles}
\begin{itemize}
    \item \textbf{Narrative Coherence}: All CP spends within a scene should connect thematically
    \item \textbf{Player Agency}: Complications create interesting choices, not insurmountable obstacles
    \item \textbf{Progressive Escalation}: Higher tier characters naturally attract greater consequences
    \item \textbf{Resolution Paths}: Every complication thread should have potential resolution
\end{itemize}

\subsection{Spending Formula}
\textbf{Base CP = 4 + Character Tier}
\begin{itemize}
    \item \textbf{Tier I (Rookie)}: 5 CP base
    \item \textbf{Tier II (Seasoned)}: 6 CP base
    \item \textbf{Tier III (Veteran)}: 7 CP base
    \item \textbf{Tier IV (Paragon)}: 8 CP base
    \item \textbf{Tier V (Mythic)}: 9 CP base
\end{itemize}

\subsection{Hard Limits}
\begin{itemize}
    \item \textbf{Standard Scenes}: Maximum 12 CP spending
    \item \textbf{Climactic Scenes}: Maximum 16 CP spending
    \item \textbf{Active Threads}: Maximum (Tier + 1) concurrent threads
    \item \textbf{Session Budget}: Maximum 20 CP total per session
\end{itemize}

\subsection{Banked CP Integration}
Banked CP from character complications count toward scene spending limits rather than adding to available CP. This prevents exponential complication stacking while honoring narrative debt.

\subsection{Thread Management}
Complication threads follow a natural escalation pattern:
\begin{itemize}
    \item \textbf{First Exposure}: 1-2 CP (Minor inconvenience)
    \item \textbf{Second Occurrence}: 2-4 CP (Moderate setback)
    \item \textbf{Third Strike}: 3-6 CP (Major consequence)
    \item \textbf{Resolution}: Thread concludes with narrative payoff
\end{itemize}

\section{Rules Clarifications}

\subsection{Follower Assist}
\begin{itemize}
    \item Assist dice come from the helper, not the leader.
    \item Total Assist on any roll (from any sources) remains hard-capped at +3. Exception: The "Exceptional Coordination" Talent allows one follower to provide +4 assist dice.
\end{itemize}

\subsection{Boon Economy}
\begin{itemize}
    \item You earn boons from failing die rolls or leveraging bonds with other player characters.
    \item Holding cap: You can hold at most 5 Boons.
    \item Conversion: Once per session, in downtime, you may convert 2 Boons → 1 XP (max 2 XP via conversion per session).
\end{itemize}

\paragraph{Bond-Driven Generation}
Players may earn boons through bond-driven resource generation (see Section 2.1.5) by taking significant actions to aid bonded allies with intricate descriptions of their bond-motivated actions.

\subsection{Asset Activations}
\begin{itemize}
    \item \textbf{Off-Screen Activation}: A player may activate an off-screen asset at the very start of a campaign or during Downtime. It costs 1 Boon or 2 XP to activate.
    \item \textbf{Off-Screen Effects}: Use each Asset's listed Off-Screen effect once per session for free.
    \item \textbf{On-Screen Activations}: To reshape the current scene, spend 1 Boon.
    \item \textbf{Plausibility Test}: The Asset must have scope and reach.
\end{itemize}

\subsection{Over-Stack}
\begin{itemize}
    \item Structural advantages: active buff/tag, favorable venue/pennant, Follower Initiative unused, on-screen Asset activation, opponent disadvantaged by fiction, ritual prep that applies now.
    \item Trigger: If the crew enters a scene with ≥ 3 structural advantages, apply Over-Stack once for that scene: either start one named rail at +1 or the GM banks +1 CP for the first Deck Twist.
\end{itemize}

\subsection{Familiar Bond}
\begin{itemize}
    \item Familiars use the standard Follower Exposure/Harm tracks and require no upkeep.
    \item Each time a familiar acts on-screen in a high-risk beat, mark Exposure +1 on the familiar after the second such beat this scene.
\end{itemize}

\subsection{Ritual Casting}
\begin{itemize}
    \item Helper cap: Maximum simultaneous helpers = $\lceil$primary caster's Ritual/Arcana/2$\rceil$, max 3.
    \item Relevant skills: Helpers may use different relevant skills if their procedure is fictionally distinct.
    \item CP distribution: CP from Channel resolves on that roller. CP from Weave is assigned to the primary caster.
\end{itemize}

\section{Inspiration}

\paragraph{Cost.}
\textbf{Inspire (Bonded)} costs \textbf{3 XP}. Uses between downtime scale by Tier (Tier I/II/III = 2/3/4).

\emph{Optional (no Tiers):} Use a 3-rank track instead — \textbf{Inspire I/II/III} at \textbf{2 XP} per rank; each rank grants \textbf{2/3/4} uses between downtime.

\textbf{Prerequisite:} a declared \textbf{Bonded PC ally} (bond, vow, or crew tie) \\
\textbf{Use:} 1 action, audible/visible cue, \textbf{Near} range \\
\textbf{Uses between downtime:} Tier I / II / III = \textbf{2 / 3 / 4} (or a 3-rank talent track at 2/3/4). Reset on downtime.

\paragraph{Effect (choose your Bonded PC ally in Near who can see/hear you).}
Apply all:
\begin{itemize}
  \item \textbf{Bonded ally:} +1 \textbf{Boon} (PC only; normal Boon limits apply) \emph{and} +1 die on their next roll this scene.
  \item \textbf{You:} +1 die on your next roll this scene.
  \item \textbf{Each other PC in Near (who perceives you):} +1 die on their next roll this scene.
\end{itemize}

\paragraph{Clarifiers.}
\begin{itemize}
  \item Followers cannot receive Boons and do not benefit from Inspire.
  \item A PC can benefit from Inspire \textbf{once per scene}; later Inspires \textbf{replace} any unspent +1 (no stacking).
  \item All +1 dice count toward the normal \textbf{+3 dice cap} (with help, assets, etc.).
  \item Boons follow the game's \textbf{normal limits} (hold/carry). There is no special expiration.
\end{itemize}

\bigskip

% =========================================
% Fate's Edge SRD: Metaphysics & Casting Color
% =========================================

\section{Metaphysics: The Eight Elements}
\label{sec:metaphysics}

\textit{All magic leans on eight Elements. Four are material (Earth, Fire, Air, Water). Four are their metaphysical counterparts (Fate, Life, Luck, Death/Dreams). Every Element casts a shadow—its \emph{opposite}. Backlash tends to express along the invoked Element or its opposite.}

\subsection*{Oppositions \& Realms (table)}
\begin{tabular}{@{}lllll@{}}
\textbf{Element} & \textbf{Domain} & \textbf{Realm (evoked)} & \textbf{Opposes} & \textbf{Backlash tends to…} \\
\midrule
Earth & stone, weight, binding & Underways, vaults, bedrock & Air & pin, slow, bury, collapse \\
Air & breath, sound, motion & High-places, drafts, echoes & Earth & scatter, drop, vertigo, broadcast \\
Fire & heat, light, fervor & Forges, lamps, emberfields & Water & sear, consume, flare attention \\
Water & flow, memory, tides & Rivers, cisterns, rain & Fire & flood, slip, drown, leach \\
Fate & causality, oaths, \emph{anti-magic} & Threads, wheels, signs & Luck & lock outcomes, seal options \\
Life & growth, vigor, repair & Greenways, hives, bloom & Death/Dreams & overgrow, fever, untidy healing \\
Luck (Fortune) & chance, breaks, windfalls & Crossroads, coinfalls & Fate & side-coincidence, misfire elsewhere \\
Death/Dreams (Obishaal) & thresholds, sleep, endings \& \emph{Ways Between} & Doors, veils, graveways & Life & fade, sleep, thin walls, haunted echoes \\
\bottomrule
\end{tabular}

\paragraph{Realms.} Each Element "has a place" in the world—sites, signs, and materials that \emph{welcome} it. Casting aligned to a Realm reads easier; casting in an opposing Realm strains and courts backlash. Obishaal (Death/Dreams/Thresholds) is also how long travel works: step into the \textbf{Ways Between}; cross by doors, sleep, or rite; step out where the thresholds rhyme.

\subsection*{How Elements shape play (guidance, not hard rules)}
\begin{itemize}
  \item \textbf{Declare an Elemental axis} when you \textbf{Weave} a spell (\S\ref{sec:arcane-freeform}): name 1 (or at most 2) Elements that color the effect.
  \item \textbf{GM DV nudges (fiction-first):} aligned locus/tools \(\Rightarrow\) consider \(-1\) DV \emph{or} +1 Effect; hostile locus/opposition \(\Rightarrow\) consider \(+1\) DV \emph{or} drop Effect a step.
  \item \textbf{Backlash color:} on Partial/Miss (or two 1s on a Hit), express consequences via the chosen Element or its opposite (see table above).
  \item \textbf{Fate as anti-magic:} Fate-heavy effects resist other magic; the GM may treat sustained foreign magic in a Fate locus as \emph{Limited Effect} unless honored by oaths/price.
\end{itemize}

\paragraph{Ways Between (Obishaal) — travel note.}
To traverse great distance, cast \emph{PASSAGE/TRANSPORT} flavored by Death/Dreams (doors, sleep, stories). DV by fiction (shared threshold, name, token). Backlash often leaves \emph{thin places} or hitchhiking dreams.

\bigskip

% =========================================
% Arcane Access: Your Personal Art
% =========================================

\section{Art (Casting Method)}
\label{sec:art}

\textit{Your \emph{Art} is the narrative method by which your \textbf{Weave} and \textbf{Cast} work (sigils, sung names, lantern-law, bone charms, contracts, salt-thread, etc.).}

\subsection*{Declaring an Art}
When you gain \textbf{Arcane Initiate} (\S\ref{sec:arcane-freeform}), write a one–two line \textbf{Art} describing:
\begin{itemize}
  \item \textbf{Gesture \& medium} (ink, chord, breath, light, bone, law).
  \item \textbf{Typical Elements} you lean on (pick 2 you're often aligned to).
\end{itemize}

\subsection*{Art in play (small, balanced hook)}
\begin{itemize}
  \item \textbf{Spotlight bump (1/scene):} if your Art is \emph{clearly honored in fiction} (right tools/time/setting), gain \textbf{+1 die} on your \textbf{Cast}. This counts toward the +3 dice cap.
  \item \textbf{Off-style strain:} if you're forced to work wildly \emph{against} your Art (no tools, hostile locus), the GM may set a worse \textbf{Position} or require you to accept extra \textbf{Backlash} choices on a Partial.
\end{itemize}

\paragraph{Examples.}
\begin{itemize}
  \item \emph{Sealwright's Chant} — lantern, chain, chalk; Elements: Earth/Fire; \textit{I sing hinges into hearing.}
  \item \emph{Salt-Thread Scribe} — salt, silk, breath; Elements: Water/Luck; \textit{Knots remember routes and debts.}
  \item \emph{Name-Caller} — true names, oaths, bells; Elements: Fate/Air; \textit{Words stand up when I say them.}
\end{itemize}

\bigskip

% =========================================
% DV Guidance (Element-aware)
% =========================================

\section{DV Guidance}
\label{sec:dv-guidance-elements}

\noindent Use the freeform casting flow (\S\ref{sec:arcane-freeform}: Weave $\rightarrow$ Cast). Set DV by scope and situation, then \emph{shade} it with Element alignment.

\begin{tabular}{@{}llp{9cm}@{}}
\textbf{DV 2} & Small/local & One target, Near band, short veil, fragile barrier, simple shove/cleanse. Aligned Realm/tools may bump Effect to \emph{Standard}. \\
\textbf{DV 3} & Scene-scale & Small zone multi-target, sturdy barrier [2/4], move one band, strong veil/reveal, oath ping. Opposed Realm may drop Effect to \emph{Limited}. \\
\textbf{DV 4} & Big swing & Zone control, Far reach, complex transport, high-quality ward, reshape Position broadly. Expect backlash if rushed. \\
\textbf{DV 5+} & Set-piece/ritual & Battlefield rewriting, clause-level edits of reality, defying entrenched wards. Wants prep, aids, or locus. \\
\end{tabular}

\paragraph{Backlash by Element).}
On Partial/Miss (or two 1s on a Hit), color the cost:
\begin{itemize}
  \item \textbf{Earth} $\rightarrow$ rubble, pin, heavy footing; \textbf{vs Air} $\rightarrow$ sound carries, you're exposed.
  \item \textbf{Fire} $\rightarrow$ burns, flares, unwanted attention; \textbf{vs Water} $\rightarrow$ slick, shorted gear, sodden scrolls.
  \item \textbf{Air} $\rightarrow$ scatter, drop, vertigo; \textbf{vs Earth} $\rightarrow$ stuck, cramped, dust choke.
  \item \textbf{Water} $\rightarrow$ leak, flood, cold drag; \textbf{vs Fire} $\rightarrow$ smoke, sputter, dim.
  \item \textbf{Fate} $\rightarrow$ options close, "only one way"; \textbf{vs Luck} $\rightarrow$ mischance hits an ally.
  \item \textbf{Luck} $\rightarrow$ side-effect elsewhere, fragile success; \textbf{vs Fate} $\rightarrow$ your gambit snaps to a fixed, harsher outcome.
  \item \textbf{Life} $\rightarrow$ overgrowth, fever, hungry repair; \textbf{vs Death/Dreams} $\rightarrow$ numbness, sleep-tug, a door opens.
  \item \textbf{Death/Dreams} $\rightarrow$ fade, echo, threshold opens/closes; \textbf{vs Life} $\rightarrow$ pain returns, rot sets in, breath drags.
\end{itemize}

\bigskip

% =========================================
% Cross-Section Pointers
% =========================================

\paragraph{See also.}
\begin{itemize}
  \item \textbf{Arcane Casting (Freeform)} for the two-stage flow and Backlash (\S\ref{sec:arcane-freeform}).
  \item \textbf{Tags (Library)} for robust effect shorthands you can print onto spells (\S\ref{sec:tag-library}); use \texttt{[WARD]/[BANISH]} for Outsiders with the unified DV/Leash model.
  \item \textbf{Range \& Position (Hybrid)} for Close/Near/Far/Absent and CP/Boon nudges.
\end{itemize}

\section{Outsiders (Definition \& Caps)}

\textbf{Outsider:} any being \textbf{not native} to this world/setting—e.g., summoned entities, demons, extraplanar creatures, spirits, some celestials, and similar arrivals-from-beyond.

\paragraph{Cap (for Outsiders).}
\begin{itemize}
  \item \textbf{PC-summoned Outsiders:} Cap is set by the summoner's talent (Lesser = \textbf{Cap 1}, Greater = \textbf{Cap 3}).
  \item \textbf{World/NPC Outsiders:} the stat block lists Cap; if absent, the GM assigns: Lesser = 1, Greater = 3, Elder = 5 (rare).
\end{itemize}

\bigskip

\section{Rites (Pact Magic)}

\textit{Rites are precise, named effects granted by powerful Patrons. This framework keeps play fast and consistent.}

\subsection*{Becoming an Acolyte}
\begin{itemize}
  \item Choose \textbf{one} or \textbf{two Patrons}. (Two is allowed; see \emph{Cross-Patron Interference}.)
  \item Start with \textbf{2 Low} and \textbf{1 Standard} Rite total (split across your Patrons as you wish).
  \item Track an \textbf{Obligation ledger} \emph{per Patron} (segments of debt accrued through use).
\end{itemize}

\subsection*{Using a Rite}
Unless a Rite says otherwise:
\begin{enumerate}
  \item \textbf{Invoke (1 action).} Speak the name, draw the sign, or employ the proper tool. If a roll is required, the GM sets \textbf{DV by fiction} (typical scene DV 2–4).
  \item \textbf{Mark Obligation.} Invoking a Rite usually marks \textbf{+1 segment} to that Patron's ledger (some Low rites may be free at GM discretion).
  \item \textbf{Push It (optional).} Amplify the effect; mark \textbf{+1 Obligation}. A given Rite may be Pushed at most \textbf{once per scene}.
  \item \textbf{Backlash (on a 1 or Miss).} The GM inflicts a fitting consequence \emph{or} marks \textbf{+1 Obligation}.
\end{enumerate}

\paragraph{Stacking \& Caps.}
\begin{itemize}
  \item \textbf{Same-source scene buffs do not stack.} Take the best version; others are suppressed.
  \item A \textbf{single Rite} can add at most \textbf{+3 Obligation segments per scene} \emph{from that Rite} (Invoke + Push + possible Backlash).
\end{itemize}

\paragraph{Cross-Patron Interference.}
If you \textbf{switch Patrons in the same scene} (e.g., after using Gate, you Invoke Ikasha):
\begin{itemize}
  \item Immediately mark \textbf{+1 Obligation} to the second Patron (jealous attention).
  \item The GM may color the next moment with minor interference (audible hum, flicker, bystander notice).
  \item Each further \emph{switch} that scene marks \textbf{+1} again.
\end{itemize}

\paragraph{Clearing Obligation.}
Between scenes or in downtime, you may clear \textbf{1–2 segments} per Patron by:
\begin{itemize}
  \item \textbf{Service} fitting the Patron (tend hinges for the Gate, keep sanctioned watch for the Oath, etc.), or
  \item Spending \textbf{1–2 Boons} to represent offerings, favors, or perfect observance.
\end{itemize}
At \textbf{+3/+6/+9} total segments, foreshadow manifestations (omens, auditors, bailiffs) as your table prefers.

\paragraph{Tags.}
Rites may include SRD tags such as \texttt{[WARD]}, \texttt{[BANISH]}, \texttt{[UNWARD]}. \textbf{Tags only function when printed on a specific Talent, Ability, or as the result of a Spell.} For Outsiders, use the unified DV/Leash rules (see Tags section).


\bigskip

\section{Two Sample Patrons (Drop-In)}

\subsection*{The Sealed Gate — Boundaries, Jurisdiction, Closure}
\textit{You write borders into the world and prosecute trespass.}

\paragraph{Low Rites.}
\begin{description}
  \item[Seal the Threshold.] Draw a brief sign across a door/line. Enemies crossing suffer worsened position or stumble. \emph{Invoke:} +1 Obl. \emph{Push:} strengthen the edge (e.g., treat as difficult terrain) \emph{(+1 Obl)}.
  \item[Key's Rebuke.] Snap a spectral hasp at a reaching hand or tool; stagger or disarm for a beat. \emph{Invoke:} +1 Obl.
\end{description}

\paragraph{Standard Rites.}
\begin{description}
  \item[Circle of Denial \texttt{[WARD]} (affects Outsiders).] Mark a ring or arc; Outsiders that \emph{cross} must test \emph{Cross the Ward} (\textbf{DV = Cap}). \emph{Hit:} cross and \textbf{+DV} to Leash; \emph{Partial:} cross and \textbf{+1}; \emph{Miss:} fail to cross this beat. \emph{Invoke:} +1 Obl. \emph{Push:} fortify the circle \emph{(+1 Obl)}.
  \item[Writ of Passage.] Name one path as \emph{permitted}; allies on that route gain improved position/flow. \emph{Invoke:} +1 Obl. \emph{Push:} extend to one extra ally or carry across one obstacle \emph{(+1 Obl)}.
\end{description}

\paragraph{High Rites.}
\begin{description}
  \item[Banishment Knot \texttt{[BANISH]} (Outsider).] Cast out a named Outsider at Near. \emph{DV = Cap}. \emph{Hit:} \textbf{+DV} to Leash (often filling); \emph{Partial:} \textbf{+1}. \emph{Invoke:} +1--2 Obl (table preference). \emph{Push:} strip one tether/anchor \emph{(+1 Obl)}.
\end{description}

\subsection*{Ikasha — Shadow, Latent Potential, Penumbra}
\textit{You bank what might happen and cash it when the light looks away.}

\paragraph{Low Rites.}
\begin{description}
  \item[Borrowed Shade.] Cool or deepen existing dim; allies acting within gain improved position until violence occurs there. \emph{Invoke:} +1 Obl.
  \item[Mark of Penumbra.] Tag a place/object with a small shadow-glyph; later, blur sound or slip attention at that mark once. \emph{Invoke:} +1 Obl. \emph{Push:} extend the tell-skip to one ally \emph{(+1 Obl)}.
\end{description}

\paragraph{Standard Rites.}
\begin{description}
  \item[Umbra Thread.] Tether two shadows you can see; once this scene, you or a willing ally traverse between them as a single Move. \emph{Invoke:} +1 Obl. \emph{Push:} allow one additional traverse or longer reach \emph{(+1 Obl)}.
  \item[Deferred Answer.] Bank one pointed question/command aimed at you; until you answer, you resist pins and scans more easily. When you cash it, answer honestly or accept a complication. \emph{Invoke:} +1 Obl.
\end{description}

\paragraph{High Rite.}
\begin{description}
  \item[Eclipse Clause \texttt{[WARD]} (affects Outsiders).] For one scene, an area is penumbral; ranged targeting beyond Near is impaired; shadow-to-shadow movement is improved. Outsiders crossing the edge use the unified \texttt{[WARD]} crossing (DV = Cap; \emph{Hit} \textbf{+DV} Leash, \emph{Partial} \textbf{+1}). \emph{Invoke:} +2 Obl. \emph{Push:} harden the border for one beat \emph{(+1 Obl)}.
\end{description}

\bigskip

\section{SRD Notes \& Hooks (Rites)}

\paragraph{Design notes.}
\begin{itemize}
  \item Keep DVs \textbf{fictional and modest} (2–4 typical). Save 5+ for set-pieces, rituals, or decisive gambits.
  \item Use \textbf{omens at +3/+6/+9 Obligation} to foreshadow, not auto-punish; let debt matter on screen.
  \item \textbf{No same-source stacking.} If a player tries to layer multiple scene-long effects from the same Patron, take the best only.
\end{itemize}

\paragraph{Interoperability.}
\begin{itemize}
  \item Rites that list \texttt{[WARD]}, \texttt{[BANISH]}, or \texttt{[UNWARD]} use the SRD Tag rules. For \textbf{Outsiders}, \textbf{DV = Cap}; \emph{Hit} adds \textbf{+DV} to Leash; \emph{Partial} adds \textbf{+1}.
  \item Summoning from the \emph{Pact-Whisperer} rules coexists cleanly: Wards accelerate Leash; Banishing fills or advances it.
\end{itemize}

\section{Pact-Whisperer — Standalone Summoning}

\textit{Call a spirit fast; bind it with a tiny leash; keep play fiction-first.}

\subsection*{Core Procedure}
\begin{enumerate}
  \item \textbf{Call (1 action):} A spirit manifests at \textbf{Near}. Pick a \textbf{Spirit Template}.
  \item \textbf{Bind (no extra roll):} choose one: spend \textbf{1 Boon} \emph{or} mark \textbf{1 Fatigue}.
  \item \textbf{Leash (lives on the spirit):} set \textbf{Leash = Cap + 2} segments. \\
        Cap 1 $\rightarrow$ [3] \hspace{1em} Cap 3 $\rightarrow$ [5]
  \item \textbf{Tick Leash} (no rolls) when any happen: the spirit \textbf{takes harm}; you \textbf{command against its nature}; you \textbf{split focus} (you take another significant action while it acts on your order); a rival \textbf{contests} it; it \textbf{moves from Close to Far}. Crossing a \texttt{[WARD]} uses the rules under \S\ref{sec:tags}.
  \item \textbf{When the Leash fills:} the spirit \textbf{acts to its nature once}, then \textbf{departs}.
\end{enumerate}

\paragraph{Action economy.} Issuing a meaningful command uses \textbf{your action}. \\
\textbf{Limit:} one active spirit at a time (you may Call again after departure). \\
\textbf{Downtime:} all summons \textbf{depart at downtime} unless an ability explicitly says otherwise.

\paragraph{Boon Finesse }
Once per round, you may spend \textbf{1 Boon} to \textbf{clear 1 tick} from your current spirit's Leash. You cannot use this after the Leash has already filled.

\paragraph{Talent gates.}
\begin{itemize}
  \item \textbf{Lesser Pactwright (Talent):} You may Call spirits of \textbf{Cap 1}.
  \item \textbf{Greater Pactwright (Talent):} You may also Call spirits of \textbf{Cap 3}.
\end{itemize}

\bigskip

\section{Tags for Talents, Abilities, and Spell Results}
\label{sec:tags}

\noindent\textbf{Tags apply only when printed on a specific \emph{Talent}, \emph{Ability}, or as the \emph{result of a Spell}.}
They do nothing on their own. Spells are accessible but \textbf{risky}; Talents/Abilities provide safer, repeatable access.

\subsection*{\texttt{[WARD]}}
Creates a magical \textbf{edge/circle/zone} that \textbf{challenges Outsiders} (or other targets if the text explicitly says so).

\paragraph{Crossing a \texttt{[WARD]} (when it affects Outsiders).}
\begin{itemize}
  \item Roll a suitable pool (table choice, e.g., Spirit+Lore, Resolve+Lore, Presence+Command).
  \item \textbf{DV = target's Cap.}
  \item \textbf{On a Hit:} the Outsider \textbf{crosses} and its \textbf{Leash gains +DV segments} (i.e., +Cap).
  \item \textbf{On a Partial:} the Outsider \textbf{crosses} and its \textbf{Leash gains +1 segment}.
  \item \textbf{On a Miss:} it \textbf{fails to cross} this beat.
\end{itemize}

\paragraph{Non-summoned Outsider.}
If it has no Leash, create a \textbf{temporary Exit Tally = Cap + 2} the first time it is affected; treat Leash gains as ticks on that tally; when filled, it departs after a brief "acts to nature" beat.

\paragraph{Scope.}
Ward shape, duration, materials, and any advantage/disadvantage are defined by the source Talent/Ability or Spell. \textbf{DV for crossing is always the target's Cap.} \\
\textbf{Wards can affect anything,} but the ability must \textbf{explicitly state} what they affect; the unified DV/Leash math in this section applies only when the target is an \textbf{Outsider}.

\paragraph{\texttt{[UNWARD]} to remove.}
Only an ability with \texttt{[UNWARD]} (or an explicit counter named in the Ward's text) can dismiss a Ward.

\subsection*{\texttt{[BANISH]}}
Drives a visible \textbf{Outsider} toward departure.
\begin{itemize}
  \item Target a visible Outsider in range; roll the pool the source lists.
  \item \textbf{DV = target's Cap.}
  \item \textbf{On a Hit:} add \textbf{+DV segments} to its \textbf{Leash} (or Exit Tally).
  \item \textbf{On a Partial:} add \textbf{+1 segment}.
  \item \textbf{On a Miss:} no effect.
\end{itemize}
If this fills the Leash/Tally, resolve a brief "acts to nature" beat, then the Outsider departs.

\subsection*{\texttt{[UNWARD]}}
Unmakes or suppresses a \texttt{[WARD]} created by a Talent/Ability or as the result of a Spell.
\begin{itemize}
  \item \textbf{DV:} by fiction (materials, sanctity, prep, locus, opposition).
  \item \textbf{On a Hit:} Ward dismissed/suppressed per text.
  \item \textbf{On a Partial:} weakened or suppressed briefly (about one beat), per GM call.
  \item \textbf{On a Miss:} no effect.
\end{itemize}

\section{Tags}
\label{sec:tag-library}

\noindent\textbf{Tags apply only when printed on a \emph{Talent}, an \emph{Ability}, or as the \emph{result of a Spell}.} They have no effect on their own and cannot be used directly via an action.

\subsection*{Conventions}
\begin{itemize}
  \item \textbf{DV by fiction} unless noted: potency, prep, place, materials, opposition, and risk set DV (typ. 2--4).
  \item \textbf{Duration}: scene, unless the source says otherwise.
  \item \textbf{Counters}: only via other printed Tags/abilities; mundane counters work if the ability text allows.
  \item \textbf{Outsiders}: see \S\ref{sec:outsiders} and \S\ref{sec:tags-ward-banish} for \texttt{[WARD]} and \texttt{[BANISH]} specifics.
\end{itemize}

\subsection*{Control \& Countermagic}
\paragraph{\texttt{[DISPEL]}} End an ongoing magical effect/construct.
\begin{itemize}
  \item DV: by fiction. \textbf{Hit}: dismiss/suppress per text. \textbf{Partial}: suppress briefly or shrink scope. \textbf{Miss}: no effect.
\end{itemize}

\paragraph{\texttt{[COUNTER]}} Interrupt a cast/rite in progress.
\begin{itemize}
  \item Window: during the listed casting/rite window. DV: by fiction.
  \item \textbf{Hit}: cancel. \textbf{Partial}: degrade Position/Effect or impose a cost. \textbf{Miss}: no effect.
\end{itemize}

\paragraph{\texttt{[BARRIER]}} Create cover/obstruction.
\begin{itemize}
  \item DV: by fiction. \textbf{Hit}: place barrier with simple integrity [2/4/6]. \textbf{Partial}: narrow/fragile lane.
  \item Counters: \texttt{[DISPEL]}, brute force, clever route.
\end{itemize}

\paragraph{\texttt{[SEAL]} / \texttt{[UNSEAL]}} Lock or unlock a container/door/portal (not a spirit).
\begin{itemize}
  \item DV: by fiction. \textbf{Hit}: locked/unlocked; state allowed bypasses. \textbf{Partial}: short-lived/leaky seal.
\end{itemize}

\subsection*{Concealment \& Revelation}
\paragraph{\texttt{[VEIL]}} Obscure a person/thing/zone.
\begin{itemize}
  \item DV: by fiction. \textbf{Hit}: impose disadvantage on scans/Notice vs.\ the subject; specify limits (angle, distance, scent).
  \item \textbf{Partial}: works only at range or under dim. Counters: \texttt{[REVEAL]}, strong light, proof.
\end{itemize}

\paragraph{\texttt{[REVEAL]}} Expose illusions, disguises, hidden clauses.
\begin{itemize}
  \item DV: by fiction. \textbf{Hit}: surface the truth/sign; say how it shows. \textbf{Partial}: a tell/clue, not the full picture.
\end{itemize}

\paragraph{\texttt{[MARK]}} Tag a target for tracking or leverage.
\begin{itemize}
  \item DV: by fiction. \textbf{Hit}: place a visible/invisible mark; once/scene you or an ally gain +1 die when acting directly against the Marked target.
  \item \textbf{Partial}: noisy/short-lived. Counters: \texttt{[CLEANSE]}, \texttt{[DISPEL]}, salt/iron if text allows.
\end{itemize}

\subsection*{Boons \& Burdens}
\paragraph{\texttt{[CURSE]}} Inflict a sticky hindrance with a clear release.
\begin{itemize}
  \item DV: by fiction. \textbf{Hit}: apply a named condition. \textbf{Partial}: milder/intermittent. Counters: \texttt{[CLEANSE]} or listed keys.
\end{itemize}

\paragraph{\texttt{[CLEANSE]}} Remove/suppress a condition (poison, disease, \texttt{[CURSE]}, fear).
\begin{itemize}
  \item DV: by fiction. \textbf{Hit}: remove one named affliction. \textbf{Partial}: suppress/reduce for the scene.
\end{itemize}

\paragraph{\texttt{[FORTIFY]}} Harden a person/place/object against a vector (fire, blades, fear, sway).
\begin{itemize}
  \item DV: by fiction. \textbf{Hit}: raise Position \emph{or} reduce consequence severity vs.\ that vector this scene.
  \item \textbf{Partial}: limited scope (one ally/doorway).
\end{itemize}

\subsection*{Influence \& Oaths}
\paragraph{\texttt{[COMMAND]}} Issue a clear order to a sapient target.
\begin{itemize}
  \item DV: by fiction (authority, leverage, fear, ritual standing).
  \item \textbf{Hit}: comply now or suffer an immediate cost (GM says). \textbf{Partial}: hesitate/bargain/partial. \textbf{Miss}: refusal/blowback.
\end{itemize}

\paragraph{\texttt{[OATH]}} Bind parties to terms; breaking has teeth.
\begin{itemize}
  \item DV: by fiction (witnesses, sanctity, stakes). \textbf{Hit}: enforceable pact with stated boon and breach consequence.
  \item \textbf{Partial}: loophole exists. Counters: release clause, \texttt{[CLEANSE]} if text allows.
\end{itemize}

\paragraph{\texttt{[SANCTIFY]}} Consecrate a zone to a code/patron.
\begin{itemize}
  \item DV: by fiction. \textbf{Hit}: specify allowed/prohibited acts; violations start Risky or impose a soft consequence.
  \item \textbf{Partial}: patchy coverage. Counters: opposing rite, profanation, time.
\end{itemize}

\subsection*{Movement \& Making}
\paragraph{\texttt{[PASSAGE]}} Declare a route as permitted/easy.
\begin{itemize}
  \item DV: by fiction. \textbf{Hit}: allies on that path gain improved flow (Position/Effect bump or ignore 1 level of difficult terrain).
  \item \textbf{Partial}: one ally/segment only.
\end{itemize}

\paragraph{\texttt{[TRANSPORT]}} Move a target across an obstacle (blink, lift, pull).
\begin{itemize}
  \item DV: by fiction (mass, range, warding). \textbf{Hit}: relocate within listed range; state tells/costs.
  \item \textbf{Partial}: arrive off-balance, drop gear, or worse Position.
\end{itemize}

\paragraph{\texttt{[CONJURE]}} Create a useful object/cover/hazard.
\begin{itemize}
  \item DV: by fiction. \textbf{Hit}: conjure item/zone with integrity [2/4/6] or a ticking hazard (burn/freeze/haze).
  \item \textbf{Partial}: fragile/short-lived. Counters: \texttt{[DISPEL]}, force, time.
\end{itemize}

\subsection*{Outsider-Specific Tags}
\label{sec:tags-ward-banish}

\paragraph{\texttt{[WARD]}} Challenge Outsiders crossing a warded edge/zone.
\begin{itemize}
  \item \textbf{DV = target Cap.} \textbf{Hit}: crosses; \textbf{+DV} segments to Leash (or Exit Tally). \textbf{Partial}: crosses; \textbf{+1} segment. \textbf{Miss}: fails to cross this beat.
  \item Ability text must \emph{explicitly} state if it affects Outsiders (or anything else).
\end{itemize}

\paragraph{\texttt{[BANISH]}} Drive a visible Outsider toward departure.
\begin{itemize}
  \item \textbf{DV = target Cap.} \textbf{Hit}: \textbf{+DV} to Leash/Tally. \textbf{Partial}: \textbf{+1}. \textbf{Miss}: no effect.
\end{itemize}

\paragraph{\texttt{[UNWARD]}} Unmake/suppress a \texttt{[WARD]} created by a Talent/Ability or Spell result.
\begin{itemize}
  \item DV: by fiction. \textbf{Hit}: dismiss/suppress per text. \textbf{Partial}: weaken/suppress briefly (≈1 beat). \textbf{Miss}: no effect.
\end{itemize}
\bigskip

\section{Outcome Shorthand (for this Section)}

\begin{description}
  \item[Hit:] roll meets or exceeds DV.
  \item[Partial:] a Hit with a cost/complication (your table's usual "success with cost").
  \item[Miss:] roll does not meet DV.
\end{description}

\bigskip

\section{One-Glance Summary}

\begin{itemize}
  \item \textbf{Outsider} = not native (summoned, demons, extraplanar, spirits, some celestials).
  \item \textbf{Unified DV/impact for Outsiders:} \texttt{[WARD]} crossing and \texttt{[BANISH]} use \textbf{DV = Cap}; \textbf{Hit} $\rightarrow$ \textbf{+DV} Leash segments; \textbf{Partial} $\rightarrow$ \textbf{+1}.
  \item \textbf{Leash} lives on summoned spirits (\textbf{Cap + 2} segments). Non-summoned Outsiders get a temporary \textbf{Exit Tally = Cap + 2} when first affected.
  \item \textbf{Boon Finesse is core:} once per round, spend 1 Boon to clear 1 Leash tick.
  \item \textbf{Gates:} Tags apply only when printed on a \textbf{Talent}, \textbf{Ability}, or as the \textbf{result of a Spell} and cannot be applied with just an action.
  \item \textbf{Summons depart at downtime} unless stated otherwise.
\end{itemize}

\subsection{Deck of Consequences}

\subsection{Two Deck Systems (Compatibility)}
Fate's Edge uses two distinct card tools:

\paragraph{Travel Decks (regional, 52-card).}
\emph{Spade}=Place, \emph{Heart}=Actor, \emph{Club}=Pressure, \emph{Diamond}=Leverage. These power journeys and gates.

\paragraph{Deck of Consequences (scene drama).}
\emph{Hearts}=emotional/social fallout, \emph{Spades}=harm/escalation, \emph{Clubs}=material cost, \emph{Diamonds}=magical/spiritual disturbance.

\textit{Guidance:} Never mix suit meanings across decks. When a rule references "Spade/Club/Diamond," it means \emph{Travel}. When it says "Hearts/Spades/Clubs/Diamonds," it means \emph{Consequences}.

\subsection{Structure of the Deck}
\begin{itemize}
    \item \textbf{Suits} = Domains of Complications
    \begin{itemize}
        \item Hearts: Emotional, social, or relational fallout.
        \item Spades: Harm, danger, or escalation of conflict.
        \item Clubs: Resource strain, economic or material cost.
        \item Diamonds: Magical, spiritual, or cosmic disturbances.
    \end{itemize}
    \item \textbf{Ranks} = Severity of Complications
    \begin{itemize}
        \item Ace–3: Minor inconvenience or flavor complication.
        \item 4–6: Moderate setback with some narrative teeth.
        \item 7–9: Significant consequence altering the course of action.
        \item 10–King: Major fallout, introducing new problems or lasting scars.
    \end{itemize}
\end{itemize}

\subsection{Using the Deck}
\begin{enumerate}
  \item Player rolls; each 1 generates a Complication Point (CP).
  \item GM chooses one method for that roll:
  \begin{enumerate}
    \item \textbf{Direct Spend}: translate CP into immediate consequences/clock ticks; or
    \item \textbf{Deck Draw}: draw up to \textbf{min(CP, 3)} cards and \textbf{synthesize a single twist}
    guided by suit and highest rank.
  \end{enumerate}
\end{enumerate}

\section{Player Archetypes at the Table}

\subsection{The Solo}
\begin{itemize}
    \item Invests XP primarily in Attributes and Skills.
    \item Strengths: always ready, iconic spotlight.
    \item Risks: narrow toolkit; may lag in social or resource scenes.
\end{itemize}

\subsection{The Mixed Player}
\begin{itemize}
    \item Balances XP between self and assets.
    \item Strengths: adaptable, bridges party gaps.
    \item Risks: upkeep spread thin.
\end{itemize}

\subsection{The Mastermind}
\begin{itemize}
    \item Builds networks, followers, and assets.
    \item Strengths: broad reach, drives strategies.
    \item Risks: Complication fallout; vulnerable allies.
\end{itemize}

\section{Campaign Frame / Finale: The Crown Spread}

\subsection{Session 0: The Crown Spread (Initial Draw)}
Draw 5 cards: Spade, Heart, Club, Diamond, and a Wild (any suit; reveal last).

\subsection{The Campaign Clock}
Track two dials over the campaign:
\begin{itemize}
    \item \textbf{Mandate (0–6)}: The table's public legitimacy and buy-in.
    \item \textbf{Crisis (0–6)}: The opposition engine (rivals, pressure rails, attrition).
\end{itemize}

\subsection{Finale Procedure (Crown Beat)}
Use the Session 0 Crown Spread to seed setup; then run the three-beat crown.

\subsection{Legacy Conversion (Epilogue)}
After the Finale, each PC draws 2 cards and answers epilogue prompts by suit.

\section{Travel Framework}

\subsection{Core Travel Procedure}
For each leg of a journey, draw 3–4 cards using the decks for your destination and controlling authority.
\begin{itemize}
    \item Spade from the destination deck: sets the scene (place).
    \item Heart from the destination deck: introduces the local actor or faction.
    \item Club from the Wilds (general hazards) or destination (if strongly policed): brings pressure.
    \item Diamond from the authority that gates the route: papers, escorts, rights, or exceptions.
\end{itemize}

Set a travel clock by the highest rank (2–5⇒4 • 6–10⇒6 • J/Q/K⇒8 • A⇒10). On success, advance to the next leg; on failure, mark delay, debt, or diversion and resolve a consequence in the fiction.

\subsection{Mode rules}
\begin{itemize}
    \item \textbf{Sea legs} (Amaranthine/Dolmis/Aberderrin): If Theona or Valewood 9s show up anywhere in the seed, add an omission or taboo to the leg.
    \item \textbf{Aeler Aces and Valewood Corridors}: Any A means wood actively rearranges paths or wakes structures.
    \item \textbf{Rivers}: Bridges, booms, and law in Ecktoria/Viterra; reed-mazes and bell-lines in Mistlands/Linn waters.
    \item \textbf{Frontier blends}: When origin and destination disagree on law, draw two Diamonds (one from each law) and choose which you will be judged by at the end of the leg.
\end{itemize}

\subsection{Route Modules}
\subsubsection{Amaranthine Coastway}
Kahfagia → Ecktoria → Acasia → Marcott (Vhasia) → Fairport (Viterra).

\subsubsection{Astroegro Straits}
Thepyrgos controls the hinge between seas.

\subsubsection{Dolmis Circuits}
Fairport (Viterra) → Theona (Three Greens) → Ubral fjords → Aelinnel west shore.

\subsubsection{Aelerian Passes Underways}
Vhasia/Viterra/Ubral south slopes → Aeler gates → Mistlands.

\subsubsection{Shadow Corridors}
Thin Shore (Valewood east coast): risky misted corridor north–south toward Zakov.

\subsubsection{River Roads}
Belworth: forms the boundary between Vhasia and Viterra.

\subsubsection{Steppe Frontiers (Violet Steppes \& Meadows)}
Ykrul ↔ Vilikari ↔ Ecktoria/Acasia borders.

\section{Design Philosophy Guardrails}

\subsection{Core Principles}
\begin{enumerate}
    \item \textbf{Narrative Primacy}: Mechanics serve story, not replace it.
    \item \textbf{Risk as Drama}: Every roll carries potential for triumph + complication.
    \item \textbf{Meaningful Growth}: XP investment creates lasting character/world change.
    \item \textbf{Consequence Weight}: Choices ripple outward, nothing is free.
    \item \textbf{Fail Forward}: Boons are primarily earned through failing rolls. 1s become narrative fuel.
\end{enumerate}

\subsection{Mechanical Constraints}
\begin{itemize}
    \item \textbf{ASSIST MAX}: +3 dice total per roll, regardless of helpers. Exception: The "Exceptional Coordination" Talent allows one follower to provide +4 assist dice.
    \item \textbf{BOON MAX}: 5 total, 2→1 XP conversion once/session (max 2 XP via conversion per session).
    \item \textbf{INITIATIVE}: 1 Follower Action per scene crew-wide.
    \item \textbf{OVER-STACK}: 2+ structural advantages = start rails +1 OR GM banks +1 CP.
    \item \textbf{POSITION}: Controlled | Risky | Desperate (affects success/failure texture).
\end{itemize}

\paragraph{High-Tier CP Sinks.}
For 3–6+ CP spends that move the world (reputation cascades, faction instability, resonance, prophecy), see the stand-alone \emph{High CP Sinks} handout. A good default: at end of leg, \textbf{3 CP → tick 1 Front}.

\subsection{Balance Philosophy}
\begin{itemize}
    \item Quadratic follower costs ensure high follower investments are intentionally expensive for mechanical balance.
    \item Risk-reward equilibrium maintains that dangerous magic prevents caster dominance while preserving narrative impact.
    \item Viable approaches across all character builds are supported through balanced mechanics.
\end{itemize}

\subsection{Progression Clarity}
\begin{itemize}
    \item Attribute cost diminishing returns encourage diversification.
    \item Skill mastery benefits provide meaningful advancement.
    \item Prestige ability considerations include scaling options with additional XP investment.
\end{itemize}

\end{document}
