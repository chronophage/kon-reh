\documentclass[11pt]{article}
\usepackage[margin=1in]{geometry}
\usepackage{graphicx}
\usepackage{booktabs}
\usepackage{longtable}
\usepackage{array}
\usepackage{fancyhdr}
\usepackage{titlesec}
\usepackage{hyperref}
\usepackage{enumitem}
\usepackage{multicol}

\pagestyle{fancy}
\fancyhf{}
\rhead{\thepage}
\lhead{Fate's Edge SRD}
\renewcommand{\headrulewidth}{0.4pt}

\titleformat{\section}{\large\bfseries}{\thesection}{1em}{}
\titleformat{\subsection}{\bfseries}{\thesubsection}{1em}{}

\title{Fate's Edge System Reference Document (SRD)}
\author{}
\date{}

\begin{document}

\maketitle

\tableofcontents
\newpage

\section{Core Principles}

\subsection{Identity of Fate's Edge}
Fate's Edge is a narrative-first tabletop roleplaying system where every action carries weight, every choice has consequence, and every spell risks backlash. Dice are not simply a measure of success or failure—they are instruments of fate, weaving opportunity with risk.

\subsection{A World of Consequences}

\subsubsection{Design Goals}
\begin{itemize}
    \item \textbf{Narrative Primacy}: Mechanics exist to serve the story.
    \item \textbf{Risk as Drama}: Every roll carries the potential for triumph and complication.
    \item \textbf{Meaningful Growth}: Advancement is more than improving statistics.
\end{itemize}

\subsubsection{The Central Question}
What are you willing to risk, and what are you willing to pay, to reshape the world around you?

\subsubsection{Tone of Play}
\begin{itemize}
    \item Cinematic, with pacing tied to narrative beats.
    \item Consequential, where even small choices ripple outward.
    \item Collaborative, empowering both GM and players.
\end{itemize}

\subsection{Key Concepts}

\subsubsection{Narrative Time}
Time is measured by story weight:
\begin{itemize}
    \item A Moment — A heartbeat, a glance, a single strike or word.
    \item Some Time — A few minutes, enough for a skirmish.
    \item Significant Time — Hours, long enough for travel or rituals.
    \item Days — Large-scale endeavors: marches, training, recovery.
\end{itemize}

\subsubsection{Complication Points}
Whenever a player rolls dice, each result of 1 generates a Complication Point (CP). These are narrative fuel. The GM spends them to introduce twists.

\subsubsection{Affinity}
Each culture provides an Affinity: a narrative edge or metaphysical bond. Affinities make certain Arts, skills, or actions more reliable.

\subsubsection{Prestige Abilities}
High-level talents unlocked by mastering cultural arts or philosophies. They are narrative milestones as much as mechanical ones.

\subsubsection{On-Screen vs. Off-Screen}
\begin{itemize}
    \item \textbf{On-Screen Resources}: Companions, hirelings, or allies who stand beside you in danger.
    \item \textbf{Off-Screen Resources}: Taverns, estates, titles, or networks of informants.
\end{itemize}

\section{Core Mechanic}

\subsection{The Art of Consequence}

\subsubsection{Procedure}
All significant actions follow a three-step process:
\begin{enumerate}
    \item \textbf{Approach}: The player describes both what their character wants and how they attempt it.
    \item \textbf{Execution}: Build a dice pool equal to Attribute + Skill and roll that many d10s. Each die of 6 or higher counts as a success. Each 1 rolled generates a Complication Point.
    \item \textbf{Outcome}: The GM interprets total successes against the difficulty of the task. Complication Points are then spent to weave narrative setbacks.
\end{enumerate}

\subsubsection{The Description Ladder}
\begin{itemize}
    \item \textbf{Basic Action}: Roll the pool as-is. All 1s remain as Complication Points.
    \item \textbf{Detailed Action}: A clear, descriptive flourish allows the player to re-roll one die showing 1.
    \item \textbf{Intricate Action}: A richly described, multi-sensory action allows the player to re-roll all dice showing 1, and add one positive narrative flourish to the scene if they succeed.
\end{itemize}

\paragraph{Rule — Re-rolling 1s and CP}
Re-rolling 1s does \emph{not} remove the Complication Point already generated by those dice. If any re-rolled dice show 1 again, they generate additional CP as normal.
\[
\text{Let } C_{0}=\text{initial 1s,\quad } C_{r}=\text{1s on re-rolls}\;\Rightarrow\; \text{Total CP}=C_{0}+C_{r}.
\]

\textit{Example:} You roll 7d10: \(\{9,8,5,4,3,\;1,\;1\}\Rightarrow C_{0}=2\).
You re-roll both 1s (Intricate): \(\{6,2\}\Rightarrow C_{r}=0\).
Final: successes = 3, \(\text{CP}=2\) (those initial CP remain).

\subsubsection{Complication Points}
Complication Points (CP) are the engine of drama. They are not simple penalties, but narrative levers. The GM spends CP to introduce setbacks appropriate to the context:
\begin{itemize}
    \item Escalation — drawing more enemies, raising the stakes.
    \item Exhaustion — draining time, resources, or positioning.
    \item Exposure — revealing hidden actions, alerting foes.
    \item Collateral — harm or danger spilling over onto allies, innocents, or surroundings.
\end{itemize}

\subsubsection{Design Intent}
This mechanic ensures that every roll changes the story. Success without risk is rare, and even failure opens new narrative avenues.

\subsubsection{GM Quick Reference: Adjudicating Skill Checks}

\paragraph{Difficulty Ladder (Set Before the Roll)}
\begin{center}
\begin{tabular}{cll}
\toprule
\textbf{DV} & \textbf{Name} & \textbf{When to Use} \\
\midrule
2 & Routine & Clear intent, modest stakes, controlled environment. \\
3 & Pressured & Time pressure, mild resistance, partial info. \\
4 & Hard & Hostile conditions, active opposition, precise timing. \\
5+ & Extreme & Multiple constraints, high precision, dramatic failure. \\
\bottomrule
\end{tabular}
\end{center}

\begin{table}[htbp]
\centering
\caption{Boon Generation by Dice Pool and Difficulty}
\begin{tabular}{|c|c|c|c|c|c|}
\hline
\textbf{Pool Size} & \textbf{DV 2} & \textbf{DV 3} & \textbf{DV 4} & \textbf{DV 5} & \textbf{Avg Boons/Roll} \\
\hline
2d10 & 22\% CS, 35\% Miss & 33\% Miss & 42\% Miss & 49\% Miss & 0.64-0.91 \\
3d10 & 12\% CS, 45\% Miss & 45\% Miss & 51\% Miss & 55\% Miss & 0.78-0.97 \\
4d10 & 6\% CS, 53\% Miss & 53\% Miss & 57\% Miss & 58.5\% Miss & 0.87-0.995 \\
\hline
\end{tabular}
\end{table}

\paragraph{Outcome Matrix (After the Roll)}
Let $S$ be successes (≥ 6) and $C$ be Complication Points (number of 1s rolled).
\begin{center}
\begin{tabular}{lll}
\toprule
\textbf{Case} & \textbf{Name} & \textbf{Guidance} \\
\midrule
$S \geq DV$ and $C = 0$ & Clean Success & Deliver the intent crisply. \\
$S \geq DV$ and $C > 0$ & Success \& Cost & Grant the intent; spend/bank CP for complications. \\
$0 < S < DV$ & Partial & Progress with a fork. \\
$S = 0$ & Miss & No progress. Cash/bank CP or offer Devil's Bargain. \\
\bottomrule
\end{tabular}
\end{center}

\paragraph{Complication Point (CP) Spend Menu}
\begin{itemize}
    \item \textbf{1 CP}: Minor pressure: noise, trace, +1 Supply segment.
    \item \textbf{2 CP}: Moderate setback: alarm raised, lose position/cover, lesser foe or lock.
    \item \textbf{3 CP}: Serious trouble: reinforcements, key gear breaks, rail tick.
    \item \textbf{4+ CP}: Major turn: trap springs, authority arrives, scene shifts.
\end{itemize}

\paragraph{Assistance, Boons, \& Description}
\begin{itemize}
    \item \textbf{Assists}: One helper per action; up to +3 dice.
    \item \textbf{Boons}: A player may re-roll one die after seeing the pool. Once per session, in downtime, you may convert 2 Boons → 1 XP (max 2 XP via conversion per session).
    \item \textbf{Description Ladder}: Basic (roll as-is), Detailed (re-roll one 1), Intricate (re-roll all 1s and add one flourish if successful).
\end{itemize}

\paragraph{Bond-Driven Resource Generation}
When a player takes a significant action to aid an ally with whom they share a bond, and explicitly references that bond in an intricate description, they may mark that bond to gain 1 boon after the action resolves.

\textbf{Requirements:}
\begin{itemize}
    \item \textbf{Mutual Bond:} Player shares a bond with the ally they're aiding
    \item \textbf{Intricate Description:} Player describes how their bond motivates their action
    \item \textbf{Significant Aid:} Meaningful assistance (not just +1 die)
    \item \textbf{Fiction First:} The bond must genuinely drive the choice to help
\end{itemize}

\textbf{Limitations:}
\begin{itemize}
    \item Once per bond per session
    \item Must be a meaningful sacrifice or risk
    \item GM approval for "significant action"
    \item Cannot be used for basic assistance rolls
\end{itemize}

\paragraph{Setting Stakes Fast (Cheat Prompts)}
\begin{itemize}
    \item If this goes right, what changes?
    \item If this goes wrong, what bites back?
\end{itemize}

\paragraph{Banking \& Cashing CP}
\begin{itemize}
    \item Banked CP should pay off within the same scene or arc.
    \item Avoid nickel-and-diming. Prefer one memorable complication over many petty penalties.
\end{itemize}

\subsection{Worked Micro-Examples}
\begin{itemize}
    \item \textbf{Lockpick Under Watch (DV 2)}: Player rolls 6 dice: 10, 8, 5, 4, 1, 1 ⇒ S=2, C=2. Success \& Cost. Door opens; GM spends 1 CP for a squeal (patrol starts moving) and banks 1 CP to bring that patrol around on the next beat.
    \item \textbf{Charm the Captain (DV 2)}: Player rolls 5 dice: 7, 6, 6, 2, 1 ⇒ S=3, C=1. Success \& Cost. Passage granted; GM spends 1 CP: "He expects a favor on the return leg—he'll collect."
    \item \textbf{Traverse the Pass (DV 3)}: Group roll pools to a net 3 successes but produces C=3. Success \& Cost. GM spends 2 CP to add Fatigue 1 to all from cold and exposure, banks 1 CP to crack a wagon axle next scene.
\end{itemize}

\section{Integrated Combat System}

\subsection{Core Philosophy}
Combat is violent conflict resolved through the standard consequence mechanics. Every combat action generates potential for both triumph and complication, with consequences that cascade through the same economy as all other challenges.

\subsection{Resolution Procedure}
\begin{enumerate}
    \item \textbf{Declare Action}: Player states intent and approach (Attribute + Skill)
    \item \textbf{Set Position}: GM sets Controlled, Risky, or Desperate based on tactical situation
    \item \textbf{Roll Dice}: Roll pool = Attribute + Skill
    \item \textbf{Count Results}: 6+ = Success, 1 = Complication Point (CP)
    \item \textbf{Apply Outcome}: Use standard Outcome Matrix
    \item \textbf{Manage Consequences}: GM spends CP or draws from Consequences Deck
\end{enumerate}

\subsection{Position States}
\begin{itemize}
    \item \textbf{Controlled}: Advantageous position, minor consequences
    \item \textbf{Risky}: Even odds, moderate consequences  
    \item \textbf{Desperate}: Disadvantaged, severe consequences
\end{itemize}

\subsection{Combat-Specific Consequence Types}
\begin{itemize}
    \item \textbf{Hearts}: Morale, fear, command/control breakdown
    \item \textbf{Spades}: Physical harm, positioning changes, weapon status
    \item \textbf{Clubs}: Resource depletion, gear damage, fatigue
    \item \textbf{Diamonds}: Environmental hazards, reinforcements, tactical setbacks
\end{itemize}

\subsection{Harm Integration}
Harm tracks directly tie to CP economy:
\begin{itemize}
    \item \textbf{Minor (-)}: Generate 1 CP on next 2 rolls
    \item \textbf{Moderate (=)}: Generate 1 CP on next roll, -1 die to relevant actions
    \item \textbf{Severe (≈)}: Generate 2 CP on next roll, -2 dice to relevant actions  
    \item \textbf{Critical (†)}: Generate 3 CP on next roll, out of action until treated
\end{itemize}

\subsection{Tactical Clocks}
Persistent combat conditions tracked through clocks:
\begin{itemize}
    \item \textbf{Mob Overwhelm} (6): Enemy numbers become advantage
    \item \textbf{Fatigue Spiral} (4): Exhaustion affects performance
    \item \textbf{Morale Collapse} (6): Fear undermines effectiveness
    \item \textbf{Environmental Collapse} (8): Terrain/fire/building failure
\end{itemize}

\subsection{Position Dynamics}
Position can shift during combat based on CP spending:
\begin{itemize}
    \item \textbf{1 CP}: Shift position one step (GM choice)
    \item \textbf{Player Spending}: 1 CP to improve position one step
    \item \textbf{Narrative Triggers}: Flanking, reinforcement arrival, environmental changes
\end{itemize}

\subsection{Magic Combat Integration}
Spellcasting in combat feeds the same consequence economy:
\begin{itemize}
    \item Channel/Weave Backlash CP applies to tactical situation
    \item Spells can shift position, create tactical clocks, or generate combat consequences
    \item Magic consequences cascade through existing combat systems
\end{itemize}

\subsection{Asset/Follower Combat Integration}
\begin{itemize}
    \item \textbf{Follower Risk}: 2+ CP spent in combat can endanger assisting followers
    \item \textbf{Asset Compromise}: Combat in certain locations can damage relevant assets  
    \item \textbf{Offensive Activation}: 1 Boon activates asset for combat advantage
    \item \textbf{Initiative Actions}: Followers can take combat-relevant independent actions
\end{itemize}

\subsection{Outcome Matrix Application}
Same as standard resolution, but consequences are combat-specific:
\begin{itemize}
    \item \textbf{Clean Success}: Intent achieved with no tactical complications
    \item \textbf{Success \& Cost}: Intent achieved, but GM spends CP for combat consequences
    \item \textbf{Partial}: Progress with tactical fork (accept cost OR concede ground)
    \item \textbf{Miss}: No progress; GM spends CP for combat consequences OR offers tactical bargain
\end{itemize}

\section{Advancement \& XP}

\subsection{Awarding XP}
\begin{itemize}
    \item \textbf{Gritty}: 4–6 XP per session (slow burn).
    \item \textbf{Standard}: 6–10 XP per session (default pace).
    \item \textbf{Heroic}: 10–14 XP per session (fast growth).
\end{itemize}

\subsubsection{Session Awards}
\begin{itemize}
    \item Table Attendance: +2 XP
    \item Major Objective Reached: +2–4 XP
    \item Discovery or Lore Unlocked: +1–2 XP
    \item Hard Choice Embraced: +1–2 XP
    \item Complication Spotlight: +1–3 XP
    \item Bond/Flag Driven Play: +1–2 XP
    \item GM Curveball Award: +0–3 XP
\end{itemize}

\subsubsection{Milestones}
\begin{itemize}
    \item +8–12 XP to all players at the conclusion of a major story arc.
    \item +2 XP bonus to one player for a signature moment of the arc.
\end{itemize}

\subsubsection{Complication Dividend}
\begin{itemize}
    \item Face Card (J/Q/K): +1 XP
    \item Ace: +2 XP
\end{itemize}

\subsection{Spending XP}
\begin{itemize}
    \item \textbf{Attributes}: Cost = new rating × 3. Downtime = new rating in days.
    \item \textbf{Skills}: Cost = new level × 2. Downtime = new level in days.
    \item \textbf{On-Screen Followers}: Cost = Cap². Downtime = 1–3 days to recruit and brief.
    \item \textbf{Off-Screen Assets}: Minor (4 XP, 1 day), Standard (8 XP, 1 week), Major (12 XP, 1 month).
\end{itemize}

\subsection{Rush Rule}
A player may skip downtime, but the GM creates a Haste clock of four segments. If the clock fills, the new ability or asset carries flaws or narrative complications.

\subsection{Tiers of Reputation}
\begin{itemize}
    \item \textbf{Tier I – Rookie} (0–40 XP): Local reputation; prestige locked.
    \item \textbf{Tier II – Seasoned} (41–90): Regional notice; prestige abilities may be unlocked.
    \item \textbf{Tier III – Veteran} (91–150): National influence; second follower slot suggested.
    \item \textbf{Tier IV – Paragon} (151–220): Movers and shakers; rivals emerge to challenge.
    \item \textbf{Tier V – Mythic} (221+): Legendary status; kingdoms and cults respond.
\end{itemize}

\section{Character Creation}
\subsection{Starting Build Points}
Players begin with 30 Experience Points (XP) to allocate during initial character creation. This represents a balanced baseline for competent starting characters.

\subsection{Enhanced Starting Builds}
Players may exceed the standard 30 XP build through narrative engagement:

\begin{itemize}
    \item \textbf{Bonds:} Up to two player-defined mutual bonds may be taken for +2 XP total (Section \ref{sessionawards})
    \item \textbf{Complications:} Up to two initial complications may be accepted for +4 XP total (Section \ref{sessionawards}) \textit{NOTE: Scenes start with +CP per complication per character until cleared.}
\end{itemize}

This allows for a maximum starting build of 34 XP, though players are encouraged to aim for 30 XP and use bonds/complications to mitigate slight overages while maintaining narrative balance.

\subsection{Recommended Approach}
The GM should encourage players to:
\begin{itemize}
    \item Target 30 XP for balanced starting characters
    \item Use bonds and complications to enhance characterization rather than pure mechanical optimization
    \item Consider the narrative implications of any starting advantages
\end{itemize}

\subsection*{Initial Complications}
\textbf{Per Complication:}
\begin{itemize}
    \item Start each scene  with +1 banked CP per character with  initial complications until those complications have cleared.
\end{itemize}

\section{Complication Point Management}

The GM should manage Complication Point (CP) spending to maintain dramatic tension while preserving player agency and game flow. CP spending scales with character tier but is subject to hard limits to ensure playability.

\subsection{Core Principles}
\begin{itemize}
    \item \textbf{Narrative Coherence}: All CP spends within a scene should connect thematically
    \item \textbf{Player Agency}: Complications create interesting choices, not insurmountable obstacles
    \item \textbf{Progressive Escalation}: Higher tier characters naturally attract greater consequences
    \item \textbf{Resolution Paths}: Every complication thread should have potential resolution
\end{itemize}

\subsection{Spending Formula}
\textbf{Base CP = 4 + Character Tier}
\begin{itemize}
    \item \textbf{Tier I (Rookie)}: 5 CP base
    \item \textbf{Tier II (Seasoned)}: 6 CP base
    \item \textbf{Tier III (Veteran)}: 7 CP base
    \item \textbf{Tier IV (Paragon)}: 8 CP base
    \item \textbf{Tier V (Mythic)}: 9 CP base
\end{itemize}

\subsection{Hard Limits}
\begin{itemize}
    \item \textbf{Standard Scenes}: Maximum 12 CP spending
    \item \textbf{Climactic Scenes}: Maximum 16 CP spending
    \item \textbf{Active Threads}: Maximum (Tier + 1) concurrent threads
    \item \textbf{Session Budget}: Maximum 20 CP total per session
\end{itemize}

\subsection{Banked CP Integration}
Banked CP from character complications count toward scene spending limits rather than adding to available CP. This prevents exponential complication stacking while honoring narrative debt.

\subsection{Thread Management}
Complication threads follow a natural escalation pattern:
\begin{itemize}
    \item \textbf{First Exposure}: 1-2 CP (Minor inconvenience)
    \item \textbf{Second Occurrence}: 2-4 CP (Moderate setback)
    \item \textbf{Third Strike}: 3-6 CP (Major consequence)
    \item \textbf{Resolution}: Thread concludes with narrative payoff
\end{itemize}

\section{Rules Clarifications}

\subsection{Follower Assist}
\begin{itemize}
    \item Assist dice come from the helper, not the leader.
    \item Total Assist on any roll (from any sources) remains hard-capped at +3. Exception: The "Exceptional Coordination" Talent allows one follower to provide +4 assist dice.
\end{itemize}

\subsection{Boon Economy}
\begin{itemize}
    \item Holding cap: You can hold at most 5 Boons.
    \item Conversion: Once per session, in downtime, you may convert 2 Boons → 1 XP (max 2 XP via conversion per session).
\end{itemize}

\paragraph{Bond-Driven Generation}
Players may earn boons through bond-driven resource generation (see Section 2.1.5) by taking significant actions to aid bonded allies with intricate descriptions of their bond-motivated actions.

\subsection{Asset Activations}
\begin{itemize}
    \item Off-Screen effects: Use each Asset's listed Off-Screen effect once per session for free.
    \item On-Screen activations: To reshape the current scene, spend 1 Boon.
    \item Plausibility test: The Asset must have scope and reach.
\end{itemize}

\subsection{Over-Stack}
\begin{itemize}
    \item Structural advantages: active buff/tag, favorable venue/pennant, Follower Initiative unused, on-screen Asset activation, opponent disadvantaged by fiction, ritual prep that applies now.
    \item Trigger: If the crew enters a scene with ≥ 3 structural advantages, apply Over-Stack once for that scene: either start one named rail at +1 or the GM banks +1 CP for the first Deck Twist.
\end{itemize}

\subsection{Familiar Bond}
\begin{itemize}
    \item Familiars use the standard Follower Exposure/Harm tracks and require no upkeep.
    \item Each time a familiar acts on-screen in a high-risk beat, mark Exposure +1 on the familiar after the second such beat this scene.
\end{itemize}

\subsection{Ritual Casting}
\begin{itemize}
    \item Helper cap: Maximum simultaneous helpers = $\lceil$primary caster's Ritual/Arcana/2$\rceil$, max 3.
    \item Relevant skills: Helpers may use different relevant skills if their procedure is fictionally distinct.
    \item CP distribution: CP from Channel resolves on that roller. CP from Weave is assigned to the primary caster.
\end{itemize}

\subsection{Deck of Consequences}

\subsection{Two Deck Systems (Compatibility)}
Fate's Edge uses two distinct card tools:

\paragraph{Travel Decks (regional, 52-card).}
\emph{Spade}=Place, \emph{Heart}=Actor, \emph{Club}=Pressure, \emph{Diamond}=Leverage. These power journeys and gates.

\paragraph{Deck of Consequences (scene drama).}
\emph{Hearts}=emotional/social fallout, \emph{Spades}=harm/escalation, \emph{Clubs}=material cost, \emph{Diamonds}=magical/spiritual disturbance.

\textit{Guidance:} Never mix suit meanings across decks. When a rule references ``Spade/Club/Diamond,'' it means \emph{Travel}. When it says ``Hearts/Spades/Clubs/Diamonds,'' it means \emph{Consequences}.

\subsection{Structure of the Deck}
\begin{itemize}
    \item \textbf{Suits} = Domains of Complications
    \begin{itemize}
        \item Hearts: Emotional, social, or relational fallout.
        \item Spades: Harm, danger, or escalation of conflict.
        \item Clubs: Resource strain, economic or material cost.
        \item Diamonds: Magical, spiritual, or cosmic disturbances.
    \end{itemize}
    \item \textbf{Ranks} = Severity of Complications
    \begin{itemize}
        \item Ace–3: Minor inconvenience or flavor complication.
        \item 4–6: Moderate setback with some narrative teeth.
        \item 7–9: Significant consequence altering the course of action.
        \item 10–King: Major fallout, introducing new problems or lasting scars.
    \end{itemize}
\end{itemize}

\subsection{Using the Deck}
\begin{enumerate}
  \item Player rolls; each 1 generates a Complication Point (CP).
  \item GM chooses one method for that roll:
  \begin{enumerate}
    \item \textbf{Direct Spend}: translate CP into immediate consequences/clock ticks; or
    \item \textbf{Deck Draw}: draw up to \textbf{min(CP, 3)} cards and \textbf{synthesize a single twist}
    guided by suit and highest rank.
  \end{enumerate}
\end{enumerate}

\section{Player Archetypes at the Table}

\subsection{The Solo}
\begin{itemize}
    \item Invests XP primarily in Attributes and Skills.
    \item Strengths: always ready, iconic spotlight.
    \item Risks: narrow toolkit; may lag in social or resource scenes.
\end{itemize}

\subsection{The Mixed Player}
\begin{itemize}
    \item Balances XP between self and assets.
    \item Strengths: adaptable, bridges party gaps.
    \item Risks: upkeep spread thin.
\end{itemize}

\subsection{The Mastermind}
\begin{itemize}
    \item Builds networks, followers, and assets.
    \item Strengths: broad reach, drives strategies.
    \item Risks: Complication fallout; vulnerable allies.
\end{itemize}

\section{Campaign Frame / Finale: The Crown Spread}

\subsection{Session 0: The Crown Spread (Initial Draw)}
Draw 5 cards: Spade, Heart, Club, Diamond, and a Wild (any suit; reveal last).

\subsection{The Campaign Clock}
Track two dials over the campaign:
\begin{itemize}
    \item \textbf{Mandate (0–6)}: The table's public legitimacy and buy-in.
    \item \textbf{Crisis (0–6)}: The opposition engine (rivals, pressure rails, attrition).
\end{itemize}

\subsection{Finale Procedure (Crown Beat)}
Use the Session 0 Crown Spread to seed setup; then run the three-beat crown.

\subsection{Legacy Conversion (Epilogue)}
After the Finale, each PC draws 2 cards and answers epilogue prompts by suit.

\section{Travel Framework}

\subsection{Core Travel Procedure}
For each leg of a journey, draw 3–4 cards using the decks for your destination and controlling authority.
\begin{itemize}
    \item Spade from the destination deck: sets the scene (place).
    \item Heart from the destination deck: introduces the local actor or faction.
    \item Club from the Wilds (general hazards) or destination (if strongly policed): brings pressure.
    \item Diamond from the authority that gates the route: papers, escorts, rights, or exceptions.
\end{itemize}

Set a travel clock by the highest rank (2–5⇒4 • 6–10⇒6 • J/Q/K⇒8 • A⇒10). On success, advance to the next leg; on failure, mark delay, debt, or diversion and resolve a consequence in the fiction.

\subsection{Mode rules}
\begin{itemize}
    \item \textbf{Sea legs} (Amaranthine/Dolmis/Aberderrin): If Theona or Valewood 9s show up anywhere in the seed, add an omission or taboo to the leg.
    \item \textbf{Aeler Aces and Valewood Corridors}: Any A means wood actively rearranges paths or wakes structures.
    \item \textbf{Rivers}: Bridges, booms, and law in Ecktoria/Viterra; reed-mazes and bell-lines in Mistlands/Linn waters.
    \item \textbf{Frontier blends}: When origin and destination disagree on law, draw two Diamonds (one from each law) and choose which you will be judged by at the end of the leg.
\end{itemize}

\subsection{Route Modules}
\subsubsection{Amaranthine Coastway}
Kahfagia → Ecktoria → Acasia → Marcott (Vhasia) → Fairport (Viterra).

\subsubsection{Astroegro Straits}
Thepyrgos controls the hinge between seas.

\subsubsection{Dolmis Circuits}
Fairport (Viterra) → Theona (Three Greens) → Ubral fjords → Aelinnel west shore.

\subsubsection{Aelerian Passes Underways}
Vhasia/Viterra/Ubral south slopes → Aeler gates → Mistlands.

\subsubsection{Shadow Corridors}
Thin Shore (Valewood east coast): risky misted corridor north–south toward Zakov.

\subsubsection{River Roads}
Belworth: forms the boundary between Vhasia and Viterra.

\subsubsection{Steppe Frontiers (Violet Steppes \& Meadows)}
Ykrul ↔ Vilikari ↔ Ecktoria/Acasia borders.

\section{Design Philosophy Guardrails}

\subsection{Core Principles}
\begin{enumerate}
    \item \textbf{Narrative Primacy}: Mechanics serve story, not replace it.
    \item \textbf{Risk as Drama}: Every roll carries potential for triumph + complication.
    \item \textbf{Meaningful Growth}: XP investment creates lasting character/world change.
    \item \textbf{Consequence Weight}: Choices ripple outward, nothing is free.
\end{enumerate}

\subsection{Mechanical Constraints}
\begin{itemize}
    \item \textbf{ASSIST MAX}: +3 dice total per roll, regardless of helpers. Exception: The "Exceptional Coordination" Talent allows one follower to provide +4 assist dice.
    \item \textbf{BOON MAX}: 5 total, 2→1 XP conversion once/session (max 2 XP via conversion per session).
    \item \textbf{INITIATIVE}: 1 Follower Action per scene crew-wide.
    \item \textbf{OVER-STACK}: 2+ structural advantages = start rails +1 OR GM banks +1 CP.
    \item \textbf{POSITION}: Controlled | Risky | Desperate (affects success/failure texture).
\end{itemize}

\paragraph{High-Tier CP Sinks.}
For 3–6+ CP spends that move the world (reputation cascades, faction instability, resonance, prophecy), see the stand-alone \emph{High CP Sinks} handout. A good default: at end of leg, \textbf{3 CP → tick 1 Front}.

\subsection{Balance Philosophy}
\begin{itemize}
    \item Quadratic follower costs ensure high follower investments are intentionally expensive for mechanical balance.
    \item Risk-reward equilibrium maintains that dangerous magic prevents caster dominance while preserving narrative impact.
    \item Viable approaches across all character builds are supported through balanced mechanics.
\end{itemize}

\subsection{Progression Clarity}
\begin{itemize}
    \item Attribute cost diminishing returns encourage diversification.
    \item Skill mastery benefits provide meaningful advancement.
    \item Prestige ability considerations include scaling options with additional XP investment.
\end{itemize}

\end{document}

