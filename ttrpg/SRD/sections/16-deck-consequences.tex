
% --- Fate's Edge SRD — Section 16: Deck of Consequences ---
% Include this file from your main .tex with: 
% --- Fate's Edge SRD — Section 16: Deck of Consequences ---
% Include this file from your main .tex with: 
% --- Fate's Edge SRD — Section 16: Deck of Consequences ---
% Include this file from your main .tex with: 
% --- Fate's Edge SRD — Section 16: Deck of Consequences ---
% Include this file from your main .tex with: \input{16-deck-consequences.tex}

\section{Deck of Consequences}
\label{sec:deck-consequences}

The \textbf{Deck of Consequences} is a narrative tool for injecting drama, setbacks, and twists. It replaces or supplements GM fiat with randomized complications that remain thematically consistent.

\subsection{Deck Structure}
\begin{itemize}
  \item Use a standard 52-card deck (jokers optional).
  \item Divide into four suits, each tied to a \textbf{theme of complication}.
\end{itemize}

\begin{center}
\begin{tabular}{@{}llp{8cm}@{}}
\toprule
\textbf{Suit} & \textbf{Theme} & \textbf{Examples} \\
\midrule
Hearts & Social / Emotional & Betrayal, love triangle, family drama, ally under strain. \\
Clubs & Physical / Violent  & Ambush, wound, fatigue, weapon break. \\
Diamonds & Resources / Wealth & Supplies run low, theft, loss of funding, gear breaks. \\
Spades & Mystical / Supernatural & Omen, curse, patron demand, haunting. \\
\bottomrule
\end{tabular}
\end{center}

\subsection{Card Ranks \& Severity}
\begin{description}[leftmargin=1.5em, style=nextline]
  \item[Ace:] Scene-altering twist; compels immediate response.
  \item[King/Queen/Jack:] Major complication with lasting effects.
  \item[10--8:] Moderate complication that reshapes current scene.
  \item[7--5:] Minor complication; nuisance, but creates tension.
  \item[4--2:] Subtle complication or foreshadowing omen.
\end{description}

\subsection{Jokers (Optional)}
\begin{itemize}
  \item Red Joker: Catastrophic event (environmental collapse, patron intervention).
  \item Black Joker: Dark boon (immediate help, but with lasting cost or debt).
\end{itemize}

\subsection{Usage in Play}
\begin{itemize}
  \item \textbf{Trigger:} GM may draw when a roll shows multiple 1s, when CP overflows, or during travel (see \S\ref{sec:travel}).
  \item \textbf{Cadence:} Aim for 1--2 draws per session, more if the tone skews chaotic.
  \item \textbf{Integration:} Complications should align with fiction already present; do not derail core arcs.
\end{itemize}

\subsection{Crown Spread Integration}
Use the Crown Spread (see \S18) to seed campaign-scale twists. Draw 5--7 cards in Session 0 to foreshadow long-term arcs.

\subsection{Campaign Clock Tie-In}
When the Campaign Clock advances, the GM may flip a card face-up from the Deck of Consequences to signal how pressure is mounting.

\subsection{GM Quick Cues}
\begin{itemize}
  \item Translate raw card results into fiction, not mechanical penalties alone.
  \item Complications should build on what’s already happening, not restart the story.
  \item Respect player agency: allow clever mitigation, but ensure consequences land.
\end{itemize}


\section{Deck of Consequences}
\label{sec:deck-consequences}

The \textbf{Deck of Consequences} is a narrative tool for injecting drama, setbacks, and twists. It replaces or supplements GM fiat with randomized complications that remain thematically consistent.

\subsection{Deck Structure}
\begin{itemize}
  \item Use a standard 52-card deck (jokers optional).
  \item Divide into four suits, each tied to a \textbf{theme of complication}.
\end{itemize}

\begin{center}
\begin{tabular}{@{}llp{8cm}@{}}
\toprule
\textbf{Suit} & \textbf{Theme} & \textbf{Examples} \\
\midrule
Hearts & Social / Emotional & Betrayal, love triangle, family drama, ally under strain. \\
Clubs & Physical / Violent  & Ambush, wound, fatigue, weapon break. \\
Diamonds & Resources / Wealth & Supplies run low, theft, loss of funding, gear breaks. \\
Spades & Mystical / Supernatural & Omen, curse, patron demand, haunting. \\
\bottomrule
\end{tabular}
\end{center}

\subsection{Card Ranks \& Severity}
\begin{description}[leftmargin=1.5em, style=nextline]
  \item[Ace:] Scene-altering twist; compels immediate response.
  \item[King/Queen/Jack:] Major complication with lasting effects.
  \item[10--8:] Moderate complication that reshapes current scene.
  \item[7--5:] Minor complication; nuisance, but creates tension.
  \item[4--2:] Subtle complication or foreshadowing omen.
\end{description}

\subsection{Jokers (Optional)}
\begin{itemize}
  \item Red Joker: Catastrophic event (environmental collapse, patron intervention).
  \item Black Joker: Dark boon (immediate help, but with lasting cost or debt).
\end{itemize}

\subsection{Usage in Play}
\begin{itemize}
  \item \textbf{Trigger:} GM may draw when a roll shows multiple 1s, when CP overflows, or during travel (see \S\ref{sec:travel}).
  \item \textbf{Cadence:} Aim for 1--2 draws per session, more if the tone skews chaotic.
  \item \textbf{Integration:} Complications should align with fiction already present; do not derail core arcs.
\end{itemize}

\subsection{Crown Spread Integration}
Use the Crown Spread (see \S18) to seed campaign-scale twists. Draw 5--7 cards in Session 0 to foreshadow long-term arcs.

\subsection{Campaign Clock Tie-In}
When the Campaign Clock advances, the GM may flip a card face-up from the Deck of Consequences to signal how pressure is mounting.

\subsection{GM Quick Cues}
\begin{itemize}
  \item Translate raw card results into fiction, not mechanical penalties alone.
  \item Complications should build on what’s already happening, not restart the story.
  \item Respect player agency: allow clever mitigation, but ensure consequences land.
\end{itemize}


\section{Deck of Consequences}
\label{sec:deck-consequences}

The \textbf{Deck of Consequences} is a narrative tool for injecting drama, setbacks, and twists. It replaces or supplements GM fiat with randomized complications that remain thematically consistent.

\subsection{Deck Structure}
\begin{itemize}
  \item Use a standard 52-card deck (jokers optional).
  \item Divide into four suits, each tied to a \textbf{theme of complication}.
\end{itemize}

\begin{center}
\begin{tabular}{@{}llp{8cm}@{}}
\toprule
\textbf{Suit} & \textbf{Theme} & \textbf{Examples} \\
\midrule
Hearts & Social / Emotional & Betrayal, love triangle, family drama, ally under strain. \\
Clubs & Physical / Violent  & Ambush, wound, fatigue, weapon break. \\
Diamonds & Resources / Wealth & Supplies run low, theft, loss of funding, gear breaks. \\
Spades & Mystical / Supernatural & Omen, curse, patron demand, haunting. \\
\bottomrule
\end{tabular}
\end{center}

\subsection{Card Ranks \& Severity}
\begin{description}[leftmargin=1.5em, style=nextline]
  \item[Ace:] Scene-altering twist; compels immediate response.
  \item[King/Queen/Jack:] Major complication with lasting effects.
  \item[10--8:] Moderate complication that reshapes current scene.
  \item[7--5:] Minor complication; nuisance, but creates tension.
  \item[4--2:] Subtle complication or foreshadowing omen.
\end{description}

\subsection{Jokers (Optional)}
\begin{itemize}
  \item Red Joker: Catastrophic event (environmental collapse, patron intervention).
  \item Black Joker: Dark boon (immediate help, but with lasting cost or debt).
\end{itemize}

\subsection{Usage in Play}
\begin{itemize}
  \item \textbf{Trigger:} GM may draw when a roll shows multiple 1s, when CP overflows, or during travel (see \S\ref{sec:travel}).
  \item \textbf{Cadence:} Aim for 1--2 draws per session, more if the tone skews chaotic.
  \item \textbf{Integration:} Complications should align with fiction already present; do not derail core arcs.
\end{itemize}

\subsection{Crown Spread Integration}
Use the Crown Spread (see \S18) to seed campaign-scale twists. Draw 5--7 cards in Session 0 to foreshadow long-term arcs.

\subsection{Campaign Clock Tie-In}
When the Campaign Clock advances, the GM may flip a card face-up from the Deck of Consequences to signal how pressure is mounting.

\subsection{GM Quick Cues}
\begin{itemize}
  \item Translate raw card results into fiction, not mechanical penalties alone.
  \item Complications should build on what’s already happening, not restart the story.
  \item Respect player agency: allow clever mitigation, but ensure consequences land.
\end{itemize}


\section{Deck of Consequences}
\label{sec:deck-consequences}

The \textbf{Deck of Consequences} is a narrative tool for injecting drama, setbacks, and twists. It replaces or supplements GM fiat with randomized complications that remain thematically consistent.

\subsection{Deck Structure}
\begin{itemize}
  \item Use a standard 52-card deck (jokers optional).
  \item Divide into four suits, each tied to a \textbf{theme of complication}.
\end{itemize}

\begin{center}
\begin{tabular}{@{}llp{8cm}@{}}
\toprule
\textbf{Suit} & \textbf{Theme} & \textbf{Examples} \\
\midrule
Hearts & Social / Emotional & Betrayal, love triangle, family drama, ally under strain. \\
Clubs & Physical / Violent  & Ambush, wound, fatigue, weapon break. \\
Diamonds & Resources / Wealth & Supplies run low, theft, loss of funding, gear breaks. \\
Spades & Mystical / Supernatural & Omen, curse, patron demand, haunting. \\
\bottomrule
\end{tabular}
\end{center}

\subsection{Card Ranks \& Severity}
\begin{description}[leftmargin=1.5em, style=nextline]
  \item[Ace:] Scene-altering twist; compels immediate response.
  \item[King/Queen/Jack:] Major complication with lasting effects.
  \item[10--8:] Moderate complication that reshapes current scene.
  \item[7--5:] Minor complication; nuisance, but creates tension.
  \item[4--2:] Subtle complication or foreshadowing omen.
\end{description}

\subsection{Jokers (Optional)}
\begin{itemize}
  \item Red Joker: Catastrophic event (environmental collapse, patron intervention).
  \item Black Joker: Dark boon (immediate help, but with lasting cost or debt).
\end{itemize}

\subsection{Usage in Play}
\begin{itemize}
  \item \textbf{Trigger:} GM may draw when a roll shows multiple 1s, when CP overflows, or during travel (see \S\ref{sec:travel}).
  \item \textbf{Cadence:} Aim for 1--2 draws per session, more if the tone skews chaotic.
  \item \textbf{Integration:} Complications should align with fiction already present; do not derail core arcs.
\end{itemize}

\subsection{Crown Spread Integration}
Use the Crown Spread (see \S18) to seed campaign-scale twists. Draw 5--7 cards in Session 0 to foreshadow long-term arcs.

\subsection{Campaign Clock Tie-In}
When the Campaign Clock advances, the GM may flip a card face-up from the Deck of Consequences to signal how pressure is mounting.

\subsection{GM Quick Cues}
\begin{itemize}
  \item Translate raw card results into fiction, not mechanical penalties alone.
  \item Complications should build on what’s already happening, not restart the story.
  \item Respect player agency: allow clever mitigation, but ensure consequences land.
\end{itemize}
