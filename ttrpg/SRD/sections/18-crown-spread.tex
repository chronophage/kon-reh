
% --- Fate's Edge SRD — Section 18: Campaign Frame: The Crown Spread ---
% Include this file from your main .tex with: 
% --- Fate's Edge SRD — Section 18: Campaign Frame: The Crown Spread ---
% Include this file from your main .tex with: 
% --- Fate's Edge SRD — Section 18: Campaign Frame: The Crown Spread ---
% Include this file from your main .tex with: 
% --- Fate's Edge SRD — Section 18: Campaign Frame: The Crown Spread ---
% Include this file from your main .tex with: \input{18-crown-spread.tex}

\section{Campaign Frame: The Crown Spread}
\label{sec:crown-spread}

The \textbf{Crown Spread} is a campaign-framing tool that uses a spread of cards to establish the long arc of a story. It provides seeds for GMs and players alike to weave motifs, omens, and foreshadowed events.

\subsection{Setup}
\begin{itemize}
  \item In Session 0, lay out 5--7 cards in a semicircle (the “Crown”). Use either the \textbf{Deck of Consequences} or a standard card deck.
  \item Each card anchors a motif, omen, or looming event.
  \item Record the spread openly on a Campaign Sheet or digital log.
\end{itemize}

\subsection{Interpreting the Spread}
\begin{description}[leftmargin=1.5em, style=nextline]
  \item[Position 1 (Root):] The underlying tension or theme of the campaign. 
  \item[Position 2 (Crest):] A key faction or patron influence that will rise. 
  \item[Position 3 (Crown):] The climax image or major confrontation the arc builds toward. 
  \item[Position 4 (Left Hand):] A bond, ally, or relationship that anchors play. 
  \item[Position 5 (Right Hand):] A rival, betrayer, or challenger who pressures the party. 
  \item[Optional 6+7:] Expansions for setting-wide twists (environmental, mystical, or political).
\end{description}

\subsection{Using the Spread in Play}
\begin{itemize}
  \item Each drawn card becomes a \textbf{Foreshadow Clock [4]} attached to its motif. Advance the clock when events lean toward that omen.
  \item When a Foreshadow Clock fills, the motif manifests concretely in play (e.g., a faction rises, a betrayer reveals themselves).
  \item Tie Spread cards to \textbf{Campaign Clocks} for pacing (see below).
\end{itemize}

\subsection{Campaign Clock}
The \textbf{Campaign Clock} tracks rising stakes across the arc.
\begin{itemize}
  \item Default size: [8].
  \item Advance the Campaign Clock when: multiple SB overflows in a session, when travel legs resolve with major cost, or when Spread omens manifest.
  \item On fill: the Crown confrontation arrives. Play through its fallout as campaign climax.
\end{itemize}

\subsection{Ending \& Legacy}
\begin{itemize}
  \item After the Crown confrontation resolves, hold an epilogue session.
  \item Resolve any remaining Foreshadow Clocks as epilogue vignettes.
  \item Players may mark \textbf{Legacy Bonds}—new anchors for future campaigns or descendants.
\end{itemize}

\subsection{GM Quick Cues}
\begin{itemize}
  \item The Spread is not a railroad—it foreshadows, not dictates.
  \item Reinterpret cards liberally as play evolves; symbols matter more than literal events.
  \item Remind players of their omens between arcs to build tension and payoff.
\end{itemize}


\section{Campaign Frame: The Crown Spread}
\label{sec:crown-spread}

The \textbf{Crown Spread} is a campaign-framing tool that uses a spread of cards to establish the long arc of a story. It provides seeds for GMs and players alike to weave motifs, omens, and foreshadowed events.

\subsection{Setup}
\begin{itemize}
  \item In Session 0, lay out 5--7 cards in a semicircle (the “Crown”). Use either the \textbf{Deck of Consequences} or a standard card deck.
  \item Each card anchors a motif, omen, or looming event.
  \item Record the spread openly on a Campaign Sheet or digital log.
\end{itemize}

\subsection{Interpreting the Spread}
\begin{description}[leftmargin=1.5em, style=nextline]
  \item[Position 1 (Root):] The underlying tension or theme of the campaign. 
  \item[Position 2 (Crest):] A key faction or patron influence that will rise. 
  \item[Position 3 (Crown):] The climax image or major confrontation the arc builds toward. 
  \item[Position 4 (Left Hand):] A bond, ally, or relationship that anchors play. 
  \item[Position 5 (Right Hand):] A rival, betrayer, or challenger who pressures the party. 
  \item[Optional 6+7:] Expansions for setting-wide twists (environmental, mystical, or political).
\end{description}

\subsection{Using the Spread in Play}
\begin{itemize}
  \item Each drawn card becomes a \textbf{Foreshadow Clock [4]} attached to its motif. Advance the clock when events lean toward that omen.
  \item When a Foreshadow Clock fills, the motif manifests concretely in play (e.g., a faction rises, a betrayer reveals themselves).
  \item Tie Spread cards to \textbf{Campaign Clocks} for pacing (see below).
\end{itemize}

\subsection{Campaign Clock}
The \textbf{Campaign Clock} tracks rising stakes across the arc.
\begin{itemize}
  \item Default size: [8].
  \item Advance the Campaign Clock when: multiple SB overflows in a session, when travel legs resolve with major cost, or when Spread omens manifest.
  \item On fill: the Crown confrontation arrives. Play through its fallout as campaign climax.
\end{itemize}

\subsection{Ending \& Legacy}
\begin{itemize}
  \item After the Crown confrontation resolves, hold an epilogue session.
  \item Resolve any remaining Foreshadow Clocks as epilogue vignettes.
  \item Players may mark \textbf{Legacy Bonds}—new anchors for future campaigns or descendants.
\end{itemize}

\subsection{GM Quick Cues}
\begin{itemize}
  \item The Spread is not a railroad—it foreshadows, not dictates.
  \item Reinterpret cards liberally as play evolves; symbols matter more than literal events.
  \item Remind players of their omens between arcs to build tension and payoff.
\end{itemize}


\section{Campaign Frame: The Crown Spread}
\label{sec:crown-spread}

The \textbf{Crown Spread} is a campaign-framing tool that uses a spread of cards to establish the long arc of a story. It provides seeds for GMs and players alike to weave motifs, omens, and foreshadowed events.

\subsection{Setup}
\begin{itemize}
  \item In Session 0, lay out 5--7 cards in a semicircle (the “Crown”). Use either the \textbf{Deck of Consequences} or a standard card deck.
  \item Each card anchors a motif, omen, or looming event.
  \item Record the spread openly on a Campaign Sheet or digital log.
\end{itemize}

\subsection{Interpreting the Spread}
\begin{description}[leftmargin=1.5em, style=nextline]
  \item[Position 1 (Root):] The underlying tension or theme of the campaign. 
  \item[Position 2 (Crest):] A key faction or patron influence that will rise. 
  \item[Position 3 (Crown):] The climax image or major confrontation the arc builds toward. 
  \item[Position 4 (Left Hand):] A bond, ally, or relationship that anchors play. 
  \item[Position 5 (Right Hand):] A rival, betrayer, or challenger who pressures the party. 
  \item[Optional 6+7:] Expansions for setting-wide twists (environmental, mystical, or political).
\end{description}

\subsection{Using the Spread in Play}
\begin{itemize}
  \item Each drawn card becomes a \textbf{Foreshadow Clock [4]} attached to its motif. Advance the clock when events lean toward that omen.
  \item When a Foreshadow Clock fills, the motif manifests concretely in play (e.g., a faction rises, a betrayer reveals themselves).
  \item Tie Spread cards to \textbf{Campaign Clocks} for pacing (see below).
\end{itemize}

\subsection{Campaign Clock}
The \textbf{Campaign Clock} tracks rising stakes across the arc.
\begin{itemize}
  \item Default size: [8].
  \item Advance the Campaign Clock when: multiple SB overflows in a session, when travel legs resolve with major cost, or when Spread omens manifest.
  \item On fill: the Crown confrontation arrives. Play through its fallout as campaign climax.
\end{itemize}

\subsection{Ending \& Legacy}
\begin{itemize}
  \item After the Crown confrontation resolves, hold an epilogue session.
  \item Resolve any remaining Foreshadow Clocks as epilogue vignettes.
  \item Players may mark \textbf{Legacy Bonds}—new anchors for future campaigns or descendants.
\end{itemize}

\subsection{GM Quick Cues}
\begin{itemize}
  \item The Spread is not a railroad—it foreshadows, not dictates.
  \item Reinterpret cards liberally as play evolves; symbols matter more than literal events.
  \item Remind players of their omens between arcs to build tension and payoff.
\end{itemize}


\section{Campaign Frame: The Crown Spread}
\label{sec:crown-spread}

The \textbf{Crown Spread} is a campaign-framing tool that uses a spread of cards to establish the long arc of a story. It provides seeds for GMs and players alike to weave motifs, omens, and foreshadowed events.

\subsection{Setup}
\begin{itemize}
  \item In Session 0, lay out 5--7 cards in a semicircle (the “Crown”). Use either the \textbf{Deck of Consequences} or a standard card deck.
  \item Each card anchors a motif, omen, or looming event.
  \item Record the spread openly on a Campaign Sheet or digital log.
\end{itemize}

\subsection{Interpreting the Spread}
\begin{description}[leftmargin=1.5em, style=nextline]
  \item[Position 1 (Root):] The underlying tension or theme of the campaign. 
  \item[Position 2 (Crest):] A key faction or patron influence that will rise. 
  \item[Position 3 (Crown):] The climax image or major confrontation the arc builds toward. 
  \item[Position 4 (Left Hand):] A bond, ally, or relationship that anchors play. 
  \item[Position 5 (Right Hand):] A rival, betrayer, or challenger who pressures the party. 
  \item[Optional 6+7:] Expansions for setting-wide twists (environmental, mystical, or political).
\end{description}

\subsection{Using the Spread in Play}
\begin{itemize}
  \item Each drawn card becomes a \textbf{Foreshadow Clock [4]} attached to its motif. Advance the clock when events lean toward that omen.
  \item When a Foreshadow Clock fills, the motif manifests concretely in play (e.g., a faction rises, a betrayer reveals themselves).
  \item Tie Spread cards to \textbf{Campaign Clocks} for pacing (see below).
\end{itemize}

\subsection{Campaign Clock}
The \textbf{Campaign Clock} tracks rising stakes across the arc.
\begin{itemize}
  \item Default size: [8].
  \item Advance the Campaign Clock when: multiple CP overflows in a session, when travel legs resolve with major cost, or when Spread omens manifest.
  \item On fill: the Crown confrontation arrives. Play through its fallout as campaign climax.
\end{itemize}

\subsection{Ending \& Legacy}
\begin{itemize}
  \item After the Crown confrontation resolves, hold an epilogue session.
  \item Resolve any remaining Foreshadow Clocks as epilogue vignettes.
  \item Players may mark \textbf{Legacy Bonds}—new anchors for future campaigns or descendants.
\end{itemize}

\subsection{GM Quick Cues}
\begin{itemize}
  \item The Spread is not a railroad—it foreshadows, not dictates.
  \item Reinterpret cards liberally as play evolves; symbols matter more than literal events.
  \item Remind players of their omens between arcs to build tension and payoff.
\end{itemize}
