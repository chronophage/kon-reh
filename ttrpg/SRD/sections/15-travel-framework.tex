
% --- Fate's Edge SRD — Section 15: Travel Framework ---
% Include this file from your main .tex with: 
% --- Fate's Edge SRD — Section 15: Travel Framework ---
% Include this file from your main .tex with: 
% --- Fate's Edge SRD — Section 15: Travel Framework ---
% Include this file from your main .tex with: 
% --- Fate's Edge SRD — Section 15: Travel Framework ---
% Include this file from your main .tex with: \input{15-travel-framework.tex}

\section{Travel Framework}
\label{sec:travel}

Adventuring often means crossing dangerous or unknown lands. The travel framework provides pacing tools to turn journeys into dramatic scenes without micromanaging miles.

\subsection{Legs and Clocks}
\begin{itemize}
  \item Break long journeys into \textbf{legs}, each representing a significant change in terrain, danger, or political region.
  \item Each leg is tracked with a \textbf{Travel Clock [4]} by default.
  \item A leg completes when the clock fills through \textbf{progress actions, encounters, or scene resolution}.
\end{itemize}

\subsection{Roles on the Road}
Assign roles each travel leg (rotate or repeat as desired):
\begin{description}[leftmargin=1.5em, style=nextline]
  \item[Guide:] Sets course, reads signs, rolls navigation.
  \item[Scout:] Moves ahead; first contact with hazards, ambushes, or terrain.
  \item[Quartermaster:] Manages food, supplies, and encumbrance.
  \item[Watch:] Maintains vigilance; first line against ambush or mishap.
\end{description}

\subsection{Encounters \& Events}
When advancing the Travel Clock, the GM may also introduce:
\begin{itemize}
  \item \textbf{Discovery:} Ruins, strange omen, hidden site.
  \item \textbf{Hazard:} Terrain challenge, illness, weather.
  \item \textbf{Encounter:} Bandits, emissaries, beasts.
  \item \textbf{Complication:} Bond tested, supplies strained, rival pursuit.
\end{itemize}

\subsection{Encounter Frequency}
\begin{itemize}
  \item For every 1--2 segments filled, insert one encounter or hazard scene.
  \item Each leg should include \textbf{at least one meaningful encounter}.
\end{itemize}

\subsection{Resolution}
\begin{itemize}
  \item When a Travel Clock fills, narrate arrival at the new region, with fallout from any hazards or complications unresolved.
  \item If complications remain, they carry forward into the next leg until resolved.
\end{itemize}

\subsection{Deck of Consequences Integration}
\begin{itemize}
  \item Draw from the \textbf{Deck of Consequences} when rolling travel hazards or unexpected encounters.
  \item Crown Spread or Campaign Clock can foreshadow upcoming travel-related events.
\end{itemize}

\subsection{GM Quick Cues}
\begin{itemize}
  \item Travel should \textbf{advance story stakes}, not pause them. Use it to foreshadow threats, deepen bonds, or reveal patron omens.
  \item Let players showcase talents in their assigned roles.
  \item Mix discovery and hardship to keep legs tense but rewarding.
\end{itemize}


\section{Travel Framework}
\label{sec:travel}

Adventuring often means crossing dangerous or unknown lands. The travel framework provides pacing tools to turn journeys into dramatic scenes without micromanaging miles.

\subsection{Legs and Clocks}
\begin{itemize}
  \item Break long journeys into \textbf{legs}, each representing a significant change in terrain, danger, or political region.
  \item Each leg is tracked with a \textbf{Travel Clock [4]} by default.
  \item A leg completes when the clock fills through \textbf{progress actions, encounters, or scene resolution}.
\end{itemize}

\subsection{Roles on the Road}
Assign roles each travel leg (rotate or repeat as desired):
\begin{description}[leftmargin=1.5em, style=nextline]
  \item[Guide:] Sets course, reads signs, rolls navigation.
  \item[Scout:] Moves ahead; first contact with hazards, ambushes, or terrain.
  \item[Quartermaster:] Manages food, supplies, and encumbrance.
  \item[Watch:] Maintains vigilance; first line against ambush or mishap.
\end{description}

\subsection{Encounters \& Events}
When advancing the Travel Clock, the GM may also introduce:
\begin{itemize}
  \item \textbf{Discovery:} Ruins, strange omen, hidden site.
  \item \textbf{Hazard:} Terrain challenge, illness, weather.
  \item \textbf{Encounter:} Bandits, emissaries, beasts.
  \item \textbf{Complication:} Bond tested, supplies strained, rival pursuit.
\end{itemize}

\subsection{Encounter Frequency}
\begin{itemize}
  \item For every 1--2 segments filled, insert one encounter or hazard scene.
  \item Each leg should include \textbf{at least one meaningful encounter}.
\end{itemize}

\subsection{Resolution}
\begin{itemize}
  \item When a Travel Clock fills, narrate arrival at the new region, with fallout from any hazards or complications unresolved.
  \item If complications remain, they carry forward into the next leg until resolved.
\end{itemize}

\subsection{Deck of Consequences Integration}
\begin{itemize}
  \item Draw from the \textbf{Deck of Consequences} when rolling travel hazards or unexpected encounters.
  \item Crown Spread or Campaign Clock can foreshadow upcoming travel-related events.
\end{itemize}

\subsection{GM Quick Cues}
\begin{itemize}
  \item Travel should \textbf{advance story stakes}, not pause them. Use it to foreshadow threats, deepen bonds, or reveal patron omens.
  \item Let players showcase talents in their assigned roles.
  \item Mix discovery and hardship to keep legs tense but rewarding.
\end{itemize}


\section{Travel Framework}
\label{sec:travel}

Adventuring often means crossing dangerous or unknown lands. The travel framework provides pacing tools to turn journeys into dramatic scenes without micromanaging miles.

\subsection{Legs and Clocks}
\begin{itemize}
  \item Break long journeys into \textbf{legs}, each representing a significant change in terrain, danger, or political region.
  \item Each leg is tracked with a \textbf{Travel Clock [4]} by default.
  \item A leg completes when the clock fills through \textbf{progress actions, encounters, or scene resolution}.
\end{itemize}

\subsection{Roles on the Road}
Assign roles each travel leg (rotate or repeat as desired):
\begin{description}[leftmargin=1.5em, style=nextline]
  \item[Guide:] Sets course, reads signs, rolls navigation.
  \item[Scout:] Moves ahead; first contact with hazards, ambushes, or terrain.
  \item[Quartermaster:] Manages food, supplies, and encumbrance.
  \item[Watch:] Maintains vigilance; first line against ambush or mishap.
\end{description}

\subsection{Encounters \& Events}
When advancing the Travel Clock, the GM may also introduce:
\begin{itemize}
  \item \textbf{Discovery:} Ruins, strange omen, hidden site.
  \item \textbf{Hazard:} Terrain challenge, illness, weather.
  \item \textbf{Encounter:} Bandits, emissaries, beasts.
  \item \textbf{Complication:} Bond tested, supplies strained, rival pursuit.
\end{itemize}

\subsection{Encounter Frequency}
\begin{itemize}
  \item For every 1--2 segments filled, insert one encounter or hazard scene.
  \item Each leg should include \textbf{at least one meaningful encounter}.
\end{itemize}

\subsection{Resolution}
\begin{itemize}
  \item When a Travel Clock fills, narrate arrival at the new region, with fallout from any hazards or complications unresolved.
  \item If complications remain, they carry forward into the next leg until resolved.
\end{itemize}

\subsection{Deck of Consequences Integration}
\begin{itemize}
  \item Draw from the \textbf{Deck of Consequences} when rolling travel hazards or unexpected encounters.
  \item Crown Spread or Campaign Clock can foreshadow upcoming travel-related events.
\end{itemize}

\subsection{GM Quick Cues}
\begin{itemize}
  \item Travel should \textbf{advance story stakes}, not pause them. Use it to foreshadow threats, deepen bonds, or reveal patron omens.
  \item Let players showcase talents in their assigned roles.
  \item Mix discovery and hardship to keep legs tense but rewarding.
\end{itemize}


\section{Travel Framework}
\label{sec:travel}

Adventuring often means crossing dangerous or unknown lands. The travel framework provides pacing tools to turn journeys into dramatic scenes without micromanaging miles.

\subsection{Legs and Clocks}
\begin{itemize}
  \item Break long journeys into \textbf{legs}, each representing a significant change in terrain, danger, or political region.
  \item Each leg is tracked with a \textbf{Travel Clock [4]} by default.
  \item A leg completes when the clock fills through \textbf{progress actions, encounters, or scene resolution}.
\end{itemize}

\subsection{Roles on the Road}
Assign roles each travel leg (rotate or repeat as desired):
\begin{description}[leftmargin=1.5em, style=nextline]
  \item[Guide:] Sets course, reads signs, rolls navigation.
  \item[Scout:] Moves ahead; first contact with hazards, ambushes, or terrain.
  \item[Quartermaster:] Manages food, supplies, and encumbrance.
  \item[Watch:] Maintains vigilance; first line against ambush or mishap.
\end{description}

\subsection{Encounters \& Events}
When advancing the Travel Clock, the GM may also introduce:
\begin{itemize}
  \item \textbf{Discovery:} Ruins, strange omen, hidden site.
  \item \textbf{Hazard:} Terrain challenge, illness, weather.
  \item \textbf{Encounter:} Bandits, emissaries, beasts.
  \item \textbf{Complication:} Bond tested, supplies strained, rival pursuit.
\end{itemize}

\subsection{Encounter Frequency}
\begin{itemize}
  \item For every 1--2 segments filled, insert one encounter or hazard scene.
  \item Each leg should include \textbf{at least one meaningful encounter}.
\end{itemize}

\subsection{Resolution}
\begin{itemize}
  \item When a Travel Clock fills, narrate arrival at the new region, with fallout from any hazards or complications unresolved.
  \item If complications remain, they carry forward into the next leg until resolved.
\end{itemize}

\subsection{Deck of Consequences Integration}
\begin{itemize}
  \item Draw from the \textbf{Deck of Consequences} when rolling travel hazards or unexpected encounters.
  \item Crown Spread or Campaign Clock can foreshadow upcoming travel-related events.
\end{itemize}

\subsection{GM Quick Cues}
\begin{itemize}
  \item Travel should \textbf{advance story stakes}, not pause them. Use it to foreshadow threats, deepen bonds, or reveal patron omens.
  \item Let players showcase talents in their assigned roles.
  \item Mix discovery and hardship to keep legs tense but rewarding.
\end{itemize}
