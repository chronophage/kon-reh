
% --- Fate's Edge SRD — Section 3: Combat System ---
% Include this file from your main .tex with: 
% --- Fate's Edge SRD — Section 3: Combat System ---
% Include this file from your main .tex with: 
% --- Fate's Edge SRD — Section 3: Combat System ---
% Include this file from your main .tex with: 
% --- Fate's Edge SRD — Section 3: Combat System ---
% Include this file from your main .tex with: \input{03-combat.tex}

\section{Combat System}

\subsection{Core Philosophy}
Combat in Fate's Edge is not a separate mini-game; it is simply conflict under sharper focus. 
It uses the same dice pool system and SB economy as all other actions. 
The rules are designed to emphasize narrative consequence, positional play, and risk management.

\subsection{Structure of Combat}
\begin{itemize}
  \item \textbf{Rounds:} Each round represents a few seconds of action.
  \item \textbf{Turns:} Each participant takes one significant action per round.
  \item \textbf{Scenes:} A battle is one scene unless the fiction dictates otherwise.
\end{itemize}

\subsection{Taking Action}
On your turn, declare intent and method as normal:
\begin{enumerate}
  \item \textbf{Set Position:} The GM decides if you are Controlled, Risky, or Desperate.
  \item \textbf{Build Pool:} Attribute + Skill (+ gear, + assists, +1 from Imbuement if active).
  \item \textbf{Roll:} Each 6+ is a success. Each 1 generates SB.
  \item \textbf{Resolve:} Successes vs DV, SB spent by GM, Position/Effect applied.
\end{enumerate}

\subsection{Position \& Effect}
\begin{description}[leftmargin=1.5em]
  \item[Controlled] You act from safety or advantage. Failure still leaves you options.
  \item[Risky] Standard case. Failure has teeth, but not ruin.
  \item[Desperate] High stakes. Failure is severe; success may bring extra XP (mark Desperate use).
\end{description}
% =========================
% HEALTH, FATIGUE, & HARM (REVISED)
% =========================
\section{Health, Fatigue, \& Harm (Revised)}
\label{sec:health-fatigue-harm-rev}

\subsection*{Tracks \& Caps}
\begin{itemize}
  \item \textbf{Fatigue Track}: boxes equal to \textbf{Body}.
  \item \textbf{Harm Levels}: as defined elsewhere in the SRD (\textbf{Harm 1}, \textbf{Harm 2}, \textbf{Harm 3}).
\end{itemize}

\subsection*{Fatigue $\rightarrow$ Harm Conversion}
Whenever you would mark Fatigue and your Fatigue Track \emph{fills} (all boxes marked):
\begin{enumerate}
  \item \textbf{Increase} your \textbf{Harm} by one level (e.g., 0$\rightarrow$Harm~1, Harm~1$\rightarrow$Harm~2, Harm~2$\rightarrow$Harm~3).
  \item \textbf{Clear all Fatigue} (erase the Fatigue Track back to 0).
\end{enumerate}
This conversion can occur multiple times in a scene. Effects of Harm tier (disadvantage, action limits, incapacitation at Harm~3, etc.) follow your existing SRD.

\subsection*{Taking Fatigue}
Mark Fatigue for strain, exertion, travel, magic costs, or \S\ref{sec:obligation-overflow-rev} overflow. Fatigue can exceed remaining boxes only to \emph{trigger} conversion; any excess is ignored after the Harm increase and Fatigue clear.

\subsection*{Recovering Fatigue}
\begin{itemize}
  \item \textbf{Short Rest} (quiet watch, food/water): remove \textbf{2 Fatigue}.
  \item \textbf{Full Night}: remove \textbf{all Fatigue}.
\end{itemize}
\emph{Fatigue recovery does not remove Harm.} Recover Harm via your normal medical/ritual rules in the SRD.

\subsection*{Mitigation (Optional Dials)}
\begin{itemize}
  \item \textbf{Soak/Ward}: Before marking Fatigue, reduce it by 1--2 (to a minimum of 0) if protected by armor/boons/rites.
  \item \textbf{Convert}: Some effects may convert incoming \textbf{Harm 1} to \textbf{2 Fatigue}; if this \emph{fills} the track, convert as normal.
\end{itemize}

\paragraph{Effect}
Effect is narrative reach:
\begin{itemize}
  \item \textbf{Limited:} Scratch or slow progress.
  \item \textbf{Standard:} Expected impact (downing a guard, disabling a lock).
  \item \textbf{Great:} Overwhelming impact, bigger than expected.
\end{itemize}

\subsection{Damage \& Consequences}
When you take harm:
\begin{itemize}
  \item \textbf{Level 1 Harm:} Minor injury or hindrance. -1 die to related actions.
  \item \textbf{Level 2 Harm:} Serious wound. -1 die to most actions until treated.
  \item \textbf{Level 3 Harm:} Critical injury. You are incapacitated or dying.
\end{itemize}
Harm may be resisted (roll Attribute vs DV 3; 1s generate SB). On a hit, reduce harm by one level.

\subsection{Teamwork}
\begin{itemize}
  \item \textbf{Assist:} Spend 1 Stress or Boon to add +1 die. Max +3 dice from assists.
  \item \textbf{Setup:} Make a roll to improve another’s Position or Effect.
  \item \textbf{Protect:} Take harm or consequence meant for another.
\end{itemize}

\subsection{GM Guidance for SB in Combat}
Use SB to escalate combat fiction:
\begin{itemize}
  \item \textbf{1 SB:} Reinforce enemy cover, minor injury, reveal new foe.
  \item \textbf{2 SB:} Reinforcements arrive, key gear breaks, enemy gains +1 die.
  \item \textbf{3 SB:} Enemy unleashes a Rite or summon, terrain shifts, ally is endangered.
  \item \textbf{4+ SB:} Scene twists—fires spread, the floor collapses, Patron omens manifest.
\end{itemize}

\subsection{Combat and Magic}
\begin{itemize}
  \item \textbf{Casting.} Casters spend one action to \emph{Weave} and another to \emph{Cast}. Requires the \emph{Caster’s Gift} talent. 
  \item \textbf{Rites.} Invoking a Rite takes one action. Players may \emph{Push It} for $+1$ Obligation to gain the listed benefit. 
  \item \textbf{Invokers.} Invokers perform Rites via Symbol. Ritual invocation takes $\text{DV}+1$ rounds and always marks $+1$ Obligation. Alternatively, they may \emph{Crack the Seal} to cast instantly by setting the Symbol to \textsc{Compromised} and marking $+2$ Obligation ($+3$ if High-Power). Invoker Rites cannot use \emph{Push It}. 
  \item \textbf{Imbuements.} Once per scene, spend one action to activate an Imbuement. For the remainder of the scene, gain $+1$ to one Weapon and one Thematic Skill. 
\end{itemize}

\subsection{Worked Example}
\emph{Kael swings his Imbued blade at a cultist (DV 2). He rolls 5 dice: 9, 7, 5, 2, 1.}
\begin{itemize}
  \item Successes = 3 (hit), SB = 1.
  \item GM grants success: cultist is cut down.
  \item GM spends 1 SB: ``Blood sprays across the sigil—energy flares, the summoning accelerates.''
\end{itemize}


\section{Combat System}

\subsection{Core Philosophy}
Combat in Fate's Edge is not a separate mini-game; it is simply conflict under sharper focus. 
It uses the same dice pool system and SB economy as all other actions. 
The rules are designed to emphasize narrative consequence, positional play, and risk management.

\subsection{Structure of Combat}
\begin{itemize}
  \item \textbf{Rounds:} Each round represents a few seconds of action.
  \item \textbf{Turns:} Each participant takes one significant action per round.
  \item \textbf{Scenes:} A battle is one scene unless the fiction dictates otherwise.
\end{itemize}

\subsection{Taking Action}
On your turn, declare intent and method as normal:
\begin{enumerate}
  \item \textbf{Set Position:} The GM decides if you are Controlled, Risky, or Desperate.
  \item \textbf{Build Pool:} Attribute + Skill (+ gear, + assists, +1 from Imbuement if active).
  \item \textbf{Roll:} Each 6+ is a success. Each 1 generates SB.
  \item \textbf{Resolve:} Successes vs DV, SB spent by GM, Position/Effect applied.
\end{enumerate}

\subsection{Position \& Effect}
\begin{description}[leftmargin=1.5em]
  \item[Controlled] You act from safety or advantage. Failure still leaves you options.
  \item[Risky] Standard case. Failure has teeth, but not ruin.
  \item[Desperate] High stakes. Failure is severe; success may bring extra XP (mark Desperate use).
\end{description}
% =========================
% HEALTH, FATIGUE, & HARM (REVISED)
% =========================
\section{Health, Fatigue, \& Harm (Revised)}
\label{sec:health-fatigue-harm-rev}

\subsection*{Tracks \& Caps}
\begin{itemize}
  \item \textbf{Fatigue Track}: boxes equal to \textbf{Body}.
  \item \textbf{Harm Levels}: as defined elsewhere in the SRD (\textbf{Harm 1}, \textbf{Harm 2}, \textbf{Harm 3}).
\end{itemize}

\subsection*{Fatigue $\rightarrow$ Harm Conversion}
Whenever you would mark Fatigue and your Fatigue Track \emph{fills} (all boxes marked):
\begin{enumerate}
  \item \textbf{Increase} your \textbf{Harm} by one level (e.g., 0$\rightarrow$Harm~1, Harm~1$\rightarrow$Harm~2, Harm~2$\rightarrow$Harm~3).
  \item \textbf{Clear all Fatigue} (erase the Fatigue Track back to 0).
\end{enumerate}
This conversion can occur multiple times in a scene. Effects of Harm tier (disadvantage, action limits, incapacitation at Harm~3, etc.) follow your existing SRD.

\subsection*{Taking Fatigue}
Mark Fatigue for strain, exertion, travel, magic costs, or \S\ref{sec:obligation-overflow-rev} overflow. Fatigue can exceed remaining boxes only to \emph{trigger} conversion; any excess is ignored after the Harm increase and Fatigue clear.

\subsection*{Recovering Fatigue}
\begin{itemize}
  \item \textbf{Short Rest} (quiet watch, food/water): remove \textbf{2 Fatigue}.
  \item \textbf{Full Night}: remove \textbf{all Fatigue}.
\end{itemize}
\emph{Fatigue recovery does not remove Harm.} Recover Harm via your normal medical/ritual rules in the SRD.

\subsection*{Mitigation (Optional Dials)}
\begin{itemize}
  \item \textbf{Soak/Ward}: Before marking Fatigue, reduce it by 1--2 (to a minimum of 0) if protected by armor/boons/rites.
  \item \textbf{Convert}: Some effects may convert incoming \textbf{Harm 1} to \textbf{2 Fatigue}; if this \emph{fills} the track, convert as normal.
\end{itemize}

\paragraph{Effect}
Effect is narrative reach:
\begin{itemize}
  \item \textbf{Limited:} Scratch or slow progress.
  \item \textbf{Standard:} Expected impact (downing a guard, disabling a lock).
  \item \textbf{Great:} Overwhelming impact, bigger than expected.
\end{itemize}

\subsection{Damage \& Consequences}
When you take harm:
\begin{itemize}
  \item \textbf{Level 1 Harm:} Minor injury or hindrance. -1 die to related actions.
  \item \textbf{Level 2 Harm:} Serious wound. -1 die to most actions until treated.
  \item \textbf{Level 3 Harm:} Critical injury. You are incapacitated or dying.
\end{itemize}
Harm may be resisted (roll Attribute vs DV 3; 1s generate SB). On a hit, reduce harm by one level.

\subsection{Teamwork}
\begin{itemize}
  \item \textbf{Assist:} Spend 1 Stress or Boon to add +1 die. Max +3 dice from assists.
  \item \textbf{Setup:} Make a roll to improve another’s Position or Effect.
  \item \textbf{Protect:} Take harm or consequence meant for another.
\end{itemize}

\subsection{GM Guidance for SB in Combat}
Use SB to escalate combat fiction:
\begin{itemize}
  \item \textbf{1 SB:} Reinforce enemy cover, minor injury, reveal new foe.
  \item \textbf{2 SB:} Reinforcements arrive, key gear breaks, enemy gains +1 die.
  \item \textbf{3 SB:} Enemy unleashes a Rite or summon, terrain shifts, ally is endangered.
  \item \textbf{4+ SB:} Scene twists—fires spread, the floor collapses, Patron omens manifest.
\end{itemize}

\subsection{Combat and Magic}
\begin{itemize}
  \item \textbf{Casting.} Casters spend one action to \emph{Weave} and another to \emph{Cast}. Requires the \emph{Caster’s Gift} talent. 
  \item \textbf{Rites.} Invoking a Rite takes one action. Players may \emph{Push It} for $+1$ Obligation to gain the listed benefit. 
  \item \textbf{Invokers.} Invokers perform Rites via Symbol. Ritual invocation takes $\text{DV}+1$ rounds and always marks $+1$ Obligation. Alternatively, they may \emph{Crack the Seal} to cast instantly by setting the Symbol to \textsc{Compromised} and marking $+2$ Obligation ($+3$ if High-Power). Invoker Rites cannot use \emph{Push It}. 
  \item \textbf{Imbuements.} Once per scene, spend one action to activate an Imbuement. For the remainder of the scene, gain $+1$ to one Weapon and one Thematic Skill. 
\end{itemize}

\subsection{Worked Example}
\emph{Kael swings his Imbued blade at a cultist (DV 2). He rolls 5 dice: 9, 7, 5, 2, 1.}
\begin{itemize}
  \item Successes = 3 (hit), SB = 1.
  \item GM grants success: cultist is cut down.
  \item GM spends 1 SB: ``Blood sprays across the sigil—energy flares, the summoning accelerates.''
\end{itemize}


\section{Combat System}

\subsection{Core Philosophy}
Combat in Fate's Edge is not a separate mini-game; it is simply conflict under sharper focus. 
It uses the same dice pool system and SB economy as all other actions. 
The rules are designed to emphasize narrative consequence, positional play, and risk management.

\subsection{Structure of Combat}
\begin{itemize}
  \item \textbf{Rounds:} Each round represents a few seconds of action.
  \item \textbf{Turns:} Each participant takes one significant action per round.
  \item \textbf{Scenes:} A battle is one scene unless the fiction dictates otherwise.
\end{itemize}

\subsection{Taking Action}
On your turn, declare intent and method as normal:
\begin{enumerate}
  \item \textbf{Set Position:} The GM decides if you are Controlled, Risky, or Desperate.
  \item \textbf{Build Pool:} Attribute + Skill (+ gear, + assists, +1 from Imbuement if active).
  \item \textbf{Roll:} Each 6+ is a success. Each 1 generates SB.
  \item \textbf{Resolve:} Successes vs DV, SB spent by GM, Position/Effect applied.
\end{enumerate}

\subsection{Position \& Effect}
\begin{description}[leftmargin=1.5em]
  \item[Controlled] You act from safety or advantage. Failure still leaves you options.
  \item[Risky] Standard case. Failure has teeth, but not ruin.
  \item[Desperate] High stakes. Failure is severe; success may bring extra XP (mark Desperate use).
\end{description}
% =========================
% HEALTH, FATIGUE, & HARM (REVISED)
% =========================
\section{Health, Fatigue, \& Harm (Revised)}
\label{sec:health-fatigue-harm-rev}

\subsection*{Tracks \& Caps}
\begin{itemize}
  \item \textbf{Fatigue Track}: boxes equal to \textbf{Body}.
  \item \textbf{Harm Levels}: as defined elsewhere in the SRD (\textbf{Harm 1}, \textbf{Harm 2}, \textbf{Harm 3}).
\end{itemize}

\subsection*{Fatigue $\rightarrow$ Harm Conversion}
Whenever you would mark Fatigue and your Fatigue Track \emph{fills} (all boxes marked):
\begin{enumerate}
  \item \textbf{Increase} your \textbf{Harm} by one level (e.g., 0$\rightarrow$Harm~1, Harm~1$\rightarrow$Harm~2, Harm~2$\rightarrow$Harm~3).
  \item \textbf{Clear all Fatigue} (erase the Fatigue Track back to 0).
\end{enumerate}
This conversion can occur multiple times in a scene. Effects of Harm tier (disadvantage, action limits, incapacitation at Harm~3, etc.) follow your existing SRD.

\subsection*{Taking Fatigue}
Mark Fatigue for strain, exertion, travel, magic costs, or \S\ref{sec:obligation-overflow-rev} overflow. Fatigue can exceed remaining boxes only to \emph{trigger} conversion; any excess is ignored after the Harm increase and Fatigue clear.

\subsection*{Recovering Fatigue}
\begin{itemize}
  \item \textbf{Short Rest} (quiet watch, food/water): remove \textbf{2 Fatigue}.
  \item \textbf{Full Night}: remove \textbf{all Fatigue}.
\end{itemize}
\emph{Fatigue recovery does not remove Harm.} Recover Harm via your normal medical/ritual rules in the SRD.

\subsection*{Mitigation (Optional Dials)}
\begin{itemize}
  \item \textbf{Soak/Ward}: Before marking Fatigue, reduce it by 1--2 (to a minimum of 0) if protected by armor/boons/rites.
  \item \textbf{Convert}: Some effects may convert incoming \textbf{Harm 1} to \textbf{2 Fatigue}; if this \emph{fills} the track, convert as normal.
\end{itemize}

\paragraph{Effect}
Effect is narrative reach:
\begin{itemize}
  \item \textbf{Limited:} Scratch or slow progress.
  \item \textbf{Standard:} Expected impact (downing a guard, disabling a lock).
  \item \textbf{Great:} Overwhelming impact, bigger than expected.
\end{itemize}

\subsection{Damage \& Consequences}
When you take harm:
\begin{itemize}
  \item \textbf{Level 1 Harm:} Minor injury or hindrance. -1 die to related actions.
  \item \textbf{Level 2 Harm:} Serious wound. -1 die to most actions until treated.
  \item \textbf{Level 3 Harm:} Critical injury. You are incapacitated or dying.
\end{itemize}
Harm may be resisted (roll Attribute vs DV 3; 1s generate SB). On a hit, reduce harm by one level.

\subsection{Teamwork}
\begin{itemize}
  \item \textbf{Assist:} Spend 1 Stress or Boon to add +1 die. Max +3 dice from assists.
  \item \textbf{Setup:} Make a roll to improve another’s Position or Effect.
  \item \textbf{Protect:} Take harm or consequence meant for another.
\end{itemize}

\subsection{GM Guidance for SB in Combat}
Use SB to escalate combat fiction:
\begin{itemize}
  \item \textbf{1 SB:} Reinforce enemy cover, minor injury, reveal new foe.
  \item \textbf{2 SB:} Reinforcements arrive, key gear breaks, enemy gains +1 die.
  \item \textbf{3 SB:} Enemy unleashes a Rite or summon, terrain shifts, ally is endangered.
  \item \textbf{4+ SB:} Scene twists—fires spread, the floor collapses, Patron omens manifest.
\end{itemize}

\subsection{Combat and Magic}
\begin{itemize}
  \item \textbf{Casting.} Casters spend one action to \emph{Weave} and another to \emph{Cast}. Requires the \emph{Caster’s Gift} talent. 
  \item \textbf{Rites.} Invoking a Rite takes one action. Players may \emph{Push It} for $+1$ Obligation to gain the listed benefit. 
  \item \textbf{Invokers.} Invokers perform Rites via Symbol. Ritual invocation takes $\text{DV}+1$ rounds and always marks $+1$ Obligation. Alternatively, they may \emph{Crack the Seal} to cast instantly by setting the Symbol to \textsc{Compromised} and marking $+2$ Obligation ($+3$ if High-Power). Invoker Rites cannot use \emph{Push It}. 
  \item \textbf{Imbuements.} Once per scene, spend one action to activate an Imbuement. For the remainder of the scene, gain $+1$ to one Weapon and one Thematic Skill. 
\end{itemize}

\subsection{Worked Example}
\emph{Kael swings his Imbued blade at a cultist (DV 2). He rolls 5 dice: 9, 7, 5, 2, 1.}
\begin{itemize}
  \item Successes = 3 (hit), SB = 1.
  \item GM grants success: cultist is cut down.
  \item GM spends 1 SB: ``Blood sprays across the sigil—energy flares, the summoning accelerates.''
\end{itemize}


\section{Combat System}

\subsection{Core Philosophy}
Combat in Fate's Edge is not a separate mini-game; it is simply conflict under sharper focus. 
It uses the same dice pool system and SB economy as all other actions. 
The rules are designed to emphasize narrative consequence, positional play, and risk management.

\subsection{Structure of Combat}
\begin{itemize}
  \item \textbf{Rounds:} Each round represents a few seconds of action.
  \item \textbf{Turns:} Each participant takes one significant action per round.
  \item \textbf{Scenes:} A battle is one scene unless the fiction dictates otherwise.
\end{itemize}

\subsection{Taking Action}
On your turn, declare intent and method as normal:
\begin{enumerate}
  \item \textbf{Set Position:} The GM decides if you are Controlled, Risky, or Desperate.
  \item \textbf{Build Pool:} Attribute + Skill (+ gear, + assists, +1 from Imbuement if active).
  \item \textbf{Roll:} Each 6+ is a success. Each 1 generates SB.
  \item \textbf{Resolve:} Successes vs DV, SB spent by GM, Position/Effect applied.
\end{enumerate}

\subsection{Position \& Effect}
\begin{description}[leftmargin=1.5em]
  \item[Controlled] You act from safety or advantage. Failure still leaves you options.
  \item[Risky] Standard case. Failure has teeth, but not ruin.
  \item[Desperate] High stakes. Failure is severe; success may bring extra XP (mark Desperate use).
\end{description}
% =========================
% HEALTH, FATIGUE, & HARM (REVISED)
% =========================
\section{Health, Fatigue, \& Harm}
\label{sec:health-fatigue-harm-rev}

\subsection*{Tracks \& Caps}
\begin{itemize}
  \item \textbf{Fatigue Track}: boxes equal to \textbf{Body}.
  \item \textbf{Harm Levels}: as defined elsewhere in the SRD (\textbf{Harm 1}, \textbf{Harm 2}, \textbf{Harm 3}).
\end{itemize}

\subsection*{Fatigue $\rightarrow$ Harm Conversion}
Whenever you would mark Fatigue and your Fatigue Track \emph{fills} (all boxes marked):
\begin{enumerate}
  \item \textbf{Increase} your \textbf{Harm} by one level (e.g., 0$\rightarrow$Harm~1, Harm~1$\rightarrow$Harm~2, Harm~2$\rightarrow$Harm~3).
  \item \textbf{Clear all Fatigue} (erase the Fatigue Track back to 0).
\end{enumerate}
This conversion can occur multiple times in a scene. Effects of Harm tier (disadvantage, action limits, incapacitation at Harm~3, etc.) follow your existing SRD.

\subsection*{Taking Fatigue}
Mark Fatigue for strain, exertion, travel, magic costs, or \S\ref{sec:obligation-overflow-rev} overflow. Fatigue can exceed remaining boxes only to \emph{trigger} conversion; any excess is ignored after the Harm increase and Fatigue clear.

\subsection*{Recovering Fatigue}
\begin{itemize}
  \item \textbf{Short Rest} (quiet watch, food/water): remove \textbf{2 Fatigue}.
  \item \textbf{Full Night}: remove \textbf{all Fatigue}.
\end{itemize}
\emph{Fatigue recovery does not remove Harm.} Recover Harm via your normal medical/ritual rules in the SRD.

\subsection*{Mitigation (Optional Dials)}
\begin{itemize}
  \item \textbf{Soak/Ward}: Before marking Fatigue, reduce it by 1--2 (to a minimum of 0) if protected by armor/boons/rites.
  \item \textbf{Convert}: Some effects may convert incoming \textbf{Harm 1} to \textbf{2 Fatigue}; if this \emph{fills} the track, convert as normal.
\end{itemize}

\paragraph{Effect}
Effect is narrative reach:
\begin{itemize}
  \item \textbf{Limited:} Scratch or slow progress.
  \item \textbf{Standard:} Expected impact (downing a guard, disabling a lock).
  \item \textbf{Great:} Overwhelming impact, bigger than expected.
\end{itemize}

\subsection{Damage \& Consequences}
When you take harm:
\begin{itemize}
  \item \textbf{Level 1 Harm:} Minor injury or hindrance. -1 die to related actions.
  \item \textbf{Level 2 Harm:} Serious wound. -1 die to most actions until treated.
  \item \textbf{Level 3 Harm:} Critical injury. You are incapacitated or dying.
\end{itemize}
Harm may be resisted (roll Attribute vs DV 3; 1s generate SB). On a hit, reduce harm by one level.

\subsection{Teamwork}
\begin{itemize}
  \item \textbf{Assist:} Spend 1 Stress or Boon to add +1 die. Max +3 dice from assists.
  \item \textbf{Setup:} Make a roll to improve another’s Position or Effect.
  \item \textbf{Protect:} Take harm or consequence meant for another.
\end{itemize}

\subsection{GM Guidance for SB in Combat}
Use SB to escalate combat fiction:
\begin{itemize}
  \item \textbf{1 SB:} Reinforce enemy cover, minor injury, reveal new foe.
  \item \textbf{2 SB:} Reinforcements arrive, key gear breaks, enemy gains +1 die.
  \item \textbf{3 SB:} Enemy unleashes a Rite or summon, terrain shifts, ally is endangered.
  \item \textbf{4+ SB:} Scene twists—fires spread, the floor collapses, Patron omens manifest.
\end{itemize}

\subsection{Combat and Magic}
\begin{itemize}
  \item \textbf{Casting.} Casters spend one action to \emph{Weave} and another to \emph{Cast}. Requires the \emph{Caster’s Gift} talent. 
  \item \textbf{Rites.} Invoking a Rite takes one action. Players may \emph{Push It} for $+1$ Obligation to gain the listed benefit. 
  \item \textbf{Invokers.} Invokers perform Rites via Symbol. Ritual invocation takes $\text{DV}+1$ rounds and always marks $+1$ Obligation. Alternatively, they may \emph{Crack the Seal} to cast instantly by setting the Symbol to \textsc{Compromised} and marking $+2$ Obligation ($+3$ if High-Power). Invoker Rites cannot use \emph{Push It}. 
  \item \textbf{Imbuements.} Once per scene, spend one action to activate an Imbuement. For the remainder of the scene, gain $+1$ to one Weapon and one Thematic Skill. 
\end{itemize}

\subsection{Weapons \& Armor}
\label{app:weapons-armor}
\index{Weapons}\index{Armor}

\subsubsection{Weapons by Weight Class}
\begin{itemize}
  \item \textbf{Light (4 XP)} — fast, concealable.
  \item \textbf{Medium (8 XP)} — balanced, battlefield standard.
  \item \textbf{Heavy (12 XP)} — punishing, slow.
\end{itemize}

\subsubsection*{Melee}
\begin{tabular}{llll}
\toprule
\textbf{Weight} & \textbf{Close} & \textbf{Near} & \textbf{Notes} \\
\midrule
Light & +2d & +1d & Quick, tight quarters \\
Medium & +1d & +2d & \emph{Set} 1/scene or –1d first attack \\
Heavy & –1d & +3d & \emph{Set} 1/scene or –2d first attack \\
\bottomrule
\end{tabular}

\subsubsection*{Ranged \& Tempo}
\begin{tabular}{lllll}
\toprule
\textbf{Weight} & \textbf{Tempo} & \textbf{Close} & \textbf{Near} & \textbf{Far} \\
\midrule
Light (4 XP) & Fast & Risky & +1d & — \\
Medium (8 XP) & Standard & Desperate & +2d & +1d \\
Heavy (12 XP) & Slow & Desperate & +1d & +3d \\
\bottomrule
\end{tabular}

\paragraph{Tempo:} \textbf{Fast} = Move+Shoot. \textbf{Standard} = Move or Shoot, Aim +1d/Effect. \textbf{Slow} = Set/Brace, full reload, cannot Move+Shoot.

\subsubsection{Weapon Tags (Optional, +4 XP each, max 2)}
\index{Weapons!Tags}
\textbf{Reach, Close, Accurate, Brutal, Hook, Concealable, Quickdraw, Two-Handed, Off-Hand.}

\subsection{Shields (Optional)}
\begin{tabular}{llll}
\toprule
\textbf{Shield} & \textbf{XP} & \textbf{Benefit} & \textbf{Tradeoff} \\
\midrule
Buckler & 4 & +1d Defend vs melee or +1 DV & Off-hand \\
Heater  & 8 & +1d Defend; 1 Harm→Fatigue & –1d Ranged \\
Pavise  & 12 & \emph{Plant}: heavy cover cone & Bulky, immobile \\
\bottomrule
\end{tabular}

\subsection{Armor}
\begin{tabular}{llll}
\toprule
\textbf{Armor} & \textbf{XP} & \textbf{Conversion} & \textbf{Penalty} \\
\midrule
Light  & 4  & 1 Harm→1 Fatigue & — \\
Medium & 8  & 2 Harm→1 Fatigue & –1d physical \\
Heavy  & 12 & 3 Harm→2 Fatigue & –2d physical, no sprint \\
\bottomrule
\end{tabular}

\paragraph{Notes:} Conversion applies per Harm instance before Fatigue is marked. You may still Resist first.

\subsection{Condition \& Upkeep}
\begin{description}
  \item[\textbf{Neglected}] Weapons –1d; Armor: conversion worsens by 1 step.
  \item[\textbf{Compromised}] Weapons –1d first attack/round; Armor: no conversion.
\end{description}
\emph{Fix:} Short Rest/tools remove Neglected. A scene/Smith removes Compromised.

\subsection{Ranged Options (At a Glance)}
\begin{itemize}
  \item \textbf{Aim:} +1d or +1 Effect.  
  \item \textbf{Volley:} Extra ammo +1d (max +2).  
  \item \textbf{Suppress:} Zone fire, foes –1d/Limited Effect.  
  \item \textbf{Overwatch:} Ready a Risky shot on trigger.  
\end{itemize}

\subsection{Worked Example}
\emph{Kael swings his Imbued blade at a cultist (DV 2). He rolls 5 dice: 9, 7, 5, 2, 1.}
\begin{itemize}
  \item Successes = 3 (hit), SB = 1.
  \item GM grants success: cultist is cut down.
  \item GM spends 1 SB: ``Blood sprays across the sigil—energy flares, the summoning accelerates.''
\end{itemize}
