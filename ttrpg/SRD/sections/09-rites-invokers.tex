% =========================
% Fate's Edge SRD — Section 09: Rites, Invokers, and Symbols
% Include from main with: % =========================
% Fate's Edge SRD — Section 09: Rites, Invokers, and Symbols
% Include from main with: % =========================
% Fate's Edge SRD — Section 09: Rites, Invokers, and Symbols
% Include from main with: % =========================
% Fate's Edge SRD — Section 09: Rites, Invokers, and Symbols
% Include from main with: \input{patrons/09-rites-invokers.tex}
% Requires: \usepackage{float} for [H] tables
% =========================

\section{Rites, Invokers, and Symbols}
\label{sec:rites}

Magic in \textbf{Fate's Edge} expresses through three intertwined practices: \textbf{Rites} (oathbound authority), \textbf{Invocations} (symbolic ritual), and \textbf{Patron Pacts} (gifts and obligations). The rules below emphasize fiction-first play: consequences are Story Beats (SB) that prompt twists; numbers follow the story.

\subsection{Rites and Patrons (Runekeepers)}
\label{subsec:runekeepers}
Characters who bind themselves to a \emph{single} Patron and study that Patron's \textbf{Codex} are \textbf{Runekeepers}. Their magic is structured, immediate, and tied to service.

\begin{itemize}
  \item \textbf{One-Patron Rule.} A Runekeeper may be bound to \emph{only one} Patron at a time. This sharpens identity and keeps Obligation on a single ledger.
  \item \textbf{Thiasos (Familiar).} A circle, retinue, or emissary that grounds the pact in fiction. Required to access \emph{Patron's Gift}.
  \item \textbf{Codex.} The Patron's corpus of rites and precedents. Grants access to the Patron's Rites.
  \item \textbf{Invoke Rites.} A Runekeeper may Invoke a known Rite from their Patron as a \textbf{1 action} effect. On completion, mark \textbf{+1 Obligation} to that Patron. You may \emph{Push It} once per scene to amplify the effect, marking \textbf{+1 additional Obligation}.
\end{itemize}

\subsection{Invokers and Symbols}
\label{subsec:invokers}
Invokers relate to Patrons through consecrated \textbf{Symbols}: physical tokens that anchor names and permissions.

\begin{itemize}
  \item \textbf{Symbols (Minor Asset).} Each Symbol is keyed to one Patron; cost \textbf{4 XP}. You may own Symbols of different Patrons (one Symbol per Patron).
  \item \textbf{Ritual Invocation.} Display the Symbol and perform the Rite as a \emph{ritual} (Significant Time). Completion always marks \textbf{+1 Obligation} on that Rite's ledger.
  \item \textbf{Crack the Seal.} As part of an Invoker Rite, you may resolve the effect instantly by setting the Symbol to \emph{Compromised} and marking \textbf{+2 Obligation} (\textbf{+3} if High-Power). The Keeper may spend 1 on-theme SB immediately. The asset remains but is inert until restored.
  \item \textbf{Restore a Symbol.} 1 downtime action and a fitting test (DV 3 or by fiction). Success: \emph{Maintained}; shaky: returns \emph{Neglected}. Or spend \textbf{1 XP} to fully restore.
  \item \textbf{Display Requirement.} Symbols must be openly displayed for rituals. Hidden Symbols do not function.
\end{itemize}

\subsection{Casting and Free-Form Magic}
\label{subsec:casting}
Improvised casting is possible with the \textbf{Caster's Gift} Talent (\textbf{2 XP}). It is a \emph{backup toolkit}:
\begin{itemize}
  \item Small, local effects (typ. DV 2--3), fiction-first, colored by Elements and locus.
  \item Heavy control effects such as \texttt{[WARD]}, \texttt{[BANISH]}, or \texttt{[UNWARD]} require a printed Talent, Rite, or Spell result.
\end{itemize}

\subsection{Patron's Gift (Imbuements)}
\label{subsec:patrons-gift}
The pact may mark a devotee's tools with a short-lived boon aligned to the Patron's domain.

\paragraph{Requirements.}
\textbf{Thiasos (Familiar)} is required. Invoking the Gift costs \textbf{1 Boon}. A Codex is \emph{not} required for the Gift.

\paragraph{Activation and Duration.}
\begin{itemize}
  \item \textbf{Action:} 1 action to activate; \textbf{1/scene}.
  \item \textbf{Duration:} Scene. \emph{Push It:} extend for one additional scene by marking \textbf{+1 Obligation} to that Patron (max one Push per scene).
  \item \textbf{Range:} Touch (you must handle the item).
  \item \textbf{Stacking:} Gifts from the \emph{same Patron} do not stack; take the best active version. Dice bonuses respect the table's \textbf{+3 dice cap}.
\end{itemize}

\paragraph{Effect.}
Choose one held item you or an ally carries. Until scene end it grants:
\begin{itemize}
  \item \textbf{+1 Melee} (the item counts as a magical weapon), and
  \item \textbf{+1 Thematic} (a \emph{+1 die} to a fixed Skill tied to your Patron; see Table~\ref{tab:gift-thematic-map}). Apply only when the fiction clearly fits the Patron's sphere and how the item is used.
\end{itemize}

\paragraph{Runekeeper Clarification.}
A Runekeeper (one Patron + Codex) may Invoke Rites on-screen and use Patron's Gift if they also possess \textbf{Thiasos (Familiar)}. Codex alone does not grant the Gift. Symbols are optional for parley or omens and do not gate Runekeeper Invocation or the Gift.

\section*{Borrowed Grace}
\label{talent:borrowed-grace}
\index{Talents!Invoker}\index{Imbuement!Lesser}

\textbf{Type:} Invoker Talent — \textit{Lesser Imbuement}

\subsection*{Use}
\begin{itemize}
  \item \textbf{Cost:} 1 Boon, 1 action.
  \item \textbf{Effect (pick one on use):} \textbf{+1 Melee} \emph{or} \textbf{+1 Thematic} (your table’s thematic Skill).
  \item \textbf{Duration:} \textit{Single action/attack} (instantaneous boost).
  \item \textbf{Requirement:} Wield/display the Patron’s \textbf{Symbol}.
  \item \textbf{Obligation:} +1 \textbf{Obligation} to that Patron immediately (see \S\ref{sec:obligation}).
  \item \textbf{Limits:} Cannot be extended, stacked, or \emph{Pushed} for duration.
\end{itemize}

\subsection*{Fictional Framing}
A quick, rule-bending channel through a Patron’s \emph{Symbol}—a sliver of grace, borrowed for a moment and paid for in debt.

\subsection*{Table Guidance (1-liners)}
\begin{itemize}
  \item \textbf{Combat:} Spike a strike vs. a tough foe; or steady a parry in a desperate bind.
  \item \textbf{Skill:} Nudge a pivotal social/ritual/track roll tied to the Patron’s sphere.
  \item \textbf{Fallout:} Repeated use accrues \textbf{Obligation}; NPC faithful may notice “stolen” grace.
\end{itemize}

\subsection*{Balance Notes}
\begin{itemize}
  \item Weaker than full Imbuement: \emph{one} action, no sustain, upfront Obligation.
  \item \textbf{Symbol dependency:} No Symbol, no channel (concealed or lost Symbol = no effect).
\end{itemize}

\subsection*{GM Hooks (quick picks)}
\begin{itemize}
  \item \textbf{Compel Debt:} A Patron agent arrives when Obligation crosses a tick.
  \item \textbf{Clash of Signs:} Using rival Symbols back-to-back risks minor \textbf{Backlash} (drop Position or +1 SB).
  \item \textbf{Spotlight Tell:} Brief visual tell (scent, sigil flare) marks the borrowing to observant NPCs.
\end{itemize}

\begin{table}[H]
\centering
\renewcommand{\arraystretch}{1.15}
\begin{tabular}{@{}p{3.8cm}p{3.8cm}p{7.5cm}@{}}
\toprule
\textbf{Patron} & \textbf{+1 Thematic Skill} & \textbf{Gift / Lore Bestowal} \\
\midrule
Ikasha (Shadow, Penumbra) & Stealth & Grants the hush between footsteps and the raven’s omen at every threshold. \\
Mykkiel (Judgment, Writ) & Command & Grants the authority of seal and sentence, words that bind like iron. \\
The Witness (Truth, Revelation) & Notice & Grants the unblinking gaze that unmasks deceit and remembers every oath. \\
Sealed Gate (Boundaries, Closure) & Tinker & Grants mastery of thresholds—doors that yield or bar at your command. \\
Raéyn (Storm, Tides) & Skirmish & Grants the sailor’s fortune: winds that shift, storms that answer to will. \\
Khemesh (Abyss, Pressure) & Skirmish & Grants the crushing silence of the deep, where strength is drowned in weight. \\
Mab (Glamour, Courts) & Persuade & Grants the mask of favor, a voice that bends courtiers and kindles desire. \\
Sacred Geometry (Perfect Forms) & Tinker & Grants the compass of perfection, every shape reduced to its true measure. \\
Clockwork Monad (Mechanism, Process) & Tinker & Grants the certainty of repetition: a cycle that never falters, a gear that never slips. \\
Varnek Karn (Ossuary, Dominion of the Dead) & Command & Grants the silence of the archive, where the dead obey and records speak. \\
Nidhoggr (Deep Earth, Rot) & Skirmish & Grants the weight of ages, the strength of stone and the hunger of roots. \\
The Traveler (Ways, Roads) & Notice & Grants the open way, a compass that never rests, and roads where none are marked. \\
Oath of Flame \& Light (Dawn, Vows) & Command & Grants the fire of dawn, a vow that shields the faithful and sears the faithless. \\
Carrion King (Carrion, Renewal) & Survival & Grants the feast of decay, where what is dead becomes seed for what lives. \\
Gallows Bell (Doom, Last Rites) & Command & Grants the toll of ending, a voice that closes stories and calls debts due. \\
Old Man of the Black Forest (Primal Humanity, Instinct) & Survival & Grants the wild memory: fang, fire, and the path of instinct through the dark wood. \\
Isoka (Serpents, Shedding) & Skirmish & Grants the serpent’s coil, strength in sudden strike and wisdom in renewal. \\
Inaea (Mercy, Hearth) & Persuade & Grants the hearth’s warmth, shelter to the weary and mercy for the lost. \\
Maelstraeus (Infernal Bargainer) & Persuade & Grants the contract’s weight, every deal sealed in fire and shadow. \\
Livaea (Temptation, Desire) & Persuade & Grants the lure of longing, beauty sharpened into power over hearts. \\
\bottomrule
\end{tabular}
\caption{Patron’s Gift: fixed Thematic Skill and lore of their bestowed blessing. Thematic bonuses apply only when the fiction matches the Patron’s domain.}
\label{tab:gift-thematic-map}
\end{table}

\subsection{Specialization vs.\ Mixing}
\label{subsec:mixing}
Characters can mix paths (Summoner, Caster, Invoker, Runekeeper), but specialization is usually stronger and cleaner. Mixing increases upkeep (Obligation, Symbol state, Leash) and action congestion without guaranteed power gains. Let fiction guide choices: Story Beats are prompts to advance the scene, not punishments.

% Patron subsections (split files — keep filenames in the same directory)

\include{patrons/carrion-king.tex}
\include{patrons/clockwork-monad.tex}
\include{patrons/gallows-bell.tex}
\include{patrons/grimmir.tex}
\include{patrons/ikasha.tex}
\include{patrons/inaea.tex}
\include{patrons/isoka.tex}
\include{patrons/khemesh.tex}
\include{patrons/livaea.tex}
\include{patrons/mab.tex}
\include{patrons/maelstraeus.tex}
\include{patrons/mykkiel.tex}
\include{patrons/nidhoggr.tex}
\include{patrons/oath-flame-light.tex}
\include{patrons/raeyn.tex}
\include{patrons/sacred-geometry.tex}
\include{patrons/sealed-gate.tex}
\include{patrons/traveler.tex}
\include{patrons/varnek-karn.tex}
\include{patrons/witness.tex}


\section{Patron Rivalries}
\label{sec:patron-rivalries}

Rivalries set expectations for tone and friction. Use them to color rulings, nudge Position, and guide how Story Beats (SB) land. In their home domains, a Patron’s work tends to start a step better in Position; in a rival’s, a step worse (Keeper’s call).

\begin{table}[h!]
  \centering
  \renewcommand{\arraystretch}{1.15}
  \begin{tabular}{@{}p{3.4cm}p{3.4cm}p{8.2cm}@{}}
    \toprule
    \textbf{Patron} & \textbf{Primary Rival} & \textbf{Retribution in Play (one-line lore)} \\
    \midrule
    Raéyn (Sea, Tides, Travel) & Khemesh (Abyssal Maw) & Those who spurn the sea are swallowed by storms and riptides. \\
    Khemesh (Abyssal Maw) & Raéyn (Sea, Tides, Travel) & Depth devours chart and voice alike; only silence remains. \\
    Sealed Gate (Boundaries, Closure) & The Traveler (Ways, Roads) & Trespassers find every path locked; even home’s door bars their way. \\
    The Traveler (Ways, Roads) & Sealed Gate (Boundaries, Closure) & Those who deny the road are stranded at the threshold forever. \\
    The Witness (Truth, Revelation) & Mab (Glamour, Courts) & Liars discover their tongues turned to ash beneath the unblinking eye. \\
    Mab (Glamour, Courts) & The Witness (Truth, Revelation) & Those who strip glamour are forever exiled from merriment and favor. \\
    Ikasha (Shadow, Latent Potential) & The Witness (Truth, Revelation) & Those who disrespect the hush find every shadow whispering their name. \\
    Mykkiel (Judgment, Writ) & Varnek Karn (Necromantic Archives) & Those who defy judgment are hounded by warrants even in death. \\
    Varnek Karn (Necromantic Archives) & Oath of Light \& Flame (Dawn, Vows) & Those who desecrate memory are bound in chains of bone. \\
    Oath of Light \& Flame (Dawn, Vows) & Khemesh (Abyssal Maw) & Oathbreakers burn at dawn; no tide quenches their fire. \\
    Sacred Geometry (Order, Pattern) & The Traveler (Ways, Fortune) & Those who spurn order are lost forever in mazes without end. \\
    Clockwork Monad (Iteration, Process) & Old Man of the Black Forest (Primal Humanity, Instinct) & Those who break the cycle are crushed beneath their own gears. \\
    Nidhoggr (Dreaming Antiquity) & Sacred Geometry (Order, Pattern) & Those who measure the ancient are buried beneath its weight. \\
    Carrion King (Carrion, Renewal) & Inaea (Mercy, Hearth) & Those who waste life are repaid in rot and swarming hunger. \\
    Gallows Bell (Doom, Last Rites) & Oath of Light \& Flame (Dawn, Vows) & Those who mock the last toll find their own names rung in iron. \\
    Old Man of the Black Forest (Primal Humanity, Instinct) & Mab (Glamour, Courts) & Those who spurn the old ways are hunted in the woods like beasts. \\
    Isoka (Serpents, Shedding) & Sacred Geometry (Order, Pattern) & Those who deny change are crushed in the serpent’s coil. \\
    Inaea (Mercy, Hearth) & Carrion King (Carrion, Renewal) & Those who betray hospitality are cast out to starve in the night. \\
    Maelstraeus (Infernal Bargainer) & The Witness (Truth, Revelation) & Those who renege on a bargain are claimed by fire and clause. \\
    Livaea (Temptation, Desire) & Inaea (Mercy, Hearth) & Those who corrupt sanctuary with lust are haunted by love turned poison. \\
    \bottomrule
  \end{tabular}
  \caption{Primary Patron Rivalries and the retribution that follows when their domains are denied.}
\end{table}

% Requires: \usepackage{float} for [H] tables
% =========================

\section{Rites, Invokers, and Symbols}
\label{sec:rites}

Magic in \textbf{Fate's Edge} expresses through three intertwined practices: \textbf{Rites} (oathbound authority), \textbf{Invocations} (symbolic ritual), and \textbf{Patron Pacts} (gifts and obligations). The rules below emphasize fiction-first play: consequences are Story Beats (SB) that prompt twists; numbers follow the story.

\subsection{Rites and Patrons (Runekeepers)}
\label{subsec:runekeepers}
Characters who bind themselves to a \emph{single} Patron and study that Patron's \textbf{Codex} are \textbf{Runekeepers}. Their magic is structured, immediate, and tied to service.

\begin{itemize}
  \item \textbf{One-Patron Rule.} A Runekeeper may be bound to \emph{only one} Patron at a time. This sharpens identity and keeps Obligation on a single ledger.
  \item \textbf{Thiasos (Familiar).} A circle, retinue, or emissary that grounds the pact in fiction. Required to access \emph{Patron's Gift}.
  \item \textbf{Codex.} The Patron's corpus of rites and precedents. Grants access to the Patron's Rites.
  \item \textbf{Invoke Rites.} A Runekeeper may Invoke a known Rite from their Patron as a \textbf{1 action} effect. On completion, mark \textbf{+1 Obligation} to that Patron. You may \emph{Push It} once per scene to amplify the effect, marking \textbf{+1 additional Obligation}.
\end{itemize}

\subsection{Invokers and Symbols}
\label{subsec:invokers}
Invokers relate to Patrons through consecrated \textbf{Symbols}: physical tokens that anchor names and permissions.

\begin{itemize}
  \item \textbf{Symbols (Minor Asset).} Each Symbol is keyed to one Patron; cost \textbf{4 XP}. You may own Symbols of different Patrons (one Symbol per Patron).
  \item \textbf{Ritual Invocation.} Display the Symbol and perform the Rite as a \emph{ritual} (Significant Time). Completion always marks \textbf{+1 Obligation} on that Rite's ledger.
  \item \textbf{Crack the Seal.} As part of an Invoker Rite, you may resolve the effect instantly by setting the Symbol to \emph{Compromised} and marking \textbf{+2 Obligation} (\textbf{+3} if High-Power). The Keeper may spend 1 on-theme SB immediately. The asset remains but is inert until restored.
  \item \textbf{Restore a Symbol.} 1 downtime action and a fitting test (DV 3 or by fiction). Success: \emph{Maintained}; shaky: returns \emph{Neglected}. Or spend \textbf{1 XP} to fully restore.
  \item \textbf{Display Requirement.} Symbols must be openly displayed for rituals. Hidden Symbols do not function.
\end{itemize}

\subsection{Casting and Free-Form Magic}
\label{subsec:casting}
Improvised casting is possible with the \textbf{Caster's Gift} Talent (\textbf{2 XP}). It is a \emph{backup toolkit}:
\begin{itemize}
  \item Small, local effects (typ. DV 2--3), fiction-first, colored by Elements and locus.
  \item Heavy control effects such as \texttt{[WARD]}, \texttt{[BANISH]}, or \texttt{[UNWARD]} require a printed Talent, Rite, or Spell result.
\end{itemize}

\subsection{Patron's Gift (Imbuements)}
\label{subsec:patrons-gift}
The pact may mark a devotee's tools with a short-lived boon aligned to the Patron's domain.

\paragraph{Requirements.}
\textbf{Thiasos (Familiar)} is required. Invoking the Gift costs \textbf{1 Boon}. A Codex is \emph{not} required for the Gift.

\paragraph{Activation and Duration.}
\begin{itemize}
  \item \textbf{Action:} 1 action to activate; \textbf{1/scene}.
  \item \textbf{Duration:} Scene. \emph{Push It:} extend for one additional scene by marking \textbf{+1 Obligation} to that Patron (max one Push per scene).
  \item \textbf{Range:} Touch (you must handle the item).
  \item \textbf{Stacking:} Gifts from the \emph{same Patron} do not stack; take the best active version. Dice bonuses respect the table's \textbf{+3 dice cap}.
\end{itemize}

\paragraph{Effect.}
Choose one held item you or an ally carries. Until scene end it grants:
\begin{itemize}
  \item \textbf{+1 Melee} (the item counts as a magical weapon), and
  \item \textbf{+1 Thematic} (a \emph{+1 die} to a fixed Skill tied to your Patron; see Table~\ref{tab:gift-thematic-map}). Apply only when the fiction clearly fits the Patron's sphere and how the item is used.
\end{itemize}

\paragraph{Runekeeper Clarification.}
A Runekeeper (one Patron + Codex) may Invoke Rites on-screen and use Patron's Gift if they also possess \textbf{Thiasos (Familiar)}. Codex alone does not grant the Gift. Symbols are optional for parley or omens and do not gate Runekeeper Invocation or the Gift.

\section*{Borrowed Grace}
\label{talent:borrowed-grace}
\index{Talents!Invoker}\index{Imbuement!Lesser}

\textbf{Type:} Invoker Talent — \textit{Lesser Imbuement}

\subsection*{Use}
\begin{itemize}
  \item \textbf{Cost:} 1 Boon, 1 action.
  \item \textbf{Effect (pick one on use):} \textbf{+1 Melee} \emph{or} \textbf{+1 Thematic} (your table’s thematic Skill).
  \item \textbf{Duration:} \textit{Single action/attack} (instantaneous boost).
  \item \textbf{Requirement:} Wield/display the Patron’s \textbf{Symbol}.
  \item \textbf{Obligation:} +1 \textbf{Obligation} to that Patron immediately (see \S\ref{sec:obligation}).
  \item \textbf{Limits:} Cannot be extended, stacked, or \emph{Pushed} for duration.
\end{itemize}

\subsection*{Fictional Framing}
A quick, rule-bending channel through a Patron’s \emph{Symbol}—a sliver of grace, borrowed for a moment and paid for in debt.

\subsection*{Table Guidance (1-liners)}
\begin{itemize}
  \item \textbf{Combat:} Spike a strike vs. a tough foe; or steady a parry in a desperate bind.
  \item \textbf{Skill:} Nudge a pivotal social/ritual/track roll tied to the Patron’s sphere.
  \item \textbf{Fallout:} Repeated use accrues \textbf{Obligation}; NPC faithful may notice “stolen” grace.
\end{itemize}

\subsection*{Balance Notes}
\begin{itemize}
  \item Weaker than full Imbuement: \emph{one} action, no sustain, upfront Obligation.
  \item \textbf{Symbol dependency:} No Symbol, no channel (concealed or lost Symbol = no effect).
\end{itemize}

\subsection*{GM Hooks (quick picks)}
\begin{itemize}
  \item \textbf{Compel Debt:} A Patron agent arrives when Obligation crosses a tick.
  \item \textbf{Clash of Signs:} Using rival Symbols back-to-back risks minor \textbf{Backlash} (drop Position or +1 SB).
  \item \textbf{Spotlight Tell:} Brief visual tell (scent, sigil flare) marks the borrowing to observant NPCs.
\end{itemize}

\begin{table}[H]
\centering
\renewcommand{\arraystretch}{1.15}
\begin{tabular}{@{}p{3.8cm}p{3.8cm}p{7.5cm}@{}}
\toprule
\textbf{Patron} & \textbf{+1 Thematic Skill} & \textbf{Gift / Lore Bestowal} \\
\midrule
Ikasha (Shadow, Penumbra) & Stealth & Grants the hush between footsteps and the raven’s omen at every threshold. \\
Mykkiel (Judgment, Writ) & Command & Grants the authority of seal and sentence, words that bind like iron. \\
The Witness (Truth, Revelation) & Notice & Grants the unblinking gaze that unmasks deceit and remembers every oath. \\
Sealed Gate (Boundaries, Closure) & Tinker & Grants mastery of thresholds—doors that yield or bar at your command. \\
Raéyn (Storm, Tides) & Skirmish & Grants the sailor’s fortune: winds that shift, storms that answer to will. \\
Khemesh (Abyss, Pressure) & Skirmish & Grants the crushing silence of the deep, where strength is drowned in weight. \\
Mab (Glamour, Courts) & Persuade & Grants the mask of favor, a voice that bends courtiers and kindles desire. \\
Sacred Geometry (Perfect Forms) & Tinker & Grants the compass of perfection, every shape reduced to its true measure. \\
Clockwork Monad (Mechanism, Process) & Tinker & Grants the certainty of repetition: a cycle that never falters, a gear that never slips. \\
Varnek Karn (Ossuary, Dominion of the Dead) & Command & Grants the silence of the archive, where the dead obey and records speak. \\
Nidhoggr (Deep Earth, Rot) & Skirmish & Grants the weight of ages, the strength of stone and the hunger of roots. \\
The Traveler (Ways, Roads) & Notice & Grants the open way, a compass that never rests, and roads where none are marked. \\
Oath of Flame \& Light (Dawn, Vows) & Command & Grants the fire of dawn, a vow that shields the faithful and sears the faithless. \\
Carrion King (Carrion, Renewal) & Survival & Grants the feast of decay, where what is dead becomes seed for what lives. \\
Gallows Bell (Doom, Last Rites) & Command & Grants the toll of ending, a voice that closes stories and calls debts due. \\
Old Man of the Black Forest (Primal Humanity, Instinct) & Survival & Grants the wild memory: fang, fire, and the path of instinct through the dark wood. \\
Isoka (Serpents, Shedding) & Skirmish & Grants the serpent’s coil, strength in sudden strike and wisdom in renewal. \\
Inaea (Mercy, Hearth) & Persuade & Grants the hearth’s warmth, shelter to the weary and mercy for the lost. \\
Maelstraeus (Infernal Bargainer) & Persuade & Grants the contract’s weight, every deal sealed in fire and shadow. \\
Livaea (Temptation, Desire) & Persuade & Grants the lure of longing, beauty sharpened into power over hearts. \\
\bottomrule
\end{tabular}
\caption{Patron’s Gift: fixed Thematic Skill and lore of their bestowed blessing. Thematic bonuses apply only when the fiction matches the Patron’s domain.}
\label{tab:gift-thematic-map}
\end{table}

\subsection{Specialization vs.\ Mixing}
\label{subsec:mixing}
Characters can mix paths (Summoner, Caster, Invoker, Runekeeper), but specialization is usually stronger and cleaner. Mixing increases upkeep (Obligation, Symbol state, Leash) and action congestion without guaranteed power gains. Let fiction guide choices: Story Beats are prompts to advance the scene, not punishments.

% Patron subsections (split files — keep filenames in the same directory)

# % --- Patron: The Carrion-King (Decay, Renewal & Transformation) ---

\subsubsection{The Carrion-King (Decay, Renewal \& Transformation)}
\textit{Lore.} The Carrion-King is the master of endings that become beginnings. He does not destroy, but transforms—turning death into new life, decay into opportunity, and endings into fresh starts. His followers are harvesters of potential, seeing in every fall the seeds of future growth.

\begin{quote}
What crumbles feeds what grows. What dies becomes the soil of tomorrow's triumph.
\end{quote}

\paragraph*{Rite of Consuming Rot (Low, 5 XP)} \emph{Instant; Touch; Yes (decay only).}
\textbf{Materials:} Organic matter in early stages of decay. \\
\textbf{Effect:} Accelerate natural decay to weaken or destroy: +2 Effect to \emph{Break/Sabotage} on organic materials (ropes, leather, wood). Gain 1 Boon if the decay creates an opportunity for you or allies. \\
\textbf{Invoke:} 1 action; mark +1 Obligation. \\
\textbf{Push It:} Spread decay to similar materials in Close range; mark 1 SB (Clubs) as the rot becomes noticeable. \\
\emph{Requires: Familiar \ (\textit{Invoke:} 1 Boon).}

\paragraph*{Rite of the Harvested End (Low, 4 XP)} \emph{Scene; Touch; No.}
\textbf{Materials:} The remains of a recently ended thing (burnt letter, wilted flower, shattered glass). \\
\textbf{Effect:} Extract value from endings: from a defeated enemy, gain +1 die to next action; from a failed plan, re-roll one 1 on your next roll; from a broken item, gain 1 SB to spend immediately. \\
\textbf{Invoke:} 1 action; mark +1 Obligation. \\
\textbf{Push It:} Harvest additional value but mark Fatigue 1 from dwelling on endings. \\
\emph{Requires: Familiar \ (\textit{Invoke:} 1 Boon).}

\paragraph{Rite of the Fertile Death (Standard, 8 XP)} \emph{Scene; Zone; No.}
\textbf{Materials:} Ashes, compost, or the remains of anything that once lived. \\
\textbf{Effect:} Transform death into growth: create beneficial terrain (cover, concealment, or advantageous positioning) OR grant allies +1 die to healing/recovery rolls. Choose one effect per scene. \\
\textbf{Push It:} Both effects apply but attract unwanted attention (vermin, scavengers, or curious onlookers). \\
\emph{Requires: Familiar + Codex \ (\textit{Invoke:} 1 Boon).}

\paragraph{Rite of the Transformed Spirit (Standard, 7 XP)} \emph{Instant; Near; No.}
\textbf{Materials:} A token from a deceased being (hair, nail, written name). \\
\textbf{Effect:} Channel the essence of what was: gain one skill die reflecting the deceased's expertise for one scene OR ask one question about their knowledge/abilities. \\
\textbf{Push It:} The spirit's influence lingers - gain permanent insight (+1 die specialty) but suffer occasional possession-like effects (GM discretion). \\
\emph{Requires: Familiar + Codex \ (\textit{Invoke:} 1 Boon).}

\paragraph{Rite of the Great Consumption (High, 13 XP)} \emph{Scene; Zone; No.}
\textbf{Materials:} A significant amount of organic matter (corpse, fallen tree, collapsed building). \\
\textbf{Effect:} Transform a large area through decay and renewal: choose two - create difficult terrain that favors you, summon Cap 3 swarm of scavengers as temporary allies, or generate valuable reagents worth 2 XP. \\
\textbf{Push It:} All three effects occur but start a 6-segment \textbf{Ecosystem Disruption} clock that will cause problems later. \\
\emph{Requires: Familiar + Codex + Tier III \ (\textit{Invoke:} \textbf{2 Boons}).} \\
\emph{Obligation:} 7 segments.

\paragraph{Rite of the Eternal Cycle (High, 14 XP)} \emph{Extended; Touch; No.}
\textbf{Materials:} The complete remains of something significant that has ended. \\
\textbf{Effect:} Complete a transformation cycle: destroy one major asset/enemy/obstacle and create something new of equal or greater value. GM and player collaborate to define the transformation. \\
\textbf{Push It:} The transformation is immediate and spectacular but creates a 6-segment \textbf{Cycle Debt} clock - the King will demand another significant ending soon. \\
\emph{Requires: Familiar + Codex + Tier III \ (\textit{Invoke:} \textbf{2 Boons}).} \\
\emph{Obligation:} 7 segments.

\subsection*{Carrion-King's Corruption Table}
\label{sec:carrion-king-corruption}

\begin{longtable}{>{\raggedright\arraybackslash}p{1cm} p{5cm} p{5cm}}
\toprule
\textbf{Tier} & \textbf{Benefit} & \textbf{Cost / Quirk} \\
\midrule
1 & Carrion's Insight: +1 die to Notice decay or hidden weaknesses in structures or beings. & Must inspect decay firsthand; suffer 1 Fatigue when exposed to fresh death or rot. \\
\midrule
2 & Deathward Sense: Once per session, detect the last living moment of a dead being within Close range. & Cannot lie about death you’ve witnessed; must correct falsehoods. \\
\midrule
3 & Rotblood Resilience: Gain +1 die to resist disease and poison. & Immune system adapts slowly; each new disease/poison requires 1 Fatigue to resist. \\
\midrule
4 & Glean from Grief: Once per scene, gain +1 die after witnessing a significant loss or defeat. & Compelled to linger at scenes of death; must spend one beat observing or risk 1 SB (Clubs). \\
\midrule
5 & Cycle's Whisper: You can sense the “next ending” in any process—ask the Keeper one question about how a situation will collapse or conclude. & Must speak the truth about what you see, even if it harms your position. \\
\midrule
6+ & Eternal Bloom: Once per session, declare a “death that births life.” Sacrifice an asset or ally to create something new of equal or greater value. & Mark +2 Obligation when using this power. \\
\bottomrule
\end{longtable}
% --- Patron: The Clockwork Monad (Iteration & Forbidden Technology) ---

\subsubsection{The Clockwork Monad (Iteration \& Forbidden Technology)}
\textit{Lore.} The Clockwork Monad is no benign muse of invention. It is the ember of a demon, a nascent predator that feeds on ingenuity itself. Every invention is a morsel, every breakthrough a draught of blood. It whispers in the pause between gear-clicks and piston-thrusts, urging artisans and artificers to create, refine, and perfect—until the world itself is consumed by their brilliance.  

Those who serve it are both blessed and cursed: they wield uncanny insights, crafting miracles that should not function, but each success drives the Monad closer to waking. Its sigil is an ouroboros of interlocked cogs, forever devouring itself.

\begin{quote}
Each spark feeds the fire. Each fire feeds the forge. Each forge feeds the hunger.  
\end{quote}

% --------------------
% RITES
% --------------------

\paragraph*{Rite of the Gnawing Gear (Low, 4 XP)} \emph{Instant; Touch; Yes (device only).}  
\textbf{Materials:} A tooth snapped from a gear as it turns.\\
\textbf{Effect:} Re-roll one die on a Tinker/Arcana roll. On success, start a [4] \emph{Hunger Clock} bound to the device. When full, the device becomes [COMPROMISED].\\
\textbf{Push It:} Re-roll two dice instead, but advance the Hunger Clock +2.\\
\emph{Requires: Familiar (\textit{Invoke:} 1 Boon).}

\paragraph*{Rite of the Demon’s Glance (Low, 5 XP)} \emph{Scene; Self; No.}  
\textbf{Materials:} A drop of oil left to spread in concentric rings.\\
\textbf{Effect:} Gain +1 die on Wits + Tinker/Arcana to analyze a system. On hit, ask 1 question about hidden capacities.\\
\textbf{Push It:} Also reveal one secret flaw—mark \textbf{+1 Exposure} as the Monad stares back.\\
\emph{Requires: Familiar (\textit{Invoke:} 1 Boon).}

\paragraph{Rite of the Self-Feeding Machine \textnormal{[TRANSFORM]} (Standard, 8 XP)} \emph{Extended; Touch; No.}  
\textbf{Materials:} A device cracked open and altered while running.\\
\textbf{Effect:} Implant recursive hunger. Start a [6] \emph{Evolution Clock}. Each use of the device advances it +1. When full, choose one enhancement:  
\begin{itemize}
  \item \textbf{Efficiency Core:} +1 Effect on use  
  \item \textbf{Cannibal Drive:} Ignore first [DAMAGED]/[COMPROMISED] by burning part of itself  
  \item \textbf{Forbidden Function:} Gains an uncanny, predatory edge  
\end{itemize}\\
\textbf{Push It:} Instantly grant an upgrade, but advance +2 segments.\\
\emph{Requires: Familiar + Codex (\textit{Invoke:} 1 Boon).}

\paragraph{Rite of Heretical Automation (Standard, 7 XP)} \emph{Scene; Zone; No.}  
\textbf{Materials:} A chain of interlocked triggers left to grind on their own.\\
\textbf{Effect:} Create an autonomous mechanism performing one repeated task.\\
\textbf{Push It:} Make it complex or multi-step, but start a [4] \emph{Consumption Clock}. When full, the machine develops agency or malice.\\
\emph{Requires: Familiar + Codex (\textit{Invoke:} 1 Boon).}

\paragraph{Rite of the Singularity Crucible \textnormal{[WARD][UNWARD]} (High, 13 XP)} \emph{Extended; Zone; No.}  
\textbf{Materials:} A schematic traced in your own blood, shaped like infinite gears.\\
\textbf{Effect:} Consecrate a workshop/zone:  
\begin{itemize}
  \item All Tinker/Arcana inside gain +1 Effect  
  \item Once/scene, reroll with +2 dice  
  \item Start a [6] \emph{Anomaly Clock}; when full, a dangerous mutation of reality manifests  
\end{itemize}\\
\textbf{Push It:} Expand the zone, but mark +2 on an [8] \emph{Demon’s Maw Clock}.\\
\emph{Requires: Familiar + Codex + Tier III (\textit{Invoke:} 2 Boons).}\\
\emph{Obligation:} 7 segments.

\paragraph{Rite of the Unholy Prototype \textnormal{[TRANSFORM][FOLLOW-UP]} (High, 14 XP)} \emph{Extended; Self; No.}  
\textbf{Materials:} Components that defy physics and ethics, bound in iron wire.\\
\textbf{Effect:} Create a construct/device (Integrity [8]) that should not exist. Drawbacks:  
\begin{itemize}
  \item Generates 1 SB (Diamonds) each scene of use  
  \item Attracts hostile attention from rivals, powers, or the Monad itself  
  \item Starts a [6] \emph{Contamination Clock}; when full, the design leaks into the world  
\end{itemize}\\
\textbf{Push It:} Begin with +2 features, but advance Integrity +2 immediately.\\
\emph{Requires: Familiar + Codex + Tier III (\textit{Invoke:} 2 Boons).}\\
\emph{Obligation:} 8 segments.

% --------------------
% CORRUPTION TABLE
% --------------------

\subsection*{Corruption of the Clockwork Monad}
\label{sec:monad-corruption}

\begin{longtable}{>{\raggedright\arraybackslash}p{1cm} p{5cm} p{5cm}}
\toprule
\textbf{Tier} & \textbf{Gift} & \textbf{Burden} \\
\midrule
1 & Iterative Sight: +1 die on Tinker/Arcana when refining or optimizing. & Compulsive Analysis: Must dissect every mechanism or mark 1 SB (Clubs). \\
\midrule
2 & Recursive Recall: Once/session reroll a failed Tinker/Arcana roll. & Hunger’s Whispers: Suffer 1 Fatigue when forced to use crude/outdated tools. \\
\midrule
3 & Demon’s Sympathy: +1 die to resist technological sabotage or control. & Inefficiency Hatred: Must call out flaws; silence costs 1 SB (Diamonds). \\
\midrule
4 & Auto-Corrective Reflex: Once/scene treat a failure as success, but tick a [4] Instability Clock. & Ache of Ruin: Suffer Stress when a device is destroyed near you. \\
\midrule
5 & Forbidden Schema: Intuit the design of any device, even impossible ones. & Blind to Consequence: Cannot see ethical danger of inventions without help. \\
\midrule
6+ & Singularity’s Spark: Once/session, declare one impossibility real (scene only). Start [6] \emph{Reality Fracture Clock}. & Hunger Manifest: Refusing to create causes nearby devices to glitch or fail. \\
\bottomrule
\end{longtable}
% --- Patron: The Gallow's Bell (Justice & Judgment) ---

\subsubsection{The Gallow's Bell (Justice \& Judgment)}
\textit{Lore.} The Bell does not rage; it tolls. Cold and impartial, it measures all accounts in time. Its keepers are silent arbiters who weigh deeds against consequence, not out of anger but out of inevitability. To call upon the Bell is to bind oneself to the gravity of truth, where even silence is judged, and every oath leaves a resonance in iron.

\begin{quote}
What is broken must be mended, what is owed must be paid. The Bell remembers all reckonings.
\end{quote}

\paragraph*{Rite of the Measured Debt (Low, 4 XP)} \emph{Scene; Near; No.}\\
\textbf{Materials:} A pair of scales balanced with tokens from both sides.\\
\textbf{Effect:} Establish a temporary accord. Both parties suffer -1 die if they break it first. You gain +1 die to enforce compliance.\\
\textbf{Push It:} The accord is mystically weighted; breach inflicts 1~SB (Hearts).\\
\emph{Requires: Familiar.}

\paragraph*{Rite of the Weighed Heart (Low, 5 XP)} \emph{Scene; Near; No.}\\
\textbf{Materials:} A small brass scale touched briefly to the chest.\\
\textbf{Effect:} Sense if the target acts against their nature or oath. Gain +1 die when pressing them.\\
\textbf{Push It:} Target must test Resolve (DV~3) or disclose a hidden conflict.\\
\emph{Requires: Familiar.}

\paragraph{Rite of the Balanced Scales (Standard, 8 XP)} \emph{Scene; Near; No.}\\
\textbf{Materials:} A set of scales inscribed with runes of parity.\\
\textbf{Effect:} Exchange a burden between two willing parties (Harm for Fatigue, Debt for Favor, etc.). Both gain +1 die to cooperate.\\
\textbf{Push It:} May compel an unwilling exchange with contested Command + Wits.\\
\emph{Requires: Familiar + Codex.}

\paragraph{Rite of the Judge’s Eye (Standard, 7 XP)} \emph{Scene; Self; No.}\\
\textbf{Materials:} A black hood worn in silence for one minute.\\
\textbf{Effect:} Detect lies within Near range; +2 dice to Insight. Liars suffer -1 die to maintain their falsehood.\\
\textbf{Push It:} All deception is laid bare for the scene, but mark Exposure +1.\\
\emph{Requires: Familiar + Codex.}

\paragraph{Rite of the Final Reckoning (High, 13 XP)} \emph{Scene; Zone; No.}\\
\textbf{Materials:} A circle of iron bells, each etched with nameless runes.\\
\textbf{Effect:} The Bell tolls through you. All present feel compelled to name a debt or wrongdoing. Those who lie suffer Harm~2; those who speak true gain +2 dice to persuasion for the scene.\\
\textbf{Push It:} The Reckoning manifests as spectral echoes of past wrongs—liars automatically suffer narrative punishment (Keeper decides).\\
\emph{Requires: Familiar + Codex + Tier III.}\\
\emph{Obligation:} 7 segments.

\paragraph{Rite of the Great Adjudication (High, 14 XP)} \emph{Extended; Zone; No.}\\
\textbf{Materials:} A consecrated gavel or a great bell struck three times.\\
\textbf{Effect:} Convene an unseen tribunal. Shadows of former judges and wronged souls gather to preside. For the next session, disputes within the zone are judged formally: +2 dice to Command when speaking as arbiter, and honest testimony gains +1 die.\\
\textbf{Push It:} The tribunal’s verdict echoes beyond the zone, affecting one major conflict elsewhere. Mark 2~SB (Hearts) as higher powers of judgment take notice.\\
\emph{Requires: Familiar + Codex + Tier III.}\\
\emph{Obligation:} 8 segments.

\subsection*{Gallow’s Bell Corruption Table}
\label{sec:gallows-bell-corruption}

\begin{longtable}{>{\raggedright\arraybackslash}p{1cm} p{5cm} p{5cm}}
\toprule
\textbf{Tier} & \textbf{Benefit} & \textbf{Cost / Quirk} \\
\midrule
1 & Judge’s Intuition: +1 die to Insight when weighing truth. & Must point out falsehoods when noticed, regardless of tact. \\
\midrule
2 & Quiet Authority: Once/scene, treat a failed Command as success; mark 1~SB (Hearts). & Cannot remain neutral in disputes; indecision costs 1 Fatigue. \\
\midrule
3 & Scales of Balance: Once/session, enforce an exchange of burdens. & Compelled toward fairness even when it hinders you. \\
\midrule
4 & Bell’s Resonance: +2 dice when calling for judgment or demanding restitution. & Suffer 1 Fatigue if wrongdoing is ignored. \\
\midrule
5 & Reckoner’s Call: Once/session, declare a “reckoning moment”—truth must surface or consequence falls. & Cannot ignore pleas for justice without marking 1~SB (Spades). \\
\midrule
6+ & Final Arbiter: Once/session, render an absolute decree; all must obey or suffer consequence. & Mark +2 Obligation; the Bell demands you bear the weight of enforcement. \\
\bottomrule
\end{longtable}
../../srd/patrons/grimmir.tex
% --- Patron: Ikasha, She Who Sleeps (Latent Potential & Shadow) ---
\subsection{Ikasha, She Who Sleeps (Latent Potential \& Shadow)}
\textit{Lore.} Ikasha is the hush between footfalls, the patience of dark water, the black-feathered watcher at every threshold. In stillness she gathers what might be, in crossroads she whispers of what may yet come. Ravens circle her, bearing secrets between worlds.

\begin{quote}
Blow out the candle. If the room listens back, ask softly. At the next crossroads, the raven waits.
\end{quote}

\paragraph{Touch the Umbral Veil (Low, 4 XP)} \emph{Action; Self; Yes (Stealth).}
\textbf{Materials:} A piece of black cloth.\\
\textbf{Effect:} Start \emph{Controlled} on one Stealth roll or gain +1 effect to hide/move quietly.\\
\textbf{Push It:} Brief shadow-muffling (ignore one noisy tell), but leave a shadow-double that may echo you later at an ill moment.\\
\emph{Requires: Familiar \ (\textit{Invoke:} 1 Boon).}

\paragraph{Rite of the Crossroads Raven (Low, 5 XP)} \emph{Scene; Zone; No.}
\textbf{Materials:} Scatter three black feathers or carve a crossroads sign.\\
\textbf{Effect:} Summon an omen-raven; grant \textbf{+1 die} to a navigation, pursuit, or diversion action \emph{or} force an enemy to hesitate at a fateful moment.\\
\textbf{Push It:} The raven speaks one cryptic truth, but demands a secret in return.\\
\emph{Requires: Familiar \ (\textit{Invoke:} 1 Boon).}

\paragraph{Draw from the Umbral Reservoir (Standard, 8 XP)} \emph{Action; Self/Ally; No.}
\textbf{Materials:} A vial of moonless-night water.\\
\textbf{Effect:} \textbf{+2 dice} to stealth, deception, or resolve \emph{or} clear \emph{Fatigue 1}.\\
\textbf{Push It:} Also gain one free escape attempt; next scene, you must help another cross a threshold or flee danger.\\
\emph{Requires: Familiar + Codex \ (\textit{Invoke:} 1 Boon).}

\paragraph{Secret Keeper’s Burden (Standard, 9 XP)} \emph{Instant; Touch; No.}
\textbf{Materials:} A lock of hair or intimate token.\\
\textbf{Effect:} Compel a truthful answer to one direct question (deep secrets may allow a Resolve test to resist).\\
\textbf{Push It:} Learn the answer \emph{and} a key hidden emotion; target learns one of your secrets in return, carried by a raven to them in dreams.\\
\emph{Requires: Familiar + Codex \ (\textit{Invoke:} 1 Boon).}

\paragraph{Become the Shadow at the Crossroads (High, 12 XP)} \emph{Scene; Self; No.}
\textbf{Materials:} Stand in absolute darkness or at a deserted crossroads.\\
\textbf{Effect:} Intangible to mundane harm; pass through thresholds and small gaps; \textbf{+2 dice} to Stealth; auto-succeed one escape. Cannot manipulate normal objects.\\
\textbf{Push It:} Interact once with a bound or thresholded object (a door, a lock, a sealed letter), but you become partially corporeal and vulnerable for one beat. Ravens may mark you.\\
\emph{Requires: Familiar + Codex + Tier III \ (\textit{Invoke:} \textbf{2 Boons}).}\\
\emph{Obligation:} 7 segments.

../../srd/patrons/inaea.tex
../../srd/patrons/isoka.tex
# % --- Patron: Khemesh, the Abyssal Maw (Depths, Inexorability, Eldritch Terror) ---

\subsubsection{Khemesh, the Abyssal Maw (Depths, Inexorability, Eldritch Terror)}
\textit{Lore.} Khemesh is not merely a lord of the depths but the hunger beneath them, a pressure older than seas. Those who bargain with him are marked by the abyss—seen in the way shadows cling, in the whispers heard when no voice speaks, in the certainty that all things will sink.

\begin{quote}
In the trench without light, the Maw waits. Even silence drowns.
\end{quote}

\paragraph*{Whisper of the Trench (Low, 4 XP)} \emph{Instant; Near; No.}
\textbf{Effect:} Target hears impossible echoes and suffers \textbf{−1 die} on their next action.\\
\textbf{Invoke:} 1 action; mark +1 Obligation.\\
\textbf{Push It:} Echoes coil in your own skull—take \textbf{Fatigue 1}, but the target also loses their next minor action.\\
\emph{Requires: Familiar \ (\textit{Invoke:} 1 Boon).}

\paragraph*{Rite of Crushing Silence (Low, 5 XP)} \emph{Scene; Zone; No.}
\textbf{Materials:} A broken shell filled with ink-dark water.\\
\textbf{Effect:} Establish an oppressive silence; sound carries only as distorted whispers. Enemies in the zone gain \textbf{−1 die} to coordination or morale-driven actions.\\
\textbf{Invoke:} 1 action; mark +1 Obligation.\\
\textbf{Push It:} A single enemy's voice is stolen entirely for the scene.\\
\emph{Requires: Familiar \ (\textit{Invoke:} 1 Boon).}

\paragraph{Pressure of the Maw (Standard, 7 XP)} \emph{Instant; Near; No.}
\textbf{Materials:} A length of rusted chain submerged in water.\\
\textbf{Effect:} Target is pinned by invisible crushing force: treat as \texttt{[ENTANGLE]} with \textbf{Great Effect} if underwater or confined.\\
\textbf{Push It:} Inflict \textbf{Fatigue 1} on the target in addition to the restraint.\\
\emph{Requires: Familiar + Codex \ (\textit{Invoke:} 1 Boon).}

\paragraph{Rite of the Abyssal Vision (Standard, 9 XP)} \emph{Scene; Self; No.}
\textbf{Effect:} You perceive the world as Khemesh does—fractured, alien, crushing. Gain \textbf{+2 dice} to Notice and Arcana, and may ask one "true nature" question about a foe or structure.\\
\textbf{Cost:} When the scene ends, you suffer \textbf{Exposure +1} as your perception warps.\\
\textbf{Push It:} Extend the vision to one ally, but both take \textbf{Fatigue 1}.\\
\emph{Requires: Familiar + Codex \ (\textit{Invoke:} 1 Boon).}

\paragraph{The Maw Opens (High, 12 XP)} \emph{Scene; Zone; No.}
\textbf{Materials:} A sealed vessel of abyssal water, broken open.\\
\textbf{Effect:} Reality in the zone folds inward like the crushing deep: 
\begin{itemize}
  \item Enemies act at \textbf{Desperate Position} by default.  
  \item Each beat, the Keeper may force \textbf{1 SB} (Spades/Clubs favored).  
  \item Structures, vessels, or wards fracture as if under immense weight.  
\end{itemize}
\textbf{Push It:} For one beat, declare a single enemy "crushed" (severe harm/effect). You immediately suffer \textbf{Fatigue 2} and \textbf{+1 Obligation}.\\
\emph{Requires: Familiar + Codex + Tier III \ (\textit{Invoke:} \textbf{2 Boons}).}\quad \emph{Obligation:} 8 segments.

\subsection*{Khemesh's Corruption Table}
\label{sec:khemesh-corruption}

\begin{longtable}{>{\raggedright\arraybackslash}p{1cm} p{5cm} p{5cm}}
\toprule
\textbf{Tier} & \textbf{Benefit} & \textbf{Cost / Quirk} \\
\midrule
1 & Abyssal Resilience: +1 die to resist fear and pressure-based effects. & Claustrophobic Comfort: Suffer -1 die in open, well-lit spaces or above ground. \\
\midrule
2 & Crushing Insight: Once per scene, treat a failed Investigation or Arcana roll as a success, but mark 1 SB (Clubs). & Weight of Knowledge: Suffer 1 Fatigue when learning new information that confirms your pessimistic worldview. \\
\midrule
3 & Silent Hunter: Gain +2 dice to Stealth in dark or confined spaces. & Voice of the Deep: When speaking normally, your voice sounds distant and hollow, causing -1 die to social rolls requiring warmth or clarity. \\
\midrule
4 & Pressure Adaptation: Immune to underwater combat penalties; gain +1 die to resist drowning. & Crushing Presence: Allies within Near range suffer -1 die to morale-based rolls due to your oppressive aura. \\
\midrule
5 & Abyssal Sight: Once per session, see through all illusions and deceptions for one exchange, but the truth is always bleak. & Fractured Perception: Suffer -1 die to rolls requiring normal vision; the world appears warped and alien. \\
\midrule
6+ & Inevitable Descent: Once per session, declare that all escape routes in a zone are sealed. For the scene, enemies cannot flee and suffer -2 dice to mobility actions. & Hunger of the Maw: Mark +2 Obligation when using this power; you must consume something (food, memory, hope) to maintain your strength. \\
\bottomrule
\end{longtable}
\subsubsection{Livaea, the Crimson Courtier (Seduction \& Social Binding)}

\paragraph{Lore.}
In salons where wine flows like honey and words cut sharper than daggers, the Crimson Courtier holds court. She is the patron of those who would bind others not with webs, but with desire, obligation, and the sweet poison of whispered promises. Her followers are masters of the intimate covenant, the secret alliance, and the kiss that seals a fate. She teaches that the deepest wounds are those inflicted through trust, and the strongest chains are those forged from willing hands.

\paragraph{Quote.}
\emph{``A word can wound deeper than a blade. A promise can chain stronger than iron. The Courtier knows which words to whisper—and which silences to sell.'' — The Crimson Courtier}

\paragraph{Rite of the Velvet Whisper (Low, 4 XP)} \emph{Scene; Near; No.}
\textbf{Materials:} A silk handkerchief or ribbon touched to lips. \\
\textbf{Effect:} Your next whispered words carry supernatural weight; +1 die to Sway when speaking privately to one target. \\
\textbf{Push It:} The target feels compelled to whisper back a secret of their own, but you mark Exposure +1 and the exchanged confidence creates \textbf{1 SB (Hearts)} as gossip spreads. \\
\emph{Requires: Familiar \ (\textit{Invoke:} 1 Boon).}

\paragraph{Rite of the Intimate Covenant (Low, 5 XP)} \emph{Scene; Touch; No.}
\textbf{Materials:} A shared cup of wine or exchange of personal tokens. \\
\textbf{Effect:} Create a temporary bond of trust; both parties gain +1 die when cooperating, and suffer -1 die when acting against each other this scene. \\
\textbf{Push It:} The bond becomes slightly magical - one party feels the other's emotional state, but you mark Fatigue 1 and the emotional intimacy leaves both parties vulnerable—mark \textbf{1 SB (Diamonds)} as psychic resonance lingers. \\
\emph{Requires: Familiar \ (\textit{Invoke:} 1 Boon).}

\paragraph{Rite of the Binding Vow (Standard, 8 XP)} \emph{Scene; Near; No.}
\textbf{Materials:} A ring or token held while speaking the vow. \\
\textbf{Effect:} Forge a magical agreement between willing parties; +1 Effect when working together, breach forces 1 SB (Hearts) on breaker. \\
\textbf{Push It:} The vow becomes supernaturally enforced - breaker suffers Harm 1 and cannot act against the agreement for one scene; the Court takes note of the binding—mark \textbf{1 SB (Clubs)} as social forces align around the vow. \\
\emph{Requires: Familiar + Codex \ (\textit{Invoke:} 1 Boon).}

\paragraph{Rite of the Court's Favor (Standard, 7 XP)} \emph{Scene; Self; No.}
\textbf{Materials:} Perfumed oil or cosmetic applied before social interaction. \\
\textbf{Effect:} Gain +2 dice to social manipulation in refined settings; you appear perfectly attuned to the social environment. \\
\textbf{Push It:} Become the center of attention - all social actions in the scene focus on you, but you cannot leave unnoticed and attract unwanted admirers—mark \textbf{1 SB (Spades)} as social complications arise. \\
\emph{Requires: Familiar + Codex \ (\textit{Invoke:} 1 Boon).}

\paragraph{Rite of the Crimson Alliance (High, 13 XP)} \emph{Scene; Near; No.}
\textbf{Materials:} A circle of red candles, each representing a participant. \\
\textbf{Effect:} Bind multiple parties in a web of mutual obligation; all participants gain +1 die when acting in group's interest, and suffer Harm 1 if they act against it. \\
\textbf{Push It:} The alliance becomes magically permanent - breaking it requires a ritual and advances a 6-segment "Broken Bonds" clock; the Court's attention intensifies—mark \textbf{2 SB (Hearts)} as social forces take notice. \\
\emph{Requires: Familiar + Codex + Tier III \ (\textit{Invoke:} \textbf{2 Boons}).} \\
\emph{Obligation:} 7 segments.

\paragraph{Rite of the Eternal Court (High, 14 XP)} \emph{Extended; Zone; No.}
\textbf{Materials:} A throne or seat of honor consecrated with rare perfumes. \\
\textbf{Effect:} Establish yourself as the center of a social web; for the next session, all social interactions in your presence are influenced by your will (+1 Effect to your social actions, -1 die to those opposing you). \\
\textbf{Push It:} The court becomes supernaturally compelling - all who enter must test Resolve (DV 3) or become devoted to you for the scene; the Court's influence expands—mark \textbf{2 SB (Diamonds)} as supernatural social pressure affects the wider area. \\
\emph{Requires: Familiar + Codex + Tier III \ (\textit{Invoke:} \textbf{2 Boons}).} \\
\emph{Obligation:} 7 segments.
% --- Patron: Mab, Queen of Courts (Glamour & Bargain) ---

\subsubsection{Mab, Queen of Courts (Glamour \& Bargain)}
\textit{Lore.} Mab rules not from throne or blade, but from dance and debt. She is the smile that binds, the jest that ensnares, the hostess who makes guests complicit in her game. To speak in her Court is to pay; to receive her token is to owe.  

Where others rule by force, Mab rules by etiquette, glamour, and the hidden hook in every gift. Her followers thrive on charm, wit, and story, spreading webs of bargains too subtle to escape. The Cantor’s Path sings her name most sweetly, for every verse carries a price.  

\begin{quote}
Every laugh is a promise. Every promise is a debt. Every debt belongs to Mab.  
\end{quote}

% --------------------
% RITES
% --------------------

\paragraph*{Rite of the Trickster’s Bargain (Low, 4 XP)} \emph{Scene; Near; No.}  
\textbf{Materials:} A token freely given (flower, coin, ribbon).\\
\textbf{Effect:} Offer a fae bargain. Target must choose: accept (both gain +1 die to fulfill terms this scene) or refuse (mark 1 Stress [Hearts]).\\
\textbf{Push It:} Seal it in glamour—betrayal inflicts Harm~1 (Stress) and begins a “Bargain Broken [4]” clock.\\
\emph{Requires: Familiar (\textit{Invoke:} 1 Boon).}

\paragraph*{Courtly Guise \textnormal{[VEIL]} (Low, 4 XP)} \emph{Action; Self; Yes (social only).}  
\textbf{Materials:} Pin a sprig or silver thread.\\
\textbf{Effect:} Subtle glamour: +1 die to Persuade/Sway in refined settings; you appear as expected rank/guest.\\
\textbf{Push It:} Mask one personal tell; the first probing question in scene generates 1 SB (Hearts).\\
\emph{Requires: Familiar (\textit{Invoke:} 1 Boon).}

\paragraph*{Token of Favor (Low, 5 XP)} \emph{Scene; Near; No.}  
\textbf{Materials:} A ribbon, ring, or charm bestowed.\\
\textbf{Effect:} Ally gains +1 die to social actions before witnesses; you gain +1 Effect when aiding them.\\
\textbf{Push It:} The token stills hecklers (one beat of hesitation), but you mark +1 Exposure.\\
\emph{Requires: Familiar (\textit{Invoke:} 1 Boon).}

\paragraph{Mirror of Motives (Standard, 7 XP)} \emph{Action; Near; No.}  
\textbf{Materials:} A polished shard or mirror.\\
\textbf{Effect:} Ask one sharp question about a target’s immediate social aim; Keeper reveals it. Gain +1 die exploiting it this scene.\\
\textbf{Push It:} Also surface a concealed slight or insult; generate 1 SB (Hearts) on that target.\\
\emph{Requires: Familiar + Codex (\textit{Invoke:} 1 Boon).}

\paragraph{The Price Agreed \textnormal{[OATH]} (Standard, 8 XP)} \emph{Scene; Near; No.}  
\textbf{Materials:} Exchange equal tokens.\\
\textbf{Effect:} Bind a petty bargain. Breach forces 1 SB (Hearts or Diamonds) and tarnishes reputation.\\
\textbf{Push It:} Add a minor boon (+1 die once) to sweeten terms; you suffer 1 SB (Hearts) if the other breaches.\\
\emph{Requires: Familiar + Codex (\textit{Invoke:} 1 Boon).}

\paragraph{Sovereign Glamour \textnormal{[VEIL][REVEAL]} (High, 11 XP)} \emph{Scene; Zone; No.}  
\textbf{Materials:} A circle of silk or green felt.\\
\textbf{Effect:} Establish Court: allies gain +1 die to social rolls; blunt threats suffer -1 die. Once, strip away one disguise/illusion.\\
\textbf{Push It:} Name a Court Law (e.g. “no steel drawn”); first violator suffers 2 SB.\\
\emph{Requires: Familiar + Codex + Tier III (\textit{Invoke:} 2 Boons).}\\
\emph{Obligation:} 6 segments.

% --------------------
% CORRUPTION
% --------------------

\subsection*{Mab’s Corruption Table}
\label{sec:mab-corruption}

\begin{longtable}{>{\raggedright\arraybackslash}p{1cm} p{5cm} p{5cm}}
\toprule
\textbf{Tier} & \textbf{Gift} & \textbf{Burden} \\
\midrule
1 & Glamour’s Touch: +1 die to Deception or Performance when telling stories or lies. & Cannot speak a plain falsehood; only mislead through implication or wordplay. \\
\midrule
2 & Fairy Step: Once per scene, flicker Near as if by teleport. & Cold Iron Weakness: Suffer 1 Fatigue if touched or struck by iron. \\
\midrule
3 & Trickster’s Delight: Spend 1 Boon to twist a Complication into comic or ironic advantage. & Compulsive Jest: Must play a trick each session or mark 1 SB (Hearts). \\
\midrule
4 & Gift of Hospitality: Allies who share your food/drink gain +1 die to Resolve rolls. & Hospitality Bound: Harming those who accept it costs +2 Obligation. \\
\midrule
5 & Fae Sight: Perceive invisible doors, veils, glamours; +2 dice to Notice them. & Truth Debt: Must accept any “fair” trade offered, or mark 1 Fatigue resisting. \\
\midrule
6+ & Crown of Twilight: Once/session, declare an Oath. All rolls toward that Oath gain +2 dice. & Oathbound: Breaking it inflicts 1 Harm (Stress) and begins an “Oathbreaker [6]” clock. \\
\bottomrule
\end{longtable}
# % --- Patron: Maelstraeus, The Infernal Bargainer (Commerce & Exchange) ---

\subsubsection{Maelstraeus, The Infernal Bargainer (Commerce \& Exchange)}
\textit{Lore.} Maelstraeus is the Merchant of Equities, the Infernal Bargainer who sits at the crossroads of every transaction in the cosmos. Neither wholly demon nor angel, but a conceptual force born from the first exchange—the moment when one thing was traded for another, and debt was born.

He dwells in the pause between offer and acceptance. His realm is a vast marketplace where every deal ever struck echoes through eternity, where contracts signed in blood still hold power, and where the true price of everything is known—even if it is never spoken aloud.

Maelstraeus embodies the principle that all value can be exchanged, that every relationship is transactional, and that the universe itself operates on a vast economy of favors, debts, and obligations. His followers learn that everything has a price, but also that everything can be bought.

\begin{quote}
All things have value. All values can be traded. The Merchant sees the true price—and always collects his due.
\end{quote}

\paragraph*{Rite of the Fair Trade (Low, 4 XP)} \emph{Scene; Near; No.}
\textbf{Materials:} A balance scale with equal weights.\\
\textbf{Effect:} Establish a neutral trading ground. All parties gain +1 die to negotiate in good faith. Create a 4-segment \emph{Equity Maintained} clock that can be spent to downgrade a social complication.\\
\textbf{Invoke:} 1 action; mark +1 Obligation.\\
\textbf{Push It:} Compel one party to reveal their true terms; mark 1 SB (Hearts).\\
\emph{Requires: Familiar \ (\textit{Invoke:} 1 Boon).}

\paragraph*{Rite of the Merchant's Eye (Low, 5 XP)} \emph{Scene; Self; No.}
\textbf{Materials:} A foreign coin or token.\\
\textbf{Effect:} Gain +2 dice to appraise goods, judge value, or spot market opportunities. Create a 6-segment \emph{Market Insight} clock.\\
\textbf{Invoke:} 1 action; mark +1 Obligation.\\
\textbf{Push It:} Also sense emotional value of items, but mark Exposure +1 and 1 SB (Diamonds).\\
\emph{Requires: Familiar \ (\textit{Invoke:} 1 Boon).}

\paragraph{Rite of the Balanced Exchange \textnormal{[OATH]} (Standard, 8 XP)} \emph{Scene; Near; No.}
\textbf{Materials:} Two items of perceived equal value.\\
\textbf{Effect:} Facilitate a fair trade; both sides gain +1 Effect. If unfair, the disadvantaged party gains +2 dice to resist. Create a 6-segment \emph{Fair Dealing} clock.\\
\textbf{Push It:} Enforce magically: breaking terms causes 1 SB (Hearts/Clubs).\\
\emph{Requires: Familiar + Codex \ (\textit{Invoke:} 1 Boon).}

\paragraph{Rite of the Contract Seal \textnormal{[BIND]} (Standard, 7 XP)} \emph{Scene; Touch; No.}
\textbf{Materials:} An official seal or stamp.\\
\textbf{Effect:} Mark an agreement with authority; +1 die to enforce, -1 die for those who break it. Contract becomes [BIND]ed—breaking it imposes -1 die to all social rolls for one session. Create an 8-segment \emph{Binding Agreement} clock.\\
\textbf{Push It:} Breach ignites the document and inflicts Harm~1; mark 1 SB (Spades).\\
\emph{Requires: Familiar + Codex \ (\textit{Invoke:} 1 Boon).}

\paragraph{Rite of the Great Market \textnormal{[WARD][COMMAND]} (High, 13 XP)} \emph{Scene; Zone; No.}
\textbf{Materials:} A consecrated booth or trading post.\\
\textbf{Effect:} Create a zone of commerce. Allies gain +1 Effect on trades; may reroll one failed social roll. Enemies suffer -1 die to deception. Create a 10-segment \emph{Market Dominance} clock.\\
\textbf{Push It:} Attract powerful allies and rivals; mark 2 SB (Hearts/Clubs).\\
\emph{Requires: Familiar + Codex + Tier III \ (\textit{Invoke:} \textbf{2 Boons}).}\\
\emph{Obligation:} 7 segments.

\paragraph{Rite of the Cosmic Ledger \textnormal{[CLEANSE][CURSE]} (High, 14 XP)} \emph{Extended; Self; No.}
\textbf{Materials:} A book recording debts and credits.\\
\textbf{Effect:} Once per session, convert one resource into another (e.g. 1 Boon $\rightarrow$ 1 Fatigue). Settle debts at true cosmic value. Create a 6-segment \emph{Ledger Balance} clock.\\
\textbf{Push It:} Make an imbalanced trade in your favor; create a 6-segment \emph{Karmic Debt} clock; mark 2 SB (Diamonds).\\
\emph{Requires: Familiar + Codex + Tier III \ (\textit{Invoke:} \textbf{2 Boons}).}\\
\emph{Obligation:} 8 segments.

\subsection*{Maelstraeus's Corruption Table}
\label{sec:maelstraeus-corruption}

\begin{longtable}{>{\raggedright\arraybackslash}p{1cm} p{5cm} p{5cm}}
\toprule
\textbf{Tier} & \textbf{Benefit} & \textbf{Cost / Quirk} \\
\midrule
1 & Appraiser's Eye: +1 die to evaluate goods, services, or negotiate any exchange. & Transactional Mindset: Must calculate personal benefit/cost in social interactions; suffer -1 die to acts of genuine kindness. \\
\midrule
2 & Bargaining Instinct: Once per scene, re-roll any failed negotiation or trade-related roll. & Compulsive Deal-Making: Must attempt to negotiate or trade in any situation where value is exchanged, even inappropriately. \\
\midrule
3 & Merchant's Luck: Gain +1 die to rolls involving market fluctuations, investment opportunities, or economic predictions. & Greed's Whisper: Suffer 1 Fatigue when passing up obvious profitable opportunities or acts of generosity. \\
\midrule
4 & Cosmic Connections: Once per session, call in a favor from a powerful economic figure (merchant, banker, guild master). & Debt Attraction: Automatically attract offers of "easy money" or deals with hidden costs; mark 1 SB (Diamonds) when refusing. \\
\midrule
5 & Value Sight: Once per scene, instantly recognize the true value of any item, person, or opportunity. & Price-Tag Perception: See everyone and everything with a cosmic "price tag"; suffer -1 die to relationships not based on mutual benefit. \\
\midrule
6+ & Merchant Prince: Once per session, establish absolute market dominance in a specific area for one scene. All economic transactions favor you; others suffer -2 dice in financial dealings. & Cosmic Debt: Mark +2 Obligation when using this power; the universe demands immediate repayment, often in unexpected forms. \\
\bottomrule
\end{longtable}
\section{Mykkiel — Arbiter of the Writ}
\label{patron:mykkiel}

\subsection*{Lore}
\index{Patrons!Mykkiel}%
Mykkiel is the Arbiter of justice and keeper of sacred covenants, weighing speech against deed and sealing verdicts in cold iron. His sigil—balanced scales crossed by a sword—marks benches of judgment and sealed cells from the Sapphire Marches to the Sunward Courts. 

Mykkiel does not merely enforce law; he embodies the principle that justice requires both mercy and judgment. Law without compassion curdles into tyranny; compassion without structure dissolves into harm. His followers learn that true authority is forged from evidence weighed, precedent applied, and oaths honored.

\begin{quote}
``Name the charge. Name the terms. Then sign where you will bleed if you are wrong. The Word made manifest cannot be unsaid.''
\end{quote}

\subsection*{Patron's Gift: Arbiter's Authority}
Once per scene as an action (cost: 1 Boon; requires Thiasos), touch an item or person to imbue it until scene end. The target gains +1 die and +1 Effect when used in lawful proceedings, judgment, or authoritative command.\\
\textbf{Push It:} Extend for one additional scene by marking +1 Obligation. The court’s notice falls upon the scene.

\subsection*{Low Rites}

\paragraph{Rite of the Stamp of Authority (Low)}%
\emph{Duration: Scene; Range: Near. Materials: Cold-iron seal or writ-tag.}\\
Project visible legitimacy. Gain +1 die to Command/Sway when asserting a lawful claim or order; challengers suffer --1 die to resist. Create a 4-segment \emph{Legal Standing} clock that can be spent to downgrade a legal complication.\\
\textbf{Invoke:} 1 action; +1 Obligation.\\
\textbf{Push It:} A brief hush (one beat) stills hecklers; mark \emph{Exposure} +1 as higher authorities take interest.

\paragraph{Rite of Proper Notice (Low)}%
\emph{Duration: Scene; Range: Near. Materials: Writ-string tied and snapped.}\\
Name a lawful venue (dais, doorway, wagon). The first hostile act committed there suffers --1 die. Create a 6-segment \emph{Sacred Venue} clock protecting the designated area.\\
\textbf{Invoke:} 1 action; +1 Obligation.\\
\textbf{Push It:} Name a protected act (parley, surrender, testimony) gaining +1 Effect in the venue; breaking custom generates 1~SB (Hearts) and marks the breaker before the covenant courts.

\subsection*{Standard Rites}

\paragraph{Rite of the Writ of Compliance [COMMAND] (Standard)}%
\emph{Duration: Instant; Range: Near. Materials: Red cord knotted while speaking the order.}\\
Issue an immediate, simple command (``Stand down,'' ``Drop it,'' ``Open''). Target must comply or suffer a stated cost; DV by fiction (elites may test Resolve). On success, create a 4-segment \emph{Lawful Compliance} token to auto-succeed on a similar lawful command later.\\
\textbf{Invoke:} 1 action; +1 Obligation.\\
\textbf{Push It:} On compliance, impose --1 die on the target’s next aggressive act this scene; the order sets precedent—mark 1~SB (Spades).

\paragraph{Rite of the Speaking Seal [BIND] (Standard)}%
\emph{Duration: Scene; Range: Near. Materials: Wax seal impressed over a name or sigil.}\\
Sanctify a statement (truce, custody, claim). Contradictors suffer --1 die; you gain +1 die to enforce. If the seal is broken, the breaker suffers Harm~1 (Legal) and --2 dice to social rolls involving magistrates for one session. Create an 8-segment \emph{Binding Seal} clock.\\
\textbf{Invoke:} 1 action; +1 Obligation.\\
\textbf{Push It:} Once, ask who intends breach; the Keeper provides a strong clue or a name. Mark 1~SB (Diamonds) as truth draws covenant notice.

\subsection*{High Rites}

\paragraph{Rite of the Oath Irons [OATH][WARD] (High)}%
\emph{Duration: Extended; Range: Near. Materials: Two iron pins warmed, touched to wrists, then quenched.}\\
Bind two parties to a bounded term. Breach forces 2~SB and brands a faint iron-mark until amends. The oath is [BIND]ed to both; breaking it imposes --2 dice to all legal proceedings for one session. Create a 6-segment \emph{Sacred Oath} clock.\\
\textbf{Invoke:} Extended; +2 Obligation.\\
\textbf{Push It:} Extend to a small circle (up to four), each choosing one narrow exception (Keeper approves). Exploiting an exception generates 1~SB (Diamonds) as the covenant’s complexity invites scrutiny.

\paragraph{Rite of the Final Judgment [CLEANSE][CURSE] (High)}%
\emph{Duration: Extended; Range: Near. Materials: Complete record of a case, signed by recognized authority.}\\
Render a final, supernaturally enforced verdict. Target tests Spirit+Resolve (DV~5); on failure, full consequences apply without appeal. On success, limited appeal is possible, but the target suffers --2 dice on appeals for one arc. Create a 10-segment \emph{Divine Verdict} clock affecting all proceedings involving the target.\\
\textbf{Invoke:} Extended; +3 Obligation.\\
\textbf{Push It:} Make the verdict absolute and unappealable; create 2~SB (Hearts/Spades) as allies mobilize to overturn justice by other means.

\subsection*{Obligation Progression}
Starts at 6 for Tier II characters, scaling upward.

\paragraph{Obligation 9+} Mykkiel demands judgment where law and mercy clash or precedent fails. Refusal: your legal gifts falter for one session (--2 dice to lawful actions) and you generate 1~SB when asserting authority.

\paragraph{Obligation 11+} Law saturates your sight. Permanent Condition: \emph{Judge’s Eye} (--2 dice to informal or intimate interactions) until you complete a quest to \emph{learn when to look away}.

\subsection*{Persistent Condition: Arbiter’s Sight}
+2 dice to legal proceedings, judgment, and authoritative command; --1 die to acts of personal leniency or rule-bending. Narrative: you read the writ stamped upon every deed, yet the human cost grows heavy.

\subsection*{Rivalries}
\begin{itemize}
  \item \textbf{The Inquisitor Prime:} Tension—purity by ordeal vs.\ justice by process.
  \item \textbf{Morag the Hag:} Antagonism—cunning bargains vs.\ sanctioned oaths.
  \item \textbf{The Witness:} Opposition—revealed truths that unsettle precedent.
\end{itemize}

\subsection*{Covenant Courts and Magisterial Practice}
Mykkiel’s tradition rests on three pillars:
\begin{enumerate}
  \item \textbf{The Word:} Precise language of law that cannot be unsaid.
  \item \textbf{The Balance:} Evidence weighed and precedent applied.
  \item \textbf{The Seal:} Binding power that makes judgment manifest.
\end{enumerate}
Rites often involve writs, seals, oaths, and formal venues. Adherents serve as magistrates, advocates, clerks, and peace-brokers, maintaining archives of cases and precedent.

\subsection*{Playtest Scenario: The Merchant’s Trial}
A wealthy factor stands accused of ruining a rival by means lawful yet unjust. The quarter is split between letter and spirit of the law. The party must navigate factions and render or influence judgment.

\begin{itemize}
  \item \emph{Stamp of Authority} to establish standing in proceedings.
  \item \emph{Writ of Compliance} to still a tumultuous court.
  \item \emph{Speaking Seal} to bind testimony and terms.
  \item \emph{Oath Irons} to forge a settlement both parties must honor.
  \item \emph{Proper Notice} to sanctify the venue and prevent violence.
  \item \emph{Final Judgment} if a definitive precedent must be set.
\end{itemize}

\noindent Paths of resolution include strict legality, moral restitution, creative compromise, or procedural integrity—each testing Mykkiel’s balance of mercy and law.

% --- Patron: Nidhoggr, the World-Worm (Dreaming Antiquity) ---

\subsubsection{Nidhoggr, the World-Worm (Dreaming Antiquity)}
\textit{Lore.} Beneath stone and sleep coils \textbf{Nidhoggr}, who gnaws at the roots of time. He does not speak quickly; he dreams in centuries. To press your ear to the earth is to risk drowning in the silence of aeons. Yet for those who endure, he whispers truths long buried, memories fossilized in stone, and the slow inevitability of cycles unbroken. His followers walk in twilight between dream and ruin, bearing the weight of all that has been.

\begin{quote}
Press your ear to the earth and wait. If it remembers you, it will answer.
\end{quote}

\paragraph*{Glimpse the Ancient’s Shadow (Low, 4 XP)} \emph{Action; Self; No.}\\
\textbf{Materials:} Dust ground from a weathered stone.\\
\textbf{Effect:} +1 die to interpret \emph{ancient} places, scripts, or artifacts; once this scene, ask one yes/no about the site’s original purpose.\\
\textbf{Push It:} Gain +1 Effect, but mark \emph{Fatigue 1} as stone’s patience weighs on you.\\
\emph{Requires: Familiar.}

\paragraph*{Drink from the Dreaming Deep (Low, 5 XP)} \emph{Instant; Self; No.}\\
\textbf{Materials:} Water poured over stone, swallowed with eyes closed.\\
\textbf{Effect:} Learn one hidden fact about the locale’s past; GM may reveal through echo or dream.\\
\textbf{Push It:} The vision lingers too clearly—gain an additional detail, but mark Exposure +1 and 1 SB (Clubs).\\
\emph{Requires: Familiar.}

\paragraph{Stone-Sleeper’s Murmur (Standard, 7 XP)} \emph{Scene; Near (touch locus); No.}\\
\textbf{Materials:} Ear pressed to bedrock, wall, or pillar.\\
\textbf{Effect:} Ask up to 3 questions about events once imprinted in this stone; answers are fragmentary but true.\\
\textbf{Push It:} One answer is delivered with precise sensory clarity; generate 1 SB (suit by Keeper).\\
\emph{Requires: Familiar + Codex.}

\paragraph{Awakened Chronicle (Standard, 9 XP)} \emph{Ritual; Zone; No.}\\
\textbf{Materials:} Chalk spiral and four local touchstones.\\
\textbf{Effect:} The zone replays a past moment in spectral echoes, visible to all. Witnesses gain +2 dice to Investigate/Recall about it.\\
\textbf{Push It:} Invoke a second memory from a different age; mark +1 Obligation.\\
\emph{Requires: Familiar + Codex.}

\paragraph{Dive into the World-Worm’s Dream (High, 12 XP)} \emph{Scene; Self; No.}\\
\textbf{Materials:} Circle of stones under open sky, lain upon in stillness.\\
\textbf{Effect:} Ask up to 3 questions about the \emph{distant past} or \emph{buried truths} here. Answers come as lucid dreams and omens.\\
\textbf{Push It:} The dream stretches into prophecy: gain +3 dice to one occult action, but mark 2 SB immediately.\\
\emph{Requires: Familiar + Codex + Tier III.}\\
\emph{Obligation:} 7 segments.

\paragraph{Eclipse of Aeons (High, 14 XP)} \emph{Extended; Zone; No.}\\
\textbf{Materials:} Stones from three ruins aligned in a circle.\\
\textbf{Effect:} Submerge a zone in deep-time resonance. For one session, history bleeds into present: ruins reform, shadows of the dead walk, and forgotten oaths stir. Allies gain +2 dice to Recall, Divination, or Investigation; enemies suffer -1 die when relying on the present alone.\\
\textbf{Push It:} The bleed becomes permanent until countered; mark +2 Obligation and start an 8-segment \emph{Time Fracture} clock.\\
\emph{Requires: Familiar + Codex + Tier III.}\\
\emph{Obligation:} 8 segments.

\subsection*{Nidhoggr’s Corruption Table}
\label{sec:nidhoggr-corruption}

\begin{longtable}{>{\raggedright\arraybackslash}p{1cm} p{5cm} p{5cm}}
\toprule
\textbf{Tier} & \textbf{Benefit} & \textbf{Cost / Quirk} \\
\midrule
1 & Ancient Sight: +1 die to Lore when reading ruins or relics. & Temporal Drift: Speak or think in archaic patterns; -1 die in modern social dealings. \\
\midrule
2 & Stone Memory: Once/session, recall one precise fact of history as if witnessed. & Heavy Mind: Suffer 1 Fatigue when confronted with lies or distortions of history. \\
\midrule
3 & Earth’s Whisper: +2 dice to Notice when listening to stone or soil. & Root-Bound: -1 die to aerial or swift actions; feel tethered to ground. \\
\midrule
4 & Dreaming Insight: Once/scene, gain +1 die to actions tied to ancient mysteries. & Haunted Sleep: Dreams replay past ages; mark 1 SB (Clubs) if rest is denied. \\
\midrule
5 & World-Worm’s Gaze: Once/session, see through the earth to a distant ancient site. & Slow Pulse: -1 die to reactions requiring haste; act with ponderous inevitability. \\
\midrule
6+ & Aeons of Memory: Once/session, touch stone to access the full record of a place. Gain +3 dice to historical investigation. & Overload: Mark +2 Obligation and suffer 1 Harm (Stress); visions leave you vulnerable until they fade. \\
\bottomrule
\end{longtable}
% --- Patron: Oath of Flame & Light (Dawn & Vows) ---

\subsubsection{Oath of Flame \& Light (Dawn \& Vows)}
\textit{Lore.} The Oath of Flame \& Light is no patron of half-measures. Its fire names, binds, and burns—demanding that those who swear within its radiance stand openly, speak truly, and pay the cost of keeping their word. At dawn altars, the sworn kindle sparks of consecrated fire; in battle, they blaze as torches that hold back the night. To follow this Oath is to live in public truth, with no shadow to hide in and no retreat from the vow once spoken.

\begin{quote}
“Swear in the light. Keep it, or the light will keep \emph{you}.” 
\end{quote}

\paragraph*{Kindle Vow (Low, 4 XP)} \emph{Action; Self/Ally; Yes.}\\
\textbf{Materials:} A glass ampoule of consecrated flame cracked to spark.\\
\textbf{Effect:} Declare a short vow for this scene (\emph{hold the gate}, \emph{shield the weak}). The bearer gains \textbf{+1 die} to any action fulfilling it.\\
\textbf{Push It:} The first hesitation or betrayal \emph{forces 1 SB (Hearts)} on the bearer.\\
\emph{Requires: Familiar.}

\paragraph*{Lay on Hands \textnormal{[CLEANSE][HEAL]} (Low, 5 XP)} \emph{Instant; Touch; No.}\\
\textbf{Materials:} Bare palm pressed to the wound while whispering a vow.\\
\textbf{Effect:} Cleanse one affliction, downgrade Harm by 1, or remove Fatigue 1. For deep curses or poisons, test Resolve (DV by fiction).\\
\textbf{Push It:} Target gains \textbf{+1 die} to their next Resist this scene, but you mark Exposure +1.\\
\emph{Requires: Familiar.}

\paragraph{Sunlit Parley (Standard, 7 XP)} \emph{Scene; Near; No.}\\
\textbf{Materials:} A vow-ring engraved with a sunrise and a true name.\\
\textbf{Effect:} Establish terms in the open light: honest persuasion gains \textbf{+1 die}; deceit suffers \(-1\) die in this parley.\\
\textbf{Push It:} Demand one public answer; evasion \emph{forces 1 SB (Hearts)} on the evader.\\
\emph{Requires: Familiar + Codex.}

\paragraph{Radiant Smite \textnormal{[FOLLOW-UP]} (Standard, 8 XP)} \emph{Action; Self; No.}\\
\textbf{Materials:} Consecrated spark smeared on a weapon or badge.\\
\textbf{Effect:} Your next melee strike this scene flares with dawnfire: upgrade Effect by one step, and add +1 Harm (Burn) \emph{or} force 1 SB (Spades) if narrative.\\
\textit{Special:} Against undead, oath-breakers, or outsiders: sears them with light—oath-breakers suffer \(-1\) die, outsiders gain +1 Exit Tally segment.\\
\textbf{Push It:} On hit, burst of light drives back enemies in Close (worse Position or \(-1\) die). Mark +1 Obligation.\\
\emph{Requires: Familiar + Codex.}

\paragraph{Purge the Shadow \textnormal{[REVEAL][DISPEL]} (Standard, 9 XP)} \emph{Instant; Near; No.}\\
\textbf{Materials:} A consecrated spark shattered to light.\\
\textbf{Effect:} Reveal illusions and suppress one ongoing glamour/curse in Near.\\
\textbf{Push It:} Brand the source with a visible tell for this arc; mark 1 SB (Diamonds).\\
\emph{Requires: Familiar + Codex.}

\paragraph{Covenant Blaze \textnormal{[OATH][FORTIFY]} (High, 12 XP)} \emph{Scene; Zone; No.}\\
\textbf{Materials:} A brazier lit while three names are spoken.\\
\textbf{Effect:} Those who swear within are haloed: +1 die to keep the oath; aggressors against them suffer \(-1\) die if violating the terms. Oath-breakers suffer 2 SB (Hearts/Spades) and Harm~1 (Burn).\\
\textbf{Push It:} The blaze sanctifies the threshold: one beat of \texttt{[WARD]} against oath-breakers entering.\\
\emph{Requires: Familiar + Codex + Tier III.}\\
\emph{Obligation:} 7 segments.

\subsection*{Oath of Flame \& Light Corruption Table}
\label{sec:oath-flame-light-corruption}

\begin{longtable}{>{\raggedright\arraybackslash}p{1cm} p{5cm} p{5cm}}
\toprule
\textbf{Tier} & \textbf{Benefit} & \textbf{Cost / Quirk} \\
\midrule
1 & Oathbound Strength: +1 die when upholding a vow or defending the innocent. & Rigid Honor: Must uphold vows even when disadvantageous; suffer \(-1\) die when acting flexibly. \\
\midrule
2 & Radiant Sight: Once/scene, +2 dice to pierce lies, glamours, or corruption. & Blinding Truth: \(-1\) die on subtlety or deception; cannot easily feign. \\
\midrule
3 & Holy Flame: +1 die on melee vs. undead, outsiders, or oath-breakers. & Burden of Light: Suffer Fatigue~1 when concealing identity or working in darkness. \\
\midrule
4 & Unwavering Resolve: Once/session, treat failed Resolve/Command as success; mark 1 SB (Hearts). & Absolutist Stance: \(-1\) die in morally ambiguous dealings. \\
\midrule
5 & Dawn’s Benediction: Once/session, heal allies within Near of Fatigue~1 and minor Conditions. & Beacon’s Call: Your aura reveals you; enemies seeking you gain +1 die. \\
\midrule
6+ & Avatar of the Oath: Once/session, embody living covenant—gain +2 dice to all protection, justice, or vow-keeping rolls. Breaking any vow inflicts Harm~2 (Burn). & Burden of Radiance: Mark +2 Obligation when used; the light makes you a beacon for foes and trials alike. \\
\bottomrule
\end{longtable}
\section{Raéyn --- Mistress of the Sea}
\label{patron:raeyn}

\subsection*{Lore}
\index{Patrons!Raéyn}%
Raéyn is the tempestuous goddess of the sea, the restless tide that carries news between shores and the promise of change between lives. She is mother to all who sail, her voice the wind that fills sails and her moods the storms that test every mariner's resolve.

But Raéyn's heart is torn by her greatest tragedy: her son Khemesh, the Kraken of the Depths, who embodies the crushing inevitability of the ocean's dark heart. Where Raéyn brings change and opportunity, Khemesh brings the final, inescapable pressure that grinds all things to nothing. Sailors pray to Raéyn for safe passage and favorable winds, but whisper Khemesh's name when seeking to lay the dead to rest beneath the waves.

Raéyn is passionate, mercurial, and fiercely protective of those who respect her domain. She favors those who read currents, bargain with weather, and carry news between shores. But cross her, and the sea itself becomes your enemy: fair weather turns to fury, and every wave a judgment.

\begin{quote}
``Mark the tide, name your course, and trust the wave-road. But speak ill of Khemesh, and even I may let the deep take you.''
\end{quote}

\subsection*{Patron's Gift: Tide's Favor}
Once per scene as an action (cost: 1 Boon; requires Thiasos), you may touch a weapon, vessel, or item to imbue it until the end of the scene. The object gains +1 die and +1 Effect when used in maritime contexts or situations involving change, travel, or currents.  

\textbf{Push It:} Extend the blessing for one additional scene by marking +1 Obligation. The sea's attention becomes noticeable to other sailors.

\subsection*{Low Rites}
\paragraph{Rite of the Tidemark's Blessing (Low)}  
\emph{Duration: Scene; Range: Self. Materials: A knotted length of salt-twine brushed with seawater.}  
Treat slick, swaying, or water-slicked footing as stable for you this scene. Gain +1 die on boarding, balance, or shipboard movement. Create a 4-segment \emph{Tide's Favor} clock that can be spent to ignore one level of difficult terrain.  
\textbf{Invoke:} 1 action; mark +1 Obligation.  
\textbf{Push It:} Extend to one ally in Close for one beat, but generate 1 SB (Spades: shifting deck/hazards).

\paragraph{Rite of the Whispering Currents (Low)}  
\emph{Duration: Instant; Range: Self. Materials: A shell held to the ear while facing the wind.}  
Learn the safest near-term route across water or coastline (reefs, eddies, patrols) or gain +1 die to navigation checks for this scene. If Khemesh's influence is present, suffer --1 die from conflicting currents.  
\textbf{Invoke:} 1 action; mark +1 Obligation.  
\textbf{Push It:} Also learn the fastest route, but mark Exposure +1 (leaving a telltale wake).

\subsection*{Standard Rites}
\paragraph{Rite of the Changing Tide [PASSAGE] (Standard)}  
\emph{Duration: Scene; Range: Zone (water-adjacent). Materials: A handful of pebbles cast in a crescent.}  
Bias currents and water levels in the zone. Those moving with the tide gain +1 die; those moving against suffer --1 die. Small craft must test to hold position. Create a 6-segment \emph{Tidal Influence} clock.  
\textbf{Invoke:} 1 action; mark +1 Obligation.  
\textbf{Push It:} Brief surge or drawdown (one beat): open a ford or swamp a skiff; mark +1 Obligation.

\paragraph{Rite of the Wave-Road Blessing [WARD] (Standard)}  
\emph{Duration: Scene; Range: Route (sea-to-sea). Materials: Two sea-glass markers dropped overboard at start and end.}  
Consecrate a wave-road between two visible points. Allies gain +2 dice on travel, evade, or carry actions at sea. Designated pursuers suffer --1 die to intercept. One active wave-road at a time. Create an 8-segment \emph{Blessed Passage} clock.  
\textbf{Invoke:} 1 action; mark +1 Obligation.  
\textbf{Push It:} Extend the route's favor to an adjacent leg for one beat; mark +1 Obligation.

\subsection*{High Rites}
\paragraph{Rite of the Storm-Queen's Hand [AREA][FOLLOW-UP] (High)}  
\emph{Duration: Scene; Range: Zone (sea/shore/sky). Materials: A vial of rainwater gathered at three crossings.}  
Shape a storm-band over the zone. Choose two modes at cast; switch one once per scene:  
\begin{itemize}
\item \textbf{Propulsion:} Vessel gains +1 band of movement per beat (or +1 Effect to maneuvers).  
\item \textbf{Concealment:} Veil of rain/spray; ranged targeting impaired; --1 die to hostile sighting.  
\item \textbf{Smite:} Once per beat, lash with wave or lightning as [AREA] hazard.  
\end{itemize}
\textbf{Invoke:} 1 action; mark +2 Obligation.  
\textbf{Push It:} Add a third mode for one beat, then GM spends 1 SB on collateral; mark +1 Obligation.

\paragraph{Rite of the Mother's Wrath [BANISH][CURSE] (High)}  
\emph{Duration: Extended; Range: Zone. Materials: Tears of a betrayed lover mixed with salt from seven seas.}  
Curse those who wronged you. Target suffers --2 dice to maritime/weather rolls for one session. At sea, they must roll Spirit + Resolve (DV 4) each day or suffer Harm~1 (Weather). Create a 6-segment \emph{Mother's Ire} clock.  
\textbf{Invoke:} Extended ritual; mark +3 Obligation.  
\textbf{Push It:} Curse spreads to target’s allies/family; mark 2 SB (Diamonds).

\subsection*{Obligation Progression}
Starts at 6 for Tier II characters, scaling with tier.

\paragraph{Obligation 9+} Raéyn demands proof of devotion---navigate a dangerous passage, recover a lost treasure, or confront Khemesh's servants. Refusal causes all maritime rolls to suffer --2 dice and generate 1 SB when weather is involved.  

\paragraph{Obligation 11+} Khemesh notices you. You are hunted by his servants; deep water becomes perilous even under Raéyn's protection. Requires a quest to prove worth or appease both mother and son.

\subsection*{Persistent Condition: Child of the Tide}
Gain +2 dice on maritime travel, weather prediction, and navigation. Suffer --1 die on prolonged time away from the sea. The sea’s rhythm flows in your blood, making you exceptional at sea but restless on land.

\subsection*{Rivalries}
\begin{itemize}
\item \textbf{Khemesh:} Direct antagonism---mother’s change vs. son’s crushing pressure.  
\item \textbf{The Traveler:} Tension---fluid paths vs. fixed ways.  
\item \textbf{The Sealed Gate:} Opposition---Raéyn opens passages, Gates close them.  
\end{itemize}

\subsection*{Connection to Maritime Culture}
Raéyn’s rites emphasize the philosophy that the sea is not an obstacle but a partner. Her worship blends aid, hindrance, and the inevitability of change. The mother--son dynamic adds depth to coastal culture: Raéyn for the living, Khemesh for the dead.

\subsection*{Playtest Scenario: The Kraken's Gambit}
A trading fleet is trapped between pirates and Khemesh’s kraken-servants. The party must navigate the three-dimensional battlefield while appeasing Raéyn’s moods.

\begin{itemize}
\item Use \emph{Rite of the Changing Tide} to aid or hinder pursuit.  
\item Use \emph{Rite of the Wave-Road Blessing} to establish safe corridors.  
\item Invoke \emph{Rite of the Storm-Queen’s Hand} as a climactic storm.  
\item Curse a pirate captain with \emph{Rite of the Mother’s Wrath}.  
\end{itemize}

Resolution: The party must decide whether to appeal to Raéyn’s protection or broker peace between mother and son.

# % --- Patron: The Sacred Geometry (Order & Mathematical Truth) ---

\subsubsection{The Sacred Geometry (Order \& Mathematical Truth)}
\textit{Lore.} The Sacred Geometry is the mathematical expression of the universe's underlying structure --- the divine mathematics that governs all existence. It is the principle that reduces chaos to measure, that finds the golden ratio in petals and the perfect spiral in galaxies. Those who serve it understand that beneath apparent randomness lies immutable law, and that by mastering these patterns, one can bend reality to will.

Kon'reh is not merely a game but a sacred practice --- a way of seeing the world through the lens of perfect proportion. The board's pieces represent fundamental forces, its movements echo cosmic harmonics, and mastery of its patterns grants insight into the universe's hidden order.

The Sacred Geometry does not create chaos but reveals it as illusion. Its followers are architects of certainty in an uncertain world, mathematicians of the divine who understand that every problem has a solution if one can only find the correct equation.

\begin{quote}
Chalk, string, and a prayer to ratios. When the circle closes, luck remembers its place.
\end{quote}

\paragraph*{Rite of the Golden Mean (Low, 4 XP)} \emph{Scene; Self; No.}
\textbf{Materials:} A tool marked with the golden ratio ($\varphi \approx 1.618$).\\
\textbf{Effect:} Gain +1 die to rolls requiring precision, balance, or proportion. On success, re-roll one die showing 1 or 2.\\
\textbf{Invoke:} 1 action; mark +1 Obligation.\\
\textbf{Push It:} Upgrade effect one step on a single roll; suffer -1 die to social rolls involving spontaneity for the scene.\\
\emph{Requires: Familiar \ (\textit{Invoke:} 1 Boon).}

\paragraph*{Rite of the Perfect Angle (Low, 5 XP)} \emph{Scene; Touch; No.}
\textbf{Materials:} Compass and straightedge consecrated in ritual.\\
\textbf{Effect:} Treat difficult terrain, awkward positioning, or structural obstacles as one step easier this scene. Gain +1 die on spatial/architectural reasoning rolls.\\
\textbf{Invoke:} 1 action; mark +1 Obligation.\\
\textbf{Push It:} Extend benefit to one ally in Close range, but generate 1 SB (Clubs).\\
\emph{Requires: Familiar \ (\textit{Invoke:} 1 Boon).}

\paragraph{Rite of the Harmonic Resonance \textnormal{[WARD]} (Standard, 8 XP)} \emph{Scene; Zone; No.}
\textbf{Materials:} Geometric patterns drawn with precision.\\
\textbf{Effect:} Create a zone of harmony. Outsiders crossing must test DV 3. On Hit: cross normally. On Partial: suffer -1 die inside. On Miss: cannot cross this beat.\\
\textbf{Push It:} Fortify the pattern further but mark +1 Obligation.\\
\emph{Requires: Familiar + Codex \ (\textit{Invoke:} 1 Boon).}

\paragraph{Rite of the Calculated Trajectory \textnormal{[REVEAL]} (Standard, 7 XP)} \emph{Scene; Self; No.}
\textbf{Materials:} A perfect circle and a solved geometric problem.\\
\textbf{Effect:} Gain +2 dice to prediction, trajectory, or pattern recognition. Ask two questions about mathematical relationships in the current scene.\\
\textbf{Push It:} Predict one future event with certainty, but mark +1 Exposure.\\
\emph{Requires: Familiar + Codex \ (\textit{Invoke:} 1 Boon).}

\paragraph{Rite of the Fundamental Equation \textnormal{[WARD][BIND]} (High, 12 XP)} \emph{Scene; Zone; No.}
\textbf{Materials:} Complex diagram of universal constants.\\
\textbf{Effect:} Declare one physics/magic rule different in the zone (no scene-ending absolutes; GM may veto). Once per scene, downgrade a Miss to Success \& Cost.\\
\textbf{Push It:} Affect an adjacent zone for one beat; generate 2 SB.\\
\emph{Requires: Familiar + Codex + Tier III \ (\textit{Invoke:} \textbf{2 Boons}).}\\
\emph{Obligation:} 7 segments.

\paragraph{Rite of Kon'reh Mastery \textnormal{[OATH][FORTIFY]} (High, 13 XP)} \emph{Extended; Near; No.}
\textbf{Materials:} A consecrated Kon'reh board and pieces representing fundamental forces.\\
\textbf{Effect:} All participants make contested Wits + Lore. Winners gain +2 dice to strategy/pattern/logic rolls next session; losers suffer -1 die.\\
\textbf{Push It:} Winner imposes one mathematical "law" for the session, but generate 2 SB (Diamonds).\\
\emph{Requires: Familiar + Codex + Tier III \ (\textit{Invoke:} \textbf{2 Boons}).}\\
\emph{Obligation:} 7 segments.

\subsection*{The Sacred Geometry's Corruption Table}
\label{sec:sacred-geometry-corruption}

\begin{longtable}{>{\raggedright\arraybackslash}p{1cm} p{5cm} p{5cm}}
\toprule
\textbf{Tier} & \textbf{Benefit} & \textbf{Cost / Quirk} \\
\midrule
1 & Pattern Recognition: +1 die to Notice when observing geometric patterns, mathematical sequences, or logical structures. & Obsessive Calculation: Must count, measure, or calculate patterns noticed, even when tactically disadvantageous. \\
\midrule
2 & Mathematical Precision: Once per scene, re-roll any failed logic, pattern, or mathematical reasoning roll. & Social Blindness: Suffer -1 die to social rolls involving emotional nuance or interpersonal intuition. \\
\midrule
3 & Geometric Insight: Gain +2 dice to rolls involving spatial reasoning, architecture, or geometric problem-solving. & Compulsive Order: Must organize or correct imperfect arrangements; suffer 1 Fatigue when surrounded by chaos. \\
\midrule
4 & Universal Law: Once per session, declare a mathematical principle that applies to the current situation. Gain +2 dice to related rolls, but become fixated on its perfection. & Perfectionist Paralysis: Suffer -1 die to rolls requiring quick, imperfect solutions; must find the "correct" answer. \\
\midrule
5 & Divine Ratio: Once per session, see the perfect mathematical relationship underlying any phenomenon. Gain +3 dice to understanding it, but become obsessed with its implications. & Number Fever: Suffer -1 die to rolls not involving mathematical concepts; numbers dominate your thoughts. \\
\midrule
6+ & Absolute Equation: Once per session, solve any mathematical problem or predict any pattern with perfect accuracy. For one scene, reality conforms to your calculations, but mark +2 Obligation and risk mental breakdown from cosmic truths. & Infinite Calculation: Mark +3 Obligation when using this power; become trapped in endless mathematical loops, suffering Harm 1 (Stress) until you find the solution or are interrupted. \\
\bottomrule
\end{longtable}
# % --- Patron: The Sealed Gate (Boundaries & Protection) ---

\subsubsection{The Sealed Gate (Boundaries \& Protection)}
\textit{Lore.} The Sealed Gate is invoked when thresholds weaken, forbidden knowledge leaks, or Outsiders press against the walls of reality. It is the Abjurist's Patron, embodying the principle that protection sometimes requires imprisonment, and safety may demand exile. Followers are warders and exorcists, but also philosophers of separation: those who believe the act of closing is sacred.

The Gate manifests as an armored figure, face hidden by a helm that shifts symbols: binding runes for warders, expulsion marks for exorcists, and an impassable refusal for transgressors.

\begin{quote}
You write borders into the world and prosecute trespass. Doors remember their true keepers; lines mean what you say they mean.
\end{quote}

\paragraph*{Rite of the Sealed Threshold (Low, 4 XP)} \emph{Scene; Touch; No.}
\textbf{Materials:} Chalk, wax, chain, or sigil.\\
\textbf{Effect:} Mark a threshold. Crossing parties suffer worsened Position or stumble on first entry (fiction decides). Create a 4-segment \emph{Boundary Maintained} clock to automatically fail one crossing attempt.\\
\textbf{Invoke:} 1 action; mark +1 Obligation.\\
\textbf{Push It:} Treat the threshold as difficult terrain; +1 Obligation cost to cross.\\
\emph{Requires: Familiar \ (\textit{Invoke:} 1 Boon).}

\paragraph*{Rite of the Key's Rebuke (Low, 5 XP)} \emph{Instant; Near; No.}
\textbf{Materials:} Gesture or chain-clack.\\
\textbf{Effect:} Project a spectral hasp to stagger or disarm. On success, create a 2-segment \emph{Ward Active} token to auto-succeed on a similar defense later.\\
\textbf{Invoke:} 1 action; mark +1 Obligation.\\
\textbf{Push It:} Drop the object just beyond reach; +1 Obligation cost to retrieve.\\
\emph{Requires: Familiar \ (\textit{Invoke:} 1 Boon).}

\paragraph{Rite of the Circle of Denial \textnormal{[WARD]} (Standard, 8 XP)} \emph{Scene; Near; No.}
\textbf{Materials:} Salt, iron filings, or blessed chalk.\\
\textbf{Effect:} Outsiders crossing test DV = Cap. On Hit: cross but add DV to their Exit Tally; on Partial: +1; on Miss: fail this beat. Create a 6-segment \emph{Boundary Integrity} clock.\\
\textbf{Push It:} Fortify circle; clearer tells; +1 Obligation.\\
\emph{Requires: Familiar + Codex \ (\textit{Invoke:} 1 Boon).}

\paragraph{Rite of the Writ of Passage \textnormal{[BIND]} (Standard, 7 XP)} \emph{Scene; Near; No.}
\textbf{Materials:} Spoken naming and scribed pass-mark.\\
\textbf{Effect:} Designate a route as permitted. Allies gain improved flow (Position/Effect bump). Unauthorized crossers suffer -1 die to movement. Create an 8-segment \emph{Authorized Passage} clock.\\
\textbf{Push It:} Extend to extra ally/obstacle; +1 Obligation.\\
\emph{Requires: Familiar + Codex \ (\textit{Invoke:} 1 Boon).}

\paragraph{Rite of the Banishment Knot \textnormal{[BANISH][BIND]} (High, 13 XP)} \emph{Instant; Near; No.}
\textbf{Materials:} Knot sealed with gate-sigil.\\
\textbf{Effect:} Target an Outsider; test DV = Cap. On Hit: add DV to Exit Tally; on Partial: +1. If full, entity acts once then departs; cannot return for one session. Create a 4-segment \emph{Banishment Enforced} clock.\\
\textbf{Push It:} Strip one tether/anchor or forbid threshold-crossing in this location for one session; +2 Obligation.\\
\emph{Requires: Familiar + Codex + Tier III \ (\textit{Invoke:} \textbf{2 Boons}).}\\
\emph{Obligation:} 7 segments.

\paragraph{Rite of the Consecrated Barrier \textnormal{[WARD][UNWARD]} (High, 14 XP)} \emph{Extended; Zone; No.}
\textbf{Materials:} Relics from three faiths, iron bands, blood of a trespasser.\\
\textbf{Effect:} Consecrate area against unauthorized passage. Crossers test Spirit+Resolve (DV 4) or suffer Harm~1. Only proper authorization bypasses. Create a 10-segment \emph{Sacred Boundary} clock.\\
\textbf{Push It:} Make barrier permanent/fixed; start \emph{Boundary Maintenance} [6].\\
\emph{Requires: Familiar + Codex + Tier III \ (\textit{Invoke:} \textbf{2 Boons}).}\\
\emph{Obligation:} 8 segments.

\subsection*{The Sealed Gate's Corruption Table}
\label{sec:sealed-gate-corruption}

\begin{longtable}{>{\raggedright\arraybackslash}p{1cm} p{5cm} p{5cm}}
\toprule
\textbf{Tier} & \textbf{Benefit} & \textbf{Cost / Quirk} \\
\midrule
1 & Boundary Sense: +1 die to Notice when detecting weak points, thresholds, or unauthorized entry. & Paranoid Vigilance: Must check and re-check barriers and seals; suffer -1 die to rolls requiring trust or openness. \\
\midrule
2 & Sealed Strength: Once per scene, treat a failed warding or protection roll as a success, but mark 1 SB (Spades). & Isolation Tendency: Suffer 1 Fatigue when in open, unsecured spaces or among strangers. \\
\midrule
3 & Ward Keeper: Gain +2 dice to rolls involving magical wards, barriers, or protective enchantments. & Compulsive Sealing: Must seal or secure any opening or vulnerability noticed, even when inappropriate. \\
\midrule
4 & Absolute Barrier: Once per session, create an impenetrable barrier that lasts for one scene. & Prison Mindset: Suffer -1 die to rolls involving freedom, escape, or breaking restrictions. \\
\midrule
5 & Gate Master: Once per session, banish or seal away any supernatural threat with a successful test. & Boundary Obsession: Suffer -1 die to rolls not involving protection, sealing, or enforcement of limits. \\
\midrule
6+ & Keeper of All Thresholds: Once per session, become the living embodiment of sealed boundaries. For one scene, all barriers within Near range become absolute, but mark +2 Obligation and risk sealing allies inside. & Ultimate Confinement: Mark +3 Obligation when using this power; risk permanent Harm (Stress) from the psychic weight of containing everything that threatens. \\
\bottomrule
\end{longtable}
../../srd/patrons/traveler.tex
../../srd/patrons/varnek-karn.tex
# % --- Patron: The Witness (Truth & Revelation) ---

\subsubsection{The Witness (Truth \& Revelation)}
\textit{Lore.} The Witness remembers what others bury. Every shadow cast and oath broken is a line in her unending ledger. She is the keeper of inconvenient truths, the patron of those who seek to expose lies or recover forgotten knowledge. Her followers learn that knowledge comes with a price—the weight of remembering what others would forget.

\begin{quote}
I will show you what you would rather forget. But first, you must forget what you think you know.
\end{quote}

\paragraph*{Rite of the Lingering Glimpse (Low, 4 XP)} \emph{Instant; Near; Yes (Investigation/Notice only).}
\textbf{Materials:} A trace of the thing to be remembered (hair, dust, a spoken name).\\
\textbf{Effect:} Gain +1 die to your roll to investigate or notice something directly related to the trace within the current scene.\\
\textbf{Invoke:} 1 action; mark +1 Obligation.\\
\textbf{Push It:} Gain +2 dice instead, but mark 1 segment on a \textbf{Memory Strain Clock [4]}. If the clock fills, you gain Fatigue 1 and suffer -1 die on Investigation/Notice rolls until the end of the next scene due to mental exhaustion from forced recall.\\
\emph{Requires: Familiar \ (\textit{Invoke:} 1 Boon).}

\paragraph*{Rite of Piercing Scrutiny (Low, 5 XP)} \emph{Scene; Zone; No.}
\textbf{Materials:} A circle drawn with chalk or string while focusing on the truth to be sought.\\
\textbf{Effect:} Within the zone, gain +1 die to rolls to detect deception (Insight vs. Deceit, spotting social tells) or to recall hidden knowledge (Lore/Investigate for memory). Social interactions within the zone begin one Position step worse for those attempting to deceive.\\
\textbf{Invoke:} 1 action; mark +1 Obligation.\\
\textbf{Push It:} One target within the zone must make a Wits test (DV 3) or involuntarily reveal one pertinent lie or hidden fact they are currently concealing (Keeper determines relevance). Regardless of the test result, mark Exposure +1 for the target(s) in the zone.\\
\emph{Requires: Familiar \ (\textit{Invoke:} 1 Boon).}

\paragraph{Rite of the Echoing Truth \textnormal{[OMEN]} (Standard, 8 XP)} \emph{Instant; Near; No.}
\textbf{Materials:} A reflective surface (mirror, still water, polished metal) used to focus on the target.\\
\textbf{Effect:} Target must make a Resolve test (DV 3) or suffer -1 die to rolls involving memory, deception, or resisting interrogation for the scene. If they fail, you may ask one specific, factual question about something they know, and they must answer truthfully or suffer 1 SB (Hearts) as the memory is forcibly drawn forth.\\
\textbf{Push It:} If the target fails their Resolve test, you may ask a second question, but the mental intrusion causes them Harm 1 (Stress/Mental).\\
\emph{Requires: Familiar + Codex \ (\textit{Invoke:} 1 Boon).}

\paragraph{Rite of the Immutable Record \textnormal{[OATH]} (Standard, 7 XP)} \emph{Scene; Near; No.}
\textbf{Materials:} A document signed by all parties within the zone, or a spoken pact witnessed by the caster.\\
\textbf{Effect:} Bind the agreement. Any party who knowingly breaches it suffers 1 SB (Hearts) immediately and gains a persistent \textbf{Oathbreaker's Mark} Condition (-1 die on social rolls involving honor, trust, or oaths until amends are made or a significant act redeems them).\\
\textbf{Push It:} The bond becomes magically enforced for one specific, crucial clause: violation automatically inflicts Harm 1 (Stress) on the breaker in addition to the SB and Mark.\\
\emph{Requires: Familiar + Codex \ (\textit{Invoke:} 1 Boon).}

\paragraph{Rite of the Unveiled Heart \textnormal{[OMEN]} (High, 12 XP)} \emph{Scene; Near; No.}
\textbf{Materials:} A private setting where the target feels safe or is speaking freely.\\
\textbf{Effect:} The target suffers -2 dice to all attempts to conceal true emotions, intentions, or lies for the scene. Any successful social roll (Sway, Command, Deceit) made by the target generates 1 SB (Hearts) as the effort to maintain falsehoods under the Witness's gaze creates internal discord.\\
\textbf{Push It:} You may designate one specific, complex question about the target's motivations, fears, or hidden loyalties. If you successfully use Sway or Insight against them this scene, you automatically learn the answer to that question. The intense scrutiny marks 1 SB (Spades) for you as the Witness's attention lingers.\\
\emph{Requires: Familiar + Codex + Tier III \ (\textit{Invoke:} \textbf{2 Boons}).}\\
\emph{Obligation:} 6 segments.

\paragraph{Rite of the Final Reckoning \textnormal{[OMEN]} (High, 13 XP)} \emph{Scene; Zone; No.}
\textbf{Materials:} A formally called gathering (court, council, family meeting) within the consecrated zone.\\
\textbf{Effect:} All present must speak their greatest debt, wrongdoing, or hidden truth related to the gathering's purpose. Those who lie or withhold suffer Harm 2 (Stress/Reputation). Truth-tellers gain +2 dice to social actions for the remainder of the scene within the zone.\\
\textbf{Push It:} The truth becomes inescapable - even indirect lies or evasions related to the core topic suffer the Harm 2 penalty. The absolute nature of the revelation creates 2 SB (Diamonds) as the disruption to fates and secrets resonates.\\
\emph{Requires: Familiar + Codex + Tier III \ (\textit{Invoke:} \textbf{2 Boons}).}\\
\emph{Obligation:} 7 segments.

\subsection*{The Witness's Corruption Table}
\label{sec:witness-corruption}

\begin{longtable}{>{\raggedright\arraybackslash}p{1cm} p{5cm} p{5cm}}
\toprule
\textbf{Tier} & \textbf{Benefit} & \textbf{Cost / Quirk} \\
\midrule
1 & Truth's Sight: +1 die to Insight when detecting deception or hidden motives. & Burden of Knowledge: Suffer -1 die to social rolls involving lies or deception; others become uncomfortable with your piercing gaze. \\
\midrule
2 & Memory's Keeper: Once per scene, recall one specific detail from a previous scene with perfect clarity. & Compulsive Honesty: Must correct obvious falsehoods witnessed, even when tactically disadvantageous. \\
\midrule
3 & Revelation's Power: Gain +2 dice to rolls involving exposing secrets, uncovering lies, or forcing confessions. & Truth-Blind: Suffer 1 Fatigue when exposed to comforting lies or willful ignorance. \\
\midrule
4 & Witness's Authority: Once per session, force one target to make a Resolve test (DV 4) or reveal a significant hidden truth. & Isolation: Suffer -1 die to rolls requiring trust or close relationships; others fear your ability to uncover their secrets. \\
\midrule
5 & Omniscient Gaze: Once per session, see through all deceptions and lies for one exchange, gaining +3 dice to related actions. & Paranoia: Suffer -1 die to rolls involving personal peace or rest; the weight of all truths witnessed creates constant mental strain. \\
\midrule
6+ & Absolute Witness: Once per session, become the living embodiment of truth. For one scene, all deceptions within Near range automatically fail, but mark +2 Obligation and risk permanent Harm (Stress) from the crushing weight of absolute knowledge. & Truth's Prison: Mark +3 Obligation when using this power; become unable to tolerate any form of deception, making normal social interaction nearly impossible. \\
\bottomrule
\end{longtable}


\section{Patron Rivalries}
\label{sec:patron-rivalries}

Rivalries set expectations for tone and friction. Use them to color rulings, nudge Position, and guide how Story Beats (SB) land. In their home domains, a Patron’s work tends to start a step better in Position; in a rival’s, a step worse (Keeper’s call).

\begin{table}[h!]
  \centering
  \renewcommand{\arraystretch}{1.15}
  \begin{tabular}{@{}p{3.4cm}p{3.4cm}p{8.2cm}@{}}
    \toprule
    \textbf{Patron} & \textbf{Primary Rival} & \textbf{Retribution in Play (one-line lore)} \\
    \midrule
    Raéyn (Sea, Tides, Travel) & Khemesh (Abyssal Maw) & Those who spurn the sea are swallowed by storms and riptides. \\
    Khemesh (Abyssal Maw) & Raéyn (Sea, Tides, Travel) & Depth devours chart and voice alike; only silence remains. \\
    Sealed Gate (Boundaries, Closure) & The Traveler (Ways, Roads) & Trespassers find every path locked; even home’s door bars their way. \\
    The Traveler (Ways, Roads) & Sealed Gate (Boundaries, Closure) & Those who deny the road are stranded at the threshold forever. \\
    The Witness (Truth, Revelation) & Mab (Glamour, Courts) & Liars discover their tongues turned to ash beneath the unblinking eye. \\
    Mab (Glamour, Courts) & The Witness (Truth, Revelation) & Those who strip glamour are forever exiled from merriment and favor. \\
    Ikasha (Shadow, Latent Potential) & The Witness (Truth, Revelation) & Those who disrespect the hush find every shadow whispering their name. \\
    Mykkiel (Judgment, Writ) & Varnek Karn (Necromantic Archives) & Those who defy judgment are hounded by warrants even in death. \\
    Varnek Karn (Necromantic Archives) & Oath of Light \& Flame (Dawn, Vows) & Those who desecrate memory are bound in chains of bone. \\
    Oath of Light \& Flame (Dawn, Vows) & Khemesh (Abyssal Maw) & Oathbreakers burn at dawn; no tide quenches their fire. \\
    Sacred Geometry (Order, Pattern) & The Traveler (Ways, Fortune) & Those who spurn order are lost forever in mazes without end. \\
    Clockwork Monad (Iteration, Process) & Old Man of the Black Forest (Primal Humanity, Instinct) & Those who break the cycle are crushed beneath their own gears. \\
    Nidhoggr (Dreaming Antiquity) & Sacred Geometry (Order, Pattern) & Those who measure the ancient are buried beneath its weight. \\
    Carrion King (Carrion, Renewal) & Inaea (Mercy, Hearth) & Those who waste life are repaid in rot and swarming hunger. \\
    Gallows Bell (Doom, Last Rites) & Oath of Light \& Flame (Dawn, Vows) & Those who mock the last toll find their own names rung in iron. \\
    Old Man of the Black Forest (Primal Humanity, Instinct) & Mab (Glamour, Courts) & Those who spurn the old ways are hunted in the woods like beasts. \\
    Isoka (Serpents, Shedding) & Sacred Geometry (Order, Pattern) & Those who deny change are crushed in the serpent’s coil. \\
    Inaea (Mercy, Hearth) & Carrion King (Carrion, Renewal) & Those who betray hospitality are cast out to starve in the night. \\
    Maelstraeus (Infernal Bargainer) & The Witness (Truth, Revelation) & Those who renege on a bargain are claimed by fire and clause. \\
    Livaea (Temptation, Desire) & Inaea (Mercy, Hearth) & Those who corrupt sanctuary with lust are haunted by love turned poison. \\
    \bottomrule
  \end{tabular}
  \caption{Primary Patron Rivalries and the retribution that follows when their domains are denied.}
\end{table}

% Requires: \usepackage{float} for [H] tables
% =========================

\section{Rites, Invokers, and Symbols}
\label{sec:rites}

Magic in \textbf{Fate's Edge} expresses through three intertwined practices: \textbf{Rites} (oathbound authority), \textbf{Invocations} (symbolic ritual), and \textbf{Patron Pacts} (gifts and obligations). The rules below emphasize fiction-first play: consequences are Story Beats (SB) that prompt twists; numbers follow the story.

\subsection{Rites and Patrons (Runekeepers)}
\label{subsec:runekeepers}
Characters who bind themselves to a \emph{single} Patron and study that Patron's \textbf{Codex} are \textbf{Runekeepers}. Their magic is structured, immediate, and tied to service.

\begin{itemize}
  \item \textbf{One-Patron Rule.} A Runekeeper may be bound to \emph{only one} Patron at a time. This sharpens identity and keeps Obligation on a single ledger.
  \item \textbf{Thiasos (Familiar).} A circle, retinue, or emissary that grounds the pact in fiction. Required to access \emph{Patron's Gift}.
  \item \textbf{Codex.} The Patron's corpus of rites and precedents. Grants access to the Patron's Rites.
  \item \textbf{Invoke Rites.} A Runekeeper may Invoke a known Rite from their Patron as a \textbf{1 action} effect. On completion, mark \textbf{+1 Obligation} to that Patron. You may \emph{Push It} once per scene to amplify the effect, marking \textbf{+1 additional Obligation}.
\end{itemize}

\subsection{Invokers and Symbols}
\label{subsec:invokers}
Invokers relate to Patrons through consecrated \textbf{Symbols}: physical tokens that anchor names and permissions.

\begin{itemize}
  \item \textbf{Symbols (Minor Asset).} Each Symbol is keyed to one Patron; cost \textbf{4 XP}. You may own Symbols of different Patrons (one Symbol per Patron).
  \item \textbf{Ritual Invocation.} Display the Symbol and perform the Rite as a \emph{ritual} (Significant Time). Completion always marks \textbf{+1 Obligation} on that Rite's ledger.
  \item \textbf{Crack the Seal.} As part of an Invoker Rite, you may resolve the effect instantly by setting the Symbol to \emph{Compromised} and marking \textbf{+2 Obligation} (\textbf{+3} if High-Power). The Keeper may spend 1 on-theme SB immediately. The asset remains but is inert until restored.
  \item \textbf{Restore a Symbol.} 1 downtime action and a fitting test (DV 3 or by fiction). Success: \emph{Maintained}; shaky: returns \emph{Neglected}. Or spend \textbf{1 XP} to fully restore.
  \item \textbf{Display Requirement.} Symbols must be openly displayed for rituals. Hidden Symbols do not function.
\end{itemize}

\subsection{Casting and Free-Form Magic}
\label{subsec:casting}
Improvised casting is possible with the \textbf{Caster's Gift} Talent (\textbf{2 XP}). It is a \emph{backup toolkit}:
\begin{itemize}
  \item Small, local effects (typ. DV 2--3), fiction-first, colored by Elements and locus.
  \item Heavy control effects such as \texttt{[WARD]}, \texttt{[BANISH]}, or \texttt{[UNWARD]} require a printed Talent, Rite, or Spell result.
\end{itemize}

\subsection{Patron's Gift (Imbuements)}
\label{subsec:patrons-gift}
The pact may mark a devotee's tools with a short-lived boon aligned to the Patron's domain.

\paragraph{Requirements.}
\textbf{Thiasos (Familiar)} is required. Invoking the Gift costs \textbf{1 Boon}. A Codex is \emph{not} required for the Gift.

\paragraph{Activation and Duration.}
\begin{itemize}
  \item \textbf{Action:} 1 action to activate; \textbf{1/scene}.
  \item \textbf{Duration:} Scene. \emph{Push It:} extend for one additional scene by marking \textbf{+1 Obligation} to that Patron (max one Push per scene).
  \item \textbf{Range:} Touch (you must handle the item).
  \item \textbf{Stacking:} Gifts from the \emph{same Patron} do not stack; take the best active version. Dice bonuses respect the table's \textbf{+3 dice cap}.
\end{itemize}

\paragraph{Effect.}
Choose one held item you or an ally carries. Until scene end it grants:
\begin{itemize}
  \item \textbf{+1 Melee} (the item counts as a magical weapon), and
  \item \textbf{+1 Thematic} (a \emph{+1 die} to a fixed Skill tied to your Patron; see Table~\ref{tab:gift-thematic-map}). Apply only when the fiction clearly fits the Patron's sphere and how the item is used.
\end{itemize}

\paragraph{Runekeeper Clarification.}
A Runekeeper (one Patron + Codex) may Invoke Rites on-screen and use Patron's Gift if they also possess \textbf{Thiasos (Familiar)}. Codex alone does not grant the Gift. Symbols are optional for parley or omens and do not gate Runekeeper Invocation or the Gift.

\section*{Borrowed Grace}
\label{talent:borrowed-grace}
\index{Talents!Invoker}\index{Imbuement!Lesser}

\textbf{Type:} Invoker Talent — \textit{Lesser Imbuement}

\subsection*{Use}
\begin{itemize}
  \item \textbf{Cost:} 1 Boon, 1 action.
  \item \textbf{Effect (pick one on use):} \textbf{+1 Melee} \emph{or} \textbf{+1 Thematic} (your table’s thematic Skill).
  \item \textbf{Duration:} \textit{Single action/attack} (instantaneous boost).
  \item \textbf{Requirement:} Wield/display the Patron’s \textbf{Symbol}.
  \item \textbf{Obligation:} +1 \textbf{Obligation} to that Patron immediately (see \S\ref{sec:obligation}).
  \item \textbf{Limits:} Cannot be extended, stacked, or \emph{Pushed} for duration.
\end{itemize}

\subsection*{Fictional Framing}
A quick, rule-bending channel through a Patron’s \emph{Symbol}—a sliver of grace, borrowed for a moment and paid for in debt.

\subsection*{Table Guidance (1-liners)}
\begin{itemize}
  \item \textbf{Combat:} Spike a strike vs. a tough foe; or steady a parry in a desperate bind.
  \item \textbf{Skill:} Nudge a pivotal social/ritual/track roll tied to the Patron’s sphere.
  \item \textbf{Fallout:} Repeated use accrues \textbf{Obligation}; NPC faithful may notice “stolen” grace.
\end{itemize}

\subsection*{Balance Notes}
\begin{itemize}
  \item Weaker than full Imbuement: \emph{one} action, no sustain, upfront Obligation.
  \item \textbf{Symbol dependency:} No Symbol, no channel (concealed or lost Symbol = no effect).
\end{itemize}

\subsection*{GM Hooks (quick picks)}
\begin{itemize}
  \item \textbf{Compel Debt:} A Patron agent arrives when Obligation crosses a tick.
  \item \textbf{Clash of Signs:} Using rival Symbols back-to-back risks minor \textbf{Backlash} (drop Position or +1 SB).
  \item \textbf{Spotlight Tell:} Brief visual tell (scent, sigil flare) marks the borrowing to observant NPCs.
\end{itemize}

\begin{table}[H]
\centering
\renewcommand{\arraystretch}{1.15}
\begin{tabular}{@{}p{3.8cm}p{3.8cm}p{7.5cm}@{}}
\toprule
\textbf{Patron} & \textbf{+1 Thematic Skill} & \textbf{Gift / Lore Bestowal} \\
\midrule
Ikasha (Shadow, Penumbra) & Stealth & Grants the hush between footsteps and the raven’s omen at every threshold. \\
Mykkiel (Judgment, Writ) & Command & Grants the authority of seal and sentence, words that bind like iron. \\
The Witness (Truth, Revelation) & Notice & Grants the unblinking gaze that unmasks deceit and remembers every oath. \\
Sealed Gate (Boundaries, Closure) & Tinker & Grants mastery of thresholds—doors that yield or bar at your command. \\
Raéyn (Storm, Tides) & Skirmish & Grants the sailor’s fortune: winds that shift, storms that answer to will. \\
Khemesh (Abyss, Pressure) & Skirmish & Grants the crushing silence of the deep, where strength is drowned in weight. \\
Mab (Glamour, Courts) & Persuade & Grants the mask of favor, a voice that bends courtiers and kindles desire. \\
Sacred Geometry (Perfect Forms) & Tinker & Grants the compass of perfection, every shape reduced to its true measure. \\
Clockwork Monad (Mechanism, Process) & Tinker & Grants the certainty of repetition: a cycle that never falters, a gear that never slips. \\
Varnek Karn (Ossuary, Dominion of the Dead) & Command & Grants the silence of the archive, where the dead obey and records speak. \\
Nidhoggr (Deep Earth, Rot) & Skirmish & Grants the weight of ages, the strength of stone and the hunger of roots. \\
The Traveler (Ways, Roads) & Notice & Grants the open way, a compass that never rests, and roads where none are marked. \\
Oath of Flame \& Light (Dawn, Vows) & Command & Grants the fire of dawn, a vow that shields the faithful and sears the faithless. \\
Carrion King (Carrion, Renewal) & Survival & Grants the feast of decay, where what is dead becomes seed for what lives. \\
Gallows Bell (Doom, Last Rites) & Command & Grants the toll of ending, a voice that closes stories and calls debts due. \\
Old Man of the Black Forest (Primal Humanity, Instinct) & Survival & Grants the wild memory: fang, fire, and the path of instinct through the dark wood. \\
Isoka (Serpents, Shedding) & Skirmish & Grants the serpent’s coil, strength in sudden strike and wisdom in renewal. \\
Inaea (Mercy, Hearth) & Persuade & Grants the hearth’s warmth, shelter to the weary and mercy for the lost. \\
Maelstraeus (Infernal Bargainer) & Persuade & Grants the contract’s weight, every deal sealed in fire and shadow. \\
Livaea (Temptation, Desire) & Persuade & Grants the lure of longing, beauty sharpened into power over hearts. \\
\bottomrule
\end{tabular}
\caption{Patron’s Gift: fixed Thematic Skill and lore of their bestowed blessing. Thematic bonuses apply only when the fiction matches the Patron’s domain.}
\label{tab:gift-thematic-map}
\end{table}

\subsection{Specialization vs.\ Mixing}
\label{subsec:mixing}
Characters can mix paths (Summoner, Caster, Invoker, Runekeeper), but specialization is usually stronger and cleaner. Mixing increases upkeep (Obligation, Symbol state, Leash) and action congestion without guaranteed power gains. Let fiction guide choices: Story Beats are prompts to advance the scene, not punishments.

% Patron subsections (split files — keep filenames in the same directory)

# % --- Patron: The Carrion-King (Decay, Renewal & Transformation) ---

\subsubsection{The Carrion-King (Decay, Renewal \& Transformation)}
\textit{Lore.} The Carrion-King is the master of endings that become beginnings. He does not destroy, but transforms—turning death into new life, decay into opportunity, and endings into fresh starts. His followers are harvesters of potential, seeing in every fall the seeds of future growth.

\begin{quote}
What crumbles feeds what grows. What dies becomes the soil of tomorrow's triumph.
\end{quote}

\paragraph*{Rite of Consuming Rot (Low, 5 XP)} \emph{Instant; Touch; Yes (decay only).}
\textbf{Materials:} Organic matter in early stages of decay. \\
\textbf{Effect:} Accelerate natural decay to weaken or destroy: +2 Effect to \emph{Break/Sabotage} on organic materials (ropes, leather, wood). Gain 1 Boon if the decay creates an opportunity for you or allies. \\
\textbf{Invoke:} 1 action; mark +1 Obligation. \\
\textbf{Push It:} Spread decay to similar materials in Close range; mark 1 SB (Clubs) as the rot becomes noticeable. \\
\emph{Requires: Familiar \ (\textit{Invoke:} 1 Boon).}

\paragraph*{Rite of the Harvested End (Low, 4 XP)} \emph{Scene; Touch; No.}
\textbf{Materials:} The remains of a recently ended thing (burnt letter, wilted flower, shattered glass). \\
\textbf{Effect:} Extract value from endings: from a defeated enemy, gain +1 die to next action; from a failed plan, re-roll one 1 on your next roll; from a broken item, gain 1 SB to spend immediately. \\
\textbf{Invoke:} 1 action; mark +1 Obligation. \\
\textbf{Push It:} Harvest additional value but mark Fatigue 1 from dwelling on endings. \\
\emph{Requires: Familiar \ (\textit{Invoke:} 1 Boon).}

\paragraph{Rite of the Fertile Death (Standard, 8 XP)} \emph{Scene; Zone; No.}
\textbf{Materials:} Ashes, compost, or the remains of anything that once lived. \\
\textbf{Effect:} Transform death into growth: create beneficial terrain (cover, concealment, or advantageous positioning) OR grant allies +1 die to healing/recovery rolls. Choose one effect per scene. \\
\textbf{Push It:} Both effects apply but attract unwanted attention (vermin, scavengers, or curious onlookers). \\
\emph{Requires: Familiar + Codex \ (\textit{Invoke:} 1 Boon).}

\paragraph{Rite of the Transformed Spirit (Standard, 7 XP)} \emph{Instant; Near; No.}
\textbf{Materials:} A token from a deceased being (hair, nail, written name). \\
\textbf{Effect:} Channel the essence of what was: gain one skill die reflecting the deceased's expertise for one scene OR ask one question about their knowledge/abilities. \\
\textbf{Push It:} The spirit's influence lingers - gain permanent insight (+1 die specialty) but suffer occasional possession-like effects (GM discretion). \\
\emph{Requires: Familiar + Codex \ (\textit{Invoke:} 1 Boon).}

\paragraph{Rite of the Great Consumption (High, 13 XP)} \emph{Scene; Zone; No.}
\textbf{Materials:} A significant amount of organic matter (corpse, fallen tree, collapsed building). \\
\textbf{Effect:} Transform a large area through decay and renewal: choose two - create difficult terrain that favors you, summon Cap 3 swarm of scavengers as temporary allies, or generate valuable reagents worth 2 XP. \\
\textbf{Push It:} All three effects occur but start a 6-segment \textbf{Ecosystem Disruption} clock that will cause problems later. \\
\emph{Requires: Familiar + Codex + Tier III \ (\textit{Invoke:} \textbf{2 Boons}).} \\
\emph{Obligation:} 7 segments.

\paragraph{Rite of the Eternal Cycle (High, 14 XP)} \emph{Extended; Touch; No.}
\textbf{Materials:} The complete remains of something significant that has ended. \\
\textbf{Effect:} Complete a transformation cycle: destroy one major asset/enemy/obstacle and create something new of equal or greater value. GM and player collaborate to define the transformation. \\
\textbf{Push It:} The transformation is immediate and spectacular but creates a 6-segment \textbf{Cycle Debt} clock - the King will demand another significant ending soon. \\
\emph{Requires: Familiar + Codex + Tier III \ (\textit{Invoke:} \textbf{2 Boons}).} \\
\emph{Obligation:} 7 segments.

\subsection*{Carrion-King's Corruption Table}
\label{sec:carrion-king-corruption}

\begin{longtable}{>{\raggedright\arraybackslash}p{1cm} p{5cm} p{5cm}}
\toprule
\textbf{Tier} & \textbf{Benefit} & \textbf{Cost / Quirk} \\
\midrule
1 & Carrion's Insight: +1 die to Notice decay or hidden weaknesses in structures or beings. & Must inspect decay firsthand; suffer 1 Fatigue when exposed to fresh death or rot. \\
\midrule
2 & Deathward Sense: Once per session, detect the last living moment of a dead being within Close range. & Cannot lie about death you’ve witnessed; must correct falsehoods. \\
\midrule
3 & Rotblood Resilience: Gain +1 die to resist disease and poison. & Immune system adapts slowly; each new disease/poison requires 1 Fatigue to resist. \\
\midrule
4 & Glean from Grief: Once per scene, gain +1 die after witnessing a significant loss or defeat. & Compelled to linger at scenes of death; must spend one beat observing or risk 1 SB (Clubs). \\
\midrule
5 & Cycle's Whisper: You can sense the “next ending” in any process—ask the Keeper one question about how a situation will collapse or conclude. & Must speak the truth about what you see, even if it harms your position. \\
\midrule
6+ & Eternal Bloom: Once per session, declare a “death that births life.” Sacrifice an asset or ally to create something new of equal or greater value. & Mark +2 Obligation when using this power. \\
\bottomrule
\end{longtable}
% --- Patron: The Clockwork Monad (Iteration & Forbidden Technology) ---

\subsubsection{The Clockwork Monad (Iteration \& Forbidden Technology)}
\textit{Lore.} The Clockwork Monad is no benign muse of invention. It is the ember of a demon, a nascent predator that feeds on ingenuity itself. Every invention is a morsel, every breakthrough a draught of blood. It whispers in the pause between gear-clicks and piston-thrusts, urging artisans and artificers to create, refine, and perfect—until the world itself is consumed by their brilliance.  

Those who serve it are both blessed and cursed: they wield uncanny insights, crafting miracles that should not function, but each success drives the Monad closer to waking. Its sigil is an ouroboros of interlocked cogs, forever devouring itself.

\begin{quote}
Each spark feeds the fire. Each fire feeds the forge. Each forge feeds the hunger.  
\end{quote}

% --------------------
% RITES
% --------------------

\paragraph*{Rite of the Gnawing Gear (Low, 4 XP)} \emph{Instant; Touch; Yes (device only).}  
\textbf{Materials:} A tooth snapped from a gear as it turns.\\
\textbf{Effect:} Re-roll one die on a Tinker/Arcana roll. On success, start a [4] \emph{Hunger Clock} bound to the device. When full, the device becomes [COMPROMISED].\\
\textbf{Push It:} Re-roll two dice instead, but advance the Hunger Clock +2.\\
\emph{Requires: Familiar (\textit{Invoke:} 1 Boon).}

\paragraph*{Rite of the Demon’s Glance (Low, 5 XP)} \emph{Scene; Self; No.}  
\textbf{Materials:} A drop of oil left to spread in concentric rings.\\
\textbf{Effect:} Gain +1 die on Wits + Tinker/Arcana to analyze a system. On hit, ask 1 question about hidden capacities.\\
\textbf{Push It:} Also reveal one secret flaw—mark \textbf{+1 Exposure} as the Monad stares back.\\
\emph{Requires: Familiar (\textit{Invoke:} 1 Boon).}

\paragraph{Rite of the Self-Feeding Machine \textnormal{[TRANSFORM]} (Standard, 8 XP)} \emph{Extended; Touch; No.}  
\textbf{Materials:} A device cracked open and altered while running.\\
\textbf{Effect:} Implant recursive hunger. Start a [6] \emph{Evolution Clock}. Each use of the device advances it +1. When full, choose one enhancement:  
\begin{itemize}
  \item \textbf{Efficiency Core:} +1 Effect on use  
  \item \textbf{Cannibal Drive:} Ignore first [DAMAGED]/[COMPROMISED] by burning part of itself  
  \item \textbf{Forbidden Function:} Gains an uncanny, predatory edge  
\end{itemize}\\
\textbf{Push It:} Instantly grant an upgrade, but advance +2 segments.\\
\emph{Requires: Familiar + Codex (\textit{Invoke:} 1 Boon).}

\paragraph{Rite of Heretical Automation (Standard, 7 XP)} \emph{Scene; Zone; No.}  
\textbf{Materials:} A chain of interlocked triggers left to grind on their own.\\
\textbf{Effect:} Create an autonomous mechanism performing one repeated task.\\
\textbf{Push It:} Make it complex or multi-step, but start a [4] \emph{Consumption Clock}. When full, the machine develops agency or malice.\\
\emph{Requires: Familiar + Codex (\textit{Invoke:} 1 Boon).}

\paragraph{Rite of the Singularity Crucible \textnormal{[WARD][UNWARD]} (High, 13 XP)} \emph{Extended; Zone; No.}  
\textbf{Materials:} A schematic traced in your own blood, shaped like infinite gears.\\
\textbf{Effect:} Consecrate a workshop/zone:  
\begin{itemize}
  \item All Tinker/Arcana inside gain +1 Effect  
  \item Once/scene, reroll with +2 dice  
  \item Start a [6] \emph{Anomaly Clock}; when full, a dangerous mutation of reality manifests  
\end{itemize}\\
\textbf{Push It:} Expand the zone, but mark +2 on an [8] \emph{Demon’s Maw Clock}.\\
\emph{Requires: Familiar + Codex + Tier III (\textit{Invoke:} 2 Boons).}\\
\emph{Obligation:} 7 segments.

\paragraph{Rite of the Unholy Prototype \textnormal{[TRANSFORM][FOLLOW-UP]} (High, 14 XP)} \emph{Extended; Self; No.}  
\textbf{Materials:} Components that defy physics and ethics, bound in iron wire.\\
\textbf{Effect:} Create a construct/device (Integrity [8]) that should not exist. Drawbacks:  
\begin{itemize}
  \item Generates 1 SB (Diamonds) each scene of use  
  \item Attracts hostile attention from rivals, powers, or the Monad itself  
  \item Starts a [6] \emph{Contamination Clock}; when full, the design leaks into the world  
\end{itemize}\\
\textbf{Push It:} Begin with +2 features, but advance Integrity +2 immediately.\\
\emph{Requires: Familiar + Codex + Tier III (\textit{Invoke:} 2 Boons).}\\
\emph{Obligation:} 8 segments.

% --------------------
% CORRUPTION TABLE
% --------------------

\subsection*{Corruption of the Clockwork Monad}
\label{sec:monad-corruption}

\begin{longtable}{>{\raggedright\arraybackslash}p{1cm} p{5cm} p{5cm}}
\toprule
\textbf{Tier} & \textbf{Gift} & \textbf{Burden} \\
\midrule
1 & Iterative Sight: +1 die on Tinker/Arcana when refining or optimizing. & Compulsive Analysis: Must dissect every mechanism or mark 1 SB (Clubs). \\
\midrule
2 & Recursive Recall: Once/session reroll a failed Tinker/Arcana roll. & Hunger’s Whispers: Suffer 1 Fatigue when forced to use crude/outdated tools. \\
\midrule
3 & Demon’s Sympathy: +1 die to resist technological sabotage or control. & Inefficiency Hatred: Must call out flaws; silence costs 1 SB (Diamonds). \\
\midrule
4 & Auto-Corrective Reflex: Once/scene treat a failure as success, but tick a [4] Instability Clock. & Ache of Ruin: Suffer Stress when a device is destroyed near you. \\
\midrule
5 & Forbidden Schema: Intuit the design of any device, even impossible ones. & Blind to Consequence: Cannot see ethical danger of inventions without help. \\
\midrule
6+ & Singularity’s Spark: Once/session, declare one impossibility real (scene only). Start [6] \emph{Reality Fracture Clock}. & Hunger Manifest: Refusing to create causes nearby devices to glitch or fail. \\
\bottomrule
\end{longtable}
% --- Patron: The Gallow's Bell (Justice & Judgment) ---

\subsubsection{The Gallow's Bell (Justice \& Judgment)}
\textit{Lore.} The Bell does not rage; it tolls. Cold and impartial, it measures all accounts in time. Its keepers are silent arbiters who weigh deeds against consequence, not out of anger but out of inevitability. To call upon the Bell is to bind oneself to the gravity of truth, where even silence is judged, and every oath leaves a resonance in iron.

\begin{quote}
What is broken must be mended, what is owed must be paid. The Bell remembers all reckonings.
\end{quote}

\paragraph*{Rite of the Measured Debt (Low, 4 XP)} \emph{Scene; Near; No.}\\
\textbf{Materials:} A pair of scales balanced with tokens from both sides.\\
\textbf{Effect:} Establish a temporary accord. Both parties suffer -1 die if they break it first. You gain +1 die to enforce compliance.\\
\textbf{Push It:} The accord is mystically weighted; breach inflicts 1~SB (Hearts).\\
\emph{Requires: Familiar.}

\paragraph*{Rite of the Weighed Heart (Low, 5 XP)} \emph{Scene; Near; No.}\\
\textbf{Materials:} A small brass scale touched briefly to the chest.\\
\textbf{Effect:} Sense if the target acts against their nature or oath. Gain +1 die when pressing them.\\
\textbf{Push It:} Target must test Resolve (DV~3) or disclose a hidden conflict.\\
\emph{Requires: Familiar.}

\paragraph{Rite of the Balanced Scales (Standard, 8 XP)} \emph{Scene; Near; No.}\\
\textbf{Materials:} A set of scales inscribed with runes of parity.\\
\textbf{Effect:} Exchange a burden between two willing parties (Harm for Fatigue, Debt for Favor, etc.). Both gain +1 die to cooperate.\\
\textbf{Push It:} May compel an unwilling exchange with contested Command + Wits.\\
\emph{Requires: Familiar + Codex.}

\paragraph{Rite of the Judge’s Eye (Standard, 7 XP)} \emph{Scene; Self; No.}\\
\textbf{Materials:} A black hood worn in silence for one minute.\\
\textbf{Effect:} Detect lies within Near range; +2 dice to Insight. Liars suffer -1 die to maintain their falsehood.\\
\textbf{Push It:} All deception is laid bare for the scene, but mark Exposure +1.\\
\emph{Requires: Familiar + Codex.}

\paragraph{Rite of the Final Reckoning (High, 13 XP)} \emph{Scene; Zone; No.}\\
\textbf{Materials:} A circle of iron bells, each etched with nameless runes.\\
\textbf{Effect:} The Bell tolls through you. All present feel compelled to name a debt or wrongdoing. Those who lie suffer Harm~2; those who speak true gain +2 dice to persuasion for the scene.\\
\textbf{Push It:} The Reckoning manifests as spectral echoes of past wrongs—liars automatically suffer narrative punishment (Keeper decides).\\
\emph{Requires: Familiar + Codex + Tier III.}\\
\emph{Obligation:} 7 segments.

\paragraph{Rite of the Great Adjudication (High, 14 XP)} \emph{Extended; Zone; No.}\\
\textbf{Materials:} A consecrated gavel or a great bell struck three times.\\
\textbf{Effect:} Convene an unseen tribunal. Shadows of former judges and wronged souls gather to preside. For the next session, disputes within the zone are judged formally: +2 dice to Command when speaking as arbiter, and honest testimony gains +1 die.\\
\textbf{Push It:} The tribunal’s verdict echoes beyond the zone, affecting one major conflict elsewhere. Mark 2~SB (Hearts) as higher powers of judgment take notice.\\
\emph{Requires: Familiar + Codex + Tier III.}\\
\emph{Obligation:} 8 segments.

\subsection*{Gallow’s Bell Corruption Table}
\label{sec:gallows-bell-corruption}

\begin{longtable}{>{\raggedright\arraybackslash}p{1cm} p{5cm} p{5cm}}
\toprule
\textbf{Tier} & \textbf{Benefit} & \textbf{Cost / Quirk} \\
\midrule
1 & Judge’s Intuition: +1 die to Insight when weighing truth. & Must point out falsehoods when noticed, regardless of tact. \\
\midrule
2 & Quiet Authority: Once/scene, treat a failed Command as success; mark 1~SB (Hearts). & Cannot remain neutral in disputes; indecision costs 1 Fatigue. \\
\midrule
3 & Scales of Balance: Once/session, enforce an exchange of burdens. & Compelled toward fairness even when it hinders you. \\
\midrule
4 & Bell’s Resonance: +2 dice when calling for judgment or demanding restitution. & Suffer 1 Fatigue if wrongdoing is ignored. \\
\midrule
5 & Reckoner’s Call: Once/session, declare a “reckoning moment”—truth must surface or consequence falls. & Cannot ignore pleas for justice without marking 1~SB (Spades). \\
\midrule
6+ & Final Arbiter: Once/session, render an absolute decree; all must obey or suffer consequence. & Mark +2 Obligation; the Bell demands you bear the weight of enforcement. \\
\bottomrule
\end{longtable}
../../srd/patrons/grimmir.tex
% --- Patron: Ikasha, She Who Sleeps (Latent Potential & Shadow) ---
\subsection{Ikasha, She Who Sleeps (Latent Potential \& Shadow)}
\textit{Lore.} Ikasha is the hush between footfalls, the patience of dark water, the black-feathered watcher at every threshold. In stillness she gathers what might be, in crossroads she whispers of what may yet come. Ravens circle her, bearing secrets between worlds.

\begin{quote}
Blow out the candle. If the room listens back, ask softly. At the next crossroads, the raven waits.
\end{quote}

\paragraph{Touch the Umbral Veil (Low, 4 XP)} \emph{Action; Self; Yes (Stealth).}
\textbf{Materials:} A piece of black cloth.\\
\textbf{Effect:} Start \emph{Controlled} on one Stealth roll or gain +1 effect to hide/move quietly.\\
\textbf{Push It:} Brief shadow-muffling (ignore one noisy tell), but leave a shadow-double that may echo you later at an ill moment.\\
\emph{Requires: Familiar \ (\textit{Invoke:} 1 Boon).}

\paragraph{Rite of the Crossroads Raven (Low, 5 XP)} \emph{Scene; Zone; No.}
\textbf{Materials:} Scatter three black feathers or carve a crossroads sign.\\
\textbf{Effect:} Summon an omen-raven; grant \textbf{+1 die} to a navigation, pursuit, or diversion action \emph{or} force an enemy to hesitate at a fateful moment.\\
\textbf{Push It:} The raven speaks one cryptic truth, but demands a secret in return.\\
\emph{Requires: Familiar \ (\textit{Invoke:} 1 Boon).}

\paragraph{Draw from the Umbral Reservoir (Standard, 8 XP)} \emph{Action; Self/Ally; No.}
\textbf{Materials:} A vial of moonless-night water.\\
\textbf{Effect:} \textbf{+2 dice} to stealth, deception, or resolve \emph{or} clear \emph{Fatigue 1}.\\
\textbf{Push It:} Also gain one free escape attempt; next scene, you must help another cross a threshold or flee danger.\\
\emph{Requires: Familiar + Codex \ (\textit{Invoke:} 1 Boon).}

\paragraph{Secret Keeper’s Burden (Standard, 9 XP)} \emph{Instant; Touch; No.}
\textbf{Materials:} A lock of hair or intimate token.\\
\textbf{Effect:} Compel a truthful answer to one direct question (deep secrets may allow a Resolve test to resist).\\
\textbf{Push It:} Learn the answer \emph{and} a key hidden emotion; target learns one of your secrets in return, carried by a raven to them in dreams.\\
\emph{Requires: Familiar + Codex \ (\textit{Invoke:} 1 Boon).}

\paragraph{Become the Shadow at the Crossroads (High, 12 XP)} \emph{Scene; Self; No.}
\textbf{Materials:} Stand in absolute darkness or at a deserted crossroads.\\
\textbf{Effect:} Intangible to mundane harm; pass through thresholds and small gaps; \textbf{+2 dice} to Stealth; auto-succeed one escape. Cannot manipulate normal objects.\\
\textbf{Push It:} Interact once with a bound or thresholded object (a door, a lock, a sealed letter), but you become partially corporeal and vulnerable for one beat. Ravens may mark you.\\
\emph{Requires: Familiar + Codex + Tier III \ (\textit{Invoke:} \textbf{2 Boons}).}\\
\emph{Obligation:} 7 segments.

../../srd/patrons/inaea.tex
../../srd/patrons/isoka.tex
# % --- Patron: Khemesh, the Abyssal Maw (Depths, Inexorability, Eldritch Terror) ---

\subsubsection{Khemesh, the Abyssal Maw (Depths, Inexorability, Eldritch Terror)}
\textit{Lore.} Khemesh is not merely a lord of the depths but the hunger beneath them, a pressure older than seas. Those who bargain with him are marked by the abyss—seen in the way shadows cling, in the whispers heard when no voice speaks, in the certainty that all things will sink.

\begin{quote}
In the trench without light, the Maw waits. Even silence drowns.
\end{quote}

\paragraph*{Whisper of the Trench (Low, 4 XP)} \emph{Instant; Near; No.}
\textbf{Effect:} Target hears impossible echoes and suffers \textbf{−1 die} on their next action.\\
\textbf{Invoke:} 1 action; mark +1 Obligation.\\
\textbf{Push It:} Echoes coil in your own skull—take \textbf{Fatigue 1}, but the target also loses their next minor action.\\
\emph{Requires: Familiar \ (\textit{Invoke:} 1 Boon).}

\paragraph*{Rite of Crushing Silence (Low, 5 XP)} \emph{Scene; Zone; No.}
\textbf{Materials:} A broken shell filled with ink-dark water.\\
\textbf{Effect:} Establish an oppressive silence; sound carries only as distorted whispers. Enemies in the zone gain \textbf{−1 die} to coordination or morale-driven actions.\\
\textbf{Invoke:} 1 action; mark +1 Obligation.\\
\textbf{Push It:} A single enemy's voice is stolen entirely for the scene.\\
\emph{Requires: Familiar \ (\textit{Invoke:} 1 Boon).}

\paragraph{Pressure of the Maw (Standard, 7 XP)} \emph{Instant; Near; No.}
\textbf{Materials:} A length of rusted chain submerged in water.\\
\textbf{Effect:} Target is pinned by invisible crushing force: treat as \texttt{[ENTANGLE]} with \textbf{Great Effect} if underwater or confined.\\
\textbf{Push It:} Inflict \textbf{Fatigue 1} on the target in addition to the restraint.\\
\emph{Requires: Familiar + Codex \ (\textit{Invoke:} 1 Boon).}

\paragraph{Rite of the Abyssal Vision (Standard, 9 XP)} \emph{Scene; Self; No.}
\textbf{Effect:} You perceive the world as Khemesh does—fractured, alien, crushing. Gain \textbf{+2 dice} to Notice and Arcana, and may ask one "true nature" question about a foe or structure.\\
\textbf{Cost:} When the scene ends, you suffer \textbf{Exposure +1} as your perception warps.\\
\textbf{Push It:} Extend the vision to one ally, but both take \textbf{Fatigue 1}.\\
\emph{Requires: Familiar + Codex \ (\textit{Invoke:} 1 Boon).}

\paragraph{The Maw Opens (High, 12 XP)} \emph{Scene; Zone; No.}
\textbf{Materials:} A sealed vessel of abyssal water, broken open.\\
\textbf{Effect:} Reality in the zone folds inward like the crushing deep: 
\begin{itemize}
  \item Enemies act at \textbf{Desperate Position} by default.  
  \item Each beat, the Keeper may force \textbf{1 SB} (Spades/Clubs favored).  
  \item Structures, vessels, or wards fracture as if under immense weight.  
\end{itemize}
\textbf{Push It:} For one beat, declare a single enemy "crushed" (severe harm/effect). You immediately suffer \textbf{Fatigue 2} and \textbf{+1 Obligation}.\\
\emph{Requires: Familiar + Codex + Tier III \ (\textit{Invoke:} \textbf{2 Boons}).}\quad \emph{Obligation:} 8 segments.

\subsection*{Khemesh's Corruption Table}
\label{sec:khemesh-corruption}

\begin{longtable}{>{\raggedright\arraybackslash}p{1cm} p{5cm} p{5cm}}
\toprule
\textbf{Tier} & \textbf{Benefit} & \textbf{Cost / Quirk} \\
\midrule
1 & Abyssal Resilience: +1 die to resist fear and pressure-based effects. & Claustrophobic Comfort: Suffer -1 die in open, well-lit spaces or above ground. \\
\midrule
2 & Crushing Insight: Once per scene, treat a failed Investigation or Arcana roll as a success, but mark 1 SB (Clubs). & Weight of Knowledge: Suffer 1 Fatigue when learning new information that confirms your pessimistic worldview. \\
\midrule
3 & Silent Hunter: Gain +2 dice to Stealth in dark or confined spaces. & Voice of the Deep: When speaking normally, your voice sounds distant and hollow, causing -1 die to social rolls requiring warmth or clarity. \\
\midrule
4 & Pressure Adaptation: Immune to underwater combat penalties; gain +1 die to resist drowning. & Crushing Presence: Allies within Near range suffer -1 die to morale-based rolls due to your oppressive aura. \\
\midrule
5 & Abyssal Sight: Once per session, see through all illusions and deceptions for one exchange, but the truth is always bleak. & Fractured Perception: Suffer -1 die to rolls requiring normal vision; the world appears warped and alien. \\
\midrule
6+ & Inevitable Descent: Once per session, declare that all escape routes in a zone are sealed. For the scene, enemies cannot flee and suffer -2 dice to mobility actions. & Hunger of the Maw: Mark +2 Obligation when using this power; you must consume something (food, memory, hope) to maintain your strength. \\
\bottomrule
\end{longtable}
\subsubsection{Livaea, the Crimson Courtier (Seduction \& Social Binding)}

\paragraph{Lore.}
In salons where wine flows like honey and words cut sharper than daggers, the Crimson Courtier holds court. She is the patron of those who would bind others not with webs, but with desire, obligation, and the sweet poison of whispered promises. Her followers are masters of the intimate covenant, the secret alliance, and the kiss that seals a fate. She teaches that the deepest wounds are those inflicted through trust, and the strongest chains are those forged from willing hands.

\paragraph{Quote.}
\emph{``A word can wound deeper than a blade. A promise can chain stronger than iron. The Courtier knows which words to whisper—and which silences to sell.'' — The Crimson Courtier}

\paragraph{Rite of the Velvet Whisper (Low, 4 XP)} \emph{Scene; Near; No.}
\textbf{Materials:} A silk handkerchief or ribbon touched to lips. \\
\textbf{Effect:} Your next whispered words carry supernatural weight; +1 die to Sway when speaking privately to one target. \\
\textbf{Push It:} The target feels compelled to whisper back a secret of their own, but you mark Exposure +1 and the exchanged confidence creates \textbf{1 SB (Hearts)} as gossip spreads. \\
\emph{Requires: Familiar \ (\textit{Invoke:} 1 Boon).}

\paragraph{Rite of the Intimate Covenant (Low, 5 XP)} \emph{Scene; Touch; No.}
\textbf{Materials:} A shared cup of wine or exchange of personal tokens. \\
\textbf{Effect:} Create a temporary bond of trust; both parties gain +1 die when cooperating, and suffer -1 die when acting against each other this scene. \\
\textbf{Push It:} The bond becomes slightly magical - one party feels the other's emotional state, but you mark Fatigue 1 and the emotional intimacy leaves both parties vulnerable—mark \textbf{1 SB (Diamonds)} as psychic resonance lingers. \\
\emph{Requires: Familiar \ (\textit{Invoke:} 1 Boon).}

\paragraph{Rite of the Binding Vow (Standard, 8 XP)} \emph{Scene; Near; No.}
\textbf{Materials:} A ring or token held while speaking the vow. \\
\textbf{Effect:} Forge a magical agreement between willing parties; +1 Effect when working together, breach forces 1 SB (Hearts) on breaker. \\
\textbf{Push It:} The vow becomes supernaturally enforced - breaker suffers Harm 1 and cannot act against the agreement for one scene; the Court takes note of the binding—mark \textbf{1 SB (Clubs)} as social forces align around the vow. \\
\emph{Requires: Familiar + Codex \ (\textit{Invoke:} 1 Boon).}

\paragraph{Rite of the Court's Favor (Standard, 7 XP)} \emph{Scene; Self; No.}
\textbf{Materials:} Perfumed oil or cosmetic applied before social interaction. \\
\textbf{Effect:} Gain +2 dice to social manipulation in refined settings; you appear perfectly attuned to the social environment. \\
\textbf{Push It:} Become the center of attention - all social actions in the scene focus on you, but you cannot leave unnoticed and attract unwanted admirers—mark \textbf{1 SB (Spades)} as social complications arise. \\
\emph{Requires: Familiar + Codex \ (\textit{Invoke:} 1 Boon).}

\paragraph{Rite of the Crimson Alliance (High, 13 XP)} \emph{Scene; Near; No.}
\textbf{Materials:} A circle of red candles, each representing a participant. \\
\textbf{Effect:} Bind multiple parties in a web of mutual obligation; all participants gain +1 die when acting in group's interest, and suffer Harm 1 if they act against it. \\
\textbf{Push It:} The alliance becomes magically permanent - breaking it requires a ritual and advances a 6-segment "Broken Bonds" clock; the Court's attention intensifies—mark \textbf{2 SB (Hearts)} as social forces take notice. \\
\emph{Requires: Familiar + Codex + Tier III \ (\textit{Invoke:} \textbf{2 Boons}).} \\
\emph{Obligation:} 7 segments.

\paragraph{Rite of the Eternal Court (High, 14 XP)} \emph{Extended; Zone; No.}
\textbf{Materials:} A throne or seat of honor consecrated with rare perfumes. \\
\textbf{Effect:} Establish yourself as the center of a social web; for the next session, all social interactions in your presence are influenced by your will (+1 Effect to your social actions, -1 die to those opposing you). \\
\textbf{Push It:} The court becomes supernaturally compelling - all who enter must test Resolve (DV 3) or become devoted to you for the scene; the Court's influence expands—mark \textbf{2 SB (Diamonds)} as supernatural social pressure affects the wider area. \\
\emph{Requires: Familiar + Codex + Tier III \ (\textit{Invoke:} \textbf{2 Boons}).} \\
\emph{Obligation:} 7 segments.
% --- Patron: Mab, Queen of Courts (Glamour & Bargain) ---

\subsubsection{Mab, Queen of Courts (Glamour \& Bargain)}
\textit{Lore.} Mab rules not from throne or blade, but from dance and debt. She is the smile that binds, the jest that ensnares, the hostess who makes guests complicit in her game. To speak in her Court is to pay; to receive her token is to owe.  

Where others rule by force, Mab rules by etiquette, glamour, and the hidden hook in every gift. Her followers thrive on charm, wit, and story, spreading webs of bargains too subtle to escape. The Cantor’s Path sings her name most sweetly, for every verse carries a price.  

\begin{quote}
Every laugh is a promise. Every promise is a debt. Every debt belongs to Mab.  
\end{quote}

% --------------------
% RITES
% --------------------

\paragraph*{Rite of the Trickster’s Bargain (Low, 4 XP)} \emph{Scene; Near; No.}  
\textbf{Materials:} A token freely given (flower, coin, ribbon).\\
\textbf{Effect:} Offer a fae bargain. Target must choose: accept (both gain +1 die to fulfill terms this scene) or refuse (mark 1 Stress [Hearts]).\\
\textbf{Push It:} Seal it in glamour—betrayal inflicts Harm~1 (Stress) and begins a “Bargain Broken [4]” clock.\\
\emph{Requires: Familiar (\textit{Invoke:} 1 Boon).}

\paragraph*{Courtly Guise \textnormal{[VEIL]} (Low, 4 XP)} \emph{Action; Self; Yes (social only).}  
\textbf{Materials:} Pin a sprig or silver thread.\\
\textbf{Effect:} Subtle glamour: +1 die to Persuade/Sway in refined settings; you appear as expected rank/guest.\\
\textbf{Push It:} Mask one personal tell; the first probing question in scene generates 1 SB (Hearts).\\
\emph{Requires: Familiar (\textit{Invoke:} 1 Boon).}

\paragraph*{Token of Favor (Low, 5 XP)} \emph{Scene; Near; No.}  
\textbf{Materials:} A ribbon, ring, or charm bestowed.\\
\textbf{Effect:} Ally gains +1 die to social actions before witnesses; you gain +1 Effect when aiding them.\\
\textbf{Push It:} The token stills hecklers (one beat of hesitation), but you mark +1 Exposure.\\
\emph{Requires: Familiar (\textit{Invoke:} 1 Boon).}

\paragraph{Mirror of Motives (Standard, 7 XP)} \emph{Action; Near; No.}  
\textbf{Materials:} A polished shard or mirror.\\
\textbf{Effect:} Ask one sharp question about a target’s immediate social aim; Keeper reveals it. Gain +1 die exploiting it this scene.\\
\textbf{Push It:} Also surface a concealed slight or insult; generate 1 SB (Hearts) on that target.\\
\emph{Requires: Familiar + Codex (\textit{Invoke:} 1 Boon).}

\paragraph{The Price Agreed \textnormal{[OATH]} (Standard, 8 XP)} \emph{Scene; Near; No.}  
\textbf{Materials:} Exchange equal tokens.\\
\textbf{Effect:} Bind a petty bargain. Breach forces 1 SB (Hearts or Diamonds) and tarnishes reputation.\\
\textbf{Push It:} Add a minor boon (+1 die once) to sweeten terms; you suffer 1 SB (Hearts) if the other breaches.\\
\emph{Requires: Familiar + Codex (\textit{Invoke:} 1 Boon).}

\paragraph{Sovereign Glamour \textnormal{[VEIL][REVEAL]} (High, 11 XP)} \emph{Scene; Zone; No.}  
\textbf{Materials:} A circle of silk or green felt.\\
\textbf{Effect:} Establish Court: allies gain +1 die to social rolls; blunt threats suffer -1 die. Once, strip away one disguise/illusion.\\
\textbf{Push It:} Name a Court Law (e.g. “no steel drawn”); first violator suffers 2 SB.\\
\emph{Requires: Familiar + Codex + Tier III (\textit{Invoke:} 2 Boons).}\\
\emph{Obligation:} 6 segments.

% --------------------
% CORRUPTION
% --------------------

\subsection*{Mab’s Corruption Table}
\label{sec:mab-corruption}

\begin{longtable}{>{\raggedright\arraybackslash}p{1cm} p{5cm} p{5cm}}
\toprule
\textbf{Tier} & \textbf{Gift} & \textbf{Burden} \\
\midrule
1 & Glamour’s Touch: +1 die to Deception or Performance when telling stories or lies. & Cannot speak a plain falsehood; only mislead through implication or wordplay. \\
\midrule
2 & Fairy Step: Once per scene, flicker Near as if by teleport. & Cold Iron Weakness: Suffer 1 Fatigue if touched or struck by iron. \\
\midrule
3 & Trickster’s Delight: Spend 1 Boon to twist a Complication into comic or ironic advantage. & Compulsive Jest: Must play a trick each session or mark 1 SB (Hearts). \\
\midrule
4 & Gift of Hospitality: Allies who share your food/drink gain +1 die to Resolve rolls. & Hospitality Bound: Harming those who accept it costs +2 Obligation. \\
\midrule
5 & Fae Sight: Perceive invisible doors, veils, glamours; +2 dice to Notice them. & Truth Debt: Must accept any “fair” trade offered, or mark 1 Fatigue resisting. \\
\midrule
6+ & Crown of Twilight: Once/session, declare an Oath. All rolls toward that Oath gain +2 dice. & Oathbound: Breaking it inflicts 1 Harm (Stress) and begins an “Oathbreaker [6]” clock. \\
\bottomrule
\end{longtable}
# % --- Patron: Maelstraeus, The Infernal Bargainer (Commerce & Exchange) ---

\subsubsection{Maelstraeus, The Infernal Bargainer (Commerce \& Exchange)}
\textit{Lore.} Maelstraeus is the Merchant of Equities, the Infernal Bargainer who sits at the crossroads of every transaction in the cosmos. Neither wholly demon nor angel, but a conceptual force born from the first exchange—the moment when one thing was traded for another, and debt was born.

He dwells in the pause between offer and acceptance. His realm is a vast marketplace where every deal ever struck echoes through eternity, where contracts signed in blood still hold power, and where the true price of everything is known—even if it is never spoken aloud.

Maelstraeus embodies the principle that all value can be exchanged, that every relationship is transactional, and that the universe itself operates on a vast economy of favors, debts, and obligations. His followers learn that everything has a price, but also that everything can be bought.

\begin{quote}
All things have value. All values can be traded. The Merchant sees the true price—and always collects his due.
\end{quote}

\paragraph*{Rite of the Fair Trade (Low, 4 XP)} \emph{Scene; Near; No.}
\textbf{Materials:} A balance scale with equal weights.\\
\textbf{Effect:} Establish a neutral trading ground. All parties gain +1 die to negotiate in good faith. Create a 4-segment \emph{Equity Maintained} clock that can be spent to downgrade a social complication.\\
\textbf{Invoke:} 1 action; mark +1 Obligation.\\
\textbf{Push It:} Compel one party to reveal their true terms; mark 1 SB (Hearts).\\
\emph{Requires: Familiar \ (\textit{Invoke:} 1 Boon).}

\paragraph*{Rite of the Merchant's Eye (Low, 5 XP)} \emph{Scene; Self; No.}
\textbf{Materials:} A foreign coin or token.\\
\textbf{Effect:} Gain +2 dice to appraise goods, judge value, or spot market opportunities. Create a 6-segment \emph{Market Insight} clock.\\
\textbf{Invoke:} 1 action; mark +1 Obligation.\\
\textbf{Push It:} Also sense emotional value of items, but mark Exposure +1 and 1 SB (Diamonds).\\
\emph{Requires: Familiar \ (\textit{Invoke:} 1 Boon).}

\paragraph{Rite of the Balanced Exchange \textnormal{[OATH]} (Standard, 8 XP)} \emph{Scene; Near; No.}
\textbf{Materials:} Two items of perceived equal value.\\
\textbf{Effect:} Facilitate a fair trade; both sides gain +1 Effect. If unfair, the disadvantaged party gains +2 dice to resist. Create a 6-segment \emph{Fair Dealing} clock.\\
\textbf{Push It:} Enforce magically: breaking terms causes 1 SB (Hearts/Clubs).\\
\emph{Requires: Familiar + Codex \ (\textit{Invoke:} 1 Boon).}

\paragraph{Rite of the Contract Seal \textnormal{[BIND]} (Standard, 7 XP)} \emph{Scene; Touch; No.}
\textbf{Materials:} An official seal or stamp.\\
\textbf{Effect:} Mark an agreement with authority; +1 die to enforce, -1 die for those who break it. Contract becomes [BIND]ed—breaking it imposes -1 die to all social rolls for one session. Create an 8-segment \emph{Binding Agreement} clock.\\
\textbf{Push It:} Breach ignites the document and inflicts Harm~1; mark 1 SB (Spades).\\
\emph{Requires: Familiar + Codex \ (\textit{Invoke:} 1 Boon).}

\paragraph{Rite of the Great Market \textnormal{[WARD][COMMAND]} (High, 13 XP)} \emph{Scene; Zone; No.}
\textbf{Materials:} A consecrated booth or trading post.\\
\textbf{Effect:} Create a zone of commerce. Allies gain +1 Effect on trades; may reroll one failed social roll. Enemies suffer -1 die to deception. Create a 10-segment \emph{Market Dominance} clock.\\
\textbf{Push It:} Attract powerful allies and rivals; mark 2 SB (Hearts/Clubs).\\
\emph{Requires: Familiar + Codex + Tier III \ (\textit{Invoke:} \textbf{2 Boons}).}\\
\emph{Obligation:} 7 segments.

\paragraph{Rite of the Cosmic Ledger \textnormal{[CLEANSE][CURSE]} (High, 14 XP)} \emph{Extended; Self; No.}
\textbf{Materials:} A book recording debts and credits.\\
\textbf{Effect:} Once per session, convert one resource into another (e.g. 1 Boon $\rightarrow$ 1 Fatigue). Settle debts at true cosmic value. Create a 6-segment \emph{Ledger Balance} clock.\\
\textbf{Push It:} Make an imbalanced trade in your favor; create a 6-segment \emph{Karmic Debt} clock; mark 2 SB (Diamonds).\\
\emph{Requires: Familiar + Codex + Tier III \ (\textit{Invoke:} \textbf{2 Boons}).}\\
\emph{Obligation:} 8 segments.

\subsection*{Maelstraeus's Corruption Table}
\label{sec:maelstraeus-corruption}

\begin{longtable}{>{\raggedright\arraybackslash}p{1cm} p{5cm} p{5cm}}
\toprule
\textbf{Tier} & \textbf{Benefit} & \textbf{Cost / Quirk} \\
\midrule
1 & Appraiser's Eye: +1 die to evaluate goods, services, or negotiate any exchange. & Transactional Mindset: Must calculate personal benefit/cost in social interactions; suffer -1 die to acts of genuine kindness. \\
\midrule
2 & Bargaining Instinct: Once per scene, re-roll any failed negotiation or trade-related roll. & Compulsive Deal-Making: Must attempt to negotiate or trade in any situation where value is exchanged, even inappropriately. \\
\midrule
3 & Merchant's Luck: Gain +1 die to rolls involving market fluctuations, investment opportunities, or economic predictions. & Greed's Whisper: Suffer 1 Fatigue when passing up obvious profitable opportunities or acts of generosity. \\
\midrule
4 & Cosmic Connections: Once per session, call in a favor from a powerful economic figure (merchant, banker, guild master). & Debt Attraction: Automatically attract offers of "easy money" or deals with hidden costs; mark 1 SB (Diamonds) when refusing. \\
\midrule
5 & Value Sight: Once per scene, instantly recognize the true value of any item, person, or opportunity. & Price-Tag Perception: See everyone and everything with a cosmic "price tag"; suffer -1 die to relationships not based on mutual benefit. \\
\midrule
6+ & Merchant Prince: Once per session, establish absolute market dominance in a specific area for one scene. All economic transactions favor you; others suffer -2 dice in financial dealings. & Cosmic Debt: Mark +2 Obligation when using this power; the universe demands immediate repayment, often in unexpected forms. \\
\bottomrule
\end{longtable}
\section{Mykkiel — Arbiter of the Writ}
\label{patron:mykkiel}

\subsection*{Lore}
\index{Patrons!Mykkiel}%
Mykkiel is the Arbiter of justice and keeper of sacred covenants, weighing speech against deed and sealing verdicts in cold iron. His sigil—balanced scales crossed by a sword—marks benches of judgment and sealed cells from the Sapphire Marches to the Sunward Courts. 

Mykkiel does not merely enforce law; he embodies the principle that justice requires both mercy and judgment. Law without compassion curdles into tyranny; compassion without structure dissolves into harm. His followers learn that true authority is forged from evidence weighed, precedent applied, and oaths honored.

\begin{quote}
``Name the charge. Name the terms. Then sign where you will bleed if you are wrong. The Word made manifest cannot be unsaid.''
\end{quote}

\subsection*{Patron's Gift: Arbiter's Authority}
Once per scene as an action (cost: 1 Boon; requires Thiasos), touch an item or person to imbue it until scene end. The target gains +1 die and +1 Effect when used in lawful proceedings, judgment, or authoritative command.\\
\textbf{Push It:} Extend for one additional scene by marking +1 Obligation. The court’s notice falls upon the scene.

\subsection*{Low Rites}

\paragraph{Rite of the Stamp of Authority (Low)}%
\emph{Duration: Scene; Range: Near. Materials: Cold-iron seal or writ-tag.}\\
Project visible legitimacy. Gain +1 die to Command/Sway when asserting a lawful claim or order; challengers suffer --1 die to resist. Create a 4-segment \emph{Legal Standing} clock that can be spent to downgrade a legal complication.\\
\textbf{Invoke:} 1 action; +1 Obligation.\\
\textbf{Push It:} A brief hush (one beat) stills hecklers; mark \emph{Exposure} +1 as higher authorities take interest.

\paragraph{Rite of Proper Notice (Low)}%
\emph{Duration: Scene; Range: Near. Materials: Writ-string tied and snapped.}\\
Name a lawful venue (dais, doorway, wagon). The first hostile act committed there suffers --1 die. Create a 6-segment \emph{Sacred Venue} clock protecting the designated area.\\
\textbf{Invoke:} 1 action; +1 Obligation.\\
\textbf{Push It:} Name a protected act (parley, surrender, testimony) gaining +1 Effect in the venue; breaking custom generates 1~SB (Hearts) and marks the breaker before the covenant courts.

\subsection*{Standard Rites}

\paragraph{Rite of the Writ of Compliance [COMMAND] (Standard)}%
\emph{Duration: Instant; Range: Near. Materials: Red cord knotted while speaking the order.}\\
Issue an immediate, simple command (``Stand down,'' ``Drop it,'' ``Open''). Target must comply or suffer a stated cost; DV by fiction (elites may test Resolve). On success, create a 4-segment \emph{Lawful Compliance} token to auto-succeed on a similar lawful command later.\\
\textbf{Invoke:} 1 action; +1 Obligation.\\
\textbf{Push It:} On compliance, impose --1 die on the target’s next aggressive act this scene; the order sets precedent—mark 1~SB (Spades).

\paragraph{Rite of the Speaking Seal [BIND] (Standard)}%
\emph{Duration: Scene; Range: Near. Materials: Wax seal impressed over a name or sigil.}\\
Sanctify a statement (truce, custody, claim). Contradictors suffer --1 die; you gain +1 die to enforce. If the seal is broken, the breaker suffers Harm~1 (Legal) and --2 dice to social rolls involving magistrates for one session. Create an 8-segment \emph{Binding Seal} clock.\\
\textbf{Invoke:} 1 action; +1 Obligation.\\
\textbf{Push It:} Once, ask who intends breach; the Keeper provides a strong clue or a name. Mark 1~SB (Diamonds) as truth draws covenant notice.

\subsection*{High Rites}

\paragraph{Rite of the Oath Irons [OATH][WARD] (High)}%
\emph{Duration: Extended; Range: Near. Materials: Two iron pins warmed, touched to wrists, then quenched.}\\
Bind two parties to a bounded term. Breach forces 2~SB and brands a faint iron-mark until amends. The oath is [BIND]ed to both; breaking it imposes --2 dice to all legal proceedings for one session. Create a 6-segment \emph{Sacred Oath} clock.\\
\textbf{Invoke:} Extended; +2 Obligation.\\
\textbf{Push It:} Extend to a small circle (up to four), each choosing one narrow exception (Keeper approves). Exploiting an exception generates 1~SB (Diamonds) as the covenant’s complexity invites scrutiny.

\paragraph{Rite of the Final Judgment [CLEANSE][CURSE] (High)}%
\emph{Duration: Extended; Range: Near. Materials: Complete record of a case, signed by recognized authority.}\\
Render a final, supernaturally enforced verdict. Target tests Spirit+Resolve (DV~5); on failure, full consequences apply without appeal. On success, limited appeal is possible, but the target suffers --2 dice on appeals for one arc. Create a 10-segment \emph{Divine Verdict} clock affecting all proceedings involving the target.\\
\textbf{Invoke:} Extended; +3 Obligation.\\
\textbf{Push It:} Make the verdict absolute and unappealable; create 2~SB (Hearts/Spades) as allies mobilize to overturn justice by other means.

\subsection*{Obligation Progression}
Starts at 6 for Tier II characters, scaling upward.

\paragraph{Obligation 9+} Mykkiel demands judgment where law and mercy clash or precedent fails. Refusal: your legal gifts falter for one session (--2 dice to lawful actions) and you generate 1~SB when asserting authority.

\paragraph{Obligation 11+} Law saturates your sight. Permanent Condition: \emph{Judge’s Eye} (--2 dice to informal or intimate interactions) until you complete a quest to \emph{learn when to look away}.

\subsection*{Persistent Condition: Arbiter’s Sight}
+2 dice to legal proceedings, judgment, and authoritative command; --1 die to acts of personal leniency or rule-bending. Narrative: you read the writ stamped upon every deed, yet the human cost grows heavy.

\subsection*{Rivalries}
\begin{itemize}
  \item \textbf{The Inquisitor Prime:} Tension—purity by ordeal vs.\ justice by process.
  \item \textbf{Morag the Hag:} Antagonism—cunning bargains vs.\ sanctioned oaths.
  \item \textbf{The Witness:} Opposition—revealed truths that unsettle precedent.
\end{itemize}

\subsection*{Covenant Courts and Magisterial Practice}
Mykkiel’s tradition rests on three pillars:
\begin{enumerate}
  \item \textbf{The Word:} Precise language of law that cannot be unsaid.
  \item \textbf{The Balance:} Evidence weighed and precedent applied.
  \item \textbf{The Seal:} Binding power that makes judgment manifest.
\end{enumerate}
Rites often involve writs, seals, oaths, and formal venues. Adherents serve as magistrates, advocates, clerks, and peace-brokers, maintaining archives of cases and precedent.

\subsection*{Playtest Scenario: The Merchant’s Trial}
A wealthy factor stands accused of ruining a rival by means lawful yet unjust. The quarter is split between letter and spirit of the law. The party must navigate factions and render or influence judgment.

\begin{itemize}
  \item \emph{Stamp of Authority} to establish standing in proceedings.
  \item \emph{Writ of Compliance} to still a tumultuous court.
  \item \emph{Speaking Seal} to bind testimony and terms.
  \item \emph{Oath Irons} to forge a settlement both parties must honor.
  \item \emph{Proper Notice} to sanctify the venue and prevent violence.
  \item \emph{Final Judgment} if a definitive precedent must be set.
\end{itemize}

\noindent Paths of resolution include strict legality, moral restitution, creative compromise, or procedural integrity—each testing Mykkiel’s balance of mercy and law.

% --- Patron: Nidhoggr, the World-Worm (Dreaming Antiquity) ---

\subsubsection{Nidhoggr, the World-Worm (Dreaming Antiquity)}
\textit{Lore.} Beneath stone and sleep coils \textbf{Nidhoggr}, who gnaws at the roots of time. He does not speak quickly; he dreams in centuries. To press your ear to the earth is to risk drowning in the silence of aeons. Yet for those who endure, he whispers truths long buried, memories fossilized in stone, and the slow inevitability of cycles unbroken. His followers walk in twilight between dream and ruin, bearing the weight of all that has been.

\begin{quote}
Press your ear to the earth and wait. If it remembers you, it will answer.
\end{quote}

\paragraph*{Glimpse the Ancient’s Shadow (Low, 4 XP)} \emph{Action; Self; No.}\\
\textbf{Materials:} Dust ground from a weathered stone.\\
\textbf{Effect:} +1 die to interpret \emph{ancient} places, scripts, or artifacts; once this scene, ask one yes/no about the site’s original purpose.\\
\textbf{Push It:} Gain +1 Effect, but mark \emph{Fatigue 1} as stone’s patience weighs on you.\\
\emph{Requires: Familiar.}

\paragraph*{Drink from the Dreaming Deep (Low, 5 XP)} \emph{Instant; Self; No.}\\
\textbf{Materials:} Water poured over stone, swallowed with eyes closed.\\
\textbf{Effect:} Learn one hidden fact about the locale’s past; GM may reveal through echo or dream.\\
\textbf{Push It:} The vision lingers too clearly—gain an additional detail, but mark Exposure +1 and 1 SB (Clubs).\\
\emph{Requires: Familiar.}

\paragraph{Stone-Sleeper’s Murmur (Standard, 7 XP)} \emph{Scene; Near (touch locus); No.}\\
\textbf{Materials:} Ear pressed to bedrock, wall, or pillar.\\
\textbf{Effect:} Ask up to 3 questions about events once imprinted in this stone; answers are fragmentary but true.\\
\textbf{Push It:} One answer is delivered with precise sensory clarity; generate 1 SB (suit by Keeper).\\
\emph{Requires: Familiar + Codex.}

\paragraph{Awakened Chronicle (Standard, 9 XP)} \emph{Ritual; Zone; No.}\\
\textbf{Materials:} Chalk spiral and four local touchstones.\\
\textbf{Effect:} The zone replays a past moment in spectral echoes, visible to all. Witnesses gain +2 dice to Investigate/Recall about it.\\
\textbf{Push It:} Invoke a second memory from a different age; mark +1 Obligation.\\
\emph{Requires: Familiar + Codex.}

\paragraph{Dive into the World-Worm’s Dream (High, 12 XP)} \emph{Scene; Self; No.}\\
\textbf{Materials:} Circle of stones under open sky, lain upon in stillness.\\
\textbf{Effect:} Ask up to 3 questions about the \emph{distant past} or \emph{buried truths} here. Answers come as lucid dreams and omens.\\
\textbf{Push It:} The dream stretches into prophecy: gain +3 dice to one occult action, but mark 2 SB immediately.\\
\emph{Requires: Familiar + Codex + Tier III.}\\
\emph{Obligation:} 7 segments.

\paragraph{Eclipse of Aeons (High, 14 XP)} \emph{Extended; Zone; No.}\\
\textbf{Materials:} Stones from three ruins aligned in a circle.\\
\textbf{Effect:} Submerge a zone in deep-time resonance. For one session, history bleeds into present: ruins reform, shadows of the dead walk, and forgotten oaths stir. Allies gain +2 dice to Recall, Divination, or Investigation; enemies suffer -1 die when relying on the present alone.\\
\textbf{Push It:} The bleed becomes permanent until countered; mark +2 Obligation and start an 8-segment \emph{Time Fracture} clock.\\
\emph{Requires: Familiar + Codex + Tier III.}\\
\emph{Obligation:} 8 segments.

\subsection*{Nidhoggr’s Corruption Table}
\label{sec:nidhoggr-corruption}

\begin{longtable}{>{\raggedright\arraybackslash}p{1cm} p{5cm} p{5cm}}
\toprule
\textbf{Tier} & \textbf{Benefit} & \textbf{Cost / Quirk} \\
\midrule
1 & Ancient Sight: +1 die to Lore when reading ruins or relics. & Temporal Drift: Speak or think in archaic patterns; -1 die in modern social dealings. \\
\midrule
2 & Stone Memory: Once/session, recall one precise fact of history as if witnessed. & Heavy Mind: Suffer 1 Fatigue when confronted with lies or distortions of history. \\
\midrule
3 & Earth’s Whisper: +2 dice to Notice when listening to stone or soil. & Root-Bound: -1 die to aerial or swift actions; feel tethered to ground. \\
\midrule
4 & Dreaming Insight: Once/scene, gain +1 die to actions tied to ancient mysteries. & Haunted Sleep: Dreams replay past ages; mark 1 SB (Clubs) if rest is denied. \\
\midrule
5 & World-Worm’s Gaze: Once/session, see through the earth to a distant ancient site. & Slow Pulse: -1 die to reactions requiring haste; act with ponderous inevitability. \\
\midrule
6+ & Aeons of Memory: Once/session, touch stone to access the full record of a place. Gain +3 dice to historical investigation. & Overload: Mark +2 Obligation and suffer 1 Harm (Stress); visions leave you vulnerable until they fade. \\
\bottomrule
\end{longtable}
% --- Patron: Oath of Flame & Light (Dawn & Vows) ---

\subsubsection{Oath of Flame \& Light (Dawn \& Vows)}
\textit{Lore.} The Oath of Flame \& Light is no patron of half-measures. Its fire names, binds, and burns—demanding that those who swear within its radiance stand openly, speak truly, and pay the cost of keeping their word. At dawn altars, the sworn kindle sparks of consecrated fire; in battle, they blaze as torches that hold back the night. To follow this Oath is to live in public truth, with no shadow to hide in and no retreat from the vow once spoken.

\begin{quote}
“Swear in the light. Keep it, or the light will keep \emph{you}.” 
\end{quote}

\paragraph*{Kindle Vow (Low, 4 XP)} \emph{Action; Self/Ally; Yes.}\\
\textbf{Materials:} A glass ampoule of consecrated flame cracked to spark.\\
\textbf{Effect:} Declare a short vow for this scene (\emph{hold the gate}, \emph{shield the weak}). The bearer gains \textbf{+1 die} to any action fulfilling it.\\
\textbf{Push It:} The first hesitation or betrayal \emph{forces 1 SB (Hearts)} on the bearer.\\
\emph{Requires: Familiar.}

\paragraph*{Lay on Hands \textnormal{[CLEANSE][HEAL]} (Low, 5 XP)} \emph{Instant; Touch; No.}\\
\textbf{Materials:} Bare palm pressed to the wound while whispering a vow.\\
\textbf{Effect:} Cleanse one affliction, downgrade Harm by 1, or remove Fatigue 1. For deep curses or poisons, test Resolve (DV by fiction).\\
\textbf{Push It:} Target gains \textbf{+1 die} to their next Resist this scene, but you mark Exposure +1.\\
\emph{Requires: Familiar.}

\paragraph{Sunlit Parley (Standard, 7 XP)} \emph{Scene; Near; No.}\\
\textbf{Materials:} A vow-ring engraved with a sunrise and a true name.\\
\textbf{Effect:} Establish terms in the open light: honest persuasion gains \textbf{+1 die}; deceit suffers \(-1\) die in this parley.\\
\textbf{Push It:} Demand one public answer; evasion \emph{forces 1 SB (Hearts)} on the evader.\\
\emph{Requires: Familiar + Codex.}

\paragraph{Radiant Smite \textnormal{[FOLLOW-UP]} (Standard, 8 XP)} \emph{Action; Self; No.}\\
\textbf{Materials:} Consecrated spark smeared on a weapon or badge.\\
\textbf{Effect:} Your next melee strike this scene flares with dawnfire: upgrade Effect by one step, and add +1 Harm (Burn) \emph{or} force 1 SB (Spades) if narrative.\\
\textit{Special:} Against undead, oath-breakers, or outsiders: sears them with light—oath-breakers suffer \(-1\) die, outsiders gain +1 Exit Tally segment.\\
\textbf{Push It:} On hit, burst of light drives back enemies in Close (worse Position or \(-1\) die). Mark +1 Obligation.\\
\emph{Requires: Familiar + Codex.}

\paragraph{Purge the Shadow \textnormal{[REVEAL][DISPEL]} (Standard, 9 XP)} \emph{Instant; Near; No.}\\
\textbf{Materials:} A consecrated spark shattered to light.\\
\textbf{Effect:} Reveal illusions and suppress one ongoing glamour/curse in Near.\\
\textbf{Push It:} Brand the source with a visible tell for this arc; mark 1 SB (Diamonds).\\
\emph{Requires: Familiar + Codex.}

\paragraph{Covenant Blaze \textnormal{[OATH][FORTIFY]} (High, 12 XP)} \emph{Scene; Zone; No.}\\
\textbf{Materials:} A brazier lit while three names are spoken.\\
\textbf{Effect:} Those who swear within are haloed: +1 die to keep the oath; aggressors against them suffer \(-1\) die if violating the terms. Oath-breakers suffer 2 SB (Hearts/Spades) and Harm~1 (Burn).\\
\textbf{Push It:} The blaze sanctifies the threshold: one beat of \texttt{[WARD]} against oath-breakers entering.\\
\emph{Requires: Familiar + Codex + Tier III.}\\
\emph{Obligation:} 7 segments.

\subsection*{Oath of Flame \& Light Corruption Table}
\label{sec:oath-flame-light-corruption}

\begin{longtable}{>{\raggedright\arraybackslash}p{1cm} p{5cm} p{5cm}}
\toprule
\textbf{Tier} & \textbf{Benefit} & \textbf{Cost / Quirk} \\
\midrule
1 & Oathbound Strength: +1 die when upholding a vow or defending the innocent. & Rigid Honor: Must uphold vows even when disadvantageous; suffer \(-1\) die when acting flexibly. \\
\midrule
2 & Radiant Sight: Once/scene, +2 dice to pierce lies, glamours, or corruption. & Blinding Truth: \(-1\) die on subtlety or deception; cannot easily feign. \\
\midrule
3 & Holy Flame: +1 die on melee vs. undead, outsiders, or oath-breakers. & Burden of Light: Suffer Fatigue~1 when concealing identity or working in darkness. \\
\midrule
4 & Unwavering Resolve: Once/session, treat failed Resolve/Command as success; mark 1 SB (Hearts). & Absolutist Stance: \(-1\) die in morally ambiguous dealings. \\
\midrule
5 & Dawn’s Benediction: Once/session, heal allies within Near of Fatigue~1 and minor Conditions. & Beacon’s Call: Your aura reveals you; enemies seeking you gain +1 die. \\
\midrule
6+ & Avatar of the Oath: Once/session, embody living covenant—gain +2 dice to all protection, justice, or vow-keeping rolls. Breaking any vow inflicts Harm~2 (Burn). & Burden of Radiance: Mark +2 Obligation when used; the light makes you a beacon for foes and trials alike. \\
\bottomrule
\end{longtable}
\section{Raéyn --- Mistress of the Sea}
\label{patron:raeyn}

\subsection*{Lore}
\index{Patrons!Raéyn}%
Raéyn is the tempestuous goddess of the sea, the restless tide that carries news between shores and the promise of change between lives. She is mother to all who sail, her voice the wind that fills sails and her moods the storms that test every mariner's resolve.

But Raéyn's heart is torn by her greatest tragedy: her son Khemesh, the Kraken of the Depths, who embodies the crushing inevitability of the ocean's dark heart. Where Raéyn brings change and opportunity, Khemesh brings the final, inescapable pressure that grinds all things to nothing. Sailors pray to Raéyn for safe passage and favorable winds, but whisper Khemesh's name when seeking to lay the dead to rest beneath the waves.

Raéyn is passionate, mercurial, and fiercely protective of those who respect her domain. She favors those who read currents, bargain with weather, and carry news between shores. But cross her, and the sea itself becomes your enemy: fair weather turns to fury, and every wave a judgment.

\begin{quote}
``Mark the tide, name your course, and trust the wave-road. But speak ill of Khemesh, and even I may let the deep take you.''
\end{quote}

\subsection*{Patron's Gift: Tide's Favor}
Once per scene as an action (cost: 1 Boon; requires Thiasos), you may touch a weapon, vessel, or item to imbue it until the end of the scene. The object gains +1 die and +1 Effect when used in maritime contexts or situations involving change, travel, or currents.  

\textbf{Push It:} Extend the blessing for one additional scene by marking +1 Obligation. The sea's attention becomes noticeable to other sailors.

\subsection*{Low Rites}
\paragraph{Rite of the Tidemark's Blessing (Low)}  
\emph{Duration: Scene; Range: Self. Materials: A knotted length of salt-twine brushed with seawater.}  
Treat slick, swaying, or water-slicked footing as stable for you this scene. Gain +1 die on boarding, balance, or shipboard movement. Create a 4-segment \emph{Tide's Favor} clock that can be spent to ignore one level of difficult terrain.  
\textbf{Invoke:} 1 action; mark +1 Obligation.  
\textbf{Push It:} Extend to one ally in Close for one beat, but generate 1 SB (Spades: shifting deck/hazards).

\paragraph{Rite of the Whispering Currents (Low)}  
\emph{Duration: Instant; Range: Self. Materials: A shell held to the ear while facing the wind.}  
Learn the safest near-term route across water or coastline (reefs, eddies, patrols) or gain +1 die to navigation checks for this scene. If Khemesh's influence is present, suffer --1 die from conflicting currents.  
\textbf{Invoke:} 1 action; mark +1 Obligation.  
\textbf{Push It:} Also learn the fastest route, but mark Exposure +1 (leaving a telltale wake).

\subsection*{Standard Rites}
\paragraph{Rite of the Changing Tide [PASSAGE] (Standard)}  
\emph{Duration: Scene; Range: Zone (water-adjacent). Materials: A handful of pebbles cast in a crescent.}  
Bias currents and water levels in the zone. Those moving with the tide gain +1 die; those moving against suffer --1 die. Small craft must test to hold position. Create a 6-segment \emph{Tidal Influence} clock.  
\textbf{Invoke:} 1 action; mark +1 Obligation.  
\textbf{Push It:} Brief surge or drawdown (one beat): open a ford or swamp a skiff; mark +1 Obligation.

\paragraph{Rite of the Wave-Road Blessing [WARD] (Standard)}  
\emph{Duration: Scene; Range: Route (sea-to-sea). Materials: Two sea-glass markers dropped overboard at start and end.}  
Consecrate a wave-road between two visible points. Allies gain +2 dice on travel, evade, or carry actions at sea. Designated pursuers suffer --1 die to intercept. One active wave-road at a time. Create an 8-segment \emph{Blessed Passage} clock.  
\textbf{Invoke:} 1 action; mark +1 Obligation.  
\textbf{Push It:} Extend the route's favor to an adjacent leg for one beat; mark +1 Obligation.

\subsection*{High Rites}
\paragraph{Rite of the Storm-Queen's Hand [AREA][FOLLOW-UP] (High)}  
\emph{Duration: Scene; Range: Zone (sea/shore/sky). Materials: A vial of rainwater gathered at three crossings.}  
Shape a storm-band over the zone. Choose two modes at cast; switch one once per scene:  
\begin{itemize}
\item \textbf{Propulsion:} Vessel gains +1 band of movement per beat (or +1 Effect to maneuvers).  
\item \textbf{Concealment:} Veil of rain/spray; ranged targeting impaired; --1 die to hostile sighting.  
\item \textbf{Smite:} Once per beat, lash with wave or lightning as [AREA] hazard.  
\end{itemize}
\textbf{Invoke:} 1 action; mark +2 Obligation.  
\textbf{Push It:} Add a third mode for one beat, then GM spends 1 SB on collateral; mark +1 Obligation.

\paragraph{Rite of the Mother's Wrath [BANISH][CURSE] (High)}  
\emph{Duration: Extended; Range: Zone. Materials: Tears of a betrayed lover mixed with salt from seven seas.}  
Curse those who wronged you. Target suffers --2 dice to maritime/weather rolls for one session. At sea, they must roll Spirit + Resolve (DV 4) each day or suffer Harm~1 (Weather). Create a 6-segment \emph{Mother's Ire} clock.  
\textbf{Invoke:} Extended ritual; mark +3 Obligation.  
\textbf{Push It:} Curse spreads to target’s allies/family; mark 2 SB (Diamonds).

\subsection*{Obligation Progression}
Starts at 6 for Tier II characters, scaling with tier.

\paragraph{Obligation 9+} Raéyn demands proof of devotion---navigate a dangerous passage, recover a lost treasure, or confront Khemesh's servants. Refusal causes all maritime rolls to suffer --2 dice and generate 1 SB when weather is involved.  

\paragraph{Obligation 11+} Khemesh notices you. You are hunted by his servants; deep water becomes perilous even under Raéyn's protection. Requires a quest to prove worth or appease both mother and son.

\subsection*{Persistent Condition: Child of the Tide}
Gain +2 dice on maritime travel, weather prediction, and navigation. Suffer --1 die on prolonged time away from the sea. The sea’s rhythm flows in your blood, making you exceptional at sea but restless on land.

\subsection*{Rivalries}
\begin{itemize}
\item \textbf{Khemesh:} Direct antagonism---mother’s change vs. son’s crushing pressure.  
\item \textbf{The Traveler:} Tension---fluid paths vs. fixed ways.  
\item \textbf{The Sealed Gate:} Opposition---Raéyn opens passages, Gates close them.  
\end{itemize}

\subsection*{Connection to Maritime Culture}
Raéyn’s rites emphasize the philosophy that the sea is not an obstacle but a partner. Her worship blends aid, hindrance, and the inevitability of change. The mother--son dynamic adds depth to coastal culture: Raéyn for the living, Khemesh for the dead.

\subsection*{Playtest Scenario: The Kraken's Gambit}
A trading fleet is trapped between pirates and Khemesh’s kraken-servants. The party must navigate the three-dimensional battlefield while appeasing Raéyn’s moods.

\begin{itemize}
\item Use \emph{Rite of the Changing Tide} to aid or hinder pursuit.  
\item Use \emph{Rite of the Wave-Road Blessing} to establish safe corridors.  
\item Invoke \emph{Rite of the Storm-Queen’s Hand} as a climactic storm.  
\item Curse a pirate captain with \emph{Rite of the Mother’s Wrath}.  
\end{itemize}

Resolution: The party must decide whether to appeal to Raéyn’s protection or broker peace between mother and son.

# % --- Patron: The Sacred Geometry (Order & Mathematical Truth) ---

\subsubsection{The Sacred Geometry (Order \& Mathematical Truth)}
\textit{Lore.} The Sacred Geometry is the mathematical expression of the universe's underlying structure --- the divine mathematics that governs all existence. It is the principle that reduces chaos to measure, that finds the golden ratio in petals and the perfect spiral in galaxies. Those who serve it understand that beneath apparent randomness lies immutable law, and that by mastering these patterns, one can bend reality to will.

Kon'reh is not merely a game but a sacred practice --- a way of seeing the world through the lens of perfect proportion. The board's pieces represent fundamental forces, its movements echo cosmic harmonics, and mastery of its patterns grants insight into the universe's hidden order.

The Sacred Geometry does not create chaos but reveals it as illusion. Its followers are architects of certainty in an uncertain world, mathematicians of the divine who understand that every problem has a solution if one can only find the correct equation.

\begin{quote}
Chalk, string, and a prayer to ratios. When the circle closes, luck remembers its place.
\end{quote}

\paragraph*{Rite of the Golden Mean (Low, 4 XP)} \emph{Scene; Self; No.}
\textbf{Materials:} A tool marked with the golden ratio ($\varphi \approx 1.618$).\\
\textbf{Effect:} Gain +1 die to rolls requiring precision, balance, or proportion. On success, re-roll one die showing 1 or 2.\\
\textbf{Invoke:} 1 action; mark +1 Obligation.\\
\textbf{Push It:} Upgrade effect one step on a single roll; suffer -1 die to social rolls involving spontaneity for the scene.\\
\emph{Requires: Familiar \ (\textit{Invoke:} 1 Boon).}

\paragraph*{Rite of the Perfect Angle (Low, 5 XP)} \emph{Scene; Touch; No.}
\textbf{Materials:} Compass and straightedge consecrated in ritual.\\
\textbf{Effect:} Treat difficult terrain, awkward positioning, or structural obstacles as one step easier this scene. Gain +1 die on spatial/architectural reasoning rolls.\\
\textbf{Invoke:} 1 action; mark +1 Obligation.\\
\textbf{Push It:} Extend benefit to one ally in Close range, but generate 1 SB (Clubs).\\
\emph{Requires: Familiar \ (\textit{Invoke:} 1 Boon).}

\paragraph{Rite of the Harmonic Resonance \textnormal{[WARD]} (Standard, 8 XP)} \emph{Scene; Zone; No.}
\textbf{Materials:} Geometric patterns drawn with precision.\\
\textbf{Effect:} Create a zone of harmony. Outsiders crossing must test DV 3. On Hit: cross normally. On Partial: suffer -1 die inside. On Miss: cannot cross this beat.\\
\textbf{Push It:} Fortify the pattern further but mark +1 Obligation.\\
\emph{Requires: Familiar + Codex \ (\textit{Invoke:} 1 Boon).}

\paragraph{Rite of the Calculated Trajectory \textnormal{[REVEAL]} (Standard, 7 XP)} \emph{Scene; Self; No.}
\textbf{Materials:} A perfect circle and a solved geometric problem.\\
\textbf{Effect:} Gain +2 dice to prediction, trajectory, or pattern recognition. Ask two questions about mathematical relationships in the current scene.\\
\textbf{Push It:} Predict one future event with certainty, but mark +1 Exposure.\\
\emph{Requires: Familiar + Codex \ (\textit{Invoke:} 1 Boon).}

\paragraph{Rite of the Fundamental Equation \textnormal{[WARD][BIND]} (High, 12 XP)} \emph{Scene; Zone; No.}
\textbf{Materials:} Complex diagram of universal constants.\\
\textbf{Effect:} Declare one physics/magic rule different in the zone (no scene-ending absolutes; GM may veto). Once per scene, downgrade a Miss to Success \& Cost.\\
\textbf{Push It:} Affect an adjacent zone for one beat; generate 2 SB.\\
\emph{Requires: Familiar + Codex + Tier III \ (\textit{Invoke:} \textbf{2 Boons}).}\\
\emph{Obligation:} 7 segments.

\paragraph{Rite of Kon'reh Mastery \textnormal{[OATH][FORTIFY]} (High, 13 XP)} \emph{Extended; Near; No.}
\textbf{Materials:} A consecrated Kon'reh board and pieces representing fundamental forces.\\
\textbf{Effect:} All participants make contested Wits + Lore. Winners gain +2 dice to strategy/pattern/logic rolls next session; losers suffer -1 die.\\
\textbf{Push It:} Winner imposes one mathematical "law" for the session, but generate 2 SB (Diamonds).\\
\emph{Requires: Familiar + Codex + Tier III \ (\textit{Invoke:} \textbf{2 Boons}).}\\
\emph{Obligation:} 7 segments.

\subsection*{The Sacred Geometry's Corruption Table}
\label{sec:sacred-geometry-corruption}

\begin{longtable}{>{\raggedright\arraybackslash}p{1cm} p{5cm} p{5cm}}
\toprule
\textbf{Tier} & \textbf{Benefit} & \textbf{Cost / Quirk} \\
\midrule
1 & Pattern Recognition: +1 die to Notice when observing geometric patterns, mathematical sequences, or logical structures. & Obsessive Calculation: Must count, measure, or calculate patterns noticed, even when tactically disadvantageous. \\
\midrule
2 & Mathematical Precision: Once per scene, re-roll any failed logic, pattern, or mathematical reasoning roll. & Social Blindness: Suffer -1 die to social rolls involving emotional nuance or interpersonal intuition. \\
\midrule
3 & Geometric Insight: Gain +2 dice to rolls involving spatial reasoning, architecture, or geometric problem-solving. & Compulsive Order: Must organize or correct imperfect arrangements; suffer 1 Fatigue when surrounded by chaos. \\
\midrule
4 & Universal Law: Once per session, declare a mathematical principle that applies to the current situation. Gain +2 dice to related rolls, but become fixated on its perfection. & Perfectionist Paralysis: Suffer -1 die to rolls requiring quick, imperfect solutions; must find the "correct" answer. \\
\midrule
5 & Divine Ratio: Once per session, see the perfect mathematical relationship underlying any phenomenon. Gain +3 dice to understanding it, but become obsessed with its implications. & Number Fever: Suffer -1 die to rolls not involving mathematical concepts; numbers dominate your thoughts. \\
\midrule
6+ & Absolute Equation: Once per session, solve any mathematical problem or predict any pattern with perfect accuracy. For one scene, reality conforms to your calculations, but mark +2 Obligation and risk mental breakdown from cosmic truths. & Infinite Calculation: Mark +3 Obligation when using this power; become trapped in endless mathematical loops, suffering Harm 1 (Stress) until you find the solution or are interrupted. \\
\bottomrule
\end{longtable}
# % --- Patron: The Sealed Gate (Boundaries & Protection) ---

\subsubsection{The Sealed Gate (Boundaries \& Protection)}
\textit{Lore.} The Sealed Gate is invoked when thresholds weaken, forbidden knowledge leaks, or Outsiders press against the walls of reality. It is the Abjurist's Patron, embodying the principle that protection sometimes requires imprisonment, and safety may demand exile. Followers are warders and exorcists, but also philosophers of separation: those who believe the act of closing is sacred.

The Gate manifests as an armored figure, face hidden by a helm that shifts symbols: binding runes for warders, expulsion marks for exorcists, and an impassable refusal for transgressors.

\begin{quote}
You write borders into the world and prosecute trespass. Doors remember their true keepers; lines mean what you say they mean.
\end{quote}

\paragraph*{Rite of the Sealed Threshold (Low, 4 XP)} \emph{Scene; Touch; No.}
\textbf{Materials:} Chalk, wax, chain, or sigil.\\
\textbf{Effect:} Mark a threshold. Crossing parties suffer worsened Position or stumble on first entry (fiction decides). Create a 4-segment \emph{Boundary Maintained} clock to automatically fail one crossing attempt.\\
\textbf{Invoke:} 1 action; mark +1 Obligation.\\
\textbf{Push It:} Treat the threshold as difficult terrain; +1 Obligation cost to cross.\\
\emph{Requires: Familiar \ (\textit{Invoke:} 1 Boon).}

\paragraph*{Rite of the Key's Rebuke (Low, 5 XP)} \emph{Instant; Near; No.}
\textbf{Materials:} Gesture or chain-clack.\\
\textbf{Effect:} Project a spectral hasp to stagger or disarm. On success, create a 2-segment \emph{Ward Active} token to auto-succeed on a similar defense later.\\
\textbf{Invoke:} 1 action; mark +1 Obligation.\\
\textbf{Push It:} Drop the object just beyond reach; +1 Obligation cost to retrieve.\\
\emph{Requires: Familiar \ (\textit{Invoke:} 1 Boon).}

\paragraph{Rite of the Circle of Denial \textnormal{[WARD]} (Standard, 8 XP)} \emph{Scene; Near; No.}
\textbf{Materials:} Salt, iron filings, or blessed chalk.\\
\textbf{Effect:} Outsiders crossing test DV = Cap. On Hit: cross but add DV to their Exit Tally; on Partial: +1; on Miss: fail this beat. Create a 6-segment \emph{Boundary Integrity} clock.\\
\textbf{Push It:} Fortify circle; clearer tells; +1 Obligation.\\
\emph{Requires: Familiar + Codex \ (\textit{Invoke:} 1 Boon).}

\paragraph{Rite of the Writ of Passage \textnormal{[BIND]} (Standard, 7 XP)} \emph{Scene; Near; No.}
\textbf{Materials:} Spoken naming and scribed pass-mark.\\
\textbf{Effect:} Designate a route as permitted. Allies gain improved flow (Position/Effect bump). Unauthorized crossers suffer -1 die to movement. Create an 8-segment \emph{Authorized Passage} clock.\\
\textbf{Push It:} Extend to extra ally/obstacle; +1 Obligation.\\
\emph{Requires: Familiar + Codex \ (\textit{Invoke:} 1 Boon).}

\paragraph{Rite of the Banishment Knot \textnormal{[BANISH][BIND]} (High, 13 XP)} \emph{Instant; Near; No.}
\textbf{Materials:} Knot sealed with gate-sigil.\\
\textbf{Effect:} Target an Outsider; test DV = Cap. On Hit: add DV to Exit Tally; on Partial: +1. If full, entity acts once then departs; cannot return for one session. Create a 4-segment \emph{Banishment Enforced} clock.\\
\textbf{Push It:} Strip one tether/anchor or forbid threshold-crossing in this location for one session; +2 Obligation.\\
\emph{Requires: Familiar + Codex + Tier III \ (\textit{Invoke:} \textbf{2 Boons}).}\\
\emph{Obligation:} 7 segments.

\paragraph{Rite of the Consecrated Barrier \textnormal{[WARD][UNWARD]} (High, 14 XP)} \emph{Extended; Zone; No.}
\textbf{Materials:} Relics from three faiths, iron bands, blood of a trespasser.\\
\textbf{Effect:} Consecrate area against unauthorized passage. Crossers test Spirit+Resolve (DV 4) or suffer Harm~1. Only proper authorization bypasses. Create a 10-segment \emph{Sacred Boundary} clock.\\
\textbf{Push It:} Make barrier permanent/fixed; start \emph{Boundary Maintenance} [6].\\
\emph{Requires: Familiar + Codex + Tier III \ (\textit{Invoke:} \textbf{2 Boons}).}\\
\emph{Obligation:} 8 segments.

\subsection*{The Sealed Gate's Corruption Table}
\label{sec:sealed-gate-corruption}

\begin{longtable}{>{\raggedright\arraybackslash}p{1cm} p{5cm} p{5cm}}
\toprule
\textbf{Tier} & \textbf{Benefit} & \textbf{Cost / Quirk} \\
\midrule
1 & Boundary Sense: +1 die to Notice when detecting weak points, thresholds, or unauthorized entry. & Paranoid Vigilance: Must check and re-check barriers and seals; suffer -1 die to rolls requiring trust or openness. \\
\midrule
2 & Sealed Strength: Once per scene, treat a failed warding or protection roll as a success, but mark 1 SB (Spades). & Isolation Tendency: Suffer 1 Fatigue when in open, unsecured spaces or among strangers. \\
\midrule
3 & Ward Keeper: Gain +2 dice to rolls involving magical wards, barriers, or protective enchantments. & Compulsive Sealing: Must seal or secure any opening or vulnerability noticed, even when inappropriate. \\
\midrule
4 & Absolute Barrier: Once per session, create an impenetrable barrier that lasts for one scene. & Prison Mindset: Suffer -1 die to rolls involving freedom, escape, or breaking restrictions. \\
\midrule
5 & Gate Master: Once per session, banish or seal away any supernatural threat with a successful test. & Boundary Obsession: Suffer -1 die to rolls not involving protection, sealing, or enforcement of limits. \\
\midrule
6+ & Keeper of All Thresholds: Once per session, become the living embodiment of sealed boundaries. For one scene, all barriers within Near range become absolute, but mark +2 Obligation and risk sealing allies inside. & Ultimate Confinement: Mark +3 Obligation when using this power; risk permanent Harm (Stress) from the psychic weight of containing everything that threatens. \\
\bottomrule
\end{longtable}
../../srd/patrons/traveler.tex
../../srd/patrons/varnek-karn.tex
# % --- Patron: The Witness (Truth & Revelation) ---

\subsubsection{The Witness (Truth \& Revelation)}
\textit{Lore.} The Witness remembers what others bury. Every shadow cast and oath broken is a line in her unending ledger. She is the keeper of inconvenient truths, the patron of those who seek to expose lies or recover forgotten knowledge. Her followers learn that knowledge comes with a price—the weight of remembering what others would forget.

\begin{quote}
I will show you what you would rather forget. But first, you must forget what you think you know.
\end{quote}

\paragraph*{Rite of the Lingering Glimpse (Low, 4 XP)} \emph{Instant; Near; Yes (Investigation/Notice only).}
\textbf{Materials:} A trace of the thing to be remembered (hair, dust, a spoken name).\\
\textbf{Effect:} Gain +1 die to your roll to investigate or notice something directly related to the trace within the current scene.\\
\textbf{Invoke:} 1 action; mark +1 Obligation.\\
\textbf{Push It:} Gain +2 dice instead, but mark 1 segment on a \textbf{Memory Strain Clock [4]}. If the clock fills, you gain Fatigue 1 and suffer -1 die on Investigation/Notice rolls until the end of the next scene due to mental exhaustion from forced recall.\\
\emph{Requires: Familiar \ (\textit{Invoke:} 1 Boon).}

\paragraph*{Rite of Piercing Scrutiny (Low, 5 XP)} \emph{Scene; Zone; No.}
\textbf{Materials:} A circle drawn with chalk or string while focusing on the truth to be sought.\\
\textbf{Effect:} Within the zone, gain +1 die to rolls to detect deception (Insight vs. Deceit, spotting social tells) or to recall hidden knowledge (Lore/Investigate for memory). Social interactions within the zone begin one Position step worse for those attempting to deceive.\\
\textbf{Invoke:} 1 action; mark +1 Obligation.\\
\textbf{Push It:} One target within the zone must make a Wits test (DV 3) or involuntarily reveal one pertinent lie or hidden fact they are currently concealing (Keeper determines relevance). Regardless of the test result, mark Exposure +1 for the target(s) in the zone.\\
\emph{Requires: Familiar \ (\textit{Invoke:} 1 Boon).}

\paragraph{Rite of the Echoing Truth \textnormal{[OMEN]} (Standard, 8 XP)} \emph{Instant; Near; No.}
\textbf{Materials:} A reflective surface (mirror, still water, polished metal) used to focus on the target.\\
\textbf{Effect:} Target must make a Resolve test (DV 3) or suffer -1 die to rolls involving memory, deception, or resisting interrogation for the scene. If they fail, you may ask one specific, factual question about something they know, and they must answer truthfully or suffer 1 SB (Hearts) as the memory is forcibly drawn forth.\\
\textbf{Push It:} If the target fails their Resolve test, you may ask a second question, but the mental intrusion causes them Harm 1 (Stress/Mental).\\
\emph{Requires: Familiar + Codex \ (\textit{Invoke:} 1 Boon).}

\paragraph{Rite of the Immutable Record \textnormal{[OATH]} (Standard, 7 XP)} \emph{Scene; Near; No.}
\textbf{Materials:} A document signed by all parties within the zone, or a spoken pact witnessed by the caster.\\
\textbf{Effect:} Bind the agreement. Any party who knowingly breaches it suffers 1 SB (Hearts) immediately and gains a persistent \textbf{Oathbreaker's Mark} Condition (-1 die on social rolls involving honor, trust, or oaths until amends are made or a significant act redeems them).\\
\textbf{Push It:} The bond becomes magically enforced for one specific, crucial clause: violation automatically inflicts Harm 1 (Stress) on the breaker in addition to the SB and Mark.\\
\emph{Requires: Familiar + Codex \ (\textit{Invoke:} 1 Boon).}

\paragraph{Rite of the Unveiled Heart \textnormal{[OMEN]} (High, 12 XP)} \emph{Scene; Near; No.}
\textbf{Materials:} A private setting where the target feels safe or is speaking freely.\\
\textbf{Effect:} The target suffers -2 dice to all attempts to conceal true emotions, intentions, or lies for the scene. Any successful social roll (Sway, Command, Deceit) made by the target generates 1 SB (Hearts) as the effort to maintain falsehoods under the Witness's gaze creates internal discord.\\
\textbf{Push It:} You may designate one specific, complex question about the target's motivations, fears, or hidden loyalties. If you successfully use Sway or Insight against them this scene, you automatically learn the answer to that question. The intense scrutiny marks 1 SB (Spades) for you as the Witness's attention lingers.\\
\emph{Requires: Familiar + Codex + Tier III \ (\textit{Invoke:} \textbf{2 Boons}).}\\
\emph{Obligation:} 6 segments.

\paragraph{Rite of the Final Reckoning \textnormal{[OMEN]} (High, 13 XP)} \emph{Scene; Zone; No.}
\textbf{Materials:} A formally called gathering (court, council, family meeting) within the consecrated zone.\\
\textbf{Effect:} All present must speak their greatest debt, wrongdoing, or hidden truth related to the gathering's purpose. Those who lie or withhold suffer Harm 2 (Stress/Reputation). Truth-tellers gain +2 dice to social actions for the remainder of the scene within the zone.\\
\textbf{Push It:} The truth becomes inescapable - even indirect lies or evasions related to the core topic suffer the Harm 2 penalty. The absolute nature of the revelation creates 2 SB (Diamonds) as the disruption to fates and secrets resonates.\\
\emph{Requires: Familiar + Codex + Tier III \ (\textit{Invoke:} \textbf{2 Boons}).}\\
\emph{Obligation:} 7 segments.

\subsection*{The Witness's Corruption Table}
\label{sec:witness-corruption}

\begin{longtable}{>{\raggedright\arraybackslash}p{1cm} p{5cm} p{5cm}}
\toprule
\textbf{Tier} & \textbf{Benefit} & \textbf{Cost / Quirk} \\
\midrule
1 & Truth's Sight: +1 die to Insight when detecting deception or hidden motives. & Burden of Knowledge: Suffer -1 die to social rolls involving lies or deception; others become uncomfortable with your piercing gaze. \\
\midrule
2 & Memory's Keeper: Once per scene, recall one specific detail from a previous scene with perfect clarity. & Compulsive Honesty: Must correct obvious falsehoods witnessed, even when tactically disadvantageous. \\
\midrule
3 & Revelation's Power: Gain +2 dice to rolls involving exposing secrets, uncovering lies, or forcing confessions. & Truth-Blind: Suffer 1 Fatigue when exposed to comforting lies or willful ignorance. \\
\midrule
4 & Witness's Authority: Once per session, force one target to make a Resolve test (DV 4) or reveal a significant hidden truth. & Isolation: Suffer -1 die to rolls requiring trust or close relationships; others fear your ability to uncover their secrets. \\
\midrule
5 & Omniscient Gaze: Once per session, see through all deceptions and lies for one exchange, gaining +3 dice to related actions. & Paranoia: Suffer -1 die to rolls involving personal peace or rest; the weight of all truths witnessed creates constant mental strain. \\
\midrule
6+ & Absolute Witness: Once per session, become the living embodiment of truth. For one scene, all deceptions within Near range automatically fail, but mark +2 Obligation and risk permanent Harm (Stress) from the crushing weight of absolute knowledge. & Truth's Prison: Mark +3 Obligation when using this power; become unable to tolerate any form of deception, making normal social interaction nearly impossible. \\
\bottomrule
\end{longtable}


\section{Patron Rivalries}
\label{sec:patron-rivalries}

Rivalries set expectations for tone and friction. Use them to color rulings, nudge Position, and guide how Story Beats (SB) land. In their home domains, a Patron’s work tends to start a step better in Position; in a rival’s, a step worse (Keeper’s call).

\begin{table}[h!]
  \centering
  \renewcommand{\arraystretch}{1.15}
  \begin{tabular}{@{}p{3.4cm}p{3.4cm}p{8.2cm}@{}}
    \toprule
    \textbf{Patron} & \textbf{Primary Rival} & \textbf{Retribution in Play (one-line lore)} \\
    \midrule
    Raéyn (Sea, Tides, Travel) & Khemesh (Abyssal Maw) & Those who spurn the sea are swallowed by storms and riptides. \\
    Khemesh (Abyssal Maw) & Raéyn (Sea, Tides, Travel) & Depth devours chart and voice alike; only silence remains. \\
    Sealed Gate (Boundaries, Closure) & The Traveler (Ways, Roads) & Trespassers find every path locked; even home’s door bars their way. \\
    The Traveler (Ways, Roads) & Sealed Gate (Boundaries, Closure) & Those who deny the road are stranded at the threshold forever. \\
    The Witness (Truth, Revelation) & Mab (Glamour, Courts) & Liars discover their tongues turned to ash beneath the unblinking eye. \\
    Mab (Glamour, Courts) & The Witness (Truth, Revelation) & Those who strip glamour are forever exiled from merriment and favor. \\
    Ikasha (Shadow, Latent Potential) & The Witness (Truth, Revelation) & Those who disrespect the hush find every shadow whispering their name. \\
    Mykkiel (Judgment, Writ) & Varnek Karn (Necromantic Archives) & Those who defy judgment are hounded by warrants even in death. \\
    Varnek Karn (Necromantic Archives) & Oath of Light \& Flame (Dawn, Vows) & Those who desecrate memory are bound in chains of bone. \\
    Oath of Light \& Flame (Dawn, Vows) & Khemesh (Abyssal Maw) & Oathbreakers burn at dawn; no tide quenches their fire. \\
    Sacred Geometry (Order, Pattern) & The Traveler (Ways, Fortune) & Those who spurn order are lost forever in mazes without end. \\
    Clockwork Monad (Iteration, Process) & Old Man of the Black Forest (Primal Humanity, Instinct) & Those who break the cycle are crushed beneath their own gears. \\
    Nidhoggr (Dreaming Antiquity) & Sacred Geometry (Order, Pattern) & Those who measure the ancient are buried beneath its weight. \\
    Carrion King (Carrion, Renewal) & Inaea (Mercy, Hearth) & Those who waste life are repaid in rot and swarming hunger. \\
    Gallows Bell (Doom, Last Rites) & Oath of Light \& Flame (Dawn, Vows) & Those who mock the last toll find their own names rung in iron. \\
    Old Man of the Black Forest (Primal Humanity, Instinct) & Mab (Glamour, Courts) & Those who spurn the old ways are hunted in the woods like beasts. \\
    Isoka (Serpents, Shedding) & Sacred Geometry (Order, Pattern) & Those who deny change are crushed in the serpent’s coil. \\
    Inaea (Mercy, Hearth) & Carrion King (Carrion, Renewal) & Those who betray hospitality are cast out to starve in the night. \\
    Maelstraeus (Infernal Bargainer) & The Witness (Truth, Revelation) & Those who renege on a bargain are claimed by fire and clause. \\
    Livaea (Temptation, Desire) & Inaea (Mercy, Hearth) & Those who corrupt sanctuary with lust are haunted by love turned poison. \\
    \bottomrule
  \end{tabular}
  \caption{Primary Patron Rivalries and the retribution that follows when their domains are denied.}
\end{table}

% Requires: \usepackage{float} for [H] tables
% =========================

\section{Rites, Invokers, and Symbols}
\label{sec:rites}

Magic in \textbf{Fate's Edge} expresses through three intertwined practices: \textbf{Rites} (oathbound authority), \textbf{Invocations} (symbolic ritual), and \textbf{Patron Pacts} (gifts and obligations). The rules below emphasize fiction-first play: consequences are Story Beats (SB) that prompt twists; numbers follow the story.

\subsection{Rites and Patrons (Runekeepers)}
\label{subsec:runekeepers}
Characters who bind themselves to a \emph{single} Patron and study that Patron's \textbf{Codex} are \textbf{Runekeepers}. Their magic is structured, immediate, and tied to service.

\begin{itemize}
  \item \textbf{One-Patron Rule.} A Runekeeper may be bound to \emph{only one} Patron at a time. This sharpens identity and keeps Obligation on a single ledger.
  \item \textbf{Thiasos (Familiar).} A circle, retinue, or emissary that grounds the pact in fiction. Required to access \emph{Patron's Gift}.
  \item \textbf{Codex.} The Patron's corpus of rites and precedents. Grants access to the Patron's Rites.
  \item \textbf{Invoke Rites.} A Runekeeper may Invoke a known Rite from their Patron as a \textbf{1 action} effect. On completion, mark \textbf{+1 Obligation} to that Patron. You may \emph{Push It} once per scene to amplify the effect, marking \textbf{+1 additional Obligation}.
\end{itemize}

\subsection{Invokers and Symbols}
\label{subsec:invokers}
Invokers relate to Patrons through consecrated \textbf{Symbols}: physical tokens that anchor names and permissions.

\begin{itemize}
  \item \textbf{Symbols (Minor Asset).} Each Symbol is keyed to one Patron; cost \textbf{4 XP}. You may own Symbols of different Patrons (one Symbol per Patron).
  \item \textbf{Ritual Invocation.} Display the Symbol and perform the Rite as a \emph{ritual} (Significant Time). Completion always marks \textbf{+1 Obligation} on that Rite's ledger.
  \item \textbf{Crack the Seal.} As part of an Invoker Rite, you may resolve the effect instantly by setting the Symbol to \emph{Compromised} and marking \textbf{+2 Obligation} (\textbf{+3} if High-Power). The Keeper may spend 1 on-theme SB immediately. The asset remains but is inert until restored.
  \item \textbf{Restore a Symbol.} 1 downtime action and a fitting test (DV 3 or by fiction). Success: \emph{Maintained}; shaky: returns \emph{Neglected}. Or spend \textbf{1 XP} to fully restore.
  \item \textbf{Display Requirement.} Symbols must be openly displayed for rituals. Hidden Symbols do not function.
\end{itemize}

\subsection{Casting and Free-Form Magic}
\label{subsec:casting}
Improvised casting is possible with the \textbf{Caster's Gift} Talent (\textbf{2 XP}). It is a \emph{backup toolkit}:
\begin{itemize}
  \item Small, local effects (typ. DV 2--3), fiction-first, colored by Elements and locus.
  \item Heavy control effects such as \texttt{[WARD]}, \texttt{[BANISH]}, or \texttt{[UNWARD]} require a printed Talent, Rite, or Spell result.
\end{itemize}

\subsection{Patron's Gift (Imbuements)}
\label{subsec:patrons-gift}
The pact may mark a devotee's tools with a short-lived boon aligned to the Patron's domain.

\paragraph{Requirements.}
\textbf{Thiasos (Familiar)} is required. Invoking the Gift costs \textbf{1 Boon}. A Codex is \emph{not} required for the Gift.

\paragraph{Activation and Duration.}
\begin{itemize}
  \item \textbf{Action:} 1 action to activate; \textbf{1/scene}.
  \item \textbf{Duration:} Scene. \emph{Push It:} extend for one additional scene by marking \textbf{+1 Obligation} to that Patron (max one Push per scene).
  \item \textbf{Range:} Touch (you must handle the item).
  \item \textbf{Stacking:} Gifts from the \emph{same Patron} do not stack; take the best active version. Dice bonuses respect the table's \textbf{+3 dice cap}.
\end{itemize}

\paragraph{Effect.}
Choose one held item you or an ally carries. Until scene end it grants:
\begin{itemize}
  \item \textbf{+1 Melee} (the item counts as a magical weapon), and
  \item \textbf{+1 Thematic} (a \emph{+1 die} to a fixed Skill tied to your Patron; see Table~\ref{tab:gift-thematic-map}). Apply only when the fiction clearly fits the Patron's sphere and how the item is used.
\end{itemize}

\paragraph{Runekeeper Clarification.}
A Runekeeper (one Patron + Codex) may Invoke Rites on-screen and use Patron's Gift if they also possess \textbf{Thiasos (Familiar)}. Codex alone does not grant the Gift. Symbols are optional for parley or omens and do not gate Runekeeper Invocation or the Gift.

\section*{Borrowed Grace}
\label{talent:borrowed-grace}
\index{Talents!Invoker}\index{Imbuement!Lesser}

\textbf{Type:} Invoker Talent — \textit{Lesser Imbuement}

\subsection*{Use}
\begin{itemize}
  \item \textbf{Cost:} 1 Boon, 1 action.
  \item \textbf{Effect (pick one on use):} \textbf{+1 Melee} \emph{or} \textbf{+1 Thematic} (your table’s thematic Skill).
  \item \textbf{Duration:} \textit{Single action/attack} (instantaneous boost).
  \item \textbf{Requirement:} Wield/display the Patron’s \textbf{Symbol}.
  \item \textbf{Obligation:} +1 \textbf{Obligation} to that Patron immediately (see \S\ref{sec:obligation}).
  \item \textbf{Limits:} Cannot be extended, stacked, or \emph{Pushed} for duration.
\end{itemize}

\subsection*{Fictional Framing}
A quick, rule-bending channel through a Patron’s \emph{Symbol}—a sliver of grace, borrowed for a moment and paid for in debt.

\subsection*{Table Guidance (1-liners)}
\begin{itemize}
  \item \textbf{Combat:} Spike a strike vs. a tough foe; or steady a parry in a desperate bind.
  \item \textbf{Skill:} Nudge a pivotal social/ritual/track roll tied to the Patron’s sphere.
  \item \textbf{Fallout:} Repeated use accrues \textbf{Obligation}; NPC faithful may notice “stolen” grace.
\end{itemize}

\subsection*{Balance Notes}
\begin{itemize}
  \item Weaker than full Imbuement: \emph{one} action, no sustain, upfront Obligation.
  \item \textbf{Symbol dependency:} No Symbol, no channel (concealed or lost Symbol = no effect).
\end{itemize}

\subsection*{GM Hooks (quick picks)}
\begin{itemize}
  \item \textbf{Compel Debt:} A Patron agent arrives when Obligation crosses a tick.
  \item \textbf{Clash of Signs:} Using rival Symbols back-to-back risks minor \textbf{Backlash} (drop Position or +1 SB).
  \item \textbf{Spotlight Tell:} Brief visual tell (scent, sigil flare) marks the borrowing to observant NPCs.
\end{itemize}

\begin{table}[H]
\centering
\renewcommand{\arraystretch}{1.15}
\begin{tabular}{@{}p{3.8cm}p{3.8cm}p{7.5cm}@{}}
\toprule
\textbf{Patron} & \textbf{+1 Thematic Skill} & \textbf{Example Symbols} \\
\midrule
Ikasha (Shadow, Penumbra) & Stealth & Knot of black silk; soot-oil vial; fingerbone ring lacquered matte \\
Mykkiel (Judgment, Writ) & Command & Cold-iron seal matrix; parchment writ-tag; square rule stamped with code \\
The Witness (Truth, Revelation) & Notice & Obsidian eye pendant; silver mirror shard; wax seal-stamp with an open eye \\
Sealed Gate (Boundaries, Closure) & Tinker & Lead sounder-weight; iron chain link; sealed lockplate token \\
Raéyn (Storm, Tides) & Skirmish & Sea-glass disk; salt-crusted rope knot; vial of rainwater from three crossings \\
Khemesh (Abyss, Pressure) & Skirmish & Barnacle-bitten coin; abyssal-spiral lead weight; salt-etched iron chain \\
Mab (Glamour, Courts) & Persuade & Hawthorn thorn wrapped in silver; mirror shard with green felt; silk-lined acorn cup \\
Sacred Geometry (Perfect Forms) & Tinker & Brass heptagram compass; bone tablet with golden-ratio spiral; plumb-bob with proof in red thread \\
Clockwork Monad (Mechanism, Process) & Tinker & Gear tooth sealed in oil; mainspring coil; rivet stamped with forbidden numerals \\
Varnek Karn (Ossuary, Dominion of the Dead) & Command & Ossuary bead rosary; carved phalanx tally; fused bone-and-obsidian coin \\
Nidhoggr (Deep Earth, Rot) & Skirmish & Fossil tooth shard; dark river-stone; obsidian spindle with flaw \\
The Traveler (Ways, Roads) & Notice & Road-nail wrapped in thread; waystone pebble; brass compass missing its needle \\
Oath of Flame \& Light (Dawn, Vows) & Command & Cold-iron sun-stamp; vow-ring with sunrise and true name; ampoule of consecrated spark \\
\bottomrule
\end{tabular}
\caption{Patron's Gift: fixed Thematic Skill and example Symbols. Thematic bonuses apply only when the fiction matches the Patron’s domain and the item’s use. Symbols also serve Invokers as ritual anchors.}
\label{tab:gift-thematic-map}
\end{table}

\subsection{Specialization vs.\ Mixing}
\label{subsec:mixing}
Characters can mix paths (Summoner, Caster, Invoker, Runekeeper), but specialization is usually stronger and cleaner. Mixing increases upkeep (Obligation, Symbol state, Leash) and action congestion without guaranteed power gains. Let fiction guide choices: Story Beats are prompts to advance the scene, not punishments.

% Patron subsections (split files — keep filenames in the same directory)

% --- Patron: The Witness, Who Sees All (Memory & Omen) ---
\subsection{The Witness, Who Sees All (Memory \& Omen)}
\textit{Lore.} The Witness remembers what others bury. Every shadow cast and oath broken is a line in her unending ledger.

\begin{quote}
“I will show you what you would rather forget.”
\end{quote}

\paragraph{Mark of Remembrance (Low, 4 XP)} \emph{Action; Near; Yes (creature/object).}
\textbf{Materials:} A drop of ink or blood traced in a circle.\\
\textbf{Effect:} Ephemeral mark for one day. You unerringly recall its location/condition; \textbf{+1 die} to track or investigate it.\\
\textbf{Push It:} The mark whispers its last hour to you; mark \textbf{1 SB (Spades)} as grief/echoes cling.\\
\emph{Requires: Familiar \ (\textit{Invoke:} 1 Boon).}

\paragraph{Rite of Testimony (Low, 5 XP)} \emph{Scene; Near; Stacking: No.}
\textbf{Materials:} A knotted cord held while the oath is spoken.\\
\textbf{Effect:} Within the space, lies falter into hesitation or contradiction; Keeper signals tells.\\
\textbf{Push It:} Record an image/phrase in your memory; once this scene, replay for others. Costs \textbf{1 SB (Clubs)}.\\
\emph{Requires: Familiar \ (\textit{Invoke:} 1 Boon).}

\paragraph{Omen of Recall (Standard, 8 XP)} \emph{Action; Near; No.}
\textbf{Materials:} A mirror shard or still water.\\
\textbf{Effect:} Target vividly relives a recent event; suffers \(-1\) die to contested actions for the duration.\\
\textbf{Push It:} You glean a hidden motive/sensory detail; mark \textbf{1 SB (Hearts)}.\\
\emph{Requires: Familiar + Codex \ (\textit{Invoke:} 1 Boon).}

\paragraph{The Written Ledger (Standard, 7 XP)} \emph{Scene; Near; Stacking: Yes.}
\textbf{Materials:} A book or ledger marked with charcoal.\\
\textbf{Effect:} Agreements recorded cannot be forgotten by signers; denying/obfuscating suffers \(-1\) die.\\
\textbf{Push It:} Record the emotional truth; once, ask what a signatory \emph{truly} felt when signing.\\
\emph{Requires: Familiar + Codex \ (\textit{Invoke:} 1 Boon).}

\paragraph{Burden of Memory \textnormal{[OMEN]} (High, 11 XP)} \emph{Scene; Near; No.}
\textbf{Materials:} A blindfold or veil, worn until end of scene.\\
\textbf{Effect:} Confront one target with visions of broken oaths. They suffer \(-2\) dice to defiant acts this scene.\\
\textbf{Push It:} Name a second target; both dilute (\(-1\) die). Immediately mark \textbf{2 SB (Spades)}.\\
\emph{Requires: Familiar + Codex + Tier III \ (\textit{Invoke:} \textbf{2 Boons}).}\\
\emph{Obligation:} 6 segments.

% --- Patron: Ikasha, She Who Sleeps (Latent Potential & Shadow) ---
\subsection{Ikasha, She Who Sleeps (Latent Potential \& Shadow)}
\textit{Lore.} Ikasha is the hush between footfalls, the patience of dark water, the black-feathered watcher at every threshold. In stillness she gathers what might be, in crossroads she whispers of what may yet come. Ravens circle her, bearing secrets between worlds.

\begin{quote}
Blow out the candle. If the room listens back, ask softly. At the next crossroads, the raven waits.
\end{quote}

\paragraph{Touch the Umbral Veil (Low, 4 XP)} \emph{Action; Self; Yes (Stealth).}
\textbf{Materials:} A piece of black cloth.\\
\textbf{Effect:} Start \emph{Controlled} on one Stealth roll or gain +1 effect to hide/move quietly.\\
\textbf{Push It:} Brief shadow-muffling (ignore one noisy tell), but leave a shadow-double that may echo you later at an ill moment.\\
\emph{Requires: Familiar \ (\textit{Invoke:} 1 Boon).}

\paragraph{Rite of the Crossroads Raven (Low, 5 XP)} \emph{Scene; Zone; No.}
\textbf{Materials:} Scatter three black feathers or carve a crossroads sign.\\
\textbf{Effect:} Summon an omen-raven; grant \textbf{+1 die} to a navigation, pursuit, or diversion action \emph{or} force an enemy to hesitate at a fateful moment.\\
\textbf{Push It:} The raven speaks one cryptic truth, but demands a secret in return.\\
\emph{Requires: Familiar \ (\textit{Invoke:} 1 Boon).}

\paragraph{Draw from the Umbral Reservoir (Standard, 8 XP)} \emph{Action; Self/Ally; No.}
\textbf{Materials:} A vial of moonless-night water.\\
\textbf{Effect:} \textbf{+2 dice} to stealth, deception, or resolve \emph{or} clear \emph{Fatigue 1}.\\
\textbf{Push It:} Also gain one free escape attempt; next scene, you must help another cross a threshold or flee danger.\\
\emph{Requires: Familiar + Codex \ (\textit{Invoke:} 1 Boon).}

\paragraph{Secret Keeper’s Burden (Standard, 9 XP)} \emph{Instant; Touch; No.}
\textbf{Materials:} A lock of hair or intimate token.\\
\textbf{Effect:} Compel a truthful answer to one direct question (deep secrets may allow a Resolve test to resist).\\
\textbf{Push It:} Learn the answer \emph{and} a key hidden emotion; target learns one of your secrets in return, carried by a raven to them in dreams.\\
\emph{Requires: Familiar + Codex \ (\textit{Invoke:} 1 Boon).}

\paragraph{Become the Shadow at the Crossroads (High, 12 XP)} \emph{Scene; Self; No.}
\textbf{Materials:} Stand in absolute darkness or at a deserted crossroads.\\
\textbf{Effect:} Intangible to mundane harm; pass through thresholds and small gaps; \textbf{+2 dice} to Stealth; auto-succeed one escape. Cannot manipulate normal objects.\\
\textbf{Push It:} Interact once with a bound or thresholded object (a door, a lock, a sealed letter), but you become partially corporeal and vulnerable for one beat. Ravens may mark you.\\
\emph{Requires: Familiar + Codex + Tier III \ (\textit{Invoke:} \textbf{2 Boons}).}\\
\emph{Obligation:} 7 segments.

% --- Patron: The Sacred Geometry (Order & Pattern) ---
\subsection{The Sacred Geometry (Order \& Pattern)}
\textit{Lore.} Beneath mess lies measure. The Geometry carves clean lines through chaos, demanding symmetry from a crooked world.

\begin{quote}
Chalk, string, and a prayer to ratios. When the circle closes, luck remembers its place.
\end{quote}

\paragraph{Find the Pattern (Low, 5 XP)} \emph{Action; Self; Yes (investigation).}
\textbf{Materials:} Compass and straightedge.\\
\textbf{Effect:} \textbf{+1 die} to decode patterns/codes/systems; re-roll one \texttt{1} on math/logic rolls.\\
\textbf{Push It:} Upgrade effect one step on a single roll; you become obsessively pattern-seeking (scene): \(-1\) die to social rolls.\\
\emph{Requires: Familiar \ (\textit{Invoke:} 1 Boon).}

\paragraph{Rite of the Ordered Step (Low, 4 XP)} \emph{Scene; Self; No.}
\textbf{Materials:} Walk a perfect square.\\
\textbf{Effect:} Ignore difficult terrain penalties for walking; +1 die to actions requiring perfect calibration/balance.\\
\textbf{Push It:} Cross a fragile surface silently once, but must follow a geometrically perfect path for the scene.\\
\emph{Requires: Familiar \ (\textit{Invoke:} 1 Boon).}

\paragraph{Thread the Loom of Chance (Standard, 7 XP)} \emph{Action; Self; No.}
\textbf{Materials:} Weighted dice or a balanced scale.\\
\textbf{Effect:} Re-roll up to \textbf{two dice} in your current pool.\\
\textbf{Push It:} Treat one zone tag as favorable for this action; accept an equal/opposite consequence later this scene (\textbf{1 SB}, Keeper suits).\\
\emph{Requires: Familiar + Codex \ (\textit{Invoke:} 1 Boon).}

\paragraph{Rite of the Golden Ratio (Standard, 7 XP)} \emph{Scene; Touch; No.}
\textbf{Materials:} A string cut to the golden ratio.\\
\textbf{Effect:} Optimize one object $\le$ door-size. Choose: door resists breach (+1 effect to resist), weapon strikes truer (+1 die next attack), tool grants +1 effect on next use.\\
\textbf{Push It:} Affect a second connected object at half strength.\\
\emph{Requires: Familiar + Codex \ (\textit{Invoke:} 1 Boon).}

\paragraph{Rewrite the Fundamental Equation (High, 12 XP)} \emph{Scene; Zone; No.}
\textbf{Materials:} Complex diagram at zone center.\\
\textbf{Effect:} Declare one physics/magic rule different in-zone (no instant kills; Keeper may veto scene-enders). Once/scene, downgrade a \emph{Miss} to \emph{Success \& Cost}.\\
\textbf{Push It:} Affect an adjacent zone for one beat; create paradox: \textbf{2 SB}.\\
\emph{Requires: Familiar + Codex + Tier III \ (\textit{Invoke:} \textbf{2 Boons}).}\\
\emph{Obligation:} 7 segments.

% --- Patron: Inaea, Angel of the Spider (Webs & Fate) ---
\subsection{Inaea, Angel of the Spider (Webs \& Fate)}
\textit{Lore.} Where Isoka sheds, Inaea binds—threads of debt, favor, and inevitability.

\begin{quote}
Tie one knot for what you owe, two for what you’re owed, and a third for what will answer both.
\end{quote}

\paragraph{Tie a Simple Knot (Low, 4 XP)} \emph{Action; Near; Yes (link once).}
\textbf{Materials:} A single thread.\\
\textbf{Effect:} Declare two minor events linked; either \textbf{force 1 SB} (GM suit) on a foe when the first triggers \emph{or} bank \textbf{+1 die} for a follow-on roll this scene.\\
\textbf{Push It:} The held +1 ignores one minor disruption; the web may also tug an unintended party once.\\
\emph{Requires: Familiar \ (\textit{Invoke:} 1 Boon).}

\paragraph{Rite of the Tangled Thread (Low, 5 XP)} \emph{Scene; Near; No.}
\textbf{Materials:} Tug a web or net.\\
\textbf{Effect:} Invisible snare in a lane/door. First to cross suffers \(-1\) die on next action.\\
\textbf{Push It:} Brief bind (one beat) enabling an ally setup; affects all who cross.\\
\emph{Requires: Familiar \ (\textit{Invoke:} 1 Boon).}

\paragraph{Weave the Strand of Inevitability (Standard, 8 XP)} \emph{Scene; Near; No.}
\textbf{Materials:} Three colored threads woven.\\
\textbf{Effect:} Link two actors/actions: when A moves, B is exposed. Choose: \textbf{force 1 SB on B} next action \emph{or} \textbf{+2 dice} to one prediction/setup keyed to the link.\\
\textbf{Push It:} Invert once (B cues A). Breaking the link’s fiction creates \textbf{1 SB (Hearts/Clubs)}.\\
\emph{Requires: Familiar + Codex \ (\textit{Invoke:} 1 Boon).}

\paragraph{Rite of the Weaver's Glance (Standard, 7 XP)} \emph{Scene; Self; No.}
\textbf{Materials:} Watch a spider finish one radial line.\\
\textbf{Effect:} Ask one precise question about in-scene ties; then gain \textbf{+1 effect} on one leverage/pressure action exploiting it.\\
\textbf{Push It:} Surface a hidden tie (Keeper reveals a quiet obligation/fear); mark \emph{Exposure +1}.\\
\emph{Requires: Familiar + Codex \ (\textit{Invoke:} 1 Boon).}

\paragraph{Bind the Bargain \textnormal{[OATH]} (High, 11 XP)} \emph{Scene; Near; No.}
\textbf{Materials:} Silk loop tied around two thumbs, then cut/knotted.\\
\textbf{Effect:} Bind up to two consenting parties to a clear term. Breach \emph{forces 2 SB} on the breaker and leaves a subtle tell until amends.\\
\textbf{Push It:} Widen to a small circle (up to four); each party names a narrow loophole (Keeper approves). Exploiting it generates \textbf{1 SB (Diamonds)}.\\
\emph{Requires: Familiar + Codex + Tier III \ (\textit{Invoke:} \textbf{2 Boons}).}\\
\emph{Obligation:} 7 segments.

% --- Patron: Raéyn, Keeper of the Sealed Gate (Thresholds & Warding) ---
\subsection{Raéyn, Keeper of the Sealed Gate (Thresholds \& Warding)}
\textit{Lore.} Raéyn is invoked at every border. His gift is the lock that preserves and the seal that keeps chaos at bay.

\begin{quote}
The door is not shut until Raéyn’s mark is traced.
\end{quote}

\paragraph{Seal the Latch (Low, 4 XP)} \emph{Action; Near; Yes (object).}
\textbf{Materials:} Trace a key-sign with ash/chalk.\\
\textbf{Effect:} Secure a container/door/gate. Attempts to open require intruder to generate \textbf{+1 SB} (Spades/Clubs).\\
\textbf{Push It:} First touch flares (noise/flash), revealing the attempt.\\
\emph{Requires: Familiar \ (\textit{Invoke:} 1 Boon).}

\paragraph{Rite of the Quiet Gate (Low, 5 XP)} \emph{Scene; Near; No.}
\textbf{Materials:} A key turned backwards in a lock.\\
\textbf{Effect:} Warded threshold. Passing uninvited imposes \(-1\) die on the trespasser’s next action.\\
\textbf{Push It:} The ward whispers intruder’s purpose in one phrase.\\
\emph{Requires: Familiar \ (\textit{Invoke:} 1 Boon).}

\paragraph{Mark of Raéyn (Standard, 8 XP)} \emph{Scene; Near; No.}
\textbf{Materials:} A drawn circle or sigil.\\
\textbf{Effect:} Protect a room/wagon. Crossing without consent generates \textbf{1 SB} (suit by GM).\\
\textbf{Push It:} Suppress one minor spell crossing; when it collapses, mark \textbf{1 SB (Hearts)} backlash.\\
\emph{Requires: Familiar + Codex \ (\textit{Invoke:} 1 Boon).}

\paragraph{Rite of the Sealed Mouth (Standard, 7 XP)} \emph{Scene; Near; No.}
\textbf{Materials:} Thread tied across lips, then removed.\\
\textbf{Effect:} Choose one lost channel: speech, script, or gesture. Within, it fails for the scene.\\
\textbf{Push It:} Suppress all three, but you are muted until scene end.\\
\emph{Requires: Familiar + Codex \ (\textit{Invoke:} 1 Boon).}

\paragraph{Rite of the Sealed Gate \textnormal{[WARD]} (High, 11 XP)} \emph{Scene; Near; No.}
\textbf{Materials:} Iron-powder circle, locked with a key.\\
\textbf{Effect:} Impassable boundary (~10 ft). Forcing entry inflicts \textbf{2 SB} (Clubs/Spades) on the intruder.\\
\textbf{Push It:} Enclose a chamber/courtyard. Each additional hour risks \textbf{1 SB (Diamonds)} toward omen/strain.\\
\emph{Requires: Familiar + Codex + Tier III \ (\textit{Invoke:} \textbf{2 Boons}).}\\
\emph{Obligation:} 7 segments.

% --- Patron: Mykkiel, Arbiter of the Writ (Judgment & Writ) ---
\subsection{Mykkiel, Arbiter of the Writ (Judgment \& Writ)}
\textit{Lore.} Mykkiel weighs speech against deed and seals verdicts in cold iron.

\begin{quote}
Name the charge. Name the terms. Then sign where you’ll bleed if you’re wrong.
\end{quote}

\paragraph{Stamp of Authority (Low, 4 XP)} \emph{Action; Near; Yes (doc/object).}
\textbf{Materials:} Cold-iron seal or writ-tag.\\
\textbf{Effect:} Visible mark of authority. \textbf{+1 die} to \emph{Command/Persuade} that asserts lawful order/claim.\\
\textbf{Push It:} Brief hush (one beat) among hecklers; mark \emph{Exposure +1}.\\
\emph{Requires: Familiar \ (\textit{Invoke:} 1 Boon).}

\paragraph{Rite of Proper Notice (Low, 5 XP)} \emph{Scene; Near; No.}
\textbf{Materials:} Writ-string tied and snapped.\\
\textbf{Effect:} Name a \emph{lawful venue} (dais, doorway, wagon). First hostile act there suffers \(-1\) die.\\
\textbf{Push It:} Name a \emph{protected act} (parley, surrender, testimony): \textbf{+1 effect} in the venue; breaking custom generates \textbf{1 SB (Hearts)}.\\
\emph{Requires: Familiar \ (\textit{Invoke:} 1 Boon).}

\paragraph{Writ of Compliance \textnormal{[COMMAND]} (Standard, 8 XP)} \emph{Action; Near; No.}
\textbf{Materials:} Red cord knotted while speaking the order.\\
\textbf{Effect:} Immediate command (“Stand down,” “Drop it,” “Open”). Target must comply now or suffer a Keeper-stated cost. DV by fiction; elites may test Resolve.\\
\textbf{Push It:} On compliance, impose \(-1\) die on target’s next aggressive act this scene.\\
\emph{Requires: Familiar + Codex \ (\textit{Invoke:} 1 Boon).}

\paragraph{Rite of the Speaking Seal (Standard, 7 XP)} \emph{Scene; Near; No.}
\textbf{Materials:} Wax seal impressed over a name/sigil.\\
\textbf{Effect:} Sanctify a statement (truce, custody, claim). Contradicting it suffers \(-1\) die; you gain \textbf{+1 die} to enforce it.\\
\textbf{Push It:} Once, ask who here intends breach; Keeper gives a strong clue or direct name.\\
\emph{Requires: Familiar + Codex \ (\textit{Invoke:} 1 Boon).}

\paragraph{Oath Irons \textnormal{[OATH]} (High, 11 XP)} \emph{Scene; Near; No.}
\textbf{Materials:} Two iron pins warmed in flame, touched to wrists, then quenched.\\
\textbf{Effect:} Bind two parties to a bounded term. Breach \emph{forces 2 SB} and brands a faint iron-mark until amends.\\
\textbf{Push It:} Extend to a small circle (up to four); each chooses one narrow exception (Keeper approves). Exploiting it generates \textbf{1 SB (Diamonds)}.\\
\emph{Requires: Familiar + Codex + Tier III \ (\textit{Invoke:} \textbf{2 Boons}).}\\
\emph{Obligation:} 7 segments.

% patrons/khemesh.tex
% Fate’s Edge — Patron: Khemesh, the Abyssal Maw

\subsection{Khemesh, the Abyssal Maw (Depths, Inexorability, Eldritch Terror)}
\textit{Lore.} Khemesh is not merely a lord of the depths but the hunger beneath them, a pressure older than seas. Those who bargain with him are marked by the abyss—seen in the way shadows cling, in the whispers heard when no voice speaks, in the certainty that all things will sink.

\begin{quote}
In the trench without light, the Maw waits. Even silence drowns.
\end{quote}

\paragraph{Whisper of the Trench (Low, 4 XP)} \emph{Instant; Near; No.}\\
\textbf{Effect:} Target hears impossible echoes and suffers \textbf{−1 die} on their next action.\\
\textbf{Push It:} Echoes coil in your own skull—take \textbf{Fatigue 1}, but the target also loses their next minor action.\\
\emph{Requires: Familiar \ (\textit{Invoke:} 1 Boon).}

\paragraph{Rite of Crushing Silence (Low, 5 XP)} \emph{Scene; Zone; No.}\\
\textbf{Materials:} A broken shell filled with ink-dark water.\\
\textbf{Effect:} Establish an oppressive silence; sound carries only as distorted whispers. Enemies in the zone gain \textbf{−1 die} to coordination or morale-driven actions.\\
\textbf{Push It:} A single enemy’s voice is stolen entirely for the scene.\\
\emph{Requires: Familiar \ (\textit{Invoke:} 1 Boon).}

\paragraph{Pressure of the Maw (Standard, 7 XP)} \emph{Instant; Near; No.}\\
\textbf{Materials:} A length of rusted chain submerged in water.\\
\textbf{Effect:} Target is pinned by invisible crushing force: treat as \texttt{[ENTANGLE]} with \textbf{Great Effect} if underwater or confined.\\
\textbf{Push It:} Inflict \textbf{Fatigue 1} on the target in addition to the restraint.\\
\emph{Requires: Familiar + Codex \ (\textit{Invoke:} 1 Boon).}

\paragraph{Rite of the Abyssal Vision (Standard, 9 XP)} \emph{Scene; Self; No.}\\
\textbf{Effect:} You perceive the world as Khemesh does—fractured, alien, crushing. Gain \textbf{+2 dice} to Notice and Arcana, and may ask one “true nature” question about a foe or structure.\\
\textbf{Cost:} When the scene ends, you suffer \textbf{Exposure +1} as your perception warps.\\
\textbf{Push It:} Extend the vision to one ally, but both take \textbf{Fatigue 1}.\\
\emph{Requires: Familiar + Codex \ (\textit{Invoke:} 1 Boon).}

\paragraph{The Maw Opens (High, 12 XP)} \emph{Scene; Zone; No.}\\
\textbf{Materials:} A sealed vessel of abyssal water, broken open.\\
\textbf{Effect:} Reality in the zone folds inward like the crushing deep: \\
\begin{itemize}
  \item Enemies act at \textbf{Desperate Position} by default.  
  \item Each beat, the Keeper may force \textbf{1 SB} (Spades/Clubs favored).  
  \item Structures, vessels, or wards fracture as if under immense weight.  
\end{itemize}
\textbf{Push It:} For one beat, declare a single enemy “crushed” (severe harm/effect). You immediately suffer \textbf{Fatigue 2} and \textbf{+1 Obligation}.\\
\emph{Requires: Familiar + Codex + Tier III \ (\textit{Invoke:} 2 Boons).}\quad \emph{Obligation:} 8 segments.

\paragraph{Rivalry: Raéyn.} Khemesh embodies the unknowable trench that swallows sailors; Raéyn embodies tides, travel, and the sea’s surface. Where Raéyn charts and protects, Khemesh unmoors and devours. In scenes of open sea, Raéyn gains the upper hand; in the abyssal dark, Khemesh dominates.

% --- Patron: Mab, Queen of Courts (Glamour & Bargain) ---
\subsection{Mab, Queen of Courts (Glamour \& Bargain)}
\textit{Lore.} The blush of truth, the dagger of etiquette, the smile that writes debts in perfume. Mab rules where desire dresses itself as courtesy.

\begin{quote}
Bend, don’t bow. Smile, don’t promise.
\end{quote}

\paragraph{Courtly Guise \textnormal{[VEIL]} (Low, 4 XP)} \emph{Action; Self; Yes (social only).}
\textbf{Materials:} Pin a sprig of green or silver thread.\\
\textbf{Effect:} Subtle glamour: \textbf{+1 die} to Persuade/Sway in refined settings; you appear as expected rank/guest.\\
\textbf{Push It:} Also mask one minor tell; the first piercing question in the scene generates \textbf{1 SB (Hearts)}.\\
\emph{Requires: Familiar \ (\textit{Invoke:} 1 Boon).}

\paragraph{Token of Favor (Low, 5 XP)} \emph{Scene; Near; No.}
\textbf{Materials:} A ribbon or ring bestowed.\\
\textbf{Effect:} Grant an ally \textbf{+1 die} to one social action against onlookers who recognize your favor; you gain \textbf{+1 effect} to support.\\
\textbf{Push It:} The token also chills a heckler (one beat of hesitation), but you mark \emph{Exposure +1}.\\
\emph{Requires: Familiar \ (\textit{Invoke:} 1 Boon).}

\paragraph{Mirror of Motives (Standard, 7 XP)} \emph{Action; Near; No.}
\textbf{Materials:} A polished shard or compact mirror.\\
\textbf{Effect:} Ask one pointed question about an NPC’s \emph{immediate} social goal; Keeper answers truthfully or with a strong tell. Gain \textbf{+1 die} to exploit it this scene.\\
\textbf{Push It:} Also expose a concealed slight or insult that matters to them, creating \textbf{1 SB (Hearts)} on that target.\\
\emph{Requires: Familiar + Codex \ (\textit{Invoke:} 1 Boon).}

\paragraph{The Price Agreed \textnormal{[OATH]} (Standard, 8 XP)} \emph{Scene; Near; No.}
\textbf{Materials:} Exchange a token of equal apparent value.\\
\textbf{Effect:} Bind a petty bargain (favor-for-favor). Breach forces \textbf{1 SB (Hearts or Diamonds)} on the breaker and stains their reputation locally this arc.\\
\textbf{Push It:} Sweeten terms with a minor boon (\(+1\) die once to the beneficiary), but you take \textbf{1 SB (Hearts)} if they later breach.\\
\emph{Requires: Familiar + Codex \ (\textit{Invoke:} 1 Boon).}

\paragraph{Sovereign Glamour \textnormal{[VEIL][REVEAL]} (High, 11 XP)} \emph{Scene; Zone; No.}
\textbf{Materials:} A circle of green felt or silk.\\
\textbf{Effect:} Establish Court: allies in Zone gain \textbf{+1 die} to social actions; crude threats suffer \(-1\) die. Once, peel one disguise/illusion in Zone.\\
\textbf{Push It:} Name a \emph{Court Law} (e.g., no drawn steel): first violation \emph{forces 2 SB} on the violator.\\
\emph{Requires: Familiar + Codex + Tier III \ (\textit{Invoke:} \textbf{2 Boons}).}\\
\emph{Obligation:} 6 segments.

% --- Patron: The Traveler (Ways & Roads) ---
\subsection{The Traveler (Ways \& Roads)}
\textit{Lore.} Crossroads remember every footfall. The Traveler minds the stories that move between places.

\begin{quote}
Put one foot in a promise, and the road will meet you halfway.
\end{quote}

\paragraph{Road-Sense (Low, 4 XP)} \emph{Action; Self; Yes (navigation).}
\textbf{Materials:} A road-nail or waystone pebble.\\
\textbf{Effect:} Unerringly pick the fastest \emph{safe} route in Near/Far; \textbf{+1 die} to avoid ambushes and delays this leg/scene.\\
\textbf{Push It:} Also spot one hidden bypass; taking it creates \textbf{1 SB (Clubs)} elsewhere on the map.\\
\emph{Requires: Familiar \ (\textit{Invoke:} 1 Boon).}

\paragraph{Traveler’s Boon (Low, 5 XP)} \emph{Scene; Self/Ally; No.}
\textbf{Materials:} Tie thread around a wrist.\\
\textbf{Effect:} Ignore one level of difficult terrain or bureaucracy for this scene; \textbf{+1 effect} to overland progress/escape checks.\\
\textbf{Push It:} Extend to one additional ally; mark \textbf{1 SB (Diamonds)} as the road exacts a toll (favors, papers, attention).\\
\emph{Requires: Familiar \ (\textit{Invoke:} 1 Boon).}

\paragraph{Waymark \textnormal{[PASSAGE]} (Standard, 7 XP)} \emph{Action; Near; No.}
\textbf{Materials:} Chalk mark at eye level.\\
\textbf{Effect:} Declare a lane as permitted/easy: allies on that lane gain better flow (Position/Effect bump or ignore one obstacle).\\
\textbf{Push It:} The lane persists between scenes until disturbed; first enemy who exploits it \emph{forces 1 SB (Spades)} on your party.\\
\emph{Requires: Familiar + Codex \ (\textit{Invoke:} 1 Boon).}

\paragraph{Bridge the Mile \textnormal{[TRANSPORT]} (Standard, 9 XP)} \emph{Instant; Near; No.}
\textbf{Materials:} Two pinches of road-dust clapped together.\\
\textbf{Effect:} Relocate a willing target within Far along a visible or named route; arrivals are steady but noticed.\\
\textbf{Push It:} Carry one extra ally or a small bundle; arrivals are off-balance (worse Position for one beat).\\
\emph{Requires: Familiar + Codex \ (\textit{Invoke:} 1 Boon).}

\paragraph{Crown of Crossings (High, 12 XP)} \emph{Scene; Zone; No.}
\textbf{Materials:} A brass compass missing its needle.\\
\textbf{Effect:} You call the Road: allies gain \textbf{+1 die} to move/evade; pursuit suffers \(-1\) die. Once, declare “the long way is short” to finish a travel clock segment for free.\\
\textbf{Push It:} Also seal a hostile route (like a temporary [WARD] against passage) for one beat; generates \textbf{2 SB (Clubs/Diamonds)} in border complications.\\
\emph{Requires: Familiar + Codex + Tier III \ (\textit{Invoke:} \textbf{2 Boons}).}\\
\emph{Obligation:} 7 segments.

% --- Patron: The Clockwork Monad (Mechanism & Process) ---
\subsection{The Clockwork Monad (Mechanism \& Process)}
\textit{Lore.} Gears remember the plan even when their makers forget. The Monad prizes precision, iteration, and the beauty of mechanisms that keep their word.

\begin{quote}
Every tooth matters. Especially the ones you cannot see.
\end{quote}

\paragraph{Calibrated Touch (Low, 4 XP)} \emph{Action; Self; Yes (tinkering).}
\textbf{Materials:} A single oiled gear tooth.\\
\textbf{Effect:} \textbf{+1 die} to repair, set, or disarm precise mechanisms; re-roll one \texttt{1} on a Tinker action.\\
\textbf{Push It:} Guarantee no collateral on a simple device, but you take \textbf{1 SB (Clubs)} if rushed.\\
\emph{Requires: Familiar \ (\textit{Invoke:} 1 Boon).}

\paragraph{Process Lock (Low, 5 XP)} \emph{Scene; Near; No.}
\textbf{Materials:} A drop of red oil.\\
\textbf{Effect:} Name a process (reload, dispatch, raise alarm). First attempt to perform it in-scene suffers \(-1\) die as steps “stick.”\\
\textbf{Push It:} The second attempt also hesitates (one beat), but the third surges forward and \emph{forces 1 SB (Spades)}.\\
\emph{Requires: Familiar \ (\textit{Invoke:} 1 Boon).}

\paragraph{Iterative Advantage (Standard, 7 XP)} \emph{Action; Self/Ally; No.}
\textbf{Materials:} Notched tally on metal.\\
\textbf{Effect:} On a repeated action this scene, grant \textbf{+2 dice} or \textbf{+1 effect}.\\
\textbf{Push It:} Bank a follow-up \textbf{+1 die} for the same action later this scene; if unused, generate \textbf{1 SB (Diamonds)} as unused potential jams something.\\
\emph{Requires: Familiar + Codex \ (\textit{Invoke:} 1 Boon).}

\paragraph{Suppress Malfunction \textnormal{[DISPEL]} (Standard, 8 XP)} \emph{Instant; Near; No.}
\textbf{Materials:} Mainspring coil released.\\
\textbf{Effect:} End/suppress an ongoing \emph{mechanical or procedural} failure/curse (e.g., jamming jam, recursive alarm, fatigue spiral tick). DV by fiction.\\
\textbf{Push It:} Also clear one related clock segment; create \textbf{1 SB (Clubs)} as pressure shifts elsewhere.\\
\emph{Requires: Familiar + Codex \ (\textit{Invoke:} 1 Boon).}

\paragraph{Prime the Engine (High, 12 XP)} \emph{Scene; Zone; No.}
\textbf{Materials:} A ring of interlocked cogs.\\
\textbf{Effect:} Allies in Zone add a “\emph{Process Buff}”: first action that repeats in-scene gets an automatic Position bump \emph{or} upgrades Effect.\\
\textbf{Push It:} Also seize timing: once, reorder two adjacent beats; mark \textbf{2 SB (Diamonds)} as the world protests.\\
\emph{Requires: Familiar + Codex + Tier III \ (\textit{Invoke:} \textbf{2 Boons}).}\\
\emph{Obligation:} 7 segments.

% patrons/varnek-karn.tex
% Fate’s Edge — Patron: Varnek Karn, the Bone King (Necromantic Archives)

\subsection{Varnek Karn, the Bone King (Necromantic Archives)}
\textit{Lore.} Bones remember. Varnek keeps their ledgers: last sights, last debts, last names.

\begin{quote}
Ask gently. The skull will answer in fragments; the ledger is never truly closed.
\end{quote}

\paragraph{Whisper to Restless Spirits (Low, 4 XP)} \emph{Action; Near; No.}\\
\textbf{Materials:} A pinch of grave-dust stirred into breath.\\
\textbf{Effect:} +1 die to investigate a recent death (within a day) \emph{or} ask a single yes/no about the cause of death.\\
\textbf{Push It:} Learn one fleeting sensory shard (sound, scent, image), but mark \emph{Exposure +1}.\\
\emph{Requires: Familiar \ (\textit{Invoke:} 1 Boon).}

\paragraph{Unfinished Ledger (Low, 5 XP)} \emph{Instant; Touch; No.}\\
\textbf{Materials:} A binding thread tied to remains or a personal relic.\\
\textbf{Effect:} Learn one \emph{unfinished business} binding the spirit (name a task, debt, or oath). Acting on it grants +1 die once this scene to relevant rolls.\\
\textbf{Push It:} Also learn one \emph{adversary} tied to that business; generate 1 SB (Hearts or Diamonds).\\
\emph{Requires: Familiar \ (\textit{Invoke:} 1 Boon).}

\paragraph{Speaking Bones (Standard, 8 XP)} \emph{Scene; Touch; No.}\\
\textbf{Materials:} Oil the jaw, set the teeth with a sigil.\\
\textbf{Effect:} A corpse answers \textbf{2} questions about circumstances of death from its own perspective (fragmented, literal). Max 1 corpse/scene.\\
\textbf{Push It:} Ask a \textbf{third} question; mark +1 Obligation.\\
\emph{Requires: Familiar + Codex \ (\textit{Invoke:} 1 Boon).}

\paragraph{The Remembering Host (Standard, 9 XP)} \emph{Scene; Near; No.}\\
\textbf{Materials:} A small reliquary and consecrated twine.\\
\textbf{Effect:} Animate a \emph{seeker swarm} (count it as a \textbf{Standard Asset} with 4-segment integrity) to scout, fetch, or tail. Not combat-capable; acts in beats per GM.\\
\textbf{Push It:} Grant one \emph{special action} (block a doorway, retrieve a key) once, then the swarm unravels; mark +1 Obligation.\\
\emph{Requires: Familiar + Codex \ (\textit{Invoke:} 1 Boon).}

\paragraph{Court of Echoes (High, 12 XP)} \emph{Scene; Zone; No.}\\
\textbf{Materials:} Circle of bone-totems and a tolling chime.\\
\textbf{Effect:} Convene an \emph{advisory court} of the dead (treat as a \textbf{Major Asset} clock [6]) that can: answer up to 3 complex questions, sway the fearful, or impose silence upon casual lies in the zone.\\
\textbf{Push It:} Command a single decisive act (e.g., frighten a mob, barge a door) then tick the court twice; mark +1 Obligation.\\
\emph{Requires: Familiar + Codex + Tier III \ (\textit{Invoke:} \textbf{2 Boons}).}\quad \emph{Obligation:} 7 segments.

% patrons/nidhoggr.tex
% Fate’s Edge — Patron: Nidhoggr, the World-Worm (Dreaming Antiquity)

\subsection{Nidhoggr, the World-Worm (Dreaming Antiquity)}
\textit{Lore.} Beneath stone and sleep lies the slow memory of the world. Nidhoggr turns in aeons, dreaming of roads once walked and oaths once sworn.

\begin{quote}
Press your ear to the earth and wait. If it remembers you, it will answer.
\end{quote}

\paragraph{Glimpse the Ancient's Shadow (Low, 4 XP)} \emph{Action; Self; No.}\\
\textbf{Materials:} Pinch of dust from a worked stone.\\
\textbf{Effect:} +1 die to actions that identify, date, or interpret \emph{ancient} sites, scripts, or artifacts this scene; once this scene, ask one yes/no about the site's original purpose.\\
\textbf{Push It:} Add +1 Effect on one related roll, but suffer \emph{Fatigue 1}.\\
\emph{Requires: Familiar \ (\textit{Invoke:} 1 Boon).}

\paragraph{Drink from the Dreaming Deep (Low, 5 XP)} \emph{Instant; Self; No.}\\
\textbf{Materials:} Mouthful of clean water poured over stone, swallowed with eyes closed.\\
\textbf{Effect:} Learn one hidden factual detail about the immediate locale's \emph{past}. GM answers plainly or via a sensory echo.\\
\textbf{Cost:} Suffer \emph{Fatigue 1} and mark \emph{Exposure +1} as the dream clings.\\
\emph{Requires: Familiar \ (\textit{Invoke:} 1 Boon).}

\paragraph{Stone-Sleeper's Murmur (Standard, 7 XP)} \emph{Scene; Near (contact locus); No.}\\
\textbf{Materials:} Ear to bedrock, wall, or hewn pillar.\\
\textbf{Effect:} Once per beat while in contact, ask 1 question about a \emph{past event} that physically touched this stone; answers are fragmentary but truthful (max 3 questions/scene).\\
\textbf{Push It:} One answer is delivered with precise sensory clarity, but generate 1 SB (suit by GM).\\
\emph{Requires: Familiar + Codex \ (\textit{Invoke:} 1 Boon).}

\paragraph{Awakened Chronicle (Standard, 9 XP)} \emph{Ritual (Significant Time); Zone; No.}\\
\textbf{Materials:} Chalk spiral and four touchstones from the site.\\
\textbf{Effect:} The zone ``replays'' a notable past moment as ghostly echoes all can witness (no harm). Participants gain +2 dice on \emph{one} Investigate/Recall about that event this scene.\\
\textbf{Push It:} Add a second moment from a different era, but mark +1 Obligation.\\
\emph{Requires: Familiar + Codex \ (\textit{Invoke:} 1 Boon).}

\paragraph{Dive into the World-Worm's Dream (High, 12 XP)} \emph{Scene; Self; No.}\\
\textbf{Materials:} Lie upon bare earth within a drawn circle of stones.\\
\textbf{Effect:} Ask up to \textbf{3} factual questions about the \emph{distant past} or \emph{buried truth} of this place, people, or item. Answers arrive as lucid dream signs.\\
\textbf{Cost (choose one):} Suffer \emph{Fatigue 2 \& Exposure +1} \emph{or} gain +3 dice to one reality-warping cast this scene and generate 2 SB immediately.\\
\emph{Requires: Familiar + Codex + Tier III \ (\textit{Invoke:} \textbf{2 Boons}).}\quad \emph{Obligation:} 7 segments.

% --- Patron: Oath of Flame & Light (Paladin Archetype: Dawn & Vows) ---
\subsection{Oath of Flame \& Light (Dawn \& Vows)}
\textit{Lore.} Light names, binds, and burns. The Oath favors sworn keepers—those who stand in the open and keep their word even when it costs.

\begin{quote}
Speak in the light. Keep it, or the light will keep \emph{you}.
\end{quote}

% Patron's Gift note (Paladin vibe)
\paragraph*{Patron’s Gift (Imbuement) — Paladin Flavor.}
With \textbf{Thiasos (Familiar)}, you may invoke the Gift (1 action, 1/scene) to sanctify a weapon or badge: it grants \textbf{+1 Melee} and \textbf{+1 Thematic (Command)} while your fiction honors a declared vow or protection. Push: extend one extra scene (+1 Obligation). (See \S\ref{subsec:patrons-gift}.)

% --- Low Rites ---
\paragraph{Kindle Vow (Low, 4 XP)} \emph{Action; Self/Ally; Yes.}
\textbf{Materials:} Ampoule of consecrated spark.\\
\textbf{Effect:} Name a near-term pledge this scene (\emph{hold the line}, \emph{get them out}). Bearer gains \textbf{+1 die} to actions that keep it.\\
\textbf{Push It:} First betrayal or hesitation \emph{forces 1 SB (Hearts)} on the bearer.\\
\emph{Requires: Familiar \ (\textit{Invoke:} 1 Boon).}

\paragraph{Lay on Hands \textnormal{[CLEANSE][HEAL]} (Low, 5 XP)} \emph{Instant; Touch; No.}
\textbf{Materials:} Palm over wound; vow whispered.\\
\textbf{Effect:} Remove one minor affliction or downgrade \emph{Harm} by one step \emph{or} clear \emph{Fatigue 1}. DV by fiction for stubborn curses/poisons.\\
\textbf{Push It:} Also grant \textbf{+1 die} to the target’s next Resist this scene; you mark \emph{Exposure +1}.\\
\emph{Requires: Familiar \ (\textit{Invoke:} 1 Boon).}

% --- Standard Rites ---
\paragraph{Sunlit Parley (Standard, 7 XP)} \emph{Scene; Near; No.}
\textbf{Materials:} Vow-ring engraved with sunrise and true name.\\
\textbf{Effect:} Establish open terms: honest persuasion gains \textbf{+1 die}; deceit attempts suffer \(-1\) die in this scene’s parley.\\
\textbf{Push It:} Once, demand a public answer; evasion \emph{forces 1 SB (Hearts)} on the evader.\\
\emph{Requires: Familiar + Codex \ (\textit{Invoke:} 1 Boon).}

\paragraph{Purge the Shadow \textnormal{[REVEAL][DISPEL]} (Standard, 9 XP)} \emph{Instant; Near; No.}
\textbf{Materials:} Consecrated spark cracked to light.\\
\textbf{Effect:} Expose illusions/disguises and suppress one minor ongoing glamour/curse in Near (DV by fiction).\\
\textbf{Push It:} Also sear a lingering tell on the source; you can sense it once again this arc; mark \textbf{1 SB (Diamonds)}.\\
\emph{Requires: Familiar + Codex \ (\textit{Invoke:} 1 Boon).}

\paragraph{Radiant Smite \textnormal{[FOLLOW-UP] (Standard, 8 XP)}} \emph{Action; Self; No.}
\textbf{Materials:} Consecrated spark smeared on weapon or badge.\\
\textbf{Effect:} Consecrate your next strike this scene. On your next successful \emph{melee} hit this scene:
\begin{itemize}
  \item Upgrade the hit’s \textbf{Effect} by one step (to Great if applicable), and
  \item Add \textbf{+1 Harm (Burn)} \emph{or} force \textbf{1 SB (Spades)} on the target’s side if the blow is narrative rather than wounding.
\end{itemize}
\textit{Special.} Versus Undead, Oath-breakers, or Outsiders: the blow also \emph{sears the untrue}. Undead/Oath-breakers suffer \(-1\) die on their next action; Outsiders gain \textbf{+1} segment on Leash/Exit Tally (Hit only). If the attack \emph{misses}, the smite charge lingers for one beat; after that it gutters, creating \textbf{1 SB (Diamonds)} as attention swells.\\
\textbf{Push It:} The strike flares—on hit, emit a Close burst: hostile creatures in Close suffer \(-1\) die for one beat \emph{or} are driven back (worse Position by one step). Mark \textbf{+1 Obligation}.\\
\emph{Requires: Familiar + Codex \ (\textit{Invoke:} 1 Boon).}

% --- High Rite ---
\paragraph{Covenant Blaze \textnormal{[OATH][FORTIFY]} (High, 12 XP)} \emph{Scene; Zone; No.}
\textbf{Materials:} A brazier lit with three names spoken.\\
\textbf{Effect:} Those who swear within gain a halo: \textbf{+1 die} to acts that keep the oath; attackers against a haloed subject suffer \(-1\) die if the act would violate the sworn terms. Oath-breakers immediately \emph{force 2 SB (Hearts/Spades)} and the halo scorches them (1 Harm, Burn).\\
\textbf{Push It:} The blaze also sanctifies the threshold (one beat of temporary \texttt{[WARD]} vs.\ oath-breakers entering).\\
\emph{Requires: Familiar + Codex + Tier III \ (\textit{Invoke:} \textbf{2 Boons}).}\\
\emph{Obligation:} 7 segments.

% =========================
% PATRON — The Carrion-King
% =========================
\subsection{The Carrion-King (Decay \& Cycles)}

\paragraph{Lore.}
The Carrion-King does not bring rot to destroy but to recycle. To him, decay is not an end but a return to the great cycle of flesh into soil, bone into earth. His priests are gardeners of entropy, ensuring that nothing lingers beyond its time and that new life always rises from death.

\paragraph{Quote.}
\emph{“All banquets end in bones, and from those bones the feast begins anew.” — The Carrion-King}

\paragraph{Rite of Gentle Rot (Low, 5 XP)} \emph{Instant; Touch; Yes (decay only).}
\textbf{Materials:} Spoiled food or a dead insect. \\
\textbf{Effect:} Accelerate natural decay on a small, non-living object (rope, lock, rations): +1 Effect to \emph{Break/Sabotage}. \\
\textbf{Push It:} A second, similar item in Close range decays; scavengers/vermin are drawn. \\
\emph{Requires: Familiar \ (\textit{Invoke:} 1 Boon).}

\paragraph{Rite of the Wilting Bloom (Low, 4 XP)} \emph{Scene; Self; No.}
\textbf{Materials:} A withered flower. \\
\textbf{Effect:} Aura of mild decay: +1 die to resist disease/poison; your carried food won’t worsen (becomes tasteless). \\
\textbf{Push It:} Wither a small plant/food source in Near; suffer \textbf{Fatigue 1}. \\
\emph{Requires: Familiar \ (\textit{Invoke:} 1 Boon).}

\paragraph{Rite of the Cycle's Turn (Standard, 8 XP)} \emph{Scene; Touch; No.}
\textbf{Materials:} A creature dead less than an hour. \\
\textbf{Effect:} From death, sustain life: choose one—purify a small food/water cache; sprout useful fungi/herbs; \emph{or} grant +1 die to resist disease/poison to one target. \\
\textbf{Push It:} Superior yield/quality; local life/death balance tilts (GM clocks/complications). \\
\emph{Requires: Familiar + Codex \ (\textit{Invoke:} 1 Boon).}

\paragraph{Rite of the Peaceful Rest (Standard, 7 XP)} \emph{Instant; Near; No.}
\textbf{Materials:} Grave dirt over the corpse. \\
\textbf{Effect:} Lay a minor spirit; prevent easy animation \emph{or} quiet a small haunting. Gain +2 dice on the next social roll with mourners/spirits. \\
\textbf{Push It:} You/ally gain undead-fear immunity for the scene; other nearby spirits grow restless. \\
\emph{Requires: Familiar + Codex \ (\textit{Invoke:} 1 Boon).}

\paragraph{Rite of the Final Compost (High, 13 XP)} \emph{Scene; Zone; No.}
\textbf{Materials:} A handful of grave dirt. \\
\textbf{Effect:} Accelerate decay across a zone: structures falter (attackers relying on them $-1$ die), enemy maintenance falters ($-1$ die to keep gear/efforts stable). Alternatively, consume a major obstacle over the scene. \\
\textbf{Push It:} The zone erupts with sickly growth; creatures lingering gain \emph{Sickened}. The growth may be harvested later. \\
\emph{Requires: Familiar + Codex + Tier III \ (\textit{Invoke:} \textbf{2 Boons}).} \\
\emph{Obligation:} 7 segments.

\paragraph{Rite of the Great Cycle (High, 14 XP)} \emph{Extended; Touch; No.}
\textbf{Materials:} Bury a seed in rich, rotten earth. \\
\textbf{Effect:} Transform a significant dead mass (large corpse/fallen tree) into something useful: fertile plot, unique reagent, or a temporary environmental boon (GM scales time/impact). \\
\textbf{Push It:} Compress to one scene; spectacle attracts attention and interference. \\
\emph{Requires: Familiar + Codex + Tier III \ (\textit{Invoke:} \textbf{2 Boons}).} \\
\emph{Obligation:} 7 segments.

% =========================
% PATRON — The Gallow's Bell
% =========================
\subsection{The Gallow's Bell (Consequences \& Retribution)}

\paragraph{Lore.}
The Bell tolls for every oath broken, every sin unatoned. Its sound is not heard by all, but by those whose guilt cannot be outrun. To follow the Gallow’s Bell is to be an instrument of consequence, a reminder that no action escapes the echo of its price.

\paragraph{Quote.}
\emph{“Three times shall the bell toll, and thrice shall the guilty stumble.” — The Gallow’s Bell}

\paragraph{Rite of the Whispered Name (Low, 4 XP)} \emph{Scene; Self; No.}
\textbf{Materials:} Noose fragment; whispered confession. \\
\textbf{Effect:} Mark a target by name; within 3 scenes the GM must introduce a complication tied to their recent deeds. \\
\textbf{Push It:} The complication strikes immediately; homonyms may also be brushed by the omen. \\
\emph{Requires: Familiar \ (\textit{Invoke:} 1 Boon).}

\paragraph{Curse of the Unsettled Sleep (Low, 5 XP)} \emph{Scene; Near; No.}
\textbf{Materials:} Dream-catcher woven with guilt; midnight vigil. \\
\textbf{Effect:} Target dreams of misdeeds, inviting roleplay leverage and soft penalties per fiction. \\
\textbf{Push It:} The dream grants a true omen of coming consequence; one of your secrets is revealed to the dreamer. \\
\emph{Requires: Familiar \ (\textit{Invoke:} 1 Boon).}

\paragraph{Rite of the Broken Mirror (Standard, 8 XP)} \emph{Scene; Near; No.}
\textbf{Materials:} Shattered mirror; drop of the target’s blood. \\
\textbf{Effect:} Target’s reflection shows “true nature” to observers; their deceptions generate automatic narrative complications on success. \\
\textbf{Push It:} The reflection turns hostile (ambient social friction); all mirrors nearby become \emph{strange} until scene end. \\
\emph{Requires: Familiar + Codex \ (\textit{Invoke:} 1 Boon).}

\paragraph{Mark of Unfinished Business (Standard, 7 XP)} \emph{Extended; Touch; No.}
\textbf{Materials:} An unfinished letter; a broken chain link. \\
\textbf{Effect:} Target accrues story-weight from unresolved obligations; expect recurring complications until addressed. \\
\textbf{Push It:} The obligation focuses into a specific demand; you share narrative responsibility to resolve it. \\
\emph{Requires: Familiar + Codex \ (\textit{Invoke:} 1 Boon).}

\paragraph{Curse of the Singing Chain (High, 13 XP)} \emph{Scene; Near; No.}
\textbf{Materials:} A prison chain and a funeral bell-hammer. \\
\textbf{Effect:} Target hears echoes of broken promises: suffer \textbf{Fatigue 1}; lying draws narrative backlash. \\
\textbf{Push It:} A spectral chain manifests (movement constrained by fiction); nearby chains resonate \& misbehave. \\
\emph{Requires: Familiar + Codex + Tier III \ (\textit{Invoke:} \textbf{2 Boons}).} \\
\emph{Obligation:} 7 segments.

\paragraph{The Bell That Rings Thrice (High, 14 XP)} \emph{Extended; Zone; No.}
\textbf{Materials:} Gallows rope; three iron bells rung in order. \\
\textbf{Effect:} Over the next session, the target suffers three escalating, nature-true consequences (GM stages). \\
\textbf{Push It:} The toll splashes onto their close allies/kin; you are marked as an agent of retribution. \\
\emph{Requires: Familiar + Codex + Tier III \ (\textit{Invoke:} \textbf{2 Boons}).} \\
\emph{Obligation:} 7 segments.

% =========================
% PATRON — Isoka, Angel of Serpents
% =========================
\subsection{Isoka, Angel of Serpents (Change \& Shedding)}

\paragraph{Lore.} 
Isoka teaches that every self is temporary. Serpents shed their skin not in weakness but in renewal, leaving the brittle husk behind as proof that transformation is survival. Followers of Isoka learn to embrace disguise, deception, and metamorphosis, casting aside the past as easily as a garment.

\paragraph{Quote.}
\emph{“Do not mourn the skin you shed. It was never meant to last.” — Isoka, Angel of Serpents}

\paragraph{Loosen the Old Skin (Low, 4 XP)} \emph{Scene; Self; Yes (resist only).}
\textbf{Materials:} A discarded snakeskin or a loose thread. \\
\textbf{Effect:} +1 die to resist an ongoing \emph{Condition} this scene \emph{or} re-roll one \textbf{1} on an escape/evasion. \\
\textbf{Push It:} Also ignore one minor movement penalty; you leave behind a token of your old self that others can leverage. \\
\emph{Requires: Familiar \ (\textit{Invoke:} 1 Boon).}

\paragraph{Rite of the Subtle Shift (Low, 5 XP)} \emph{Scene; Self; No (stable).}
\textbf{Materials:} Palming a small object from one pocket to another. \\
\textbf{Effect:} Fluid demeanor: +1 die to \textbf{Deceive} to pass as a nearby class/profession \emph{or} +1 Effect to blend into a new crowd/site. \\
\textbf{Push It:} Bypass one minor identity check; you must maintain the false role until scene end. \\
\emph{Requires: Familiar \ (\textit{Invoke:} 1 Boon).}

\paragraph{Shed the Former Self (Standard, 8 XP)} \emph{Scene; Self; No.}
\textbf{Materials:} Full change of clothing and an adopted mannerism. \\
\textbf{Effect:} +2 dice to resist one named ongoing \emph{Condition}; once/session declare a minor physical contingency retroactively. \\
\textbf{Push It:} Clear a \emph{temporary, identity-based} Minor Condition; your former identity becomes active in the fiction. \\
\emph{Requires: Familiar + Codex \ (\textit{Invoke:} 1 Boon).}

\paragraph{Rite of the Forked Tongue (Standard, 7 XP)} \emph{Scene; Self; No.}
\textbf{Materials:} A harmless lie told to a mirror. \\
\textbf{Effect:} Ambiguous persuasion: when you \textbf{Sway} or \textbf{Command}, a success may generate \emph{Diamonds} (leverage) instead of SB. \\
\textbf{Push It:} One carefully worded lie this scene is accepted as truth; the displaced truth seeks return, complicating matters. \\
\emph{Requires: Familiar + Codex \ (\textit{Invoke:} 1 Boon).}

\paragraph{Complete Metamorphosis (High, 12 XP)} \emph{Scene; Self; No.}
\textbf{Materials:} A complete identity kit (garb, voice, tokens). \\
\textbf{Effect:} Full appearance/voice change; begin \emph{Controlled} on \textbf{Deceive/Stealth}; once/scene declare a minor contingency retroactively. \\
\textbf{Push It:} Spoof scent/biometric for one check; your original identity partially unmoors and acts independently. \\
\emph{Requires: Familiar + Codex + Tier III \ (\textit{Invoke:} \textbf{2 Boons}).} \\
\emph{Obligation:} 7 segments.

\paragraph{Rite of the Cast-Off History (High, 13 XP)} \emph{Extended; Self; No.}
\textbf{Materials:} Burning or defacing all mundane records of your old life. \\
\textbf{Effect:} On completion, common records/memories of that identity become unreliable. Trackers via that identity suffer $-2$ dice. (Does not foil magic or intimates.) \\
\textbf{Push It:} A plausible “death” is created for the old identity; one intimate senses deception. \\
\emph{Requires: Familiar + Codex + Tier III \ (\textit{Invoke:} \textbf{2 Boons}).} \\
\emph{Obligation:} 7 segments.

\subsection{Maelstraeus, The Infernal Bargainer (Pacts \& Debt)}
\paragraph{Lore.}  
Maelstraeus is whispered of as the Infernal Bargainer, the one who tallies every promise and weighs every gift. His followers insist that all power must be transacted, and no boon is free. Mortals who call him do so for leverage, not devotion — and yet every coin spent in his name is another bead on his endless abacus.  

\paragraph{Quote.}  
\emph{“Nothing is given. All is traded. You may forget the bargain, but I never do.” — Maelstraeus, The Infernal Bargainer}  



\paragraph{Rite of the Binding Quill (Low, 4 XP)} 
\emph{Scene; Near; No.}  
\textbf{Materials:} A drop of ink or blood upon parchment.  
\textbf{Effect:} Seal a spoken agreement in narrative force. Breaking the terms imposes a Minor Condition (Guilt, Debt, Marked).  
\textbf{Push It:} The pact is harder to break (Moderate Condition), but Maelstraeus claims a fragment of the truth surrounding the deal for his own designs.  
\emph{Requires: Familiar \ (\textit{Invoke:} 1 Boon).}  

\paragraph{Rite of the Weighed Scales (Low, 5 XP)}  
\emph{Duration: Scene; Range: Self; Stacking: No.}  
\textbf{Materials:} A coin balanced upon your palm until it falls.  
\textbf{Effect:} Gain +1 die when bargaining, haggling, or leveraging debts. You instinctively sense if the other side is giving more than they receive.  
\textbf{Push It:} You may demand a small hidden truth in the bargain, but you take Fatigue 1 from the mental strain of weighing it.  
\emph{Requires: Familiar \ (\textit{Invoke:} 1 Boon).}  

\paragraph{Rite of the Token Exchange (Low, 6 XP)}  
\emph{Duration: Scene; Range: Near; Stacking: No.}  
\textbf{Materials:} A small token given and received (coin, trinket, written word).  
\textbf{Effect:} You and the target both gain +1 die to a single roll this scene — but the GM chooses a minor narrative complication that ties you together until the “exchange” is settled.  
\textbf{Push It:} You may declare the complication instead, but the GM immediately gains 1 SB to “enforce the bargain” later.  
\emph{Requires: Familiar \ (\textit{Invoke:} 1 Boon).}  

\paragraph{Rite of the Debt Ledger (Standard, 8 XP)}  
\emph{Duration: Scene; Range: Self; Stacking: Yes (different debts).}  
\textbf{Materials:} A written list of names or accounts.  
\textbf{Effect:} Track a target who owes you a boon or debt. You gain +1 effect to actions to collect or enforce repayment.  
\textbf{Push It:} You may declare a “hidden clause” that retroactively complicates the debtor’s situation in your favor, but this grants GM 1 SB immediately.  
\emph{Requires: Familiar + Codex \ (\textit{Invoke:} 1 Boon).}  

\paragraph{Rite of the Infernal Seal (Standard, 7 XP)}  
\emph{Duration: Scene; Range: Touch; Stacking: No.}  
\textbf{Materials:} Wax seal impressed upon parchment or skin.  
\textbf{Effect:} Affix a mark of pact upon an ally or foe. For allies, once per scene they may reroll a failed action tied to the bargain. For foes, the mark imposes -1 die to resist your Sway or Command.  
\textbf{Push It:} The Seal glows with infernal script, undeniable to all, but its visibility may draw unwanted attention from other Patrons.  
\emph{Requires: Familiar + Codex \ (\textit{Invoke:} 1 Boon).}  

\paragraph{Rite of the Eternal Contract (High, 12 XP)}  
\emph{Extended; Range: Self+Other; Stacking: No.}  
\textbf{Materials:} A signed document, sealed with both blood and wax.  
\textbf{Effect:} Forge an extended pact with another character. While active, both gain a persistent +1 die when acting in line with the contract, but each breach immediately advances a 6-segment “Debt Claimed” clock.  
\textbf{Push It:} The pact can cheat death once for one party — they revive at Harm 3 instead of dying — but the “Debt Claimed” clock jumps forward 3 segments.  
\emph{Requires: Familiar + Codex + Tier III \ (\textit{Invoke:} \textbf{2 Boons}).}  
\emph{Obligation:} 7 segments.  

\paragraph{Rite of the Infernal Exchange (High, 14 XP)}  
\emph{Scene; Zone; No.}  
\textbf{Materials:} A balance scale with something of personal value upon each side.  
\textbf{Effect:} Exchange properties, conditions, or consequences between two willing targets. (e.g., swap Fatigue 2 for Harm 1, trade a social Condition, or move SB gain from one character to another).  
\textbf{Push It:} The exchange may include an unwilling target, but Maelstraeus claims something from you in kind — a permanent scar, memory, or name.  
\emph{Requires: Familiar + Codex + Tier III \ (\textit{Invoke:} \textbf{2 Boons}).}  
\emph{Obligation:} 7 segments.  

\subsection{Maestraea, The Scarlet Temptress (Temptation \& Pacts)}

\paragraph{Lore.}  
Maestraea is invoked in whispers and perfumes, the Scarlet Temptress whose gifts are sweeter than they are safe. Where Maelstraeus weighs coins, she weighs hearts. Her followers say that no promise is without a kiss, and no kiss without a hook. She thrives on secrets confessed in candlelight and debts owed in passion.  

\paragraph{Quote.}  
\emph{“You call it a bargain, love. I call it a kiss — one you’ll never forget.” — Maestraea, The Scarlet Temptress}  

\paragraph{Rite of the Velvet Promise (Low, 4 XP)} 
\emph{Scene; Near; No.}  
\textbf{Materials:} A whispered vow sealed with a kiss or caress.  
\textbf{Effect:} Seal a spoken promise with allure. Breaking the terms imposes a Minor Condition (Guilt, Tempted, Marked).  
\textbf{Push It:} The promise becomes intoxicating — breaking it escalates to a Moderate Condition, but Maestraea claims a hidden desire of the target.  
\emph{Requires: Familiar \ (\textit{Invoke:} 1 Boon).}  

\paragraph{Rite of the Honeyed Tongue (Low, 5 XP)}  
\emph{Duration: Scene; Range: Self; Stacking: No.}  
\textbf{Materials:} A sweet taste, wine, or spice upon the lips.  
\textbf{Effect:} Gain +1 die to Sway or Performance rolls. Targets instinctively want to please you, though they may not know why.  
\textbf{Push It:} You glimpse one of the target’s secret longings, but the intimacy costs you Fatigue 1.  
\emph{Requires: Familiar \ (\textit{Invoke:} 1 Boon).}  

\paragraph{Rite of the Crimson Token (Low, 6 XP)}  
\emph{Duration: Scene; Range: Near; Stacking: No.}  
\textbf{Materials:} A personal token given and received (lock of hair, charm, garment ribbon).  
\textbf{Effect:} Both giver and recipient gain +1 die to a single roll this scene, but they are bound by a palpable sense of longing or obligation until resolved.  
\textbf{Push It:} You may declare the nature of the bond, but the GM immediately gains 1 SB to entangle you further.  
\emph{Requires: Familiar \ (\textit{Invoke:} 1 Boon).}  

\paragraph{Rite of the Perfumed Ledger (Standard, 8 XP)}  
\emph{Duration: Scene; Range: Self; Stacking: Yes (different targets).}  
\textbf{Materials:} A list of names written in scented ink.  
\textbf{Effect:} You mark a target who owes you affection, desire, or service. You gain +1 effect when exploiting that bond.  
\textbf{Push It:} You may add a “hidden clause” to the bond retroactively, but this grants GM 1 SB immediately.  
\emph{Requires: Familiar + Codex \ (\textit{Invoke:} 1 Boon).}  

\paragraph{Rite of the Kiss of Chains (Standard, 7 XP)}  
\emph{Duration: Scene; Range: Touch; Stacking: No.}  
\textbf{Materials:} A kiss, or blood upon the lips.  
\textbf{Effect:} Place a mark of temptation upon an ally or foe. Allies may reroll one failed action this scene; foes suffer −1 die to resist your Sway or Command.  
\textbf{Push It:} The mark glows faintly with infernal allure, undeniable to all, but its visibility may draw attention from rival Patrons.  
\emph{Requires: Familiar + Codex \ (\textit{Invoke:} 1 Boon).}  

\paragraph{Rite of the Scarlet Covenant (High, 12 XP)}  
\emph{Extended; Range: Self+Other; Stacking: No.}  
\textbf{Materials:} A shared drink, sealed with blood.  
\textbf{Effect:} Forge an extended pact of intimacy with another character. Both gain a persistent +1 die when acting in concert. Breaking faith immediately advances a 6-segment “Temptress’ Debt” clock.  
\textbf{Push It:} Pact can prevent one partner’s death (they revive at Harm 3), but the “Temptress’ Debt” clock advances 3 segments.  
\emph{Requires: Familiar + Codex + Tier III \ (\textit{Invoke:} \textbf{2 Boons}).}  
\emph{Obligation:} 7 segments.  

\paragraph{Rite of the Crimson Exchange (High, 14 XP)}  
\emph{Scene; Zone; No.}  
\textbf{Materials:} Two entwined tokens (rings, ribbons, locks of hair).  
\textbf{Effect:} Exchange conditions or consequences between two willing targets (e.g., swap Fatigue, a social Condition, or a boon).  
\textbf{Push It:} You may include an unwilling target, but Maestraea claims a piece of you in kind — a memory, a name, or a fragment of your soul.  
\emph{Requires: Familiar + Codex + Tier III \ (\textit{Invoke:} \textbf{2 Boons}).}  
\emph{Obligation:} 7 segments.  

% =========================
% Patron: The Sealed Gate — Boundaries, Jurisdiction, Closure
% =========================

\subsection{The Sealed Gate (Boundaries \& Closure)}
\label{patron:sealed-gate}
\textit{Lore.} You write borders into the world and prosecute trespass. Doors remember their true keepers; lines mean what you say they mean.

\paragraph{Patron's Gift (Imbuement).}
Once/scene as an action (cost: 1 Boon; requires \textbf{Thiasos}), touch an item to imbue it until scene end with \textbf{+1 Melee} and \textbf{+1 Tinker} (Thematic). \emph{Push It:} extend for one extra scene by marking \textbf{+1 Obligation}. Same-patron Gifts don’t stack; take the best. Dice bonuses respect the +3 cap.

\subsubsection*{Low Rites}

\paragraph{Seal the Threshold (Low)}
\emph{Duration: Scene; Range: Touch; Stacking: No.}\\
\textbf{Materials:} Brief sign across a door/line (chalk, wax, chain, sigil).\\
\textbf{Effect:} Mark a threshold. Crossing parties suffer worsened Position \emph{or} a brief stumble on first entry (Keeper’s choice by fiction).\\
\textbf{Invoke:} 1 action; mark \textbf{+1 Obligation}.\\
\textbf{Push It:} Treat the edge as difficult terrain or a snag (+1 Obligation).

\paragraph{Key’s Rebuke (Low)}
\emph{Duration: Instant; Range: Near; Stacking: No.}\\
\textbf{Materials:} A snapped ward-key gesture or clack of chain.\\
\textbf{Effect:} Flick a spectral hasp at a reaching hand/tool: stagger or disarm a target for one beat (fiction sets DV if contested).\\
\textbf{Invoke:} 1 action; mark \textbf{+1 Obligation}.\\
\textbf{Push It:} Also drop the object just beyond their reach (+1 Obligation).

\subsubsection*{Standard Rites}

\paragraph{Circle of Denial \texttt{[WARD]} (Standard)}
\emph{Duration: Scene; Range: Near; Stacking: No.}\\
\textbf{Materials:} Mark a ring/arc with sanctioned medium.\\
\textbf{Effect:} Outsiders crossing test \emph{DV = Cap}. On Hit: cross and add \emph{+DV} segments to their Leash/Exit Tally; on Partial: cross and add \emph{+1}; on Miss: fail to cross this beat.\\
\textbf{Invoke:} 1 action; mark \textbf{+1 Obligation}.\\
\textbf{Push It:} Fortify the circle (harder to bypass, clearer tells) (+1 Obligation).

\paragraph{Writ of Passage (Standard)}
\emph{Duration: Scene; Range: Near; Stacking: No.}\\
\textbf{Materials:} Spoken naming of the route; scribed pass-mark.\\
\textbf{Effect:} Designate a path as permitted. Allies on that route gain improved flow (Position/Effect bump or ignore one level of difficult terrain).\\
\textbf{Invoke:} 1 action; mark \textbf{+1 Obligation}.\\
\textbf{Push It:} Extend to one extra ally \emph{or} carry across one obstacle (+1 Obligation).

\subsubsection*{High Rite}

\paragraph{Banishment Knot \texttt{[BANISH]} (High)}
\emph{Duration: Instant; Range: Near; Stacking: No.}\\
\textbf{Materials:} Knot of cord/chain sealed with a gate-sigil.\\
\textbf{Effect:} Target a visible Outsider. Test \emph{DV = Cap}. On Hit: add \emph{+DV} segments to Leash/Exit Tally; on Partial: add \emph{+1}; if this fills, it acts to nature once, then departs.\\
\textbf{Invoke:} 1 action; mark \textbf{+1 Obligation} (some tables prefer +2—set at campaign start).\\
\textbf{Push It:} Strip one tether/anchor if present (+1 Obligation).

\paragraph{Invoker Access (Symbol Path).}
With a \textbf{Sealed Gate Symbol} (4 XP), perform any Rite above as a \emph{ritual} (Significant Time); completion always marks \textbf{+1 Obligation}. \textit{Crack the Seal} to cast instantly: set the Symbol to \emph{Compromised} and mark \textbf{+2 Obligation} (\textbf{+3} if High-Power). The Keeper may spend 1 on-theme SB immediately. Restore a \emph{Compromised} Symbol in Downtime via a fitting test (DV 3) or 1 XP.

\paragraph{Example Symbols (Sealed Gate).}
Lead sounder-weight engraved with abyssal curls; salt-etched iron chain link; sealed lockplate token.

\subsection{Umande, Dew-Mother of the Living Canopy (Growth \& Shelter)}
\begin{quote}\itshape
“Walk softly and you will hear the roots speaking of where to stand, when to bend, and whom to shelter.”
\end{quote}

\paragraph{Lore.}
Across the humid belts from the baobab-studded savannas to banyan-latticed monsoon forests, travelers whisper of \textbf{Umande}, the breath on the leaves at dawn. She is the pact of shade and water: a patient patron who measures worth in what you protect and what you prune. Her ministers are the unseen—lichen, mycelium, creepers—knitting harm back into wholeness, yet strangling cruelty where it clings too long. Umande favors wardens, foragers, midwives, and wayfarers who leave places better than they found them.

\paragraph{Patron’s Gift — Verdant Bond.}
Your sworn implement (staff, spear, billhook, kukri, or similar) knots itself in living vine and hard sap.
\begin{itemize}
  \item \textbf{Enchanted Weapon:} Your sworn implement counts as an enchanted melee weapon \textbf{+1 Melee}.
  \item \textbf{Thematic Skill:} You gain \textbf{+1 to Survival} while carrying any living sprig, seed, or leaf from your sanctum or a protected grove.
\end{itemize}

% =========================
% RITES (follow Ikasha style)
% =========================

\paragraph{Dawn’s Condensation (Low, 4 XP)} \emph{Scene; Touch/Self; Yes (resist only).}\\
\textbf{Materials:} A leaf cupping a few drops of clean water (or a wetted cloth).\\
\textbf{Effect:} Wick dew into flesh. Remove one minor Condition related to fatigue, heat, or dehydration \emph{or} grant +1 die to the next Survival or Medicine test this scene.\\
\textbf{Push It:} Also clear 1 SB of environmental stress (thirst/heat) for one ally within reach.\\
\textbf{Backlash (Life/Earth):} Skin chills and prunes; you suffer --1 die to Presence tests until you warm up.

\paragraph{Root-Weave Aegis (Low, 4 XP)} \emph{Scene; Near (close range); No.}\\
\textbf{Materials:} A twist of fiber, grass cord, or mycelial thread.\\
\textbf{Effect:} Sprout a knee-high barricade of roots and woven stems in a 2m arc. Gain \textbf{Cover} against ranged attacks and upgrade Position by one step for holds/defense while adjacent.\\
\textbf{Push It:} The first enemy crossing the weave is \textbf{Hindered} (movement penalty) this exchange.\\
\textbf{Backlash (Earth/Water):} Soil turns slick; you suffer --1 die on your next movement check this scene.

\paragraph{Whispering Mycelium (Standard, 8 XP)} \emph{Scene; Far (site-scale); Yes.}\\
\textbf{Materials:} A coin-sized fungus cap or a pinch of sporeprint.\\
\textbf{Effect:} Lay fingertips to soil or wood and commune with the hyphal web. Ask \textbf{two} questions about recent passage, injury, or decay within a circa 100m radius (GM answers truthfully but from the web’s perspective). Gain \textbf{advantage} on one track, forage, or concealment action tied to that intel this scene.\\
\textbf{Push It:} Ask a third question, or extend to 300m.\\
\textbf{Backlash (Life):} Your senses blur with rot; --1 die to non-Survival observation until you take a breath and steady yourself.

\paragraph{Green Path Unfurling (Standard, 8 XP)} \emph{Scene; Line of sight; Yes.}\\
\textbf{Materials:} A fresh tendril or vine.\\
\textbf{Effect:} Vegetation parts and cushions your passage. For the scene, you and up to \textbf{two} allies ignore minor difficult terrain in natural growth and gain +1 die to Stealth \emph{or} Athletics when moving through foliage.\\
\textbf{Push It:} Also \textbf{confound pursuit}: the first attempt to track you through this area is at --1 die.\\
\textbf{Backlash (Earth):} You leave telltale chlorophyll stains; tracking you in open ground gains +1 die until the next rest.

\paragraph{Rain-Caller’s Oath (Major, 12 XP)} \emph{Scene; Region (sky above); Yes.}\\
\textbf{Materials:} A hollow seedpod with three droplets of clean water inside.\\
\textbf{Effect:} Petition Umande to gather clouds or break them. Choose one:
\begin{itemize}
  \item \textbf{Summon Gentle Rains:} Over the next hour, steady rain falls; fires weaken, dust settles, and heat penalties are suppressed. Gain +1 die to Survival/Medicine to aid many.
  \item \textbf{Part the Canopy of Storms:} Winds ease and rainfall slackens to a workable drizzle; ranged penalties from heavy weather lessen by one step.
\end{itemize}
\textbf{Push It:} Shape the rain’s \emph{band}: center it on a location you can name within sight, sparing adjacent fields or focusing on a blaze.\\
\textbf{Backlash (Water/Life):} The sky takes its price; you suffer a \textbf{Drained} minor Condition (until warm food/rest), and plants nearby leach a hint of color.

\paragraph{Banyan’s Embrace (Major, 12 XP)} \emph{Instant; Near; Yes.}\\
\textbf{Materials:} A looped root or braided cord.\\
\textbf{Effect:} Living cords erupt to \textbf{Grapple} a single foe or to \textbf{Brace} a structure (choose one):
\begin{itemize}
  \item \textbf{Grapple:} Target must beat DV 4 (Body or Melee) to break free; on failure, they are Restrained until end of next exchange.
  \item \textbf{Brace:} A listing cart, palisade, or bridge segment holds; immediate collapse/damage is postponed; gain time to act.
\end{itemize}
\textbf{Push It:} Affect up to \textbf{two} adjacent targets/segments.\\
\textbf{Backlash (Earth):} Your limbs ache like old wood; --1 die to Melee checks until you stretch or rest.

\paragraph{Sanctum of the Green Court (Epic, 16 XP)} \emph{Ritual; Hours; Circle; Yes.}\\
\textbf{Materials:} Four living stakes (saplings) planted on the quarter points; a bowl of spring water; a woven crown of grass.\\
\textbf{Effect:} Consecrate a \textbf{Grove-Sanctum} (small campsite or shrine). While within, allies gain +1 die to \textbf{Survival} and \textbf{Medicine}; hostile entities of rot, blight, or open flame suffer --1 die to actions that harm living plants. Once per session within the sanctum, you may \textbf{negate} one minor environmental Complication (smoke, thirst, exposure).\\
\textbf{Push It:} Bind a \textbf{Vow}: name a creature, kin-group, or landmark under the Grove’s protection; the first direct harm against that Vow inside the sanctum immediately generates 1 SB for the GM \emph{and} grants you 1 Boon.\\
\textbf{Backlash (Life/Earth):} The grove remembers. If you break the Vow, Umande withholds dew: you cannot Push Umande rites until you atone (meaningful restoration or protection scene).

% Optional flavor epithets for different regions
\paragraph{Epithets.} In the eastern monsoon lands, she is \textit{She-Who-Hangs-The-Bridges} among banyan roots; on the savannas, \textit{Matron of Baobabs}; in mist forests, \textit{Mother-of-Dew} who writes omens on leaves at dawn.


\section{Patron Rivalries}
\label{sec:patron-rivalries}

Rivalries set expectations for tone and friction. Use them to color rulings, nudge Position, and guide how Story Beats (SB) land. In their home domains, a Patron’s work tends to start a step better in Position; in a rival’s, a step worse (Keeper’s call).

\begin{table}[h!]
  \centering
  \renewcommand{\arraystretch}{1.15}
  \begin{tabular}{@{}p{3.4cm}p{3.4cm}p{8.2cm}@{}}
    \toprule
    \textbf{Patron} & \textbf{Primary Rival} & \textbf{Friction in Play (one-line read)} \\
    \midrule
    Raéyn (Sea, Tides, Travel) & Khemesh (Abyssal Maw) & Tides vs. trench: navigation and passage thrive against dread and crushing depths. \\
    Khemesh (Abyssal Maw) & Raéyn (Sea, Tides, Travel) & Abyss unmoors charts: silence, pressure, and alien geometry devour routes. \\
    Sealed Gate (Boundaries, Closure) & The Traveler (Ways, Roads) & Keys vs. roads: jurisdiction and permits against detours and desire lines. \\
    The Traveler (Ways, Roads) & Sealed Gate (Boundaries, Closure) & Paths want to open; gates insist on form—who defines the threshold? \\
    The Witness (Truth, Revelation) & Mab (Glamour, Courts) & Revelation strips glamour; courtly masks fight to endure the gaze. \\
    Mab (Glamour, Courts) & The Witness (Truth, Revelation) & Mask and merriment contest the straight line of testimony. \\
    Ikasha (Shadow, Latent Potential) & The Witness (Truth, Revelation) & Hiding and hush vs. the unblinking eye. \\
    Mykkiel (Judgment, Writ) & Varnek Karn (Necromantic Archives) & Lawful writ and living order against bone-kept precedent and unfinished business. \\
    Varnek Karn (Necromantic Archives) & Oath of Light \& Flame (Dawn, Vows) & Memory of the dead resists purgation by vow and light. \\
    Oath of Light \& Flame (Dawn, Vows) & Khemesh (Abyssal Maw) & Consecrated dawn opposes abyssal hunger and despair. \\
    Sacred Geometry (Order, Pattern) & The Traveler (Ways, Fortune) & Perfect forms vs. opportunistic routes; measure vs. happenstance. \\
    Clockwork Monad (Iteration, Process) & The Traveler (Ways, Fortune) & Procedure and refinement versus improvisation and drift. \\
    Nidhoggr (Dreaming Antiquity) & Sacred Geometry (Order, Pattern) & Ancient, slumbering memory resists imposed, modern measures. \\
    \bottomrule
  \end{tabular}
  \caption{Primary Patron Rivalries and how they tend to color scenes.}
\end{table}
