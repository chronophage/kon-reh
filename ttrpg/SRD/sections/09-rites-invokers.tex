% =========================
% Fate's Edge SRD — Section 09: Rites, Invokers, and Symbols
% Include from main with: % =========================
% Fate's Edge SRD — Section 09: Rites, Invokers, and Symbols
% Include from main with: % =========================
% Fate's Edge SRD — Section 09: Rites, Invokers, and Symbols
% Include from main with: % =========================
% Fate's Edge SRD — Section 09: Rites, Invokers, and Symbols
% Include from main with: \input{patrons/09-rites-invokers.tex}
% Requires: \usepackage{float} for [H] tables
% =========================

\section{Rites, Invokers, and Symbols}
\label{sec:rites}

Magic in \textbf{Fate's Edge} expresses through three intertwined practices: \textbf{Rites} (oathbound authority), \textbf{Invocations} (symbolic ritual), and \textbf{Patron Pacts} (gifts and obligations). The rules below emphasize fiction-first play: consequences are Story Beats (SB) that prompt twists; numbers follow the story.

\subsection{Rites and Patrons (Runekeepers)}
\label{subsec:runekeepers}
Characters who bind themselves to a \emph{single} Patron and study that Patron's \textbf{Codex} are \textbf{Runekeepers}. Their magic is structured, immediate, and tied to service.

\begin{itemize}
  \item \textbf{One-Patron Rule.} A Runekeeper may be bound to \emph{only one} Patron at a time. This sharpens identity and keeps Obligation on a single ledger.
  \item \textbf{Thiasos (Familiar).} A circle, retinue, or emissary that grounds the pact in fiction. Required to access \emph{Patron's Gift}.
  \item \textbf{Codex.} The Patron's corpus of rites and precedents. Grants access to the Patron's Rites.
  \item \textbf{Invoke Rites.} A Runekeeper may Invoke a known Rite from their Patron as a \textbf{1 action} effect. On completion, mark \textbf{+1 Obligation} to that Patron. You may \emph{Push It} once per scene to amplify the effect, marking \textbf{+1 additional Obligation}.
\end{itemize}

\subsection{Invokers and Symbols}
\label{subsec:invokers}
Invokers relate to Patrons through consecrated \textbf{Symbols}: physical tokens that anchor names and permissions.

\begin{itemize}
  \item \textbf{Symbols (Minor Asset).} Each Symbol is keyed to one Patron; cost \textbf{4 XP}. You may own Symbols of different Patrons (one Symbol per Patron).
  \item \textbf{Ritual Invocation.} Display the Symbol and perform the Rite as a \emph{ritual} (Significant Time). Completion always marks \textbf{+1 Obligation} on that Rite's ledger.
  \item \textbf{Crack the Seal.} As part of an Invoker Rite, you may resolve the effect instantly by setting the Symbol to \emph{Compromised} and marking \textbf{+2 Obligation} (\textbf{+3} if High-Power). The Keeper may spend 1 on-theme SB immediately. The asset remains but is inert until restored.
  \item \textbf{Restore a Symbol.} 1 downtime action and a fitting test (DV 3 or by fiction). Success: \emph{Maintained}; shaky: returns \emph{Neglected}. Or spend \textbf{1 XP} to fully restore.
  \item \textbf{Display Requirement.} Symbols must be openly displayed for rituals. Hidden Symbols do not function.
\end{itemize}

\subsection{Casting and Free-Form Magic}
\label{subsec:casting}
Improvised casting is possible with the \textbf{Caster's Gift} Talent (\textbf{2 XP}). It is a \emph{backup toolkit}:
\begin{itemize}
  \item Small, local effects (typ. DV 2--3), fiction-first, colored by Elements and locus.
  \item Heavy control effects such as \texttt{[WARD]}, \texttt{[BANISH]}, or \texttt{[UNWARD]} require a printed Talent, Rite, or Spell result.
\end{itemize}

\subsection{Patron's Gift (Imbuements)}
\label{subsec:patrons-gift}
The pact may mark a devotee's tools with a short-lived boon aligned to the Patron's domain.

\paragraph{Requirements.}
\textbf{Thiasos (Familiar)} is required. Invoking the Gift costs \textbf{1 Boon}. A Codex is \emph{not} required for the Gift.

\paragraph{Activation and Duration.}
\begin{itemize}
  \item \textbf{Action:} 1 action to activate; \textbf{1/scene}.
  \item \textbf{Duration:} Scene. \emph{Push It:} extend for one additional scene by marking \textbf{+1 Obligation} to that Patron (max one Push per scene).
  \item \textbf{Range:} Touch (you must handle the item).
  \item \textbf{Stacking:} Gifts from the \emph{same Patron} do not stack; take the best active version. Dice bonuses respect the table's \textbf{+3 dice cap}.
\end{itemize}

\paragraph{Effect.}
Choose one held item you or an ally carries. Until scene end it grants:
\begin{itemize}
  \item \textbf{+1 Melee} (the item counts as a magical weapon), and
  \item \textbf{+1 Thematic} (a \emph{+1 die} to a fixed Skill tied to your Patron; see Table~\ref{tab:gift-thematic-map}). Apply only when the fiction clearly fits the Patron's sphere and how the item is used.
\end{itemize}

\paragraph{Runekeeper Clarification.}
A Runekeeper (one Patron + Codex) may Invoke Rites on-screen and use Patron's Gift if they also possess \textbf{Thiasos (Familiar)}. Codex alone does not grant the Gift. Symbols are optional for parley or omens and do not gate Runekeeper Invocation or the Gift.

\begin{table}[H]
\centering
\renewcommand{\arraystretch}{1.15}
\begin{tabular}{@{}p{3.8cm}p{3.8cm}p{7.5cm}@{}}
\toprule
\textbf{Patron} & \textbf{+1 Thematic Skill} & \textbf{Example Symbols} \\
\midrule
Ikasha (Shadow, Penumbra) & Stealth & Knot of black silk; soot-oil vial; fingerbone ring lacquered matte \\
Mykkiel (Judgment, Writ) & Command & Cold-iron seal matrix; parchment writ-tag; square rule stamped with code \\
The Witness (Truth, Revelation) & Notice & Obsidian eye pendant; silver mirror shard; wax seal-stamp with an open eye \\
Sealed Gate (Boundaries, Closure) & Tinker & Lead sounder-weight; iron chain link; sealed lockplate token \\
Raéyn (Storm, Tides) & Skirmish & Sea-glass disk; salt-crusted rope knot; vial of rainwater from three crossings \\
Khemesh (Abyss, Pressure) & Skirmish & Barnacle-bitten coin; abyssal-spiral lead weight; salt-etched iron chain \\
Mab (Glamour, Courts) & Persuade & Hawthorn thorn wrapped in silver; mirror shard with green felt; silk-lined acorn cup \\
Sacred Geometry (Perfect Forms) & Tinker & Brass heptagram compass; bone tablet with golden-ratio spiral; plumb-bob with proof in red thread \\
Clockwork Monad (Mechanism, Process) & Tinker & Gear tooth sealed in oil; mainspring coil; rivet stamped with forbidden numerals \\
Varnek Karn (Ossuary, Dominion of the Dead) & Command & Ossuary bead rosary; carved phalanx tally; fused bone-and-obsidian coin \\
Nidhoggr (Deep Earth, Rot) & Skirmish & Fossil tooth shard; dark river-stone; obsidian spindle with flaw \\
The Traveler (Ways, Roads) & Notice & Road-nail wrapped in thread; waystone pebble; brass compass missing its needle \\
Oath of Flame \& Light (Dawn, Vows) & Command & Cold-iron sun-stamp; vow-ring with sunrise and true name; ampoule of consecrated spark \\
\bottomrule
\end{tabular}
\caption{Patron's Gift: fixed Thematic Skill and example Symbols. Thematic bonuses apply only when the fiction matches the Patron’s domain and the item’s use. Symbols also serve Invokers as ritual anchors.}
\label{tab:gift-thematic-map}
\end{table}

\subsection{Specialization vs.\ Mixing}
\label{subsec:mixing}
Characters can mix paths (Summoner, Caster, Invoker, Runekeeper), but specialization is usually stronger and cleaner. Mixing increases upkeep (Obligation, Symbol state, Leash) and action congestion without guaranteed power gains. Let fiction guide choices: Story Beats are prompts to advance the scene, not punishments.

% Patron subsections (split files — keep filenames in the same directory)
\input{patrons/09a-patron-witness.tex}
\input{patrons/09b-patron-ikasha.tex}
\input{patrons/09c-patron-sacred-geometry.tex}
\input{patrons/09d-patron-inaea.tex}
\input{patrons/09e-patron-raeyn.tex}
\input{patrons/09f-patron-mykkiel.tex}
\input{patrons/09g-patron-khemesh.tex}
\input{patrons/09h-patron-mab.tex}
\input{patrons/09i-patron-traveler.tex}
\input{patrons/09j-patron-clockwork-monad.tex}
\input{patrons/09k-patron-varnek-karn.tex}
\input{patrons/09l-patron-nidhoggr.tex}
\input{patrons/09m-patron-oath-flame-light.tex}

% Requires: \usepackage{float} for [H] tables
% =========================

\section{Rites, Invokers, and Symbols}
\label{sec:rites}

Magic in \textbf{Fate's Edge} expresses through three intertwined practices: \textbf{Rites} (oathbound authority), \textbf{Invocations} (symbolic ritual), and \textbf{Patron Pacts} (gifts and obligations). The rules below emphasize fiction-first play: consequences are Story Beats (SB) that prompt twists; numbers follow the story.

\subsection{Rites and Patrons (Runekeepers)}
\label{subsec:runekeepers}
Characters who bind themselves to a \emph{single} Patron and study that Patron's \textbf{Codex} are \textbf{Runekeepers}. Their magic is structured, immediate, and tied to service.

\begin{itemize}
  \item \textbf{One-Patron Rule.} A Runekeeper may be bound to \emph{only one} Patron at a time. This sharpens identity and keeps Obligation on a single ledger.
  \item \textbf{Thiasos (Familiar).} A circle, retinue, or emissary that grounds the pact in fiction. Required to access \emph{Patron's Gift}.
  \item \textbf{Codex.} The Patron's corpus of rites and precedents. Grants access to the Patron's Rites.
  \item \textbf{Invoke Rites.} A Runekeeper may Invoke a known Rite from their Patron as a \textbf{1 action} effect. On completion, mark \textbf{+1 Obligation} to that Patron. You may \emph{Push It} once per scene to amplify the effect, marking \textbf{+1 additional Obligation}.
\end{itemize}

\subsection{Invokers and Symbols}
\label{subsec:invokers}
Invokers relate to Patrons through consecrated \textbf{Symbols}: physical tokens that anchor names and permissions.

\begin{itemize}
  \item \textbf{Symbols (Minor Asset).} Each Symbol is keyed to one Patron; cost \textbf{4 XP}. You may own Symbols of different Patrons (one Symbol per Patron).
  \item \textbf{Ritual Invocation.} Display the Symbol and perform the Rite as a \emph{ritual} (Significant Time). Completion always marks \textbf{+1 Obligation} on that Rite's ledger.
  \item \textbf{Crack the Seal.} As part of an Invoker Rite, you may resolve the effect instantly by setting the Symbol to \emph{Compromised} and marking \textbf{+2 Obligation} (\textbf{+3} if High-Power). The Keeper may spend 1 on-theme SB immediately. The asset remains but is inert until restored.
  \item \textbf{Restore a Symbol.} 1 downtime action and a fitting test (DV 3 or by fiction). Success: \emph{Maintained}; shaky: returns \emph{Neglected}. Or spend \textbf{1 XP} to fully restore.
  \item \textbf{Display Requirement.} Symbols must be openly displayed for rituals. Hidden Symbols do not function.
\end{itemize}

\subsection{Casting and Free-Form Magic}
\label{subsec:casting}
Improvised casting is possible with the \textbf{Caster's Gift} Talent (\textbf{2 XP}). It is a \emph{backup toolkit}:
\begin{itemize}
  \item Small, local effects (typ. DV 2--3), fiction-first, colored by Elements and locus.
  \item Heavy control effects such as \texttt{[WARD]}, \texttt{[BANISH]}, or \texttt{[UNWARD]} require a printed Talent, Rite, or Spell result.
\end{itemize}

\subsection{Patron's Gift (Imbuements)}
\label{subsec:patrons-gift}
The pact may mark a devotee's tools with a short-lived boon aligned to the Patron's domain.

\paragraph{Requirements.}
\textbf{Thiasos (Familiar)} is required. Invoking the Gift costs \textbf{1 Boon}. A Codex is \emph{not} required for the Gift.

\paragraph{Activation and Duration.}
\begin{itemize}
  \item \textbf{Action:} 1 action to activate; \textbf{1/scene}.
  \item \textbf{Duration:} Scene. \emph{Push It:} extend for one additional scene by marking \textbf{+1 Obligation} to that Patron (max one Push per scene).
  \item \textbf{Range:} Touch (you must handle the item).
  \item \textbf{Stacking:} Gifts from the \emph{same Patron} do not stack; take the best active version. Dice bonuses respect the table's \textbf{+3 dice cap}.
\end{itemize}

\paragraph{Effect.}
Choose one held item you or an ally carries. Until scene end it grants:
\begin{itemize}
  \item \textbf{+1 Melee} (the item counts as a magical weapon), and
  \item \textbf{+1 Thematic} (a \emph{+1 die} to a fixed Skill tied to your Patron; see Table~\ref{tab:gift-thematic-map}). Apply only when the fiction clearly fits the Patron's sphere and how the item is used.
\end{itemize}

\paragraph{Runekeeper Clarification.}
A Runekeeper (one Patron + Codex) may Invoke Rites on-screen and use Patron's Gift if they also possess \textbf{Thiasos (Familiar)}. Codex alone does not grant the Gift. Symbols are optional for parley or omens and do not gate Runekeeper Invocation or the Gift.

\begin{table}[H]
\centering
\renewcommand{\arraystretch}{1.15}
\begin{tabular}{@{}p{3.8cm}p{3.8cm}p{7.5cm}@{}}
\toprule
\textbf{Patron} & \textbf{+1 Thematic Skill} & \textbf{Example Symbols} \\
\midrule
Ikasha (Shadow, Penumbra) & Stealth & Knot of black silk; soot-oil vial; fingerbone ring lacquered matte \\
Mykkiel (Judgment, Writ) & Command & Cold-iron seal matrix; parchment writ-tag; square rule stamped with code \\
The Witness (Truth, Revelation) & Notice & Obsidian eye pendant; silver mirror shard; wax seal-stamp with an open eye \\
Sealed Gate (Boundaries, Closure) & Tinker & Lead sounder-weight; iron chain link; sealed lockplate token \\
Raéyn (Storm, Tides) & Skirmish & Sea-glass disk; salt-crusted rope knot; vial of rainwater from three crossings \\
Khemesh (Abyss, Pressure) & Skirmish & Barnacle-bitten coin; abyssal-spiral lead weight; salt-etched iron chain \\
Mab (Glamour, Courts) & Persuade & Hawthorn thorn wrapped in silver; mirror shard with green felt; silk-lined acorn cup \\
Sacred Geometry (Perfect Forms) & Tinker & Brass heptagram compass; bone tablet with golden-ratio spiral; plumb-bob with proof in red thread \\
Clockwork Monad (Mechanism, Process) & Tinker & Gear tooth sealed in oil; mainspring coil; rivet stamped with forbidden numerals \\
Varnek Karn (Ossuary, Dominion of the Dead) & Command & Ossuary bead rosary; carved phalanx tally; fused bone-and-obsidian coin \\
Nidhoggr (Deep Earth, Rot) & Skirmish & Fossil tooth shard; dark river-stone; obsidian spindle with flaw \\
The Traveler (Ways, Roads) & Notice & Road-nail wrapped in thread; waystone pebble; brass compass missing its needle \\
Oath of Flame \& Light (Dawn, Vows) & Command & Cold-iron sun-stamp; vow-ring with sunrise and true name; ampoule of consecrated spark \\
\bottomrule
\end{tabular}
\caption{Patron's Gift: fixed Thematic Skill and example Symbols. Thematic bonuses apply only when the fiction matches the Patron’s domain and the item’s use. Symbols also serve Invokers as ritual anchors.}
\label{tab:gift-thematic-map}
\end{table}

\subsection{Specialization vs.\ Mixing}
\label{subsec:mixing}
Characters can mix paths (Summoner, Caster, Invoker, Runekeeper), but specialization is usually stronger and cleaner. Mixing increases upkeep (Obligation, Symbol state, Leash) and action congestion without guaranteed power gains. Let fiction guide choices: Story Beats are prompts to advance the scene, not punishments.

% Patron subsections (split files — keep filenames in the same directory)
\input{patrons/09a-patron-witness.tex}
\input{patrons/09b-patron-ikasha.tex}
\input{patrons/09c-patron-sacred-geometry.tex}
\input{patrons/09d-patron-inaea.tex}
% --- Patron: Raéyn, Keeper of the Sealed Gate (Thresholds & Warding) ---
\subsection{Raéyn, Keeper of the Sealed Gate (Thresholds \& Warding)}
\textit{Lore.} Raéyn is invoked at every border. His gift is the lock that preserves and the seal that keeps chaos at bay.

\begin{quote}
The door is not shut until Raéyn’s mark is traced.
\end{quote}

\paragraph{Seal the Latch (Low, 4 XP)} \emph{Action; Near; Yes (object).}
\textbf{Materials:} Trace a key-sign with ash/chalk.\\
\textbf{Effect:} Secure a container/door/gate. Attempts to open require intruder to generate \textbf{+1 SB} (Spades/Clubs).\\
\textbf{Push It:} First touch flares (noise/flash), revealing the attempt.\\
\emph{Requires: Familiar \ (\textit{Invoke:} 1 Boon).}

\paragraph{Rite of the Quiet Gate (Low, 5 XP)} \emph{Scene; Near; No.}
\textbf{Materials:} A key turned backwards in a lock.\\
\textbf{Effect:} Warded threshold. Passing uninvited imposes \(-1\) die on the trespasser’s next action.\\
\textbf{Push It:} The ward whispers intruder’s purpose in one phrase.\\
\emph{Requires: Familiar \ (\textit{Invoke:} 1 Boon).}

\paragraph{Mark of Raéyn (Standard, 8 XP)} \emph{Scene; Near; No.}
\textbf{Materials:} A drawn circle or sigil.\\
\textbf{Effect:} Protect a room/wagon. Crossing without consent generates \textbf{1 SB} (suit by GM).\\
\textbf{Push It:} Suppress one minor spell crossing; when it collapses, mark \textbf{1 SB (Hearts)} backlash.\\
\emph{Requires: Familiar + Codex \ (\textit{Invoke:} 1 Boon).}

\paragraph{Rite of the Sealed Mouth (Standard, 7 XP)} \emph{Scene; Near; No.}
\textbf{Materials:} Thread tied across lips, then removed.\\
\textbf{Effect:} Choose one lost channel: speech, script, or gesture. Within, it fails for the scene.\\
\textbf{Push It:} Suppress all three, but you are muted until scene end.\\
\emph{Requires: Familiar + Codex \ (\textit{Invoke:} 1 Boon).}

\paragraph{Rite of the Sealed Gate \textnormal{[WARD]} (High, 11 XP)} \emph{Scene; Near; No.}
\textbf{Materials:} Iron-powder circle, locked with a key.\\
\textbf{Effect:} Impassable boundary (~10 ft). Forcing entry inflicts \textbf{2 SB} (Clubs/Spades) on the intruder.\\
\textbf{Push It:} Enclose a chamber/courtyard. Each additional hour risks \textbf{1 SB (Diamonds)} toward omen/strain.\\
\emph{Requires: Familiar + Codex + Tier III \ (\textit{Invoke:} \textbf{2 Boons}).}\\
\emph{Obligation:} 7 segments.

\input{patrons/09f-patron-mykkiel.tex}
% --- Patron: Khemesh, the Kraken Lord (Depths & Inevitable Power) ---
\subsection{Khemesh, the Kraken Lord (Depths \& Inevitable Power)}
\textit{Lore.} Khemesh is the unseen weight beneath the waves—the patience of the abyss and the certainty of drowning.

\begin{quote}
In the end, all things sink. The depths remember everything.
\end{quote}

\paragraph{Grasp of the Abyss (Low, 4 XP)} \emph{Action; Near; Yes (single target).}
\textbf{Materials:} A knot tied and dropped in water.\\
\textbf{Effect:} Phantom tentacles clutch; target suffers \(-1\) die next action or is briefly held in place.\\
\textbf{Push It:} Inflict 1 Harm (Crush); you gain Fatigue 1 (phantom drowning).\\
\emph{Requires: Familiar \ (\textit{Invoke:} 1 Boon).}

\paragraph{Rite of the Drowning Silence (Low, 5 XP)} \emph{Scene; Zone; No.}
\textbf{Materials:} A seashell filled with water, then broken.\\
\textbf{Effect:} Muffle sound; rolls relying on voice/noise suffer \(-1\) die; Stealth gains +1 die.\\
\textbf{Push It:} Silence deepens to pressure; others resist or suffer Fatigue 1.\\
\emph{Requires: Familiar \ (\textit{Invoke:} 1 Boon).}

\paragraph{Weight of the Deep (Standard, 7 XP)} \emph{Scene; Near; No.}
\textbf{Materials:} A stone dropped into black water.\\
\textbf{Effect:} Target moves as if burdened; \(-1\) die to physical actions for the scene.\\
\textbf{Push It:} Briefly pin the target to deck/ground; mark \textbf{1 SB (Clubs)} from collateral strain.\\
\emph{Requires: Familiar + Codex \ (\textit{Invoke:} 1 Boon).}

\paragraph{Rite of the Crushing Coil (Standard, 9 XP)} \emph{Instant; Near; No.}
\textbf{Materials:} Rope/chain wrapped around your arm.\\
\textbf{Effect:} Spectral tentacle lashes or constricts: deal 2 Harm (Crush) to one target in Near.\\
\textbf{Push It:} Instead immobilize for one beat; you suffer Fatigue 1.\\
\emph{Requires: Familiar + Codex \ (\textit{Invoke:} 1 Boon).}

\paragraph{Abyssal Dominion (High, 12 XP)} \emph{Scene; Zone; No.}
\textbf{Materials:} Seawater poured in a circle.\\
\textbf{Effect:} Unseen tides hinder foes: enemies in Zone \(-1\) die to move/attack; allies \textbf{+1 die} to resist.\\
\textbf{Push It:} One massive tentacle strikes (3 Harm) once; draws the Keeper’s tide: \textbf{2 SB (Clubs/Spades)}.\\
\emph{Requires: Familiar + Codex + Tier III \ (\textit{Invoke:} \textbf{2 Boons}).}\\
\emph{Obligation:} 7 segments.

% --- Patron: Mab, Queen of Courts (Glamour & Bargain) ---
\subsection{Mab, Queen of Courts (Glamour \& Bargain)}
\textit{Lore.} The blush of truth, the dagger of etiquette, the smile that writes debts in perfume. Mab rules where desire dresses itself as courtesy.

\begin{quote}
Bend, don’t bow. Smile, don’t promise.
\end{quote}

\paragraph{Courtly Guise \textnormal{[VEIL]} (Low, 4 XP)} \emph{Action; Self; Yes (social only).}
\textbf{Materials:} Pin a sprig of green or silver thread.\\
\textbf{Effect:} Subtle glamour: \textbf{+1 die} to Persuade/Sway in refined settings; you appear as expected rank/guest.\\
\textbf{Push It:} Also mask one minor tell; the first piercing question in the scene generates \textbf{1 SB (Hearts)}.\\
\emph{Requires: Familiar \ (\textit{Invoke:} 1 Boon).}

\paragraph{Token of Favor (Low, 5 XP)} \emph{Scene; Near; No.}
\textbf{Materials:} A ribbon or ring bestowed.\\
\textbf{Effect:} Grant an ally \textbf{+1 die} to one social action against onlookers who recognize your favor; you gain \textbf{+1 effect} to support.\\
\textbf{Push It:} The token also chills a heckler (one beat of hesitation), but you mark \emph{Exposure +1}.\\
\emph{Requires: Familiar \ (\textit{Invoke:} 1 Boon).}

\paragraph{Mirror of Motives (Standard, 7 XP)} \emph{Action; Near; No.}
\textbf{Materials:} A polished shard or compact mirror.\\
\textbf{Effect:} Ask one pointed question about an NPC’s \emph{immediate} social goal; Keeper answers truthfully or with a strong tell. Gain \textbf{+1 die} to exploit it this scene.\\
\textbf{Push It:} Also expose a concealed slight or insult that matters to them, creating \textbf{1 SB (Hearts)} on that target.\\
\emph{Requires: Familiar + Codex \ (\textit{Invoke:} 1 Boon).}

\paragraph{The Price Agreed \textnormal{[OATH]} (Standard, 8 XP)} \emph{Scene; Near; No.}
\textbf{Materials:} Exchange a token of equal apparent value.\\
\textbf{Effect:} Bind a petty bargain (favor-for-favor). Breach forces \textbf{1 SB (Hearts or Diamonds)} on the breaker and stains their reputation locally this arc.\\
\textbf{Push It:} Sweeten terms with a minor boon (\(+1\) die once to the beneficiary), but you take \textbf{1 SB (Hearts)} if they later breach.\\
\emph{Requires: Familiar + Codex \ (\textit{Invoke:} 1 Boon).}

\paragraph{Sovereign Glamour \textnormal{[VEIL][REVEAL]} (High, 11 XP)} \emph{Scene; Zone; No.}
\textbf{Materials:} A circle of green felt or silk.\\
\textbf{Effect:} Establish Court: allies in Zone gain \textbf{+1 die} to social actions; crude threats suffer \(-1\) die. Once, peel one disguise/illusion in Zone.\\
\textbf{Push It:} Name a \emph{Court Law} (e.g., no drawn steel): first violation \emph{forces 2 SB} on the violator.\\
\emph{Requires: Familiar + Codex + Tier III \ (\textit{Invoke:} \textbf{2 Boons}).}\\
\emph{Obligation:} 6 segments.

\input{patrons/09i-patron-traveler.tex}
% --- Patron: The Clockwork Monad (Mechanism & Process) ---
\subsection{The Clockwork Monad (Mechanism \& Process)}
\textit{Lore.} Gears remember the plan even when their makers forget. The Monad prizes precision, iteration, and the beauty of mechanisms that keep their word.

\begin{quote}
Every tooth matters. Especially the ones you cannot see.
\end{quote}

\paragraph{Calibrated Touch (Low, 4 XP)} \emph{Action; Self; Yes (tinkering).}
\textbf{Materials:} A single oiled gear tooth.\\
\textbf{Effect:} \textbf{+1 die} to repair, set, or disarm precise mechanisms; re-roll one \texttt{1} on a Tinker action.\\
\textbf{Push It:} Guarantee no collateral on a simple device, but you take \textbf{1 SB (Clubs)} if rushed.\\
\emph{Requires: Familiar \ (\textit{Invoke:} 1 Boon).}

\paragraph{Process Lock (Low, 5 XP)} \emph{Scene; Near; No.}
\textbf{Materials:} A drop of red oil.\\
\textbf{Effect:} Name a process (reload, dispatch, raise alarm). First attempt to perform it in-scene suffers \(-1\) die as steps “stick.”\\
\textbf{Push It:} The second attempt also hesitates (one beat), but the third surges forward and \emph{forces 1 SB (Spades)}.\\
\emph{Requires: Familiar \ (\textit{Invoke:} 1 Boon).}

\paragraph{Iterative Advantage (Standard, 7 XP)} \emph{Action; Self/Ally; No.}
\textbf{Materials:} Notched tally on metal.\\
\textbf{Effect:} On a repeated action this scene, grant \textbf{+2 dice} or \textbf{+1 effect}.\\
\textbf{Push It:} Bank a follow-up \textbf{+1 die} for the same action later this scene; if unused, generate \textbf{1 SB (Diamonds)} as unused potential jams something.\\
\emph{Requires: Familiar + Codex \ (\textit{Invoke:} 1 Boon).}

\paragraph{Suppress Malfunction \textnormal{[DISPEL]} (Standard, 8 XP)} \emph{Instant; Near; No.}
\textbf{Materials:} Mainspring coil released.\\
\textbf{Effect:} End/suppress an ongoing \emph{mechanical or procedural} failure/curse (e.g., jamming jam, recursive alarm, fatigue spiral tick). DV by fiction.\\
\textbf{Push It:} Also clear one related clock segment; create \textbf{1 SB (Clubs)} as pressure shifts elsewhere.\\
\emph{Requires: Familiar + Codex \ (\textit{Invoke:} 1 Boon).}

\paragraph{Prime the Engine (High, 12 XP)} \emph{Scene; Zone; No.}
\textbf{Materials:} A ring of interlocked cogs.\\
\textbf{Effect:} Allies in Zone add a “\emph{Process Buff}”: first action that repeats in-scene gets an automatic Position bump \emph{or} upgrades Effect.\\
\textbf{Push It:} Also seize timing: once, reorder two adjacent beats; mark \textbf{2 SB (Diamonds)} as the world protests.\\
\emph{Requires: Familiar + Codex + Tier III \ (\textit{Invoke:} \textbf{2 Boons}).}\\
\emph{Obligation:} 7 segments.

\input{patrons/09k-patron-varnek-karn.tex}
% --- Patron: Nidhoggr (Deep Earth & Rot) ---
\subsection{Nidhoggr (Deep Earth \& Rot)}
\textit{Lore.} What grows must fall. What stands must sink. Nidhoggr is patient ruin—roots splitting stone, rot reclaiming pride.

\begin{quote}
Lie down. The earth knows what to make of you.
\end{quote}

\paragraph{Root-Grip (Low, 4 XP)} \emph{Action; Near; Yes (grounded target).}
\textbf{Materials:} A seed pressed into soil.\\
\textbf{Effect:} Roots or sinkholes hinder a grounded foe; \(-1\) die on their next move/strike.\\
\textbf{Push It:} Brief immobilize (one beat) if they stand on soil/wood; you take \textbf{1 SB (Clubs)} as structures complain.\\
\emph{Requires: Familiar \ (\textit{Invoke:} 1 Boon).}

\paragraph{Rot’s Kiss (Low, 5 XP)} \emph{Scene; Touch; No.}
\textbf{Materials:} A smear of compost or mold.\\
\textbf{Effect:} Tag gear or cover: first use this scene is \emph{Limited Effect} as rot softens it.\\
\textbf{Push It:} Also chip 1 integrity from mundane barriers/props touched in the scene.\\
\emph{Requires: Familiar \ (\textit{Invoke:} 1 Boon).}

\paragraph{Burden of Old Stone (Standard, 7 XP)} \emph{Scene; Near; No.}
\textbf{Materials:} A river-stone that has never seen sunlight.\\
\textbf{Effect:} Drop morale and vigor: \(-1\) die to strenuous actions for enemies who can feel the ground; allies gain steadiness (+1 die to resist knockdown).\\
\textbf{Push It:} Collapse a minor lintel/ledge to reshape cover; creates \textbf{1 SB (Clubs)}.\\
\emph{Requires: Familiar + Codex \ (\textit{Invoke:} 1 Boon).}

\paragraph{Devour the Pillar (Standard, 9 XP)} \emph{Instant; Near; No.}
\textbf{Materials:} Obsidian spindle with a hairline flaw.\\
\textbf{Effect:} Target a single support, axle, or keystone; reduce its integrity drastically (Keeper: equivalent to a big bite out of a [4] barrier).\\
\textbf{Push It:} Chain reaction threatens another support in Zone; you must choose to save someone or secure footing (\textbf{1 SB (Spades/Clubs)}).\\
\emph{Requires: Familiar + Codex \ (\textit{Invoke:} 1 Boon).}

\paragraph{Grave-Quiet Dominion (High, 12 XP)} \emph{Scene; Zone; No.}
\textbf{Materials:} Fossil tooth shard and grave loam.\\
\textbf{Effect:} The ground asserts itself: enemies in Zone suffer \(-1\) die to rush/charge/leap; all falls worsen by one step; once, open a sink to isolate a foe.\\
\textbf{Push It:} Call the slow crush: \emph{force 2 SB (Clubs/Spades)} as supports, beams, or roots shift at the worst time.\\
\emph{Requires: Familiar + Codex + Tier III \ (\textit{Invoke:} \textbf{2 Boons}).}\\
\emph{Obligation:} 7 segments.

\input{patrons/09m-patron-oath-flame-light.tex}

% Requires: \usepackage{float} for [H] tables
% =========================

\section{Rites, Invokers, and Symbols}
\label{sec:rites}

Magic in \textbf{Fate's Edge} expresses through three intertwined practices: \textbf{Rites} (oathbound authority), \textbf{Invocations} (symbolic ritual), and \textbf{Patron Pacts} (gifts and obligations). The rules below emphasize fiction-first play: consequences are Story Beats (SB) that prompt twists; numbers follow the story.

\subsection{Rites and Patrons (Runekeepers)}
\label{subsec:runekeepers}
Characters who bind themselves to a \emph{single} Patron and study that Patron's \textbf{Codex} are \textbf{Runekeepers}. Their magic is structured, immediate, and tied to service.

\begin{itemize}
  \item \textbf{One-Patron Rule.} A Runekeeper may be bound to \emph{only one} Patron at a time. This sharpens identity and keeps Obligation on a single ledger.
  \item \textbf{Thiasos (Familiar).} A circle, retinue, or emissary that grounds the pact in fiction. Required to access \emph{Patron's Gift}.
  \item \textbf{Codex.} The Patron's corpus of rites and precedents. Grants access to the Patron's Rites.
  \item \textbf{Invoke Rites.} A Runekeeper may Invoke a known Rite from their Patron as a \textbf{1 action} effect. On completion, mark \textbf{+1 Obligation} to that Patron. You may \emph{Push It} once per scene to amplify the effect, marking \textbf{+1 additional Obligation}.
\end{itemize}

\subsection{Invokers and Symbols}
\label{subsec:invokers}
Invokers relate to Patrons through consecrated \textbf{Symbols}: physical tokens that anchor names and permissions.

\begin{itemize}
  \item \textbf{Symbols (Minor Asset).} Each Symbol is keyed to one Patron; cost \textbf{4 XP}. You may own Symbols of different Patrons (one Symbol per Patron).
  \item \textbf{Ritual Invocation.} Display the Symbol and perform the Rite as a \emph{ritual} (Significant Time). Completion always marks \textbf{+1 Obligation} on that Rite's ledger.
  \item \textbf{Crack the Seal.} As part of an Invoker Rite, you may resolve the effect instantly by setting the Symbol to \emph{Compromised} and marking \textbf{+2 Obligation} (\textbf{+3} if High-Power). The Keeper may spend 1 on-theme SB immediately. The asset remains but is inert until restored.
  \item \textbf{Restore a Symbol.} 1 downtime action and a fitting test (DV 3 or by fiction). Success: \emph{Maintained}; shaky: returns \emph{Neglected}. Or spend \textbf{1 XP} to fully restore.
  \item \textbf{Display Requirement.} Symbols must be openly displayed for rituals. Hidden Symbols do not function.
\end{itemize}

\subsection{Casting and Free-Form Magic}
\label{subsec:casting}
Improvised casting is possible with the \textbf{Caster's Gift} Talent (\textbf{2 XP}). It is a \emph{backup toolkit}:
\begin{itemize}
  \item Small, local effects (typ. DV 2--3), fiction-first, colored by Elements and locus.
  \item Heavy control effects such as \texttt{[WARD]}, \texttt{[BANISH]}, or \texttt{[UNWARD]} require a printed Talent, Rite, or Spell result.
\end{itemize}

\subsection{Patron's Gift (Imbuements)}
\label{subsec:patrons-gift}
The pact may mark a devotee's tools with a short-lived boon aligned to the Patron's domain.

\paragraph{Requirements.}
\textbf{Thiasos (Familiar)} is required. Invoking the Gift costs \textbf{1 Boon}. A Codex is \emph{not} required for the Gift.

\paragraph{Activation and Duration.}
\begin{itemize}
  \item \textbf{Action:} 1 action to activate; \textbf{1/scene}.
  \item \textbf{Duration:} Scene. \emph{Push It:} extend for one additional scene by marking \textbf{+1 Obligation} to that Patron (max one Push per scene).
  \item \textbf{Range:} Touch (you must handle the item).
  \item \textbf{Stacking:} Gifts from the \emph{same Patron} do not stack; take the best active version. Dice bonuses respect the table's \textbf{+3 dice cap}.
\end{itemize}

\paragraph{Effect.}
Choose one held item you or an ally carries. Until scene end it grants:
\begin{itemize}
  \item \textbf{+1 Melee} (the item counts as a magical weapon), and
  \item \textbf{+1 Thematic} (a \emph{+1 die} to a fixed Skill tied to your Patron; see Table~\ref{tab:gift-thematic-map}). Apply only when the fiction clearly fits the Patron's sphere and how the item is used.
\end{itemize}

\paragraph{Runekeeper Clarification.}
A Runekeeper (one Patron + Codex) may Invoke Rites on-screen and use Patron's Gift if they also possess \textbf{Thiasos (Familiar)}. Codex alone does not grant the Gift. Symbols are optional for parley or omens and do not gate Runekeeper Invocation or the Gift.

\begin{table}[H]
\centering
\renewcommand{\arraystretch}{1.15}
\begin{tabular}{@{}p{3.8cm}p{3.8cm}p{7.5cm}@{}}
\toprule
\textbf{Patron} & \textbf{+1 Thematic Skill} & \textbf{Example Symbols} \\
\midrule
Ikasha (Shadow, Penumbra) & Stealth & Knot of black silk; soot-oil vial; fingerbone ring lacquered matte \\
Mykkiel (Judgment, Writ) & Command & Cold-iron seal matrix; parchment writ-tag; square rule stamped with code \\
The Witness (Truth, Revelation) & Notice & Obsidian eye pendant; silver mirror shard; wax seal-stamp with an open eye \\
Sealed Gate (Boundaries, Closure) & Tinker & Lead sounder-weight; iron chain link; sealed lockplate token \\
Raéyn (Storm, Tides) & Skirmish & Sea-glass disk; salt-crusted rope knot; vial of rainwater from three crossings \\
Khemesh (Abyss, Pressure) & Skirmish & Barnacle-bitten coin; abyssal-spiral lead weight; salt-etched iron chain \\
Mab (Glamour, Courts) & Persuade & Hawthorn thorn wrapped in silver; mirror shard with green felt; silk-lined acorn cup \\
Sacred Geometry (Perfect Forms) & Tinker & Brass heptagram compass; bone tablet with golden-ratio spiral; plumb-bob with proof in red thread \\
Clockwork Monad (Mechanism, Process) & Tinker & Gear tooth sealed in oil; mainspring coil; rivet stamped with forbidden numerals \\
Varnek Karn (Ossuary, Dominion of the Dead) & Command & Ossuary bead rosary; carved phalanx tally; fused bone-and-obsidian coin \\
Nidhoggr (Deep Earth, Rot) & Skirmish & Fossil tooth shard; dark river-stone; obsidian spindle with flaw \\
The Traveler (Ways, Roads) & Notice & Road-nail wrapped in thread; waystone pebble; brass compass missing its needle \\
Oath of Flame \& Light (Dawn, Vows) & Command & Cold-iron sun-stamp; vow-ring with sunrise and true name; ampoule of consecrated spark \\
\bottomrule
\end{tabular}
\caption{Patron's Gift: fixed Thematic Skill and example Symbols. Thematic bonuses apply only when the fiction matches the Patron’s domain and the item’s use. Symbols also serve Invokers as ritual anchors.}
\label{tab:gift-thematic-map}
\end{table}

\subsection{Specialization vs.\ Mixing}
\label{subsec:mixing}
Characters can mix paths (Summoner, Caster, Invoker, Runekeeper), but specialization is usually stronger and cleaner. Mixing increases upkeep (Obligation, Symbol state, Leash) and action congestion without guaranteed power gains. Let fiction guide choices: Story Beats are prompts to advance the scene, not punishments.

% Patron subsections (split files — keep filenames in the same directory)
\input{patrons/09a-patron-witness.tex}
\input{patrons/09b-patron-ikasha.tex}
\input{patrons/09c-patron-sacred-geometry.tex}
\input{patrons/09d-patron-inaea.tex}
% --- Patron: Raéyn, Keeper of the Sealed Gate (Thresholds & Warding) ---
\subsection{Raéyn, Keeper of the Sealed Gate (Thresholds \& Warding)}
\textit{Lore.} Raéyn is invoked at every border. His gift is the lock that preserves and the seal that keeps chaos at bay.

\begin{quote}
The door is not shut until Raéyn’s mark is traced.
\end{quote}

\paragraph{Seal the Latch (Low, 4 XP)} \emph{Action; Near; Yes (object).}
\textbf{Materials:} Trace a key-sign with ash/chalk.\\
\textbf{Effect:} Secure a container/door/gate. Attempts to open require intruder to generate \textbf{+1 SB} (Spades/Clubs).\\
\textbf{Push It:} First touch flares (noise/flash), revealing the attempt.\\
\emph{Requires: Familiar \ (\textit{Invoke:} 1 Boon).}

\paragraph{Rite of the Quiet Gate (Low, 5 XP)} \emph{Scene; Near; No.}
\textbf{Materials:} A key turned backwards in a lock.\\
\textbf{Effect:} Warded threshold. Passing uninvited imposes \(-1\) die on the trespasser’s next action.\\
\textbf{Push It:} The ward whispers intruder’s purpose in one phrase.\\
\emph{Requires: Familiar \ (\textit{Invoke:} 1 Boon).}

\paragraph{Mark of Raéyn (Standard, 8 XP)} \emph{Scene; Near; No.}
\textbf{Materials:} A drawn circle or sigil.\\
\textbf{Effect:} Protect a room/wagon. Crossing without consent generates \textbf{1 SB} (suit by GM).\\
\textbf{Push It:} Suppress one minor spell crossing; when it collapses, mark \textbf{1 SB (Hearts)} backlash.\\
\emph{Requires: Familiar + Codex \ (\textit{Invoke:} 1 Boon).}

\paragraph{Rite of the Sealed Mouth (Standard, 7 XP)} \emph{Scene; Near; No.}
\textbf{Materials:} Thread tied across lips, then removed.\\
\textbf{Effect:} Choose one lost channel: speech, script, or gesture. Within, it fails for the scene.\\
\textbf{Push It:} Suppress all three, but you are muted until scene end.\\
\emph{Requires: Familiar + Codex \ (\textit{Invoke:} 1 Boon).}

\paragraph{Rite of the Sealed Gate \textnormal{[WARD]} (High, 11 XP)} \emph{Scene; Near; No.}
\textbf{Materials:} Iron-powder circle, locked with a key.\\
\textbf{Effect:} Impassable boundary (~10 ft). Forcing entry inflicts \textbf{2 SB} (Clubs/Spades) on the intruder.\\
\textbf{Push It:} Enclose a chamber/courtyard. Each additional hour risks \textbf{1 SB (Diamonds)} toward omen/strain.\\
\emph{Requires: Familiar + Codex + Tier III \ (\textit{Invoke:} \textbf{2 Boons}).}\\
\emph{Obligation:} 7 segments.

\input{patrons/09f-patron-mykkiel.tex}
% --- Patron: Khemesh, the Kraken Lord (Depths & Inevitable Power) ---
\subsection{Khemesh, the Kraken Lord (Depths \& Inevitable Power)}
\textit{Lore.} Khemesh is the unseen weight beneath the waves—the patience of the abyss and the certainty of drowning.

\begin{quote}
In the end, all things sink. The depths remember everything.
\end{quote}

\paragraph{Grasp of the Abyss (Low, 4 XP)} \emph{Action; Near; Yes (single target).}
\textbf{Materials:} A knot tied and dropped in water.\\
\textbf{Effect:} Phantom tentacles clutch; target suffers \(-1\) die next action or is briefly held in place.\\
\textbf{Push It:} Inflict 1 Harm (Crush); you gain Fatigue 1 (phantom drowning).\\
\emph{Requires: Familiar \ (\textit{Invoke:} 1 Boon).}

\paragraph{Rite of the Drowning Silence (Low, 5 XP)} \emph{Scene; Zone; No.}
\textbf{Materials:} A seashell filled with water, then broken.\\
\textbf{Effect:} Muffle sound; rolls relying on voice/noise suffer \(-1\) die; Stealth gains +1 die.\\
\textbf{Push It:} Silence deepens to pressure; others resist or suffer Fatigue 1.\\
\emph{Requires: Familiar \ (\textit{Invoke:} 1 Boon).}

\paragraph{Weight of the Deep (Standard, 7 XP)} \emph{Scene; Near; No.}
\textbf{Materials:} A stone dropped into black water.\\
\textbf{Effect:} Target moves as if burdened; \(-1\) die to physical actions for the scene.\\
\textbf{Push It:} Briefly pin the target to deck/ground; mark \textbf{1 SB (Clubs)} from collateral strain.\\
\emph{Requires: Familiar + Codex \ (\textit{Invoke:} 1 Boon).}

\paragraph{Rite of the Crushing Coil (Standard, 9 XP)} \emph{Instant; Near; No.}
\textbf{Materials:} Rope/chain wrapped around your arm.\\
\textbf{Effect:} Spectral tentacle lashes or constricts: deal 2 Harm (Crush) to one target in Near.\\
\textbf{Push It:} Instead immobilize for one beat; you suffer Fatigue 1.\\
\emph{Requires: Familiar + Codex \ (\textit{Invoke:} 1 Boon).}

\paragraph{Abyssal Dominion (High, 12 XP)} \emph{Scene; Zone; No.}
\textbf{Materials:} Seawater poured in a circle.\\
\textbf{Effect:} Unseen tides hinder foes: enemies in Zone \(-1\) die to move/attack; allies \textbf{+1 die} to resist.\\
\textbf{Push It:} One massive tentacle strikes (3 Harm) once; draws the Keeper’s tide: \textbf{2 SB (Clubs/Spades)}.\\
\emph{Requires: Familiar + Codex + Tier III \ (\textit{Invoke:} \textbf{2 Boons}).}\\
\emph{Obligation:} 7 segments.

% --- Patron: Mab, Queen of Courts (Glamour & Bargain) ---
\subsection{Mab, Queen of Courts (Glamour \& Bargain)}
\textit{Lore.} The blush of truth, the dagger of etiquette, the smile that writes debts in perfume. Mab rules where desire dresses itself as courtesy.

\begin{quote}
Bend, don’t bow. Smile, don’t promise.
\end{quote}

\paragraph{Courtly Guise \textnormal{[VEIL]} (Low, 4 XP)} \emph{Action; Self; Yes (social only).}
\textbf{Materials:} Pin a sprig of green or silver thread.\\
\textbf{Effect:} Subtle glamour: \textbf{+1 die} to Persuade/Sway in refined settings; you appear as expected rank/guest.\\
\textbf{Push It:} Also mask one minor tell; the first piercing question in the scene generates \textbf{1 SB (Hearts)}.\\
\emph{Requires: Familiar \ (\textit{Invoke:} 1 Boon).}

\paragraph{Token of Favor (Low, 5 XP)} \emph{Scene; Near; No.}
\textbf{Materials:} A ribbon or ring bestowed.\\
\textbf{Effect:} Grant an ally \textbf{+1 die} to one social action against onlookers who recognize your favor; you gain \textbf{+1 effect} to support.\\
\textbf{Push It:} The token also chills a heckler (one beat of hesitation), but you mark \emph{Exposure +1}.\\
\emph{Requires: Familiar \ (\textit{Invoke:} 1 Boon).}

\paragraph{Mirror of Motives (Standard, 7 XP)} \emph{Action; Near; No.}
\textbf{Materials:} A polished shard or compact mirror.\\
\textbf{Effect:} Ask one pointed question about an NPC’s \emph{immediate} social goal; Keeper answers truthfully or with a strong tell. Gain \textbf{+1 die} to exploit it this scene.\\
\textbf{Push It:} Also expose a concealed slight or insult that matters to them, creating \textbf{1 SB (Hearts)} on that target.\\
\emph{Requires: Familiar + Codex \ (\textit{Invoke:} 1 Boon).}

\paragraph{The Price Agreed \textnormal{[OATH]} (Standard, 8 XP)} \emph{Scene; Near; No.}
\textbf{Materials:} Exchange a token of equal apparent value.\\
\textbf{Effect:} Bind a petty bargain (favor-for-favor). Breach forces \textbf{1 SB (Hearts or Diamonds)} on the breaker and stains their reputation locally this arc.\\
\textbf{Push It:} Sweeten terms with a minor boon (\(+1\) die once to the beneficiary), but you take \textbf{1 SB (Hearts)} if they later breach.\\
\emph{Requires: Familiar + Codex \ (\textit{Invoke:} 1 Boon).}

\paragraph{Sovereign Glamour \textnormal{[VEIL][REVEAL]} (High, 11 XP)} \emph{Scene; Zone; No.}
\textbf{Materials:} A circle of green felt or silk.\\
\textbf{Effect:} Establish Court: allies in Zone gain \textbf{+1 die} to social actions; crude threats suffer \(-1\) die. Once, peel one disguise/illusion in Zone.\\
\textbf{Push It:} Name a \emph{Court Law} (e.g., no drawn steel): first violation \emph{forces 2 SB} on the violator.\\
\emph{Requires: Familiar + Codex + Tier III \ (\textit{Invoke:} \textbf{2 Boons}).}\\
\emph{Obligation:} 6 segments.

\input{patrons/09i-patron-traveler.tex}
% --- Patron: The Clockwork Monad (Mechanism & Process) ---
\subsection{The Clockwork Monad (Mechanism \& Process)}
\textit{Lore.} Gears remember the plan even when their makers forget. The Monad prizes precision, iteration, and the beauty of mechanisms that keep their word.

\begin{quote}
Every tooth matters. Especially the ones you cannot see.
\end{quote}

\paragraph{Calibrated Touch (Low, 4 XP)} \emph{Action; Self; Yes (tinkering).}
\textbf{Materials:} A single oiled gear tooth.\\
\textbf{Effect:} \textbf{+1 die} to repair, set, or disarm precise mechanisms; re-roll one \texttt{1} on a Tinker action.\\
\textbf{Push It:} Guarantee no collateral on a simple device, but you take \textbf{1 SB (Clubs)} if rushed.\\
\emph{Requires: Familiar \ (\textit{Invoke:} 1 Boon).}

\paragraph{Process Lock (Low, 5 XP)} \emph{Scene; Near; No.}
\textbf{Materials:} A drop of red oil.\\
\textbf{Effect:} Name a process (reload, dispatch, raise alarm). First attempt to perform it in-scene suffers \(-1\) die as steps “stick.”\\
\textbf{Push It:} The second attempt also hesitates (one beat), but the third surges forward and \emph{forces 1 SB (Spades)}.\\
\emph{Requires: Familiar \ (\textit{Invoke:} 1 Boon).}

\paragraph{Iterative Advantage (Standard, 7 XP)} \emph{Action; Self/Ally; No.}
\textbf{Materials:} Notched tally on metal.\\
\textbf{Effect:} On a repeated action this scene, grant \textbf{+2 dice} or \textbf{+1 effect}.\\
\textbf{Push It:} Bank a follow-up \textbf{+1 die} for the same action later this scene; if unused, generate \textbf{1 SB (Diamonds)} as unused potential jams something.\\
\emph{Requires: Familiar + Codex \ (\textit{Invoke:} 1 Boon).}

\paragraph{Suppress Malfunction \textnormal{[DISPEL]} (Standard, 8 XP)} \emph{Instant; Near; No.}
\textbf{Materials:} Mainspring coil released.\\
\textbf{Effect:} End/suppress an ongoing \emph{mechanical or procedural} failure/curse (e.g., jamming jam, recursive alarm, fatigue spiral tick). DV by fiction.\\
\textbf{Push It:} Also clear one related clock segment; create \textbf{1 SB (Clubs)} as pressure shifts elsewhere.\\
\emph{Requires: Familiar + Codex \ (\textit{Invoke:} 1 Boon).}

\paragraph{Prime the Engine (High, 12 XP)} \emph{Scene; Zone; No.}
\textbf{Materials:} A ring of interlocked cogs.\\
\textbf{Effect:} Allies in Zone add a “\emph{Process Buff}”: first action that repeats in-scene gets an automatic Position bump \emph{or} upgrades Effect.\\
\textbf{Push It:} Also seize timing: once, reorder two adjacent beats; mark \textbf{2 SB (Diamonds)} as the world protests.\\
\emph{Requires: Familiar + Codex + Tier III \ (\textit{Invoke:} \textbf{2 Boons}).}\\
\emph{Obligation:} 7 segments.

\input{patrons/09k-patron-varnek-karn.tex}
% --- Patron: Nidhoggr (Deep Earth & Rot) ---
\subsection{Nidhoggr (Deep Earth \& Rot)}
\textit{Lore.} What grows must fall. What stands must sink. Nidhoggr is patient ruin—roots splitting stone, rot reclaiming pride.

\begin{quote}
Lie down. The earth knows what to make of you.
\end{quote}

\paragraph{Root-Grip (Low, 4 XP)} \emph{Action; Near; Yes (grounded target).}
\textbf{Materials:} A seed pressed into soil.\\
\textbf{Effect:} Roots or sinkholes hinder a grounded foe; \(-1\) die on their next move/strike.\\
\textbf{Push It:} Brief immobilize (one beat) if they stand on soil/wood; you take \textbf{1 SB (Clubs)} as structures complain.\\
\emph{Requires: Familiar \ (\textit{Invoke:} 1 Boon).}

\paragraph{Rot’s Kiss (Low, 5 XP)} \emph{Scene; Touch; No.}
\textbf{Materials:} A smear of compost or mold.\\
\textbf{Effect:} Tag gear or cover: first use this scene is \emph{Limited Effect} as rot softens it.\\
\textbf{Push It:} Also chip 1 integrity from mundane barriers/props touched in the scene.\\
\emph{Requires: Familiar \ (\textit{Invoke:} 1 Boon).}

\paragraph{Burden of Old Stone (Standard, 7 XP)} \emph{Scene; Near; No.}
\textbf{Materials:} A river-stone that has never seen sunlight.\\
\textbf{Effect:} Drop morale and vigor: \(-1\) die to strenuous actions for enemies who can feel the ground; allies gain steadiness (+1 die to resist knockdown).\\
\textbf{Push It:} Collapse a minor lintel/ledge to reshape cover; creates \textbf{1 SB (Clubs)}.\\
\emph{Requires: Familiar + Codex \ (\textit{Invoke:} 1 Boon).}

\paragraph{Devour the Pillar (Standard, 9 XP)} \emph{Instant; Near; No.}
\textbf{Materials:} Obsidian spindle with a hairline flaw.\\
\textbf{Effect:} Target a single support, axle, or keystone; reduce its integrity drastically (Keeper: equivalent to a big bite out of a [4] barrier).\\
\textbf{Push It:} Chain reaction threatens another support in Zone; you must choose to save someone or secure footing (\textbf{1 SB (Spades/Clubs)}).\\
\emph{Requires: Familiar + Codex \ (\textit{Invoke:} 1 Boon).}

\paragraph{Grave-Quiet Dominion (High, 12 XP)} \emph{Scene; Zone; No.}
\textbf{Materials:} Fossil tooth shard and grave loam.\\
\textbf{Effect:} The ground asserts itself: enemies in Zone suffer \(-1\) die to rush/charge/leap; all falls worsen by one step; once, open a sink to isolate a foe.\\
\textbf{Push It:} Call the slow crush: \emph{force 2 SB (Clubs/Spades)} as supports, beams, or roots shift at the worst time.\\
\emph{Requires: Familiar + Codex + Tier III \ (\textit{Invoke:} \textbf{2 Boons}).}\\
\emph{Obligation:} 7 segments.

\input{patrons/09m-patron-oath-flame-light.tex}

% Requires: \usepackage{float} for [H] tables
% =========================

\section{Rites, Invokers, and Symbols}
\label{sec:rites}

Magic in \textbf{Fate's Edge} expresses through three intertwined practices: \textbf{Rites} (oathbound authority), \textbf{Invocations} (symbolic ritual), and \textbf{Patron Pacts} (gifts and obligations). The rules below emphasize fiction-first play: consequences are Story Beats (SB) that prompt twists; numbers follow the story.

\subsection{Rites and Patrons (Runekeepers)}
\label{subsec:runekeepers}
Characters who bind themselves to a \emph{single} Patron and study that Patron's \textbf{Codex} are \textbf{Runekeepers}. Their magic is structured, immediate, and tied to service.

\begin{itemize}
  \item \textbf{One-Patron Rule.} A Runekeeper may be bound to \emph{only one} Patron at a time. This sharpens identity and keeps Obligation on a single ledger.
  \item \textbf{Thiasos (Familiar).} A circle, retinue, or emissary that grounds the pact in fiction. Required to access \emph{Patron's Gift}.
  \item \textbf{Codex.} The Patron's corpus of rites and precedents. Grants access to the Patron's Rites.
  \item \textbf{Invoke Rites.} A Runekeeper may Invoke a known Rite from their Patron as a \textbf{1 action} effect. On completion, mark \textbf{+1 Obligation} to that Patron. You may \emph{Push It} once per scene to amplify the effect, marking \textbf{+1 additional Obligation}.
\end{itemize}

\subsection{Invokers and Symbols}
\label{subsec:invokers}
Invokers relate to Patrons through consecrated \textbf{Symbols}: physical tokens that anchor names and permissions.

\begin{itemize}
  \item \textbf{Symbols (Minor Asset).} Each Symbol is keyed to one Patron; cost \textbf{4 XP}. You may own Symbols of different Patrons (one Symbol per Patron).
  \item \textbf{Ritual Invocation.} Display the Symbol and perform the Rite as a \emph{ritual} (Significant Time). Completion always marks \textbf{+1 Obligation} on that Rite's ledger.
  \item \textbf{Crack the Seal.} As part of an Invoker Rite, you may resolve the effect instantly by setting the Symbol to \emph{Compromised} and marking \textbf{+2 Obligation} (\textbf{+3} if High-Power). The Keeper may spend 1 on-theme SB immediately. The asset remains but is inert until restored.
  \item \textbf{Restore a Symbol.} 1 downtime action and a fitting test (DV 3 or by fiction). Success: \emph{Maintained}; shaky: returns \emph{Neglected}. Or spend \textbf{1 XP} to fully restore.
  \item \textbf{Display Requirement.} Symbols must be openly displayed for rituals. Hidden Symbols do not function.
\end{itemize}

\subsection{Casting and Free-Form Magic}
\label{subsec:casting}
Improvised casting is possible with the \textbf{Caster's Gift} Talent (\textbf{2 XP}). It is a \emph{backup toolkit}:
\begin{itemize}
  \item Small, local effects (typ. DV 2--3), fiction-first, colored by Elements and locus.
  \item Heavy control effects such as \texttt{[WARD]}, \texttt{[BANISH]}, or \texttt{[UNWARD]} require a printed Talent, Rite, or Spell result.
\end{itemize}

\subsection{Patron's Gift (Imbuements)}
\label{subsec:patrons-gift}
The pact may mark a devotee's tools with a short-lived boon aligned to the Patron's domain.

\paragraph{Requirements.}
\textbf{Thiasos (Familiar)} is required. Invoking the Gift costs \textbf{1 Boon}. A Codex is \emph{not} required for the Gift.

\paragraph{Activation and Duration.}
\begin{itemize}
  \item \textbf{Action:} 1 action to activate; \textbf{1/scene}.
  \item \textbf{Duration:} Scene. \emph{Push It:} extend for one additional scene by marking \textbf{+1 Obligation} to that Patron (max one Push per scene).
  \item \textbf{Range:} Touch (you must handle the item).
  \item \textbf{Stacking:} Gifts from the \emph{same Patron} do not stack; take the best active version. Dice bonuses respect the table's \textbf{+3 dice cap}.
\end{itemize}

\paragraph{Effect.}
Choose one held item you or an ally carries. Until scene end it grants:
\begin{itemize}
  \item \textbf{+1 Melee} (the item counts as a magical weapon), and
  \item \textbf{+1 Thematic} (a \emph{+1 die} to a fixed Skill tied to your Patron; see Table~\ref{tab:gift-thematic-map}). Apply only when the fiction clearly fits the Patron's sphere and how the item is used.
\end{itemize}

\paragraph{Runekeeper Clarification.}
A Runekeeper (one Patron + Codex) may Invoke Rites on-screen and use Patron's Gift if they also possess \textbf{Thiasos (Familiar)}. Codex alone does not grant the Gift. Symbols are optional for parley or omens and do not gate Runekeeper Invocation or the Gift.

\section*{Borrowed Grace}
\label{talent:borrowed-grace}
\index{Talents!Invoker}\index{Imbuement!Lesser}

\textbf{Type:} Invoker Talent — \textit{Lesser Imbuement}

\subsection*{Use}
\begin{itemize}
  \item \textbf{Cost:} 1 Boon, 1 action.
  \item \textbf{Effect (pick one on use):} \textbf{+1 Melee} \emph{or} \textbf{+1 Thematic} (your table’s thematic Skill).
  \item \textbf{Duration:} \textit{Single action/attack} (instantaneous boost).
  \item \textbf{Requirement:} Wield/display the Patron’s \textbf{Symbol}.
  \item \textbf{Obligation:} +1 \textbf{Obligation} to that Patron immediately (see \S\ref{sec:obligation}).
  \item \textbf{Limits:} Cannot be extended, stacked, or \emph{Pushed} for duration.
\end{itemize}

\subsection*{Fictional Framing}
A quick, rule-bending channel through a Patron’s \emph{Symbol}—a sliver of grace, borrowed for a moment and paid for in debt.

\subsection*{Table Guidance (1-liners)}
\begin{itemize}
  \item \textbf{Combat:} Spike a strike vs. a tough foe; or steady a parry in a desperate bind.
  \item \textbf{Skill:} Nudge a pivotal social/ritual/track roll tied to the Patron’s sphere.
  \item \textbf{Fallout:} Repeated use accrues \textbf{Obligation}; NPC faithful may notice “stolen” grace.
\end{itemize}

\subsection*{Balance Notes}
\begin{itemize}
  \item Weaker than full Imbuement: \emph{one} action, no sustain, upfront Obligation.
  \item \textbf{Symbol dependency:} No Symbol, no channel (concealed or lost Symbol = no effect).
\end{itemize}

\subsection*{GM Hooks (quick picks)}
\begin{itemize}
  \item \textbf{Compel Debt:} A Patron agent arrives when Obligation crosses a tick.
  \item \textbf{Clash of Signs:} Using rival Symbols back-to-back risks minor \textbf{Backlash} (drop Position or +1 SB).
  \item \textbf{Spotlight Tell:} Brief visual tell (scent, sigil flare) marks the borrowing to observant NPCs.
\end{itemize}

\begin{table}[H]
\centering
\renewcommand{\arraystretch}{1.15}
\begin{tabular}{@{}p{3.8cm}p{3.8cm}p{7.5cm}@{}}
\toprule
\textbf{Patron} & \textbf{+1 Thematic Skill} & \textbf{Example Symbols} \\
\midrule
Ikasha (Shadow, Penumbra) & Stealth & Knot of black silk; soot-oil vial; fingerbone ring lacquered matte \\
Mykkiel (Judgment, Writ) & Command & Cold-iron seal matrix; parchment writ-tag; square rule stamped with code \\
The Witness (Truth, Revelation) & Notice & Obsidian eye pendant; silver mirror shard; wax seal-stamp with an open eye \\
Sealed Gate (Boundaries, Closure) & Tinker & Lead sounder-weight; iron chain link; sealed lockplate token \\
Raéyn (Storm, Tides) & Skirmish & Sea-glass disk; salt-crusted rope knot; vial of rainwater from three crossings \\
Khemesh (Abyss, Pressure) & Skirmish & Barnacle-bitten coin; abyssal-spiral lead weight; salt-etched iron chain \\
Mab (Glamour, Courts) & Persuade & Hawthorn thorn wrapped in silver; mirror shard with green felt; silk-lined acorn cup \\
Sacred Geometry (Perfect Forms) & Tinker & Brass heptagram compass; bone tablet with golden-ratio spiral; plumb-bob with proof in red thread \\
Clockwork Monad (Mechanism, Process) & Tinker & Gear tooth sealed in oil; mainspring coil; rivet stamped with forbidden numerals \\
Varnek Karn (Ossuary, Dominion of the Dead) & Command & Ossuary bead rosary; carved phalanx tally; fused bone-and-obsidian coin \\
Nidhoggr (Deep Earth, Rot) & Skirmish & Fossil tooth shard; dark river-stone; obsidian spindle with flaw \\
The Traveler (Ways, Roads) & Notice & Road-nail wrapped in thread; waystone pebble; brass compass missing its needle \\
Oath of Flame \& Light (Dawn, Vows) & Command & Cold-iron sun-stamp; vow-ring with sunrise and true name; ampoule of consecrated spark \\
\bottomrule
\end{tabular}
\caption{Patron's Gift: fixed Thematic Skill and example Symbols. Thematic bonuses apply only when the fiction matches the Patron’s domain and the item’s use. Symbols also serve Invokers as ritual anchors.}
\label{tab:gift-thematic-map}
\end{table}

\subsection{Specialization vs.\ Mixing}
\label{subsec:mixing}
Characters can mix paths (Summoner, Caster, Invoker, Runekeeper), but specialization is usually stronger and cleaner. Mixing increases upkeep (Obligation, Symbol state, Leash) and action congestion without guaranteed power gains. Let fiction guide choices: Story Beats are prompts to advance the scene, not punishments.

% Patron subsections (split files — keep filenames in the same directory)

\input{patrons/09a-patron-witness.tex}
\input{patrons/09b-patron-ikasha.tex}
\input{patrons/09c-patron-sacred-geometry.tex}
\input{patrons/09d-patron-inaea.tex}
% --- Patron: Raéyn, Keeper of the Sealed Gate (Thresholds & Warding) ---
\subsection{Raéyn, Keeper of the Sealed Gate (Thresholds \& Warding)}
\textit{Lore.} Raéyn is invoked at every border. His gift is the lock that preserves and the seal that keeps chaos at bay.

\begin{quote}
The door is not shut until Raéyn’s mark is traced.
\end{quote}

\paragraph{Seal the Latch (Low, 4 XP)} \emph{Action; Near; Yes (object).}
\textbf{Materials:} Trace a key-sign with ash/chalk.\\
\textbf{Effect:} Secure a container/door/gate. Attempts to open require intruder to generate \textbf{+1 SB} (Spades/Clubs).\\
\textbf{Push It:} First touch flares (noise/flash), revealing the attempt.\\
\emph{Requires: Familiar \ (\textit{Invoke:} 1 Boon).}

\paragraph{Rite of the Quiet Gate (Low, 5 XP)} \emph{Scene; Near; No.}
\textbf{Materials:} A key turned backwards in a lock.\\
\textbf{Effect:} Warded threshold. Passing uninvited imposes \(-1\) die on the trespasser’s next action.\\
\textbf{Push It:} The ward whispers intruder’s purpose in one phrase.\\
\emph{Requires: Familiar \ (\textit{Invoke:} 1 Boon).}

\paragraph{Mark of Raéyn (Standard, 8 XP)} \emph{Scene; Near; No.}
\textbf{Materials:} A drawn circle or sigil.\\
\textbf{Effect:} Protect a room/wagon. Crossing without consent generates \textbf{1 SB} (suit by GM).\\
\textbf{Push It:} Suppress one minor spell crossing; when it collapses, mark \textbf{1 SB (Hearts)} backlash.\\
\emph{Requires: Familiar + Codex \ (\textit{Invoke:} 1 Boon).}

\paragraph{Rite of the Sealed Mouth (Standard, 7 XP)} \emph{Scene; Near; No.}
\textbf{Materials:} Thread tied across lips, then removed.\\
\textbf{Effect:} Choose one lost channel: speech, script, or gesture. Within, it fails for the scene.\\
\textbf{Push It:} Suppress all three, but you are muted until scene end.\\
\emph{Requires: Familiar + Codex \ (\textit{Invoke:} 1 Boon).}

\paragraph{Rite of the Sealed Gate \textnormal{[WARD]} (High, 11 XP)} \emph{Scene; Near; No.}
\textbf{Materials:} Iron-powder circle, locked with a key.\\
\textbf{Effect:} Impassable boundary (~10 ft). Forcing entry inflicts \textbf{2 SB} (Clubs/Spades) on the intruder.\\
\textbf{Push It:} Enclose a chamber/courtyard. Each additional hour risks \textbf{1 SB (Diamonds)} toward omen/strain.\\
\emph{Requires: Familiar + Codex + Tier III \ (\textit{Invoke:} \textbf{2 Boons}).}\\
\emph{Obligation:} 7 segments.

\input{patrons/09f-patron-mykkiel.tex}
% --- Patron: Khemesh, the Kraken Lord (Depths & Inevitable Power) ---
\subsection{Khemesh, the Kraken Lord (Depths \& Inevitable Power)}
\textit{Lore.} Khemesh is the unseen weight beneath the waves—the patience of the abyss and the certainty of drowning.

\begin{quote}
In the end, all things sink. The depths remember everything.
\end{quote}

\paragraph{Grasp of the Abyss (Low, 4 XP)} \emph{Action; Near; Yes (single target).}
\textbf{Materials:} A knot tied and dropped in water.\\
\textbf{Effect:} Phantom tentacles clutch; target suffers \(-1\) die next action or is briefly held in place.\\
\textbf{Push It:} Inflict 1 Harm (Crush); you gain Fatigue 1 (phantom drowning).\\
\emph{Requires: Familiar \ (\textit{Invoke:} 1 Boon).}

\paragraph{Rite of the Drowning Silence (Low, 5 XP)} \emph{Scene; Zone; No.}
\textbf{Materials:} A seashell filled with water, then broken.\\
\textbf{Effect:} Muffle sound; rolls relying on voice/noise suffer \(-1\) die; Stealth gains +1 die.\\
\textbf{Push It:} Silence deepens to pressure; others resist or suffer Fatigue 1.\\
\emph{Requires: Familiar \ (\textit{Invoke:} 1 Boon).}

\paragraph{Weight of the Deep (Standard, 7 XP)} \emph{Scene; Near; No.}
\textbf{Materials:} A stone dropped into black water.\\
\textbf{Effect:} Target moves as if burdened; \(-1\) die to physical actions for the scene.\\
\textbf{Push It:} Briefly pin the target to deck/ground; mark \textbf{1 SB (Clubs)} from collateral strain.\\
\emph{Requires: Familiar + Codex \ (\textit{Invoke:} 1 Boon).}

\paragraph{Rite of the Crushing Coil (Standard, 9 XP)} \emph{Instant; Near; No.}
\textbf{Materials:} Rope/chain wrapped around your arm.\\
\textbf{Effect:} Spectral tentacle lashes or constricts: deal 2 Harm (Crush) to one target in Near.\\
\textbf{Push It:} Instead immobilize for one beat; you suffer Fatigue 1.\\
\emph{Requires: Familiar + Codex \ (\textit{Invoke:} 1 Boon).}

\paragraph{Abyssal Dominion (High, 12 XP)} \emph{Scene; Zone; No.}
\textbf{Materials:} Seawater poured in a circle.\\
\textbf{Effect:} Unseen tides hinder foes: enemies in Zone \(-1\) die to move/attack; allies \textbf{+1 die} to resist.\\
\textbf{Push It:} One massive tentacle strikes (3 Harm) once; draws the Keeper’s tide: \textbf{2 SB (Clubs/Spades)}.\\
\emph{Requires: Familiar + Codex + Tier III \ (\textit{Invoke:} \textbf{2 Boons}).}\\
\emph{Obligation:} 7 segments.

% --- Patron: Mab, Queen of Courts (Glamour & Bargain) ---
\subsection{Mab, Queen of Courts (Glamour \& Bargain)}
\textit{Lore.} The blush of truth, the dagger of etiquette, the smile that writes debts in perfume. Mab rules where desire dresses itself as courtesy.

\begin{quote}
Bend, don’t bow. Smile, don’t promise.
\end{quote}

\paragraph{Courtly Guise \textnormal{[VEIL]} (Low, 4 XP)} \emph{Action; Self; Yes (social only).}
\textbf{Materials:} Pin a sprig of green or silver thread.\\
\textbf{Effect:} Subtle glamour: \textbf{+1 die} to Persuade/Sway in refined settings; you appear as expected rank/guest.\\
\textbf{Push It:} Also mask one minor tell; the first piercing question in the scene generates \textbf{1 SB (Hearts)}.\\
\emph{Requires: Familiar \ (\textit{Invoke:} 1 Boon).}

\paragraph{Token of Favor (Low, 5 XP)} \emph{Scene; Near; No.}
\textbf{Materials:} A ribbon or ring bestowed.\\
\textbf{Effect:} Grant an ally \textbf{+1 die} to one social action against onlookers who recognize your favor; you gain \textbf{+1 effect} to support.\\
\textbf{Push It:} The token also chills a heckler (one beat of hesitation), but you mark \emph{Exposure +1}.\\
\emph{Requires: Familiar \ (\textit{Invoke:} 1 Boon).}

\paragraph{Mirror of Motives (Standard, 7 XP)} \emph{Action; Near; No.}
\textbf{Materials:} A polished shard or compact mirror.\\
\textbf{Effect:} Ask one pointed question about an NPC’s \emph{immediate} social goal; Keeper answers truthfully or with a strong tell. Gain \textbf{+1 die} to exploit it this scene.\\
\textbf{Push It:} Also expose a concealed slight or insult that matters to them, creating \textbf{1 SB (Hearts)} on that target.\\
\emph{Requires: Familiar + Codex \ (\textit{Invoke:} 1 Boon).}

\paragraph{The Price Agreed \textnormal{[OATH]} (Standard, 8 XP)} \emph{Scene; Near; No.}
\textbf{Materials:} Exchange a token of equal apparent value.\\
\textbf{Effect:} Bind a petty bargain (favor-for-favor). Breach forces \textbf{1 SB (Hearts or Diamonds)} on the breaker and stains their reputation locally this arc.\\
\textbf{Push It:} Sweeten terms with a minor boon (\(+1\) die once to the beneficiary), but you take \textbf{1 SB (Hearts)} if they later breach.\\
\emph{Requires: Familiar + Codex \ (\textit{Invoke:} 1 Boon).}

\paragraph{Sovereign Glamour \textnormal{[VEIL][REVEAL]} (High, 11 XP)} \emph{Scene; Zone; No.}
\textbf{Materials:} A circle of green felt or silk.\\
\textbf{Effect:} Establish Court: allies in Zone gain \textbf{+1 die} to social actions; crude threats suffer \(-1\) die. Once, peel one disguise/illusion in Zone.\\
\textbf{Push It:} Name a \emph{Court Law} (e.g., no drawn steel): first violation \emph{forces 2 SB} on the violator.\\
\emph{Requires: Familiar + Codex + Tier III \ (\textit{Invoke:} \textbf{2 Boons}).}\\
\emph{Obligation:} 6 segments.

\input{patrons/09i-patron-traveler.tex}
% --- Patron: The Clockwork Monad (Mechanism & Process) ---
\subsection{The Clockwork Monad (Mechanism \& Process)}
\textit{Lore.} Gears remember the plan even when their makers forget. The Monad prizes precision, iteration, and the beauty of mechanisms that keep their word.

\begin{quote}
Every tooth matters. Especially the ones you cannot see.
\end{quote}

\paragraph{Calibrated Touch (Low, 4 XP)} \emph{Action; Self; Yes (tinkering).}
\textbf{Materials:} A single oiled gear tooth.\\
\textbf{Effect:} \textbf{+1 die} to repair, set, or disarm precise mechanisms; re-roll one \texttt{1} on a Tinker action.\\
\textbf{Push It:} Guarantee no collateral on a simple device, but you take \textbf{1 SB (Clubs)} if rushed.\\
\emph{Requires: Familiar \ (\textit{Invoke:} 1 Boon).}

\paragraph{Process Lock (Low, 5 XP)} \emph{Scene; Near; No.}
\textbf{Materials:} A drop of red oil.\\
\textbf{Effect:} Name a process (reload, dispatch, raise alarm). First attempt to perform it in-scene suffers \(-1\) die as steps “stick.”\\
\textbf{Push It:} The second attempt also hesitates (one beat), but the third surges forward and \emph{forces 1 SB (Spades)}.\\
\emph{Requires: Familiar \ (\textit{Invoke:} 1 Boon).}

\paragraph{Iterative Advantage (Standard, 7 XP)} \emph{Action; Self/Ally; No.}
\textbf{Materials:} Notched tally on metal.\\
\textbf{Effect:} On a repeated action this scene, grant \textbf{+2 dice} or \textbf{+1 effect}.\\
\textbf{Push It:} Bank a follow-up \textbf{+1 die} for the same action later this scene; if unused, generate \textbf{1 SB (Diamonds)} as unused potential jams something.\\
\emph{Requires: Familiar + Codex \ (\textit{Invoke:} 1 Boon).}

\paragraph{Suppress Malfunction \textnormal{[DISPEL]} (Standard, 8 XP)} \emph{Instant; Near; No.}
\textbf{Materials:} Mainspring coil released.\\
\textbf{Effect:} End/suppress an ongoing \emph{mechanical or procedural} failure/curse (e.g., jamming jam, recursive alarm, fatigue spiral tick). DV by fiction.\\
\textbf{Push It:} Also clear one related clock segment; create \textbf{1 SB (Clubs)} as pressure shifts elsewhere.\\
\emph{Requires: Familiar + Codex \ (\textit{Invoke:} 1 Boon).}

\paragraph{Prime the Engine (High, 12 XP)} \emph{Scene; Zone; No.}
\textbf{Materials:} A ring of interlocked cogs.\\
\textbf{Effect:} Allies in Zone add a “\emph{Process Buff}”: first action that repeats in-scene gets an automatic Position bump \emph{or} upgrades Effect.\\
\textbf{Push It:} Also seize timing: once, reorder two adjacent beats; mark \textbf{2 SB (Diamonds)} as the world protests.\\
\emph{Requires: Familiar + Codex + Tier III \ (\textit{Invoke:} \textbf{2 Boons}).}\\
\emph{Obligation:} 7 segments.

\input{patrons/09k-patron-varnek-karn.tex}
% --- Patron: Nidhoggr (Deep Earth & Rot) ---
\subsection{Nidhoggr (Deep Earth \& Rot)}
\textit{Lore.} What grows must fall. What stands must sink. Nidhoggr is patient ruin—roots splitting stone, rot reclaiming pride.

\begin{quote}
Lie down. The earth knows what to make of you.
\end{quote}

\paragraph{Root-Grip (Low, 4 XP)} \emph{Action; Near; Yes (grounded target).}
\textbf{Materials:} A seed pressed into soil.\\
\textbf{Effect:} Roots or sinkholes hinder a grounded foe; \(-1\) die on their next move/strike.\\
\textbf{Push It:} Brief immobilize (one beat) if they stand on soil/wood; you take \textbf{1 SB (Clubs)} as structures complain.\\
\emph{Requires: Familiar \ (\textit{Invoke:} 1 Boon).}

\paragraph{Rot’s Kiss (Low, 5 XP)} \emph{Scene; Touch; No.}
\textbf{Materials:} A smear of compost or mold.\\
\textbf{Effect:} Tag gear or cover: first use this scene is \emph{Limited Effect} as rot softens it.\\
\textbf{Push It:} Also chip 1 integrity from mundane barriers/props touched in the scene.\\
\emph{Requires: Familiar \ (\textit{Invoke:} 1 Boon).}

\paragraph{Burden of Old Stone (Standard, 7 XP)} \emph{Scene; Near; No.}
\textbf{Materials:} A river-stone that has never seen sunlight.\\
\textbf{Effect:} Drop morale and vigor: \(-1\) die to strenuous actions for enemies who can feel the ground; allies gain steadiness (+1 die to resist knockdown).\\
\textbf{Push It:} Collapse a minor lintel/ledge to reshape cover; creates \textbf{1 SB (Clubs)}.\\
\emph{Requires: Familiar + Codex \ (\textit{Invoke:} 1 Boon).}

\paragraph{Devour the Pillar (Standard, 9 XP)} \emph{Instant; Near; No.}
\textbf{Materials:} Obsidian spindle with a hairline flaw.\\
\textbf{Effect:} Target a single support, axle, or keystone; reduce its integrity drastically (Keeper: equivalent to a big bite out of a [4] barrier).\\
\textbf{Push It:} Chain reaction threatens another support in Zone; you must choose to save someone or secure footing (\textbf{1 SB (Spades/Clubs)}).\\
\emph{Requires: Familiar + Codex \ (\textit{Invoke:} 1 Boon).}

\paragraph{Grave-Quiet Dominion (High, 12 XP)} \emph{Scene; Zone; No.}
\textbf{Materials:} Fossil tooth shard and grave loam.\\
\textbf{Effect:} The ground asserts itself: enemies in Zone suffer \(-1\) die to rush/charge/leap; all falls worsen by one step; once, open a sink to isolate a foe.\\
\textbf{Push It:} Call the slow crush: \emph{force 2 SB (Clubs/Spades)} as supports, beams, or roots shift at the worst time.\\
\emph{Requires: Familiar + Codex + Tier III \ (\textit{Invoke:} \textbf{2 Boons}).}\\
\emph{Obligation:} 7 segments.

\input{patrons/09m-patron-oath-flame-light.tex}
% =========================
% PATRON — The Carrion-King
% =========================
\subsection{The Carrion-King (Decay \& Cycles)}

\paragraph{Lore.}
The Carrion-King does not bring rot to destroy but to recycle. To him, decay is not an end but a return to the great cycle of flesh into soil, bone into earth. His priests are gardeners of entropy, ensuring that nothing lingers beyond its time and that new life always rises from death.

\paragraph{Quote.}
\emph{“All banquets end in bones, and from those bones the feast begins anew.” — The Carrion-King}

\paragraph{Rite of Gentle Rot (Low, 5 XP)} \emph{Instant; Touch; Yes (decay only).}
\textbf{Materials:} Spoiled food or a dead insect. \\
\textbf{Effect:} Accelerate natural decay on a small, non-living object (rope, lock, rations): +1 Effect to \emph{Break/Sabotage}. \\
\textbf{Push It:} A second, similar item in Close range decays; scavengers/vermin are drawn. \\
\emph{Requires: Familiar \ (\textit{Invoke:} 1 Boon).}

\paragraph{Rite of the Wilting Bloom (Low, 4 XP)} \emph{Scene; Self; No.}
\textbf{Materials:} A withered flower. \\
\textbf{Effect:} Aura of mild decay: +1 die to resist disease/poison; your carried food won’t worsen (becomes tasteless). \\
\textbf{Push It:} Wither a small plant/food source in Near; suffer \textbf{Fatigue 1}. \\
\emph{Requires: Familiar \ (\textit{Invoke:} 1 Boon).}

\paragraph{Rite of the Cycle's Turn (Standard, 8 XP)} \emph{Scene; Touch; No.}
\textbf{Materials:} A creature dead less than an hour. \\
\textbf{Effect:} From death, sustain life: choose one—purify a small food/water cache; sprout useful fungi/herbs; \emph{or} grant +1 die to resist disease/poison to one target. \\
\textbf{Push It:} Superior yield/quality; local life/death balance tilts (GM clocks/complications). \\
\emph{Requires: Familiar + Codex \ (\textit{Invoke:} 1 Boon).}

\paragraph{Rite of the Peaceful Rest (Standard, 7 XP)} \emph{Instant; Near; No.}
\textbf{Materials:} Grave dirt over the corpse. \\
\textbf{Effect:} Lay a minor spirit; prevent easy animation \emph{or} quiet a small haunting. Gain +2 dice on the next social roll with mourners/spirits. \\
\textbf{Push It:} You/ally gain undead-fear immunity for the scene; other nearby spirits grow restless. \\
\emph{Requires: Familiar + Codex \ (\textit{Invoke:} 1 Boon).}

\paragraph{Rite of the Final Compost (High, 13 XP)} \emph{Scene; Zone; No.}
\textbf{Materials:} A handful of grave dirt. \\
\textbf{Effect:} Accelerate decay across a zone: structures falter (attackers relying on them $-1$ die), enemy maintenance falters ($-1$ die to keep gear/efforts stable). Alternatively, consume a major obstacle over the scene. \\
\textbf{Push It:} The zone erupts with sickly growth; creatures lingering gain \emph{Sickened}. The growth may be harvested later. \\
\emph{Requires: Familiar + Codex + Tier III \ (\textit{Invoke:} \textbf{2 Boons}).} \\
\emph{Obligation:} 7 segments.

\paragraph{Rite of the Great Cycle (High, 14 XP)} \emph{Extended; Touch; No.}
\textbf{Materials:} Bury a seed in rich, rotten earth. \\
\textbf{Effect:} Transform a significant dead mass (large corpse/fallen tree) into something useful: fertile plot, unique reagent, or a temporary environmental boon (GM scales time/impact). \\
\textbf{Push It:} Compress to one scene; spectacle attracts attention and interference. \\
\emph{Requires: Familiar + Codex + Tier III \ (\textit{Invoke:} \textbf{2 Boons}).} \\
\emph{Obligation:} 7 segments.

\input{patrons/09p-patron-gallows-bell.tex}
\input{patrons/09q-patron-isoka.tex}
\subsection{Maelstraeus, The Infernal Bargainer (Pacts \& Debt)}
\paragraph{Lore.}  
Maelstraeus is whispered of as the Infernal Bargainer, the one who tallies every promise and weighs every gift. His followers insist that all power must be transacted, and no boon is free. Mortals who call him do so for leverage, not devotion — and yet every coin spent in his name is another bead on his endless abacus.  

\paragraph{Quote.}  
\emph{“Nothing is given. All is traded. You may forget the bargain, but I never do.” — Maelstraeus, The Infernal Bargainer}  



\paragraph{Rite of the Binding Quill (Low, 4 XP)} 
\emph{Scene; Near; No.}  
\textbf{Materials:} A drop of ink or blood upon parchment.  
\textbf{Effect:} Seal a spoken agreement in narrative force. Breaking the terms imposes a Minor Condition (Guilt, Debt, Marked).  
\textbf{Push It:} The pact is harder to break (Moderate Condition), but Maelstraeus claims a fragment of the truth surrounding the deal for his own designs.  
\emph{Requires: Familiar \ (\textit{Invoke:} 1 Boon).}  

\paragraph{Rite of the Weighed Scales (Low, 5 XP)}  
\emph{Duration: Scene; Range: Self; Stacking: No.}  
\textbf{Materials:} A coin balanced upon your palm until it falls.  
\textbf{Effect:} Gain +1 die when bargaining, haggling, or leveraging debts. You instinctively sense if the other side is giving more than they receive.  
\textbf{Push It:} You may demand a small hidden truth in the bargain, but you take Fatigue 1 from the mental strain of weighing it.  
\emph{Requires: Familiar \ (\textit{Invoke:} 1 Boon).}  

\paragraph{Rite of the Token Exchange (Low, 6 XP)}  
\emph{Duration: Scene; Range: Near; Stacking: No.}  
\textbf{Materials:} A small token given and received (coin, trinket, written word).  
\textbf{Effect:} You and the target both gain +1 die to a single roll this scene — but the GM chooses a minor narrative complication that ties you together until the “exchange” is settled.  
\textbf{Push It:} You may declare the complication instead, but the GM immediately gains 1 SB to “enforce the bargain” later.  
\emph{Requires: Familiar \ (\textit{Invoke:} 1 Boon).}  

\paragraph{Rite of the Debt Ledger (Standard, 8 XP)}  
\emph{Duration: Scene; Range: Self; Stacking: Yes (different debts).}  
\textbf{Materials:} A written list of names or accounts.  
\textbf{Effect:} Track a target who owes you a boon or debt. You gain +1 effect to actions to collect or enforce repayment.  
\textbf{Push It:} You may declare a “hidden clause” that retroactively complicates the debtor’s situation in your favor, but this grants GM 1 SB immediately.  
\emph{Requires: Familiar + Codex \ (\textit{Invoke:} 1 Boon).}  

\paragraph{Rite of the Infernal Seal (Standard, 7 XP)}  
\emph{Duration: Scene; Range: Touch; Stacking: No.}  
\textbf{Materials:} Wax seal impressed upon parchment or skin.  
\textbf{Effect:} Affix a mark of pact upon an ally or foe. For allies, once per scene they may reroll a failed action tied to the bargain. For foes, the mark imposes -1 die to resist your Sway or Command.  
\textbf{Push It:} The Seal glows with infernal script, undeniable to all, but its visibility may draw unwanted attention from other Patrons.  
\emph{Requires: Familiar + Codex \ (\textit{Invoke:} 1 Boon).}  

\paragraph{Rite of the Eternal Contract (High, 12 XP)}  
\emph{Extended; Range: Self+Other; Stacking: No.}  
\textbf{Materials:} A signed document, sealed with both blood and wax.  
\textbf{Effect:} Forge an extended pact with another character. While active, both gain a persistent +1 die when acting in line with the contract, but each breach immediately advances a 6-segment “Debt Claimed” clock.  
\textbf{Push It:} The pact can cheat death once for one party — they revive at Harm 3 instead of dying — but the “Debt Claimed” clock jumps forward 3 segments.  
\emph{Requires: Familiar + Codex + Tier III \ (\textit{Invoke:} \textbf{2 Boons}).}  
\emph{Obligation:} 7 segments.  

\paragraph{Rite of the Infernal Exchange (High, 14 XP)}  
\emph{Scene; Zone; No.}  
\textbf{Materials:} A balance scale with something of personal value upon each side.  
\textbf{Effect:} Exchange properties, conditions, or consequences between two willing targets. (e.g., swap Fatigue 2 for Harm 1, trade a social Condition, or move SB gain from one character to another).  
\textbf{Push It:} The exchange may include an unwilling target, but Maelstraeus claims something from you in kind — a permanent scar, memory, or name.  
\emph{Requires: Familiar + Codex + Tier III \ (\textit{Invoke:} \textbf{2 Boons}).}  
\emph{Obligation:} 7 segments.  

\subsection{Maestraea, The Scarlet Temptress (Temptation \& Pacts)}

\paragraph{Lore.}  
Maestraea is invoked in whispers and perfumes, the Scarlet Temptress whose gifts are sweeter than they are safe. Where Maelstraeus weighs coins, she weighs hearts. Her followers say that no promise is without a kiss, and no kiss without a hook. She thrives on secrets confessed in candlelight and debts owed in passion.  

\paragraph{Quote.}  
\emph{“You call it a bargain, love. I call it a kiss — one you’ll never forget.” — Maestraea, The Scarlet Temptress}  

\paragraph{Rite of the Velvet Promise (Low, 4 XP)} 
\emph{Scene; Near; No.}  
\textbf{Materials:} A whispered vow sealed with a kiss or caress.  
\textbf{Effect:} Seal a spoken promise with allure. Breaking the terms imposes a Minor Condition (Guilt, Tempted, Marked).  
\textbf{Push It:} The promise becomes intoxicating — breaking it escalates to a Moderate Condition, but Maestraea claims a hidden desire of the target.  
\emph{Requires: Familiar \ (\textit{Invoke:} 1 Boon).}  

\paragraph{Rite of the Honeyed Tongue (Low, 5 XP)}  
\emph{Duration: Scene; Range: Self; Stacking: No.}  
\textbf{Materials:} A sweet taste, wine, or spice upon the lips.  
\textbf{Effect:} Gain +1 die to Sway or Performance rolls. Targets instinctively want to please you, though they may not know why.  
\textbf{Push It:} You glimpse one of the target’s secret longings, but the intimacy costs you Fatigue 1.  
\emph{Requires: Familiar \ (\textit{Invoke:} 1 Boon).}  

\paragraph{Rite of the Crimson Token (Low, 6 XP)}  
\emph{Duration: Scene; Range: Near; Stacking: No.}  
\textbf{Materials:} A personal token given and received (lock of hair, charm, garment ribbon).  
\textbf{Effect:} Both giver and recipient gain +1 die to a single roll this scene, but they are bound by a palpable sense of longing or obligation until resolved.  
\textbf{Push It:} You may declare the nature of the bond, but the GM immediately gains 1 SB to entangle you further.  
\emph{Requires: Familiar \ (\textit{Invoke:} 1 Boon).}  

\paragraph{Rite of the Perfumed Ledger (Standard, 8 XP)}  
\emph{Duration: Scene; Range: Self; Stacking: Yes (different targets).}  
\textbf{Materials:} A list of names written in scented ink.  
\textbf{Effect:} You mark a target who owes you affection, desire, or service. You gain +1 effect when exploiting that bond.  
\textbf{Push It:} You may add a “hidden clause” to the bond retroactively, but this grants GM 1 SB immediately.  
\emph{Requires: Familiar + Codex \ (\textit{Invoke:} 1 Boon).}  

\paragraph{Rite of the Kiss of Chains (Standard, 7 XP)}  
\emph{Duration: Scene; Range: Touch; Stacking: No.}  
\textbf{Materials:} A kiss, or blood upon the lips.  
\textbf{Effect:} Place a mark of temptation upon an ally or foe. Allies may reroll one failed action this scene; foes suffer −1 die to resist your Sway or Command.  
\textbf{Push It:} The mark glows faintly with infernal allure, undeniable to all, but its visibility may draw attention from rival Patrons.  
\emph{Requires: Familiar + Codex \ (\textit{Invoke:} 1 Boon).}  

\paragraph{Rite of the Scarlet Covenant (High, 12 XP)}  
\emph{Extended; Range: Self+Other; Stacking: No.}  
\textbf{Materials:} A shared drink, sealed with blood.  
\textbf{Effect:} Forge an extended pact of intimacy with another character. Both gain a persistent +1 die when acting in concert. Breaking faith immediately advances a 6-segment “Temptress’ Debt” clock.  
\textbf{Push It:} Pact can prevent one partner’s death (they revive at Harm 3), but the “Temptress’ Debt” clock advances 3 segments.  
\emph{Requires: Familiar + Codex + Tier III \ (\textit{Invoke:} \textbf{2 Boons}).}  
\emph{Obligation:} 7 segments.  

\paragraph{Rite of the Crimson Exchange (High, 14 XP)}  
\emph{Scene; Zone; No.}  
\textbf{Materials:} Two entwined tokens (rings, ribbons, locks of hair).  
\textbf{Effect:} Exchange conditions or consequences between two willing targets (e.g., swap Fatigue, a social Condition, or a boon).  
\textbf{Push It:} You may include an unwilling target, but Maestraea claims a piece of you in kind — a memory, a name, or a fragment of your soul.  
\emph{Requires: Familiar + Codex + Tier III \ (\textit{Invoke:} \textbf{2 Boons}).}  
\emph{Obligation:} 7 segments.  

\input{patrons/09s-patron-the-sealed-gate.tex}
\input{patrons/09t-patron-umande.tex}

\section{Patron Rivalries}
\label{sec:patron-rivalries}

Rivalries set expectations for tone and friction. Use them to color rulings, nudge Position, and guide how Story Beats (SB) land. In their home domains, a Patron’s work tends to start a step better in Position; in a rival’s, a step worse (Keeper’s call).

\begin{table}[h!]
  \centering
  \renewcommand{\arraystretch}{1.15}
  \begin{tabular}{@{}p{3.4cm}p{3.4cm}p{8.2cm}@{}}
    \toprule
    \textbf{Patron} & \textbf{Primary Rival} & \textbf{Friction in Play (one-line read)} \\
    \midrule
    Raéyn (Sea, Tides, Travel) & Khemesh (Abyssal Maw) & Tides vs. trench: navigation and passage thrive against dread and crushing depths. \\
    Khemesh (Abyssal Maw) & Raéyn (Sea, Tides, Travel) & Abyss unmoors charts: silence, pressure, and alien geometry devour routes. \\
    Sealed Gate (Boundaries, Closure) & The Traveler (Ways, Roads) & Keys vs. roads: jurisdiction and permits against detours and desire lines. \\
    The Traveler (Ways, Roads) & Sealed Gate (Boundaries, Closure) & Paths want to open; gates insist on form—who defines the threshold? \\
    The Witness (Truth, Revelation) & Mab (Glamour, Courts) & Revelation strips glamour; courtly masks fight to endure the gaze. \\
    Mab (Glamour, Courts) & The Witness (Truth, Revelation) & Mask and merriment contest the straight line of testimony. \\
    Ikasha (Shadow, Latent Potential) & The Witness (Truth, Revelation) & Hiding and hush vs. the unblinking eye. \\
    Mykkiel (Judgment, Writ) & Varnek Karn (Necromantic Archives) & Lawful writ and living order against bone-kept precedent and unfinished business. \\
    Varnek Karn (Necromantic Archives) & Oath of Light \& Flame (Dawn, Vows) & Memory of the dead resists purgation by vow and light. \\
    Oath of Light \& Flame (Dawn, Vows) & Khemesh (Abyssal Maw) & Consecrated dawn opposes abyssal hunger and despair. \\
    Sacred Geometry (Order, Pattern) & The Traveler (Ways, Fortune) & Perfect forms vs. opportunistic routes; measure vs. happenstance. \\
    Clockwork Monad (Iteration, Process) & The Traveler (Ways, Fortune) & Procedure and refinement versus improvisation and drift. \\
    Nidhoggr (Dreaming Antiquity) & Sacred Geometry (Order, Pattern) & Ancient, slumbering memory resists imposed, modern measures. \\
    \bottomrule
  \end{tabular}
  \caption{Primary Patron Rivalries and how they tend to color scenes.}
\end{table}
