
% --- Fate's Edge SRD — Section 17: Player Archetypes at the Table ---
% Include this file from your main .tex with: 
% --- Fate's Edge SRD — Section 17: Player Archetypes at the Table ---
% Include this file from your main .tex with: 
% --- Fate's Edge SRD — Section 17: Player Archetypes at the Table ---
% Include this file from your main .tex with: 
% --- Fate's Edge SRD — Section 17: Player Archetypes at the Table ---
% Include this file from your main .tex with: \input{17-player-archetypes.tex}

\section{Player Archetypes at the Table}
\label{sec:player-archetypes}

Fate’s Edge gameplay flexes to different player archetypes. These roles are not strict classes but rather \textbf{approaches to play} that help a group balance spotlight and tone.

\subsection{The Solo}
\begin{itemize}
  \item \textbf{Focus:} Mastery of self, independence, one-on-one drama.
  \item \textbf{Strengths:} Quick spotlight scenes, duelists, scouts, specialists.
  \item \textbf{Risks:} Can drift into isolation or hog solo arcs. Needs explicit ties to group goals.
  \item \textbf{GM Tools:} Use Bonds, rival duels, and one-on-one omens to keep engagement tethered to the group.
\end{itemize}

\subsection{The Mixed}
\begin{itemize}
  \item \textbf{Focus:} Hybrid adaptability—both support and lead.
  \item \textbf{Strengths:} Bridges gaps in group dynamics; excels in team tactics.
  \item \textbf{Risks:} May feel overshadowed by extreme specialists.
  \item \textbf{GM Tools:} Spotlight them when versatility matters: flexible magic, leadership, negotiation, or sudden pivots.
\end{itemize}

\subsection{The Mastermind}
\begin{itemize}
  \item \textbf{Focus:} Schemes, plans, and command over the long arc.
  \item \textbf{Strengths:} Drives strategic play, coordinates others, excels in intrigue.
  \item \textbf{Risks:} May over-plan or dominate spotlight with metagame thinking.
  \item \textbf{GM Tools:} Challenge them with shifting information, patron demands, and rivals who anticipate their moves.
\end{itemize}

\subsection{Balancing Archetypes}
\begin{itemize}
  \item A healthy table mixes all three archetypes, ensuring solo tension, group cohesion, and strategic play coexist.
  \item Encourage players to flex between archetypes scene by scene; they are fluid, not fixed.
  \item Spotlight balance: in a 3-hour session, each archetype should anchor at least one major scene.
\end{itemize}

\subsection{GM Quick Cues}
\begin{itemize}
  \item Use archetypes as a lens for framing scenes: duel for Solos, shifting tactics for Mixed, grand reveals for Masterminds.
  \item When spotlight imbalances arise, rotate complications or bonds to another archetype’s strength.
  \item Archetypes inform how patrons and factions court PCs: Solos as champions, Mixed as envoys, Masterminds as plotters.
\end{itemize}


\section{Player Archetypes at the Table}
\label{sec:player-archetypes}

Fate’s Edge gameplay flexes to different player archetypes. These roles are not strict classes but rather \textbf{approaches to play} that help a group balance spotlight and tone.

\subsection{The Solo}
\begin{itemize}
  \item \textbf{Focus:} Mastery of self, independence, one-on-one drama.
  \item \textbf{Strengths:} Quick spotlight scenes, duelists, scouts, specialists.
  \item \textbf{Risks:} Can drift into isolation or hog solo arcs. Needs explicit ties to group goals.
  \item \textbf{GM Tools:} Use Bonds, rival duels, and one-on-one omens to keep engagement tethered to the group.
\end{itemize}

\subsection{The Mixed}
\begin{itemize}
  \item \textbf{Focus:} Hybrid adaptability—both support and lead.
  \item \textbf{Strengths:} Bridges gaps in group dynamics; excels in team tactics.
  \item \textbf{Risks:} May feel overshadowed by extreme specialists.
  \item \textbf{GM Tools:} Spotlight them when versatility matters: flexible magic, leadership, negotiation, or sudden pivots.
\end{itemize}

\subsection{The Mastermind}
\begin{itemize}
  \item \textbf{Focus:} Schemes, plans, and command over the long arc.
  \item \textbf{Strengths:} Drives strategic play, coordinates others, excels in intrigue.
  \item \textbf{Risks:} May over-plan or dominate spotlight with metagame thinking.
  \item \textbf{GM Tools:} Challenge them with shifting information, patron demands, and rivals who anticipate their moves.
\end{itemize}

\subsection{Balancing Archetypes}
\begin{itemize}
  \item A healthy table mixes all three archetypes, ensuring solo tension, group cohesion, and strategic play coexist.
  \item Encourage players to flex between archetypes scene by scene; they are fluid, not fixed.
  \item Spotlight balance: in a 3-hour session, each archetype should anchor at least one major scene.
\end{itemize}

\subsection{GM Quick Cues}
\begin{itemize}
  \item Use archetypes as a lens for framing scenes: duel for Solos, shifting tactics for Mixed, grand reveals for Masterminds.
  \item When spotlight imbalances arise, rotate complications or bonds to another archetype’s strength.
  \item Archetypes inform how patrons and factions court PCs: Solos as champions, Mixed as envoys, Masterminds as plotters.
\end{itemize}


\section{Player Archetypes at the Table}
\label{sec:player-archetypes}

Fate’s Edge gameplay flexes to different player archetypes. These roles are not strict classes but rather \textbf{approaches to play} that help a group balance spotlight and tone.

\subsection{The Solo}
\begin{itemize}
  \item \textbf{Focus:} Mastery of self, independence, one-on-one drama.
  \item \textbf{Strengths:} Quick spotlight scenes, duelists, scouts, specialists.
  \item \textbf{Risks:} Can drift into isolation or hog solo arcs. Needs explicit ties to group goals.
  \item \textbf{GM Tools:} Use Bonds, rival duels, and one-on-one omens to keep engagement tethered to the group.
\end{itemize}

\subsection{The Mixed}
\begin{itemize}
  \item \textbf{Focus:} Hybrid adaptability—both support and lead.
  \item \textbf{Strengths:} Bridges gaps in group dynamics; excels in team tactics.
  \item \textbf{Risks:} May feel overshadowed by extreme specialists.
  \item \textbf{GM Tools:} Spotlight them when versatility matters: flexible magic, leadership, negotiation, or sudden pivots.
\end{itemize}

\subsection{The Mastermind}
\begin{itemize}
  \item \textbf{Focus:} Schemes, plans, and command over the long arc.
  \item \textbf{Strengths:} Drives strategic play, coordinates others, excels in intrigue.
  \item \textbf{Risks:} May over-plan or dominate spotlight with metagame thinking.
  \item \textbf{GM Tools:} Challenge them with shifting information, patron demands, and rivals who anticipate their moves.
\end{itemize}

\subsection{Balancing Archetypes}
\begin{itemize}
  \item A healthy table mixes all three archetypes, ensuring solo tension, group cohesion, and strategic play coexist.
  \item Encourage players to flex between archetypes scene by scene; they are fluid, not fixed.
  \item Spotlight balance: in a 3-hour session, each archetype should anchor at least one major scene.
\end{itemize}

\subsection{GM Quick Cues}
\begin{itemize}
  \item Use archetypes as a lens for framing scenes: duel for Solos, shifting tactics for Mixed, grand reveals for Masterminds.
  \item When spotlight imbalances arise, rotate complications or bonds to another archetype’s strength.
  \item Archetypes inform how patrons and factions court PCs: Solos as champions, Mixed as envoys, Masterminds as plotters.
\end{itemize}


\section{Player Archetypes at the Table}
\label{sec:player-archetypes}

Fate’s Edge gameplay flexes to different player archetypes. These roles are not strict classes but rather \textbf{approaches to play} that help a group balance spotlight and tone.

\subsection{The Solo}
\begin{itemize}
  \item \textbf{Focus:} Mastery of self, independence, one-on-one drama.
  \item \textbf{Strengths:} Quick spotlight scenes, duelists, scouts, specialists.
  \item \textbf{Risks:} Can drift into isolation or hog solo arcs. Needs explicit ties to group goals.
  \item \textbf{GM Tools:} Use Bonds, rival duels, and one-on-one omens to keep engagement tethered to the group.
\end{itemize}

\subsection{The Mixed}
\begin{itemize}
  \item \textbf{Focus:} Hybrid adaptability—both support and lead.
  \item \textbf{Strengths:} Bridges gaps in group dynamics; excels in team tactics.
  \item \textbf{Risks:} May feel overshadowed by extreme specialists.
  \item \textbf{GM Tools:} Spotlight them when versatility matters: flexible magic, leadership, negotiation, or sudden pivots.
\end{itemize}

\subsection{The Mastermind}
\begin{itemize}
  \item \textbf{Focus:} Schemes, plans, and command over the long arc.
  \item \textbf{Strengths:} Drives strategic play, coordinates others, excels in intrigue.
  \item \textbf{Risks:} May over-plan or dominate spotlight with metagame thinking.
  \item \textbf{GM Tools:} Challenge them with shifting information, patron demands, and rivals who anticipate their moves.
\end{itemize}

\subsection{Balancing Archetypes}
\begin{itemize}
  \item A healthy table mixes all three archetypes, ensuring solo tension, group cohesion, and strategic play coexist.
  \item Encourage players to flex between archetypes scene by scene; they are fluid, not fixed.
  \item Spotlight balance: in a 3-hour session, each archetype should anchor at least one major scene.
\end{itemize}

\subsection{GM Quick Cues}
\begin{itemize}
  \item Use archetypes as a lens for framing scenes: duel for Solos, shifting tactics for Mixed, grand reveals for Masterminds.
  \item When spotlight imbalances arise, rotate complications or bonds to another archetype’s strength.
  \item Archetypes inform how patrons and factions court PCs: Solos as champions, Mixed as envoys, Masterminds as plotters.
\end{itemize}
