
% --- Fate's Edge SRD — Section 17: Player Archetypes at the Table ---
% Include this file from your main .tex with: 
% --- Fate's Edge SRD — Section 17: Player Archetypes at the Table ---
% Include this file from your main .tex with: 
% --- Fate's Edge SRD — Section 17: Player Archetypes at the Table ---
% Include this file from your main .tex with: 
% --- Fate's Edge SRD — Section 17: Player Archetypes at the Table ---
% Include this file from your main .tex with: \input{17-player-archetypes.tex}

\section{Player Archetypes at the Table}
\label{sec:player-archetypes}

Fate’s Edge gameplay flexes to different player archetypes. These roles are not strict classes but rather \textbf{approaches to play} that help a group balance spotlight and tone.

\subsection{The Solo}
\begin{itemize}
  \item \textbf{Focus:} Mastery of self, independence, one-on-one drama.
  \item \textbf{Strengths:} Quick spotlight scenes, duelists, scouts, specialists.
  \item \textbf{Risks:} Can drift into isolation or hog solo arcs. Needs explicit ties to group goals.
  \item \textbf{GM Tools:} Use Bonds, rival duels, and one-on-one omens to keep engagement tethered to the group.
\end{itemize}

\subsection{The Mixed}
\begin{itemize}
  \item \textbf{Focus:} Hybrid adaptability—both support and lead.
  \item \textbf{Strengths:} Bridges gaps in group dynamics; excels in team tactics.
  \item \textbf{Risks:} May feel overshadowed by extreme specialists.
  \item \textbf{GM Tools:} Spotlight them when versatility matters: flexible magic, leadership, negotiation, or sudden pivots.
\end{itemize}

\subsection{The Mastermind}
\begin{itemize}
  \item \textbf{Focus:} Schemes, plans, and command over the long arc.
  \item \textbf{Strengths:} Drives strategic play, coordinates others, excels in intrigue.
  \item \textbf{Risks:} May over-plan or dominate spotlight with metagame thinking.
  \item \textbf{GM Tools:} Challenge them with shifting information, patron demands, and rivals who anticipate their moves.
\end{itemize}

\subsection{Balancing Archetypes}
\begin{itemize}
  \item A healthy table mixes all three archetypes, ensuring solo tension, group cohesion, and strategic play coexist.
  \item Encourage players to flex between archetypes scene by scene; they are fluid, not fixed.
  \item Spotlight balance: in a 3-hour session, each archetype should anchor at least one major scene.
\end{itemize}

\subsection{GM Quick Cues}
\begin{itemize}
  \item Use archetypes as a lens for framing scenes: duel for Solos, shifting tactics for Mixed, grand reveals for Masterminds.
  \item When spotlight imbalances arise, rotate complications or bonds to another archetype’s strength.
  \item Archetypes inform how patrons and factions court PCs: Solos as champions, Mixed as envoys, Masterminds as plotters.
\end{itemize}


\section{Player Archetypes at the Table}
\label{sec:player-archetypes}

Fate’s Edge gameplay flexes to different player archetypes. These roles are not strict classes but rather \textbf{approaches to play} that help a group balance spotlight and tone.

\subsection{The Solo}
\begin{itemize}
  \item \textbf{Focus:} Mastery of self, independence, one-on-one drama.
  \item \textbf{Strengths:} Quick spotlight scenes, duelists, scouts, specialists.
  \item \textbf{Risks:} Can drift into isolation or hog solo arcs. Needs explicit ties to group goals.
  \item \textbf{GM Tools:} Use Bonds, rival duels, and one-on-one omens to keep engagement tethered to the group.
\end{itemize}

\subsection{The Mixed}
\begin{itemize}
  \item \textbf{Focus:} Hybrid adaptability—both support and lead.
  \item \textbf{Strengths:} Bridges gaps in group dynamics; excels in team tactics.
  \item \textbf{Risks:} May feel overshadowed by extreme specialists.
  \item \textbf{GM Tools:} Spotlight them when versatility matters: flexible magic, leadership, negotiation, or sudden pivots.
\end{itemize}

\subsection{The Mastermind}
\begin{itemize}
  \item \textbf{Focus:} Schemes, plans, and command over the long arc.
  \item \textbf{Strengths:} Drives strategic play, coordinates others, excels in intrigue.
  \item \textbf{Risks:} May over-plan or dominate spotlight with metagame thinking.
  \item \textbf{GM Tools:} Challenge them with shifting information, patron demands, and rivals who anticipate their moves.
\end{itemize}

\subsection{Balancing Archetypes}
\begin{itemize}
  \item A healthy table mixes all three archetypes, ensuring solo tension, group cohesion, and strategic play coexist.
  \item Encourage players to flex between archetypes scene by scene; they are fluid, not fixed.
  \item Spotlight balance: in a 3-hour session, each archetype should anchor at least one major scene.
\end{itemize}

\subsection{GM Quick Cues}
\begin{itemize}
  \item Use archetypes as a lens for framing scenes: duel for Solos, shifting tactics for Mixed, grand reveals for Masterminds.
  \item When spotlight imbalances arise, rotate complications or bonds to another archetype’s strength.
  \item Archetypes inform how patrons and factions court PCs: Solos as champions, Mixed as envoys, Masterminds as plotters.
\end{itemize}


\section{Player Archetypes at the Table}
\label{sec:player-archetypes}

Fate’s Edge gameplay flexes to different player archetypes. These roles are not strict classes but rather \textbf{approaches to play} that help a group balance spotlight and tone.

\subsection{The Solo}
\begin{itemize}
  \item \textbf{Focus:} Mastery of self, independence, one-on-one drama.
  \item \textbf{Strengths:} Quick spotlight scenes, duelists, scouts, specialists.
  \item \textbf{Risks:} Can drift into isolation or hog solo arcs. Needs explicit ties to group goals.
  \item \textbf{GM Tools:} Use Bonds, rival duels, and one-on-one omens to keep engagement tethered to the group.
\end{itemize}

\subsection{The Mixed}
\begin{itemize}
  \item \textbf{Focus:} Hybrid adaptability—both support and lead.
  \item \textbf{Strengths:} Bridges gaps in group dynamics; excels in team tactics.
  \item \textbf{Risks:} May feel overshadowed by extreme specialists.
  \item \textbf{GM Tools:} Spotlight them when versatility matters: flexible magic, leadership, negotiation, or sudden pivots.
\end{itemize}

\subsection{The Mastermind}
\begin{itemize}
  \item \textbf{Focus:} Schemes, plans, and command over the long arc.
  \item \textbf{Strengths:} Drives strategic play, coordinates others, excels in intrigue.
  \item \textbf{Risks:} May over-plan or dominate spotlight with metagame thinking.
  \item \textbf{GM Tools:} Challenge them with shifting information, patron demands, and rivals who anticipate their moves.
\end{itemize}

\subsection{Balancing Archetypes}
\begin{itemize}
  \item A healthy table mixes all three archetypes, ensuring solo tension, group cohesion, and strategic play coexist.
  \item Encourage players to flex between archetypes scene by scene; they are fluid, not fixed.
  \item Spotlight balance: in a 3-hour session, each archetype should anchor at least one major scene.
\end{itemize}

\subsection{GM Quick Cues}
\begin{itemize}
  \item Use archetypes as a lens for framing scenes: duel for Solos, shifting tactics for Mixed, grand reveals for Masterminds.
  \item When spotlight imbalances arise, rotate complications or bonds to another archetype’s strength.
  \item Archetypes inform how patrons and factions court PCs: Solos as champions, Mixed as envoys, Masterminds as plotters.
\end{itemize}


\section{Player Archetypes at the Table}
\label{sec:player-archetypes}

Fate’s Edge gameplay flexes to different player archetypes. These roles are not strict classes but rather \textbf{approaches to play} that help a group balance spotlight and tone.

\subsection{The Solo}
\begin{itemize}
  \item \textbf{Focus:} Mastery of self, independence, one-on-one drama.
  \item \textbf{Strengths:} Quick spotlight scenes, duelists, scouts, specialists.
  \item \textbf{Risks:} Can drift into isolation or hog solo arcs. Needs explicit ties to group goals.
  \item \textbf{GM Tools:} Use Bonds, rival duels, and one-on-one omens to keep engagement tethered to the group.
\end{itemize}

\subsection{The Mixed}
\begin{itemize}
  \item \textbf{Focus:} Hybrid adaptability—both support and lead.
  \item \textbf{Strengths:} Bridges gaps in group dynamics; excels in team tactics.
  \item \textbf{Risks:} May feel overshadowed by extreme specialists.
  \item \textbf{GM Tools:} Spotlight them when versatility matters: flexible magic, leadership, negotiation, or sudden pivots.
\end{itemize}

\subsection{The Mastermind}
\begin{itemize}
  \item \textbf{Focus:} Schemes, plans, and command over the long arc.
  \item \textbf{Strengths:} Drives strategic play, coordinates others, excels in intrigue.
  \item \textbf{Risks:} May over-plan or dominate spotlight with metagame thinking.
  \item \textbf{GM Tools:} Challenge them with shifting information, patron demands, and rivals who anticipate their moves.
\end{itemize}

\subsection{Balancing Archetypes}
\begin{itemize}
  \item A healthy table mixes all three archetypes, ensuring solo tension, group cohesion, and strategic play coexist.
  \item Encourage players to flex between archetypes scene by scene; they are fluid, not fixed.
  \item Spotlight balance: in a 3-hour session, each archetype should anchor at least one major scene.
\end{itemize}

\subsection{GM Quick Cues}
\begin{itemize}
  \item Use archetypes as a lens for framing scenes: duel for Solos, shifting tactics for Mixed, grand reveals for Masterminds.
  \item When spotlight imbalances arise, rotate complications or bonds to another archetype’s strength.
  \item Archetypes inform how patrons and factions court PCs: Solos as champions, Mixed as envoys, Masterminds as plotters.
\end{itemize}

\subsection*{Terrestrial Patrons}
Not all patrons are gods, demons, or cosmic forces. Mortals create power too: nobles, guilds, conspiracies, temples, syndicates, and commanders. A Terrestrial Patron represents an ongoing relationship with a powerful mortal faction.

\begin{description}[leftmargin=*]
\item[\textbf{Why They Matter}]
A terrestrial patron doesn't grant magic. They grant \emph{leverage}: protection, resources, sanctuary, information, and political shifts. Their rewards arrive through fiction and consequence.

\item[\textbf{Obligation (Terrestrial)}]
Use the same Obligation track, but the consequences are social, legal, or economic instead of supernatural.

When you call on a Patron's influence, add +1 Obligation.

\item[\textbf{Getting a Patron}]
To gain a Terrestrial Patron, complete one of the following:
\begin{itemize}[noitemsep]
\item a major job for them,
\item a sworn Oath,
\item legal or financial binding,
\item blackmail or shared crime.
\end{itemize}

Mark them on your sheet and write one sentence:
\textit{``They want me because \_\_\_''}

\item[\textbf{Perks}]
Each Patron offers 2--3 repeatable benefits, such as:
\begin{itemize}[noitemsep]
\item sanctuary,
\item legal relief,
\item black market goods,
\item elite followers,
\item forged documents,
\item military backing,
\item rumors and spywork.
\end{itemize}

Using a Perk never requires a roll. Fate has already been paid—it simply comes with Obligation.

\item[\textbf{Demands}]
Terrestrial Patrons always want something back:
\begin{itemize}[noitemsep]
\item silence,
\item loyalty,
\item a job,
\item a name,
\item a secret.
\end{itemize}

Refusing raises Obligation by 1. Betrayal may have immediate consequences.

\end{description}

\subsection*{When Obligation Fills}
At 6 Obligation, the Patron acts. This is \textbf{not} optional.

Choose one:
\begin{itemize}[noitemsep]
\item You do a job you cannot refuse.
\item You pay a severe price (legal, social, material).
\item They strike first—reputation, warrants, bounty, blackmail.
\end{itemize}

Reduce Obligation to 3 after the consequence lands.

\subsection*{Cutting Ties}
You may sever a terrestrial tie, but doing so has fallout:
\begin{itemize}[noitemsep]
\item lose all current perks,
\item gain a new Rival faction,
\item take a Curse, Bounty, or Scandal that follows you.
\end{itemize}

Some patrons never forgive. Others can be bought off.

\subsection*{Redemption or Favor}
If you do something monumental for them—beyond what was asked—reduce Obligation by 2 and gain a permanent Favor:
\begin{itemize}[noitemsep]
\item title,
\item land,
\item permanent access,
\item unique asset.
\end{itemize}

\subsection*{Quick Example}
The Black Ledger smuggling syndicate gives sanctuary and illegal gear. Rellan calls on the Ledger for a smuggled border crossing. The GM rules it succeeds automatically, but adds +1 Obligation. Rellan now owes the Ledger. Later, the Ledger demands he silence a witness. If he refuses, Obligation rises again. If Obligation ever reaches 6, the Ledger collects: accounts frozen, bounty posted, or a rival informant sent after him.