
% --- Fate's Edge SRD — Section 7: Story Beats & Boons ---
% Include this file from your main .tex with: 
% --- Fate's Edge SRD — Section 7: Story Beats & Boons ---
% Include this file from your main .tex with: 
% --- Fate's Edge SRD — Section 7: Story Beats & Boons ---
% Include this file from your main .tex with: 
% --- Fate's Edge SRD — Section 7: Story Beats & Boons ---
% Include this file from your main .tex with: \input{07-complications.tex}

\section{Story Beats and Boons}

\subsection{Story Beats (SB)}
Story Beats are the core dramatic currency of Fate’s Edge. They represent the risks, twists, and unintended consequences that emerge from every action.

\subsubsection{Generating SB}
\begin{itemize}
  \item Each die result of \textbf{1} generates 1 SB for the GM.
  \item Re-rolling a 1 does not remove its SB; if the re-rolled die also shows 1, it generates additional SB.
  \item Certain Harm effects or narrative triggers may also generate SB on future rolls.
\end{itemize}

\subsubsection{Spending SB}
The GM spends SB to introduce complications:
\begin{description}[leftmargin=1.5em, style=nextline]
  \item[Escalation:] Draw more enemies, raise the stakes.
  \item[Exhaustion:] Drain time, resources, or positioning.
  \item[Exposure:] Reveal hidden actions, alert foes.
  \item[Collateral:] Harm or danger spills onto allies, innocents, or surroundings.
\end{description}

\textbf{Menu of SB Spends (Guideline):}
\begin{itemize}
  \item 1 SB: Minor pressure (noise, trace, +1 Supply segment).
  \item 2 SB: Moderate setback (alarm raised, lose cover, lesser foe arrives).
  \item 3 SB: Serious trouble (reinforcements, broken gear, major twist).
  \item 4+ SB: Scene-shaping turn (trap springs, authority arrives, narrative shift).
\end{itemize}

\subsubsection{Limits on SB}
\begin{itemize}
  \item \textbf{Base SB Budget:} 4 + Character Tier (e.g. Tier I = 5, Tier II = 6).
  \item \textbf{Scene Limits:} 12 SB max for standard scenes, 16 SB max for climactic scenes.
  \item \textbf{Session Limit:} 20 SB total per session.
  \item \textbf{Threads:} Max concurrent complication threads = Tier + 1.
\end{itemize}

\subsection{Boons}
Boons are the player-facing reward for meaningful failure or bond-driven actions. They represent insight, opportunity, or a sudden edge.

\subsubsection{Earning Boons}
\begin{itemize}
	\item On a \textbf{Miss} (0 successes), if the stakes are meaningful and SB is spent/banked, the player gains 2 Boons, on a \textbf{Partial} success (1 or more successes < DV) award 1 Boon.
  \item On a bond-driven assist with an \textbf{Intricate Description}, the player may gain 1 Boon (once per bond per session).
  \item Other narrative rewards: The GM may award Boons for spotlighting bonds, sacrifices, or creative solutions.
\end{itemize}

\subsubsection{Spending Boons}
\begin{itemize}
  \item Re-roll a single die in a pool.
  \item Activate an on-screen Asset.
  \item Power a Rite or magical ability.
  \item Improve Position by 1 step.
  \item Convert into XP: Once per session, during downtime, convert 2 Boons → 1 XP (max 2 XP).
\end{itemize}

\subsubsection{Limits on Boons}
\begin{itemize}
  \item Hold up to 5 Boons at a time.
  \item At the end of a scene, reduce held Boons to 2 (excess are lost).
  \item Max 2 Boons earned from failure per scene per character.
\end{itemize}

\subsection{Interplay: SB and Boons}
\begin{itemize}
  \item SB fuels the GM’s complications; Boons fuel the players’ resilience.
  \item Every roll potentially adds to both sides: Successes drive story, 1s feed the GM, and misses feed the players.
  \item This dual economy ensures narrative momentum—every result matters.
\end{itemize}

\subsection{Example}
\emph{Kael rolls 6 dice to pick a lock under watch. Results: \{9, 7, 5, 3, 1, 1\}. Successes = 2, SB = 2. He succeeds, but the GM spends 1 SB for a squealing hinge and banks 1 SB for guards incoming. Because it was a Success \& Cost, no Boon is awarded. If Kael had missed entirely, he would have gained 2 Boon or 1 on a partial success.}

\subsection{Embrace the Void (Major Talent, 8 XP)}
\paragraph{For those who walk the knife-edge between power and damnation.}

\textbf{Prerequisites:} Any character with 2+ levels in a skill tied to their Patron's domain, and at least one segment of \indexterm{Obligation} to that Patron.

\textbf{Effect:} Once per session, you may choose to fully embrace your Patron’s corrupting influence to gain significant temporary power.

\textbf{Activation:}
\begin{itemize}
    \item Immediately mark 2 segments of \indexterm{Obligation} to your chosen Patron.
    \item Mark 1 segment on that Patron’s specific \indexterm{Corruption Table}.
    \item Gain one of the following benefits for the remainder of the scene:
    \begin{itemize}
        \item \textbf{Power Surge:} +1 die and +1 effect on all rolls related to that Patron’s domain.
        \item \textbf{Defiance:} Immunity to one specific consequence type (fear, charm, physical harm, etc.) for the scene.
        \item \textbf{Forbidden Rite:} Use one Rite of that Patron without marking additional Obligation (Backlash still applies).
        \item \textbf{Tempting Tongue:} +1 effect on all social manipulations for the scene.
    \end{itemize}
\end{itemize}

\textbf{Cost:}
\begin{itemize}
    \item A permanent mark on your character sheet indicating embraced corruption.
    \item Your Patron’s influence deepens: the GM gains +1 \indexterm{Story Beat} to spend against you whenever that Patron is relevant.
    \item You must roleplay the corruption’s manifestations in future scenes.
    \item This Talent cannot be activated again until you clear at least 2 segments of Obligation through proper service to your Patron.
\end{itemize}

\textbf{Narrative Integration:} This Talent represents the Faustian bargain at the heart of Patron magic—power for a price. Players gain agency over their corruption, while ensuring that it always carries meaningful consequences.

\begin{tcolorbox}[title=Example Corruptions by Patron]
\textbf{Ikasha (Shadows):} You cannot lie about secrets you have learned; you compulsively seek hidden truths. \\
\textbf{Aliyah (Chains \& Curses):} You bear a visible corruption mark; you crave increasingly dangerous curses to feel alive. \\
\textbf{Raéyn (Sea):} You draw the attention of sea creatures; you suffer --1 die on land-based actions. \\
\textbf{The Sealed Gate:} You attract entities seeking to cross thresholds; you compulsively seal or lock doors, gates, and bindings.
\end{tcolorbox}


\section{Story Beats and Boons}

\subsection{Story Beats (SB)}
Story Beats are the core dramatic currency of Fate’s Edge. They represent the risks, twists, and unintended consequences that emerge from every action.

\subsubsection{Generating SB}
\begin{itemize}
  \item Each die result of \textbf{1} generates 1 SB for the GM.
  \item Re-rolling a 1 does not remove its SB; if the re-rolled die also shows 1, it generates additional SB.
  \item Certain Harm effects or narrative triggers may also generate SB on future rolls.
\end{itemize}

\subsubsection{Spending SB}
The GM spends SB to introduce complications:
\begin{description}[leftmargin=1.5em, style=nextline]
  \item[Escalation:] Draw more enemies, raise the stakes.
  \item[Exhaustion:] Drain time, resources, or positioning.
  \item[Exposure:] Reveal hidden actions, alert foes.
  \item[Collateral:] Harm or danger spills onto allies, innocents, or surroundings.
\end{description}

\textbf{Menu of SB Spends (Guideline):}
\begin{itemize}
  \item 1 SB: Minor pressure (noise, trace, +1 Supply segment).
  \item 2 SB: Moderate setback (alarm raised, lose cover, lesser foe arrives).
  \item 3 SB: Serious trouble (reinforcements, broken gear, major twist).
  \item 4+ SB: Scene-shaping turn (trap springs, authority arrives, narrative shift).
\end{itemize}

\subsubsection{Limits on SB}
\begin{itemize}
  \item \textbf{Base SB Budget:} 4 + Character Tier (e.g. Tier I = 5, Tier II = 6).
  \item \textbf{Scene Limits:} 12 SB max for standard scenes, 16 SB max for climactic scenes.
  \item \textbf{Session Limit:} 20 SB total per session.
  \item \textbf{Threads:} Max concurrent complication threads = Tier + 1.
\end{itemize}

\subsection{Boons}
Boons are the player-facing reward for meaningful failure or bond-driven actions. They represent insight, opportunity, or a sudden edge.

\subsubsection{Earning Boons}
\begin{itemize}
	\item On a \textbf{Miss} (0 successes), if the stakes are meaningful and SB is spent/banked, the player gains 2 Boons, on a \textbf{Partial} success (1 or more successes < DV) award 1 Boon.
  \item On a bond-driven assist with an \textbf{Intricate Description}, the player may gain 1 Boon (once per bond per session).
  \item Other narrative rewards: The GM may award Boons for spotlighting bonds, sacrifices, or creative solutions.
\end{itemize}

\subsubsection{Spending Boons}
\begin{itemize}
  \item Re-roll a single die in a pool.
  \item Activate an on-screen Asset.
  \item Power a Rite or magical ability.
  \item Improve Position by 1 step.
  \item Convert into XP: Once per session, during downtime, convert 2 Boons → 1 XP (max 2 XP).
\end{itemize}

\subsubsection{Limits on Boons}
\begin{itemize}
  \item Hold up to 5 Boons at a time.
  \item At the end of a scene, reduce held Boons to 2 (excess are lost).
  \item Max 2 Boons earned from failure per scene per character.
\end{itemize}

\subsection{Interplay: SB and Boons}
\begin{itemize}
  \item SB fuels the GM’s complications; Boons fuel the players’ resilience.
  \item Every roll potentially adds to both sides: Successes drive story, 1s feed the GM, and misses feed the players.
  \item This dual economy ensures narrative momentum—every result matters.
\end{itemize}

\subsection{Example}
\emph{Kael rolls 6 dice to pick a lock under watch. Results: \{9, 7, 5, 3, 1, 1\}. Successes = 2, SB = 2. He succeeds, but the GM spends 1 SB for a squealing hinge and banks 1 SB for guards incoming. Because it was a Success \& Cost, no Boon is awarded. If Kael had missed entirely, he would have gained 2 Boon or 1 on a partial success.}

\subsection{Embrace the Void (Major Talent, 8 XP)}
\paragraph{For those who walk the knife-edge between power and damnation.}

\textbf{Prerequisites:} Any character with 2+ levels in a skill tied to their Patron's domain, and at least one segment of \indexterm{Obligation} to that Patron.

\textbf{Effect:} Once per session, you may choose to fully embrace your Patron’s corrupting influence to gain significant temporary power.

\textbf{Activation:}
\begin{itemize}
    \item Immediately mark 2 segments of \indexterm{Obligation} to your chosen Patron.
    \item Mark 1 segment on that Patron’s specific \indexterm{Corruption Table}.
    \item Gain one of the following benefits for the remainder of the scene:
    \begin{itemize}
        \item \textbf{Power Surge:} +1 die and +1 effect on all rolls related to that Patron’s domain.
        \item \textbf{Defiance:} Immunity to one specific consequence type (fear, charm, physical harm, etc.) for the scene.
        \item \textbf{Forbidden Rite:} Use one Rite of that Patron without marking additional Obligation (Backlash still applies).
        \item \textbf{Tempting Tongue:} +1 effect on all social manipulations for the scene.
    \end{itemize}
\end{itemize}

\textbf{Cost:}
\begin{itemize}
    \item A permanent mark on your character sheet indicating embraced corruption.
    \item Your Patron’s influence deepens: the GM gains +1 \indexterm{Story Beat} to spend against you whenever that Patron is relevant.
    \item You must roleplay the corruption’s manifestations in future scenes.
    \item This Talent cannot be activated again until you clear at least 2 segments of Obligation through proper service to your Patron.
\end{itemize}

\textbf{Narrative Integration:} This Talent represents the Faustian bargain at the heart of Patron magic—power for a price. Players gain agency over their corruption, while ensuring that it always carries meaningful consequences.

\begin{tcolorbox}[title=Example Corruptions by Patron]
\textbf{Ikasha (Shadows):} You cannot lie about secrets you have learned; you compulsively seek hidden truths. \\
\textbf{Aliyah (Chains \& Curses):} You bear a visible corruption mark; you crave increasingly dangerous curses to feel alive. \\
\textbf{Raéyn (Sea):} You draw the attention of sea creatures; you suffer --1 die on land-based actions. \\
\textbf{The Sealed Gate:} You attract entities seeking to cross thresholds; you compulsively seal or lock doors, gates, and bindings.
\end{tcolorbox}


\section{Story Beats and Boons}

\subsection{Story Beats (SB)}
Story Beats are the core dramatic currency of Fate’s Edge. They represent the risks, twists, and unintended consequences that emerge from every action.

\subsubsection{Generating SB}
\begin{itemize}
  \item Each die result of \textbf{1} generates 1 SB for the GM.
  \item Re-rolling a 1 does not remove its SB; if the re-rolled die also shows 1, it generates additional SB.
  \item Certain Harm effects or narrative triggers may also generate SB on future rolls.
\end{itemize}

\subsubsection{Spending SB}
The GM spends SB to introduce complications:
\begin{description}[leftmargin=1.5em, style=nextline]
  \item[Escalation:] Draw more enemies, raise the stakes.
  \item[Exhaustion:] Drain time, resources, or positioning.
  \item[Exposure:] Reveal hidden actions, alert foes.
  \item[Collateral:] Harm or danger spills onto allies, innocents, or surroundings.
\end{description}

\textbf{Menu of SB Spends (Guideline):}
\begin{itemize}
  \item 1 SB: Minor pressure (noise, trace, +1 Supply segment).
  \item 2 SB: Moderate setback (alarm raised, lose cover, lesser foe arrives).
  \item 3 SB: Serious trouble (reinforcements, broken gear, major twist).
  \item 4+ SB: Scene-shaping turn (trap springs, authority arrives, narrative shift).
\end{itemize}

\subsubsection{Limits on SB}
\begin{itemize}
  \item \textbf{Base SB Budget:} 4 + Character Tier (e.g. Tier I = 5, Tier II = 6).
  \item \textbf{Scene Limits:} 12 SB max for standard scenes, 16 SB max for climactic scenes.
  \item \textbf{Session Limit:} 20 SB total per session.
  \item \textbf{Threads:} Max concurrent complication threads = Tier + 1.
\end{itemize}

\subsection{Boons}
Boons are the player-facing reward for meaningful failure or bond-driven actions. They represent insight, opportunity, or a sudden edge.

\subsubsection{Earning Boons}
\begin{itemize}
	\item On a \textbf{Miss} (0 successes), if the stakes are meaningful and SB is spent/banked, the player gains 2 Boons, on a \textbf{Partial} success (1 or more successes < DV) award 1 Boon.
  \item On a bond-driven assist with an \textbf{Intricate Description}, the player may gain 1 Boon (once per bond per session).
  \item Other narrative rewards: The GM may award Boons for spotlighting bonds, sacrifices, or creative solutions.
\end{itemize}

\subsubsection{Spending Boons}
\begin{itemize}
  \item Re-roll a single die in a pool.
  \item Activate an on-screen Asset.
  \item Power a Rite or magical ability.
  \item Improve Position by 1 step.
  \item Convert into XP: Once per session, during downtime, convert 2 Boons → 1 XP (max 2 XP).
\end{itemize}

\subsubsection{Limits on Boons}
\begin{itemize}
  \item Hold up to 5 Boons at a time.
  \item At the end of a scene, reduce held Boons to 2 (excess are lost).
  \item Max 2 Boons earned from failure per scene per character.
\end{itemize}

\subsection{Interplay: SB and Boons}
\begin{itemize}
  \item SB fuels the GM’s complications; Boons fuel the players’ resilience.
  \item Every roll potentially adds to both sides: Successes drive story, 1s feed the GM, and misses feed the players.
  \item This dual economy ensures narrative momentum—every result matters.
\end{itemize}

\subsection{Example}
\emph{Kael rolls 6 dice to pick a lock under watch. Results: \{9, 7, 5, 3, 1, 1\}. Successes = 2, SB = 2. He succeeds, but the GM spends 1 SB for a squealing hinge and banks 1 SB for guards incoming. Because it was a Success \& Cost, no Boon is awarded. If Kael had missed entirely, he would have gained 2 Boon or 1 on a partial success.}

\subsection{Embrace the Void (Major Talent, 8 XP)}
\paragraph{For those who walk the knife-edge between power and damnation.}

\textbf{Prerequisites:} Any character with 2+ levels in a skill tied to their Patron's domain, and at least one segment of \indexterm{Obligation} to that Patron.

\textbf{Effect:} Once per session, you may choose to fully embrace your Patron’s corrupting influence to gain significant temporary power.

\textbf{Activation:}
\begin{itemize}
    \item Immediately mark 2 segments of \indexterm{Obligation} to your chosen Patron.
    \item Mark 1 segment on that Patron’s specific \indexterm{Corruption Table}.
    \item Gain one of the following benefits for the remainder of the scene:
    \begin{itemize}
        \item \textbf{Power Surge:} +1 die and +1 effect on all rolls related to that Patron’s domain.
        \item \textbf{Defiance:} Immunity to one specific consequence type (fear, charm, physical harm, etc.) for the scene.
        \item \textbf{Forbidden Rite:} Use one Rite of that Patron without marking additional Obligation (Backlash still applies).
        \item \textbf{Tempting Tongue:} +1 effect on all social manipulations for the scene.
    \end{itemize}
\end{itemize}

\textbf{Cost:}
\begin{itemize}
    \item A permanent mark on your character sheet indicating embraced corruption.
    \item Your Patron’s influence deepens: the GM gains +1 \indexterm{Story Beat} to spend against you whenever that Patron is relevant.
    \item You must roleplay the corruption’s manifestations in future scenes.
    \item This Talent cannot be activated again until you clear at least 2 segments of Obligation through proper service to your Patron.
\end{itemize}

\textbf{Narrative Integration:} This Talent represents the Faustian bargain at the heart of Patron magic—power for a price. Players gain agency over their corruption, while ensuring that it always carries meaningful consequences.

\begin{tcolorbox}[title=Example Corruptions by Patron]
\textbf{Ikasha (Shadows):} You cannot lie about secrets you have learned; you compulsively seek hidden truths. \\
\textbf{Aliyah (Chains \& Curses):} You bear a visible corruption mark; you crave increasingly dangerous curses to feel alive. \\
\textbf{Raéyn (Sea):} You draw the attention of sea creatures; you suffer --1 die on land-based actions. \\
\textbf{The Sealed Gate:} You attract entities seeking to cross thresholds; you compulsively seal or lock doors, gates, and bindings.
\end{tcolorbox}


\section{Story Beats and Boons}

\subsection{Story Beats (SB)}
Story Beats are the core dramatic currency of Fate’s Edge. They represent the risks, twists, and unintended consequences that emerge from every action.

\subsubsection{Generating SB}
\begin{itemize}
  \item Each die result of \textbf{1} generates 1 SB for the GM.
  \item Re-rolling a 1 does not remove its SB; if the re-rolled die also shows 1, it generates additional SB.
  \item Certain Harm effects or narrative triggers may also generate SB on future rolls.
\end{itemize}

\subsubsection{Spending SB}
The GM spends SB to introduce complications:
\begin{description}[leftmargin=1.5em, style=nextline]
  \item[Escalation:] Draw more enemies, raise the stakes.
  \item[Exhaustion:] Drain time, resources, or positioning.
  \item[Exposure:] Reveal hidden actions, alert foes.
  \item[Collateral:] Harm or danger spills onto allies, innocents, or surroundings.
\end{description}

\textbf{Menu of SB Spends (Guideline):}
\begin{itemize}
  \item 1 SB: Minor pressure (noise, trace, +1 Supply segment).
  \item 2 SB: Moderate setback (alarm raised, lose cover, lesser foe arrives).
  \item 3 SB: Serious trouble (reinforcements, broken gear, major twist).
  \item 4+ SB: Scene-shaping turn (trap springs, authority arrives, narrative shift).
\end{itemize}

\subsubsection{Limits on SB}
\begin{itemize}
  \item \textbf{Base SB Budget:} 4 + Character Tier (e.g. Tier I = 5, Tier II = 6).
  \item \textbf{Scene Limits:} 12 SB max for standard scenes, 16 SB max for climactic scenes.
  \item \textbf{Session Limit:} 20 SB total per session.
  \item \textbf{Threads:} Max concurrent complication threads = Tier + 1.
\end{itemize}

\subsection{Boons}
Boons are the player-facing reward for meaningful failure or bond-driven actions. They represent insight, opportunity, or a sudden edge.

\subsubsection{Earning Boons}
\begin{itemize}
	\item On a \textbf{Miss} (0 successes), if the stakes are meaningful and SB is spent/banked, the player gains 2 Boons, on a \textbf{Partial} success (1 or more successes < DV) award 1 Boon.
  \item On a bond-driven assist with an \textbf{Intricate Description}, the player may gain 1 Boon (once per bond per session).
  \item Other narrative rewards: The GM may award Boons for spotlighting bonds, sacrifices, or creative solutions.
\end{itemize}

\subsubsection{Spending Boons}
\begin{itemize}
  \item Re-roll a single die in a pool.
  \item Activate an on-screen Asset.
  \item Power a Rite or magical ability.
  \item Improve Position by 1 step.
  \item Convert into XP: Once per session, during downtime, convert 2 Boons → 1 XP (max 2 XP).
\end{itemize}

\subsubsection{Limits on Boons}
\begin{itemize}
  \item Hold up to 5 Boons at a time.
  \item At the end of a scene, reduce held Boons to 2 (excess are lost).
  \item Max 2 Boons earned from failure per scene per character.
\end{itemize}

\subsection{Interplay: SB and Boons}
\begin{itemize}
  \item SB fuels the GM’s complications; Boons fuel the players’ resilience.
  \item Every roll potentially adds to both sides: Successes drive story, 1s feed the GM, and misses feed the players.
  \item This dual economy ensures narrative momentum—every result matters.
\end{itemize}

\subsection{Example}
\emph{Kael rolls 6 dice to pick a lock under watch. Results: \{9, 7, 5, 3, 1, 1\}. Successes = 2, SB = 2. He succeeds, but the GM spends 1 SB for a squealing hinge and banks 1 SB for guards incoming. Because it was a Success \& Cost, no Boon is awarded. If Kael had missed entirely, he would have gained 2 Boon or 1 on a partial success.}

\subsection{Embrace the Void (Major Talent, 8 XP)}
\paragraph{For those who walk the knife-edge between power and damnation.}

\textbf{Prerequisites:} Any character with 2+ levels in a skill tied to their Patron's domain, and at least one segment of \indexterm{Obligation} to that Patron.

\textbf{Effect:} Once per session, you may choose to fully embrace your Patron’s corrupting influence to gain significant temporary power.

\textbf{Activation:}
\begin{itemize}
    \item Immediately mark 2 segments of \indexterm{Obligation} to your chosen Patron.
    \item Mark 1 segment on that Patron’s specific \indexterm{Corruption Table}.
    \item Gain one of the following benefits for the remainder of the scene:
    \begin{itemize}
        \item \textbf{Power Surge:} +1 die and +1 effect on all rolls related to that Patron’s domain.
        \item \textbf{Defiance:} Immunity to one specific consequence type (fear, charm, physical harm, etc.) for the scene.
        \item \textbf{Forbidden Rite:} Use one Rite of that Patron without marking additional Obligation (Backlash still applies).
        \item \textbf{Tempting Tongue:} +1 effect on all social manipulations for the scene.
    \end{itemize}
\end{itemize}

\textbf{Cost:}
\begin{itemize}
    \item A permanent mark on your character sheet indicating embraced corruption.
    \item Your Patron’s influence deepens: the GM gains +1 \indexterm{Story Beat} to spend against you whenever that Patron is relevant.
    \item You must roleplay the corruption’s manifestations in future scenes.
    \item This Talent cannot be activated again until you clear at least 2 segments of Obligation through proper service to your Patron.
\end{itemize}

\textbf{Narrative Integration:} This Talent represents the Faustian bargain at the heart of Patron magic—power for a price. Players gain agency over their corruption, while ensuring that it always carries meaningful consequences.

\begin{tcolorbox}[title=Example Corruptions by Patron]
\textbf{Ikasha (Shadows):} You cannot lie about secrets you have learned; you compulsively seek hidden truths. \\
\textbf{Aliyah (Chains \& Curses):} You bear a visible corruption mark; you crave increasingly dangerous curses to feel alive. \\
\textbf{Raéyn (Sea):} You draw the attention of sea creatures; you suffer --1 die on land-based actions. \\
\textbf{The Sealed Gate:} You attract entities seeking to cross thresholds; you compulsively seal or lock doors, gates, and bindings.
\end{tcolorbox}
