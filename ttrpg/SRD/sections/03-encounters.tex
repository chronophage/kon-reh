
% --- Fate's Edge SRD — Section 3: Combat System ---
% Include this file from your main .tex with: 
% --- Fate's Edge SRD — Section 3: Combat System ---
% Include this file from your main .tex with: 
% --- Fate's Edge SRD — Section 3: Combat System ---
% Include this file from your main .tex with: 
% --- Fate's Edge SRD — Section 3: Combat System ---
% Include this file from your main .tex with: \input{03-combat.tex}

\section{Combat System}

\subsection{Core Philosophy}
Combat in Fate's Edge is not a separate mini-game; it is simply conflict under sharper focus. 
It uses the same dice pool system and SB economy as all other actions. 
The rules are designed to emphasize narrative consequence, positional play, and risk management.

\subsection{Structure of Combat}
\begin{itemize}
  \item \textbf{Rounds:} Each round represents a few seconds of action.
  \item \textbf{Turns:} Each participant takes one significant action per round.
  \item \textbf{Scenes:} A battle is one scene unless the fiction dictates otherwise.
\end{itemize}

\subsection{Taking Action}
On your turn, declare intent and method as normal:
\begin{enumerate}
  \item \textbf{Set Position:} The GM decides if you are Controlled, Risky, or Desperate.
  \item \textbf{Build Pool:} Attribute + Skill (+ gear, + assists, +1 from Imbuement if active).
  \item \textbf{Roll:} Each 6+ is a success. Each 1 generates SB.
  \item \textbf{Resolve:} Successes vs DV, SB spent by GM, Position/Effect applied.
\end{enumerate}

\subsection{Position \& Effect}
\begin{description}[leftmargin=1.5em]
  \item[Controlled] You act from safety or advantage. Failure still leaves you options.
  \item[Risky] Standard case. Failure has teeth, but not ruin.
  \item[Desperate] High stakes. Failure is severe; success may bring extra XP (mark Desperate use).
\end{description}
% =========================
% HEALTH, FATIGUE, & HARM (REVISED)
% =========================
\section{Health, Fatigue, \& Harm (Revised)}
\label{sec:health-fatigue-harm-rev}

\subsection*{Tracks \& Caps}
\begin{itemize}
  \item \textbf{Fatigue Track}: boxes equal to \textbf{Body}.
  \item \textbf{Harm Levels}: as defined elsewhere in the SRD (\textbf{Harm 1}, \textbf{Harm 2}, \textbf{Harm 3}).
\end{itemize}

\subsection*{Fatigue $\rightarrow$ Harm Conversion}
Whenever you would mark Fatigue and your Fatigue Track \emph{fills} (all boxes marked):
\begin{enumerate}
  \item \textbf{Increase} your \textbf{Harm} by one level (e.g., 0$\rightarrow$Harm~1, Harm~1$\rightarrow$Harm~2, Harm~2$\rightarrow$Harm~3).
  \item \textbf{Clear all Fatigue} (erase the Fatigue Track back to 0).
\end{enumerate}
This conversion can occur multiple times in a scene. Effects of Harm tier (disadvantage, action limits, incapacitation at Harm~3, etc.) follow your existing SRD.

\subsection*{Taking Fatigue}
Mark Fatigue for strain, exertion, travel, magic costs, or \S\ref{sec:obligation-overflow-rev} overflow. Fatigue can exceed remaining boxes only to \emph{trigger} conversion; any excess is ignored after the Harm increase and Fatigue clear.

\subsection*{Recovering Fatigue}
\begin{itemize}
  \item \textbf{Short Rest} (quiet watch, food/water): remove \textbf{2 Fatigue}.
  \item \textbf{Full Night}: remove \textbf{all Fatigue}.
\end{itemize}
\emph{Fatigue recovery does not remove Harm.} Recover Harm via your normal medical/ritual rules in the SRD.

\subsection*{Mitigation (Optional Dials)}
\begin{itemize}
  \item \textbf{Soak/Ward}: Before marking Fatigue, reduce it by 1--2 (to a minimum of 0) if protected by armor/boons/rites.
  \item \textbf{Convert}: Some effects may convert incoming \textbf{Harm 1} to \textbf{2 Fatigue}; if this \emph{fills} the track, convert as normal.
\end{itemize}

\paragraph{Effect}
Effect is narrative reach:
\begin{itemize}
  \item \textbf{Limited:} Scratch or slow progress.
  \item \textbf{Standard:} Expected impact (downing a guard, disabling a lock).
  \item \textbf{Great:} Overwhelming impact, bigger than expected.
\end{itemize}

\subsection{Damage \& Consequences}
When you take harm:
\begin{itemize}
  \item \textbf{Level 1 Harm:} Minor injury or hindrance. -1 die to related actions.
  \item \textbf{Level 2 Harm:} Serious wound. -1 die to most actions until treated.
  \item \textbf{Level 3 Harm:} Critical injury. You are incapacitated or dying.
\end{itemize}
Harm may be resisted (roll Attribute vs DV 3; 1s generate SB). On a hit, reduce harm by one level.

\subsection{Teamwork}
\begin{itemize}
  \item \textbf{Assist:} Spend 1 Stress or Boon to add +1 die. Max +3 dice from assists.
  \item \textbf{Setup:} Make a roll to improve another’s Position or Effect.
  \item \textbf{Protect:} Take harm or consequence meant for another.
\end{itemize}

\subsection{GM Guidance for SB in Combat}
Use SB to escalate combat fiction:
\begin{itemize}
  \item \textbf{1 SB:} Reinforce enemy cover, minor injury, reveal new foe.
  \item \textbf{2 SB:} Reinforcements arrive, key gear breaks, enemy gains +1 die.
  \item \textbf{3 SB:} Enemy unleashes a Rite or summon, terrain shifts, ally is endangered.
  \item \textbf{4+ SB:} Scene twists—fires spread, the floor collapses, Patron omens manifest.
\end{itemize}

\subsection{Combat and Magic}
\begin{itemize}
  \item \textbf{Casting.} Casters spend one action to \emph{Weave} and another to \emph{Cast}. Requires the \emph{Caster’s Gift} talent. 
  \item \textbf{Rites.} Invoking a Rite takes one action. Players may \emph{Push It} for $+1$ Obligation to gain the listed benefit. 
  \item \textbf{Invokers.} Invokers perform Rites via Symbol. Ritual invocation takes $\text{DV}+1$ rounds and always marks $+1$ Obligation. Alternatively, they may \emph{Crack the Seal} to cast instantly by setting the Symbol to \textsc{Compromised} and marking $+2$ Obligation ($+3$ if High-Power). Invoker Rites cannot use \emph{Push It}. 
  \item \textbf{Imbuements.} Once per scene, spend one action to activate an Imbuement. For the remainder of the scene, gain $+1$ to one Weapon and one Thematic Skill. 
\end{itemize}

\subsection{Worked Example}
\emph{Kael swings his Imbued blade at a cultist (DV 2). He rolls 5 dice: 9, 7, 5, 2, 1.}
\begin{itemize}
  \item Successes = 3 (hit), SB = 1.
  \item GM grants success: cultist is cut down.
  \item GM spends 1 SB: ``Blood sprays across the sigil—energy flares, the summoning accelerates.''
\end{itemize}


\section{Combat System}

\subsection{Core Philosophy}
Combat in Fate's Edge is not a separate mini-game; it is simply conflict under sharper focus. 
It uses the same dice pool system and SB economy as all other actions. 
The rules are designed to emphasize narrative consequence, positional play, and risk management.

\subsection{Structure of Combat}
\begin{itemize}
  \item \textbf{Rounds:} Each round represents a few seconds of action.
  \item \textbf{Turns:} Each participant takes one significant action per round.
  \item \textbf{Scenes:} A battle is one scene unless the fiction dictates otherwise.
\end{itemize}

\subsection{Taking Action}
On your turn, declare intent and method as normal:
\begin{enumerate}
  \item \textbf{Set Position:} The GM decides if you are Controlled, Risky, or Desperate.
  \item \textbf{Build Pool:} Attribute + Skill (+ gear, + assists, +1 from Imbuement if active).
  \item \textbf{Roll:} Each 6+ is a success. Each 1 generates SB.
  \item \textbf{Resolve:} Successes vs DV, SB spent by GM, Position/Effect applied.
\end{enumerate}

\subsection{Position \& Effect}
\begin{description}[leftmargin=1.5em]
  \item[Controlled] You act from safety or advantage. Failure still leaves you options.
  \item[Risky] Standard case. Failure has teeth, but not ruin.
  \item[Desperate] High stakes. Failure is severe; success may bring extra XP (mark Desperate use).
\end{description}
% =========================
% HEALTH, FATIGUE, & HARM (REVISED)
% =========================
\section{Health, Fatigue, \& Harm (Revised)}
\label{sec:health-fatigue-harm-rev}

\subsection*{Tracks \& Caps}
\begin{itemize}
  \item \textbf{Fatigue Track}: boxes equal to \textbf{Body}.
  \item \textbf{Harm Levels}: as defined elsewhere in the SRD (\textbf{Harm 1}, \textbf{Harm 2}, \textbf{Harm 3}).
\end{itemize}

\subsection*{Fatigue $\rightarrow$ Harm Conversion}
Whenever you would mark Fatigue and your Fatigue Track \emph{fills} (all boxes marked):
\begin{enumerate}
  \item \textbf{Increase} your \textbf{Harm} by one level (e.g., 0$\rightarrow$Harm~1, Harm~1$\rightarrow$Harm~2, Harm~2$\rightarrow$Harm~3).
  \item \textbf{Clear all Fatigue} (erase the Fatigue Track back to 0).
\end{enumerate}
This conversion can occur multiple times in a scene. Effects of Harm tier (disadvantage, action limits, incapacitation at Harm~3, etc.) follow your existing SRD.

\subsection*{Taking Fatigue}
Mark Fatigue for strain, exertion, travel, magic costs, or \S\ref{sec:obligation-overflow-rev} overflow. Fatigue can exceed remaining boxes only to \emph{trigger} conversion; any excess is ignored after the Harm increase and Fatigue clear.

\subsection*{Recovering Fatigue}
\begin{itemize}
  \item \textbf{Short Rest} (quiet watch, food/water): remove \textbf{2 Fatigue}.
  \item \textbf{Full Night}: remove \textbf{all Fatigue}.
\end{itemize}
\emph{Fatigue recovery does not remove Harm.} Recover Harm via your normal medical/ritual rules in the SRD.

\subsection*{Mitigation (Optional Dials)}
\begin{itemize}
  \item \textbf{Soak/Ward}: Before marking Fatigue, reduce it by 1--2 (to a minimum of 0) if protected by armor/boons/rites.
  \item \textbf{Convert}: Some effects may convert incoming \textbf{Harm 1} to \textbf{2 Fatigue}; if this \emph{fills} the track, convert as normal.
\end{itemize}

\paragraph{Effect}
Effect is narrative reach:
\begin{itemize}
  \item \textbf{Limited:} Scratch or slow progress.
  \item \textbf{Standard:} Expected impact (downing a guard, disabling a lock).
  \item \textbf{Great:} Overwhelming impact, bigger than expected.
\end{itemize}

\subsection{Damage \& Consequences}
When you take harm:
\begin{itemize}
  \item \textbf{Level 1 Harm:} Minor injury or hindrance. -1 die to related actions.
  \item \textbf{Level 2 Harm:} Serious wound. -1 die to most actions until treated.
  \item \textbf{Level 3 Harm:} Critical injury. You are incapacitated or dying.
\end{itemize}
Harm may be resisted (roll Attribute vs DV 3; 1s generate SB). On a hit, reduce harm by one level.

\subsection{Teamwork}
\begin{itemize}
  \item \textbf{Assist:} Spend 1 Stress or Boon to add +1 die. Max +3 dice from assists.
  \item \textbf{Setup:} Make a roll to improve another’s Position or Effect.
  \item \textbf{Protect:} Take harm or consequence meant for another.
\end{itemize}

\subsection{GM Guidance for SB in Combat}
Use SB to escalate combat fiction:
\begin{itemize}
  \item \textbf{1 SB:} Reinforce enemy cover, minor injury, reveal new foe.
  \item \textbf{2 SB:} Reinforcements arrive, key gear breaks, enemy gains +1 die.
  \item \textbf{3 SB:} Enemy unleashes a Rite or summon, terrain shifts, ally is endangered.
  \item \textbf{4+ SB:} Scene twists—fires spread, the floor collapses, Patron omens manifest.
\end{itemize}

\subsection{Combat and Magic}
\begin{itemize}
  \item \textbf{Casting.} Casters spend one action to \emph{Weave} and another to \emph{Cast}. Requires the \emph{Caster’s Gift} talent. 
  \item \textbf{Rites.} Invoking a Rite takes one action. Players may \emph{Push It} for $+1$ Obligation to gain the listed benefit. 
  \item \textbf{Invokers.} Invokers perform Rites via Symbol. Ritual invocation takes $\text{DV}+1$ rounds and always marks $+1$ Obligation. Alternatively, they may \emph{Crack the Seal} to cast instantly by setting the Symbol to \textsc{Compromised} and marking $+2$ Obligation ($+3$ if High-Power). Invoker Rites cannot use \emph{Push It}. 
  \item \textbf{Imbuements.} Once per scene, spend one action to activate an Imbuement. For the remainder of the scene, gain $+1$ to one Weapon and one Thematic Skill. 
\end{itemize}

\subsection{Worked Example}
\emph{Kael swings his Imbued blade at a cultist (DV 2). He rolls 5 dice: 9, 7, 5, 2, 1.}
\begin{itemize}
  \item Successes = 3 (hit), SB = 1.
  \item GM grants success: cultist is cut down.
  \item GM spends 1 SB: ``Blood sprays across the sigil—energy flares, the summoning accelerates.''
\end{itemize}


\section{Combat System}

\subsection{Core Philosophy}
Combat in Fate's Edge is not a separate mini-game; it is simply conflict under sharper focus. 
It uses the same dice pool system and SB economy as all other actions. 
The rules are designed to emphasize narrative consequence, positional play, and risk management.

\subsection{Structure of Combat}
\begin{itemize}
  \item \textbf{Rounds:} Each round represents a few seconds of action.
  \item \textbf{Turns:} Each participant takes one significant action per round.
  \item \textbf{Scenes:} A battle is one scene unless the fiction dictates otherwise.
\end{itemize}

\subsection{Taking Action}
On your turn, declare intent and method as normal:
\begin{enumerate}
  \item \textbf{Set Position:} The GM decides if you are Controlled, Risky, or Desperate.
  \item \textbf{Build Pool:} Attribute + Skill (+ gear, + assists, +1 from Imbuement if active).
  \item \textbf{Roll:} Each 6+ is a success. Each 1 generates SB.
  \item \textbf{Resolve:} Successes vs DV, SB spent by GM, Position/Effect applied.
\end{enumerate}

\subsection{Position \& Effect}
\begin{description}[leftmargin=1.5em]
  \item[Controlled] You act from safety or advantage. Failure still leaves you options.
  \item[Risky] Standard case. Failure has teeth, but not ruin.
  \item[Desperate] High stakes. Failure is severe; success may bring extra XP (mark Desperate use).
\end{description}
% =========================
% HEALTH, FATIGUE, & HARM (REVISED)
% =========================
\section{Health, Fatigue, \& Harm (Revised)}
\label{sec:health-fatigue-harm-rev}

\subsection*{Tracks \& Caps}
\begin{itemize}
  \item \textbf{Fatigue Track}: boxes equal to \textbf{Body}.
  \item \textbf{Harm Levels}: as defined elsewhere in the SRD (\textbf{Harm 1}, \textbf{Harm 2}, \textbf{Harm 3}).
\end{itemize}

\subsection*{Fatigue $\rightarrow$ Harm Conversion}
Whenever you would mark Fatigue and your Fatigue Track \emph{fills} (all boxes marked):
\begin{enumerate}
  \item \textbf{Increase} your \textbf{Harm} by one level (e.g., 0$\rightarrow$Harm~1, Harm~1$\rightarrow$Harm~2, Harm~2$\rightarrow$Harm~3).
  \item \textbf{Clear all Fatigue} (erase the Fatigue Track back to 0).
\end{enumerate}
This conversion can occur multiple times in a scene. Effects of Harm tier (disadvantage, action limits, incapacitation at Harm~3, etc.) follow your existing SRD.

\subsection*{Taking Fatigue}
Mark Fatigue for strain, exertion, travel, magic costs, or \S\ref{sec:obligation-overflow-rev} overflow. Fatigue can exceed remaining boxes only to \emph{trigger} conversion; any excess is ignored after the Harm increase and Fatigue clear.

\subsection*{Recovering Fatigue}
\begin{itemize}
  \item \textbf{Short Rest} (quiet watch, food/water): remove \textbf{2 Fatigue}.
  \item \textbf{Full Night}: remove \textbf{all Fatigue}.
\end{itemize}
\emph{Fatigue recovery does not remove Harm.} Recover Harm via your normal medical/ritual rules in the SRD.

\subsection*{Mitigation (Optional Dials)}
\begin{itemize}
  \item \textbf{Soak/Ward}: Before marking Fatigue, reduce it by 1--2 (to a minimum of 0) if protected by armor/boons/rites.
  \item \textbf{Convert}: Some effects may convert incoming \textbf{Harm 1} to \textbf{2 Fatigue}; if this \emph{fills} the track, convert as normal.
\end{itemize}

\paragraph{Effect}
Effect is narrative reach:
\begin{itemize}
  \item \textbf{Limited:} Scratch or slow progress.
  \item \textbf{Standard:} Expected impact (downing a guard, disabling a lock).
  \item \textbf{Great:} Overwhelming impact, bigger than expected.
\end{itemize}

\subsection{Damage \& Consequences}
When you take harm:
\begin{itemize}
  \item \textbf{Level 1 Harm:} Minor injury or hindrance. -1 die to related actions.
  \item \textbf{Level 2 Harm:} Serious wound. -1 die to most actions until treated.
  \item \textbf{Level 3 Harm:} Critical injury. You are incapacitated or dying.
\end{itemize}
Harm may be resisted (roll Attribute vs DV 3; 1s generate SB). On a hit, reduce harm by one level.

\subsection{Teamwork}
\begin{itemize}
  \item \textbf{Assist:} Spend 1 Stress or Boon to add +1 die. Max +3 dice from assists.
  \item \textbf{Setup:} Make a roll to improve another’s Position or Effect.
  \item \textbf{Protect:} Take harm or consequence meant for another.
\end{itemize}

\subsection{GM Guidance for SB in Combat}
Use SB to escalate combat fiction:
\begin{itemize}
  \item \textbf{1 SB:} Reinforce enemy cover, minor injury, reveal new foe.
  \item \textbf{2 SB:} Reinforcements arrive, key gear breaks, enemy gains +1 die.
  \item \textbf{3 SB:} Enemy unleashes a Rite or summon, terrain shifts, ally is endangered.
  \item \textbf{4+ SB:} Scene twists—fires spread, the floor collapses, Patron omens manifest.
\end{itemize}

\subsection{Combat and Magic}
\begin{itemize}
  \item \textbf{Casting.} Casters spend one action to \emph{Weave} and another to \emph{Cast}. Requires the \emph{Caster’s Gift} talent. 
  \item \textbf{Rites.} Invoking a Rite takes one action. Players may \emph{Push It} for $+1$ Obligation to gain the listed benefit. 
  \item \textbf{Invokers.} Invokers perform Rites via Symbol. Ritual invocation takes $\text{DV}+1$ rounds and always marks $+1$ Obligation. Alternatively, they may \emph{Crack the Seal} to cast instantly by setting the Symbol to \textsc{Compromised} and marking $+2$ Obligation ($+3$ if High-Power). Invoker Rites cannot use \emph{Push It}. 
  \item \textbf{Imbuements.} Once per scene, spend one action to activate an Imbuement. For the remainder of the scene, gain $+1$ to one Weapon and one Thematic Skill. 
\end{itemize}

\subsection{Worked Example}
\emph{Kael swings his Imbued blade at a cultist (DV 2). He rolls 5 dice: 9, 7, 5, 2, 1.}
\begin{itemize}
  \item Successes = 3 (hit), SB = 1.
  \item GM grants success: cultist is cut down.
  \item GM spends 1 SB: ``Blood sprays across the sigil—energy flares, the summoning accelerates.''
\end{itemize}



\section*{Core DV Philosophy}

Difficulty Values (DV) in Fate's Edge represent \textbf{narrative weight}, not simulationist challenge. The DV system should answer: "How much does this matter to the story right now?"

\section{The Standard DV Ladder}

\begin{center}
\begin{longtable}{clp{3.5in}}
\toprule
\textbf{DV} & \textbf{Category} & \textbf{When to Use} \\
\midrule
2 & Routine & Clear intent, modest stakes, controlled environment \\
3 & Easy & Minor challenge, familiar task, slight pressure \\
4 & Moderate & Notable challenge, active opposition, time limits \\
5 & Hard & Significant challenge, hostile conditions, precision required \\
6 & Very Hard & Exceptional challenge, multiple constraints, high drama \\
7+ & Extreme & Mythic challenge, campaign-defining, near-impossible \\
\bottomrule
\end{longtable}
\end{center}

\section{DV Setting by Narrative Context}

\subsection{Character Capability Baseline}

Start with the character's Tier and adjust based on the specific challenge:

\begin{center}
\begin{longtable}{cll}
\toprule
\textbf{Tier} & \textbf{Baseline DV} & \textbf{Example Character} \\
\midrule
I & 3-4 & Rookie, local threat \\
II & 4-5 & Seasoned, regional threat \\
III & 5-6 & Veteran, national threat \\
IV & 6-7 & Paragon, legendary threat \\
V & 7-8 & Mythic, world-changing threat \\
\bottomrule
\end{longtable}
\end{center}

\subsection{Position Modifiers}

\begin{center}
\begin{longtable}{cl}
\toprule
\textbf{Position} & \textbf{DV Modifier} \\
\midrule
Dominant & -1 \\
Controlled & +0 \\
Desperate & +1 \\
\bottomrule
\end{longtable}
\end{center}

\section{Contextual DV Modifiers}

\subsection{Environmental Factors}

\begin{itemize}
\item \textbf{Favorable Conditions:} -1 DV (good lighting, stable ground, clear weather)
\item \textbf{Neutral Conditions:} +0 DV (typical environment)
\item \textbf{Challenging Conditions:} +1 DV (dim light, uneven ground, light wind)
\item \textbf{Hostile Conditions:} +2 DV (darkness, slippery surfaces, heavy rain)
\item \textbf{Extreme Conditions:} +3 DV (blizzard, earthquake, magical storm)
\end{itemize}

\subsection{Time Pressure}

\begin{itemize}
\item \textbf{No Time Pressure:} -1 DV (deliberate, careful approach)
\item \textbf{Standard Timing:} +0 DV (normal pace)
\item \textbf{Moderate Pressure:} +1 DV (limited time, but manageable)
\item \textbf{Severe Pressure:} +2 DV (countdown, immediate consequences)
\item \textbf{Critical Timing:} +3 DV (split-second timing, life-or-death)
\end{itemize}

\subsection{Character Condition}

\begin{itemize}
\item \textbf{Well-rested, Focused:} -1 DV (clear mind, full attention)
\item \textbf{Normal Condition:} +0 DV (typical state)
\item \textbf{Fatigued (1-2):} +1 DV (minor exhaustion, distraction)
\item \textbf{Fatigued (3-4):} +2 DV (significant strain, impaired focus)
\item \textbf{Harm 1-2:} +1-2 DV (injury effects, pain penalties)
\item \textbf{Harm 3+:} +3 DV (severe injury, near incapacity)
\end{itemize}

\section{Skill and Attribute Considerations}

\subsection{Skill Mastery Modifiers}

\begin{itemize}
\item \textbf{Skill 0:} +2 DV (untrained attempt)
\item \textbf{Skill 1-2:} +0 DV (basic competence)
\item \textbf{Skill 3-4:} -1 DV (skilled practitioner)
\item \textbf{Skill 5+:} -2 DV (mastery level)
\end{itemize}

\subsection{Attribute Relevance}

When the primary Attribute is exceptionally high or low:

\begin{itemize}
\item \textbf{Attribute 5:} -1 DV (exceptional natural talent)
\item \textbf{Attribute 1:} +2 DV (significant natural limitation)
\end{itemize}

\section{Group Actions and Assistance}

\subsection{Assistance Modifiers}

\begin{itemize}
\item \textbf{One Competent Helper:} -1 DV (relevant expertise)
\item \textbf{Two Helpers:} -1 DV (combined assistance, diminishing returns)
\item \textbf{Three+ Helpers:} -1 DV (maximum assistance benefit)
\item \textbf{Unhelpful Environment:} +1-2 DV (crowded, chaotic, obstructive)
\end{itemize}

\subsection{Group vs. Individual Challenges}

\begin{itemize}
\item \textbf{Individual Task:} Standard DV
\item \textbf{Group Coordination Required:} +1-2 DV (communication complexity)
\item \textbf{Massive Scale:} +2-3 DV (beyond individual scope)
\item \textbf{Specialized Roles Needed:} +1 DV per missing expertise
\end{itemize}

\section{Equipment and Tools}

\subsection{Tool Quality Modifiers}

\begin{itemize}
\item \textbf{Superior Tools:} -1 DV (specialized, well-maintained)
\item \textbf{Adequate Tools:} +0 DV (standard equipment)
\item \textbf{Poor Tools:} +1 DV (worn, improvised, inadequate)
\item \textbf{Wrong Tools:} +2-3 DV (completely inappropriate)
\item \textbf{Magical/Advanced Tools:} -1 to -2 DV (depending on power)
\end{itemize}

\subsection{Tool Condition}

\begin{itemize}
\item \textbf{Maintained:} +0 DV
\item \textbf{Neglected:} +1 DV
\item \textbf{Compromised:} +2 DV
\item \textbf{Broken:} Task impossible without repair
\end{itemize}

\section{Opposition and Resistance}

\subsection{Opposition Level}

\begin{itemize}
\item \textbf{No Active Opposition:} -1 DV (unopposed action)
\item \textbf{Passive Resistance:} +0 DV (natural resistance, no active counter)
\item \textbf{Active Opposition:} +1-2 DV (opponent actively countering)
\item \textbf{Skilled Opposition:} +2-3 DV (opponent with relevant expertise)
\item \textbf{Superior Opposition:} +3-4 DV (opponent significantly more capable)
\end{itemize}

\section{Scenario-Specific DV Guidelines}

\subsection{Combat DV Modifiers}

\begin{itemize}
\item \textbf{Target Size:} -1 to +2 DV (tiny to huge)
\item \textbf{Cover:} +1-2 DV (partial to full cover)
\item \textbf{Range:} +0 to +2 DV (Close to Far)
\item \textbf{Mobility:} +1-2 DV (moving target)
\item \textbf{Illumination:} +1-2 DV (dim to darkness)
\end{itemize}

\subsection{Social DV Modifiers}

\begin{itemize}
\item \textbf{Relationship:} -2 to +2 DV (close ally to bitter enemy)
\item \textbf{Social Distance:} +0 to +2 DV (intimate to formal/professional)
\item \textbf{Cultural Familiarity:} -1 to +2 DV (native customs to foreign protocols)
\item \textbf{Stakes Clarity:} -1 to +2 DV (clear, mutual benefit to ambiguous/harmful)
\item \textbf{Time Pressure:} +0 to +2 DV (leisurely discussion to immediate deadline)
\end{itemize}

\section{Calculating Final DV}

To determine the final DV for any action:

\begin{enumerate}
\item Start with the \textbf{Base DV} from the Standard Ladder (2-7+)
\item Add the character's \textbf{Tier Modifier}: DV = Base DV + (Character Tier - 1)
\item Apply relevant \textbf{Contextual Modifiers} from previous sections
\item Consider \textbf{Position Effects}: Dominant (-1), Controlled (±0), Desperate (+1)
\item Adjust for \textbf{Environmental and Circumstantial Factors}
\end{enumerate}

\textbf{Minimum DV:} No roll can have a DV lower than 2. If modifiers would reduce it further, treat the action as automatic success with narrative description of the easy victory.

\textbf{Maximum DV:} For extremely challenging tasks, DV may exceed 7. Consider using clocks or extended challenges for DV 8+ tasks rather than single rolls.

\section{Special DV Considerations}

\subsection{Group Actions}

When multiple characters act together on a single goal:

\begin{itemize}
\item One character leads the action (sets main DV)
\item Helpers provide assistance (typically -1 DV or +1 Effect)
\item Each helper accepts shared risk from complications
\item Complex coordination may increase DV by +1
\end{itemize}

\subsection{Extended Challenges}

For tasks requiring multiple successes over time:

\begin{itemize}
\item Set a \textbf{Challenge Clock} (4-8 segments)
\item Each successful roll advances the clock
\item Complications may tick the clock backward
\item Partial successes may advance clock slowly
\end{itemize}

\subsection{Contested Actions}

When two parties oppose each other directly:

\begin{itemize}
\item Both parties roll against the same DV
\item Higher successes win the contest
\item Tie results favor the defender or status quo
\item Story Beats generated by either side may be spent by the GM
\end{itemize}

\section{DV Quick Reference}

For rapid gameplay, use these guidelines:

\begin{center}
\begin{longtable}{cl}
\toprule
\textbf{Situation} & \textbf{Quick DV} \\
\midrule
Clear, no pressure & 2 \\
Standard challenge & 3 \\
Notable opposition & 4 \\
Serious danger & 5 \\
Extreme circumstances & 6 \\
Mythic challenge & 7+ \\
\bottomrule
\end{longtable}
\end{center}

Remember: DV represents \textbf{narrative weight}, not simulationist difficulty. Adjust based on story importance, not just mechanical challenge.

\section{Combat Encounters}

\subsection{Core Philosophy}
Combat in Fate's Edge is not a separate mini-game; it is simply conflict under sharper focus. 
It uses the same dice pool system and SB economy as all other actions. 
The rules are designed to emphasize narrative consequence, positional play, and risk management.

\subsection{Structure of Combat}
\begin{itemize}
  \item \textbf{Rounds:} Each round represents a few seconds of action.
  \item \textbf{Turns:} Each participant takes one significant action per round.
  \item \textbf{Scenes:} A battle is one scene unless the fiction dictates otherwise.
\end{itemize}

\subsection{Taking Action}
On your turn, declare intent and method as normal:
\begin{enumerate}
  \item \textbf{Set Position:} The GM decides if you are Dominant, Controlled, or Desperate.
  \item \textbf{Build Pool:} Attribute + Skill (+ gear, + assists, +1 from Imbuement if active).
  \item \textbf{Roll:} Each 6+ is a success. Each 1 generates SB.
  \item \textbf{Resolve:} Successes vs DV, SB spent by GM, Position/Effect applied.
\end{enumerate}

\subsection{Position \& Effect}
\begin{description}[leftmargin=1.5em]
  \item[Dominant] You act from safety or advantage. Failure still leaves you options.
  \item[Controlled] Standard case. Failure has teeth, but not ruin.
  \item[Desperate] High stakes. Failure is severe; success may bring extra XP (mark Desperate use).
\end{description}
% =========================
% HEALTH, FATIGUE, & HARM
% =========================
\section{Health, Fatigue, \& Harm}
\label{sec:health-fatigue-harm-rev}

\subsection*{Tracks \& Caps}
\begin{itemize}
  \item \textbf{Fatigue Track}: boxes equal to \textbf{Body}.
  \item \textbf{Harm Levels}: as defined elsewhere in the SRD (\textbf{Harm 1}, \textbf{Harm 2}, \textbf{Harm 3}).
\end{itemize}

\subsection*{Fatigue $\rightarrow$ Harm Conversion}
Whenever you would mark Fatigue and your Fatigue Track \emph{fills} (all boxes marked):
\begin{enumerate}
  \item \textbf{Increase} your \textbf{Harm} by one level (e.g., 0$\rightarrow$Harm~1, Harm~1$\rightarrow$Harm~2, Harm~2$\rightarrow$Harm~3).
  \item \textbf{Clear all Fatigue} (erase the Fatigue Track back to 0).
\end{enumerate}
This conversion can occur multiple times in a scene. Effects of Harm tier (disadvantage, action limits, incapacitation at Harm~3, etc.) follow your existing SRD.

\subsection*{Taking Fatigue}
Mark Fatigue for strain, exertion, travel, magic costs, or \S\ref{sec:obligation-overflow-rev} overflow. Fatigue can exceed remaining boxes only to \emph{trigger} conversion; any excess is ignored after the Harm increase and Fatigue clear.

\subsection*{Recovering Fatigue}
\begin{itemize}
  \item \textbf{Short Rest} (quiet watch, food/water): remove \textbf{2 Fatigue}.
  \item \textbf{Full Night}: remove \textbf{all Fatigue}.
\end{itemize}
\emph{Fatigue recovery does not remove Harm.} Recover Harm via your normal medical/ritual rules in the SRD.

\subsection*{Mitigation (Optional Dials)}
\begin{itemize}
  \item \textbf{Soak/Ward}: Before marking Fatigue, reduce it by 1--2 (to a minimum of 0) if protected by armor/boons/rites.
  \item \textbf{Convert}: Some effects may convert incoming \textbf{Harm 1} to \textbf{2 Fatigue}; if this \emph{fills} the track, convert as normal.
\end{itemize}

\paragraph{Effect}
Effect is narrative reach:
\begin{itemize}
  \item \textbf{Limited:} Scratch or slow progress.
  \item \textbf{Standard:} Expected impact (downing a guard, disabling a lock).
  \item \textbf{Great:} Overwhelming impact, bigger than expected.
\end{itemize}

\subsection{Damage \& Consequences}
When you take harm:
\begin{itemize}
  \item \textbf{Level 1 Harm:} Minor injury or hindrance. -1 die to related actions.
  \item \textbf{Level 2 Harm:} Serious wound. -1 die to most actions until treated.
  \item \textbf{Level 3 Harm:} Critical injury. You are incapacitated or dying.
\end{itemize}
Harm may be resisted (roll Attribute vs DV 3; 1s generate SB). On a hit, reduce harm by one level.

\subsection{Teamwork}
\begin{itemize}
  \item \textbf{Assist:} Spend 1 Stress or Boon to add +1 die. Max +3 dice from assists.
  \item \textbf{Setup:} Make a roll to improve another’s Position or Effect.
  \item \textbf{Protect:} Take harm or consequence meant for another.
\end{itemize}

\subsection{GM Guidance for SB in Combat}
Use SB to escalate combat fiction:
\begin{itemize}
  \item \textbf{1 SB:} Reinforce enemy cover, minor injury, reveal new foe.
  \item \textbf{2 SB:} Reinforcements arrive, key gear breaks, enemy gains +1 die.
  \item \textbf{3 SB:} Enemy unleashes a Rite or summon, terrain shifts, ally is endangered.
  \item \textbf{4+ SB:} Scene twists—fires spread, the floor collapses, Patron omens manifest.
\end{itemize}

\subsection{Combat and Magic}
\begin{itemize}
  \item \textbf{Casting.} Casters spend one action to \emph{Weave} and another to \emph{Cast}. Requires the \emph{Caster’s Gift} talent. 
  \item \textbf{Rites.} Invoking a Rite takes one action. Players may \emph{Push It} for $+1$ Obligation to gain the listed benefit. 
  \item \textbf{Invokers.} Invokers perform Rites via Symbol. Ritual invocation takes $\text{DV}+1$ rounds and always marks $+1$ Obligation. Alternatively, they may \emph{Crack the Seal} to cast instantly by setting the Symbol to \textsc{Compromised} and marking $+2$ Obligation ($+3$ if High-Power). Invoker Rites cannot use \emph{Push It}. 
  \item \textbf{Imbuements.} Once per scene, spend one action to activate an Imbuement. For the remainder of the scene, gain $+1$ to one Weapon and one Thematic Skill. 
\end{itemize}

\subsection{Weapons \& Armor}
\label{app:weapons-armor}
\index{Weapons}\index{Armor}

\subsubsection{Weapons by Weight Class}
\begin{itemize}
  \item \textbf{Light (4 XP)} — fast, concealable.
  \item \textbf{Medium (8 XP)} — balanced, battlefield standard.
  \item \textbf{Heavy (12 XP)} — punishing, slow.
\end{itemize}

\subsubsection*{Melee}
\begin{longtable}{llll}
\toprule
\textbf{Weight} & \textbf{Close} & \textbf{Near} & \textbf{Notes} \\
\midrule
Light & +2d & +1d & Quick, tight quarters \\
Medium & +1d & +2d & \emph{Set} 1/scene or –1d first attack \\
Heavy & –1d & +3d & \emph{Set} 1/scene or –2d first attack \\
\bottomrule
\end{longtable}

\subsubsection*{Ranged \& Tempo}
\begin{longtable}{lllll}
\toprule
\textbf{Weight} & \textbf{Tempo} & \textbf{Close} & \textbf{Near} & \textbf{Far} \\
\midrule
Light (4 XP) & Fast & Controlled & +1d & — \\
Medium (8 XP) & Standard & Desperate & +2d & +1d \\
Heavy (12 XP) & Slow & Desperate & +1d & +3d \\
\bottomrule
\end{longtable}

\paragraph{Tempo:} \textbf{Fast} = Move+Shoot. \textbf{Standard} = Move or Shoot, Aim +1d/Effect. \textbf{Slow} = Set/Brace, full reload, cannot Move+Shoot.

\subsubsection{Weapon Tags (Optional, +4 XP each, max 2)}
\index{Weapons!Tags}
\textbf{Reach, Close, Accurate, Brutal, Hook, Concealable, Quickdraw, Two-Handed, Off-Hand.}

\subsection{Shields (Optional)}
\begin{longtable}{llll}
\toprule
\textbf{Shield} & \textbf{XP} & \textbf{Benefit} & \textbf{Tradeoff} \\
\midrule
Buckler & 4 & +1d Defend vs melee or +1 DV & Off-hand \\
Heater  & 8 & +1d Defend; 1 Harm→Fatigue & –1d Ranged \\
Pavise  & 12 & \emph{Plant}: heavy cover cone & Bulky, immobile \\
\bottomrule
\end{longtable}
\textbf{Note: }\textit{Using a shield limits weapon size to Light or Medium}

\subsection{Armor}
\begin{longtable}{llll}
\toprule
\textbf{Armor} & \textbf{XP} & \textbf{Conversion} & \textbf{Penalty} \\
\midrule
Light  & 4  & 1 Harm→1 Fatigue & — \\
Medium & 8  & 2 Harm→1 Fatigue & –1d physical \\
Heavy  & 12 & 3 Harm→2 Fatigue & –2d physical, no sprint \\
\bottomrule
\end{longtable}

\paragraph{Notes:} Conversion applies per Harm instance before Fatigue is marked. You may still Resist first.

\subsection{Condition \& Upkeep}
\begin{description}
  \item[\textbf{Neglected}] Weapons –1d; Armor: conversion worsens by 1 step.
  \item[\textbf{Compromised}] Weapons –1d first attack/round; Armor: no conversion.
\end{description}
\emph{Fix:} Short Rest/tools remove Neglected. A scene/Smith removes Compromised.

\subsection{Ranged Options (At a Glance)}
\begin{itemize}
  \item \textbf{Aim:} +1d or +1 Effect.  
  \item \textbf{Volley:} Extra ammo +1d (max +2).  
  \item \textbf{Suppress:} Zone fire, foes –1d/Limited Effect.  
  \item \textbf{Overwatch:} Ready a Controlled shot on trigger.  
\end{itemize}

\subsection{Worked Example}
\emph{Kael swings his Imbued blade at a cultist (DV 2). He rolls 5 dice: 9, 7, 5, 2, 1.}
\begin{itemize}
  \item Successes = 3 (hit), SB = 1.
  \item GM grants success: cultist is cut down.
  \item GM spends 1 SB: ``Blood sprays across the sigil—energy flares, the summoning accelerates.''
\end{itemize}
\subsubsection{Enchanted Equipment Conditions}

Enchanted equipment follows the same maintenance rules as other assets:

\textbf{Neglected:}
\begin{itemize}
\item Minor enchantments function at -1 die penalty
\item Major enchantments lose 1 benefit or become unusable
\end{itemize}

\textbf{Compromised:}
\begin{itemize}
\item All enchantments cease functioning
\item Item provides no magical benefits until repaired
\end{itemize}

\textbf{Repair:}
\begin{itemize}
\item Short Rest + Tinker DV 3: Remove Neglected status
\item Downtime + Tinker DV 4: Remove Compromised status
\item Failed repair attempts may cause permanent enchantment degradation
\end{itemize}

\section{Monk Talents}
\label{sec:monk-talents}

\subsection*{Core Concept}
Monks channel inner discipline into supernatural martial prowess, combining unarmed combat mastery with spiritual focus.

\subsection*{Starting Talent}
\paragraph{Disciplined Body (3 XP --- Minor Talent)} 
\textbf{Requirements:} Melee 1+, Body 2+. \\
\textbf{Benefits:}
\begin{itemize}
  \item +1 die to unarmed combat attacks.
  \item Convert 1 Harm to Fatigue once per scene.
  \item Once per scene, improve Position by one step.
\end{itemize}

\subsection*{Advanced Talents}
\paragraph{Iron Fist Way (6 XP --- Minor Talent)} 
\textbf{Benefits:} +1 die to unarmed attacks; strikes count as enchanted.

\paragraph{Flowing Spirit Way (8 XP --- Major Talent)} 
\textbf{Benefits:} Convert up to 1 Harm into Fatigue per attack; +1 die against fear or charm.

\paragraph{Perfect Timing Way (7 XP --- Major Talent)} 
\textbf{Benefits:} Twice per scene, improve Position by +1 step; +1 die to reactions.

\paragraph{Untouchable Way (12 XP --- Major Talent)} 
\textbf{Prerequisites:} Iron Fist + Flowing Spirit. \\
\textbf{Benefits:} +1 die to unarmed attacks; convert 2 Harm into Fatigue; cannot be grappled.

\paragraph{Inevitable Way (15 XP --- Major Talent)} 
\textbf{Prerequisites:} Iron Fist + Perfect Timing. \\
\textbf{Benefits:} +2 dice to unarmed attacks; ignore 1 Armor; may counterattack when an enemy misses.

\paragraph{Transcendent Harmony (18 XP --- Epic Talent)} 
\textbf{Prerequisites:} Flowing Spirit + Perfect Timing, Spirit 4+. \\
\textbf{Benefits:} Convert 2 Harm into Fatigue; once per session become immune to Harm; allies gain +1 defense.

\subsection*{Progression Path}
Monks specialize early (6--8 XP), combine paths mid-tier (12--15 XP), and achieve transcendence late (18 XP). Each path represents a distinct combat philosophy and playstyle.

\section{Miniatures and Tactical Layer}
\label{sec:miniatures}

\subsection{Core Concepts}
\begin{itemize}
  \item Works on square or hex grids; declare grid type at setup.
  \item Units have base sizes (Small, Medium, Large, Huge) and a facing.
  \item Actions per turn: Move and Act (attack, cast, interact, etc.), in either order.
  \item All checks use normal SRD roll + DV system.
\end{itemize}

\subsection{Turn Structure}
\begin{enumerate}
  \item Start: resolve ongoing effects.
  \item Move: up to Speed; obey Zones of Control (ZOC).
  \item Act: attack, test, assist, cast, rally, shove, guard, etc.
  \item End: resolve end effects and reactions.
\end{enumerate}

\subsection{Zones of Control (ZOC)}
\begin{itemize}
  \item \textbf{Squares:} 4 orthogonal adjacents (optional: 8). 
  \item \textbf{Hexes:} 6 adjacents.
  \item Large/Huge project ZOC from edges; Reach may extend ZOC by +1 ring.
  \item \textbf{Rules:} 
    \begin{itemize}
      \item Entering enemy ZOC ends movement (you are engaged).
      \item Cannot move through enemy ZOC.
      \item Leaving requires Disengage (DV 4–6) or spend 1 Boon.
      \item Multiple ZOCs increase DV by +1 per extra controller.
    \end{itemize}
\end{itemize}

\subsection{Facing and Flanking}
\begin{itemize}
  \item Choose a facing at end of movement.
  \item Flank: +1 die if attacked from opposite arcs; Rear: +1 die and +1 Effect.
\end{itemize}

\subsection{Special Actions}
\begin{itemize}
  \item \textbf{Guard:} Ready a strike when enemy leaves ZOC.
  \item \textbf{Dash:} +2 movement this turn.
  \item \textbf{Brace:} Resist Shoves/Pulls and extend ZOC (opportunity only).
  \item \textbf{Tackle:} Knock target prone (DV 4–6).
\end{itemize}

\subsection{Magic Integration}
\begin{itemize}
  \item Magic uses \textbf{[TAGS]} (e.g., [WARD], [BANISH], [CONJURE]) tied to ZOC, range, and LoS.
  \item Casting while engaged worsens Position unless [INSTANT] or aided by Talent.
  \item Rituals require clear space and visible Symbols; disrupted rituals fail or require a test.
\end{itemize}

\subsection{Quick Reference}
\begin{itemize}
  \item Entering enemy ZOC ends movement; leaving requires Disengage.
  \item Flank = +1 die; Rear = +1 die and +1 Effect.
  \item Difficult terrain +1 cost; moving up elevation +1.
  \item Boons may break ZOC rules: auto-Disengage, change facing, or Heroic Rush.
\end{itemize}

 \begin{tcolorbox}[title=\textbf{Miniatures Mode — Speed Defaults},colback=white!98!gray,colframe=black!50!gray,boxrule=0.4pt]
\textbf{DV:} $\mathrm{DV}=\mathrm{Tier}+2+\text{Keywords}$ \quad(Elevation +1, Altar[WARD] +1, Disengage=4).\\
\textbf{Crit:} Bump Position one step; if already Dominant, Push/Pull 1 hex \emph{or} gain +1 Success.\\
\textbf{ZOC:} Enter/leave an adjacent hex provokes 1 \emph{Reaction} (Free Strike \emph{or} Shove 1 hex). Each unit has 1 Reaction/round.\\
\textbf{Tags:} Max 2 active tags per unit. [WARD] = -1 die vs target; attacker may accept 2 Fatigue to ignore once.\\
\textbf{Terrain:} Difficult=2 MP/hex. Elevation=+1 DV from below.\\
\textbf{Heat:} On any Crit, GM immediately spends 1 Heat to degrade Position or trigger terrain.
\end{tcolorbox}

\begin{tcolorbox}[title=\textbf{Hex Keywords},colback=white!98!gray,colframe=black!50!gray,boxrule=0.4pt]
\textbf{Difficult:} 2 MP/hex \quad \textbf{Elevation:} +1 DV from below \quad \textbf{ZOC:} Reaction on cross\\
\textbf{Altar [WARD]:} -1 die to target (or attacker takes 2 Fatigue to ignore)\\
\textbf{Incorporeal:} Ignore Difficult; may pass through occupied hexes; cannot end there\\
\textbf{Assist (mini):} +1 Effect (not dice); max 1 helper
\end{tcolorbox}

\begin{fatebox}[TPK Resolution Options]
  A Total Party Kill doesn’t have to end the campaign. Choose one of the following resolutions that fits the fiction and stakes.
  
  \begin{itemize}
    \item \textbf{Capture \& Consequences.} PCs live but are captured, bound, or indebted. Advance relevant clocks +2, strip 1–2 assets, apply a permanent Condition to 1–2 PCs.
  
    \item \textbf{Patron’s Claim (Bargain).} A Patron intervenes. PCs survive, but each accepts a non-negotiable term (e.g., \emph{Obligation +2}, lose a Gift, sworn service for a season). Record the Claim as a front.
  
    \item \textbf{Last Stand \emph{→} Legacy.} Convert the TPK scene into a Last Stand: for the remainder of the scene, +1 Effect and ignore new Harm; each action creates 1 SB. When it ends, the party dies. Next session, new PCs inherit one Relationship/Tool/Lesson from the fallen.
  
    \item \textbf{Dramatic Exit \& Inheritance.} Players choose meaningful deaths that save others or seal a danger. Next PCs begin with one inherited tie (bond/contact/rival), a degraded Tool, or +1 XP toward a relevant Talent.
  
    \item \textbf{Reprieve at a Price.} Buy back from death by marking \textbf{+2 Obligation} each (or one steep Patron Claim). All scene clocks advance +1; introduce a new front tied to the price paid.
  
    \item \textbf{New Torchbearers.} End the chapter cleanly. Start with new characters directly connected to the fallen (apprentices, kin, witnesses). Carry forward consequences and open clocks.
  \end{itemize}
  
  \textbf{GM Notes.} Name the cost before rolls at the brink; pick one option, don’t stack. Tie outcomes to Patron themes, and change the world (factions/clocks) accordingly.
  \end{fatebox}

\subsection{Persuasion Encounters}
\index{Persuasion}\index{Social!Clock}\index{DV}\index{Position}

Use a \textbf{Persuasion Clock} to track progress toward agreement. If the outcome is truly binary, skip the clock and resolve with a single roll.

\paragraph{Clock Size.}
Set segments by difficulty/resolve of the target:
\begin{itemize}
  \item Easy: 4-segment \quad Moderate: 6-segment \quad Hard: 8-segment
\end{itemize}

\paragraph{Position → DV.}
Set Position from fiction and map to DV (Dominant = DV 2, Controlled = DV 3, Desperate = DV 4–5+).

\paragraph{Actions.}
Each attempt must be fictionally distinct (new appeal, leverage, evidence, witness, or framing). Bonds and Boons may assist as normal.

\paragraph{Outcomes.}
\begin{itemize}
  \item \textbf{Strong Hit (Success):} Tick the clock +2.
  \item \textbf{Mixed (Partial):} Tick the clock +1 and accept a complication (GM may start/advance a small \textsc{Face Lost} or \textsc{Rebuttal} clock, or claim 1 SB).
  \item \textbf{Miss:} No progress; GM may \emph{decrease} the clock −1 or advance an \textsc{Opposition} clock +1–2.
\end{itemize}

\paragraph{Opposition.}
For contested scenes, add an \textbf{Opposition Clock} (4–6). When it fills first, the target hardens: Position worsens one step or the ask narrows (concession required).

\paragraph{Finish.}
When the Persuasion Clock fills, the target agrees as framed. If there are unresolved costs/clocks, pay them or renegotiate a smaller ask.

\paragraph{Limits.}
Repeat-spam of the same line of argument does not tick the clock; change the fiction.
