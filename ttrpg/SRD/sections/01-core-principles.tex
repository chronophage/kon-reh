
% --- Fate's Edge SRD — Section 1: Core Principles ---
% Include this file from your main .tex with: 
% --- Fate's Edge SRD — Section 1: Core Principles ---
% Include this file from your main .tex with: 
% --- Fate's Edge SRD — Section 1: Core Principles ---
% Include this file from your main .tex with: 
% --- Fate's Edge SRD — Section 1: Core Principles ---
% Include this file from your main .tex with: \input{01-core-principles.tex}

\section{Core Principles}

\subsection{Identity of Fate's Edge}
Fate's Edge is a narrative-first tabletop roleplaying system where every action carries weight, every choice has consequence, and every spell risks backlash. Dice are not simply a measure of success or failure---they are instruments of fate, weaving opportunity with risk.

\subsection{A World of Consequences}

\subsubsection{Design Goals}
\begin{itemize}
  \item \textbf{Narrative Primacy:} Mechanics exist to serve the story.
  \item \textbf{Risk as Drama:} Every roll carries the potential for triumph and complication.
  \item \textbf{Meaningful Growth:} Advancement is more than improving statistics.
\end{itemize}

\subsubsection{The Central Question}
\emph{What are you willing to risk, and what are you willing to pay, to reshape the world around you?}

\subsubsection{Tone of Play}
\begin{itemize}
  \item Cinematic, with pacing tied to narrative beats.
  \item Consequential, where even small choices ripple outward.
  \item Collaborative, empowering both GM and players.
\end{itemize}

\subsection{Key Concepts}

\subsubsection{Narrative Time}
Time is measured by story weight:
\begin{itemize}
  \item \textbf{A Moment} --- A heartbeat, a glance, a single strike or word.
  \item \textbf{Some Time} --- A few minutes, enough for a skirmish.
  \item \textbf{Significant Time} --- Hours, long enough for travel or rituals.
  \item \textbf{Days} --- Large-scale endeavors: marches, training, recovery.
\end{itemize}

\subsubsection{Story Beats}
Whenever a player rolls dice, each result of 1 generates a Story Beat (SB). These are narrative fuel. The GM spends them to introduce twists.

\subsubsection{Affinity}
Each culture provides an Affinity: a narrative edge or metaphysical bond. Affinities make certain Arts, skills, or actions more reliable.

\subsubsection{Prestige Abilities}
High-level talents unlocked by mastering cultural arts or philosophies. They are narrative milestones as much as mechanical ones.

\subsubsection{On-Screen vs.\ Off-Screen}
\begin{itemize}
  \item \textbf{On-Screen Resources:} Companions, hirelings, or allies who stand beside you in danger.
  \item \textbf{Off-Screen Resources:} Taverns, estates, titles, or networks of informants.
\end{itemize}


\section{Core Principles}

\subsection{Identity of Fate's Edge}
Fate's Edge is a narrative-first tabletop roleplaying system where every action carries weight, every choice has consequence, and every spell risks backlash. Dice are not simply a measure of success or failure---they are instruments of fate, weaving opportunity with risk.

\subsection{A World of Consequences}

\subsubsection{Design Goals}
\begin{itemize}
  \item \textbf{Narrative Primacy:} Mechanics exist to serve the story.
  \item \textbf{Risk as Drama:} Every roll carries the potential for triumph and complication.
  \item \textbf{Meaningful Growth:} Advancement is more than improving statistics.
\end{itemize}

\subsubsection{The Central Question}
\emph{What are you willing to risk, and what are you willing to pay, to reshape the world around you?}

\subsubsection{Tone of Play}
\begin{itemize}
  \item Cinematic, with pacing tied to narrative beats.
  \item Consequential, where even small choices ripple outward.
  \item Collaborative, empowering both GM and players.
\end{itemize}

\subsection{Key Concepts}

\subsubsection{Narrative Time}
Time is measured by story weight:
\begin{itemize}
  \item \textbf{A Moment} --- A heartbeat, a glance, a single strike or word.
  \item \textbf{Some Time} --- A few minutes, enough for a skirmish.
  \item \textbf{Significant Time} --- Hours, long enough for travel or rituals.
  \item \textbf{Days} --- Large-scale endeavors: marches, training, recovery.
\end{itemize}

\subsubsection{Story Beats}
Whenever a player rolls dice, each result of 1 generates a Story Beat (SB). These are narrative fuel. The GM spends them to introduce twists.

\subsubsection{Affinity}
Each culture provides an Affinity: a narrative edge or metaphysical bond. Affinities make certain Arts, skills, or actions more reliable.

\subsubsection{Prestige Abilities}
High-level talents unlocked by mastering cultural arts or philosophies. They are narrative milestones as much as mechanical ones.

\subsubsection{On-Screen vs.\ Off-Screen}
\begin{itemize}
  \item \textbf{On-Screen Resources:} Companions, hirelings, or allies who stand beside you in danger.
  \item \textbf{Off-Screen Resources:} Taverns, estates, titles, or networks of informants.
\end{itemize}


\section{Core Principles}

\subsection{Identity of Fate's Edge}
Fate's Edge is a narrative-first tabletop roleplaying system where every action carries weight, every choice has consequence, and every spell risks backlash. Dice are not simply a measure of success or failure---they are instruments of fate, weaving opportunity with risk.

\subsection{A World of Consequences}

\subsubsection{Design Goals}
\begin{itemize}
  \item \textbf{Narrative Primacy:} Mechanics exist to serve the story.
  \item \textbf{Risk as Drama:} Every roll carries the potential for triumph and complication.
  \item \textbf{Meaningful Growth:} Advancement is more than improving statistics.
\end{itemize}

\subsubsection{The Central Question}
\emph{What are you willing to risk, and what are you willing to pay, to reshape the world around you?}

\subsubsection{Tone of Play}
\begin{itemize}
  \item Cinematic, with pacing tied to narrative beats.
  \item Consequential, where even small choices ripple outward.
  \item Collaborative, empowering both GM and players.
\end{itemize}

\subsection{Key Concepts}

\subsubsection{Narrative Time}
Time is measured by story weight:
\begin{itemize}
  \item \textbf{A Moment} --- A heartbeat, a glance, a single strike or word.
  \item \textbf{Some Time} --- A few minutes, enough for a skirmish.
  \item \textbf{Significant Time} --- Hours, long enough for travel or rituals.
  \item \textbf{Days} --- Large-scale endeavors: marches, training, recovery.
\end{itemize}

\subsubsection{Story Beats}
Whenever a player rolls dice, each result of 1 generates a Story Beat (SB). These are narrative fuel. The GM spends them to introduce twists.

\subsubsection{Affinity}
Each culture provides an Affinity: a narrative edge or metaphysical bond. Affinities make certain Arts, skills, or actions more reliable.

\subsubsection{Prestige Abilities}
High-level talents unlocked by mastering cultural arts or philosophies. They are narrative milestones as much as mechanical ones.

\subsubsection{On-Screen vs.\ Off-Screen}
\begin{itemize}
  \item \textbf{On-Screen Resources:} Companions, hirelings, or allies who stand beside you in danger.
  \item \textbf{Off-Screen Resources:} Taverns, estates, titles, or networks of informants.
\end{itemize}


\section{Core Principles}

\subsection{Identity of Fate's Edge}
Fate's Edge is a narrative-first tabletop roleplaying system where every action carries weight, every choice has consequence, and every spell risks backlash. Dice are not simply a measure of success or failure---they are instruments of fate, weaving opportunity with risk.

\subsection{A World of Consequences}

\subsubsection{Design Goals}
\begin{itemize}
  \item \textbf{Narrative Primacy:} Mechanics exist to serve the story.
  \item \textbf{Risk as Drama:} Every roll carries the potential for triumph and complication.
  \item \textbf{Meaningful Growth:} Advancement is more than improving statistics.
\end{itemize}

\subsubsection{The Central Question}
\emph{What are you willing to risk, and what are you willing to pay, to reshape the world around you?}

\subsubsection{Tone of Play}
\begin{itemize}
  \item Cinematic, with pacing tied to narrative beats.
  \item Consequential, where even small choices ripple outward.
  \item Collaborative, empowering both GM and players.
\end{itemize}

\subsection{Key Concepts}

\subsubsection{Narrative Time}
Time is measured by story weight:
\begin{itemize}
  \item \textbf{A Moment} --- A heartbeat, a glance, a single strike or word.
  \item \textbf{Some Time} --- A few minutes, enough for a skirmish.
  \item \textbf{Significant Time} --- Hours, long enough for travel or rituals.
  \item \textbf{Days} --- Large-scale endeavors: marches, training, recovery.
\end{itemize}

\subsubsection{Story Beats}
Whenever a player rolls dice, each result of 1 generates a Story Beat (SB). These are narrative fuel. The GM spends them to introduce twists.

\subsubsection{Affinity}
Each culture provides an Affinity: a narrative edge or metaphysical bond. Affinities make certain Arts, skills, or actions more reliable.

\subsubsection{Prestige Abilities}
High-level talents unlocked by mastering cultural arts or philosophies. They are narrative milestones as much as mechanical ones.

\subsubsection{On-Screen vs.\ Off-Screen}
\begin{itemize}
  \item \textbf{On-Screen Resources:} Companions, hirelings, or allies who stand beside you in danger.
  \item \textbf{Off-Screen Resources:} Taverns, estates, titles, or networks of informants.
\end{itemize}

% --- SRD CORE RULE ---

\section{Standard Rule: Player-Managed Modules}
\label{sec:player-managed-modules}

This rule makes each player the primary steward of their character-facing trackers (\emph{modules}). It keeps table pace high, reduces hidden bookkeeping, and clarifies when mechanical thresholds trigger. The GM retains authority over world-facing clocks, faction fronts, and all major narrative consequences.

\subsection{Scope (\emph{What Counts as a Module})}
Player-managed modules are any \textbf{character-facing} clocks, counters, or discrete states that sit on a single character sheet:
\begin{itemize}
  \item \textbf{Obligation} (per Patron or Symbol).
  \item \textbf{Corruption Clock} (e.g., Cantor).
  \item \textbf{Leash} (Summoned spirit strain) and \textbf{Spirit Bond Clock}.
  \item \textbf{Repertoire Clock} (Cantor) or similar progression clocks.
  \item \textbf{Asset States} (e.g., Symbol: Maintained / Neglected / \textsc{Compromised} / \textsc{Shattered}).
  \item \textbf{Scene Counters} explicitly tied to a PC (e.g., Exposure on that PC, personal Buff/Debuff durations).
\end{itemize}
\textit{Not included:} GM story resources (global \textbf{Story Beats}), location/faction clocks, and mystery/doom fronts.

\begin{tcolorbox}[colback=black!3,colframe=black!40!white,title={What Players Track (at a Glance)}]
\begin{tabularx}{\textwidth}{l l X}
\toprule
\textbf{Module} & \textbf{Owner} & \textbf{Tick / Change Triggers (examples)} \\
\midrule
Obligation (by Patron) & Player & Invoke/Push/ritual text; Invoker \emph{Borrowed Grace}; cracking a Symbol; bargain costs. \\
Corruption Clock & Player & Cantor Push; Resonant Rite; GM spends a Beat tied to the PC’s occult actions. \\
Leash (Summoning) & Player & Harm to spirit; commands against nature; split focus; crossing \texttt{[WARD]} (DV = Cap). \\
Spirit Bond [4] & Player & Shared victories, mutual aid, meaningful attempts (\emph{near-miss progress} once/session/type). \\
Repertoire [6] & Player & Learn a new unique Song/rite-as-song; practice milestones. \\
Asset State (Symbol) & Player & Maintenance/downtime checks; \emph{Crack the Seal} \(\rightarrow\) \textsc{Compromised}; breakage \(\rightarrow\) \textsc{Shattered}. \\
\bottomrule
\end{tabularx}
\end{tcolorbox}

\subsection{Core Principle}
Players \textbf{immediately} mark their own modules when a rule says ``mark $+X$'' or a trigger fires. Threshold effects resolve as soon as they are reached.

\subsection{Player Duties}
\begin{enumerate}
  \item \textbf{Mark Increases/Decreases on Cue.} When you Invoke a Rite, Push, spend/clear per rules text, or a trigger fires, update your module \emph{now}, not later.
  \item \textbf{Declare Thresholds.} If marking fills a clock or crosses capacity, say so aloud; thresholds resolve before the scene proceeds.
  \item \textbf{State Ownership.} Keep per-Patron Obligation tallies distinct; track each Symbol’s state if you use Symbols.
  \item \textbf{Keep It Visible.} Use a tracker the GM and table can see (sheet boxes, index cards, or shared digital).
\end{enumerate}

\subsection{GM Duties}
\begin{enumerate}
  \item \textbf{Spot-Check.} At need, ask any player: current Obligation by Patron, Corruption segments, Leash state, Asset states.
  \item \textbf{Enforce Thresholds.} When a player reports a threshold, apply the standard effects below \emph{immediately}.
  \item \textbf{Own the Fallout.} Patron intrusions, faction reactions, front clocks, and major narrative consequences remain GM authority.
\end{enumerate}

\subsection{Standard Thresholds \& Effects}
\paragraph{Obligation Capacity}
\label{sec:obligation-capacity}
\[
\textbf{Obligation Capacity} \;=\; \textit{Spirit} + \textit{Presence}
\]
\begin{itemize}
  \item \textbf{Over Capacity:} Immediately mark \textbf{+1 Fatigue} per segment over capacity.
  \item \textbf{Over \(\mathbf{2\times}\) Capacity:} Immediately clear all Fatigue, mark \textbf{+1 Harm (Stress)}, and a \textbf{Patron Intrusion} occurs (GM frames on-theme demand/complication).
\end{itemize}

\paragraph{Corruption Full}
When a \textbf{Corruption Clock} fills:
\begin{itemize}
  \item Apply the last-Patron \textbf{benefit \& burden} (per Patron table or setting guidance) to the PC (and any listed followers/retainers).
  \item \textbf{Reset} the Corruption Clock to empty.
  \item If the player chooses \textbf{Embrace Corruption}, convert the current Patron theme into a permanent boon/curse per \S\ref{subsec:corruption-fading}.
\end{itemize}

\paragraph{Leash Full (Summoning)}
When the \textbf{Leash} fills:
\begin{itemize}
  \item The spirit acts once to its nature, then \textbf{departs} (or turns hostile at GM discretion and fiction).
\end{itemize}
\textbf{Leash Capacity:} \(\textit{Cap} + \textit{Spirit}\) segments. (\textit{Cap} is the outsider’s tier: Cap~1 for Lesser, Cap~3 for Greater.)

\paragraph{Symbol State (Invoker)}
\begin{itemize}
  \item \textbf{Maintained} \(\rightarrow\) normal function. \quad
        \textbf{Neglected} \(\rightarrow\) GM may impose $+1$ DV to related rites.
  \item \textbf{\textsc{Compromised}} (e.g., \emph{Crack the Seal}) \(\rightarrow\) instant resolution per rules; mark extra Obligation; repair in Downtime or pay 1 XP.
  \item \textbf{\textsc{Shattered}} \(\rightarrow\) unusable until replaced or ritually restored per fiction.
\end{itemize}

\subsection{Table Procedure (90-Second Loop)}
\paragraph{Start of Session}
Players read out: per-Patron \textbf{Obligation} totals, \textbf{Corruption} segments, standing \textbf{Asset States}, and any personal clocks at 3+.

\paragraph{End of Scene}
Quick pass: ``\emph{Any marks?}'' Players tick modules from scene events. If a threshold triggers, resolve now.

\paragraph{Downtime}
Players apply clears (service, contrition, purification, study) to their own modules. GM verifies any costs or fiction.

\subsection{Disputes \& Order of Operations}
If two marks would land simultaneously, apply them in the \textbf{least advantageous order for the acting character}, unless a rule specifies otherwise. The GM is final arbiter.

\subsection{Accessibility \& Tools}
Use highly visible trackers: bold boxes on sheets, poker chips for segments, or a shared table of per-Patron Obligation. Keep modules at-a-glance to minimize interruption.

\subsection{Worked Micro-Examples}
\begin{itemize}
  \item \textbf{Invoker Rites Twice:} Vessa Invokes two different Patrons. She marks each Patron’s \textbf{Obligation} separately. Hitting capacity with Patron A causes Fatigue; Patron B remains below capacity.
  \item \textbf{Cantor Pushes:} Jorel Pushes a Song (mark +1 Corruption). That fill triggers the last-Patron boon/burden immediately; then he resets to 0.
  \item \textbf{Summoner Clash:} Kestra’s Cap~3 elemental takes Harm and crosses a \texttt{[WARD]}; she ticks her \textbf{Leash} twice. On fill, the elemental flares once and departs.
\end{itemize}