
% --- Fate's Edge SRD — Section 20: Design Philosophy Guardrails ---
% Include this file from your main .tex with: \geometry{margin=1in}
\sectionfont{\large\bfseries}
\subsectionfont{\normalsize\bfseries}

\definecolor{shadecolor}{gray}{0.95}
\definecolor{headercolor}{gray}{0.2}

\newenvironment{gmbox}[1]{%
  \begin{shaded}%
  \textbf{#1}\\%
}{\end{shaded}}

\newenvironment{quicktool}[1]{%
  \begin{gmbox}{#1}
}{\end{gmbox}}

\newenvironment{recommendation}[1]{%
  \begin{gmbox}{Recommendation: #1}
}{\end{gmbox}}

% --- Fate's Edge SRD — Section 20: Design Philosophy Guardrails ---
% Include this file from your main .tex with: \geometry{margin=1in}
\sectionfont{\large\bfseries}
\subsectionfont{\normalsize\bfseries}

\definecolor{shadecolor}{gray}{0.95}
\definecolor{headercolor}{gray}{0.2}

\newenvironment{gmbox}[1]{%
  \begin{shaded}%
  \textbf{#1}\\%
}{\end{shaded}}

\newenvironment{quicktool}[1]{%
  \begin{gmbox}{#1}
}{\end{gmbox}}

\newenvironment{recommendation}[1]{%
  \begin{gmbox}{Recommendation: #1}
}{\end{gmbox}}

% --- Fate's Edge SRD — Section 20: Design Philosophy Guardrails ---
% Include this file from your main .tex with: \geometry{margin=1in}
\sectionfont{\large\bfseries}
\subsectionfont{\normalsize\bfseries}

\definecolor{shadecolor}{gray}{0.95}
\definecolor{headercolor}{gray}{0.2}

\newenvironment{gmbox}[1]{%
  \begin{shaded}%
  \textbf{#1}\\%
}{\end{shaded}}

\newenvironment{quicktool}[1]{%
  \begin{gmbox}{#1}
}{\end{gmbox}}

\newenvironment{recommendation}[1]{%
  \begin{gmbox}{Recommendation: #1}
}{\end{gmbox}}

% --- Fate's Edge SRD — Section 20: Design Philosophy Guardrails ---
% Include this file from your main .tex with: \input{20-design-guardrails.tex}

\section{The Director Mindset}

You are not a rules engine. You are a director choosing the next shot:
\begin{itemize}[leftmargin=*]
\item \textbf{Clean win} - Character succeeds, story progresses
\item \textbf{Costly win} - Success with complications
\item \textbf{Partial success} - Progress made, but with gaps
\item \textbf{Spiraling disaster} - Situation escalates dramatically
\end{itemize}

\begin{recommendation}{Embrace Imperfection}
Perfect mechanical resolution is less important than maintaining story momentum. A quick, clear call that keeps the session moving is better than a precise ruling that derails the flow.
\end{recommendation}

\section{Core Quick Tools}

\subsection{1. Five-Second Position \& Effect}

\textbf{Risk Assessment:} What is the Risk?
\begin{itemize}[leftmargin=*]
\item \textbf{Controlled:} Safe, prepared, low threat
\item \textbf{Risky (default):} Pressure, danger, uncertainty
\item \textbf{Desperate:} Immediate danger, overmatched, exposed
\end{itemize}

\textbf{Impact Assessment:} What is the Impact?
\begin{itemize}[leftmargin=*]
\item \textbf{Limited:} Partial progress or minor effect
\item \textbf{Standard (default):} Normal success
\item \textbf{Great:} Powerful, overwhelming, high-impact
\end{itemize}

\begin{quicktool}{Example Usage}
Player charges tougher foe in melee. Risk = Risky, Impact = Standard. "Risky / Standard, DV 3. Roll!"
\end{quicktool}

\subsection{2. The Lazy DV Table}

\begin{center}
\begin{tabularx}{\textwidth}{clX}
\toprule
\textbf{DV} & \textbf{Use When} & \textbf{Example} \\
\midrule
2 & Routine, low stakes & Pick a lock in a safe house \\
3 & Pressured (default) & Pick a lock with guards nearby \\
4 & Hard, hostile conditions & Pick a trapped arcane chest \\
5+ & Extreme, dramatic risk & Pick a magical lock underwater \\
\bottomrule
\end{tabularx}
\end{center}

\begin{recommendation}{Default to DV 3}
When in doubt, DV 3 provides appropriate challenge for most meaningful actions without overcomplicating resolution.
\end{recommendation}

\subsection{3. No-Thinking SB Spending}

When players roll 1s, react instantly. Pick one option and move on:

\begin{center}
\begin{tabularx}{\textwidth}{clX}
\toprule
\textbf{SB} & \textbf{Complication Type} & \textbf{Examples} \\
\midrule
1 & Minor & Noise, track, small loss, clock +1 \\
2 & Moderate & Alarm, lose Position, broken tool, clock +2 \\
3+ & Major & Reinforcements, terrain shift, ally endangered \\
\bottomrule
\end{tabularx}
\end{center}

\begin{recommendation}{Never Overthink SB Spends}
One strong spend is better than multiple minor taxes. Bank sparingly and resolve quickly to maintain momentum.
\end{recommendation}

\subsection{4. The Three-Clock Rule}

Only three active clocks at once:
\begin{itemize}[leftmargin=*]
\item \textbf{Scene Clock:} e.g., Guards Alerted [4]
\item \textbf{Journey Clock:} e.g., Mist Encroaches [6]
\item \textbf{Campaign Clock:} e.g., Baron's Suspicion [8]
\end{itemize}

\begin{recommendation}{Fold New Complications}
When new complications arise, integrate them into existing clocks rather than creating new ones. This keeps focus tight and prevents clock sprawl.
\end{recommendation}

\subsection{5. Magic Adjudication Shortcut}

If a player invents a spell on the fly:
\begin{enumerate}[leftmargin=*]
\item Hear the intent
\item Assign a fitting TAG (Veil, Ward, Barrier, Glamour, etc.)
\item Pick DV based on scale:
  \begin{itemize}
  \item Personal / Subtle = DV 2
  \item Scene-wide / Strong = DV 3
  \item Area / Devastating = DV 4+
  \end{itemize}
\item Choose Position based on danger while casting
\end{enumerate}

\begin{quicktool}{Example}
Creating force wall under fire: TAG = Barrier, scale = Scene $\Rightarrow$ DV 3. Casting in melee $\Rightarrow$ Risky. "Risky / Standard, DV 3. Roll."
\end{quicktool}

\section{Preparation Workflow}

\subsection{Pre-Session (15 minutes)}

\begin{enumerate}[leftmargin=*]
\item Use Five-Second Position/Effect for major scenes
\item Apply Lazy DV Table to key challenges
\item Set up Three-Clock framework
\item Bank 2-3 SB spends per major scene
\item Prep 1-2 magic adjudications
\end{enumerate}

\begin{recommendation}{Prep as Story Beats}
Focus on "What happens if..." rather than mechanical stats. Prepare compelling complications and interesting outcomes.
\end{recommendation}

\section{Session Management}

\subsection{Core Principles}

\begin{itemize}[leftmargin=*]
\item Keep mechanical resolution under 10 seconds
\item Always spend SB immediately, don't hoard
\item Never exceed three active clocks
\item Reassess Position every major beat
\end{itemize}

\begin{recommendation}{Trust the Framework}
The Quick-Kit tools aren't training wheels—they're professional equipment that experienced directors use to keep focus on story.
\end{recommendation}

\section{Character Integration}

\subsection{New Player Approach}

\begin{enumerate}[leftmargin=*]
\item Ask "What story role do you want to play?"
\item Help build mechanics to support that role
\item Focus on narrative contributions over mechanical optimization
\item Introduce complexity gradually as player comfort grows
\end{enumerate}

\subsection{Veteran Player Integration}

\begin{enumerate}[leftmargin=*]
\item Leverage existing system knowledge
\item Encourage creative mechanical combinations
\item Challenge with complex multi-domain scenarios
\item Support high-tier mythic storytelling
\end{enumerate}

\section{Domain Management}

\subsection{Multi-Domain Adventures}

\begin{itemize}[leftmargin=*]
\item Each character's specialties should have moments to shine
\item Integrate assets naturally as story elements, not just bonuses
\item Use followers for support roles that enhance scenes
\item Let clocks interact and influence each other naturally
\end{itemize}

\begin{recommendation}{Domain Integration}
Set 2-3 domain-appropriate clocks, ask "Who can contribute their specialty here?" each scene, and let mechanical elements serve narrative advancement.
\end{recommendation}

\section{High-Tier Play}

\subsection{Tier V Considerations}

\begin{itemize}[leftmargin=*]
\item Scale fiction, not mechanics
\item Treat assets as enablers for realm-level actions
\item Ensure consequences match mythic stakes
\item Make follower management strategic rather than tactical
\item Use clocks to represent ongoing sagas, not temporary complications
\end{itemize}

\begin{recommendation}{Mythic Scope}
Tier V adventures should create legends that define the world. Focus on permanent changes to setting and character transformation rather than bigger numbers.
\end{recommendation}

\section{Continuity Management}

\subsection{New Character Integration}

\begin{enumerate}[leftmargin=*]
\item Position new characters as essential support roles
\item Establish immediate bonds with existing characters
\item Connect new character skills to ongoing campaign threads
\item Provide meaningful contributions without overshadowing veterans
\end{enumerate}

\begin{recommendation}{Supporting Cast Approach}
New characters don't need to be mythic heroes—essential specialists, trusted allies, or community anchors can be equally engaging and easier to integrate.
\end{recommendation}

\section{Troubleshooting Guide}

\subsection{Common Issues and Solutions}

\begin{description}[leftmargin=*]
\item[Analysis Paralysis:] Use the Lazy DV Table and default to DV 3. Make quick decisions and maintain momentum.
\item[Mechanical Overload:] Apply the Three-Clock Rule. Simplify asset states to Maintained/Neglected/Compromised.
\item[Player Disengagement:] Ensure each character has spotlight moments. Rotate complications and bonds to different strengths.
\item[Session Lag:] Trust the fail-forward system. Every roll should change the story, even misses through Boons.
\end{description}

\subsection{Emergency Reset}

When sessions lose focus:
\begin{enumerate}[leftmargin=*]
\item Identify the core story question
\item Strip down to Three-Clock framework
\item Give each player one clear action to take
\item Resolve with quick Position/DV calls
\item Escalate through SB spends until momentum returns
\end{enumerate}

\section{Final Wisdom}

\begin{gmbox}{Remember Your Role}
You are not the rules expert—you are the story facilitator. The players will help you remember details. Focus on pacing, tension, and narrative flow.
\end{gmbox}

\begin{gmbox}{Embrace the Economy}
Story Beats and Boons are your storytelling tools, not bookkeeping. They represent the world responding to character actions, not mechanical penalties.
\end{gmbox}

\begin{gmbox}{Trust Your Players}
Players want the story to succeed as much as you do. When in doubt, ask for their input or go with the interpretation that makes the story more interesting.
\end{gmbox}

\section{Design Philosophy Guardrails (Flow-First GMing)}
\label{sec:design-guardrails}

Fate’s Edge is built to \textbf{keep play flowing}. If you remember nothing else: \textbf{The Narrative is primary}. 
Mechanics exist to shape \emph{how} the story changes, not \emph{whether} it moves. 
This section translates the rules into plain, table-ready guidance—especially for new GMs.

\subsection{Simple Translations}
\begin{description}[leftmargin=1.5em, style=nextline]
  \item[Story Beats (SB) $\Rightarrow$ Story Beats:] 1s on dice give you \emph{beats} to spend. Spend them on twists, escalations, or new information. One strong beat is better than three tiny ones.
  \item[Clocks $\Rightarrow$ Checkboxes/Lists:] A Clock is just a short checklist that tracks progress or rising danger. When it fills, the listed thing \emph{happens}. Name it and tick it when fiction leans that way.
  \item[\texttt{[TAGS]} $\Rightarrow$ Gates with a Cost:] Tags are labels that unlock specific effects (e.g.\ \texttt{[WARD]}, \texttt{[BANISH]}). They don’t do anything alone. They appear on Talents, Rites, or Spells to say, ``Yes, you can do this—\emph{here’s the price and limits}.''
\end{description}

\subsection{The 30-Second Adjudication Loop}
Use this loop to resolve almost anything without breaking flow.
\begin{enumerate}
  \item \textbf{Clarify intent and approach.} ``What do you want, and how?''
  \item \textbf{Set stakes and Position.} ``If it works, what changes? If it fails, what bites?'' Start \emph{Controlled/Standard} unless fiction says otherwise.
  \item \textbf{Roll \& read.} Count 6+ as successes; each \textbf{1} gives you SB (beats). Compare successes to DV.
  \item \textbf{Spend one beat well.} Cash SB on one memorable twist or tick a relevant Clock.
  \item \textbf{Push forward.} Describe how the fiction is now different; ask, ``Who moves next?''
\end{enumerate}

\subsection{When to Reach for Mechanics (and When Not To)}
\begin{itemize}
  \item \textbf{Roll} when uncertainty + meaningful stakes exist \emph{now}. Otherwise, say ``Yes'' or offer a choice/cost.
  \item \textbf{Use a Clock} when danger or progress builds over time (guard alert, ritual, chase, social sway).
  \item \textbf{Draw from the Deck} when you want an oracular twist consistent with the current tone.
  \item \textbf{Skip subsystems} if they slow the table. You can always tick a Clock and move on.
\end{itemize}

\subsection{Defaults That Keep Things Moving}
\begin{itemize}
  \item \textbf{Range/Position:} Assume \emph{Near} and \emph{Controlled/Standard}. Ask: ``Do you need a beat to get there?''
  \item \textbf{DV:} 2 for small/local, 3 for scene-scale, 4 for big swings, 5+ for set-pieces/rituals.
  \item \textbf{Boons:} Misses on meaningful actions grant Boons (player fuel). Trim to 2 at scene end.
  \item \textbf{SB Budget:} Prefer one strong spend over many petty taxes. Bank sparingly and pay off soon.
\end{itemize}

\subsection{Rookie GM Comfort Dials}
You can use these dials to simplify play, then loosen them later.
\begin{description}[leftmargin=1.5em, style=nextline]
  \item[Soft SB:] For your first 2 sessions, cap each roll’s SB spend to \textbf{1--2} unless it’s a set-piece.
  \item[Visible Clocks:] Put Clocks on the table. Name them aloud: \emph{``Guards Incoming [4]''}. Tick them in ink.
  \item[Tag Cards:] Print a one-liner for frequently used Tags (\texttt{[WARD]}, \texttt{[BANISH]}, \texttt{[COUNTER]}). Hand them out when a power is active.
  \item[One Move, One Sentence:] Every ruling should end with one sentence that states the new situation.
\end{description}

\subsection{Narrative-First Rulings (with Examples)}
\paragraph{Example 1: The Locked Gate}
Player: ``I pick the lock fast before the patrol rounds the corner.''\\
GM: ``Controlled/Standard, DV 3. If it works, you’re through; if it fails, the patrol clocks closer.''\\
Roll shows 1 SB. GM spends 1 SB to tick \emph{Guards Incoming [4]}. ``You’re through, but boots echo—two ticks left.'' \emph{Flow continues.}

\paragraph{Example 2: The Shadow Rite}
Player Invokes a \texttt{[WARD]}. ``You’re safe unless Outsiders test the edge: DV = Cap. If one hits, its Leash gains +DV. Your Push would add +1 Obligation—do you Push?'' The scene stays in motion; costs and gates are clear.

\paragraph{Example 3: Fire Cast Backlash}
Caster hits but shows two 1s. GM picks one strong backlash: ``Flare blinds you; Position -1 for the next action.'' No rules dive; \emph{one beat lands}, story moves.

\subsection{Let the Fiction Lead}
\begin{itemize}
  \item Say what the world does next. If a rule is unclear, follow the fiction and note a ruling; refine between sessions.
  \item If you forget a tag nuance, ask: ``What is the effect trying to \emph{gate}?'' Charge a cost (time, risk, Obligation, or a tick), then go.
  \item Tie SB spends to \textbf{visible} outcomes: a new foe appears, a path closes, a clock advances.
\end{itemize}

\subsection{Common Pitfalls and Fixes}
\begin{description}[leftmargin=1.5em, style=nextline]
  \item[Over-cranking SB:] If scenes feel punitive, halve your SB spends for a while or cash them into visible Clocks instead of immediate penalties.
  \item[Clock Sprawl:] Merge redundant Clocks. Each active scene rarely needs more than \textbf{2--3}.
  \item[Tag Paralysis:] If a player stalls waiting for a perfect tag, paraphrase: ``Sounds like \texttt{[VEIL]}. DV 3. Want to roll?''
  \item[Rules Drift:] If table memory conflicts with text, pick the ruling that keeps flow, then sticky-note a TODO to reconcile after play.
\end{description}

\subsection{The Four Questions (Cheat Prompts)}
When stuck, ask out loud:
\begin{enumerate}
  \item If this goes right, what changes? (\emph{Intent})
  \item If this goes wrong, what bites back? (\emph{Stakes})
  \item What single twist will make this memorable? (\emph{SB spend})
  \item Who moves next? (\emph{Momentum})
\end{enumerate}

\subsection{Design Guardrails (for Consistency)}
\begin{itemize}
  \item \textbf{Narrative Primacy:} Mechanics serve story, not replace it.
  \item \textbf{Risk as Drama:} Every roll carries potential for triumph+complication.
  \item \textbf{Meaningful Growth:} XP changes characters and the world.
  \item \textbf{Consequence Weight:} Choices ripple outward; nothing is free.
  \item \textbf{Fail Forward:} Misses fuel Boons; 1s become SB (beats).
\end{itemize}

\subsection{Session Checklist (One Page)}
Before play: set tone, stakes, and clocks in plain sight.\\
During play: adjudicate with the 30-second loop; spend one strong beat; move on.\\
After play: award XP, clear/advance Clocks, note rulings to revisit.

\bigskip
\noindent\textit{If you keep the flow, the game will carry you. The rules are rails you lay just ahead of the train.}


\section{The Director Mindset}

You are not a rules engine. You are a director choosing the next shot:
\begin{itemize}[leftmargin=*]
\item \textbf{Clean win} - Character succeeds, story progresses
\item \textbf{Costly win} - Success with complications
\item \textbf{Partial success} - Progress made, but with gaps
\item \textbf{Spiraling disaster} - Situation escalates dramatically
\end{itemize}

\begin{recommendation}{Embrace Imperfection}
Perfect mechanical resolution is less important than maintaining story momentum. A quick, clear call that keeps the session moving is better than a precise ruling that derails the flow.
\end{recommendation}

\section{Core Quick Tools}

\subsection{1. Five-Second Position \& Effect}

\textbf{Risk Assessment:} What is the Risk?
\begin{itemize}[leftmargin=*]
\item \textbf{Controlled:} Safe, prepared, low threat
\item \textbf{Risky (default):} Pressure, danger, uncertainty
\item \textbf{Desperate:} Immediate danger, overmatched, exposed
\end{itemize}

\textbf{Impact Assessment:} What is the Impact?
\begin{itemize}[leftmargin=*]
\item \textbf{Limited:} Partial progress or minor effect
\item \textbf{Standard (default):} Normal success
\item \textbf{Great:} Powerful, overwhelming, high-impact
\end{itemize}

\begin{quicktool}{Example Usage}
Player charges tougher foe in melee. Risk = Risky, Impact = Standard. "Risky / Standard, DV 3. Roll!"
\end{quicktool}

\subsection{2. The Lazy DV Table}

\begin{center}
\begin{tabularx}{\textwidth}{clX}
\toprule
\textbf{DV} & \textbf{Use When} & \textbf{Example} \\
\midrule
2 & Routine, low stakes & Pick a lock in a safe house \\
3 & Pressured (default) & Pick a lock with guards nearby \\
4 & Hard, hostile conditions & Pick a trapped arcane chest \\
5+ & Extreme, dramatic risk & Pick a magical lock underwater \\
\bottomrule
\end{tabularx}
\end{center}

\begin{recommendation}{Default to DV 3}
When in doubt, DV 3 provides appropriate challenge for most meaningful actions without overcomplicating resolution.
\end{recommendation}

\subsection{3. No-Thinking SB Spending}

When players roll 1s, react instantly. Pick one option and move on:

\begin{center}
\begin{tabularx}{\textwidth}{clX}
\toprule
\textbf{SB} & \textbf{Complication Type} & \textbf{Examples} \\
\midrule
1 & Minor & Noise, track, small loss, clock +1 \\
2 & Moderate & Alarm, lose Position, broken tool, clock +2 \\
3+ & Major & Reinforcements, terrain shift, ally endangered \\
\bottomrule
\end{tabularx}
\end{center}

\begin{recommendation}{Never Overthink SB Spends}
One strong spend is better than multiple minor taxes. Bank sparingly and resolve quickly to maintain momentum.
\end{recommendation}

\subsection{4. The Three-Clock Rule}

Only three active clocks at once:
\begin{itemize}[leftmargin=*]
\item \textbf{Scene Clock:} e.g., Guards Alerted [4]
\item \textbf{Journey Clock:} e.g., Mist Encroaches [6]
\item \textbf{Campaign Clock:} e.g., Baron's Suspicion [8]
\end{itemize}

\begin{recommendation}{Fold New Complications}
When new complications arise, integrate them into existing clocks rather than creating new ones. This keeps focus tight and prevents clock sprawl.
\end{recommendation}

\subsection{5. Magic Adjudication Shortcut}

If a player invents a spell on the fly:
\begin{enumerate}[leftmargin=*]
\item Hear the intent
\item Assign a fitting TAG (Veil, Ward, Barrier, Glamour, etc.)
\item Pick DV based on scale:
  \begin{itemize}
  \item Personal / Subtle = DV 2
  \item Scene-wide / Strong = DV 3
  \item Area / Devastating = DV 4+
  \end{itemize}
\item Choose Position based on danger while casting
\end{enumerate}

\begin{quicktool}{Example}
Creating force wall under fire: TAG = Barrier, scale = Scene $\Rightarrow$ DV 3. Casting in melee $\Rightarrow$ Risky. "Risky / Standard, DV 3. Roll."
\end{quicktool}

\section{Preparation Workflow}

\subsection{Pre-Session (15 minutes)}

\begin{enumerate}[leftmargin=*]
\item Use Five-Second Position/Effect for major scenes
\item Apply Lazy DV Table to key challenges
\item Set up Three-Clock framework
\item Bank 2-3 SB spends per major scene
\item Prep 1-2 magic adjudications
\end{enumerate}

\begin{recommendation}{Prep as Story Beats}
Focus on "What happens if..." rather than mechanical stats. Prepare compelling complications and interesting outcomes.
\end{recommendation}

\section{Session Management}

\subsection{Core Principles}

\begin{itemize}[leftmargin=*]
\item Keep mechanical resolution under 10 seconds
\item Always spend SB immediately, don't hoard
\item Never exceed three active clocks
\item Reassess Position every major beat
\end{itemize}

\begin{recommendation}{Trust the Framework}
The Quick-Kit tools aren't training wheels—they're professional equipment that experienced directors use to keep focus on story.
\end{recommendation}

\section{Character Integration}

\subsection{New Player Approach}

\begin{enumerate}[leftmargin=*]
\item Ask "What story role do you want to play?"
\item Help build mechanics to support that role
\item Focus on narrative contributions over mechanical optimization
\item Introduce complexity gradually as player comfort grows
\end{enumerate}

\subsection{Veteran Player Integration}

\begin{enumerate}[leftmargin=*]
\item Leverage existing system knowledge
\item Encourage creative mechanical combinations
\item Challenge with complex multi-domain scenarios
\item Support high-tier mythic storytelling
\end{enumerate}

\section{Domain Management}

\subsection{Multi-Domain Adventures}

\begin{itemize}[leftmargin=*]
\item Each character's specialties should have moments to shine
\item Integrate assets naturally as story elements, not just bonuses
\item Use followers for support roles that enhance scenes
\item Let clocks interact and influence each other naturally
\end{itemize}

\begin{recommendation}{Domain Integration}
Set 2-3 domain-appropriate clocks, ask "Who can contribute their specialty here?" each scene, and let mechanical elements serve narrative advancement.
\end{recommendation}

\section{High-Tier Play}

\subsection{Tier V Considerations}

\begin{itemize}[leftmargin=*]
\item Scale fiction, not mechanics
\item Treat assets as enablers for realm-level actions
\item Ensure consequences match mythic stakes
\item Make follower management strategic rather than tactical
\item Use clocks to represent ongoing sagas, not temporary complications
\end{itemize}

\begin{recommendation}{Mythic Scope}
Tier V adventures should create legends that define the world. Focus on permanent changes to setting and character transformation rather than bigger numbers.
\end{recommendation}

\section{Continuity Management}

\subsection{New Character Integration}

\begin{enumerate}[leftmargin=*]
\item Position new characters as essential support roles
\item Establish immediate bonds with existing characters
\item Connect new character skills to ongoing campaign threads
\item Provide meaningful contributions without overshadowing veterans
\end{enumerate}

\begin{recommendation}{Supporting Cast Approach}
New characters don't need to be mythic heroes—essential specialists, trusted allies, or community anchors can be equally engaging and easier to integrate.
\end{recommendation}

\section{Troubleshooting Guide}

\subsection{Common Issues and Solutions}

\begin{description}[leftmargin=*]
\item[Analysis Paralysis:] Use the Lazy DV Table and default to DV 3. Make quick decisions and maintain momentum.
\item[Mechanical Overload:] Apply the Three-Clock Rule. Simplify asset states to Maintained/Neglected/Compromised.
\item[Player Disengagement:] Ensure each character has spotlight moments. Rotate complications and bonds to different strengths.
\item[Session Lag:] Trust the fail-forward system. Every roll should change the story, even misses through Boons.
\end{description}

\subsection{Emergency Reset}

When sessions lose focus:
\begin{enumerate}[leftmargin=*]
\item Identify the core story question
\item Strip down to Three-Clock framework
\item Give each player one clear action to take
\item Resolve with quick Position/DV calls
\item Escalate through SB spends until momentum returns
\end{enumerate}

\section{Final Wisdom}

\begin{gmbox}{Remember Your Role}
You are not the rules expert—you are the story facilitator. The players will help you remember details. Focus on pacing, tension, and narrative flow.
\end{gmbox}

\begin{gmbox}{Embrace the Economy}
Story Beats and Boons are your storytelling tools, not bookkeeping. They represent the world responding to character actions, not mechanical penalties.
\end{gmbox}

\begin{gmbox}{Trust Your Players}
Players want the story to succeed as much as you do. When in doubt, ask for their input or go with the interpretation that makes the story more interesting.
\end{gmbox}

\section{Design Philosophy Guardrails (Flow-First GMing)}
\label{sec:design-guardrails}

Fate’s Edge is built to \textbf{keep play flowing}. If you remember nothing else: \textbf{The Narrative is primary}. 
Mechanics exist to shape \emph{how} the story changes, not \emph{whether} it moves. 
This section translates the rules into plain, table-ready guidance—especially for new GMs.

\subsection{Simple Translations}
\begin{description}[leftmargin=1.5em, style=nextline]
  \item[Story Beats (SB) $\Rightarrow$ Story Beats:] 1s on dice give you \emph{beats} to spend. Spend them on twists, escalations, or new information. One strong beat is better than three tiny ones.
  \item[Clocks $\Rightarrow$ Checkboxes/Lists:] A Clock is just a short checklist that tracks progress or rising danger. When it fills, the listed thing \emph{happens}. Name it and tick it when fiction leans that way.
  \item[\texttt{[TAGS]} $\Rightarrow$ Gates with a Cost:] Tags are labels that unlock specific effects (e.g.\ \texttt{[WARD]}, \texttt{[BANISH]}). They don’t do anything alone. They appear on Talents, Rites, or Spells to say, ``Yes, you can do this—\emph{here’s the price and limits}.''
\end{description}

\subsection{The 30-Second Adjudication Loop}
Use this loop to resolve almost anything without breaking flow.
\begin{enumerate}
  \item \textbf{Clarify intent and approach.} ``What do you want, and how?''
  \item \textbf{Set stakes and Position.} ``If it works, what changes? If it fails, what bites?'' Start \emph{Controlled/Standard} unless fiction says otherwise.
  \item \textbf{Roll \& read.} Count 6+ as successes; each \textbf{1} gives you SB (beats). Compare successes to DV.
  \item \textbf{Spend one beat well.} Cash SB on one memorable twist or tick a relevant Clock.
  \item \textbf{Push forward.} Describe how the fiction is now different; ask, ``Who moves next?''
\end{enumerate}

\subsection{When to Reach for Mechanics (and When Not To)}
\begin{itemize}
  \item \textbf{Roll} when uncertainty + meaningful stakes exist \emph{now}. Otherwise, say ``Yes'' or offer a choice/cost.
  \item \textbf{Use a Clock} when danger or progress builds over time (guard alert, ritual, chase, social sway).
  \item \textbf{Draw from the Deck} when you want an oracular twist consistent with the current tone.
  \item \textbf{Skip subsystems} if they slow the table. You can always tick a Clock and move on.
\end{itemize}

\subsection{Defaults That Keep Things Moving}
\begin{itemize}
  \item \textbf{Range/Position:} Assume \emph{Near} and \emph{Controlled/Standard}. Ask: ``Do you need a beat to get there?''
  \item \textbf{DV:} 2 for small/local, 3 for scene-scale, 4 for big swings, 5+ for set-pieces/rituals.
  \item \textbf{Boons:} Misses on meaningful actions grant Boons (player fuel). Trim to 2 at scene end.
  \item \textbf{SB Budget:} Prefer one strong spend over many petty taxes. Bank sparingly and pay off soon.
\end{itemize}

\subsection{Rookie GM Comfort Dials}
You can use these dials to simplify play, then loosen them later.
\begin{description}[leftmargin=1.5em, style=nextline]
  \item[Soft SB:] For your first 2 sessions, cap each roll’s SB spend to \textbf{1--2} unless it’s a set-piece.
  \item[Visible Clocks:] Put Clocks on the table. Name them aloud: \emph{``Guards Incoming [4]''}. Tick them in ink.
  \item[Tag Cards:] Print a one-liner for frequently used Tags (\texttt{[WARD]}, \texttt{[BANISH]}, \texttt{[COUNTER]}). Hand them out when a power is active.
  \item[One Move, One Sentence:] Every ruling should end with one sentence that states the new situation.
\end{description}

\subsection{Narrative-First Rulings (with Examples)}
\paragraph{Example 1: The Locked Gate}
Player: ``I pick the lock fast before the patrol rounds the corner.''\\
GM: ``Controlled/Standard, DV 3. If it works, you’re through; if it fails, the patrol clocks closer.''\\
Roll shows 1 SB. GM spends 1 SB to tick \emph{Guards Incoming [4]}. ``You’re through, but boots echo—two ticks left.'' \emph{Flow continues.}

\paragraph{Example 2: The Shadow Rite}
Player Invokes a \texttt{[WARD]}. ``You’re safe unless Outsiders test the edge: DV = Cap. If one hits, its Leash gains +DV. Your Push would add +1 Obligation—do you Push?'' The scene stays in motion; costs and gates are clear.

\paragraph{Example 3: Fire Cast Backlash}
Caster hits but shows two 1s. GM picks one strong backlash: ``Flare blinds you; Position -1 for the next action.'' No rules dive; \emph{one beat lands}, story moves.

\subsection{Let the Fiction Lead}
\begin{itemize}
  \item Say what the world does next. If a rule is unclear, follow the fiction and note a ruling; refine between sessions.
  \item If you forget a tag nuance, ask: ``What is the effect trying to \emph{gate}?'' Charge a cost (time, risk, Obligation, or a tick), then go.
  \item Tie SB spends to \textbf{visible} outcomes: a new foe appears, a path closes, a clock advances.
\end{itemize}

\subsection{Common Pitfalls and Fixes}
\begin{description}[leftmargin=1.5em, style=nextline]
  \item[Over-cranking SB:] If scenes feel punitive, halve your SB spends for a while or cash them into visible Clocks instead of immediate penalties.
  \item[Clock Sprawl:] Merge redundant Clocks. Each active scene rarely needs more than \textbf{2--3}.
  \item[Tag Paralysis:] If a player stalls waiting for a perfect tag, paraphrase: ``Sounds like \texttt{[VEIL]}. DV 3. Want to roll?''
  \item[Rules Drift:] If table memory conflicts with text, pick the ruling that keeps flow, then sticky-note a TODO to reconcile after play.
\end{description}

\subsection{The Four Questions (Cheat Prompts)}
When stuck, ask out loud:
\begin{enumerate}
  \item If this goes right, what changes? (\emph{Intent})
  \item If this goes wrong, what bites back? (\emph{Stakes})
  \item What single twist will make this memorable? (\emph{SB spend})
  \item Who moves next? (\emph{Momentum})
\end{enumerate}

\subsection{Design Guardrails (for Consistency)}
\begin{itemize}
  \item \textbf{Narrative Primacy:} Mechanics serve story, not replace it.
  \item \textbf{Risk as Drama:} Every roll carries potential for triumph+complication.
  \item \textbf{Meaningful Growth:} XP changes characters and the world.
  \item \textbf{Consequence Weight:} Choices ripple outward; nothing is free.
  \item \textbf{Fail Forward:} Misses fuel Boons; 1s become SB (beats).
\end{itemize}

\subsection{Session Checklist (One Page)}
Before play: set tone, stakes, and clocks in plain sight.\\
During play: adjudicate with the 30-second loop; spend one strong beat; move on.\\
After play: award XP, clear/advance Clocks, note rulings to revisit.

\bigskip
\noindent\textit{If you keep the flow, the game will carry you. The rules are rails you lay just ahead of the train.}


\section{The Director Mindset}

You are not a rules engine. You are a director choosing the next shot:
\begin{itemize}[leftmargin=*]
\item \textbf{Clean win} - Character succeeds, story progresses
\item \textbf{Costly win} - Success with complications
\item \textbf{Partial success} - Progress made, but with gaps
\item \textbf{Spiraling disaster} - Situation escalates dramatically
\end{itemize}

\begin{recommendation}{Embrace Imperfection}
Perfect mechanical resolution is less important than maintaining story momentum. A quick, clear call that keeps the session moving is better than a precise ruling that derails the flow.
\end{recommendation}

\section{Core Quick Tools}

\subsection{1. Five-Second Position \& Effect}

\textbf{Risk Assessment:} What is the Risk?
\begin{itemize}[leftmargin=*]
\item \textbf{Controlled:} Safe, prepared, low threat
\item \textbf{Risky (default):} Pressure, danger, uncertainty
\item \textbf{Desperate:} Immediate danger, overmatched, exposed
\end{itemize}

\textbf{Impact Assessment:} What is the Impact?
\begin{itemize}[leftmargin=*]
\item \textbf{Limited:} Partial progress or minor effect
\item \textbf{Standard (default):} Normal success
\item \textbf{Great:} Powerful, overwhelming, high-impact
\end{itemize}

\begin{quicktool}{Example Usage}
Player charges tougher foe in melee. Risk = Risky, Impact = Standard. "Risky / Standard, DV 3. Roll!"
\end{quicktool}

\subsection{2. The Lazy DV Table}

\begin{center}
\begin{tabularx}{\textwidth}{clX}
\toprule
\textbf{DV} & \textbf{Use When} & \textbf{Example} \\
\midrule
2 & Routine, low stakes & Pick a lock in a safe house \\
3 & Pressured (default) & Pick a lock with guards nearby \\
4 & Hard, hostile conditions & Pick a trapped arcane chest \\
5+ & Extreme, dramatic risk & Pick a magical lock underwater \\
\bottomrule
\end{tabularx}
\end{center}

\begin{recommendation}{Default to DV 3}
When in doubt, DV 3 provides appropriate challenge for most meaningful actions without overcomplicating resolution.
\end{recommendation}

\subsection{3. No-Thinking SB Spending}

When players roll 1s, react instantly. Pick one option and move on:

\begin{center}
\begin{tabularx}{\textwidth}{clX}
\toprule
\textbf{SB} & \textbf{Complication Type} & \textbf{Examples} \\
\midrule
1 & Minor & Noise, track, small loss, clock +1 \\
2 & Moderate & Alarm, lose Position, broken tool, clock +2 \\
3+ & Major & Reinforcements, terrain shift, ally endangered \\
\bottomrule
\end{tabularx}
\end{center}

\begin{recommendation}{Never Overthink SB Spends}
One strong spend is better than multiple minor taxes. Bank sparingly and resolve quickly to maintain momentum.
\end{recommendation}

\subsection{4. The Three-Clock Rule}

Only three active clocks at once:
\begin{itemize}[leftmargin=*]
\item \textbf{Scene Clock:} e.g., Guards Alerted [4]
\item \textbf{Journey Clock:} e.g., Mist Encroaches [6]
\item \textbf{Campaign Clock:} e.g., Baron's Suspicion [8]
\end{itemize}

\begin{recommendation}{Fold New Complications}
When new complications arise, integrate them into existing clocks rather than creating new ones. This keeps focus tight and prevents clock sprawl.
\end{recommendation}

\subsection{5. Magic Adjudication Shortcut}

If a player invents a spell on the fly:
\begin{enumerate}[leftmargin=*]
\item Hear the intent
\item Assign a fitting TAG (Veil, Ward, Barrier, Glamour, etc.)
\item Pick DV based on scale:
  \begin{itemize}
  \item Personal / Subtle = DV 2
  \item Scene-wide / Strong = DV 3
  \item Area / Devastating = DV 4+
  \end{itemize}
\item Choose Position based on danger while casting
\end{enumerate}

\begin{quicktool}{Example}
Creating force wall under fire: TAG = Barrier, scale = Scene $\Rightarrow$ DV 3. Casting in melee $\Rightarrow$ Risky. "Risky / Standard, DV 3. Roll."
\end{quicktool}

\section{Preparation Workflow}

\subsection{Pre-Session (15 minutes)}

\begin{enumerate}[leftmargin=*]
\item Use Five-Second Position/Effect for major scenes
\item Apply Lazy DV Table to key challenges
\item Set up Three-Clock framework
\item Bank 2-3 SB spends per major scene
\item Prep 1-2 magic adjudications
\end{enumerate}

\begin{recommendation}{Prep as Story Beats}
Focus on "What happens if..." rather than mechanical stats. Prepare compelling complications and interesting outcomes.
\end{recommendation}

\section{Session Management}

\subsection{Core Principles}

\begin{itemize}[leftmargin=*]
\item Keep mechanical resolution under 10 seconds
\item Always spend SB immediately, don't hoard
\item Never exceed three active clocks
\item Reassess Position every major beat
\end{itemize}

\begin{recommendation}{Trust the Framework}
The Quick-Kit tools aren't training wheels—they're professional equipment that experienced directors use to keep focus on story.
\end{recommendation}

\section{Character Integration}

\subsection{New Player Approach}

\begin{enumerate}[leftmargin=*]
\item Ask "What story role do you want to play?"
\item Help build mechanics to support that role
\item Focus on narrative contributions over mechanical optimization
\item Introduce complexity gradually as player comfort grows
\end{enumerate}

\subsection{Veteran Player Integration}

\begin{enumerate}[leftmargin=*]
\item Leverage existing system knowledge
\item Encourage creative mechanical combinations
\item Challenge with complex multi-domain scenarios
\item Support high-tier mythic storytelling
\end{enumerate}

\section{Domain Management}

\subsection{Multi-Domain Adventures}

\begin{itemize}[leftmargin=*]
\item Each character's specialties should have moments to shine
\item Integrate assets naturally as story elements, not just bonuses
\item Use followers for support roles that enhance scenes
\item Let clocks interact and influence each other naturally
\end{itemize}

\begin{recommendation}{Domain Integration}
Set 2-3 domain-appropriate clocks, ask "Who can contribute their specialty here?" each scene, and let mechanical elements serve narrative advancement.
\end{recommendation}

\section{High-Tier Play}

\subsection{Tier V Considerations}

\begin{itemize}[leftmargin=*]
\item Scale fiction, not mechanics
\item Treat assets as enablers for realm-level actions
\item Ensure consequences match mythic stakes
\item Make follower management strategic rather than tactical
\item Use clocks to represent ongoing sagas, not temporary complications
\end{itemize}

\begin{recommendation}{Mythic Scope}
Tier V adventures should create legends that define the world. Focus on permanent changes to setting and character transformation rather than bigger numbers.
\end{recommendation}

\section{Continuity Management}

\subsection{New Character Integration}

\begin{enumerate}[leftmargin=*]
\item Position new characters as essential support roles
\item Establish immediate bonds with existing characters
\item Connect new character skills to ongoing campaign threads
\item Provide meaningful contributions without overshadowing veterans
\end{enumerate}

\begin{recommendation}{Supporting Cast Approach}
New characters don't need to be mythic heroes—essential specialists, trusted allies, or community anchors can be equally engaging and easier to integrate.
\end{recommendation}

\section{Troubleshooting Guide}

\subsection{Common Issues and Solutions}

\begin{description}[leftmargin=*]
\item[Analysis Paralysis:] Use the Lazy DV Table and default to DV 3. Make quick decisions and maintain momentum.
\item[Mechanical Overload:] Apply the Three-Clock Rule. Simplify asset states to Maintained/Neglected/Compromised.
\item[Player Disengagement:] Ensure each character has spotlight moments. Rotate complications and bonds to different strengths.
\item[Session Lag:] Trust the fail-forward system. Every roll should change the story, even misses through Boons.
\end{description}

\subsection{Emergency Reset}

When sessions lose focus:
\begin{enumerate}[leftmargin=*]
\item Identify the core story question
\item Strip down to Three-Clock framework
\item Give each player one clear action to take
\item Resolve with quick Position/DV calls
\item Escalate through SB spends until momentum returns
\end{enumerate}

\section{Final Wisdom}

\begin{gmbox}{Remember Your Role}
You are not the rules expert—you are the story facilitator. The players will help you remember details. Focus on pacing, tension, and narrative flow.
\end{gmbox}

\begin{gmbox}{Embrace the Economy}
Story Beats and Boons are your storytelling tools, not bookkeeping. They represent the world responding to character actions, not mechanical penalties.
\end{gmbox}

\begin{gmbox}{Trust Your Players}
Players want the story to succeed as much as you do. When in doubt, ask for their input or go with the interpretation that makes the story more interesting.
\end{gmbox}

\section{Design Philosophy Guardrails (Flow-First GMing)}
\label{sec:design-guardrails}

Fate’s Edge is built to \textbf{keep play flowing}. If you remember nothing else: \textbf{The Narrative is primary}. 
Mechanics exist to shape \emph{how} the story changes, not \emph{whether} it moves. 
This section translates the rules into plain, table-ready guidance—especially for new GMs.

\subsection{Simple Translations}
\begin{description}[leftmargin=1.5em, style=nextline]
  \item[Story Beats (SB) $\Rightarrow$ Story Beats:] 1s on dice give you \emph{beats} to spend. Spend them on twists, escalations, or new information. One strong beat is better than three tiny ones.
  \item[Clocks $\Rightarrow$ Checkboxes/Lists:] A Clock is just a short checklist that tracks progress or rising danger. When it fills, the listed thing \emph{happens}. Name it and tick it when fiction leans that way.
  \item[\texttt{[TAGS]} $\Rightarrow$ Gates with a Cost:] Tags are labels that unlock specific effects (e.g.\ \texttt{[WARD]}, \texttt{[BANISH]}). They don’t do anything alone. They appear on Talents, Rites, or Spells to say, ``Yes, you can do this—\emph{here’s the price and limits}.''
\end{description}

\subsection{The 30-Second Adjudication Loop}
Use this loop to resolve almost anything without breaking flow.
\begin{enumerate}
  \item \textbf{Clarify intent and approach.} ``What do you want, and how?''
  \item \textbf{Set stakes and Position.} ``If it works, what changes? If it fails, what bites?'' Start \emph{Controlled/Standard} unless fiction says otherwise.
  \item \textbf{Roll \& read.} Count 6+ as successes; each \textbf{1} gives you SB (beats). Compare successes to DV.
  \item \textbf{Spend one beat well.} Cash SB on one memorable twist or tick a relevant Clock.
  \item \textbf{Push forward.} Describe how the fiction is now different; ask, ``Who moves next?''
\end{enumerate}

\subsection{When to Reach for Mechanics (and When Not To)}
\begin{itemize}
  \item \textbf{Roll} when uncertainty + meaningful stakes exist \emph{now}. Otherwise, say ``Yes'' or offer a choice/cost.
  \item \textbf{Use a Clock} when danger or progress builds over time (guard alert, ritual, chase, social sway).
  \item \textbf{Draw from the Deck} when you want an oracular twist consistent with the current tone.
  \item \textbf{Skip subsystems} if they slow the table. You can always tick a Clock and move on.
\end{itemize}

\subsection{Defaults That Keep Things Moving}
\begin{itemize}
  \item \textbf{Range/Position:} Assume \emph{Near} and \emph{Controlled/Standard}. Ask: ``Do you need a beat to get there?''
  \item \textbf{DV:} 2 for small/local, 3 for scene-scale, 4 for big swings, 5+ for set-pieces/rituals.
  \item \textbf{Boons:} Misses on meaningful actions grant Boons (player fuel). Trim to 2 at scene end.
  \item \textbf{SB Budget:} Prefer one strong spend over many petty taxes. Bank sparingly and pay off soon.
\end{itemize}

\subsection{Rookie GM Comfort Dials}
You can use these dials to simplify play, then loosen them later.
\begin{description}[leftmargin=1.5em, style=nextline]
  \item[Soft SB:] For your first 2 sessions, cap each roll’s SB spend to \textbf{1--2} unless it’s a set-piece.
  \item[Visible Clocks:] Put Clocks on the table. Name them aloud: \emph{``Guards Incoming [4]''}. Tick them in ink.
  \item[Tag Cards:] Print a one-liner for frequently used Tags (\texttt{[WARD]}, \texttt{[BANISH]}, \texttt{[COUNTER]}). Hand them out when a power is active.
  \item[One Move, One Sentence:] Every ruling should end with one sentence that states the new situation.
\end{description}

\subsection{Narrative-First Rulings (with Examples)}
\paragraph{Example 1: The Locked Gate}
Player: ``I pick the lock fast before the patrol rounds the corner.''\\
GM: ``Controlled/Standard, DV 3. If it works, you’re through; if it fails, the patrol clocks closer.''\\
Roll shows 1 SB. GM spends 1 SB to tick \emph{Guards Incoming [4]}. ``You’re through, but boots echo—two ticks left.'' \emph{Flow continues.}

\paragraph{Example 2: The Shadow Rite}
Player Invokes a \texttt{[WARD]}. ``You’re safe unless Outsiders test the edge: DV = Cap. If one hits, its Leash gains +DV. Your Push would add +1 Obligation—do you Push?'' The scene stays in motion; costs and gates are clear.

\paragraph{Example 3: Fire Cast Backlash}
Caster hits but shows two 1s. GM picks one strong backlash: ``Flare blinds you; Position -1 for the next action.'' No rules dive; \emph{one beat lands}, story moves.

\subsection{Let the Fiction Lead}
\begin{itemize}
  \item Say what the world does next. If a rule is unclear, follow the fiction and note a ruling; refine between sessions.
  \item If you forget a tag nuance, ask: ``What is the effect trying to \emph{gate}?'' Charge a cost (time, risk, Obligation, or a tick), then go.
  \item Tie SB spends to \textbf{visible} outcomes: a new foe appears, a path closes, a clock advances.
\end{itemize}

\subsection{Common Pitfalls and Fixes}
\begin{description}[leftmargin=1.5em, style=nextline]
  \item[Over-cranking SB:] If scenes feel punitive, halve your SB spends for a while or cash them into visible Clocks instead of immediate penalties.
  \item[Clock Sprawl:] Merge redundant Clocks. Each active scene rarely needs more than \textbf{2--3}.
  \item[Tag Paralysis:] If a player stalls waiting for a perfect tag, paraphrase: ``Sounds like \texttt{[VEIL]}. DV 3. Want to roll?''
  \item[Rules Drift:] If table memory conflicts with text, pick the ruling that keeps flow, then sticky-note a TODO to reconcile after play.
\end{description}

\subsection{The Four Questions (Cheat Prompts)}
When stuck, ask out loud:
\begin{enumerate}
  \item If this goes right, what changes? (\emph{Intent})
  \item If this goes wrong, what bites back? (\emph{Stakes})
  \item What single twist will make this memorable? (\emph{SB spend})
  \item Who moves next? (\emph{Momentum})
\end{enumerate}

\subsection{Design Guardrails (for Consistency)}
\begin{itemize}
  \item \textbf{Narrative Primacy:} Mechanics serve story, not replace it.
  \item \textbf{Risk as Drama:} Every roll carries potential for triumph+complication.
  \item \textbf{Meaningful Growth:} XP changes characters and the world.
  \item \textbf{Consequence Weight:} Choices ripple outward; nothing is free.
  \item \textbf{Fail Forward:} Misses fuel Boons; 1s become SB (beats).
\end{itemize}

\subsection{Session Checklist (One Page)}
Before play: set tone, stakes, and clocks in plain sight.\\
During play: adjudicate with the 30-second loop; spend one strong beat; move on.\\
After play: award XP, clear/advance Clocks, note rulings to revisit.

\bigskip
\noindent\textit{If you keep the flow, the game will carry you. The rules are rails you lay just ahead of the train.}


\section{Design Philosophy Guardrails (Flow-First GMing)}
\label{sec:design-guardrails}

Fate’s Edge is built to \textbf{keep play flowing}. If you remember nothing else: \textbf{The Narrative is primary}. 
Mechanics exist to shape \emph{how} the story changes, not \emph{whether} it moves. 
This section translates the rules into plain, table-ready guidance—especially for new GMs.

\subsection{Simple Translations}
\begin{description}[leftmargin=1.5em, style=nextline]
  \item[Complication Points (CP) $\Rightarrow$ Story Beats:] 1s on dice give you \emph{beats} to spend. Spend them on twists, escalations, or new information. One strong beat is better than three tiny ones.
  \item[Clocks $\Rightarrow$ Checkboxes/Lists:] A Clock is just a short checklist that tracks progress or rising danger. When it fills, the listed thing \emph{happens}. Name it and tick it when fiction leans that way.
  \item[\texttt{[TAGS]} $\Rightarrow$ Gates with a Cost:] Tags are labels that unlock specific effects (e.g.\ \texttt{[WARD]}, \texttt{[BANISH]}). They don’t do anything alone. They appear on Talents, Rites, or Spells to say, ``Yes, you can do this—\emph{here’s the price and limits}.''
\end{description}

\subsection{The 30-Second Adjudication Loop}
Use this loop to resolve almost anything without breaking flow.
\begin{enumerate}
  \item \textbf{Clarify intent and approach.} ``What do you want, and how?''
  \item \textbf{Set stakes and Position.} ``If it works, what changes? If it fails, what bites?'' Start \emph{Risky/Standard} unless fiction says otherwise.
  \item \textbf{Roll \& read.} Count 6+ as successes; each \textbf{1} gives you CP (beats). Compare successes to DV.
  \item \textbf{Spend one beat well.} Cash CP on one memorable twist or tick a relevant Clock.
  \item \textbf{Push forward.} Describe how the fiction is now different; ask, ``Who moves next?''
\end{enumerate}

\subsection{When to Reach for Mechanics (and When Not To)}
\begin{itemize}
  \item \textbf{Roll} when uncertainty + meaningful stakes exist \emph{now}. Otherwise, say ``Yes'' or offer a choice/cost.
  \item \textbf{Use a Clock} when danger or progress builds over time (guard alert, ritual, chase, social sway).
  \item \textbf{Draw from the Deck} when you want an oracular twist consistent with the current tone.
  \item \textbf{Skip subsystems} if they slow the table. You can always tick a Clock and move on.
\end{itemize}

\subsection{Defaults That Keep Things Moving}
\begin{itemize}
  \item \textbf{Range/Position:} Assume \emph{Near} and \emph{Risky/Standard}. Ask: ``Do you need a beat to get there?''
  \item \textbf{DV:} 2 for small/local, 3 for scene-scale, 4 for big swings, 5+ for set-pieces/rituals.
  \item \textbf{Boons:} Misses on meaningful actions grant Boons (player fuel). Trim to 2 at scene end.
  \item \textbf{CP Budget:} Prefer one strong spend over many petty taxes. Bank sparingly and pay off soon.
\end{itemize}

\subsection{Rookie GM Comfort Dials}
You can use these dials to simplify play, then loosen them later.
\begin{description}[leftmargin=1.5em, style=nextline]
  \item[Soft CP:] For your first 2 sessions, cap each roll’s CP spend to \textbf{1--2} unless it’s a set-piece.
  \item[Visible Clocks:] Put Clocks on the table. Name them aloud: \emph{``Guards Incoming [4]''}. Tick them in ink.
  \item[Tag Cards:] Print a one-liner for frequently used Tags (\texttt{[WARD]}, \texttt{[BANISH]}, \texttt{[COUNTER]}). Hand them out when a power is active.
  \item[One Move, One Sentence:] Every ruling should end with one sentence that states the new situation.
\end{description}

\subsection{Narrative-First Rulings (with Examples)}
\paragraph{Example 1: The Locked Gate}
Player: ``I pick the lock fast before the patrol rounds the corner.''\\
GM: ``Risky/Standard, DV 3. If it works, you’re through; if it fails, the patrol clocks closer.''\\
Roll shows 1 CP. GM spends 1 CP to tick \emph{Guards Incoming [4]}. ``You’re through, but boots echo—two ticks left.'' \emph{Flow continues.}

\paragraph{Example 2: The Shadow Rite}
Player Invokes a \texttt{[WARD]}. ``You’re safe unless Outsiders test the edge: DV = Cap. If one hits, its Leash gains +DV. Your Push would add +1 Obligation—do you Push?'' The scene stays in motion; costs and gates are clear.

\paragraph{Example 3: Fire Cast Backlash}
Caster hits but shows two 1s. GM picks one strong backlash: ``Flare blinds you; Position -1 for the next action.'' No rules dive; \emph{one beat lands}, story moves.

\subsection{Let the Fiction Lead}
\begin{itemize}
  \item Say what the world does next. If a rule is unclear, follow the fiction and note a ruling; refine between sessions.
  \item If you forget a tag nuance, ask: ``What is the effect trying to \emph{gate}?'' Charge a cost (time, risk, Obligation, or a tick), then go.
  \item Tie CP spends to \textbf{visible} outcomes: a new foe appears, a path closes, a clock advances.
\end{itemize}

\subsection{Common Pitfalls and Fixes}
\begin{description}[leftmargin=1.5em, style=nextline]
  \item[Over-cranking CP:] If scenes feel punitive, halve your CP spends for a while or cash them into visible Clocks instead of immediate penalties.
  \item[Clock Sprawl:] Merge redundant Clocks. Each active scene rarely needs more than \textbf{2--3}.
  \item[Tag Paralysis:] If a player stalls waiting for a perfect tag, paraphrase: ``Sounds like \texttt{[VEIL]}. DV 3. Want to roll?''
  \item[Rules Drift:] If table memory conflicts with text, pick the ruling that keeps flow, then sticky-note a TODO to reconcile after play.
\end{description}

\subsection{The Four Questions (Cheat Prompts)}
When stuck, ask out loud:
\begin{enumerate}
  \item If this goes right, what changes? (\emph{Intent})
  \item If this goes wrong, what bites back? (\emph{Stakes})
  \item What single twist will make this memorable? (\emph{CP spend})
  \item Who moves next? (\emph{Momentum})
\end{enumerate}

\subsection{Design Guardrails (for Consistency)}
\begin{itemize}
  \item \textbf{Narrative Primacy:} Mechanics serve story, not replace it.
  \item \textbf{Risk as Drama:} Every roll carries potential for triumph+complication.
  \item \textbf{Meaningful Growth:} XP changes characters and the world.
  \item \textbf{Consequence Weight:} Choices ripple outward; nothing is free.
  \item \textbf{Fail Forward:} Misses fuel Boons; 1s become CP (beats).
\end{itemize}

\subsection{Session Checklist (One Page)}
Before play: set tone, stakes, and clocks in plain sight.\\
During play: adjudicate with the 30-second loop; spend one strong beat; move on.\\
After play: award XP, clear/advance Clocks, note rulings to revisit.

\bigskip
\noindent\textit{If you keep the flow, the game will carry you. The rules are rails you lay just ahead of the train.}
