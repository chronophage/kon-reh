
% --- Fate's Edge SRD — Section 20: Design Philosophy Guardrails ---
% Include this file from your main .tex with: 
% --- Fate's Edge SRD — Section 20: Design Philosophy Guardrails ---
% Include this file from your main .tex with: 
% --- Fate's Edge SRD — Section 20: Design Philosophy Guardrails ---
% Include this file from your main .tex with: 
% --- Fate's Edge SRD — Section 20: Design Philosophy Guardrails ---
% Include this file from your main .tex with: \input{20-design-guardrails.tex}

\section{Design Philosophy Guardrails (Flow-First GMing)}
\label{sec:design-guardrails}

Fate’s Edge is built to \textbf{keep play flowing}. If you remember nothing else: \textbf{The Narrative is primary}. 
Mechanics exist to shape \emph{how} the story changes, not \emph{whether} it moves. 
This section translates the rules into plain, table-ready guidance—especially for new GMs.

\subsection{Simple Translations}
\begin{description}[leftmargin=1.5em, style=nextline]
  \item[Story Beats (SB) $\Rightarrow$ Story Beats:] 1s on dice give you \emph{beats} to spend. Spend them on twists, escalations, or new information. One strong beat is better than three tiny ones.
  \item[Clocks $\Rightarrow$ Checkboxes/Lists:] A Clock is just a short checklist that tracks progress or rising danger. When it fills, the listed thing \emph{happens}. Name it and tick it when fiction leans that way.
  \item[\texttt{[TAGS]} $\Rightarrow$ Gates with a Cost:] Tags are labels that unlock specific effects (e.g.\ \texttt{[WARD]}, \texttt{[BANISH]}). They don’t do anything alone. They appear on Talents, Rites, or Spells to say, ``Yes, you can do this—\emph{here’s the price and limits}.''
\end{description}

\subsection{The 30-Second Adjudication Loop}
Use this loop to resolve almost anything without breaking flow.
\begin{enumerate}
  \item \textbf{Clarify intent and approach.} ``What do you want, and how?''
  \item \textbf{Set stakes and Position.} ``If it works, what changes? If it fails, what bites?'' Start \emph{Risky/Standard} unless fiction says otherwise.
  \item \textbf{Roll \& read.} Count 6+ as successes; each \textbf{1} gives you SB (beats). Compare successes to DV.
  \item \textbf{Spend one beat well.} Cash SB on one memorable twist or tick a relevant Clock.
  \item \textbf{Push forward.} Describe how the fiction is now different; ask, ``Who moves next?''
\end{enumerate}

\subsection{When to Reach for Mechanics (and When Not To)}
\begin{itemize}
  \item \textbf{Roll} when uncertainty + meaningful stakes exist \emph{now}. Otherwise, say ``Yes'' or offer a choice/cost.
  \item \textbf{Use a Clock} when danger or progress builds over time (guard alert, ritual, chase, social sway).
  \item \textbf{Draw from the Deck} when you want an oracular twist consistent with the current tone.
  \item \textbf{Skip subsystems} if they slow the table. You can always tick a Clock and move on.
\end{itemize}

\subsection{Defaults That Keep Things Moving}
\begin{itemize}
  \item \textbf{Range/Position:} Assume \emph{Near} and \emph{Risky/Standard}. Ask: ``Do you need a beat to get there?''
  \item \textbf{DV:} 2 for small/local, 3 for scene-scale, 4 for big swings, 5+ for set-pieces/rituals.
  \item \textbf{Boons:} Misses on meaningful actions grant Boons (player fuel). Trim to 2 at scene end.
  \item \textbf{SB Budget:} Prefer one strong spend over many petty taxes. Bank sparingly and pay off soon.
\end{itemize}

\subsection{Rookie GM Comfort Dials}
You can use these dials to simplify play, then loosen them later.
\begin{description}[leftmargin=1.5em, style=nextline]
  \item[Soft SB:] For your first 2 sessions, cap each roll’s SB spend to \textbf{1--2} unless it’s a set-piece.
  \item[Visible Clocks:] Put Clocks on the table. Name them aloud: \emph{``Guards Incoming [4]''}. Tick them in ink.
  \item[Tag Cards:] Print a one-liner for frequently used Tags (\texttt{[WARD]}, \texttt{[BANISH]}, \texttt{[COUNTER]}). Hand them out when a power is active.
  \item[One Move, One Sentence:] Every ruling should end with one sentence that states the new situation.
\end{description}

\subsection{Narrative-First Rulings (with Examples)}
\paragraph{Example 1: The Locked Gate}
Player: ``I pick the lock fast before the patrol rounds the corner.''\\
GM: ``Risky/Standard, DV 3. If it works, you’re through; if it fails, the patrol clocks closer.''\\
Roll shows 1 SB. GM spends 1 SB to tick \emph{Guards Incoming [4]}. ``You’re through, but boots echo—two ticks left.'' \emph{Flow continues.}

\paragraph{Example 2: The Shadow Rite}
Player Invokes a \texttt{[WARD]}. ``You’re safe unless Outsiders test the edge: DV = Cap. If one hits, its Leash gains +DV. Your Push would add +1 Obligation—do you Push?'' The scene stays in motion; costs and gates are clear.

\paragraph{Example 3: Fire Cast Backlash}
Caster hits but shows two 1s. GM picks one strong backlash: ``Flare blinds you; Position -1 for the next action.'' No rules dive; \emph{one beat lands}, story moves.

\subsection{Let the Fiction Lead}
\begin{itemize}
  \item Say what the world does next. If a rule is unclear, follow the fiction and note a ruling; refine between sessions.
  \item If you forget a tag nuance, ask: ``What is the effect trying to \emph{gate}?'' Charge a cost (time, risk, Obligation, or a tick), then go.
  \item Tie SB spends to \textbf{visible} outcomes: a new foe appears, a path closes, a clock advances.
\end{itemize}

\subsection{Common Pitfalls and Fixes}
\begin{description}[leftmargin=1.5em, style=nextline]
  \item[Over-cranking SB:] If scenes feel punitive, halve your SB spends for a while or cash them into visible Clocks instead of immediate penalties.
  \item[Clock Sprawl:] Merge redundant Clocks. Each active scene rarely needs more than \textbf{2--3}.
  \item[Tag Paralysis:] If a player stalls waiting for a perfect tag, paraphrase: ``Sounds like \texttt{[VEIL]}. DV 3. Want to roll?''
  \item[Rules Drift:] If table memory conflicts with text, pick the ruling that keeps flow, then sticky-note a TODO to reconcile after play.
\end{description}

\subsection{The Four Questions (Cheat Prompts)}
When stuck, ask out loud:
\begin{enumerate}
  \item If this goes right, what changes? (\emph{Intent})
  \item If this goes wrong, what bites back? (\emph{Stakes})
  \item What single twist will make this memorable? (\emph{SB spend})
  \item Who moves next? (\emph{Momentum})
\end{enumerate}

\subsection{Design Guardrails (for Consistency)}
\begin{itemize}
  \item \textbf{Narrative Primacy:} Mechanics serve story, not replace it.
  \item \textbf{Risk as Drama:} Every roll carries potential for triumph+complication.
  \item \textbf{Meaningful Growth:} XP changes characters and the world.
  \item \textbf{Consequence Weight:} Choices ripple outward; nothing is free.
  \item \textbf{Fail Forward:} Misses fuel Boons; 1s become SB (beats).
\end{itemize}

\subsection{Session Checklist (One Page)}
Before play: set tone, stakes, and clocks in plain sight.\\
During play: adjudicate with the 30-second loop; spend one strong beat; move on.\\
After play: award XP, clear/advance Clocks, note rulings to revisit.

\bigskip
\noindent\textit{If you keep the flow, the game will carry you. The rules are rails you lay just ahead of the train.}


\section{Design Philosophy Guardrails (Flow-First GMing)}
\label{sec:design-guardrails}

Fate’s Edge is built to \textbf{keep play flowing}. If you remember nothing else: \textbf{The Narrative is primary}. 
Mechanics exist to shape \emph{how} the story changes, not \emph{whether} it moves. 
This section translates the rules into plain, table-ready guidance—especially for new GMs.

\subsection{Simple Translations}
\begin{description}[leftmargin=1.5em, style=nextline]
  \item[Story Beats (SB) $\Rightarrow$ Story Beats:] 1s on dice give you \emph{beats} to spend. Spend them on twists, escalations, or new information. One strong beat is better than three tiny ones.
  \item[Clocks $\Rightarrow$ Checkboxes/Lists:] A Clock is just a short checklist that tracks progress or rising danger. When it fills, the listed thing \emph{happens}. Name it and tick it when fiction leans that way.
  \item[\texttt{[TAGS]} $\Rightarrow$ Gates with a Cost:] Tags are labels that unlock specific effects (e.g.\ \texttt{[WARD]}, \texttt{[BANISH]}). They don’t do anything alone. They appear on Talents, Rites, or Spells to say, ``Yes, you can do this—\emph{here’s the price and limits}.''
\end{description}

\subsection{The 30-Second Adjudication Loop}
Use this loop to resolve almost anything without breaking flow.
\begin{enumerate}
  \item \textbf{Clarify intent and approach.} ``What do you want, and how?''
  \item \textbf{Set stakes and Position.} ``If it works, what changes? If it fails, what bites?'' Start \emph{Risky/Standard} unless fiction says otherwise.
  \item \textbf{Roll \& read.} Count 6+ as successes; each \textbf{1} gives you SB (beats). Compare successes to DV.
  \item \textbf{Spend one beat well.} Cash SB on one memorable twist or tick a relevant Clock.
  \item \textbf{Push forward.} Describe how the fiction is now different; ask, ``Who moves next?''
\end{enumerate}

\subsection{When to Reach for Mechanics (and When Not To)}
\begin{itemize}
  \item \textbf{Roll} when uncertainty + meaningful stakes exist \emph{now}. Otherwise, say ``Yes'' or offer a choice/cost.
  \item \textbf{Use a Clock} when danger or progress builds over time (guard alert, ritual, chase, social sway).
  \item \textbf{Draw from the Deck} when you want an oracular twist consistent with the current tone.
  \item \textbf{Skip subsystems} if they slow the table. You can always tick a Clock and move on.
\end{itemize}

\subsection{Defaults That Keep Things Moving}
\begin{itemize}
  \item \textbf{Range/Position:} Assume \emph{Near} and \emph{Risky/Standard}. Ask: ``Do you need a beat to get there?''
  \item \textbf{DV:} 2 for small/local, 3 for scene-scale, 4 for big swings, 5+ for set-pieces/rituals.
  \item \textbf{Boons:} Misses on meaningful actions grant Boons (player fuel). Trim to 2 at scene end.
  \item \textbf{SB Budget:} Prefer one strong spend over many petty taxes. Bank sparingly and pay off soon.
\end{itemize}

\subsection{Rookie GM Comfort Dials}
You can use these dials to simplify play, then loosen them later.
\begin{description}[leftmargin=1.5em, style=nextline]
  \item[Soft SB:] For your first 2 sessions, cap each roll’s SB spend to \textbf{1--2} unless it’s a set-piece.
  \item[Visible Clocks:] Put Clocks on the table. Name them aloud: \emph{``Guards Incoming [4]''}. Tick them in ink.
  \item[Tag Cards:] Print a one-liner for frequently used Tags (\texttt{[WARD]}, \texttt{[BANISH]}, \texttt{[COUNTER]}). Hand them out when a power is active.
  \item[One Move, One Sentence:] Every ruling should end with one sentence that states the new situation.
\end{description}

\subsection{Narrative-First Rulings (with Examples)}
\paragraph{Example 1: The Locked Gate}
Player: ``I pick the lock fast before the patrol rounds the corner.''\\
GM: ``Risky/Standard, DV 3. If it works, you’re through; if it fails, the patrol clocks closer.''\\
Roll shows 1 SB. GM spends 1 SB to tick \emph{Guards Incoming [4]}. ``You’re through, but boots echo—two ticks left.'' \emph{Flow continues.}

\paragraph{Example 2: The Shadow Rite}
Player Invokes a \texttt{[WARD]}. ``You’re safe unless Outsiders test the edge: DV = Cap. If one hits, its Leash gains +DV. Your Push would add +1 Obligation—do you Push?'' The scene stays in motion; costs and gates are clear.

\paragraph{Example 3: Fire Cast Backlash}
Caster hits but shows two 1s. GM picks one strong backlash: ``Flare blinds you; Position -1 for the next action.'' No rules dive; \emph{one beat lands}, story moves.

\subsection{Let the Fiction Lead}
\begin{itemize}
  \item Say what the world does next. If a rule is unclear, follow the fiction and note a ruling; refine between sessions.
  \item If you forget a tag nuance, ask: ``What is the effect trying to \emph{gate}?'' Charge a cost (time, risk, Obligation, or a tick), then go.
  \item Tie SB spends to \textbf{visible} outcomes: a new foe appears, a path closes, a clock advances.
\end{itemize}

\subsection{Common Pitfalls and Fixes}
\begin{description}[leftmargin=1.5em, style=nextline]
  \item[Over-cranking SB:] If scenes feel punitive, halve your SB spends for a while or cash them into visible Clocks instead of immediate penalties.
  \item[Clock Sprawl:] Merge redundant Clocks. Each active scene rarely needs more than \textbf{2--3}.
  \item[Tag Paralysis:] If a player stalls waiting for a perfect tag, paraphrase: ``Sounds like \texttt{[VEIL]}. DV 3. Want to roll?''
  \item[Rules Drift:] If table memory conflicts with text, pick the ruling that keeps flow, then sticky-note a TODO to reconcile after play.
\end{description}

\subsection{The Four Questions (Cheat Prompts)}
When stuck, ask out loud:
\begin{enumerate}
  \item If this goes right, what changes? (\emph{Intent})
  \item If this goes wrong, what bites back? (\emph{Stakes})
  \item What single twist will make this memorable? (\emph{SB spend})
  \item Who moves next? (\emph{Momentum})
\end{enumerate}

\subsection{Design Guardrails (for Consistency)}
\begin{itemize}
  \item \textbf{Narrative Primacy:} Mechanics serve story, not replace it.
  \item \textbf{Risk as Drama:} Every roll carries potential for triumph+complication.
  \item \textbf{Meaningful Growth:} XP changes characters and the world.
  \item \textbf{Consequence Weight:} Choices ripple outward; nothing is free.
  \item \textbf{Fail Forward:} Misses fuel Boons; 1s become SB (beats).
\end{itemize}

\subsection{Session Checklist (One Page)}
Before play: set tone, stakes, and clocks in plain sight.\\
During play: adjudicate with the 30-second loop; spend one strong beat; move on.\\
After play: award XP, clear/advance Clocks, note rulings to revisit.

\bigskip
\noindent\textit{If you keep the flow, the game will carry you. The rules are rails you lay just ahead of the train.}


\section{Design Philosophy Guardrails (Flow-First GMing)}
\label{sec:design-guardrails}

Fate’s Edge is built to \textbf{keep play flowing}. If you remember nothing else: \textbf{The Narrative is primary}. 
Mechanics exist to shape \emph{how} the story changes, not \emph{whether} it moves. 
This section translates the rules into plain, table-ready guidance—especially for new GMs.

\subsection{Simple Translations}
\begin{description}[leftmargin=1.5em, style=nextline]
  \item[Story Beats (SB) $\Rightarrow$ Story Beats:] 1s on dice give you \emph{beats} to spend. Spend them on twists, escalations, or new information. One strong beat is better than three tiny ones.
  \item[Clocks $\Rightarrow$ Checkboxes/Lists:] A Clock is just a short checklist that tracks progress or rising danger. When it fills, the listed thing \emph{happens}. Name it and tick it when fiction leans that way.
  \item[\texttt{[TAGS]} $\Rightarrow$ Gates with a Cost:] Tags are labels that unlock specific effects (e.g.\ \texttt{[WARD]}, \texttt{[BANISH]}). They don’t do anything alone. They appear on Talents, Rites, or Spells to say, ``Yes, you can do this—\emph{here’s the price and limits}.''
\end{description}

\subsection{The 30-Second Adjudication Loop}
Use this loop to resolve almost anything without breaking flow.
\begin{enumerate}
  \item \textbf{Clarify intent and approach.} ``What do you want, and how?''
  \item \textbf{Set stakes and Position.} ``If it works, what changes? If it fails, what bites?'' Start \emph{Risky/Standard} unless fiction says otherwise.
  \item \textbf{Roll \& read.} Count 6+ as successes; each \textbf{1} gives you SB (beats). Compare successes to DV.
  \item \textbf{Spend one beat well.} Cash SB on one memorable twist or tick a relevant Clock.
  \item \textbf{Push forward.} Describe how the fiction is now different; ask, ``Who moves next?''
\end{enumerate}

\subsection{When to Reach for Mechanics (and When Not To)}
\begin{itemize}
  \item \textbf{Roll} when uncertainty + meaningful stakes exist \emph{now}. Otherwise, say ``Yes'' or offer a choice/cost.
  \item \textbf{Use a Clock} when danger or progress builds over time (guard alert, ritual, chase, social sway).
  \item \textbf{Draw from the Deck} when you want an oracular twist consistent with the current tone.
  \item \textbf{Skip subsystems} if they slow the table. You can always tick a Clock and move on.
\end{itemize}

\subsection{Defaults That Keep Things Moving}
\begin{itemize}
  \item \textbf{Range/Position:} Assume \emph{Near} and \emph{Risky/Standard}. Ask: ``Do you need a beat to get there?''
  \item \textbf{DV:} 2 for small/local, 3 for scene-scale, 4 for big swings, 5+ for set-pieces/rituals.
  \item \textbf{Boons:} Misses on meaningful actions grant Boons (player fuel). Trim to 2 at scene end.
  \item \textbf{SB Budget:} Prefer one strong spend over many petty taxes. Bank sparingly and pay off soon.
\end{itemize}

\subsection{Rookie GM Comfort Dials}
You can use these dials to simplify play, then loosen them later.
\begin{description}[leftmargin=1.5em, style=nextline]
  \item[Soft SB:] For your first 2 sessions, cap each roll’s SB spend to \textbf{1--2} unless it’s a set-piece.
  \item[Visible Clocks:] Put Clocks on the table. Name them aloud: \emph{``Guards Incoming [4]''}. Tick them in ink.
  \item[Tag Cards:] Print a one-liner for frequently used Tags (\texttt{[WARD]}, \texttt{[BANISH]}, \texttt{[COUNTER]}). Hand them out when a power is active.
  \item[One Move, One Sentence:] Every ruling should end with one sentence that states the new situation.
\end{description}

\subsection{Narrative-First Rulings (with Examples)}
\paragraph{Example 1: The Locked Gate}
Player: ``I pick the lock fast before the patrol rounds the corner.''\\
GM: ``Risky/Standard, DV 3. If it works, you’re through; if it fails, the patrol clocks closer.''\\
Roll shows 1 SB. GM spends 1 SB to tick \emph{Guards Incoming [4]}. ``You’re through, but boots echo—two ticks left.'' \emph{Flow continues.}

\paragraph{Example 2: The Shadow Rite}
Player Invokes a \texttt{[WARD]}. ``You’re safe unless Outsiders test the edge: DV = Cap. If one hits, its Leash gains +DV. Your Push would add +1 Obligation—do you Push?'' The scene stays in motion; costs and gates are clear.

\paragraph{Example 3: Fire Cast Backlash}
Caster hits but shows two 1s. GM picks one strong backlash: ``Flare blinds you; Position -1 for the next action.'' No rules dive; \emph{one beat lands}, story moves.

\subsection{Let the Fiction Lead}
\begin{itemize}
  \item Say what the world does next. If a rule is unclear, follow the fiction and note a ruling; refine between sessions.
  \item If you forget a tag nuance, ask: ``What is the effect trying to \emph{gate}?'' Charge a cost (time, risk, Obligation, or a tick), then go.
  \item Tie SB spends to \textbf{visible} outcomes: a new foe appears, a path closes, a clock advances.
\end{itemize}

\subsection{Common Pitfalls and Fixes}
\begin{description}[leftmargin=1.5em, style=nextline]
  \item[Over-cranking SB:] If scenes feel punitive, halve your SB spends for a while or cash them into visible Clocks instead of immediate penalties.
  \item[Clock Sprawl:] Merge redundant Clocks. Each active scene rarely needs more than \textbf{2--3}.
  \item[Tag Paralysis:] If a player stalls waiting for a perfect tag, paraphrase: ``Sounds like \texttt{[VEIL]}. DV 3. Want to roll?''
  \item[Rules Drift:] If table memory conflicts with text, pick the ruling that keeps flow, then sticky-note a TODO to reconcile after play.
\end{description}

\subsection{The Four Questions (Cheat Prompts)}
When stuck, ask out loud:
\begin{enumerate}
  \item If this goes right, what changes? (\emph{Intent})
  \item If this goes wrong, what bites back? (\emph{Stakes})
  \item What single twist will make this memorable? (\emph{SB spend})
  \item Who moves next? (\emph{Momentum})
\end{enumerate}

\subsection{Design Guardrails (for Consistency)}
\begin{itemize}
  \item \textbf{Narrative Primacy:} Mechanics serve story, not replace it.
  \item \textbf{Risk as Drama:} Every roll carries potential for triumph+complication.
  \item \textbf{Meaningful Growth:} XP changes characters and the world.
  \item \textbf{Consequence Weight:} Choices ripple outward; nothing is free.
  \item \textbf{Fail Forward:} Misses fuel Boons; 1s become SB (beats).
\end{itemize}

\subsection{Session Checklist (One Page)}
Before play: set tone, stakes, and clocks in plain sight.\\
During play: adjudicate with the 30-second loop; spend one strong beat; move on.\\
After play: award XP, clear/advance Clocks, note rulings to revisit.

\bigskip
\noindent\textit{If you keep the flow, the game will carry you. The rules are rails you lay just ahead of the train.}


\section{Design Philosophy Guardrails (Flow-First GMing)}
\label{sec:design-guardrails}

Fate’s Edge is built to \textbf{keep play flowing}. If you remember nothing else: \textbf{The Narrative is primary}. 
Mechanics exist to shape \emph{how} the story changes, not \emph{whether} it moves. 
This section translates the rules into plain, table-ready guidance—especially for new GMs.

\subsection{Simple Translations}
\begin{description}[leftmargin=1.5em, style=nextline]
  \item[Story Beats (SB) $\Rightarrow$ Story Beats:] 1s on dice give you \emph{beats} to spend. Spend them on twists, escalations, or new information. One strong beat is better than three tiny ones.
  \item[Clocks $\Rightarrow$ Checkboxes/Lists:] A Clock is just a short checklist that tracks progress or rising danger. When it fills, the listed thing \emph{happens}. Name it and tick it when fiction leans that way.
  \item[\texttt{[TAGS]} $\Rightarrow$ Gates with a Cost:] Tags are labels that unlock specific effects (e.g.\ \texttt{[WARD]}, \texttt{[BANISH]}). They don’t do anything alone. They appear on Talents, Rites, or Spells to say, ``Yes, you can do this—\emph{here’s the price and limits}.''
\end{description}

\subsection{The 30-Second Adjudication Loop}
Use this loop to resolve almost anything without breaking flow.
\begin{enumerate}
  \item \textbf{Clarify intent and approach.} ``What do you want, and how?''
  \item \textbf{Set stakes and Position.} ``If it works, what changes? If it fails, what bites?'' Start \emph{Risky/Standard} unless fiction says otherwise.
  \item \textbf{Roll \& read.} Count 6+ as successes; each \textbf{1} gives you SB (beats). Compare successes to DV.
  \item \textbf{Spend one beat well.} Cash SB on one memorable twist or tick a relevant Clock.
  \item \textbf{Push forward.} Describe how the fiction is now different; ask, ``Who moves next?''
\end{enumerate}

\subsection{When to Reach for Mechanics (and When Not To)}
\begin{itemize}
  \item \textbf{Roll} when uncertainty + meaningful stakes exist \emph{now}. Otherwise, say ``Yes'' or offer a choice/cost.
  \item \textbf{Use a Clock} when danger or progress builds over time (guard alert, ritual, chase, social sway).
  \item \textbf{Draw from the Deck} when you want an oracular twist consistent with the current tone.
  \item \textbf{Skip subsystems} if they slow the table. You can always tick a Clock and move on.
\end{itemize}

\subsection{Defaults That Keep Things Moving}
\begin{itemize}
  \item \textbf{Range/Position:} Assume \emph{Near} and \emph{Risky/Standard}. Ask: ``Do you need a beat to get there?''
  \item \textbf{DV:} 2 for small/local, 3 for scene-scale, 4 for big swings, 5+ for set-pieces/rituals.
  \item \textbf{Boons:} Misses on meaningful actions grant Boons (player fuel). Trim to 2 at scene end.
  \item \textbf{SB Budget:} Prefer one strong spend over many petty taxes. Bank sparingly and pay off soon.
\end{itemize}

\subsection{Rookie GM Comfort Dials}
You can use these dials to simplify play, then loosen them later.
\begin{description}[leftmargin=1.5em, style=nextline]
  \item[Soft SB:] For your first 2 sessions, cap each roll’s SB spend to \textbf{1--2} unless it’s a set-piece.
  \item[Visible Clocks:] Put Clocks on the table. Name them aloud: \emph{``Guards Incoming [4]''}. Tick them in ink.
  \item[Tag Cards:] Print a one-liner for frequently used Tags (\texttt{[WARD]}, \texttt{[BANISH]}, \texttt{[COUNTER]}). Hand them out when a power is active.
  \item[One Move, One Sentence:] Every ruling should end with one sentence that states the new situation.
\end{description}

\subsection{Narrative-First Rulings (with Examples)}
\paragraph{Example 1: The Locked Gate}
Player: ``I pick the lock fast before the patrol rounds the corner.''\\
GM: ``Risky/Standard, DV 3. If it works, you’re through; if it fails, the patrol clocks closer.''\\
Roll shows 1 SB. GM spends 1 SB to tick \emph{Guards Incoming [4]}. ``You’re through, but boots echo—two ticks left.'' \emph{Flow continues.}

\paragraph{Example 2: The Shadow Rite}
Player Invokes a \texttt{[WARD]}. ``You’re safe unless Outsiders test the edge: DV = Cap. If one hits, its Leash gains +DV. Your Push would add +1 Obligation—do you Push?'' The scene stays in motion; costs and gates are clear.

\paragraph{Example 3: Fire Cast Backlash}
Caster hits but shows two 1s. GM picks one strong backlash: ``Flare blinds you; Position -1 for the next action.'' No rules dive; \emph{one beat lands}, story moves.

\subsection{Let the Fiction Lead}
\begin{itemize}
  \item Say what the world does next. If a rule is unclear, follow the fiction and note a ruling; refine between sessions.
  \item If you forget a tag nuance, ask: ``What is the effect trying to \emph{gate}?'' Charge a cost (time, risk, Obligation, or a tick), then go.
  \item Tie SB spends to \textbf{visible} outcomes: a new foe appears, a path closes, a clock advances.
\end{itemize}

\subsection{Common Pitfalls and Fixes}
\begin{description}[leftmargin=1.5em, style=nextline]
  \item[Over-cranking SB:] If scenes feel punitive, halve your SB spends for a while or cash them into visible Clocks instead of immediate penalties.
  \item[Clock Sprawl:] Merge redundant Clocks. Each active scene rarely needs more than \textbf{2--3}.
  \item[Tag Paralysis:] If a player stalls waiting for a perfect tag, paraphrase: ``Sounds like \texttt{[VEIL]}. DV 3. Want to roll?''
  \item[Rules Drift:] If table memory conflicts with text, pick the ruling that keeps flow, then sticky-note a TODO to reconcile after play.
\end{description}

\subsection{The Four Questions (Cheat Prompts)}
When stuck, ask out loud:
\begin{enumerate}
  \item If this goes right, what changes? (\emph{Intent})
  \item If this goes wrong, what bites back? (\emph{Stakes})
  \item What single twist will make this memorable? (\emph{SB spend})
  \item Who moves next? (\emph{Momentum})
\end{enumerate}

\subsection{Design Guardrails (for Consistency)}
\begin{itemize}
  \item \textbf{Narrative Primacy:} Mechanics serve story, not replace it.
  \item \textbf{Risk as Drama:} Every roll carries potential for triumph+complication.
  \item \textbf{Meaningful Growth:} XP changes characters and the world.
  \item \textbf{Consequence Weight:} Choices ripple outward; nothing is free.
  \item \textbf{Fail Forward:} Misses fuel Boons; 1s become SB (beats).
\end{itemize}

\subsection{Session Checklist (One Page)}
Before play: set tone, stakes, and clocks in plain sight.\\
During play: adjudicate with the 30-second loop; spend one strong beat; move on.\\
After play: award XP, clear/advance Clocks, note rulings to revisit.

\bigskip
\noindent\textit{If you keep the flow, the game will carry you. The rules are rails you lay just ahead of the train.}
