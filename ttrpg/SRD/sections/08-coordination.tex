\chapter{Coordination Rules \& Strategies}
\label{chap:coordination}
\index{Coordination}\index{Teamwork}\index{Assistance}\index{Boon Sharing}\index{Bonds}

\section{Purpose}
\noindent Coordination turns individual actions into decisive outcomes. This chapter defines how allies combine moves, share resources, and sequence actions to overcome threats without slowing play.

\section{Core Principles}
\begin{itemize}
  \item \textbf{Fiction First.} Describe how you help. Mechanics follow the fiction.
  \item \textbf{Clear Stakes.} State Position $\to$ DV, what help changes, and what risk the helper accepts.
  \item \textbf{One Spotlight at a Time.} Resolve one acting character’s roll; fold assistance into that action.
  \item \textbf{Visible Costs.} Story Beats (SB), Obligation, Fatigue, and asset states are tracked openly.
\end{itemize}

\section{Shared Vocabulary}
\begin{description}[leftmargin=1.8em]
  \item[Acting Character] The PC whose roll resolves the team’s immediate goal.
  \item[Assistant] A PC who contributes fictionally; they don’t roll unless the move calls for it.
  \item[Exchange] A short beat of simultaneous activity (often 1 round of table time).
  \item[Range Bands] \emph{Close, Near, Far}—coordination options often require \emph{Near}.
\end{description}

\section{Assistance (Baseline)}
\label{sec:assistance}
\index{Assistance}
\begin{itemize}
  \item \textbf{Declare Help.} An assistant states a concrete contribution (tools, opening, lure, cover).
  \item \textbf{Benefit.} Acting character gains \textbf{+1d} (up to the table’s assist cap). The GM may instead allow \textbf{+1 Position} or \textbf{+1 Effect} if the fiction fits.
  \item \textbf{Limits.} One assistant per PC per exchange by default; followers can assist per their stat block.
  \item \textbf{Cost.} The assistant accepts any oncoming risk named by the GM (SB, Fatigue, collateral).
\end{itemize}

\section{Position $\to$ DV}
\label{sec:position-to-dv}
\index{Position}\index{DV}
\noindent The GM sets Position from fiction; Position maps to DV for the acting roll (typical ladder):
\begin{itemize}
  \item \textbf{Controlled} $\Rightarrow$ \textbf{DV 2} \quad (time, tools, clear access)
  \item \textbf{Risky} $\Rightarrow$ \textbf{DV 3} \quad (pressure, partial access)
  \item \textbf{Desperate} $\Rightarrow$ \textbf{DV 4--5+} \quad (hostile field, countdown)
\end{itemize}

\section{Bonds \& Boon Sharing (Summary)}
\label{sec:bonds-boons-summary}
\index{Bonds}\index{Boon Sharing}
\noindent Bonds signal trusted ties; Boons are the table’s spotlight currency.
\begin{itemize}
  \item \textbf{Hybrid Sharing.} PCs may gift \textbf{1 Boon/scene} to an ally with a brief justification; \textbf{2 Boons} if Bonded (see \S\ref{sec:hybrid-boon-sharing}).
  \item \textbf{Assistance via Boons.} A gifted Boon may count as help on the ally’s next roll.
  \item \textbf{Tracking.} Record shared Boons openly to avoid double-counting.
\end{itemize}

\section{Stacking Limits}
\label{sec:stacking}
\noindent To prevent “one true combo,” a PC may benefit from at most \textbf{two} cooperative effects on the same action (choose which apply). \emph{Inspire}, \emph{Tactical Relay}, and similar talents each count as one.

\section{Timing \& Sequencing}
\label{sec:timing}
\begin{itemize}
  \item \textbf{Declare Order.} GM frames the exchange; players state intent in any order; resolve the acting roll, then apply assists/boons that were declared for it.
  \item \textbf{Ready/Overwatch.} Players may hold an action with a clear trigger; if triggered, resolve before the next exchange starts.
  \item \textbf{Refresh Windows.} “Once/scene” effects reset at scene end; “once/exchange” refresh at the next beat.
\end{itemize}

\section{Followers in Coordination (Brief)}
\label{sec:followers-coord}
\index{Followers}
\noindent A follower may assist for up to \textbf{+3d} (or \textbf{+4d} with \emph{Exceptional Coordination+}); they can’t receive PC-only benefits (e.g., Inspire) unless a talent states otherwise.

\section{Invoker \& Caster Notes (Brief)}
\index{Invokers}\index{Casters}
\begin{itemize}
  \item \textbf{Invokers.} Ritual via Symbol takes \(\text{DV}+1\) rounds and marks +1 Obligation; \emph{Crack the Seal} is instant at the stated costs. Invoker Rites cannot use \emph{Push It}.
  \item \textbf{Casters.} Weave (action) then Cast (action). Allies can assist either step if the fiction allows (lenses, spotters, cover).
\end{itemize}

\section{GM Guidance (One Page)}
\label{sec:coord-gm}
\begin{itemize}
  \item \textbf{Ask for the Beat.} Require one sentence of how help changes the fiction.
  \item \textbf{Name the Risk.} Before rolling, say what the helper risks (SB, collateral, position flip).
  \item \textbf{Favor Position Shifts.} When in doubt, let excellent help improve Position rather than stack dice.
  \item \textbf{Spread the Love.} Rotate who can meaningfully help each exchange; spotlight bonds and distinct roles.
\end{itemize}

\subsection{Cooperative Talents (Options)}

\paragraph{Inspire (3 XP)}
\index{Talents!Inspire}\index{Boon Sharing}\index{Coordination}
Once/scene, spend 1 Boon and provide a brief narrative justification. Choose one:
\begin{itemize}
  \item \textbf{Bonded Ally (Near):} That ally gains \textbf{+1 Boon} and \textbf{+1d} on their next roll this scene.
  \item \textbf{Self:} You gain \textbf{+1d} on your next roll this scene.
  \item \textbf{Rally (Near Allies):} Each other PC in \textbf{Near} gains \textbf{+1d} on their next roll this exchange.
  \item \textbf{Tactical Coordination (Near Allies):} All allies currently acting gain \textbf{+1 Position} on their next action this exchange.
\end{itemize}
\textbf{Limits:} Followers cannot benefit. Each PC can benefit from \emph{Inspire} at most once per scene. Requires \emph{Near} unless targeting \emph{Self}. Not usable during Downtime or purely non-conflict social scenes. \emph{Inspire} counts toward the stacking limit of cooperative effects (see \S\ref{sec:stacking}).

\paragraph{Tactical Relay (3 XP)}
Once/scene, spend 1 Boon: all allies currently acting in \textbf{Near} gain \textbf{+1 Position} on their next action this exchange. Followers excluded.

\paragraph{Shield Wall (4 XP)}
If you and at least one ally each wield a shield and are adjacent: as a \emph{Defend} action, grant \textbf{+1d Defend} to all in the Wall and convert the first incoming Harm (any one) to Fatigue. Ends if formation breaks.

\paragraph{Spotter’s Mark (3 XP)}
\emph{Aim} a target (1 action). Until end of scene or until target breaks line of sight, each \textbf{PC in Near} may claim \textbf{+1d} or \textbf{+1 Effect} once vs. that target. Once/scene you may spend 1 Boon to refresh the mark.

\paragraph{Battle Cant (2 XP)}
Once/scene, establish silent signals. On the next coordinated action where at least two PCs act on the same beat, those PCs gain \textbf{+1 Position}. Spend 1 Boon to include a third PC.

\paragraph{Medic’s Hand (3 XP)}
When you \emph{Stabilize} an ally mid-scene, also \textbf{clear 1 Fatigue} or \textbf{downgrade Harm 2→1}. Spend 1 Boon to do both. Not usable in Downtime healing.

\paragraph{Anchor Sigil (Runekeeper, 4 XP)}
Bank 1 Boon on a prepared sigil. Once this scene, when an ally \emph{Casts} or \emph{Invokes a Rite}, discharge: \textbf{–1 DV} for that action \emph{or} redirect minor backlash to you as \textbf{Fatigue +1}.

\paragraph{Exceptional Coordination+ (8 XP)}
Your follower assist cap increases to \textbf{+4d}. If \textbf{Bonded} with that follower, you may split as \textbf{+2d} to two different allies on the same exchange.


\section{Optional Rule: Multi-Character Followers}
\label{sec:multi-character-followers}

\subsection*{Overview}
A player may control more than one character in a campaign by linking them as mutual Followers. This rule also covers how to use absent players' characters temporarily.

\subsection*{Mechanics}
\begin{itemize}
  \item \textbf{Cap Tier +1:} A Follower may advance up to one Tier higher than the patron’s current Tier (e.g. a Tier~II PC supports a Follower who may reach Cap~III).
  \item \textbf{Multi-Character Play:} A player may designate two PCs as Followers of each other. Only one acts as the \emph{active} character at a time; the other functions as a Follower under normal rules.
  \item \textbf{Absent PCs:} With the absent player’s consent, their PC may be run as a temporary Follower at Cap~Tier +1, ensuring they remain useful without overshadowing the party.
\end{itemize}

\subsection*{Balance Notes}
\begin{itemize}
  \item This option preserves fairness by keeping Followers close in power but still slightly behind their leader.
  \item The action economy remains stable: only one PC per player is active per scene.
  \item It expands narrative flexibility, allowing for legacy characters, continuity, and temporary coverage.
\end{itemize}