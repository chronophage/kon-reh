
% --- Fate's Edge SRD — Section 2: Core Mechanic ---
% Include this file from your main .tex with: 
% --- Fate's Edge SRD — Section 2: Core Mechanic ---
% Include this file from your main .tex with: 
% --- Fate's Edge SRD — Section 2: Core Mechanic ---
% Include this file from your main .tex with: 
% --- Fate's Edge SRD — Section 2: Core Mechanic ---
% Include this file from your main .tex with: \input{02-core-mechanic.tex}

\section{Core Mechanic}

\subsection{The Art of Consequence}

\subsubsection{Procedure}
All significant actions follow a three-step process:
\begin{enumerate}
  \item \textbf{Approach:} The player describes both what their character wants and how they attempt it.
  \item \textbf{Execution:} Build a dice pool equal to \emph{Attribute + Skill} and roll that many d10s. Each die of \textbf{6+} counts as a success. Each \textbf{1} rolled generates a \textbf{Story Beat (SB)}.
  \item \textbf{Outcome:} The GM interprets total successes against the difficulty (DV) of the task. Story Beats are then spent to weave narrative setbacks.
\end{enumerate}

\subsubsection{The Description Ladder}
\begin{itemize}
  \item \textbf{Basic Action:} Roll the pool as-is. All 1s remain as Story Beats.
  \item \textbf{Detailed Action:} A clear, descriptive flourish allows the player to re-roll one die showing 1.
  \item \textbf{Intricate Action:} A richly described, multi-sensory action allows the player to re-roll \emph{all} dice showing 1, and add one positive narrative flourish to the scene if they succeed.
\end{itemize}

\paragraph{Rule — Re-rolling 1s and SB}
Re-rolling 1s does not remove the Story Beats already generated by those dice. If any re-rolled dice show 1 again, they generate additional SB as normal.\\
Let $C_0$ = initial 1s, $C_r$ = 1s on re-rolls $\Rightarrow$ \textbf{Total SB} $= C_0 + C_r$.\\
\emph{Example:} You roll 7d10: \{9, 8, 5, 4, 3, \textbf{1}, \textbf{1}\} $\Rightarrow C_0=2$. You re-roll both 1s (Intricate): \{6, 2\} $\Rightarrow C_r=0$. Final: successes = 3, SB = 2 (the initial SB remain).

\subsubsection{Story Beats}
Story Beats (SB) are the engine of drama. They are not simple penalties, but narrative levers. The GM spends SB to introduce setbacks appropriate to the context:
\begin{itemize}
  \item \textbf{Escalation} — drawing more enemies, raising the stakes.
  \item \textbf{Exhaustion} — draining time, resources, or positioning.
  \item \textbf{Exposure} — revealing hidden actions, alerting foes.
  \item \textbf{Collateral} — harm or danger spilling over onto allies, innocents, or surroundings.
\end{itemize}

\subsubsection{Design Intent}
This mechanic ensures that every roll changes the story. Success without risk is rare, and even failure opens new narrative avenues.

\subsubsection{GM Quick Reference: Adjudicating Skill Checks}

\paragraph{Difficulty Ladder (set before the roll)}
\begin{center}
\begin{tabular}{@{}lll@{}}
\toprule
\textbf{DV} & \textbf{Name} & \textbf{When to Use} \\
\midrule
2 & Routine   & Clear intent, modest stakes, controlled environment. \\
3 & Pressured & Time pressure, mild resistance, partial info. \\
4 & Hard      & Hostile conditions, active opposition, precise timing. \\
5+& Extreme   & Multiple constraints, high precision, dramatic failure. \\
\bottomrule
\end{tabular}
\end{center}

\paragraph{Outcome Matrix (after the roll)}
Let $S$ be successes ($\geq 6$) and $C$ be SB (number of 1s rolled).
\begin{center}
\begin{tabular}{@{}ll@{}}
\toprule
\textbf{Case} & \textbf{Guidance} \\
\midrule
$S \geq DV$ and $C=0$ & \textbf{Clean Success}: Deliver the intent crisply. \\
$S \geq DV$ and $C>0$ & \textbf{Success \& Cost}: Grant the intent; spend/bank SB for complications. \\
$0 < S < DV$          & \textbf{Partial}: Progress with a fork. Award 1 boon \\
$S = 0$               & \textbf{Miss}: No progress. Cash/bank SB. Award 2 boons \\
\bottomrule
\end{tabular}
\end{center}

\paragraph{SB Spend Menu (guidance)}
\begin{itemize}
  \item \textbf{1 SB}: Minor pressure: noise, trace, +1 Supply segment.
  \item \textbf{2 SB}: Moderate setback: alarm raised, lose position/cover, lesser foe or lock.
  \item \textbf{3 SB}: Serious trouble: reinforcements, key gear breaks, rail tick.
  \item \textbf{4+ SB}: Major turn: trap springs, authority arrives, scene shifts.
\end{itemize}

\paragraph{Assistance, Boons, \& Description}
\begin{itemize}
  \item \textbf{Assists:} One helper per action; total Assist dice across sources are capped at +3 (unless a specific Talent states otherwise).
  \item \textbf{Boons:} A player may re-roll one die after seeing the pool. Once per session, in downtime, you may convert 2 Boons $\rightarrow$ 1 XP (max 2 XP via conversion per session). Hold cap: 5. Trim to 2 at scene end.
  \item \textbf{Description Ladder:} Basic (roll as-is), Detailed (re-roll one 1), Intricate (re-roll all 1s and add one flourish on success).
\end{itemize}

\subsection{Boon Sharing}

Players may gift \textbf{1 Boon per scene} to an ally with a brief narrative justification.  
\begin{itemize}
  \item \textbf{Bonded Allies:} If characters share a bond, they may gift \textbf{2 Boons per scene}.  
  \item \textbf{Assistance:} Boons may be spent to enhance an ally’s roll (counts as assistance).  
  \item \textbf{Campaign Events:} Major victories or setbacks may generate shared Boons for the party.  
\end{itemize}

\textbf{Table Use:} Require a short story beat for each gift. Normal Boon limits apply. Track shared Boons openly.  
\textbf{GM Notes:} Reward generosity with extra opportunities, introduce occasional complications from dependence, and balance group vs.\ individual needs.

\subsection{Time Guidance Framework}

\subsubsection{Narrative Time Scales}
Time in Fate's Edge is measured by story weight, not by clocks:
\begin{itemize}
  \item \textbf{A Moment} — A heartbeat, a glance, a single strike or word.
  \item \textbf{Some Time} — A few minutes: a skirmish, a careful lockpick, a short negotiation.
  \item \textbf{Significant Time} — Hours: travel between locations, work a ritual, endure a siege.
  \item \textbf{Days} — Large-scale endeavors: marches across countryside, training a cadre, recovery.
\end{itemize}

\subsubsection{Game Structure Definitions}
\begin{description}[leftmargin=1.5em, labelindent=0em]
  \item[Scene] The basic unit of narrative play (Some Time to Significant Time); resolves a specific question or conflict.
  \item[Player Turn (Beat)] Declare action $\rightarrow$ GM sets position $\rightarrow$ roll $\rightarrow$ resolve outcome $\rightarrow$ manage consequences.
  \item[Round] Simultaneous or near-simultaneous actions within a scene (primarily for combat), representing a few seconds.
  \item[Session] One complete game session (typically 3–6 hours), containing 2–4 major scenes and resolving significant narrative progress.
  \item[Campaign] Entire story arc (6–20+ sessions) with major character development and lasting consequences.
\end{description}

\subsubsection{Magic and Ritual Time}
\begin{itemize}
  \item \textbf{Standard Casting:} Channel and Weave phases each take 1 Player Turn; resolves within a single scene.
  \item \textbf{Ritual Casting (Optional Rule):} Channel and Weave phases each require 1 Scene (Significant Time).
  \item \textbf{Rites Invocation:} Invoke takes 1 Player Turn; Weave takes 1 Player Turn. High-Power rites may require extended time by fiction.
\end{itemize}

\paragraph{Extended Rituals}
Attach long rituals to clocks:
\begin{itemize}
  \item 4-segment clock: Significant Time (hours)
  \item 6-segment clock: Extended Time (days)
  \item 8+ segment clock: Campaign Time (weeks/months)
\end{itemize}
Advance the clock through player actions, scenes, or set intervals.

\subsection{Worked Micro-Examples}
\begin{itemize}
  \item \textbf{Lockpick Under Watch (DV 2):} Roll 6 dice: 10, 8, 5, 4, 1, 1 $\Rightarrow S=2, C=2$. \emph{Success \& Cost.} Door opens; GM spends 1 SB for a squeal (patrol starts moving) and banks 1 SB to bring that patrol around on the next beat.
  \item \textbf{Charm the Captain (DV 2):} Roll 5 dice: 7, 6, 6, 2, 1 $\Rightarrow S=3, C=1$. \emph{Success \& Cost.} Passage granted; GM spends 1 SB: ``He expects a favor on the return leg—he'll collect.''
  \item \textbf{Traverse the Pass (DV 3):} Group pools to net 3 successes but produces $C=3$. \emph{Success \& Cost.} GM spends 2 SB to add Fatigue 1 to all from cold and exposure, banks 1 SB to crack a wagon axle next scene.
\end{itemize}

\paragraph{Fail Forward: Every Roll Matters}
When you \textbf{MISS} on a \emph{meaningful action}, you gain 2 \textbf{Boons}. When you have a \textbf{PARTIAL}, you gain 1 \textbf{Boon}. Boons can be spent immediately for re-rolls, Asset activations, Rites, and other abilities. You can hold up to 5 Boons (trim to 2 at scene end).\\
A miss only awards Boons if all three are true:
\begin{enumerate}
  \item Procedure followed: intent and approach declared; DV set; roll resolved.
  \item Stakes stated: what changes on success; what bites on failure.
  \item Consequence lands now: the GM spends or banks SB, applies a condition, or advances a thread.
\end{enumerate}
Typically, failures reward boons. Rehearsal/null-risk probes and repeated identical attempts in the same scene do not award Boons. Rule of thumb, if it feels like an obvious fishing attempt, do not award a boon.

\subsection{Session Loop}

\textbf{Off-Screen (Downtime).} Clear/mark clocks, pay Upkeep, manage Obligation, craft, gather info, frame intents.

\textbf{On-Screen (Adventure).} Play scenes, make moves, trigger Rites/Casting, advance fronts.

\textbf{Wrap-Up.} Award XP, mark Story Beats (SB), resolve Harm/Fatigue conversion, advance faction clocks, note Patron Largess.

\textbf{Off-Screen Hooks.} Record next Downtime intents (projects, service to Patrons, upkeep needs) and any cliffhangers.



\section{Core Mechanic}

\subsection{The Art of Consequence}

\subsubsection{Procedure}
All significant actions follow a three-step process:
\begin{enumerate}
  \item \textbf{Approach:} The player describes both what their character wants and how they attempt it.
  \item \textbf{Execution:} Build a dice pool equal to \emph{Attribute + Skill} and roll that many d10s. Each die of \textbf{6+} counts as a success. Each \textbf{1} rolled generates a \textbf{Story Beat (SB)}.
  \item \textbf{Outcome:} The GM interprets total successes against the difficulty (DV) of the task. Story Beats are then spent to weave narrative setbacks.
\end{enumerate}

\subsubsection{The Description Ladder}
\begin{itemize}
  \item \textbf{Basic Action:} Roll the pool as-is. All 1s remain as Story Beats.
  \item \textbf{Detailed Action:} A clear, descriptive flourish allows the player to re-roll one die showing 1.
  \item \textbf{Intricate Action:} A richly described, multi-sensory action allows the player to re-roll \emph{all} dice showing 1, and add one positive narrative flourish to the scene if they succeed.
\end{itemize}

\paragraph{Rule — Re-rolling 1s and SB}
Re-rolling 1s does not remove the Story Beats already generated by those dice. If any re-rolled dice show 1 again, they generate additional SB as normal.\\
Let $C_0$ = initial 1s, $C_r$ = 1s on re-rolls $\Rightarrow$ \textbf{Total SB} $= C_0 + C_r$.\\
\emph{Example:} You roll 7d10: \{9, 8, 5, 4, 3, \textbf{1}, \textbf{1}\} $\Rightarrow C_0=2$. You re-roll both 1s (Intricate): \{6, 2\} $\Rightarrow C_r=0$. Final: successes = 3, SB = 2 (the initial SB remain).

\subsubsection{Story Beats}
Story Beats (SB) are the engine of drama. They are not simple penalties, but narrative levers. The GM spends SB to introduce setbacks appropriate to the context:
\begin{itemize}
  \item \textbf{Escalation} — drawing more enemies, raising the stakes.
  \item \textbf{Exhaustion} — draining time, resources, or positioning.
  \item \textbf{Exposure} — revealing hidden actions, alerting foes.
  \item \textbf{Collateral} — harm or danger spilling over onto allies, innocents, or surroundings.
\end{itemize}

\subsubsection{Design Intent}
This mechanic ensures that every roll changes the story. Success without risk is rare, and even failure opens new narrative avenues.

\subsubsection{GM Quick Reference: Adjudicating Skill Checks}

\paragraph{Difficulty Ladder (set before the roll)}
\begin{center}
\begin{tabular}{@{}lll@{}}
\toprule
\textbf{DV} & \textbf{Name} & \textbf{When to Use} \\
\midrule
2 & Routine   & Clear intent, modest stakes, controlled environment. \\
3 & Pressured & Time pressure, mild resistance, partial info. \\
4 & Hard      & Hostile conditions, active opposition, precise timing. \\
5+& Extreme   & Multiple constraints, high precision, dramatic failure. \\
\bottomrule
\end{tabular}
\end{center}

\paragraph{Outcome Matrix (after the roll)}
Let $S$ be successes ($\geq 6$) and $C$ be SB (number of 1s rolled).
\begin{center}
\begin{tabular}{@{}ll@{}}
\toprule
\textbf{Case} & \textbf{Guidance} \\
\midrule
$S \geq DV$ and $C=0$ & \textbf{Clean Success}: Deliver the intent crisply. \\
$S \geq DV$ and $C>0$ & \textbf{Success \& Cost}: Grant the intent; spend/bank SB for complications. \\
$0 < S < DV$          & \textbf{Partial}: Progress with a fork. Award 1 boon \\
$S = 0$               & \textbf{Miss}: No progress. Cash/bank SB. Award 2 boons \\
\bottomrule
\end{tabular}
\end{center}

\paragraph{SB Spend Menu (guidance)}
\begin{itemize}
  \item \textbf{1 SB}: Minor pressure: noise, trace, +1 Supply segment.
  \item \textbf{2 SB}: Moderate setback: alarm raised, lose position/cover, lesser foe or lock.
  \item \textbf{3 SB}: Serious trouble: reinforcements, key gear breaks, rail tick.
  \item \textbf{4+ SB}: Major turn: trap springs, authority arrives, scene shifts.
\end{itemize}

\paragraph{Assistance, Boons, \& Description}
\begin{itemize}
  \item \textbf{Assists:} One helper per action; total Assist dice across sources are capped at +3 (unless a specific Talent states otherwise).
  \item \textbf{Boons:} A player may re-roll one die after seeing the pool. Once per session, in downtime, you may convert 2 Boons $\rightarrow$ 1 XP (max 2 XP via conversion per session). Hold cap: 5. Trim to 2 at scene end.
  \item \textbf{Description Ladder:} Basic (roll as-is), Detailed (re-roll one 1), Intricate (re-roll all 1s and add one flourish on success).
\end{itemize}

\subsection{Boon Sharing}

Players may gift \textbf{1 Boon per scene} to an ally with a brief narrative justification.  
\begin{itemize}
  \item \textbf{Bonded Allies:} If characters share a bond, they may gift \textbf{2 Boons per scene}.  
  \item \textbf{Assistance:} Boons may be spent to enhance an ally’s roll (counts as assistance).  
  \item \textbf{Campaign Events:} Major victories or setbacks may generate shared Boons for the party.  
\end{itemize}

\textbf{Table Use:} Require a short story beat for each gift. Normal Boon limits apply. Track shared Boons openly.  
\textbf{GM Notes:} Reward generosity with extra opportunities, introduce occasional complications from dependence, and balance group vs.\ individual needs.

\subsection{Time Guidance Framework}

\subsubsection{Narrative Time Scales}
Time in Fate's Edge is measured by story weight, not by clocks:
\begin{itemize}
  \item \textbf{A Moment} — A heartbeat, a glance, a single strike or word.
  \item \textbf{Some Time} — A few minutes: a skirmish, a careful lockpick, a short negotiation.
  \item \textbf{Significant Time} — Hours: travel between locations, work a ritual, endure a siege.
  \item \textbf{Days} — Large-scale endeavors: marches across countryside, training a cadre, recovery.
\end{itemize}

\subsubsection{Game Structure Definitions}
\begin{description}[leftmargin=1.5em, labelindent=0em]
  \item[Scene] The basic unit of narrative play (Some Time to Significant Time); resolves a specific question or conflict.
  \item[Player Turn (Beat)] Declare action $\rightarrow$ GM sets position $\rightarrow$ roll $\rightarrow$ resolve outcome $\rightarrow$ manage consequences.
  \item[Round] Simultaneous or near-simultaneous actions within a scene (primarily for combat), representing a few seconds.
  \item[Session] One complete game session (typically 3–6 hours), containing 2–4 major scenes and resolving significant narrative progress.
  \item[Campaign] Entire story arc (6–20+ sessions) with major character development and lasting consequences.
\end{description}

\subsubsection{Magic and Ritual Time}
\begin{itemize}
  \item \textbf{Standard Casting:} Channel and Weave phases each take 1 Player Turn; resolves within a single scene.
  \item \textbf{Ritual Casting (Optional Rule):} Channel and Weave phases each require 1 Scene (Significant Time).
  \item \textbf{Rites Invocation:} Invoke takes 1 Player Turn; Weave takes 1 Player Turn. High-Power rites may require extended time by fiction.
\end{itemize}

\paragraph{Extended Rituals}
Attach long rituals to clocks:
\begin{itemize}
  \item 4-segment clock: Significant Time (hours)
  \item 6-segment clock: Extended Time (days)
  \item 8+ segment clock: Campaign Time (weeks/months)
\end{itemize}
Advance the clock through player actions, scenes, or set intervals.

\subsection{Worked Micro-Examples}
\begin{itemize}
  \item \textbf{Lockpick Under Watch (DV 2):} Roll 6 dice: 10, 8, 5, 4, 1, 1 $\Rightarrow S=2, C=2$. \emph{Success \& Cost.} Door opens; GM spends 1 SB for a squeal (patrol starts moving) and banks 1 SB to bring that patrol around on the next beat.
  \item \textbf{Charm the Captain (DV 2):} Roll 5 dice: 7, 6, 6, 2, 1 $\Rightarrow S=3, C=1$. \emph{Success \& Cost.} Passage granted; GM spends 1 SB: ``He expects a favor on the return leg—he'll collect.''
  \item \textbf{Traverse the Pass (DV 3):} Group pools to net 3 successes but produces $C=3$. \emph{Success \& Cost.} GM spends 2 SB to add Fatigue 1 to all from cold and exposure, banks 1 SB to crack a wagon axle next scene.
\end{itemize}

\paragraph{Fail Forward: Every Roll Matters}
When you \textbf{MISS} on a \emph{meaningful action}, you gain 2 \textbf{Boons}. When you have a \textbf{PARTIAL}, you gain 1 \textbf{Boon}. Boons can be spent immediately for re-rolls, Asset activations, Rites, and other abilities. You can hold up to 5 Boons (trim to 2 at scene end).\\
A miss only awards Boons if all three are true:
\begin{enumerate}
  \item Procedure followed: intent and approach declared; DV set; roll resolved.
  \item Stakes stated: what changes on success; what bites on failure.
  \item Consequence lands now: the GM spends or banks SB, applies a condition, or advances a thread.
\end{enumerate}
Typically, failures reward boons. Rehearsal/null-risk probes and repeated identical attempts in the same scene do not award Boons. Rule of thumb, if it feels like an obvious fishing attempt, do not award a boon.

\subsection{Session Loop}

\textbf{Off-Screen (Downtime).} Clear/mark clocks, pay Upkeep, manage Obligation, craft, gather info, frame intents.

\textbf{On-Screen (Adventure).} Play scenes, make moves, trigger Rites/Casting, advance fronts.

\textbf{Wrap-Up.} Award XP, mark Story Beats (SB), resolve Harm/Fatigue conversion, advance faction clocks, note Patron Largess.

\textbf{Off-Screen Hooks.} Record next Downtime intents (projects, service to Patrons, upkeep needs) and any cliffhangers.



\section{Core Mechanic}

\subsection{The Art of Consequence}

\subsubsection{Procedure}
All significant actions follow a three-step process:
\begin{enumerate}
  \item \textbf{Approach:} The player describes both what their character wants and how they attempt it.
  \item \textbf{Execution:} Build a dice pool equal to \emph{Attribute + Skill} and roll that many d10s. Each die of \textbf{6+} counts as a success. Each \textbf{1} rolled generates a \textbf{Story Beat (SB)}.
  \item \textbf{Outcome:} The GM interprets total successes against the difficulty (DV) of the task. Story Beats are then spent to weave narrative setbacks.
\end{enumerate}

\subsubsection{The Description Ladder}
\begin{itemize}
  \item \textbf{Basic Action:} Roll the pool as-is. All 1s remain as Story Beats.
  \item \textbf{Detailed Action:} A clear, descriptive flourish allows the player to re-roll one die showing 1.
  \item \textbf{Intricate Action:} A richly described, multi-sensory action allows the player to re-roll \emph{all} dice showing 1, and add one positive narrative flourish to the scene if they succeed.
\end{itemize}

\paragraph{Rule — Re-rolling 1s and SB}
Re-rolling 1s does not remove the Story Beats already generated by those dice. If any re-rolled dice show 1 again, they generate additional SB as normal.\\
Let $C_0$ = initial 1s, $C_r$ = 1s on re-rolls $\Rightarrow$ \textbf{Total SB} $= C_0 + C_r$.\\
\emph{Example:} You roll 7d10: \{9, 8, 5, 4, 3, \textbf{1}, \textbf{1}\} $\Rightarrow C_0=2$. You re-roll both 1s (Intricate): \{6, 2\} $\Rightarrow C_r=0$. Final: successes = 3, SB = 2 (the initial SB remain).

\subsubsection{Story Beats}
Story Beats (SB) are the engine of drama. They are not simple penalties, but narrative levers. The GM spends SB to introduce setbacks appropriate to the context:
\begin{itemize}
  \item \textbf{Escalation} — drawing more enemies, raising the stakes.
  \item \textbf{Exhaustion} — draining time, resources, or positioning.
  \item \textbf{Exposure} — revealing hidden actions, alerting foes.
  \item \textbf{Collateral} — harm or danger spilling over onto allies, innocents, or surroundings.
\end{itemize}

\subsubsection{Design Intent}
This mechanic ensures that every roll changes the story. Success without risk is rare, and even failure opens new narrative avenues.

\subsubsection{GM Quick Reference: Adjudicating Skill Checks}

\paragraph{Difficulty Ladder (set before the roll)}
\begin{center}
\begin{tabular}{@{}lll@{}}
\toprule
\textbf{DV} & \textbf{Name} & \textbf{When to Use} \\
\midrule
2 & Routine   & Clear intent, modest stakes, controlled environment. \\
3 & Pressured & Time pressure, mild resistance, partial info. \\
4 & Hard      & Hostile conditions, active opposition, precise timing. \\
5+& Extreme   & Multiple constraints, high precision, dramatic failure. \\
\bottomrule
\end{tabular}
\end{center}

\paragraph{Outcome Matrix (after the roll)}
Let $S$ be successes ($\geq 6$) and $C$ be SB (number of 1s rolled).
\begin{center}
\begin{tabular}{@{}ll@{}}
\toprule
\textbf{Case} & \textbf{Guidance} \\
\midrule
$S \geq DV$ and $C=0$ & \textbf{Clean Success}: Deliver the intent crisply. \\
$S \geq DV$ and $C>0$ & \textbf{Success \& Cost}: Grant the intent; spend/bank SB for complications. \\
$0 < S < DV$          & \textbf{Partial}: Progress with a fork. Award 1 boon \\
$S = 0$               & \textbf{Miss}: No progress. Cash/bank SB. Award 2 boons \\
\bottomrule
\end{tabular}
\end{center}

\paragraph{SB Spend Menu (guidance)}
\begin{itemize}
  \item \textbf{1 SB}: Minor pressure: noise, trace, +1 Supply segment.
  \item \textbf{2 SB}: Moderate setback: alarm raised, lose position/cover, lesser foe or lock.
  \item \textbf{3 SB}: Serious trouble: reinforcements, key gear breaks, rail tick.
  \item \textbf{4+ SB}: Major turn: trap springs, authority arrives, scene shifts.
\end{itemize}

\paragraph{Assistance, Boons, \& Description}
\begin{itemize}
  \item \textbf{Assists:} One helper per action; total Assist dice across sources are capped at +3 (unless a specific Talent states otherwise).
  \item \textbf{Boons:} A player may re-roll one die after seeing the pool. Once per session, in downtime, you may convert 2 Boons $\rightarrow$ 1 XP (max 2 XP via conversion per session). Hold cap: 5. Trim to 2 at scene end.
  \item \textbf{Description Ladder:} Basic (roll as-is), Detailed (re-roll one 1), Intricate (re-roll all 1s and add one flourish on success).
\end{itemize}

\subsection{Boon Sharing}

Players may gift \textbf{1 Boon per scene} to an ally with a brief narrative justification.  
\begin{itemize}
  \item \textbf{Bonded Allies:} If characters share a bond, they may gift \textbf{2 Boons per scene}.  
  \item \textbf{Assistance:} Boons may be spent to enhance an ally’s roll (counts as assistance).  
  \item \textbf{Campaign Events:} Major victories or setbacks may generate shared Boons for the party.  
\end{itemize}

\textbf{Table Use:} Require a short story beat for each gift. Normal Boon limits apply. Track shared Boons openly.  
\textbf{GM Notes:} Reward generosity with extra opportunities, introduce occasional complications from dependence, and balance group vs.\ individual needs.

\subsection{Time Guidance Framework}

\subsubsection{Narrative Time Scales}
Time in Fate's Edge is measured by story weight, not by clocks:
\begin{itemize}
  \item \textbf{A Moment} — A heartbeat, a glance, a single strike or word.
  \item \textbf{Some Time} — A few minutes: a skirmish, a careful lockpick, a short negotiation.
  \item \textbf{Significant Time} — Hours: travel between locations, work a ritual, endure a siege.
  \item \textbf{Days} — Large-scale endeavors: marches across countryside, training a cadre, recovery.
\end{itemize}

\subsubsection{Game Structure Definitions}
\begin{description}[leftmargin=1.5em, labelindent=0em]
  \item[Scene] The basic unit of narrative play (Some Time to Significant Time); resolves a specific question or conflict.
  \item[Player Turn (Beat)] Declare action $\rightarrow$ GM sets position $\rightarrow$ roll $\rightarrow$ resolve outcome $\rightarrow$ manage consequences.
  \item[Round] Simultaneous or near-simultaneous actions within a scene (primarily for combat), representing a few seconds.
  \item[Session] One complete game session (typically 3–6 hours), containing 2–4 major scenes and resolving significant narrative progress.
  \item[Campaign] Entire story arc (6–20+ sessions) with major character development and lasting consequences.
\end{description}

\subsubsection{Magic and Ritual Time}
\begin{itemize}
  \item \textbf{Standard Casting:} Channel and Weave phases each take 1 Player Turn; resolves within a single scene.
  \item \textbf{Ritual Casting (Optional Rule):} Channel and Weave phases each require 1 Scene (Significant Time).
  \item \textbf{Rites Invocation:} Invoke takes 1 Player Turn; Weave takes 1 Player Turn. High-Power rites may require extended time by fiction.
\end{itemize}

\paragraph{Extended Rituals}
Attach long rituals to clocks:
\begin{itemize}
  \item 4-segment clock: Significant Time (hours)
  \item 6-segment clock: Extended Time (days)
  \item 8+ segment clock: Campaign Time (weeks/months)
\end{itemize}
Advance the clock through player actions, scenes, or set intervals.

\subsection{Worked Micro-Examples}
\begin{itemize}
  \item \textbf{Lockpick Under Watch (DV 2):} Roll 6 dice: 10, 8, 5, 4, 1, 1 $\Rightarrow S=2, C=2$. \emph{Success \& Cost.} Door opens; GM spends 1 SB for a squeal (patrol starts moving) and banks 1 SB to bring that patrol around on the next beat.
  \item \textbf{Charm the Captain (DV 2):} Roll 5 dice: 7, 6, 6, 2, 1 $\Rightarrow S=3, C=1$. \emph{Success \& Cost.} Passage granted; GM spends 1 SB: ``He expects a favor on the return leg—he'll collect.''
  \item \textbf{Traverse the Pass (DV 3):} Group pools to net 3 successes but produces $C=3$. \emph{Success \& Cost.} GM spends 2 SB to add Fatigue 1 to all from cold and exposure, banks 1 SB to crack a wagon axle next scene.
\end{itemize}

\paragraph{Fail Forward: Every Roll Matters}
When you \textbf{MISS} on a \emph{meaningful action}, you gain 2 \textbf{Boons}. When you have a \textbf{PARTIAL}, you gain 1 \textbf{Boon}. Boons can be spent immediately for re-rolls, Asset activations, Rites, and other abilities. You can hold up to 5 Boons (trim to 2 at scene end).\\
A miss only awards Boons if all three are true:
\begin{enumerate}
  \item Procedure followed: intent and approach declared; DV set; roll resolved.
  \item Stakes stated: what changes on success; what bites on failure.
  \item Consequence lands now: the GM spends or banks SB, applies a condition, or advances a thread.
\end{enumerate}
Typically, failures reward boons. Rehearsal/null-risk probes and repeated identical attempts in the same scene do not award Boons. Rule of thumb, if it feels like an obvious fishing attempt, do not award a boon.

\subsection{Session Loop}

\textbf{Off-Screen (Downtime).} Clear/mark clocks, pay Upkeep, manage Obligation, craft, gather info, frame intents.

\textbf{On-Screen (Adventure).} Play scenes, make moves, trigger Rites/Casting, advance fronts.

\textbf{Wrap-Up.} Award XP, mark Story Beats (SB), resolve Harm/Fatigue conversion, advance faction clocks, note Patron Largess.

\textbf{Off-Screen Hooks.} Record next Downtime intents (projects, service to Patrons, upkeep needs) and any cliffhangers.



\section{Core Mechanic}

\subsection{The Art of Consequence}

\subsubsection{Procedure}
All significant actions follow a three-step process:
\begin{enumerate}
  \item \textbf{Approach:} The player describes both what their character wants and how they attempt it.
  \item \textbf{Execution:} Build a dice pool equal to \emph{Attribute + Skill} and roll that many d10s. Each die of \textbf{6+} counts as a success. Each \textbf{1} rolled generates a \textbf{Story Beat (SB)}.
  \item \textbf{Outcome:} The GM interprets total successes against the difficulty (DV) of the task. Story Beats are then spent to weave narrative setbacks.
\end{enumerate}

\subsubsection{The Description Ladder}
\begin{itemize}
  \item \textbf{Basic Action:} Roll the pool as-is. All 1s remain as Story Beats.
  \item \textbf{Detailed Action:} A clear, descriptive flourish allows the player to re-roll one die showing 1.
  \item \textbf{Intricate Action:} A richly described, multi-sensory action allows the player to re-roll \emph{all} dice showing 1, and add one positive narrative flourish to the scene if they succeed.
\end{itemize}

\paragraph{Rule — Re-rolling 1s and SB}
Re-rolling 1s does not remove the Story Beats already generated by those dice. If any re-rolled dice show 1 again, they generate additional SB as normal.\\
Let $C_0$ = initial 1s, $C_r$ = 1s on re-rolls $\Rightarrow$ \textbf{Total SB} $= C_0 + C_r$.\\
\emph{Example:} You roll 7d10: \{9, 8, 5, 4, 3, \textbf{1}, \textbf{1}\} $\Rightarrow C_0=2$. You re-roll both 1s (Intricate): \{6, 2\} $\Rightarrow C_r=0$. Final: successes = 3, SB = 2 (the initial SB remain).

\subsubsection{Story Beats}
Story Beats (SB) are the engine of drama. They are not simple penalties, but narrative levers. The GM spends SB to introduce setbacks appropriate to the context:
\begin{itemize}
  \item \textbf{Escalation} — drawing more enemies, raising the stakes.
  \item \textbf{Exhaustion} — draining time, resources, or positioning.
  \item \textbf{Exposure} — revealing hidden actions, alerting foes.
  \item \textbf{Collateral} — harm or danger spilling over onto allies, innocents, or surroundings.
\end{itemize}

\subsubsection{Design Intent}
This mechanic ensures that every roll changes the story. Success without risk is rare, and even failure opens new narrative avenues.

\subsubsection{GM Quick Reference: Adjudicating Skill Checks}

\paragraph{Difficulty Ladder (set before the roll)}
\begin{center}
\begin{tabular}{@{}lll@{}}
\toprule
\textbf{DV} & \textbf{Name} & \textbf{When to Use} \\
\midrule
2 & Routine   & Clear intent, modest stakes, controlled environment. \\
3 & Pressured & Time pressure, mild resistance, partial info. \\
4 & Hard      & Hostile conditions, active opposition, precise timing. \\
5+& Extreme   & Multiple constraints, high precision, dramatic failure. \\
\bottomrule
\end{tabular}
\end{center}

\paragraph{Outcome Matrix (after the roll)}
Let $S$ be successes ($\geq 6$) and $C$ be SB (number of 1s rolled).
\begin{center}
\begin{tabular}{@{}ll@{}}
\toprule
\textbf{Case} & \textbf{Guidance} \\
\midrule
$S \geq DV$ and $C=0$ & \textbf{Clean Success}: Deliver the intent crisply. \\
$S \geq DV$ and $C>0$ & \textbf{Success \& Cost}: Grant the intent; spend/bank SB for complications. \\
$0 < S < DV$          & \textbf{Partial}: Progress with a fork. Award 1 boon \\
$S = 0$               & \textbf{Miss}: No progress. Cash/bank SB. Award 2 boons \\
\bottomrule
\end{tabular}
\end{center}

\paragraph{SB Spend Menu (guidance)}
\begin{itemize}
  \item \textbf{1 SB}: Minor pressure: noise, trace, +1 Supply segment.
  \item \textbf{2 SB}: Moderate setback: alarm raised, lose position/cover, lesser foe or lock.
  \item \textbf{3 SB}: Serious trouble: reinforcements, key gear breaks, rail tick.
  \item \textbf{4+ SB}: Major turn: trap springs, authority arrives, scene shifts.
\end{itemize}

\paragraph{Assistance, Boons, \& Description}
\begin{itemize}
  \item \textbf{Assists:} One helper per action; total Assist dice across sources are capped at +3 (unless a specific Talent states otherwise).
  \item \textbf{Boons:} A player may re-roll one die after seeing the pool. Once per session, in downtime, you may convert 2 Boons $\rightarrow$ 1 XP (max 2 XP via conversion per session). Hold cap: 5. Trim to 2 at scene end.
  \item \textbf{Description Ladder:} Basic (roll as-is), Detailed (re-roll one 1), Intricate (re-roll all 1s and add one flourish on success).
\end{itemize}

\subsection{Boon Sharing}

Players may gift \textbf{1 Boon per scene} to an ally with a brief narrative justification.  
\begin{itemize}
  \item \textbf{Bonded Allies:} If characters share a bond, they may gift \textbf{2 Boons per scene}.  
  \item \textbf{Assistance:} Boons may be spent to enhance an ally’s roll (counts as assistance).  
  \item \textbf{Campaign Events:} Major victories or setbacks may generate shared Boons for the party.  
\end{itemize}

\textbf{Table Use:} Require a short story beat for each gift. Normal Boon limits apply. Track shared Boons openly.  
\textbf{GM Notes:} Reward generosity with extra opportunities, introduce occasional complications from dependence, and balance group vs.\ individual needs.

\subsection{Time Guidance Framework}

\subsubsection{Narrative Time Scales}
Time in Fate's Edge is measured by story weight, not by clocks:
\begin{itemize}
  \item \textbf{A Moment} — A heartbeat, a glance, a single strike or word.
  \item \textbf{Some Time} — A few minutes: a skirmish, a careful lockpick, a short negotiation.
  \item \textbf{Significant Time} — Hours: travel between locations, work a ritual, endure a siege.
  \item \textbf{Days} — Large-scale endeavors: marches across countryside, training a cadre, recovery.
\end{itemize}

\subsubsection{Game Structure Definitions}
\begin{description}[leftmargin=1.5em, labelindent=0em]
  \item[Scene] The basic unit of narrative play (Some Time to Significant Time); resolves a specific question or conflict.
  \item[Player Turn (Beat)] Declare action $\rightarrow$ GM sets position $\rightarrow$ roll $\rightarrow$ resolve outcome $\rightarrow$ manage consequences.
  \item[Round] Simultaneous or near-simultaneous actions within a scene (primarily for combat), representing a few seconds.
  \item[Session] One complete game session (typically 3–6 hours), containing 2–4 major scenes and resolving significant narrative progress.
  \item[Campaign] Entire story arc (6–20+ sessions) with major character development and lasting consequences.
\end{description}

\subsubsection{Magic and Ritual Time}
\begin{itemize}
  \item \textbf{Standard Casting:} Channel and Weave phases each take 1 Player Turn; resolves within a single scene.
  \item \textbf{Ritual Casting (Optional Rule):} Channel and Weave phases each require 1 Scene (Significant Time).
  \item \textbf{Rites Invocation:} Invoke takes 1 Player Turn; Weave takes 1 Player Turn. High-Power rites may require extended time by fiction.
\end{itemize}

\paragraph{Extended Rituals}
Attach long rituals to clocks:
\begin{itemize}
  \item 4-segment clock: Significant Time (hours)
  \item 6-segment clock: Extended Time (days)
  \item 8+ segment clock: Campaign Time (weeks/months)
\end{itemize}
Advance the clock through player actions, scenes, or set intervals.

\subsection{Worked Micro-Examples}
\begin{itemize}
  \item \textbf{Lockpick Under Watch (DV 2):} Roll 6 dice: 10, 8, 5, 4, 1, 1 $\Rightarrow S=2, C=2$. \emph{Success \& Cost.} Door opens; GM spends 1 SB for a squeal (patrol starts moving) and banks 1 SB to bring that patrol around on the next beat.
  \item \textbf{Charm the Captain (DV 2):} Roll 5 dice: 7, 6, 6, 2, 1 $\Rightarrow S=3, C=1$. \emph{Success \& Cost.} Passage granted; GM spends 1 SB: ``He expects a favor on the return leg—he'll collect.''
  \item \textbf{Traverse the Pass (DV 3):} Group pools to net 3 successes but produces $C=3$. \emph{Success \& Cost.} GM spends 2 SB to add Fatigue 1 to all from cold and exposure, banks 1 SB to crack a wagon axle next scene.
\end{itemize}

\paragraph{Fail Forward: Every Roll Matters}
When you \textbf{MISS} on a \emph{meaningful action}, you gain 2 \textbf{Boons}. When you have a \textbf{PARTIAL}, you gain 1 \textbf{Boon}. Boons can be spent immediately for re-rolls, Asset activations, Rites, and other abilities. You can hold up to 5 Boons (trim to 2 at scene end).\\
A miss only awards Boons if all three are true:
\begin{enumerate}
  \item Procedure followed: intent and approach declared; DV set; roll resolved.
  \item Stakes stated: what changes on success; what bites on failure.
  \item Consequence lands now: the GM spends or banks SB, applies a condition, or advances a thread.
\end{enumerate}
Typically, failures reward boons. Rehearsal/null-risk probes and repeated identical attempts in the same scene do not award Boons. Rule of thumb, if it feels like an obvious fishing attempt, do not award a boon.

\subsection{Session Loop}

\textbf{Off-Screen (Downtime).} Clear/mark clocks, pay Upkeep, manage Obligation, craft, gather info, frame intents.

\textbf{On-Screen (Adventure).} Play scenes, make moves, trigger Rites/Casting, advance fronts.

\textbf{Wrap-Up.} Award XP, mark Story Beats (SB), resolve Harm/Fatigue conversion, advance faction clocks, note Patron Largess.

\textbf{Off-Screen Hooks.} Record next Downtime intents (projects, service to Patrons, upkeep needs) and any cliffhangers.

