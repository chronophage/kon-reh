
% --- Fate's Edge SRD — Section 2: Core Mechanic ---
% Include this file from your main .tex with: 
% --- Fate's Edge SRD — Section 2: Core Mechanic ---
% Include this file from your main .tex with: 
% --- Fate's Edge SRD — Section 2: Core Mechanic ---
% Include this file from your main .tex with: 
% --- Fate's Edge SRD — Section 2: Core Mechanic ---
% Include this file from your main .tex with: \input{02-core-mechanic.tex}

\section{Core Mechanic}

% =========================
% Turn Economy (Quick Rules)
% =========================

\subsection{The Art of Consequence}
\paragraph{Adjudicating Rolls: The Core Resolution Cycle}\index{roll adjudication}

When a player rolls, they are not simply trying to \emph{beat a number}. They are engaging the world through risk, consequence, and discovery. This section walks through the full cycle.

\paragraph{Step-by-Step Roll Resolution}

\begin{enumerate}
    \item \textbf{Declare Action \& Approach:} Player states intent, Attribute + Skill.
    \item \textbf{Set Difficulty Value (DV):}\index{Difficulty Value (DV)} Based on narrative stakes, not just mechanics.
    \item \textbf{Establish Position:}\index{Position} GM sets whether the roll is \textbf{Dominant}, \textbf{Controlled}, or \textbf{Desperate}.
    \item \textbf{Roll Pool of d10s.}
    \item \textbf{Count:} \textbf{Successes (6+)} and \textbf{Story Beats (1s)}.\index{Story Beats} 
    Each \textbf{10 counts as two successes} but does not auto-succeed if total $< DV$.
    \item \textbf{Check Against DV:} Apply the Outcome Matrix.
    \item \textbf{Spend SB:} GM spends/banks Story Beats or draws from the Deck of Consequences.\index{Deck of Consequences}
\end{enumerate}

\begin{fatebox}[Position Effects]\index{Position}
\begin{tabularx}{\textwidth}{lX}
\toprule
\textbf{Position} & \textbf{Effect} \\
\midrule
Dominant & May re-roll one \textbf{failure} (die $<6$). \\
Controlled & Default state; no re-rolls. \\
Desperate & Must re-roll one \textbf{success} (die $6+$), keeping the second result. \\
\bottomrule
\end{tabularx}
\end{fatebox}

\begin{fatebox}[Difficulty Ladder]\index{Difficulty Ladder}
  \begin{tabularx}{\textwidth}{lX}
  \toprule
  \textbf{DV} & \textbf{Typical Case} \\
  \midrule
  3 & Routine: clear intent, modest stakes, stable setting \\
  4 & Pressured: time limits, mild resistance, incomplete information \\
  5 & Hard: hostile conditions, active opposition, precision required \\
  6+ & Extreme: stacked constraints, dangerous failure, high drama \\
  \bottomrule
  \end{tabularx}
  \end{fatebox}

\textit{A DV should measure narrative weight as much as difficulty. Scaling a wall is routine. Scaling it while lantern-wardens pursue is pressured—or worse.}

\begin{tcolorbox}[title=\textbf{Difficulty Values (DV) by Tier},colback=white!97!gray,
colframe=black!80!gray,sharp corners,boxrule=0.4pt]
\textbf{Guideline.}
The base Difficulty Value (DV) for an opposed or environmental test scales with Tier:
\[
\boxed{\text{DV} = \text{Tier} + 2 + \text{Modifiers}}
\]

\textbf{Typical DVs.}
\begin{center}
\renewcommand{\arraystretch}{1.1}
\begin{longtable}{l c c}
\toprule
\textbf{Tier} & \textbf{Base DV} & \textbf{Example Challenge}\\
\midrule
I & 5 & Local threat / novice test\\
II & 6 & Veteran foe or skilled task\\
III & 7 & Elite / magical challenge\\
IV & 8 & Mythic or cosmic threat\\
\bottomrule
\end{longtable}
\end{center}

\textbf{Positional Modifiers.}
\begin{itemize}[leftmargin=*]
  \item \textbf{Desperate:} +2  \textbf{Dominant:} +1  \textbf{Controlled:} +0  \textbf{Desperate:} –1
\end{itemize}
Use \textit{DV = Tier + 2} as the default; adjust for environment, advantage, or narrative pressure.
\end{tcolorbox}

\begin{fatebox}[Outcome Matrix]\index{Outcome Matrix}
\begin{tabularx}{\textwidth}{lX}
\toprule
\textbf{Result} & \textbf{GM Guidance} \\
\midrule
$S \geq DV$, $C = 0$ & Clean Success: Grant intent, no added friction. \\
$S \geq DV$, $C > 0$ & Success \& Cost: Intent achieved; GM spends SB for complications. \\
$0 < S < DV$ & Partial: Progress \emph{proportional} to hits; intent advances but with gaps or risk. Player gains 1 Boon. \\
$S = 0$ & Miss: No progress. GM escalates with SB/Clocks. Player gains 2 Boons. \\
\bottomrule
\end{tabularx}
\end{fatebox}

\paragraph{Fail Forward: Every Roll Matters}\index{Fail Forward}

\textbf{Partials are the most common form of “success.”} They always move the fiction forward in proportion to the progress rolled.

\begin{quote}
\textit{One success on DV 4:} “The lock is stubborn. You think you can get it if you keep trying.”  
\textit{Three successes on DV 4:} “The lock springs open with a loud clank—you’re sure the guards heard.” (Upgrade to Success \& Cost; add 2 SB).  
\end{quote}

\textbf{Misses} fuel escalation but always generate player resources: 2 Boons and a consequence.  

A roll is \emph{meaningful} or \emph{significant} if:  
\begin{enumerate}
  \item The standard procedure is followed (intent + DV + roll).  
  \item Stakes are stated up front (what changes on success, what bites on failure).  
  \item Real consequences occur now (SB spent, condition applied, or thread advanced).  
\end{enumerate}

\paragraph{Important Notes}
\begin{itemize}
  \item Rolling a 1 always creates SB for the GM. Rerolls do not erase SB.  
  \item No Boons for rehearsal, trivial probes, or repeating an identical approach without changing fiction.  
  \item Controlled tests with no bite give positioning/info, not Boons.  
\end{itemize}

\subsubsection{Anti-Fishing Measures}
\begin{itemize}
  \item \textbf{Cap:} At most 2 Boons from failures per character per scene (further misses still make SB).  
  \item \textbf{Repetition Rule:} Same action + same stakes in the same scene can’t grant another Boon.  
\end{itemize}

\paragraph{Example}
Lockpicking under watch (\emph{Desperate}, DV 3).  
\textbf{Miss:} GM spends 2 SB to start \emph{Guards Incoming [6]}. Player earns 2 Boons.  
\textbf{Partial (2 successes):} Door opens halfway; guard footsteps approach. Player earns 1 Boon.  

\subsubsection{Boon Sharing}
Players may gift 1 Boon per scene to an ally with narrative justification.  
\begin{itemize}
  \item \textbf{Bonded Allies:} Up to 2 Boons gifted per scene.  
  \item \textbf{Assistance:} Shared Boons can enhance an ally’s roll.  
  \item \textbf{Campaign Events:} Major milestones may generate party-wide Boons.  
\end{itemize}

\textbf{GM Note:} Encourage gifts with roleplay beats, but balance generosity with potential dependency or group tension.

\paragraph{Rule — Re-rolling 1s and SB}
Re-rolling 1s does not remove the Story Beats already generated by those dice. If any re-rolled dice show 1 again, they generate additional SB as normal.\\
Let $C_0$ = initial 1s, $C_r$ = 1s on re-rolls $\Rightarrow$ \textbf{Total SB} $= C_0 + C_r$.\\
\emph{Example:} You roll 7d10: \{9, 8, 5, 4, 3, \textbf{1}, \textbf{1}\} $\Rightarrow C_0=2$. You re-roll both 1s (Intricate): \{6, 2\} $\Rightarrow C_r=0$. Final: successes = 3, SB = 2 (the initial SB remain).

\subsubsection{Story Beats}
Story Beats (SB) are the engine of drama. They are not simple penalties, but narrative levers. The GM spends SB to introduce setbacks appropriate to the context:
\begin{itemize}
  \item \textbf{Escalation} — drawing more enemies, raising the stakes.
  \item \textbf{Exhaustion} — draining time, resources, or positioning.
  \item \textbf{Exposure} — revealing hidden actions, alerting foes.
  \item \textbf{Collateral} — harm or danger spilling over onto allies, innocents, or surroundings.
\end{itemize}

\subsubsection{Design Intent}
This mechanic ensures that every roll changes the story. Success without risk is rare, and even failure opens new narrative avenues.

\subsubsection{GM Quick Reference: Adjudicating Skill Checks}

\paragraph{Difficulty Ladder (set before the roll)}
\begin{center}
\begin{longtable}{@{}lll@{}}
\toprule
\textbf{DV} & \textbf{Name} & \textbf{When to Use} \\
\midrule
2 & Routine   & Clear intent, modest stakes, controlled environment. \\
3 & Pressured & Time pressure, mild resistance, partial info. \\
4 & Hard      & Hostile conditions, active opposition, precise timing. \\
5+& Extreme   & Multiple constraints, high precision, dramatic failure. \\
\bottomrule
\end{longtable}
\end{center}

\paragraph{Outcome Matrix (after the roll)}
Let $S$ be successes ($\geq 6$) and $C$ be SB (number of 1s rolled).
\begin{center}
\begin{longtable}{@{}ll@{}}
\toprule
\textbf{Case} & \textbf{Guidance} \\
\midrule
$S \geq DV$ and $C=0$ & \textbf{Clean Success}: Deliver the intent crisply. \\
$S \geq DV$ and $C>0$ & \textbf{Success \& Cost}: Grant the intent; spend/bank SB for complications. \\
$0 < S < DV$          & \textbf{Partial}: Progress with a fork. Award 1 boon \\
$S = 0$               & \textbf{Miss}: No progress. Cash/bank SB. Award 2 boons \\
\bottomrule
\end{longtable}
\end{center}

\subsection{Critical Success}

Rolling a \textbf{10} on any die indicates a critical tier of success. Each 10 adds weight to the outcome:

\begin{itemize}
  \item \textbf{One 10:} Strong success with a free boon, improved Position, or other narrative flourish.
  \item \textbf{Two 10s:} Exceptional success; choose two benefits or a single powerful effect.
  \item \textbf{Three 10s:} Legendary success; resolve the conflict dramatically and progress or clear 1 segment on a secondary clock (generally, a clock tied to the scene, not the overarching campaign).
  \item \textbf{Four+ 10s:} Mythic success; progress or clear 1--2 segments from a secondary clock or create a significant story development.
\end{itemize}

\noindent If no 10s are rolled, resolve the action normally by the highest die result.

\noindent \textbf{10s are never re-rolled by Position effects or other mechanics. Critical hit effects always take place if the roll is successful, despite any SB rolled. Critical successes may reduce Backlash/Obligation/Corruption severity by one tier.}

\subsection{Position}
\label{subsec:position}
\index{Position}

Every action in \indexterm{Fate's Edge} takes place from a \textbf{Position} that reflects the character’s advantage or disadvantage in the scene. Position sets the tone for the roll, narratively and mechanically. It comes in three states:

\begin{itemize}
  \item \textbf{Dominant:} You act from a place of control, leverage, or overwhelming advantage.
  \item \textbf{Controlled:} The standard state of play. Outcomes are uncertain but balanced.
  \item \textbf{Desperate:} You act from dire straits, cornered or overmatched, with everything at stake.
\end{itemize}

\paragraph{Re-roll Mechanic.}  
Position modifies the dice pool through simple re-rolls:
\begin{center}
\begin{longtable}{@{}lll@{}}
\toprule
\textbf{Position} & \textbf{Narrative Frame} & \textbf{Mechanical Effect} \\
\midrule
Dominant & You press your advantage & Re-roll one \emph{failure} \\
Controlled    & The balanced norm & No re-rolls \\
Desperate & You act under duress & Re-roll one \emph{success} \\
\bottomrule
\end{longtable}
\end{center}

\subsubsection{Fatigue and Harm}\index{Fatigue}\index{Harm}

\paragraph{Fatigue.}
Fatigue is capped by \textbf{Body}. It reflects cumulative strain from combat, travel, or channeling. Fatigue never reduces dice directly; it shifts \textbf{Position} to riskier states.

\begin{center}
\feTableStart
\begin{tabularx}{\linewidth}{@{}l l Y@{}}
\toprule
\textbf{Fatigue} & \textbf{Position Shift} & \textbf{Narrative Effect} \
\midrule
1 (Winded) & Dominant → Controlled (once/scene) & Breathing heavy, off-balance. \
2 (Strained) & Controlled rolls create +1 SB on 1s & Mistakes creep in. \
3 (Exhausted) & Controlled → Desperate (once/scene) & Desperate exertion. \
4 (Collapse) & DV 3 Body test or Severe Harm & Push past limits. \
\bottomrule
\end{tabularx}
\feTableEnd
\end{center}

\paragraph{Harm and Casting.}
Harm not only affects physical capacity, it disrupts magical focus:

\begin{center}
\feTableStart
\begin{tabularx}{\linewidth}{@{}l l Y@{}}
\toprule
\textbf{Harm} & \textbf{Casting Effect} & \textbf{Notes} \
\midrule
Minor & Channel DV +1 & Fatigue of concentration. \
Moderate & Maintain channel DV 2 & Focus falters. \
Severe & Channel breaks; freeform casting +1 SB & Magic slips dangerous. \
Critical & Casting impossible & Patron bargain may intervene. \
\bottomrule
\end{tabularx}
\feTableEnd
\end{center}

\textbf{GM Note:} Harm and Fatigue push characters toward harsher Positions; casters feel this as \emph{lost control}, warriors as \emph{physical collapse}. Both create escalating drama without trivializing recovery.

\paragraph{Boon Interaction with Fatigue}
When Fatigue would force one or more success re-rolls:
\begin{itemize}
  \item Spending \textbf{1 Boon} negates \textbf{one} Fatigue-imposed re-roll.
  \item Additional Fatigue re-rolls must be resolved normally.
\end{itemize}

\noindent
\emph{Note:} Boons may be spent \textbf{after seeing} the result of a Fatigue re-roll.

\paragraph{SB Spend Menu (guidance)}
\begin{itemize}
  \item \textbf{1 SB}: Minor pressure: noise, trace, +1 Supply segment.
  \item \textbf{2 SB}: Moderate setback: alarm raised, lose position/cover, lesser foe or lock.
  \item \textbf{3 SB}: Serious trouble: reinforcements, key gear breaks, rail tick.
  \item \textbf{4+ SB}: Major turn: trap springs, authority arrives, scene shifts.
\end{itemize}

\paragraph{Assistance, Boons, \& Description}
\begin{itemize}
  \item \textbf{Assists:} One helper per action; total Assist dice across sources are capped at +3 (unless a specific Talent states otherwise).
  \item \textbf{Boons:} A player may re-roll one die after seeing the pool. Once per session, in downtime, you may convert 2 Boons $\rightarrow$ 1 XP (max 2 XP via conversion per session). Hold cap: 5. Trim to 2 at scene end.
  \item \textbf{Description Ladder:} Basic (roll as-is), Detailed (re-roll one 1), Intricate (re-roll all 1s and add one flourish on success).
\end{itemize}
 
\paragraph{Maximum die pool}

An individual can have a max die pool of 10d10. All extra are converted to auto-successes. 

\subsection{Boon Sharing}

Players may gift \textbf{1 Boon per scene} to an ally with a brief narrative justification.  
\begin{itemize}
  \item \textbf{Bonded Allies:} If characters share a bond, they may gift \textbf{2 Boons per scene}.  
  \item \textbf{Assistance:} Boons may be spent to enhance an ally’s roll (counts as assistance).  
  \item \textbf{Campaign Events:} Major victories or setbacks may generate shared Boons for the party.  
\end{itemize}

\textbf{Table Use:} Require a short story beat for each gift. Normal Boon limits apply. Track shared Boons openly.  
\textbf{GM Notes:} Reward generosity with extra opportunities, introduce occasional complications from dependence, and balance group vs.\ individual needs.

\subsection{Time Guidance Framework}

\subsubsection{Narrative Time Scales}
Time in Fate's Edge is measured by story weight, not by clocks:
\begin{itemize}
  \item \textbf{A Moment} — A heartbeat, a glance, a single strike or word.
  \item \textbf{Some Time} — A few minutes: a skirmish, a careful lockpick, a short negotiation.
  \item \textbf{Significant Time} — Hours: travel between locations, work a ritual, endure a siege.
  \item \textbf{Days} — Large-scale endeavors: marches across countryside, training a cadre, recovery.
\end{itemize}

\subsubsection{Game Structure Definitions}
\begin{description}[leftmargin=1.5em, labelindent=0em]
  \item[Scene] The basic unit of narrative play (Some Time to Significant Time); resolves a specific question or conflict.
  \item[Player Turn (Beat)] Declare action $\rightarrow$ GM sets position $\rightarrow$ roll $\rightarrow$ resolve outcome $\rightarrow$ manage consequences.
  \item[Round] Simultaneous or near-simultaneous actions within a scene (primarily for combat), representing a few seconds.
  \item[Session] One complete game session (typically 3–6 hours), containing 2–4 major scenes and resolving significant narrative progress.
  \item[Campaign] Entire story arc (6–20+ sessions) with major character development and lasting consequences.
\end{description}

\subsection{Fatigue}
\label{subsec:fatigue}
\index{Fatigue}

\textbf{Track:} Each character has a Fatigue track equal to \textbf{Body}. Mark Fatigue for exertion, strain, or backlash.

\textbf{In Play:} Each Fatigue step worsens your \textbf{Position} by one level 
(Deominant $\rightarrow$ Controlled $\rightarrow$ Desperate). 
If you are already \textbf{Desperate}, instead apply a \textbf{--1 die} penalty per Fatigue to that roll.

\textbf{Overflow:} When your Fatigue track fills, immediately increase \textbf{Harm by 1 step} and clear all Fatigue to 0. 
If this raises Harm to a level that incapacitates you, you fall out of the scene as normal for Harm.

\textbf{Recovery:} Short rest clears 1--2 Fatigue; a full night's rest clears all Fatigue.


\begin{fatebox}[Tracking NPC Mechanics]
  Not every meter needs to be tracked for NPCs. 
  
  \begin{itemize}
      \item \textbf{Spotlight First:} NPCs only carry Obligation, Corruption, or similar mechanics if these traits matter to the current story.
      \item \textbf{Skip the Bookkeeping:} Do not track every enemy’s resource pool. If it’s not driving narrative tension, it can be abstracted away.
      \item \textbf{Focus on Impact:} Apply NPC Obligation or Corruption only when it changes how the party experiences them — e.g., a Patron visibly twisting a rival’s fate, or a recurring villain consumed by corruption.
      \item \textbf{Player-Facing First:} Keep full mechanics for PCs, since their journey is the story’s core.
  \end{itemize}
  
  This principle keeps GM effort focused where it matters: driving story beats and consequences, not filling ledgers.
  \end{fatebox}
  
\subsection{Initiative and Turn Order}

Fate's Edge does not use fixed initiative. 
Turn order follows the fiction and the GM's facilitation:
\begin{itemize}
    \item \textbf{Narrative Fiat:} The GM frames spotlight order based on circumstances, tension, and narrative flow.
    \item \textbf{Player Input:} Players may suggest acting when it makes sense in the fiction. 
    \item \textbf{Surprise:} Ambushers act first; targets respond after the opening exchange.
    \item \textbf{Flexibility:} Spotlight may shift mid-scene if fictionally appropriate (e.g., reacting to a falling ceiling, seizing a moment).
\end{itemize}

This ensures pacing and drama guide the sequence of actions, not rigid turn structures.

\subsection{Turn Economy (Quick Rules)}
\label{subsec:turn-economy-quick}

\paragraph{Two Actions.}
Each character takes \emph{1 Action and 1 Move} on their turn. Actions and Moves may be taken in any order; repeating the same Action is not allowed unless noted. A character may use a Boon to re-roll their action at the expense of their move if they still have it available. Some weapon tempos effect whether you can take an attack and a move.

\paragraph{Move.}
Traverse up to your normal movement. \emph{Disengage:} move without provoking; your next offensive action is \textbf{Controlled}. \emph{Dash:} move again this turn; your next defense is \textbf{Desperate}.

\paragraph{Attack.}
Make a melee or ranged attack versus DV set by the GM and fiction. Teamwork/Assist costs 1 Boon.

\paragraph{Observe / Change Position (+1).}
Take a beat to read the field or set angles; gain \textbf{+1 Position} for one action this turn (e.g., Controlled$\to$Dominant). Limit: once/turn; cannot exceed \textbf{Dominant}.

\paragraph{Activate an Asset.}
Use gear, symbol, tool, or feature per its text/tags (e.g., torch, grapnel, smoke vial, rune focus). Items with \texttt{[Action]} consume one Action; \texttt{[Free]} do not.

\paragraph{Setup (Teamwork).}
Create advantage for an ally; on success, grant their next action \textbf{+1 Position} or step up Effect (GM’s call).

\paragraph{Assist (Teamwork).}
Spend \emph{1 Boon} to give an ally \emph{+1 die} on their current roll; you share appropriate risk/consequence.

\paragraph{Protect.}
Adopt a guarding stance or body-block. Choose a nearby ally; until your next turn you may intercept one hit on them and roll to resist it. On success, reduce/negate Harm; you take any fallout the GM assigns.

\paragraph{Channel / Weave.}
Runekeeper/ritual flow: \emph{Channel} (prime power) then \emph{Weave} (shape/release). Disruption or engagement may worsen Position; if \emph{Interrupted}, the casting fails.

\paragraph{Cast Rite / Song (Cantor).}
Perform a Rite/Song per its write-up. You may \emph{Push} to accelerate or empower at the cost of Fatigue/Corruption per class rules.

\paragraph{Interact.}
Lift, pull, flip a lever, shove a foe, break an object, apply a poultice, reload, draw/stow, etc. GM sets DV/Effect.

\paragraph{Defend. (Standard/Move):}  
Until next turn, you count as Defending. When resisting any attack or effect, roll normally and \textbf{improve your Position by one step} (or gain \textbf{+1d} if already Dominant).

\end{itemize}

Success negates the hit.  
Partial reduces it.  
Miss means you take it—\emph{but you learn from it}.
\paragraph{Free Items.}
Short shouts, dropping an item, quick glance. Longer or tactical assessments require \emph{Observe / Change Position} or \emph{Interact}.

\paragraph{Reactions (Out of Turn).}
\emph{Protection} may trigger when an ally is hit and you are in position. Class/Asset reactions fire as written (e.g., counter-runes, ripostes). A character may only attempt to resist an attack unless they are \textbf{Defending} unless they have a talent which lets them do so.

\paragraph{Position Caps.}
Bonuses cannot raise Position above \textbf{Dominant}; penalties cannot drop below \textbf{Desperate}. Beyond these caps, adjust DV or Effect instead.


\subsubsection{Magic and Ritual Time}
\begin{itemize}
  \item \textbf{Standard Casting:} Channel and Weave phases each take 1 Player Turn; resolves within a single scene.
  \item \textbf{Ritual Casting (Optional Rule):} Channel and Weave phases each require 1 Scene (Significant Time).
  \item \textbf{Rites Invocation:} Invoke takes 1 Player Turn; Weave takes 1 Player Turn. High-Power rites may require extended time by fiction.
\end{itemize}

\paragraph{Extended Rituals}
Attach long rituals to clocks:
\begin{itemize}
  \item 4-segment clock: Significant Time (hours)
  \item 6-segment clock: Extended Time (days)
  \item 8+ segment clock: Campaign Time (weeks/months)
\end{itemize}
Advance the clock through player actions, scenes, or set intervals.

\subsection{Worked Micro-Examples}
\begin{itemize}
  \item \textbf{Lockpick Under Watch (DV 2):} Roll 6 dice: 10, 8, 5, 4, 1, 1 $\Rightarrow S=2, C=2$. \emph{Success \& Cost.} Door opens; GM spends 1 SB for a squeal (patrol starts moving) and banks 1 SB to bring that patrol around on the next beat.
  \item \textbf{Charm the Captain (DV 2):} Roll 5 dice: 7, 6, 6, 2, 1 $\Rightarrow S=3, C=1$. \emph{Success \& Cost.} Passage granted; GM spends 1 SB: ``He expects a favor on the return leg—he'll collect.''
  \item \textbf{Traverse the Pass (DV 3):} Group pools to net 3 successes but produces $C=3$. \emph{Success \& Cost.} GM spends 2 SB to add Fatigue 1 to all from cold and exposure, banks 1 SB to crack a wagon axle next scene.
\end{itemize}

\paragraph{Fail Forward: Every Roll Matters}
When you \textbf{MISS} on a \emph{meaningful action}, you gain 2 \textbf{Boons}. When you have a \textbf{PARTIAL}, you gain 1 \textbf{Boon}. Boons can be spent immediately for re-rolls, Asset activations, Rites, and other abilities. You can hold up to 5 Boons (trim to 2 at scene end).\\
A miss only awards Boons if all three are true:
\begin{enumerate}
  \item Procedure followed: intent and approach declared; DV set; roll resolved.
  \item Stakes stated: what changes on success; what bites on failure.
  \item Consequence lands now: the GM spends or banks SB, applies a condition, or advances a thread.
\end{enumerate}
Typically, failures reward boons. Rehearsal/null-risk probes and repeated identical attempts in the same scene do not award Boons. Rule of thumb, if it feels like an obvious fishing attempt, do not award a boon.

\subsection{Session Loop}

\textbf{Off-Screen (Downtime).} Clear/mark clocks, pay Upkeep, manage Obligation, craft, gather info, frame intents.

\textbf{On-Screen (Adventure).} Play scenes, make moves, trigger Rites/Casting, advance fronts.

\textbf{Wrap-Up.} Award XP, mark Story Beats (SB), resolve Harm/Fatigue conversion, advance faction clocks, note Patron Largess.

\textbf{Off-Screen Hooks.} Record next Downtime intents (projects, service to Patrons, upkeep needs) and any cliffhangers.

\subsection{Small Folk of the Threshold (Aelaerem \& Aelinnel)}
\label{subsec:small-folk-threshold}

\index{Aelaerem}\index{Aelinnel}\index{Small Folk}

The Aelaerem and Aelinnel are diminutive peoples attuned to liminal spaces and hidden ways. Their stature grants them agility and subtlety, though at the cost of bearing heavy arms or armor.

\begin{itemize}
  \item \textbf{Restriction:} Cannot use \emph{Heavy Armor} or \emph{Heavy Weapons}.
  \item \textbf{Bonus:} Gain +1 \emph{Position} when Dodging or Resisting Knockback, and +1 die on \emph{Hide} or \emph{Evasion} rolls made while in cover.
\end{itemize}

Their presence in the world is often underestimated, but their knack for slipping unseen through thresholds and enduring where others falter has earned them a quiet reverence.

\subsection{War Mount Examples}
\label{subsec:war-mount-examples}
\index{Mounts!Examples}
\index{Talents!Cavalier}

Characters with the \textbf{War Mount} asset and the \textbf{Cavalier} talent gain unique bonuses when fighting from horseback or equivalent mounts. These examples illustrate typical play.

\paragraph{Mounted Charge (Melee).}
Sir Aven, a Vhasian Knight (Body 4 + Melee 3 = 7d10), spurs his warhorse from Far to Near range against a bandit line. 
Because of \emph{Cavalier}, he rolls +2d (total 9d10). 
The charge succeeds with Great Effect, smashing through the bandits and inflicting Harm~2. 
The GM spends SB to complicate: the horse’s barding cracks, requiring repair before the next battle.  
This demonstrates the mount’s ability to convert distance into overwhelming melee impact.

\paragraph{Ride-by Shot (Ranged).}
Later, Aven switches to bowfire. He retreats from Near to Far range while loosing arrows (Body 3 + Ranged 3 = 6d10, +2d from Cavalier = 8d10). 
A clean success deals Harm~1 to a pursuing marksman. 
The GM spends SB to draw from the Deck, introducing an arcane dust ward that raises DV for further ranged attacks until repositioned.  
This shows the mount’s ability to keep pressure on enemies while maneuvering, at the cost of potential environmental complications.

\paragraph{Summary.}
The War Mount grants mobility and offensive momentum: 
\begin{itemize}
\item Melee charges gain +2d when crossing from Far to Near. 
\item Ranged volleys gain +2d when moving from Near to Far. 
\end{itemize}
GMs should introduce fatigue, supply cost, and environmental complications to balance the tactical advantage of mounted combat.


\section{Core Mechanic}

% =========================
% Turn Economy (Quick Rules)
% =========================

\subsection{The Art of Consequence}
\paragraph{Adjudicating Rolls: The Core Resolution Cycle}\index{roll adjudication}

When a player rolls, they are not simply trying to \emph{beat a number}. They are engaging the world through risk, consequence, and discovery. This section walks through the full cycle.

\paragraph{Step-by-Step Roll Resolution}

\begin{enumerate}
    \item \textbf{Declare Action \& Approach:} Player states intent, Attribute + Skill.
    \item \textbf{Set Difficulty Value (DV):}\index{Difficulty Value (DV)} Based on narrative stakes, not just mechanics.
    \item \textbf{Establish Position:}\index{Position} GM sets whether the roll is \textbf{Dominant}, \textbf{Controlled}, or \textbf{Desperate}.
    \item \textbf{Roll Pool of d10s.}
    \item \textbf{Count:} \textbf{Successes (6+)} and \textbf{Story Beats (1s)}.\index{Story Beats} 
    Each \textbf{10 counts as two successes} but does not auto-succeed if total $< DV$.
    \item \textbf{Check Against DV:} Apply the Outcome Matrix.
    \item \textbf{Spend SB:} GM spends/banks Story Beats or draws from the Deck of Consequences.\index{Deck of Consequences}
\end{enumerate}

\begin{fatebox}[Position Effects]\index{Position}
\begin{tabularx}{\textwidth}{lX}
\toprule
\textbf{Position} & \textbf{Effect} \\
\midrule
Dominant & May re-roll one \textbf{failure} (die $<6$). \\
Controlled & Default state; no re-rolls. \\
Desperate & Must re-roll one \textbf{success} (die $6+$), keeping the second result. \\
\bottomrule
\end{tabularx}
\end{fatebox}

\begin{fatebox}[Difficulty Ladder]\index{Difficulty Ladder}
  \begin{tabularx}{\textwidth}{lX}
  \toprule
  \textbf{DV} & \textbf{Typical Case} \\
  \midrule
  3 & Routine: clear intent, modest stakes, stable setting \\
  4 & Pressured: time limits, mild resistance, incomplete information \\
  5 & Hard: hostile conditions, active opposition, precision required \\
  6+ & Extreme: stacked constraints, dangerous failure, high drama \\
  \bottomrule
  \end{tabularx}
  \end{fatebox}

\textit{A DV should measure narrative weight as much as difficulty. Scaling a wall is routine. Scaling it while lantern-wardens pursue is pressured—or worse.}

\begin{tcolorbox}[title=\textbf{Difficulty Values (DV) by Tier},colback=white!97!gray,
colframe=black!80!gray,sharp corners,boxrule=0.4pt]
\textbf{Guideline.}
The base Difficulty Value (DV) for an opposed or environmental test scales with Tier:
\[
\boxed{\text{DV} = \text{Tier} + 2 + \text{Modifiers}}
\]

\textbf{Typical DVs.}
\begin{center}
\renewcommand{\arraystretch}{1.1}
\begin{longtable}{l c c}
\toprule
\textbf{Tier} & \textbf{Base DV} & \textbf{Example Challenge}\\
\midrule
I & 5 & Local threat / novice test\\
II & 6 & Veteran foe or skilled task\\
III & 7 & Elite / magical challenge\\
IV & 8 & Mythic or cosmic threat\\
\bottomrule
\end{longtable}
\end{center}

\textbf{Positional Modifiers.}
\begin{itemize}[leftmargin=*]
  \item \textbf{Desperate:} +2  \textbf{Dominant:} +1  \textbf{Controlled:} +0  \textbf{Desperate:} –1
\end{itemize}
Use \textit{DV = Tier + 2} as the default; adjust for environment, advantage, or narrative pressure.
\end{tcolorbox}

\begin{fatebox}[Outcome Matrix]\index{Outcome Matrix}
\begin{tabularx}{\textwidth}{lX}
\toprule
\textbf{Result} & \textbf{GM Guidance} \\
\midrule
$S \geq DV$, $C = 0$ & Clean Success: Grant intent, no added friction. \\
$S \geq DV$, $C > 0$ & Success \& Cost: Intent achieved; GM spends SB for complications. \\
$0 < S < DV$ & Partial: Progress \emph{proportional} to hits; intent advances but with gaps or risk. Player gains 1 Boon. \\
$S = 0$ & Miss: No progress. GM escalates with SB/Clocks. Player gains 2 Boons. \\
\bottomrule
\end{tabularx}
\end{fatebox}

\paragraph{Fail Forward: Every Roll Matters}\index{Fail Forward}

\textbf{Partials are the most common form of “success.”} They always move the fiction forward in proportion to the progress rolled.

\begin{quote}
\textit{One success on DV 4:} “The lock is stubborn. You think you can get it if you keep trying.”  
\textit{Three successes on DV 4:} “The lock springs open with a loud clank—you’re sure the guards heard.” (Upgrade to Success \& Cost; add 2 SB).  
\end{quote}

\textbf{Misses} fuel escalation but always generate player resources: 2 Boons and a consequence.  

A roll is \emph{meaningful} or \emph{significant} if:  
\begin{enumerate}
  \item The standard procedure is followed (intent + DV + roll).  
  \item Stakes are stated up front (what changes on success, what bites on failure).  
  \item Real consequences occur now (SB spent, condition applied, or thread advanced).  
\end{enumerate}

\paragraph{Important Notes}
\begin{itemize}
  \item Rolling a 1 always creates SB for the GM. Rerolls do not erase SB.  
  \item No Boons for rehearsal, trivial probes, or repeating an identical approach without changing fiction.  
  \item Controlled tests with no bite give positioning/info, not Boons.  
\end{itemize}

\subsubsection{Anti-Fishing Measures}
\begin{itemize}
  \item \textbf{Cap:} At most 2 Boons from failures per character per scene (further misses still make SB).  
  \item \textbf{Repetition Rule:} Same action + same stakes in the same scene can’t grant another Boon.  
\end{itemize}

\paragraph{Example}
Lockpicking under watch (\emph{Desperate}, DV 3).  
\textbf{Miss:} GM spends 2 SB to start \emph{Guards Incoming [6]}. Player earns 2 Boons.  
\textbf{Partial (2 successes):} Door opens halfway; guard footsteps approach. Player earns 1 Boon.  

\subsubsection{Boon Sharing}
Players may gift 1 Boon per scene to an ally with narrative justification.  
\begin{itemize}
  \item \textbf{Bonded Allies:} Up to 2 Boons gifted per scene.  
  \item \textbf{Assistance:} Shared Boons can enhance an ally’s roll.  
  \item \textbf{Campaign Events:} Major milestones may generate party-wide Boons.  
\end{itemize}

\textbf{GM Note:} Encourage gifts with roleplay beats, but balance generosity with potential dependency or group tension.

\paragraph{Rule — Re-rolling 1s and SB}
Re-rolling 1s does not remove the Story Beats already generated by those dice. If any re-rolled dice show 1 again, they generate additional SB as normal.\\
Let $C_0$ = initial 1s, $C_r$ = 1s on re-rolls $\Rightarrow$ \textbf{Total SB} $= C_0 + C_r$.\\
\emph{Example:} You roll 7d10: \{9, 8, 5, 4, 3, \textbf{1}, \textbf{1}\} $\Rightarrow C_0=2$. You re-roll both 1s (Intricate): \{6, 2\} $\Rightarrow C_r=0$. Final: successes = 3, SB = 2 (the initial SB remain).

\subsubsection{Story Beats}
Story Beats (SB) are the engine of drama. They are not simple penalties, but narrative levers. The GM spends SB to introduce setbacks appropriate to the context:
\begin{itemize}
  \item \textbf{Escalation} — drawing more enemies, raising the stakes.
  \item \textbf{Exhaustion} — draining time, resources, or positioning.
  \item \textbf{Exposure} — revealing hidden actions, alerting foes.
  \item \textbf{Collateral} — harm or danger spilling over onto allies, innocents, or surroundings.
\end{itemize}

\subsubsection{Design Intent}
This mechanic ensures that every roll changes the story. Success without risk is rare, and even failure opens new narrative avenues.

\subsubsection{GM Quick Reference: Adjudicating Skill Checks}

\paragraph{Difficulty Ladder (set before the roll)}
\begin{center}
\begin{longtable}{@{}lll@{}}
\toprule
\textbf{DV} & \textbf{Name} & \textbf{When to Use} \\
\midrule
2 & Routine   & Clear intent, modest stakes, controlled environment. \\
3 & Pressured & Time pressure, mild resistance, partial info. \\
4 & Hard      & Hostile conditions, active opposition, precise timing. \\
5+& Extreme   & Multiple constraints, high precision, dramatic failure. \\
\bottomrule
\end{longtable}
\end{center}

\paragraph{Outcome Matrix (after the roll)}
Let $S$ be successes ($\geq 6$) and $C$ be SB (number of 1s rolled).
\begin{center}
\begin{longtable}{@{}ll@{}}
\toprule
\textbf{Case} & \textbf{Guidance} \\
\midrule
$S \geq DV$ and $C=0$ & \textbf{Clean Success}: Deliver the intent crisply. \\
$S \geq DV$ and $C>0$ & \textbf{Success \& Cost}: Grant the intent; spend/bank SB for complications. \\
$0 < S < DV$          & \textbf{Partial}: Progress with a fork. Award 1 boon \\
$S = 0$               & \textbf{Miss}: No progress. Cash/bank SB. Award 2 boons \\
\bottomrule
\end{longtable}
\end{center}

\subsection{Critical Success}

Rolling a \textbf{10} on any die indicates a critical tier of success. Each 10 adds weight to the outcome:

\begin{itemize}
  \item \textbf{One 10:} Strong success with a free boon, improved Position, or other narrative flourish.
  \item \textbf{Two 10s:} Exceptional success; choose two benefits or a single powerful effect.
  \item \textbf{Three 10s:} Legendary success; resolve the conflict dramatically and progress or clear 1 segment on a secondary clock (generally, a clock tied to the scene, not the overarching campaign).
  \item \textbf{Four+ 10s:} Mythic success; progress or clear 1--2 segments from a secondary clock or create a significant story development.
\end{itemize}

\noindent If no 10s are rolled, resolve the action normally by the highest die result.

\noindent \textbf{10s are never re-rolled by Position effects or other mechanics. Critical hit effects always take place if the roll is successful, despite any SB rolled. Critical successes may reduce Backlash/Obligation/Corruption severity by one tier.}

\subsection{Position}
\label{subsec:position}
\index{Position}

Every action in \indexterm{Fate's Edge} takes place from a \textbf{Position} that reflects the character’s advantage or disadvantage in the scene. Position sets the tone for the roll, narratively and mechanically. It comes in three states:

\begin{itemize}
  \item \textbf{Dominant:} You act from a place of control, leverage, or overwhelming advantage.
  \item \textbf{Controlled:} The standard state of play. Outcomes are uncertain but balanced.
  \item \textbf{Desperate:} You act from dire straits, cornered or overmatched, with everything at stake.
\end{itemize}

\paragraph{Re-roll Mechanic.}  
Position modifies the dice pool through simple re-rolls:
\begin{center}
\begin{longtable}{@{}lll@{}}
\toprule
\textbf{Position} & \textbf{Narrative Frame} & \textbf{Mechanical Effect} \\
\midrule
Dominant & You press your advantage & Re-roll one \emph{failure} \\
Controlled    & The balanced norm & No re-rolls \\
Desperate & You act under duress & Re-roll one \emph{success} \\
\bottomrule
\end{longtable}
\end{center}

\subsubsection{Fatigue and Harm}\index{Fatigue}\index{Harm}

\paragraph{Fatigue.}
Fatigue is capped by \textbf{Body}. It reflects cumulative strain from combat, travel, or channeling. Fatigue never reduces dice directly; it shifts \textbf{Position} to riskier states.

\begin{center}
\feTableStart
\begin{tabularx}{\linewidth}{@{}l l Y@{}}
\toprule
\textbf{Fatigue} & \textbf{Position Shift} & \textbf{Narrative Effect} \
\midrule
1 (Winded) & Dominant → Controlled (once/scene) & Breathing heavy, off-balance. \
2 (Strained) & Controlled rolls create +1 SB on 1s & Mistakes creep in. \
3 (Exhausted) & Controlled → Desperate (once/scene) & Desperate exertion. \
4 (Collapse) & DV 3 Body test or Severe Harm & Push past limits. \
\bottomrule
\end{tabularx}
\feTableEnd
\end{center}

\paragraph{Harm and Casting.}
Harm not only affects physical capacity, it disrupts magical focus:

\begin{center}
\feTableStart
\begin{tabularx}{\linewidth}{@{}l l Y@{}}
\toprule
\textbf{Harm} & \textbf{Casting Effect} & \textbf{Notes} \
\midrule
Minor & Channel DV +1 & Fatigue of concentration. \
Moderate & Maintain channel DV 2 & Focus falters. \
Severe & Channel breaks; freeform casting +1 SB & Magic slips dangerous. \
Critical & Casting impossible & Patron bargain may intervene. \
\bottomrule
\end{tabularx}
\feTableEnd
\end{center}

\textbf{GM Note:} Harm and Fatigue push characters toward harsher Positions; casters feel this as \emph{lost control}, warriors as \emph{physical collapse}. Both create escalating drama without trivializing recovery.

\paragraph{Boon Interaction with Fatigue}
When Fatigue would force one or more success re-rolls:
\begin{itemize}
  \item Spending \textbf{1 Boon} negates \textbf{one} Fatigue-imposed re-roll.
  \item Additional Fatigue re-rolls must be resolved normally.
\end{itemize}

\noindent
\emph{Note:} Boons may be spent \textbf{after seeing} the result of a Fatigue re-roll.

\paragraph{SB Spend Menu (guidance)}
\begin{itemize}
  \item \textbf{1 SB}: Minor pressure: noise, trace, +1 Supply segment.
  \item \textbf{2 SB}: Moderate setback: alarm raised, lose position/cover, lesser foe or lock.
  \item \textbf{3 SB}: Serious trouble: reinforcements, key gear breaks, rail tick.
  \item \textbf{4+ SB}: Major turn: trap springs, authority arrives, scene shifts.
\end{itemize}

\paragraph{Assistance, Boons, \& Description}
\begin{itemize}
  \item \textbf{Assists:} One helper per action; total Assist dice across sources are capped at +3 (unless a specific Talent states otherwise).
  \item \textbf{Boons:} A player may re-roll one die after seeing the pool. Once per session, in downtime, you may convert 2 Boons $\rightarrow$ 1 XP (max 2 XP via conversion per session). Hold cap: 5. Trim to 2 at scene end.
  \item \textbf{Description Ladder:} Basic (roll as-is), Detailed (re-roll one 1), Intricate (re-roll all 1s and add one flourish on success).
\end{itemize}
 
\paragraph{Maximum die pool}

An individual can have a max die pool of 10d10. All extra are converted to auto-successes. 

\subsection{Boon Sharing}

Players may gift \textbf{1 Boon per scene} to an ally with a brief narrative justification.  
\begin{itemize}
  \item \textbf{Bonded Allies:} If characters share a bond, they may gift \textbf{2 Boons per scene}.  
  \item \textbf{Assistance:} Boons may be spent to enhance an ally’s roll (counts as assistance).  
  \item \textbf{Campaign Events:} Major victories or setbacks may generate shared Boons for the party.  
\end{itemize}

\textbf{Table Use:} Require a short story beat for each gift. Normal Boon limits apply. Track shared Boons openly.  
\textbf{GM Notes:} Reward generosity with extra opportunities, introduce occasional complications from dependence, and balance group vs.\ individual needs.

\subsection{Time Guidance Framework}

\subsubsection{Narrative Time Scales}
Time in Fate's Edge is measured by story weight, not by clocks:
\begin{itemize}
  \item \textbf{A Moment} — A heartbeat, a glance, a single strike or word.
  \item \textbf{Some Time} — A few minutes: a skirmish, a careful lockpick, a short negotiation.
  \item \textbf{Significant Time} — Hours: travel between locations, work a ritual, endure a siege.
  \item \textbf{Days} — Large-scale endeavors: marches across countryside, training a cadre, recovery.
\end{itemize}

\subsubsection{Game Structure Definitions}
\begin{description}[leftmargin=1.5em, labelindent=0em]
  \item[Scene] The basic unit of narrative play (Some Time to Significant Time); resolves a specific question or conflict.
  \item[Player Turn (Beat)] Declare action $\rightarrow$ GM sets position $\rightarrow$ roll $\rightarrow$ resolve outcome $\rightarrow$ manage consequences.
  \item[Round] Simultaneous or near-simultaneous actions within a scene (primarily for combat), representing a few seconds.
  \item[Session] One complete game session (typically 3–6 hours), containing 2–4 major scenes and resolving significant narrative progress.
  \item[Campaign] Entire story arc (6–20+ sessions) with major character development and lasting consequences.
\end{description}

\subsection{Fatigue}
\label{subsec:fatigue}
\index{Fatigue}

\textbf{Track:} Each character has a Fatigue track equal to \textbf{Body}. Mark Fatigue for exertion, strain, or backlash.

\textbf{In Play:} Each Fatigue step worsens your \textbf{Position} by one level 
(Deominant $\rightarrow$ Controlled $\rightarrow$ Desperate). 
If you are already \textbf{Desperate}, instead apply a \textbf{--1 die} penalty per Fatigue to that roll.

\textbf{Overflow:} When your Fatigue track fills, immediately increase \textbf{Harm by 1 step} and clear all Fatigue to 0. 
If this raises Harm to a level that incapacitates you, you fall out of the scene as normal for Harm.

\textbf{Recovery:} Short rest clears 1--2 Fatigue; a full night's rest clears all Fatigue.


\begin{fatebox}[Tracking NPC Mechanics]
  Not every meter needs to be tracked for NPCs. 
  
  \begin{itemize}
      \item \textbf{Spotlight First:} NPCs only carry Obligation, Corruption, or similar mechanics if these traits matter to the current story.
      \item \textbf{Skip the Bookkeeping:} Do not track every enemy’s resource pool. If it’s not driving narrative tension, it can be abstracted away.
      \item \textbf{Focus on Impact:} Apply NPC Obligation or Corruption only when it changes how the party experiences them — e.g., a Patron visibly twisting a rival’s fate, or a recurring villain consumed by corruption.
      \item \textbf{Player-Facing First:} Keep full mechanics for PCs, since their journey is the story’s core.
  \end{itemize}
  
  This principle keeps GM effort focused where it matters: driving story beats and consequences, not filling ledgers.
  \end{fatebox}
  
\subsection{Initiative and Turn Order}

Fate's Edge does not use fixed initiative. 
Turn order follows the fiction and the GM's facilitation:
\begin{itemize}
    \item \textbf{Narrative Fiat:} The GM frames spotlight order based on circumstances, tension, and narrative flow.
    \item \textbf{Player Input:} Players may suggest acting when it makes sense in the fiction. 
    \item \textbf{Surprise:} Ambushers act first; targets respond after the opening exchange.
    \item \textbf{Flexibility:} Spotlight may shift mid-scene if fictionally appropriate (e.g., reacting to a falling ceiling, seizing a moment).
\end{itemize}

This ensures pacing and drama guide the sequence of actions, not rigid turn structures.

\subsection{Turn Economy (Quick Rules)}
\label{subsec:turn-economy-quick}

\paragraph{Two Actions.}
Each character takes \emph{1 Action and 1 Move} on their turn. Actions and Moves may be taken in any order; repeating the same Action is not allowed unless noted. A character may use a Boon to re-roll their action at the expense of their move if they still have it available. Some weapon tempos effect whether you can take an attack and a move.

\paragraph{Move.}
Traverse up to your normal movement. \emph{Disengage:} move without provoking; your next offensive action is \textbf{Controlled}. \emph{Dash:} move again this turn; your next defense is \textbf{Desperate}.

\paragraph{Attack.}
Make a melee or ranged attack versus DV set by the GM and fiction. Teamwork/Assist costs 1 Boon.

\paragraph{Observe / Change Position (+1).}
Take a beat to read the field or set angles; gain \textbf{+1 Position} for one action this turn (e.g., Controlled$\to$Dominant). Limit: once/turn; cannot exceed \textbf{Dominant}.

\paragraph{Activate an Asset.}
Use gear, symbol, tool, or feature per its text/tags (e.g., torch, grapnel, smoke vial, rune focus). Items with \texttt{[Action]} consume one Action; \texttt{[Free]} do not.

\paragraph{Setup (Teamwork).}
Create advantage for an ally; on success, grant their next action \textbf{+1 Position} or step up Effect (GM’s call).

\paragraph{Assist (Teamwork).}
Spend \emph{1 Boon} to give an ally \emph{+1 die} on their current roll; you share appropriate risk/consequence.

\paragraph{Protect.}
Adopt a guarding stance or body-block. Choose a nearby ally; until your next turn you may intercept one hit on them and roll to resist it. On success, reduce/negate Harm; you take any fallout the GM assigns.

\paragraph{Channel / Weave.}
Runekeeper/ritual flow: \emph{Channel} (prime power) then \emph{Weave} (shape/release). Disruption or engagement may worsen Position; if \emph{Interrupted}, the casting fails.

\paragraph{Cast Rite / Song (Cantor).}
Perform a Rite/Song per its write-up. You may \emph{Push} to accelerate or empower at the cost of Fatigue/Corruption per class rules.

\paragraph{Interact.}
Lift, pull, flip a lever, shove a foe, break an object, apply a poultice, reload, draw/stow, etc. GM sets DV/Effect.

\paragraph{Defend. (Standard/Move):}  
Until next turn, you count as Defending. When resisting any attack or effect, roll normally and \textbf{improve your Position by one step} (or gain \textbf{+1d} if already Dominant).

\end{itemize}

Success negates the hit.  
Partial reduces it.  
Miss means you take it—\emph{but you learn from it}.
\paragraph{Free Items.}
Short shouts, dropping an item, quick glance. Longer or tactical assessments require \emph{Observe / Change Position} or \emph{Interact}.

\paragraph{Reactions (Out of Turn).}
\emph{Protection} may trigger when an ally is hit and you are in position. Class/Asset reactions fire as written (e.g., counter-runes, ripostes). A character may only attempt to resist an attack unless they are \textbf{Defending} unless they have a talent which lets them do so.

\paragraph{Position Caps.}
Bonuses cannot raise Position above \textbf{Dominant}; penalties cannot drop below \textbf{Desperate}. Beyond these caps, adjust DV or Effect instead.


\subsubsection{Magic and Ritual Time}
\begin{itemize}
  \item \textbf{Standard Casting:} Channel and Weave phases each take 1 Player Turn; resolves within a single scene.
  \item \textbf{Ritual Casting (Optional Rule):} Channel and Weave phases each require 1 Scene (Significant Time).
  \item \textbf{Rites Invocation:} Invoke takes 1 Player Turn; Weave takes 1 Player Turn. High-Power rites may require extended time by fiction.
\end{itemize}

\paragraph{Extended Rituals}
Attach long rituals to clocks:
\begin{itemize}
  \item 4-segment clock: Significant Time (hours)
  \item 6-segment clock: Extended Time (days)
  \item 8+ segment clock: Campaign Time (weeks/months)
\end{itemize}
Advance the clock through player actions, scenes, or set intervals.

\subsection{Worked Micro-Examples}
\begin{itemize}
  \item \textbf{Lockpick Under Watch (DV 2):} Roll 6 dice: 10, 8, 5, 4, 1, 1 $\Rightarrow S=2, C=2$. \emph{Success \& Cost.} Door opens; GM spends 1 SB for a squeal (patrol starts moving) and banks 1 SB to bring that patrol around on the next beat.
  \item \textbf{Charm the Captain (DV 2):} Roll 5 dice: 7, 6, 6, 2, 1 $\Rightarrow S=3, C=1$. \emph{Success \& Cost.} Passage granted; GM spends 1 SB: ``He expects a favor on the return leg—he'll collect.''
  \item \textbf{Traverse the Pass (DV 3):} Group pools to net 3 successes but produces $C=3$. \emph{Success \& Cost.} GM spends 2 SB to add Fatigue 1 to all from cold and exposure, banks 1 SB to crack a wagon axle next scene.
\end{itemize}

\paragraph{Fail Forward: Every Roll Matters}
When you \textbf{MISS} on a \emph{meaningful action}, you gain 2 \textbf{Boons}. When you have a \textbf{PARTIAL}, you gain 1 \textbf{Boon}. Boons can be spent immediately for re-rolls, Asset activations, Rites, and other abilities. You can hold up to 5 Boons (trim to 2 at scene end).\\
A miss only awards Boons if all three are true:
\begin{enumerate}
  \item Procedure followed: intent and approach declared; DV set; roll resolved.
  \item Stakes stated: what changes on success; what bites on failure.
  \item Consequence lands now: the GM spends or banks SB, applies a condition, or advances a thread.
\end{enumerate}
Typically, failures reward boons. Rehearsal/null-risk probes and repeated identical attempts in the same scene do not award Boons. Rule of thumb, if it feels like an obvious fishing attempt, do not award a boon.

\subsection{Session Loop}

\textbf{Off-Screen (Downtime).} Clear/mark clocks, pay Upkeep, manage Obligation, craft, gather info, frame intents.

\textbf{On-Screen (Adventure).} Play scenes, make moves, trigger Rites/Casting, advance fronts.

\textbf{Wrap-Up.} Award XP, mark Story Beats (SB), resolve Harm/Fatigue conversion, advance faction clocks, note Patron Largess.

\textbf{Off-Screen Hooks.} Record next Downtime intents (projects, service to Patrons, upkeep needs) and any cliffhangers.

\subsection{Small Folk of the Threshold (Aelaerem \& Aelinnel)}
\label{subsec:small-folk-threshold}

\index{Aelaerem}\index{Aelinnel}\index{Small Folk}

The Aelaerem and Aelinnel are diminutive peoples attuned to liminal spaces and hidden ways. Their stature grants them agility and subtlety, though at the cost of bearing heavy arms or armor.

\begin{itemize}
  \item \textbf{Restriction:} Cannot use \emph{Heavy Armor} or \emph{Heavy Weapons}.
  \item \textbf{Bonus:} Gain +1 \emph{Position} when Dodging or Resisting Knockback, and +1 die on \emph{Hide} or \emph{Evasion} rolls made while in cover.
\end{itemize}

Their presence in the world is often underestimated, but their knack for slipping unseen through thresholds and enduring where others falter has earned them a quiet reverence.

\subsection{War Mount Examples}
\label{subsec:war-mount-examples}
\index{Mounts!Examples}
\index{Talents!Cavalier}

Characters with the \textbf{War Mount} asset and the \textbf{Cavalier} talent gain unique bonuses when fighting from horseback or equivalent mounts. These examples illustrate typical play.

\paragraph{Mounted Charge (Melee).}
Sir Aven, a Vhasian Knight (Body 4 + Melee 3 = 7d10), spurs his warhorse from Far to Near range against a bandit line. 
Because of \emph{Cavalier}, he rolls +2d (total 9d10). 
The charge succeeds with Great Effect, smashing through the bandits and inflicting Harm~2. 
The GM spends SB to complicate: the horse’s barding cracks, requiring repair before the next battle.  
This demonstrates the mount’s ability to convert distance into overwhelming melee impact.

\paragraph{Ride-by Shot (Ranged).}
Later, Aven switches to bowfire. He retreats from Near to Far range while loosing arrows (Body 3 + Ranged 3 = 6d10, +2d from Cavalier = 8d10). 
A clean success deals Harm~1 to a pursuing marksman. 
The GM spends SB to draw from the Deck, introducing an arcane dust ward that raises DV for further ranged attacks until repositioned.  
This shows the mount’s ability to keep pressure on enemies while maneuvering, at the cost of potential environmental complications.

\paragraph{Summary.}
The War Mount grants mobility and offensive momentum: 
\begin{itemize}
\item Melee charges gain +2d when crossing from Far to Near. 
\item Ranged volleys gain +2d when moving from Near to Far. 
\end{itemize}
GMs should introduce fatigue, supply cost, and environmental complications to balance the tactical advantage of mounted combat.


\section{Core Mechanic}

% =========================
% Turn Economy (Quick Rules)
% =========================

\subsection{The Art of Consequence}
\paragraph{Adjudicating Rolls: The Core Resolution Cycle}\index{roll adjudication}

When a player rolls, they are not simply trying to \emph{beat a number}. They are engaging the world through risk, consequence, and discovery. This section walks through the full cycle.

\paragraph{Step-by-Step Roll Resolution}

\begin{enumerate}
    \item \textbf{Declare Action \& Approach:} Player states intent, Attribute + Skill.
    \item \textbf{Set Difficulty Value (DV):}\index{Difficulty Value (DV)} Based on narrative stakes, not just mechanics.
    \item \textbf{Establish Position:}\index{Position} GM sets whether the roll is \textbf{Dominant}, \textbf{Controlled}, or \textbf{Desperate}.
    \item \textbf{Roll Pool of d10s.}
    \item \textbf{Count:} \textbf{Successes (6+)} and \textbf{Story Beats (1s)}.\index{Story Beats} 
    Each \textbf{10 counts as two successes} but does not auto-succeed if total $< DV$.
    \item \textbf{Check Against DV:} Apply the Outcome Matrix.
    \item \textbf{Spend SB:} GM spends/banks Story Beats or draws from the Deck of Consequences.\index{Deck of Consequences}
\end{enumerate}

\begin{fatebox}[Position Effects]\index{Position}
\begin{tabularx}{\textwidth}{lX}
\toprule
\textbf{Position} & \textbf{Effect} \\
\midrule
Dominant & May re-roll one \textbf{failure} (die $<6$). \\
Controlled & Default state; no re-rolls. \\
Desperate & Must re-roll one \textbf{success} (die $6+$), keeping the second result. \\
\bottomrule
\end{tabularx}
\end{fatebox}

\begin{fatebox}[Difficulty Ladder]\index{Difficulty Ladder}
  \begin{tabularx}{\textwidth}{lX}
  \toprule
  \textbf{DV} & \textbf{Typical Case} \\
  \midrule
  3 & Routine: clear intent, modest stakes, stable setting \\
  4 & Pressured: time limits, mild resistance, incomplete information \\
  5 & Hard: hostile conditions, active opposition, precision required \\
  6+ & Extreme: stacked constraints, dangerous failure, high drama \\
  \bottomrule
  \end{tabularx}
  \end{fatebox}

\textit{A DV should measure narrative weight as much as difficulty. Scaling a wall is routine. Scaling it while lantern-wardens pursue is pressured—or worse.}

\begin{tcolorbox}[title=\textbf{Difficulty Values (DV) by Tier},colback=white!97!gray,
colframe=black!80!gray,sharp corners,boxrule=0.4pt]
\textbf{Guideline.}
The base Difficulty Value (DV) for an opposed or environmental test scales with Tier:
\[
\boxed{\text{DV} = \text{Tier} + 2 + \text{Modifiers}}
\]

\textbf{Typical DVs.}
\begin{center}
\renewcommand{\arraystretch}{1.1}
\begin{longtable}{l c c}
\toprule
\textbf{Tier} & \textbf{Base DV} & \textbf{Example Challenge}\\
\midrule
I & 5 & Local threat / novice test\\
II & 6 & Veteran foe or skilled task\\
III & 7 & Elite / magical challenge\\
IV & 8 & Mythic or cosmic threat\\
\bottomrule
\end{longtable}
\end{center}

\textbf{Positional Modifiers.}
\begin{itemize}[leftmargin=*]
  \item \textbf{Desperate:} +2  \textbf{Dominant:} +1  \textbf{Controlled:} +0  \textbf{Desperate:} –1
\end{itemize}
Use \textit{DV = Tier + 2} as the default; adjust for environment, advantage, or narrative pressure.
\end{tcolorbox}

\begin{fatebox}[Outcome Matrix]\index{Outcome Matrix}
\begin{tabularx}{\textwidth}{lX}
\toprule
\textbf{Result} & \textbf{GM Guidance} \\
\midrule
$S \geq DV$, $C = 0$ & Clean Success: Grant intent, no added friction. \\
$S \geq DV$, $C > 0$ & Success \& Cost: Intent achieved; GM spends SB for complications. \\
$0 < S < DV$ & Partial: Progress \emph{proportional} to hits; intent advances but with gaps or risk. Player gains 1 Boon. \\
$S = 0$ & Miss: No progress. GM escalates with SB/Clocks. Player gains 2 Boons. \\
\bottomrule
\end{tabularx}
\end{fatebox}

\paragraph{Fail Forward: Every Roll Matters}\index{Fail Forward}

\textbf{Partials are the most common form of “success.”} They always move the fiction forward in proportion to the progress rolled.

\begin{quote}
\textit{One success on DV 4:} “The lock is stubborn. You think you can get it if you keep trying.”  
\textit{Three successes on DV 4:} “The lock springs open with a loud clank—you’re sure the guards heard.” (Upgrade to Success \& Cost; add 2 SB).  
\end{quote}

\textbf{Misses} fuel escalation but always generate player resources: 2 Boons and a consequence.  

A roll is \emph{meaningful} or \emph{significant} if:  
\begin{enumerate}
  \item The standard procedure is followed (intent + DV + roll).  
  \item Stakes are stated up front (what changes on success, what bites on failure).  
  \item Real consequences occur now (SB spent, condition applied, or thread advanced).  
\end{enumerate}

\paragraph{Important Notes}
\begin{itemize}
  \item Rolling a 1 always creates SB for the GM. Rerolls do not erase SB.  
  \item No Boons for rehearsal, trivial probes, or repeating an identical approach without changing fiction.  
  \item Controlled tests with no bite give positioning/info, not Boons.  
\end{itemize}

\subsubsection{Anti-Fishing Measures}
\begin{itemize}
  \item \textbf{Cap:} At most 2 Boons from failures per character per scene (further misses still make SB).  
  \item \textbf{Repetition Rule:} Same action + same stakes in the same scene can’t grant another Boon.  
\end{itemize}

\paragraph{Example}
Lockpicking under watch (\emph{Desperate}, DV 3).  
\textbf{Miss:} GM spends 2 SB to start \emph{Guards Incoming [6]}. Player earns 2 Boons.  
\textbf{Partial (2 successes):} Door opens halfway; guard footsteps approach. Player earns 1 Boon.  

\subsubsection{Boon Sharing}
Players may gift 1 Boon per scene to an ally with narrative justification.  
\begin{itemize}
  \item \textbf{Bonded Allies:} Up to 2 Boons gifted per scene.  
  \item \textbf{Assistance:} Shared Boons can enhance an ally’s roll.  
  \item \textbf{Campaign Events:} Major milestones may generate party-wide Boons.  
\end{itemize}

\textbf{GM Note:} Encourage gifts with roleplay beats, but balance generosity with potential dependency or group tension.

\paragraph{Rule — Re-rolling 1s and SB}
Re-rolling 1s does not remove the Story Beats already generated by those dice. If any re-rolled dice show 1 again, they generate additional SB as normal.\\
Let $C_0$ = initial 1s, $C_r$ = 1s on re-rolls $\Rightarrow$ \textbf{Total SB} $= C_0 + C_r$.\\
\emph{Example:} You roll 7d10: \{9, 8, 5, 4, 3, \textbf{1}, \textbf{1}\} $\Rightarrow C_0=2$. You re-roll both 1s (Intricate): \{6, 2\} $\Rightarrow C_r=0$. Final: successes = 3, SB = 2 (the initial SB remain).

\subsubsection{Story Beats}
Story Beats (SB) are the engine of drama. They are not simple penalties, but narrative levers. The GM spends SB to introduce setbacks appropriate to the context:
\begin{itemize}
  \item \textbf{Escalation} — drawing more enemies, raising the stakes.
  \item \textbf{Exhaustion} — draining time, resources, or positioning.
  \item \textbf{Exposure} — revealing hidden actions, alerting foes.
  \item \textbf{Collateral} — harm or danger spilling over onto allies, innocents, or surroundings.
\end{itemize}

\subsubsection{Design Intent}
This mechanic ensures that every roll changes the story. Success without risk is rare, and even failure opens new narrative avenues.

\subsubsection{GM Quick Reference: Adjudicating Skill Checks}

\paragraph{Difficulty Ladder (set before the roll)}
\begin{center}
\begin{longtable}{@{}lll@{}}
\toprule
\textbf{DV} & \textbf{Name} & \textbf{When to Use} \\
\midrule
2 & Routine   & Clear intent, modest stakes, controlled environment. \\
3 & Pressured & Time pressure, mild resistance, partial info. \\
4 & Hard      & Hostile conditions, active opposition, precise timing. \\
5+& Extreme   & Multiple constraints, high precision, dramatic failure. \\
\bottomrule
\end{longtable}
\end{center}

\paragraph{Outcome Matrix (after the roll)}
Let $S$ be successes ($\geq 6$) and $C$ be SB (number of 1s rolled).
\begin{center}
\begin{longtable}{@{}ll@{}}
\toprule
\textbf{Case} & \textbf{Guidance} \\
\midrule
$S \geq DV$ and $C=0$ & \textbf{Clean Success}: Deliver the intent crisply. \\
$S \geq DV$ and $C>0$ & \textbf{Success \& Cost}: Grant the intent; spend/bank SB for complications. \\
$0 < S < DV$          & \textbf{Partial}: Progress with a fork. Award 1 boon \\
$S = 0$               & \textbf{Miss}: No progress. Cash/bank SB. Award 2 boons \\
\bottomrule
\end{longtable}
\end{center}

\subsection{Critical Success}

Rolling a \textbf{10} on any die indicates a critical tier of success. Each 10 adds weight to the outcome:

\begin{itemize}
  \item \textbf{One 10:} Strong success with a free boon, improved Position, or other narrative flourish.
  \item \textbf{Two 10s:} Exceptional success; choose two benefits or a single powerful effect.
  \item \textbf{Three 10s:} Legendary success; resolve the conflict dramatically and progress or clear 1 segment on a secondary clock (generally, a clock tied to the scene, not the overarching campaign).
  \item \textbf{Four+ 10s:} Mythic success; progress or clear 1--2 segments from a secondary clock or create a significant story development.
\end{itemize}

\noindent If no 10s are rolled, resolve the action normally by the highest die result.

\noindent \textbf{10s are never re-rolled by Position effects or other mechanics. Critical hit effects always take place if the roll is successful, despite any SB rolled. Critical successes may reduce Backlash/Obligation/Corruption severity by one tier.}

\subsection{Position}
\label{subsec:position}
\index{Position}

Every action in \indexterm{Fate's Edge} takes place from a \textbf{Position} that reflects the character’s advantage or disadvantage in the scene. Position sets the tone for the roll, narratively and mechanically. It comes in three states:

\begin{itemize}
  \item \textbf{Dominant:} You act from a place of control, leverage, or overwhelming advantage.
  \item \textbf{Controlled:} The standard state of play. Outcomes are uncertain but balanced.
  \item \textbf{Desperate:} You act from dire straits, cornered or overmatched, with everything at stake.
\end{itemize}

\paragraph{Re-roll Mechanic.}  
Position modifies the dice pool through simple re-rolls:
\begin{center}
\begin{longtable}{@{}lll@{}}
\toprule
\textbf{Position} & \textbf{Narrative Frame} & \textbf{Mechanical Effect} \\
\midrule
Dominant & You press your advantage & Re-roll one \emph{failure} \\
Controlled    & The balanced norm & No re-rolls \\
Desperate & You act under duress & Re-roll one \emph{success} \\
\bottomrule
\end{longtable}
\end{center}

\subsubsection{Fatigue and Harm}\index{Fatigue}\index{Harm}

\paragraph{Fatigue.}
Fatigue is capped by \textbf{Body}. It reflects cumulative strain from combat, travel, or channeling. Fatigue never reduces dice directly; it shifts \textbf{Position} to riskier states.

\begin{center}
\feTableStart
\begin{tabularx}{\linewidth}{@{}l l Y@{}}
\toprule
\textbf{Fatigue} & \textbf{Position Shift} & \textbf{Narrative Effect} \
\midrule
1 (Winded) & Dominant → Controlled (once/scene) & Breathing heavy, off-balance. \
2 (Strained) & Controlled rolls create +1 SB on 1s & Mistakes creep in. \
3 (Exhausted) & Controlled → Desperate (once/scene) & Desperate exertion. \
4 (Collapse) & DV 3 Body test or Severe Harm & Push past limits. \
\bottomrule
\end{tabularx}
\feTableEnd
\end{center}

\paragraph{Harm and Casting.}
Harm not only affects physical capacity, it disrupts magical focus:

\begin{center}
\feTableStart
\begin{tabularx}{\linewidth}{@{}l l Y@{}}
\toprule
\textbf{Harm} & \textbf{Casting Effect} & \textbf{Notes} \
\midrule
Minor & Channel DV +1 & Fatigue of concentration. \
Moderate & Maintain channel DV 2 & Focus falters. \
Severe & Channel breaks; freeform casting +1 SB & Magic slips dangerous. \
Critical & Casting impossible & Patron bargain may intervene. \
\bottomrule
\end{tabularx}
\feTableEnd
\end{center}

\textbf{GM Note:} Harm and Fatigue push characters toward harsher Positions; casters feel this as \emph{lost control}, warriors as \emph{physical collapse}. Both create escalating drama without trivializing recovery.

\paragraph{Boon Interaction with Fatigue}
When Fatigue would force one or more success re-rolls:
\begin{itemize}
  \item Spending \textbf{1 Boon} negates \textbf{one} Fatigue-imposed re-roll.
  \item Additional Fatigue re-rolls must be resolved normally.
\end{itemize}

\noindent
\emph{Note:} Boons may be spent \textbf{after seeing} the result of a Fatigue re-roll.

\paragraph{SB Spend Menu (guidance)}
\begin{itemize}
  \item \textbf{1 SB}: Minor pressure: noise, trace, +1 Supply segment.
  \item \textbf{2 SB}: Moderate setback: alarm raised, lose position/cover, lesser foe or lock.
  \item \textbf{3 SB}: Serious trouble: reinforcements, key gear breaks, rail tick.
  \item \textbf{4+ SB}: Major turn: trap springs, authority arrives, scene shifts.
\end{itemize}

\paragraph{Assistance, Boons, \& Description}
\begin{itemize}
  \item \textbf{Assists:} One helper per action; total Assist dice across sources are capped at +3 (unless a specific Talent states otherwise).
  \item \textbf{Boons:} A player may re-roll one die after seeing the pool. Once per session, in downtime, you may convert 2 Boons $\rightarrow$ 1 XP (max 2 XP via conversion per session). Hold cap: 5. Trim to 2 at scene end.
  \item \textbf{Description Ladder:} Basic (roll as-is), Detailed (re-roll one 1), Intricate (re-roll all 1s and add one flourish on success).
\end{itemize}
 
\paragraph{Maximum die pool}

An individual can have a max die pool of 10d10. All extra are converted to auto-successes. 

\subsection{Boon Sharing}

Players may gift \textbf{1 Boon per scene} to an ally with a brief narrative justification.  
\begin{itemize}
  \item \textbf{Bonded Allies:} If characters share a bond, they may gift \textbf{2 Boons per scene}.  
  \item \textbf{Assistance:} Boons may be spent to enhance an ally’s roll (counts as assistance).  
  \item \textbf{Campaign Events:} Major victories or setbacks may generate shared Boons for the party.  
\end{itemize}

\textbf{Table Use:} Require a short story beat for each gift. Normal Boon limits apply. Track shared Boons openly.  
\textbf{GM Notes:} Reward generosity with extra opportunities, introduce occasional complications from dependence, and balance group vs.\ individual needs.

\subsection{Time Guidance Framework}

\subsubsection{Narrative Time Scales}
Time in Fate's Edge is measured by story weight, not by clocks:
\begin{itemize}
  \item \textbf{A Moment} — A heartbeat, a glance, a single strike or word.
  \item \textbf{Some Time} — A few minutes: a skirmish, a careful lockpick, a short negotiation.
  \item \textbf{Significant Time} — Hours: travel between locations, work a ritual, endure a siege.
  \item \textbf{Days} — Large-scale endeavors: marches across countryside, training a cadre, recovery.
\end{itemize}

\subsubsection{Game Structure Definitions}
\begin{description}[leftmargin=1.5em, labelindent=0em]
  \item[Scene] The basic unit of narrative play (Some Time to Significant Time); resolves a specific question or conflict.
  \item[Player Turn (Beat)] Declare action $\rightarrow$ GM sets position $\rightarrow$ roll $\rightarrow$ resolve outcome $\rightarrow$ manage consequences.
  \item[Round] Simultaneous or near-simultaneous actions within a scene (primarily for combat), representing a few seconds.
  \item[Session] One complete game session (typically 3–6 hours), containing 2–4 major scenes and resolving significant narrative progress.
  \item[Campaign] Entire story arc (6–20+ sessions) with major character development and lasting consequences.
\end{description}

\subsection{Fatigue}
\label{subsec:fatigue}
\index{Fatigue}

\textbf{Track:} Each character has a Fatigue track equal to \textbf{Body}. Mark Fatigue for exertion, strain, or backlash.

\textbf{In Play:} Each Fatigue step worsens your \textbf{Position} by one level 
(Deominant $\rightarrow$ Controlled $\rightarrow$ Desperate). 
If you are already \textbf{Desperate}, instead apply a \textbf{--1 die} penalty per Fatigue to that roll.

\textbf{Overflow:} When your Fatigue track fills, immediately increase \textbf{Harm by 1 step} and clear all Fatigue to 0. 
If this raises Harm to a level that incapacitates you, you fall out of the scene as normal for Harm.

\textbf{Recovery:} Short rest clears 1--2 Fatigue; a full night's rest clears all Fatigue.


\begin{fatebox}[Tracking NPC Mechanics]
  Not every meter needs to be tracked for NPCs. 
  
  \begin{itemize}
      \item \textbf{Spotlight First:} NPCs only carry Obligation, Corruption, or similar mechanics if these traits matter to the current story.
      \item \textbf{Skip the Bookkeeping:} Do not track every enemy’s resource pool. If it’s not driving narrative tension, it can be abstracted away.
      \item \textbf{Focus on Impact:} Apply NPC Obligation or Corruption only when it changes how the party experiences them — e.g., a Patron visibly twisting a rival’s fate, or a recurring villain consumed by corruption.
      \item \textbf{Player-Facing First:} Keep full mechanics for PCs, since their journey is the story’s core.
  \end{itemize}
  
  This principle keeps GM effort focused where it matters: driving story beats and consequences, not filling ledgers.
  \end{fatebox}
  
\subsection{Initiative and Turn Order}

Fate's Edge does not use fixed initiative. 
Turn order follows the fiction and the GM's facilitation:
\begin{itemize}
    \item \textbf{Narrative Fiat:} The GM frames spotlight order based on circumstances, tension, and narrative flow.
    \item \textbf{Player Input:} Players may suggest acting when it makes sense in the fiction. 
    \item \textbf{Surprise:} Ambushers act first; targets respond after the opening exchange.
    \item \textbf{Flexibility:} Spotlight may shift mid-scene if fictionally appropriate (e.g., reacting to a falling ceiling, seizing a moment).
\end{itemize}

This ensures pacing and drama guide the sequence of actions, not rigid turn structures.

\subsection{Turn Economy (Quick Rules)}
\label{subsec:turn-economy-quick}

\paragraph{Two Actions.}
Each character takes \emph{1 Action and 1 Move} on their turn. Actions and Moves may be taken in any order; repeating the same Action is not allowed unless noted. A character may use a Boon to re-roll their action at the expense of their move if they still have it available. Some weapon tempos effect whether you can take an attack and a move.

\paragraph{Move.}
Traverse up to your normal movement. \emph{Disengage:} move without provoking; your next offensive action is \textbf{Controlled}. \emph{Dash:} move again this turn; your next defense is \textbf{Desperate}.

\paragraph{Attack.}
Make a melee or ranged attack versus DV set by the GM and fiction. Teamwork/Assist costs 1 Boon.

\paragraph{Observe / Change Position (+1).}
Take a beat to read the field or set angles; gain \textbf{+1 Position} for one action this turn (e.g., Controlled$\to$Dominant). Limit: once/turn; cannot exceed \textbf{Dominant}.

\paragraph{Activate an Asset.}
Use gear, symbol, tool, or feature per its text/tags (e.g., torch, grapnel, smoke vial, rune focus). Items with \texttt{[Action]} consume one Action; \texttt{[Free]} do not.

\paragraph{Setup (Teamwork).}
Create advantage for an ally; on success, grant their next action \textbf{+1 Position} or step up Effect (GM’s call).

\paragraph{Assist (Teamwork).}
Spend \emph{1 Boon} to give an ally \emph{+1 die} on their current roll; you share appropriate risk/consequence.

\paragraph{Protect.}
Adopt a guarding stance or body-block. Choose a nearby ally; until your next turn you may intercept one hit on them and roll to resist it. On success, reduce/negate Harm; you take any fallout the GM assigns.

\paragraph{Channel / Weave.}
Runekeeper/ritual flow: \emph{Channel} (prime power) then \emph{Weave} (shape/release). Disruption or engagement may worsen Position; if \emph{Interrupted}, the casting fails.

\paragraph{Cast Rite / Song (Cantor).}
Perform a Rite/Song per its write-up. You may \emph{Push} to accelerate or empower at the cost of Fatigue/Corruption per class rules.

\paragraph{Interact.}
Lift, pull, flip a lever, shove a foe, break an object, apply a poultice, reload, draw/stow, etc. GM sets DV/Effect.

\paragraph{Defend. (Standard/Move):}  
Until next turn, you count as Defending. When resisting any attack or effect, roll normally and \textbf{improve your Position by one step} (or gain \textbf{+1d} if already Dominant).

\end{itemize}

Success negates the hit.  
Partial reduces it.  
Miss means you take it—\emph{but you learn from it}.
\paragraph{Free Items.}
Short shouts, dropping an item, quick glance. Longer or tactical assessments require \emph{Observe / Change Position} or \emph{Interact}.

\paragraph{Reactions (Out of Turn).}
\emph{Protection} may trigger when an ally is hit and you are in position. Class/Asset reactions fire as written (e.g., counter-runes, ripostes). A character may only attempt to resist an attack unless they are \textbf{Defending} unless they have a talent which lets them do so.

\paragraph{Position Caps.}
Bonuses cannot raise Position above \textbf{Dominant}; penalties cannot drop below \textbf{Desperate}. Beyond these caps, adjust DV or Effect instead.


\subsubsection{Magic and Ritual Time}
\begin{itemize}
  \item \textbf{Standard Casting:} Channel and Weave phases each take 1 Player Turn; resolves within a single scene.
  \item \textbf{Ritual Casting (Optional Rule):} Channel and Weave phases each require 1 Scene (Significant Time).
  \item \textbf{Rites Invocation:} Invoke takes 1 Player Turn; Weave takes 1 Player Turn. High-Power rites may require extended time by fiction.
\end{itemize}

\paragraph{Extended Rituals}
Attach long rituals to clocks:
\begin{itemize}
  \item 4-segment clock: Significant Time (hours)
  \item 6-segment clock: Extended Time (days)
  \item 8+ segment clock: Campaign Time (weeks/months)
\end{itemize}
Advance the clock through player actions, scenes, or set intervals.

\subsection{Worked Micro-Examples}
\begin{itemize}
  \item \textbf{Lockpick Under Watch (DV 2):} Roll 6 dice: 10, 8, 5, 4, 1, 1 $\Rightarrow S=2, C=2$. \emph{Success \& Cost.} Door opens; GM spends 1 SB for a squeal (patrol starts moving) and banks 1 SB to bring that patrol around on the next beat.
  \item \textbf{Charm the Captain (DV 2):} Roll 5 dice: 7, 6, 6, 2, 1 $\Rightarrow S=3, C=1$. \emph{Success \& Cost.} Passage granted; GM spends 1 SB: ``He expects a favor on the return leg—he'll collect.''
  \item \textbf{Traverse the Pass (DV 3):} Group pools to net 3 successes but produces $C=3$. \emph{Success \& Cost.} GM spends 2 SB to add Fatigue 1 to all from cold and exposure, banks 1 SB to crack a wagon axle next scene.
\end{itemize}

\paragraph{Fail Forward: Every Roll Matters}
When you \textbf{MISS} on a \emph{meaningful action}, you gain 2 \textbf{Boons}. When you have a \textbf{PARTIAL}, you gain 1 \textbf{Boon}. Boons can be spent immediately for re-rolls, Asset activations, Rites, and other abilities. You can hold up to 5 Boons (trim to 2 at scene end).\\
A miss only awards Boons if all three are true:
\begin{enumerate}
  \item Procedure followed: intent and approach declared; DV set; roll resolved.
  \item Stakes stated: what changes on success; what bites on failure.
  \item Consequence lands now: the GM spends or banks SB, applies a condition, or advances a thread.
\end{enumerate}
Typically, failures reward boons. Rehearsal/null-risk probes and repeated identical attempts in the same scene do not award Boons. Rule of thumb, if it feels like an obvious fishing attempt, do not award a boon.

\subsection{Session Loop}

\textbf{Off-Screen (Downtime).} Clear/mark clocks, pay Upkeep, manage Obligation, craft, gather info, frame intents.

\textbf{On-Screen (Adventure).} Play scenes, make moves, trigger Rites/Casting, advance fronts.

\textbf{Wrap-Up.} Award XP, mark Story Beats (SB), resolve Harm/Fatigue conversion, advance faction clocks, note Patron Largess.

\textbf{Off-Screen Hooks.} Record next Downtime intents (projects, service to Patrons, upkeep needs) and any cliffhangers.

\subsection{Small Folk of the Threshold (Aelaerem \& Aelinnel)}
\label{subsec:small-folk-threshold}

\index{Aelaerem}\index{Aelinnel}\index{Small Folk}

The Aelaerem and Aelinnel are diminutive peoples attuned to liminal spaces and hidden ways. Their stature grants them agility and subtlety, though at the cost of bearing heavy arms or armor.

\begin{itemize}
  \item \textbf{Restriction:} Cannot use \emph{Heavy Armor} or \emph{Heavy Weapons}.
  \item \textbf{Bonus:} Gain +1 \emph{Position} when Dodging or Resisting Knockback, and +1 die on \emph{Hide} or \emph{Evasion} rolls made while in cover.
\end{itemize}

Their presence in the world is often underestimated, but their knack for slipping unseen through thresholds and enduring where others falter has earned them a quiet reverence.

\subsection{War Mount Examples}
\label{subsec:war-mount-examples}
\index{Mounts!Examples}
\index{Talents!Cavalier}

Characters with the \textbf{War Mount} asset and the \textbf{Cavalier} talent gain unique bonuses when fighting from horseback or equivalent mounts. These examples illustrate typical play.

\paragraph{Mounted Charge (Melee).}
Sir Aven, a Vhasian Knight (Body 4 + Melee 3 = 7d10), spurs his warhorse from Far to Near range against a bandit line. 
Because of \emph{Cavalier}, he rolls +2d (total 9d10). 
The charge succeeds with Great Effect, smashing through the bandits and inflicting Harm~2. 
The GM spends SB to complicate: the horse’s barding cracks, requiring repair before the next battle.  
This demonstrates the mount’s ability to convert distance into overwhelming melee impact.

\paragraph{Ride-by Shot (Ranged).}
Later, Aven switches to bowfire. He retreats from Near to Far range while loosing arrows (Body 3 + Ranged 3 = 6d10, +2d from Cavalier = 8d10). 
A clean success deals Harm~1 to a pursuing marksman. 
The GM spends SB to draw from the Deck, introducing an arcane dust ward that raises DV for further ranged attacks until repositioned.  
This shows the mount’s ability to keep pressure on enemies while maneuvering, at the cost of potential environmental complications.

\paragraph{Summary.}
The War Mount grants mobility and offensive momentum: 
\begin{itemize}
\item Melee charges gain +2d when crossing from Far to Near. 
\item Ranged volleys gain +2d when moving from Near to Far. 
\end{itemize}
GMs should introduce fatigue, supply cost, and environmental complications to balance the tactical advantage of mounted combat.


\section{Core Mechanic}

\subsection{The Art of Consequence}

\subsubsection{Procedure}
All significant actions follow a three-step process:
\begin{enumerate}
  \item \textbf{Approach:} The player describes both what their character wants and how they attempt it.
  \item \textbf{Execution:} Build a dice pool equal to \emph{Attribute + Skill} and roll that many d10s. Each die of \textbf{6+} counts as a success. Each \textbf{1} rolled generates a \textbf{Story Beat (SB)}.
  \item \textbf{Outcome:} The GM interprets total successes against the difficulty (DV) of the task. Story Beats are then spent to weave narrative setbacks.
\end{enumerate}

\subsubsection{The Description Ladder}
\begin{itemize}
  \item \textbf{Basic Action:} Roll the pool as-is. All 1s remain as Story Beats.
  \item \textbf{Detailed Action:} A clear, descriptive flourish allows the player to re-roll one die showing 1.
  \item \textbf{Intricate Action:} A richly described, multi-sensory action allows the player to re-roll \emph{all} dice showing 1, and add one positive narrative flourish to the scene if they succeed.
\end{itemize}

\paragraph{Rule — Re-rolling 1s and SB}
Re-rolling 1s does not remove the Story Beats already generated by those dice. If any re-rolled dice show 1 again, they generate additional SB as normal.\\
Let $C_0$ = initial 1s, $C_r$ = 1s on re-rolls $\Rightarrow$ \textbf{Total SB} $= C_0 + C_r$.\\
\emph{Example:} You roll 7d10: \{9, 8, 5, 4, 3, \textbf{1}, \textbf{1}\} $\Rightarrow C_0=2$. You re-roll both 1s (Intricate): \{6, 2\} $\Rightarrow C_r=0$. Final: successes = 3, SB = 2 (the initial SB remain).

\subsubsection{Story Beats}
Story Beats (SB) are the engine of drama. They are not simple penalties, but narrative levers. The GM spends SB to introduce setbacks appropriate to the context:
\begin{itemize}
  \item \textbf{Escalation} — drawing more enemies, raising the stakes.
  \item \textbf{Exhaustion} — draining time, resources, or positioning.
  \item \textbf{Exposure} — revealing hidden actions, alerting foes.
  \item \textbf{Collateral} — harm or danger spilling over onto allies, innocents, or surroundings.
\end{itemize}

\subsubsection{Design Intent}
This mechanic ensures that every roll changes the story. Success without risk is rare, and even failure opens new narrative avenues.

\subsubsection{GM Quick Reference: Adjudicating Skill Checks}

\paragraph{Difficulty Ladder (set before the roll)}
\begin{center}
\begin{tabular}{@{}lll@{}}
\toprule
\textbf{DV} & \textbf{Name} & \textbf{When to Use} \\
\midrule
2 & Routine   & Clear intent, modest stakes, controlled environment. \\
3 & Pressured & Time pressure, mild resistance, partial info. \\
4 & Hard      & Hostile conditions, active opposition, precise timing. \\
5+& Extreme   & Multiple constraints, high precision, dramatic failure. \\
\bottomrule
\end{tabular}
\end{center}

\paragraph{Outcome Matrix (after the roll)}
Let $S$ be successes ($\geq 6$) and $C$ be SB (number of 1s rolled).
\begin{center}
\begin{tabular}{@{}ll@{}}
\toprule
\textbf{Case} & \textbf{Guidance} \\
\midrule
$S \geq DV$ and $C=0$ & \textbf{Clean Success}: Deliver the intent crisply. \\
$S \geq DV$ and $C>0$ & \textbf{Success \& Cost}: Grant the intent; spend/bank SB for complications. \\
$0 < S < DV$          & \textbf{Partial}: Progress with a fork. Award 1 boon \\
$S = 0$               & \textbf{Miss}: No progress. Cash/bank SB. Award 2 boons \\
\bottomrule
\end{tabular}
\end{center}

\subsection{Critical Success}
Rolling a \textbf{10} on any die indicates a critical tier of success. Each 10 adds weight to the outcome:

\begin{itemize}
  \item \textbf{One 10:} Strong success with a free boon, improved Position, or other narrative flourish.
  \item \textbf{Two 10s:} Exceptional success; choose two benefits or a single powerful effect.
  \item \textbf{Three 10s:} Legendary success; the action transcends mortal limits and resolves the conflict dramatically.
  \item \textbf{Four+ 10s:} Mythic success; the GM and table agree the result reshapes the scene or story outright.
\end{itemize}

\noindent
If no 10s are rolled, resolve the action normally by the highest die result.

\subsection{Position}
\label{subsec:position}
\index{Position}

Every action in \indexterm{Fate's Edge} takes place from a \textbf{Position} that reflects the character’s advantage or disadvantage in the scene. Position sets the tone for the roll, narratively and mechanically. It comes in three states:

\begin{itemize}
  \item \textbf{Dominant:} You act from a place of control, leverage, or overwhelming advantage.
  \item \textbf{Risky:} The standard state of play. Outcomes are uncertain but balanced.
  \item \textbf{Desperate:} You act from dire straits, cornered or overmatched, with everything at stake.
\end{itemize}

\paragraph{Re-roll Mechanic.}  
Position modifies the dice pool through simple re-rolls:
\begin{center}
\begin{tabular}{@{}lll@{}}
\toprule
\textbf{Position} & \textbf{Narrative Frame} & \textbf{Mechanical Effect} \\
\midrule
Dominant & You press your advantage & Re-roll one \emph{failure} \\
Risky    & The balanced norm & No re-rolls \\
Desperate & You act under duress & Re-roll one \emph{success} \\
\bottomrule
\end{tabular}
\end{center}

\paragraph{SB Spend Menu (guidance)}
\begin{itemize}
  \item \textbf{1 SB}: Minor pressure: noise, trace, +1 Supply segment.
  \item \textbf{2 SB}: Moderate setback: alarm raised, lose position/cover, lesser foe or lock.
  \item \textbf{3 SB}: Serious trouble: reinforcements, key gear breaks, rail tick.
  \item \textbf{4+ SB}: Major turn: trap springs, authority arrives, scene shifts.
\end{itemize}

\paragraph{Assistance, Boons, \& Description}
\begin{itemize}
  \item \textbf{Assists:} One helper per action; total Assist dice across sources are capped at +3 (unless a specific Talent states otherwise).
  \item \textbf{Boons:} A player may re-roll one die after seeing the pool. Once per session, in downtime, you may convert 2 Boons $\rightarrow$ 1 XP (max 2 XP via conversion per session). Hold cap: 5. Trim to 2 at scene end.
  \item \textbf{Description Ladder:} Basic (roll as-is), Detailed (re-roll one 1), Intricate (re-roll all 1s and add one flourish on success).
\end{itemize}

\subsection{Boon Sharing}

Players may gift \textbf{1 Boon per scene} to an ally with a brief narrative justification.  
\begin{itemize}
  \item \textbf{Bonded Allies:} If characters share a bond, they may gift \textbf{2 Boons per scene}.  
  \item \textbf{Assistance:} Boons may be spent to enhance an ally’s roll (counts as assistance).  
  \item \textbf{Campaign Events:} Major victories or setbacks may generate shared Boons for the party.  
\end{itemize}

\textbf{Table Use:} Require a short story beat for each gift. Normal Boon limits apply. Track shared Boons openly.  
\textbf{GM Notes:} Reward generosity with extra opportunities, introduce occasional complications from dependence, and balance group vs.\ individual needs.

\subsection{Time Guidance Framework}

\subsubsection{Narrative Time Scales}
Time in Fate's Edge is measured by story weight, not by clocks:
\begin{itemize}
  \item \textbf{A Moment} — A heartbeat, a glance, a single strike or word.
  \item \textbf{Some Time} — A few minutes: a skirmish, a careful lockpick, a short negotiation.
  \item \textbf{Significant Time} — Hours: travel between locations, work a ritual, endure a siege.
  \item \textbf{Days} — Large-scale endeavors: marches across countryside, training a cadre, recovery.
\end{itemize}

\subsubsection{Game Structure Definitions}
\begin{description}[leftmargin=1.5em, labelindent=0em]
  \item[Scene] The basic unit of narrative play (Some Time to Significant Time); resolves a specific question or conflict.
  \item[Player Turn (Beat)] Declare action $\rightarrow$ GM sets position $\rightarrow$ roll $\rightarrow$ resolve outcome $\rightarrow$ manage consequences.
  \item[Round] Simultaneous or near-simultaneous actions within a scene (primarily for combat), representing a few seconds.
  \item[Session] One complete game session (typically 3–6 hours), containing 2–4 major scenes and resolving significant narrative progress.
  \item[Campaign] Entire story arc (6–20+ sessions) with major character development and lasting consequences.
\end{description}

\subsubsection{Magic and Ritual Time}
\begin{itemize}
  \item \textbf{Standard Casting:} Channel and Weave phases each take 1 Player Turn; resolves within a single scene.
  \item \textbf{Ritual Casting (Optional Rule):} Channel and Weave phases each require 1 Scene (Significant Time).
  \item \textbf{Rites Invocation:} Invoke takes 1 Player Turn; Weave takes 1 Player Turn. High-Power rites may require extended time by fiction.
\end{itemize}

\paragraph{Extended Rituals}
Attach long rituals to clocks:
\begin{itemize}
  \item 4-segment clock: Significant Time (hours)
  \item 6-segment clock: Extended Time (days)
  \item 8+ segment clock: Campaign Time (weeks/months)
\end{itemize}
Advance the clock through player actions, scenes, or set intervals.

\subsection{Worked Micro-Examples}
\begin{itemize}
  \item \textbf{Lockpick Under Watch (DV 2):} Roll 6 dice: 10, 8, 5, 4, 1, 1 $\Rightarrow S=2, C=2$. \emph{Success \& Cost.} Door opens; GM spends 1 SB for a squeal (patrol starts moving) and banks 1 SB to bring that patrol around on the next beat.
  \item \textbf{Charm the Captain (DV 2):} Roll 5 dice: 7, 6, 6, 2, 1 $\Rightarrow S=3, C=1$. \emph{Success \& Cost.} Passage granted; GM spends 1 SB: ``He expects a favor on the return leg—he'll collect.''
  \item \textbf{Traverse the Pass (DV 3):} Group pools to net 3 successes but produces $C=3$. \emph{Success \& Cost.} GM spends 2 SB to add Fatigue 1 to all from cold and exposure, banks 1 SB to crack a wagon axle next scene.
\end{itemize}

\paragraph{Fail Forward: Every Roll Matters}
When you \textbf{MISS} on a \emph{meaningful action}, you gain 2 \textbf{Boons}. When you have a \textbf{PARTIAL}, you gain 1 \textbf{Boon}. Boons can be spent immediately for re-rolls, Asset activations, Rites, and other abilities. You can hold up to 5 Boons (trim to 2 at scene end).\\
A miss only awards Boons if all three are true:
\begin{enumerate}
  \item Procedure followed: intent and approach declared; DV set; roll resolved.
  \item Stakes stated: what changes on success; what bites on failure.
  \item Consequence lands now: the GM spends or banks SB, applies a condition, or advances a thread.
\end{enumerate}
Typically, failures reward boons. Rehearsal/null-risk probes and repeated identical attempts in the same scene do not award Boons. Rule of thumb, if it feels like an obvious fishing attempt, do not award a boon.

\subsection{Session Loop}

\textbf{Off-Screen (Downtime).} Clear/mark clocks, pay Upkeep, manage Obligation, craft, gather info, frame intents.

\textbf{On-Screen (Adventure).} Play scenes, make moves, trigger Rites/Casting, advance fronts.

\textbf{Wrap-Up.} Award XP, mark Story Beats (SB), resolve Harm/Fatigue conversion, advance faction clocks, note Patron Largess.

\textbf{Off-Screen Hooks.} Record next Downtime intents (projects, service to Patrons, upkeep needs) and any cliffhangers.

\subsection{Small Folk of the Threshold (Aelaerem \& Aelinnel)}
\label{subsec:small-folk-threshold}

\index{Aelaerem}\index{Aelinnel}\index{Small Folk}

The Aelaerem and Aelinnel are diminutive peoples attuned to liminal spaces and hidden ways. Their stature grants them agility and subtlety, though at the cost of bearing heavy arms or armor.

\begin{itemize}
  \item \textbf{Restriction:} Cannot use \emph{Heavy Armor} or \emph{Heavy Weapons}.
  \item \textbf{Bonus:} Gain +1 \emph{Position} when Dodging or Resisting Knockback, and +1 die on \emph{Hide} or \emph{Evasion} rolls made while in cover.
\end{itemize}

Their presence in the world is often underestimated, but their knack for slipping unseen through thresholds and enduring where others falter has earned them a quiet reverence.