
% --- Fate's Edge SRD — Section 4: Talents & Advancement ---
% Include this file from your main .tex with: 
% --- Fate's Edge SRD — Section 4: Talents & Advancement ---
% Include this file from your main .tex with: 
% --- Fate's Edge SRD — Section 4: Talents & Advancement ---
% Include this file from your main .tex with: 
% --- Fate's Edge SRD — Section 4: Talents & Advancement ---
% Include this file from your main .tex with: \input{04-talents.tex}

\section{Talents and Advancement}

\subsection{What are Talents?}
Talents are the building blocks of character specialization. They represent learned techniques, supernatural gifts, or cultural inheritances. 
Each Talent costs XP, and their costs are tied to impact.

\subsection{Talent Costs}
\begin{itemize}
  \item \textbf{2 XP} — Minor edge (e.g., Caster’s Gift, +1 situational bonus, small narrative trick).
  \item \textbf{4 XP} — Major edge (e.g., Patron’s Symbol, a strong summon upgrade, permanent +1 effect in a niche).
  \item \textbf{6+ XP} — Prestige abilities, rare and campaign-defining.
\end{itemize}

\subsection{Gaining Talents}
\begin{itemize}
  \item Spend XP earned through play. 
  \item XP comes from fulfilling Drives, resolving Arcs, trading Boons (2 Boons = 1 XP, max 2 XP/session), and GM awards.
  \item XP is spent between sessions or during downtime.
\end{itemize}

\subsection{Magic Access Through Talents}
\begin{description}[leftmargin=1.5em, style=nextline]
  \item[Caster’s Gift (2 XP):] Grants access to Weave \& Cast freeform spellcasting using the Eight Elements. Without this, characters cannot freeform cast.
  \item[Familiar (2 XP):] Required to access Patron features such as \emph{Patron’s Gift}. Binds a Thiasos.
  \item[Codex (4 XP):] Required to fully join a Patron’s service as a Runekeeper. Grants access to that Patron’s Rites and Obligation system.
  \item[Patron’s Symbol (4 XP):] Minor Asset. Allows an Invoker to access a Patron’s Rites via ritual precision. Each Patron requires its own Symbol.
\end{description}

\subsection{Imbuements}
\textbf{Patron’s Gift (Free, Requires Thiasos)}\\
Duration: Scene; Range: Touch; Stacking: No.\\
Effect: Imbue one item with temporary magical power related to your Patron’s domain. The item functions as a magical weapon (+1 Melee) and specialized tool (+1 thematic Skill) for the scene.\\
Activation: Requires 1 Action once per scene.\\
Push It: The item’s power persists for one additional scene but marks +1 Obligation.\\
Requires: Familiar (Invoke: 1 Boon).

\subsection{Advancement}
\begin{itemize}
  \item Characters advance primarily through acquiring Talents.
  \item Each 8–12 XP represents significant growth.
  \item Characters may retrain (swap out) one Talent per arc with GM approval, if fictionally justified.
  \item Advancement should always reinforce narrative identity: what drives, obligations, and affinities are shaping the character?
\end{itemize}

\subsection{Prestige Abilities}
Prestige abilities are narrative milestones unlocked through mastery or story events. They are priced at 6+ XP and include campaign-shaping effects:
\begin{itemize}
  \item Breaking fundamental limits of casting or rites.
  \item Access to forbidden summons.
  \item Rewriting obligations or reshaping patron bargains.
\end{itemize}

\subsection{Worked Example}
\emph{Sable earns 2 XP from fulfilling a Drive and 1 XP from trading in 2 Boons at session’s end. He now has 4 XP. He buys the Caster’s Gift (2 XP) and saves the other 2 XP toward a Patron’s Symbol. Next downtime, he will be able to invoke Ikasha’s rites through ritual.}

\subsection{Equipment Enchantments}

Equipment enchantments function as Talents, using the same XP costs and mechanical principles. They represent permanent magical modifications to weapons, armor, and gear.

\subsubsection{Core Principles}

\begin{itemize}
\item Enchantments cost XP like Talents (2, 4, 6+ XP for minor, major, prestige effects)
\item Each enchantment provides a specific, limited mechanical benefit
\item Enchantments follow the same stacking and limitation rules as Talents
\item Equipment must be maintained and can become Neglected/Compromised like other assets
\end{itemize}

\subsubsection{Enchantment Categories}

\textbf{Minor Enchantments (2-4 XP):}
\begin{itemize}
\item Provide small, consistent benefits
\item Often situational or single-effect modifiers
\item Examples: +1 die to specific rolls, minor damage resistance
\end{itemize}

\textbf{Major Enchantments (6+ XP):}
\begin{itemize}
\item Significant mechanical advantages
\item May provide new capabilities or action options
\item Examples: Ignore armor, special damage types, area effects
\end{itemize}

\subsubsection{Sample Enchantments}

\textbf{Weapon Enchantments:}
\begin{itemize}
\item \textbf{Keen Edge (2 XP):} +1 die to hit against armored targets
\item \textbf{Flaming Blade (4 XP):} Deals Fire elemental damage, +1 Effect vs cold creatures
\item \textbf{Soulfire Weapon (6 XP):} Ignores 1 point of armor, deals Spirit damage
\item \textbf{Thunder Hammer (8 XP):} On critical hit, knockback adjacent targets (Hazard +2)
\end{itemize}

\textbf{Armor Enchantments:}
\begin{itemize}
\item \textbf{Shadowweave (2 XP):} +1 die to Stealth rolls while moving silently
\item \textbf{Runed Plate (4 XP):} Reduce magical Backlash by 1 SB
\item \textbf{Wraithmail (8 XP):} Once per scene, phase through 1 attack (become intangible)
\end{itemize}

\subsubsection{Enchantment Limitations}

\begin{itemize}
\item Maximum enchantments = Spirit attribute (prevents stacking abuse)
\item Enchantments on same item cannot stack if they provide identical benefits
\item Damaged/Neglected equipment may lose enchantment benefits temporarily
\item Removing enchantments requires ritual (Arcana DV 4) and costs 1 XP per enchantment level
\end{itemize}

\subsubsection{Creating New Enchantments}

Use existing Talents as benchmarks:
\begin{itemize}
\item Minor (2-4 XP): Equivalent to small narrative tricks or situational bonuses
\item Major (6-8 XP): Comparable to significant mechanical edges or new capabilities
\item Prestige (10+ XP): Campaign-defining effects similar to Patron abilities
\end{itemize}

Price enchantments relative to their mechanical impact and campaign power level.


\section{Talents and Advancement}

\subsection{What are Talents?}
Talents are the building blocks of character specialization. They represent learned techniques, supernatural gifts, or cultural inheritances. 
Each Talent costs XP, and their costs are tied to impact.

\subsection{Talent Costs}
\begin{itemize}
  \item \textbf{2 XP} — Minor edge (e.g., Caster’s Gift, +1 situational bonus, small narrative trick).
  \item \textbf{4 XP} — Major edge (e.g., Patron’s Symbol, a strong summon upgrade, permanent +1 effect in a niche).
  \item \textbf{6+ XP} — Prestige abilities, rare and campaign-defining.
\end{itemize}

\subsection{Gaining Talents}
\begin{itemize}
  \item Spend XP earned through play. 
  \item XP comes from fulfilling Drives, resolving Arcs, trading Boons (2 Boons = 1 XP, max 2 XP/session), and GM awards.
  \item XP is spent between sessions or during downtime.
\end{itemize}

\subsection{Magic Access Through Talents}
\begin{description}[leftmargin=1.5em, style=nextline]
  \item[Caster’s Gift (2 XP):] Grants access to Weave \& Cast freeform spellcasting using the Eight Elements. Without this, characters cannot freeform cast.
  \item[Familiar (2 XP):] Required to access Patron features such as \emph{Patron’s Gift}. Binds a Thiasos.
  \item[Codex (4 XP):] Required to fully join a Patron’s service as a Runekeeper. Grants access to that Patron’s Rites and Obligation system.
  \item[Patron’s Symbol (4 XP):] Minor Asset. Allows an Invoker to access a Patron’s Rites via ritual precision. Each Patron requires its own Symbol.
\end{description}

\subsection{Imbuements}
\textbf{Patron’s Gift (Free, Requires Thiasos)}\\
Duration: Scene; Range: Touch; Stacking: No.\\
Effect: Imbue one item with temporary magical power related to your Patron’s domain. The item functions as a magical weapon (+1 Melee) and specialized tool (+1 thematic Skill) for the scene.\\
Activation: Requires 1 Action once per scene.\\
Push It: The item’s power persists for one additional scene but marks +1 Obligation.\\
Requires: Familiar (Invoke: 1 Boon).

\subsection{Advancement}
\begin{itemize}
  \item Characters advance primarily through acquiring Talents.
  \item Each 8–12 XP represents significant growth.
  \item Characters may retrain (swap out) one Talent per arc with GM approval, if fictionally justified.
  \item Advancement should always reinforce narrative identity: what drives, obligations, and affinities are shaping the character?
\end{itemize}

\subsection{Prestige Abilities}
Prestige abilities are narrative milestones unlocked through mastery or story events. They are priced at 6+ XP and include campaign-shaping effects:
\begin{itemize}
  \item Breaking fundamental limits of casting or rites.
  \item Access to forbidden summons.
  \item Rewriting obligations or reshaping patron bargains.
\end{itemize}

\subsection{Worked Example}
\emph{Sable earns 2 XP from fulfilling a Drive and 1 XP from trading in 2 Boons at session’s end. He now has 4 XP. He buys the Caster’s Gift (2 XP) and saves the other 2 XP toward a Patron’s Symbol. Next downtime, he will be able to invoke Ikasha’s rites through ritual.}

\subsection{Equipment Enchantments}

Equipment enchantments function as Talents, using the same XP costs and mechanical principles. They represent permanent magical modifications to weapons, armor, and gear.

\subsubsection{Core Principles}

\begin{itemize}
\item Enchantments cost XP like Talents (2, 4, 6+ XP for minor, major, prestige effects)
\item Each enchantment provides a specific, limited mechanical benefit
\item Enchantments follow the same stacking and limitation rules as Talents
\item Equipment must be maintained and can become Neglected/Compromised like other assets
\end{itemize}

\subsubsection{Enchantment Categories}

\textbf{Minor Enchantments (2-4 XP):}
\begin{itemize}
\item Provide small, consistent benefits
\item Often situational or single-effect modifiers
\item Examples: +1 die to specific rolls, minor damage resistance
\end{itemize}

\textbf{Major Enchantments (6+ XP):}
\begin{itemize}
\item Significant mechanical advantages
\item May provide new capabilities or action options
\item Examples: Ignore armor, special damage types, area effects
\end{itemize}

\subsubsection{Sample Enchantments}

\textbf{Weapon Enchantments:}
\begin{itemize}
\item \textbf{Keen Edge (2 XP):} +1 die to hit against armored targets
\item \textbf{Flaming Blade (4 XP):} Deals Fire elemental damage, +1 Effect vs cold creatures
\item \textbf{Soulfire Weapon (6 XP):} Ignores 1 point of armor, deals Spirit damage
\item \textbf{Thunder Hammer (8 XP):} On critical hit, knockback adjacent targets (Hazard +2)
\end{itemize}

\textbf{Armor Enchantments:}
\begin{itemize}
\item \textbf{Shadowweave (2 XP):} +1 die to Stealth rolls while moving silently
\item \textbf{Runed Plate (4 XP):} Reduce magical Backlash by 1 SB
\item \textbf{Wraithmail (8 XP):} Once per scene, phase through 1 attack (become intangible)
\end{itemize}

\subsubsection{Enchantment Limitations}

\begin{itemize}
\item Maximum enchantments = Spirit attribute (prevents stacking abuse)
\item Enchantments on same item cannot stack if they provide identical benefits
\item Damaged/Neglected equipment may lose enchantment benefits temporarily
\item Removing enchantments requires ritual (Arcana DV 4) and costs 1 XP per enchantment level
\end{itemize}

\subsubsection{Creating New Enchantments}

Use existing Talents as benchmarks:
\begin{itemize}
\item Minor (2-4 XP): Equivalent to small narrative tricks or situational bonuses
\item Major (6-8 XP): Comparable to significant mechanical edges or new capabilities
\item Prestige (10+ XP): Campaign-defining effects similar to Patron abilities
\end{itemize}

Price enchantments relative to their mechanical impact and campaign power level.


\section{Talents and Advancement}

\subsection{What are Talents?}
Talents are the building blocks of character specialization. They represent learned techniques, supernatural gifts, or cultural inheritances. 
Each Talent costs XP, and their costs are tied to impact.

\subsection{Talent Costs}
\begin{itemize}
  \item \textbf{2 XP} — Minor edge (e.g., Caster’s Gift, +1 situational bonus, small narrative trick).
  \item \textbf{4 XP} — Major edge (e.g., Patron’s Symbol, a strong summon upgrade, permanent +1 effect in a niche).
  \item \textbf{6+ XP} — Prestige abilities, rare and campaign-defining.
\end{itemize}

\subsection{Gaining Talents}
\begin{itemize}
  \item Spend XP earned through play. 
  \item XP comes from fulfilling Drives, resolving Arcs, trading Boons (2 Boons = 1 XP, max 2 XP/session), and GM awards.
  \item XP is spent between sessions or during downtime.
\end{itemize}

\subsection{Magic Access Through Talents}
\begin{description}[leftmargin=1.5em, style=nextline]
  \item[Caster’s Gift (2 XP):] Grants access to Weave \& Cast freeform spellcasting using the Eight Elements. Without this, characters cannot freeform cast.
  \item[Familiar (2 XP):] Required to access Patron features such as \emph{Patron’s Gift}. Binds a Thiasos.
  \item[Codex (4 XP):] Required to fully join a Patron’s service as a Runekeeper. Grants access to that Patron’s Rites and Obligation system.
  \item[Patron’s Symbol (4 XP):] Minor Asset. Allows an Invoker to access a Patron’s Rites via ritual precision. Each Patron requires its own Symbol.
\end{description}

\subsection{Imbuements}
\textbf{Patron’s Gift (Free, Requires Thiasos)}\\
Duration: Scene; Range: Touch; Stacking: No.\\
Effect: Imbue one item with temporary magical power related to your Patron’s domain. The item functions as a magical weapon (+1 Melee) and specialized tool (+1 thematic Skill) for the scene.\\
Activation: Requires 1 Action once per scene.\\
Push It: The item’s power persists for one additional scene but marks +1 Obligation.\\
Requires: Familiar (Invoke: 1 Boon).

\subsection{Advancement}
\begin{itemize}
  \item Characters advance primarily through acquiring Talents.
  \item Each 8–12 XP represents significant growth.
  \item Characters may retrain (swap out) one Talent per arc with GM approval, if fictionally justified.
  \item Advancement should always reinforce narrative identity: what drives, obligations, and affinities are shaping the character?
\end{itemize}

\subsection{Prestige Abilities}
Prestige abilities are narrative milestones unlocked through mastery or story events. They are priced at 6+ XP and include campaign-shaping effects:
\begin{itemize}
  \item Breaking fundamental limits of casting or rites.
  \item Access to forbidden summons.
  \item Rewriting obligations or reshaping patron bargains.
\end{itemize}

\subsection{Worked Example}
\emph{Sable earns 2 XP from fulfilling a Drive and 1 XP from trading in 2 Boons at session’s end. He now has 4 XP. He buys the Caster’s Gift (2 XP) and saves the other 2 XP toward a Patron’s Symbol. Next downtime, he will be able to invoke Ikasha’s rites through ritual.}

\subsection{Equipment Enchantments}

Equipment enchantments function as Talents, using the same XP costs and mechanical principles. They represent permanent magical modifications to weapons, armor, and gear.

\subsubsection{Core Principles}

\begin{itemize}
\item Enchantments cost XP like Talents (2, 4, 6+ XP for minor, major, prestige effects)
\item Each enchantment provides a specific, limited mechanical benefit
\item Enchantments follow the same stacking and limitation rules as Talents
\item Equipment must be maintained and can become Neglected/Compromised like other assets
\end{itemize}

\subsubsection{Enchantment Categories}

\textbf{Minor Enchantments (2-4 XP):}
\begin{itemize}
\item Provide small, consistent benefits
\item Often situational or single-effect modifiers
\item Examples: +1 die to specific rolls, minor damage resistance
\end{itemize}

\textbf{Major Enchantments (6+ XP):}
\begin{itemize}
\item Significant mechanical advantages
\item May provide new capabilities or action options
\item Examples: Ignore armor, special damage types, area effects
\end{itemize}

\subsubsection{Sample Enchantments}

\textbf{Weapon Enchantments:}
\begin{itemize}
\item \textbf{Keen Edge (2 XP):} +1 die to hit against armored targets
\item \textbf{Flaming Blade (4 XP):} Deals Fire elemental damage, +1 Effect vs cold creatures
\item \textbf{Soulfire Weapon (6 XP):} Ignores 1 point of armor, deals Spirit damage
\item \textbf{Thunder Hammer (8 XP):} On critical hit, knockback adjacent targets (Hazard +2)
\end{itemize}

\textbf{Armor Enchantments:}
\begin{itemize}
\item \textbf{Shadowweave (2 XP):} +1 die to Stealth rolls while moving silently
\item \textbf{Runed Plate (4 XP):} Reduce magical Backlash by 1 SB
\item \textbf{Wraithmail (8 XP):} Once per scene, phase through 1 attack (become intangible)
\end{itemize}

\subsubsection{Enchantment Limitations}

\begin{itemize}
\item Maximum enchantments = Spirit attribute (prevents stacking abuse)
\item Enchantments on same item cannot stack if they provide identical benefits
\item Damaged/Neglected equipment may lose enchantment benefits temporarily
\item Removing enchantments requires ritual (Arcana DV 4) and costs 1 XP per enchantment level
\end{itemize}

\subsubsection{Creating New Enchantments}

Use existing Talents as benchmarks:
\begin{itemize}
\item Minor (2-4 XP): Equivalent to small narrative tricks or situational bonuses
\item Major (6-8 XP): Comparable to significant mechanical edges or new capabilities
\item Prestige (10+ XP): Campaign-defining effects similar to Patron abilities
\end{itemize}

Price enchantments relative to their mechanical impact and campaign power level.


\section{Talents and Advancement}

\subsection{What are Talents?}
Talents are the building blocks of character specialization. They represent learned techniques, supernatural gifts, or cultural inheritances. 
Each Talent costs XP, and their costs are tied to impact.

\subsection{Talent Costs}
\begin{itemize}
  \item \textbf{2 XP} — Minor edge (e.g., Caster’s Gift, +1 situational bonus, small narrative trick).
  \item \textbf{4 XP} — Major edge (e.g., Patron’s Symbol, a strong summon upgrade, permanent +1 effect in a niche).
  \item \textbf{6+ XP} — Prestige abilities, rare and campaign-defining.
\end{itemize}

\subsection{Gaining Talents}
\begin{itemize}
  \item Spend XP earned through play. 
  \item XP comes from fulfilling Drives, resolving Arcs, trading Boons (2 Boons = 1 XP, max 2 XP/session), and GM awards.
  \item XP is spent between sessions or during downtime.
\end{itemize}

\subsection{Magic Access Through Talents}
\begin{description}[leftmargin=1.5em, style=nextline]
  \item[Caster’s Gift (2 XP):] Grants access to Weave \& Cast freeform spellcasting using the Eight Elements. Without this, characters cannot freeform cast.
  \item[Familiar (2 XP):] Required to access Patron features such as \emph{Patron’s Gift}. Binds a Thiasos.
  \item[Codex (4 XP):] Required to fully join a Patron’s service as a Runekeeper. Grants access to that Patron’s Rites and Obligation system.
  \item[Patron’s Symbol (4 XP):] Minor Asset. Allows an Invoker to access a Patron’s Rites via ritual precision. Each Patron requires its own Symbol.
\end{description}

\subsection{Imbuements}
\textbf{Patron’s Gift (Free, Requires Thiasos)}\\
Duration: Scene; Range: Touch; Stacking: No.\\
Effect: Imbue one item with temporary magical power related to your Patron’s domain. The item functions as a magical weapon (+1 Melee) and specialized tool (+1 thematic Skill) for the scene.\\
Activation: Requires 1 Action once per scene.\\
Push It: The item’s power persists for one additional scene but marks +1 Obligation.\\
Requires: Familiar (Invoke: 1 Boon).

\section{Melee Combat Talents}

\subsection{Minor Talents}

\subsubsection{Defensive Survival (3 XP)}
\textbf{Requirements:} Melee 2+ \\
\textbf{Effect:} +1 die to defense rolls while engaged in melee. Once per scene, convert first Harm 1 from melee to Fatigue. \\
\textbf{Narrative:} Years of combat teaching you to read attacks and flow with them.

\subsubsection{Tactical Movement (4 XP)}
\textbf{Requirements:} Athletics 2+ \\
\textbf{Effect:} Move within engagement zone as Move action (instead of full action). Once per scene, disengage from Close as Move action. \\
\textbf{Narrative:} Footwork and positioning that keeps you alive in the press.

\subsubsection{Conditioning (4 XP)}
\textbf{Requirements:} Body 3+ \\
\textbf{Effect:} Body attribute counts as +1 for Fatigue track calculations. +1 die to resist Fatigue overflow effects. \\
\textbf{Narrative:} Physical conditioning that lets you endure punishment.

\subsubsection{Weapon Master (5 XP)}
\textbf{Requirements:} Melee 2+ \\
\textbf{Effect:} +2 dice (instead of +1) with chosen weapon category. Once per scene, +1 Effect with signature weapon. \\
\textbf{Narrative:} Mastery of specific weapons that makes them extensions of yourself.

\subsection{Major Talents}

\subsubsection{Flurry Strike (7 XP)}
\textbf{Requirements:} Melee 3+, Body 3+ \\
\textbf{Effect:} When engaged with multiple opponents, make 2 attacks as one action. Each attack at -1 die. \\
\textbf{Narrative:} Training that lets you fight multiple enemies simultaneously.

\subsubsection{Duelist's Edge (8 XP)}
\textbf{Requirements:} Melee 3+, Wits 3+ \\
\textbf{Effect:} When engaged with single opponent: +1 die to all melee rolls. Once per scene, ignore first Harm 1 or 2 from that opponent. \\
\textbf{Narrative:} Psychological and tactical dominance in one-on-one combat.

\subsubsection{Battlefield Mastery (8 XP)}
\textbf{Requirements:} Melee 4+, Wits 4+, Command 2+ \\
\textbf{Effect:} Once per scene, when engaged with 3+ opponents, declare "Battlefield Mastery." For next 3 exchanges:
\begin{itemize}
    \item All melee attacks gain +1 Effect
    \item Enemies act at -1 die due to disorientation
    \item Your Position improves by one step
    \item Convert one Harm 1→Fatigue per exchange
\end{itemize}
\textbf{Narrative:} When surrounded, you enter a state of perfect combat flow where enemies become obstacles rather than threats.

\subsection*{Subtle Casting (Major Talent --- 8 XP)}
\index{Talents!Subtle Casting}

\textbf{Requirements:} Lore 3+, Performance 2+ \emph{or} Runekeeper with Codex

\textbf{Effect:} Make a \textbf{Performance + Lore} roll to quietly cast a spell, invoke a Rite, or sing a Cantos against DV (Tier). 
If successful, the casting does not generate \index{Story Beats} on the \emph{Channel} or initial roll. 
This talent allows the caster to veil magic in story, song, or symbol rather than force.

\textbf{Limitations:}
\begin{itemize}
  \item Cannot be used for \emph{Great} or \emph{Extreme} Tier effects.
  \item The \emph{Weave} phase (if applicable) still generates normal SB.
  \item Obvious magical manifestations still occur (glowing sigils, strange sounds, sudden winds, etc.).
\end{itemize}

\subsection{Prestige Talents}

\subsubsection{Battlefield Terror (12 XP)}
\textbf{Requirements:} Melee 4+, Body 4+, Harm 2+ experience \\
\textbf{Effect:} Enemies in Close range act at -1 die due to intimidation. Once per scene, convert enemy's success to partial with cost. \\
\textbf{Narrative:} Reputation and presence that makes opponents hesitate.


\subsection{Epic Talents}

\subsubsection{Blade Dance (18 XP)}
\textbf{Requirements:} Melee 5+, Duelist's Edge, Flurry Strike \\
\textbf{Effect:} Engage and attack up to 3 targets in one action. Each attack at -1 die, but Position improves by one step. \\
\textbf{Narrative:} Legendary skill that makes you a whirlwind of death.


\subsection{Combat Balance Notes}

These talents are designed to enhance melee viability while maintaining Fate's Edge's core tension between risk and reward. Melee combat should remain \textbf{manageably deadly} - dangerous enough to require tactical skill, but with meaningful options for skilled fighters to excel.

\textbf{Key Principles:}
\begin{itemize}
    \item Talents enhance existing mechanics rather than replace them
    \item Specialization provides clear advantages for focused builds
    \item High-cap opponents remain genuinely threatening
    \item Positioning and tactical decision-making remain crucial
    \item Story Beat escalation continues to compound challenges
\end{itemize}

\textbf{Role Balance:} Enhanced melee fighters complement rather than overshadow other roles. Ranged characters maintain mobility advantages, magic users provide battlefield control, and support characters enable team effectiveness.

\subsection{Advancement}
\begin{itemize}
  \item Characters advance primarily through acquiring Talents.
  \item Each 8–12 XP represents significant growth.
  \item Characters may retrain (swap out) one Talent per arc with GM approval, if fictionally justified.
  \item Advancement should always reinforce narrative identity: what drives, obligations, and affinities are shaping the character?
\end{itemize}

\subsection{Prestige Abilities}
Prestige abilities are narrative milestones unlocked through mastery or story events. They are priced at 6+ XP and include campaign-shaping effects:
\begin{itemize}
  \item Breaking fundamental limits of casting or rites.
  \item Access to forbidden summons.
  \item Rewriting obligations or reshaping patron bargains.
\end{itemize}

\subsection{Worked Example}
\emph{Sable earns 2 XP from fulfilling a Drive and 1 XP from trading in 2 Boons at session’s end. He now has 4 XP. He buys the Caster’s Gift (2 XP) and saves the other 2 XP toward a Patron’s Symbol. Next downtime, he will be able to invoke Ikasha’s rites through ritual.}

\subsection{Equipment Enchantments}

Equipment enchantments function as Talents, using the same XP costs and mechanical principles. They represent permanent magical modifications to weapons, armor, and gear.

\subsubsection{Core Principles}

\begin{itemize}
\item Enchantments cost XP like Talents (2, 4, 6+ XP for minor, major, prestige effects)
\item Each enchantment provides a specific, limited mechanical benefit
\item Enchantments follow the same stacking and limitation rules as Talents
\item Equipment must be maintained and can become Neglected/Compromised like other assets
\end{itemize}

\subsubsection{Enchantment Categories}

\textbf{Minor Enchantments (2-4 XP):}
\begin{itemize}
\item Provide small, consistent benefits
\item Often situational or single-effect modifiers
\item Examples: +1 die to specific rolls, minor damage resistance
\end{itemize}

\textbf{Major Enchantments (6+ XP):}
\begin{itemize}
\item Significant mechanical advantages
\item May provide new capabilities or action options
\item Examples: Ignore armor, special damage types, area effects
\end{itemize}

\subsubsection{Sample Enchantments}

\textbf{Weapon Enchantments:}
\begin{itemize}
\item \textbf{Keen Edge (2 XP):} +1 die to hit against armored targets
\item \textbf{Flaming Blade (4 XP):} Deals Fire elemental damage, +1 Effect vs cold creatures
\item \textbf{Soulfire Weapon (6 XP):} Ignores 1 point of armor, deals Spirit damage
\item \textbf{Thunder Hammer (8 XP):} On critical hit, knockback adjacent targets (Hazard +2)
\end{itemize}

\textbf{Armor Enchantments:}
\begin{itemize}
\item \textbf{Shadowweave (2 XP):} +1 die to Stealth rolls while moving silently
\item \textbf{Runed Plate (4 XP):} Reduce magical Backlash by 1 SB
\item \textbf{Wraithmail (8 XP):} Once per scene, phase through 1 attack (become intangible)
\end{itemize}

\subsubsection{Enchantment Limitations}

\begin{itemize}
\item Maximum enchantments = Spirit attribute (prevents stacking abuse)
\item Enchantments on same item cannot stack if they provide identical benefits
\item Damaged/Neglected equipment may lose enchantment benefits temporarily
\item Removing enchantments requires ritual (Arcana DV 4) and costs 1 XP per enchantment level
\end{itemize}

\subsubsection{Creating New Enchantments}

Use existing Talents as benchmarks:
\begin{itemize}
\item Minor (2-4 XP): Equivalent to small narrative tricks or situational bonuses
\item Major (6-8 XP): Comparable to significant mechanical edges or new capabilities
\item Prestige (10+ XP): Campaign-defining effects similar to Patron abilities
\end{itemize}

Price enchantments relative to their mechanical impact and campaign power level.
