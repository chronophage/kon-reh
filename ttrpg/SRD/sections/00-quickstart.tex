\section{Welcome to Fate's Edge}

Fate's Edge is a narrative-first tabletop RPG where every action carries weight, every choice has consequence, and every spell risks backlash. This quickstart guide covers the core rules from the SRD.

\section{Core Resolution}

\subsection{The Art of Consequence}
All significant actions follow this three-step process:
\begin{enumerate}
    \item \textbf{Approach}: Player describes intent and method
    \item \textbf{Execution}: Roll Attribute + Skill d10s (6+ = success, 1 = SB)
    \item \textbf{Outcome}: GM interprets successes vs DV, spends SB for complications
\end{enumerate}

\subsection{Attributes (1-5)}
\begin{itemize}
    \item \textbf{Body}: Strength, endurance, physical action
    \item \textbf{Wits}: Perception, cleverness, reaction speed  
    \item \textbf{Spirit}: Willpower, intuition, resilience
    \item \textbf{Presence}: Charm, command, social force
\end{itemize}

\subsection{Skills (0-5)}
\begin{itemize}
    \item \textbf{Melee}, \textbf{Ranged}, \textbf{Athletics}
    \item \textbf{Sway}, \textbf{Deception}, \textbf{Insight}
    \item \textbf{Stealth}, \textbf{Survival}, \textbf{Command}
    \item \textbf{Arcana}, \textbf{Lore}, \textbf{Craft}
\end{itemize}

\subsection{Difficulty Values (DV)}
\begin{center}
\begin{tabular}{cl}
\toprule
\textbf{DV} & \textbf{Situation} \\
\midrule
2 & Routine: Clear intent, modest stakes \\
3 & Pressured: Time pressure, mild resistance \\
4 & Hard: Hostile conditions, active opposition \\
5+ & Extreme: Multiple constraints, high precision \\
\bottomrule
\end{tabular}
\end{center}

\subsection{Position \& Effect}
\begin{itemize}
    \item \textbf{Position}: Controlled (safe), Risky (default), Desperate (high stakes)
    \item \textbf{Effect}: Limited (weak), Standard (expected), Great (strong)
\end{itemize}

\subsection{Shifting Position in Play}
Position changes as fiction changes:
\begin{itemize}
  \item \textbf{Improve Position}: Smart prep, flanking, leverage, correct tools, or spending a Boon (+1 step).
  \item \textbf{Worsen Position}: Alarms, time pressure, being pinned, darkness, or GM SB spend (–1 step).
\end{itemize}
\begin{examplebox}[Evolving Position]
\textbf{Start}: Controlled (quiet corridor).  
\textbf{After a noisy lockpick (SB=1)}: Risky (guards alerted).  
\textbf{After reinforcements arrive (SB=2)}: Desperate (crossfire, no cover).
\end{examplebox}

\subsection{Outcome Matrix}
\begin{center}
\begin{tabular}{ll}
\toprule
\textbf{Result} & \textbf{What Happens} \\
\midrule
Successes ≥ DV, 0 SB & Clean Success: Intent achieved \\
Successes ≥ DV, 1+ SB & Success \& Cost: Intent + complications \\
0 < Successes < DV & Partial: Progress with complication (gain 1 Boon) \\
Successes = 0 & Miss: No progress (gain 2 Boons) \\
\bottomrule
\end{tabular}
\end{center}

\begin{fatebox}[Story Beats (SB)]
GM spends SB to introduce narrative twists:
\begin{itemize}
    \item \textbf{1 SB}: Minor pressure (noise, trace, +1 Supply)
    \item \textbf{2 SB}: Moderate setback (alarm, lose position, lesser foe)
    \item \textbf{3 SB}: Serious trouble (reinforcements, gear breaks)
    \item \textbf{4+ SB}: Major turn (trap springs, authority arrives)
\end{itemize}
\end{fatebox}

\section{Description Ladder}

How you describe actions affects the roll:
\begin{itemize}
    \item \textbf{Basic}: Roll as-is (all 1s generate SB)
    \item \textbf{Detailed}: Re-roll one die showing 1
    \item \textbf{Intricate}: Re-roll all 1s + add flourish on success
\end{itemize}

\textbf{Note}: Re-rolling 1s doesn't remove their SB; new 1s on re-rolls add more SB.

\section{Boons: Reward for Meaningful Failure}

When you \textbf{Miss} (0 successes) on a significant action with stated stakes, gain \textbf{2 Boons}.\\
When you achieve a \textbf{Partial} (successes < DV but > 0), gain \textbf{1 Boon}.

\subsection{Using Boons}
\begin{itemize}
    \item Re-roll one die in any pool
    \item Activate an on-screen Asset
    \item Improve Position by 1 step
    \item Convert 2 Boons → 1 XP (once per session, max 2 XP)
\end{itemize}

\subsection{Boon Limits}
\begin{itemize}
    \item Hold up to 5 Boons (trim to 2 at scene end)
    \item Max 2 Boons from failure per scene
\end{itemize}

\subsection{Action Economy}
Each character turn consists of:
\begin{itemize}
    \item \textbf{1 Action}: Attack, cast, use a skill, or perform another significant task.
    \item \textbf{1 Move}: Shift one Range Band (Close ↔ Near or Near ↔ Far). 
          A Dash (costs the Action) moves two bands.
\end{itemize}

Some talents or abilities may grant bonus actions, reactions, or allow splitting one action into multiple smaller tasks. 
Channeled magic may require multiple actions across turns, and fatigue or harm may restrict available actions.

\section{Combat System}

Combat uses the same core mechanic with tactical positioning.

\subsection{Combat Procedure}
\begin{enumerate}
    \item Declare action and approach
    \item GM sets Position (Controlled/Risky/Desperate)
    \item Roll Attribute + Skill dice
    \item Count successes vs DV, 1s generate SB
    \item GM resolves outcome and spends SB
\end{enumerate}

\begin{fatebox}[Combat Position]
    In combat, Position reflects immediate danger \emph{this exchange}:
    \begin{itemize}
      \item \textbf{Controlled}: Advantage, cover, tempo.
      \item \textbf{Risky}: Default melee exchange, shifting ground.
      \item \textbf{Desperate}: Outnumbered, flanked, exposed, or under a doom clock.
    \end{itemize}
    GM can shift Position mid-round with SB spends or fictional turns (terrain, reinforcements, hazards).
\end{fatebox}

\subsection{Initiative and Turn Order}

Fate's Edge does not use fixed initiative. 
Turn order follows the fiction and the GM's facilitation:
\begin{itemize}
    \item \textbf{Narrative Fiat:} The GM frames spotlight order based on circumstances, tension, and narrative flow.
    \item \textbf{Player Input:} Players may suggest acting when it makes sense in the fiction. 
    \item \textbf{Surprise:} Ambushers act first; targets respond after the opening exchange.
    \item \textbf{Flexibility:} Spotlight may shift mid-scene if fictionally appropriate (e.g., reacting to a falling ceiling, seizing a moment).
\end{itemize}

This ensures pacing and drama guide the sequence of actions, not rigid turn structures.

\subsection{Fatigue}
\label{subsec:fatigue}
\index{Fatigue}

\textbf{Track:} Each character has a Fatigue track equal to \textbf{Body}. Mark Fatigue for exertion, strain, or backlash.

\textbf{In Play:} Each Fatigue step worsens your \textbf{Position} by one level 
(Controlled $\rightarrow$ Risky $\rightarrow$ Desperate). 
If you are already \textbf{Desperate}, instead apply a \textbf{--1 die} penalty per Fatigue to that roll.

\textbf{Overflow:} When your Fatigue track fills, immediately increase \textbf{Harm by 1 step} and clear all Fatigue to 0. 
If this raises Harm to a level that incapacitates you, you fall out of the scene as normal for Harm.

\textbf{Recovery:} Short rest clears 1--2 Fatigue; a full night's rest clears all Fatigue.

\subsection{Harm System}
\begin{itemize}
    \item \textbf{Level 1}: Minor injury (-1 die to related actions)
    \item \textbf{Level 2}: Serious wound (-1 die to most actions)
    \item \textbf{Level 3}: Critical injury (incapacitated/dying)
\end{itemize}

\subsection{Tactical Clocks}
Track persistent combat conditions:
\begin{itemize}
    \item Mob Overwhelm [6]
    \item Fatigue Spiral [4] 
    \item Morale Collapse [6]
    \item Environmental Collapse [8]
\end{itemize}

\section{Magic System}

\subsection{Three Paths of Magic}
\begin{itemize}
    \item \textbf{Caster (Freeform)}: Weave \& Cast using Eight Elements (requires Caster's Gift)
    \item \textbf{Rites User (Runekeeper)}: Patron-based rituals with Obligation (requires Codex)
    \item \textbf{Invoker (Symbol Path)}: Ritual magic using Patron Symbols
\end{itemize}

\subsection{Magic in Combat}
\begin{itemize}
    \item Casting: Channel + Weave = 2 actions
    \item Rites: 1 action to Invoke (can Push for +1 Obligation)
    \item Invokers: DV + 1 Player Turns for rituals, or Crack the Seal for instant cast
\end{itemize}

\subsection{Examples by Path}
\paragraph{Caster (Freeform).}
\textbf{Weave Fire + Channel Spirit} to lash a flaming whip (DV 3, Risky, Standard Effect).  
\textbf{Weave Stone} to raise cover (DV 2, Controlled, Limited$\rightarrow$Standard with tools).

\paragraph{Runekeeper (Rites).}
\textbf{Bind Sigil}: Restrain a foe in spectral chains (DV = \texttt{max(Obligation -- Spirit, Tier)}; +1 Obligation on Push).  
\textbf{Ward Line}: Draw a boundary spirits cannot cross (as above; Partial = shorter duration).

\paragraph{Invoker (Symbols).}
\textbf{Seal of Storms}: Call a lightning strike (DV 4, Risky; \emph{Crack the Seal} for instant cast but Desperate).  
\textbf{Seal of Veils}: Cloak an area in shimmering concealment (DV 3; Partial = flickers under stress).

\subsection{Disruption \& Push}
\begin{itemize}
  \item \textbf{Interrupted}: If a caster is silenced, disarmed, or \emph{harmed before resolution}, the spell/rite fails.
  \item \textbf{Push}: Resolve now with added risk—mark +1 Fatigue or +1 Obligation/Corruption as appropriate.
\end{itemize}
\begin{examplebox}[Interrupted Casting]
A Cantor begins a hymn (DV 3). Before resolution, an arrow deals Harm 1. The Song is \emph{Interrupted}: treat as Failure and generate SB from any rolled 1s.
\end{examplebox}

\section{Character Creation}

\subsection{Starting Build}
\begin{itemize}
    \item \textbf{30 XP} to spend
    \item Attributes: 1-3, Skills: 0-2
    \item Can take bonds (+2 XP) and complications (+4 XP total)
\end{itemize}

\subsection{Key Talents}
\begin{itemize}
    \item \textbf{Caster's Gift (2 XP)}: Freeform magic access
    \item \textbf{Familiar (2 XP)}: Patron features access
    \item \textbf{Codex (4 XP)}: Runekeeper rites and Obligation
    \item \textbf{Patron's Symbol (4 XP)}: Invoker ritual access
\end{itemize}

\section{Range Bands \& Movement}

\subsection{Range Bands}
\begin{itemize}
    \item \textbf{Close}: Arm's reach, melee combat
    \item \textbf{Near}: Same room/area, quick movement
    \item \textbf{Far}: Distant but same location
    \item \textbf{Absent}: Off-screen, requires travel
\end{itemize}

\subsection{Movement}
\begin{itemize}
    \item 1 Move shifts one band (Close↔Near or Near↔Far)
    \item Dash (action) shifts two bands
    \item Melee Flag: Mark engaged opponents in Near range
\end{itemize}

\section{Travel Framework}

\subsection{Travel Procedure}
\begin{enumerate}
    \item Break journey into legs with Travel Clock [4]
    \item Assign roles: Guide, Scout, Quartermaster, Watch
    \item Advance clock through actions/encounters
    \item Resolve complications when clock fills
\end{enumerate}

\section{Deck of Consequences}

Optional tool for narrative complications:
\begin{itemize}
    \item \textbf{Hearts}: Social/emotional complications
    \item \textbf{Spades}: Physical/violent setbacks  
    \item \textbf{Clubs}: Resource/wealth problems
    \item \textbf{Diamonds}: Mystical/supernatural events
\end{itemize}

\subsection{Deck Triggers \& Suits}
\textbf{Draw 1} when any of the following occurs (max 3 draws/scene):
\begin{itemize}
  \item A \textbf{Desperate} roll succeeds with cost.
  \item The GM spends \textbf{2+ SB at once}.
  \item A major clock fills (scene pivot).
\end{itemize}
\textbf{Suits}: \emph{Spades} = harm/danger, \emph{Hearts} = social/reputation, \emph{Clubs} = resources/fatigue, \emph{Diamonds} = arcane/mystical.

\section{Advancement}

\subsection{XP Awards}
\begin{itemize}
    \item Standard: 6-10 XP per session
    \item Major objectives: +2-4 XP
    \item Discovery: +1-2 XP
    \item Hard choices: +1-2 XP
    \item Milestones: +8-12 XP per arc
\end{itemize}

\subsection{Spending XP}
\begin{itemize}
    \item Attributes: New rating × 3 XP
    \item Skills: New level × 2 XP
    \item Talents: As listed (2-6+ XP)
    \item Followers: Cap² XP
\end{itemize}

\section{GM Quick Reference}

\subsection{Adjudication Loop}
\begin{enumerate}
    \item Player describes intent and approach
    \item Set DV (2-5+) and Position
    \item Roll pool = Attribute + Skill
    \item Count successes (6+) and SB (1s)
    \item Resolve outcome from matrix
    \item Spend SB for complications
\end{enumerate}

\begin{fatebox}[SB Spend: Scene Dials]
    \begin{tabularx}{\linewidth}{>{\bfseries}l X}
    1 SB & Minor pressure: worsen Position \emph{or} tick a 4/6 clock by 1, reveal a soft complication.\\
    2 SB & Moderate setback: remove cover, split the party, tick environmental clock, call a lesser foe.\\
    3 SB & Serious trouble: reinforcements, break a key asset, force a hard choice, escalate Position to Desperate.\\
    4+ SB & Major turn: scene transformation, authority arrival, collapse/eruption, Patron interference.
    \end{tabularx}
    \end{fatebox}

\section{Worked Examples}

\begin{examplebox}[From Controlled to Desperate]
    \textbf{Setup}: Thane (Controlled) charges a hex-brute (DV 3).  
    \textbf{Roll}: 6d10 $\rightarrow$ 10, 9, 7, 6, 3, 1. Successes=5, SB=1.  
    \textbf{Outcome}: Hit lands (Great Effect). GM spends 1 SB: rubble collapses, reducing cover.  
    \textbf{Next Beat}: Lyra’s shot is now \emph{Risky}. Another 2 SB later, reinforcements arrive—Position drops to \emph{Desperate}.
    \end{examplebox}

\begin{examplebox}[Lockpick Under Pressure]
\textbf{Situation}: Picking a lock while guards patrol nearby\\
\textbf{Roll}: Wits 2 + Stealth 2 = 4d10 → {8, 6, 3, 1}\\
\textbf{Result}: 2 successes (≥ DV 2), 1 SB\\
\textbf{Outcome}: Lock opens, GM spends 1 SB: "The lock clicks open but the last tumbler makes a loud snap - guards turn toward the sound."
\end{examplebox}

\begin{examplebox}[Combat Attack]
\textbf{Situation}: Kael attacks cultist with imbued blade\\
\textbf{Roll}: Body 3 + Melee 2 = 5d10 → {9, 7, 5, 2, 1}\\
\textbf{Result}: 3 successes (≥ DV 2), 1 SB\\
\textbf{Outcome}: Cultist defeated, GM spends 1 SB: "Blood sprays across the ritual sigil - energy flares, accelerating the summoning."
\end{examplebox}

\begin{examplebox}[Partial Success]
\textbf{Situation}: Negotiating with a suspicious merchant\\
\textbf{Roll}: Presence 3 + Sway 2 = 5d10 → {7, 6, 4, 3, 1}\\
\textbf{Result}: 2 successes (< DV 3), 1 SB\\
\textbf{Outcome}: Partial success - deal possible but with harsh terms, gain 1 Boon. GM spends 1 SB: "The merchant agrees but demands an additional favor later."
\end{examplebox}

\section{Getting Started}

\subsection{Session 1 Agenda (60–120 minutes fast start)}
\begin{enumerate}[leftmargin=*,label=\arabic*.]
  \item Cold open: a concrete problem with clear stakes (5–10 min).
  \item Teach the core loop with a short obstacle (10–15 min).
  \item Run one travel leg or social scene (15–25 min).
  \item Spotlight combat with clocks and SB spends (25–40 min).
  \item Debrief: XP, Boons, next hook (5–10 min).
\end{enumerate}

\begin{enumerate}
    \item Create characters with 30 XP
    \item GM prepares starting situation using travel framework
    \item Play through scenes using core resolution system
    \item Award XP based on accomplishments and choices
    \item Let consequences drive the narrative forward
\end{enumerate}

\begin{fatebox}[Key Design Principles]
\begin{itemize}
    \item \textbf{Narrative Primacy}: Mechanics serve the story
    \item \textbf{Risk as Drama}: Every roll carries potential cost
    \item \textbf{Meaningful Growth}: Advancement changes characters and world
    \item \textbf{Consequence Weight}: Choices ripple outward
    \item \textbf{Fail Forward}: Misses fuel future opportunities
\end{itemize}
\end{fatebox}

\begin{center}
\textbf{Remember: In Fate's Edge, nothing is free. Every victory has a price,}\\
\textbf{and every choice shapes the world around you.}
\end{center}

\section{Core Resolution}

\subsection{The Art of Consequence}
All significant actions follow this three-step process:
\begin{enumerate}
    \item \textbf{Approach}: Player describes intent and method in narrative terms.
    \item \textbf{Execution}: Roll a pool equal to Attribute + Skill (in d10s). 
          Each die of 6–10 counts as a success. Each die showing 1 generates a Story Beat (SB), 
          which the GM spends to introduce complications.
    \item \textbf{Outcome}: Compare successes against the Difficulty Value (DV). 
          GM interprets results using Position \& Effect, resolving clean success, 
          partial progress, or costly failure.
\end{enumerate}

\subsection{Group Actions}
When multiple PCs attempt the \emph{same} task together, choose a leader:
\begin{itemize}
  \item \textbf{Pool}: Leader rolls normally; up to two allies add +1 die each if fictionally helping.
  \item \textbf{SB}: Every 1 rolled by any participant still generates SB for the GM.
  \item \textbf{Consequence Spread}: On Partial/Miss, the GM may split costs among helpers (harm, position loss, resource ticks).
\end{itemize}

\section{Sample Characters}

These pre-generated characters let you jump into play immediately.

\subsection{Kael, Blade of the Dawn}
\begin{tabularx}{\linewidth}{lX}
\toprule
\textbf{Attributes} & Body 3, Wits 2, Spirit 2, Presence 2 \\
\textbf{Skills} & Melee 2, Athletics 1, Survival 1, Command 1 \\
\textbf{Talents} & Familiar (Patron: Isoka), Caster's Gift \\
\textbf{Notes} & A wandering knight who channels sunlight through his blade. Strong frontline presence. \\
\bottomrule
\end{tabularx}

\subsection{Seris, Whisper of Ash}
\begin{tabularx}{\linewidth}{lX}
\toprule
\textbf{Attributes} & Body 1, Wits 3, Spirit 3, Presence 2 \\
\textbf{Skills} & Stealth 2, Arcana 2, Deception 1, Insight 1 \\
\textbf{Talents} & Codex (Runekeeper Rites), Patron’s Symbol \\
\textbf{Notes} & Shadow-touched rite-user who bargains with forbidden patrons. Plays with risk and consequence. \\
\bottomrule
\end{tabularx}

\subsection{Daro, Lantern-Guide}
\begin{tabularx}{\linewidth}{lX}
\toprule
\textbf{Attributes} & Body 2, Wits 2, Spirit 3, Presence 3 \\
\textbf{Skills} & Sway 2, Insight 2, Lore 2, Craft 1 \\
\textbf{Talents} & Invoker (Patron Symbols), Boon Recovery \\
\textbf{Notes} & A traveling mystic who keeps groups together with charm and ritual light. Excellent support character. \\
\bottomrule
\end{tabularx}

\section{Regional Seeds}

\section{Regional Seeds}

\subsection{Roadside Waystation (d6)}
\begin{tabular}{cl}
\toprule
d6 & Detail \\
\midrule
1 & Overcrowded caravan yard; prices up 50\%.\\
2 & Quiet, watchful guards; rumors of a witch-binder.\\
3 & Shrine with flickering offerings; patron omens.\\
4 & Black-market stall (forbidden components).\\
5 & Hungry refugees; Supplies Clock starts at 2/6.\\
6 & Rival mercs arrived first; Reputation Clock +1.\\
\bottomrule
\end{tabular}

\subsection{Collapsing Ruins (d6)}
\begin{tabular}{cl}
\toprule
d6 & Feature \\
\midrule
1 & Singing stone; echoes worsen Position on a 1.\\
2 & Rune vents; Arcane Backwash [6] starts at 1/6.\\
3 & Knife-thin ledges; Athletics DV +1 while moving.\\
4 & Bound spirit; can bargain (Boon cost) for guidance.\\
5 & Unstable pillars; Environmental Collapse +1/tick.\\
6 & Hidden reliquary; discovery grants 1 XP each.\\
\bottomrule
\end{tabular}

\subsection{Roadside Waystation (d6)}
\begin{tabular}{cl}
\toprule
d6 & Detail \\
\midrule
1 & Overcrowded caravan yard; prices up 50\%.\\
2 & Quiet, watchful guards; rumors of a witch-binder.\\
3 & Shrine with flickering offerings; patron omens.\\
4 & Black-market stall (forbidden components).\\
5 & Hungry refugees; Supplies Clock starts at 2/6.\\
6 & Rival mercs arrived first; Reputation Clock +1.\\
\bottomrule
\end{tabular}

\subsection{Collapsing Ruins (d6)}
\begin{tabular}{cl}
\toprule
d6 & Feature \\
\midrule
1 & Singing stone; echoes worsen Position on a 1.\\
2 & Rune vents; Arcane Backwash [6] starts at 1/6.\\
3 & Knife-thin ledges; Athletics DV +1 while moving.\\
4 & Bound spirit; can bargain (Boon cost) for guidance.\\
5 & Unstable pillars; Environmental Collapse +1/tick.\\
6 & Hidden reliquary; discovery grants 1 XP each.\\
\bottomrule
\end{tabular}

\subsection{Travel Roles \& Beats}
Assign any of the following:
\begin{itemize}
  \item \textbf{Guide}: Sets route; on success, DV –1 for movement challenges.
  \item \textbf{Scout}: Spots hazards; on success, ignore the next 1 SB spend.
  \item \textbf{Quartermaster}: Manages supplies; on success, reduce Supplies Clock by 1.
  \item \textbf{Watch}: Prevents ambush; on success, improve Position to Controlled at scene start.
\end{itemize}
Advance the \textbf{Travel Clock [4]}; on completion, roll a regional generator or trigger a set-piece.

\section{Starter Scenario: The Lantern at Dusk}

This scenario is designed for one session (3--4 hours) with 3--5 players.  
It introduces travel, core resolution, combat, and magic.

\subsection{Premise}
The party is hired to escort a relic lantern across the haunted trade road known as the Duskway. Rumor says that when the lantern is lit, forgotten souls stir. Your Patron, allies, or enemies may all have stakes in whether it arrives intact.

\subsection{Setup}
\begin{itemize}
    \item Party begins at \textbf{Tier I} unless continuing from pregens.
    \item Relic Lantern: counts as an Asset. If broken or stolen, Patron consequences are triggered.
    \item GM prepares a \textbf{Travel Clock [4]} to represent the Duskway journey.
\end{itemize}

\subsection{Encounter Seeds}

\paragraph{1. Crossing the Shattered Bridge}
\begin{itemize}
    \item DV 3 to cross safely (Athletics, Survival, or Craft).
    \item SB Spend: falling stones, reinforcements on far side, time pressure.
    \item On failure: bridge partially collapses, advance Travel Clock +1.
\end{itemize}

\paragraph{2. The Toll of Cinders}
\begin{itemize}
    \item Bandits demand toll: 4 SB already banked by GM.
    \item Negotiation (Presence + Sway/Deception) or Combat.
    \item Success = pass unharmed; Partial = pass but lose 1 valuable item; Miss = fight breaks out.
\end{itemize}

\paragraph{3. Lantern Awakens}
\begin{itemize}
    \item When Travel Clock fills, lantern flares with ghostlight.
    \item PCs face \textbf{Restless Dead Mob [6-Clock]}.
    \item SB Spend: Lantern leaks corruption; terrain becomes treacherous; Patron stirs.
\end{itemize}

\subsection{Scene Clocks}
\begin{itemize}
    \item \textbf{Travel Clock [4]}: Each leg of the journey. Complications on completion.
    \item \textbf{Lantern Integrity [6]}: Advances whenever lantern is damaged, corrupted, or stolen.
    \item \textbf{Bandit Morale [4]}: Collapse if filled by player pressure or SB spend.
\end{itemize}

\subsection{Possible Outcomes}
\begin{itemize}
    \item \textbf{Success}: Relic Lantern delivered. Each PC gains +6 XP, Patron’s favor, and 1 new contact.
    \item \textbf{Mixed}: Lantern arrives damaged or corrupted. PCs gain XP but also an ongoing Obligation to a Patron.
    \item \textbf{Failure}: Lantern lost. PCs earn only +2 XP, and GM introduces a new Patron Foe in the next session.
\end{itemize}

\subsection{Scaling Notes}
\begin{itemize}
    \item At Tier III or IV, increase DV by +1 and add an additional Mob clock.
    \item Higher-tier parties should face rival Patron agents instead of common bandits.
\end{itemize}


\subsection{Travel Roles \& Beats}
Assign any of the following:
\begin{itemize}
  \item \textbf{Guide}: Sets route; on success, DV –1 for movement challenges.
  \item \textbf{Scout}: Spots hazards; on success, ignore the next 1 SB spend.
  \item \textbf{Quartermaster}: Manages supplies; on success, reduce Supplies Clock by 1.
  \item \textbf{Watch}: Prevents ambush; on success, improve Position to Controlled at scene start.
\end{itemize}
Advance the \textbf{Travel Clock [4]}; on completion, roll a regional generator or trigger a set-piece.
\section{Quick GM Toolkit}

This section provides essential tools for running Fate's Edge in one-shots or campaigns.

\subsection{Session Pacing}
\begin{itemize}
    \item \textbf{Opening Hook (15--20 minutes)}: Establish stakes, characters, and first challenge.
    \item \textbf{Rising Action (60--90 minutes)}: Build tension through Travel, Social, or Combat encounters.
    \item \textbf{Climax (45--60 minutes)}: Present a high-stakes scene where SB economy and Position matter most.
    \item \textbf{Resolution (15 minutes)}: Award XP, check Boons, note Patron consequences.
\end{itemize}

\subsection{The GM Loop}
\begin{enumerate}
    \item Set Position (Controlled / Risky / Desperate) and DV (2--5+).
    \item Roll Attribute + Skill; 6+ counts as a success, 1s generate SB.
    \item Check outcome matrix (Success, Success+Cost, Partial, Miss).
    \item Spend SB to escalate the scene or introduce complications.
    \item Advance clocks, adjust fiction, keep momentum moving.
\end{enumerate}

\subsection{SB Economy in Play}
\begin{itemize}
    \item \textbf{Start with 0 SB.} Build tension by banking SB from rolls.
    \item \textbf{Spend early and often.} Players should feel pressure from minor setbacks before the climax.
    \item \textbf{Escalate deliberately.} Move from 1 SB annoyances (noise, broken gear) toward 3+ SB scene-changers.
    \item \textbf{Never hoard SB.} Idle SB deflates tension; active SB drives story forward.
\end{itemize}

\begin{fatebox}[GM as Facilitator]
    Your job isn’t to defeat the players—it’s to \emph{spend SB to keep the story moving}.  
    State stakes, set Position and DV, honor player creativity, and let consequences reframe scenes rather than stall them.
    \end{fatebox}

\subsection{Using Clocks}
Clocks are visual timers representing threats, conditions, or long-term stakes.
\begin{itemize}
    \item \textbf{4-Clock}: Short obstacle (guard patrol, quick chase).
    \item \textbf{6-Clock}: Standard challenge (mob fight, corruption spread).
    \item \textbf{8-Clock}: Major arc or environmental shift (siege, collapsing fortress).
\end{itemize}
Mark slices as progress is made or SB triggers complications.

\subsection{Framing Consequences}
When spending SB, consider four categories:
\begin{itemize}
    \item \textbf{Harm}: Injuries, fatigue, or resource drain.
    \item \textbf{Position}: Shift from Controlled → Risky → Desperate.
    \item \textbf{Resources}: Supplies, wealth, or equipment lost.
    \item \textbf{Fictional Twist}: Patron interference, rival arrival, mystical surge.
\end{itemize}

\subsection{GM Principles}
\begin{itemize}
    \item \textbf{Consequence First}: Always tie outcomes back to the fiction.
    \item \textbf{Escalate, Don’t Stall}: Failure should push story forward, not shut it down.
    \item \textbf{Respect Player Creativity}: Reward intricate descriptions with rerolls and narrative weight.
    \item \textbf{Patrons Matter}: Use them as ever-present forces shaping the world and complicating choices.
\end{itemize}

\subsection{Improvisation Tools}
\begin{itemize}
    \item \textbf{Deck of Consequences}: Draw a suit to inspire a complication type.
    \item \textbf{Yes, But…}: On partial success, grant progress with a cost.
    \item \textbf{No, And…}: On a miss, deny intent and introduce new pressure.
    \item \textbf{Flashbacks}: Allow players to spend Boons or XP to retroactively prepare for a twist.
\end{itemize}

\subsection{XP Reminders}
At session end, award XP for:
\begin{itemize}
    \item Meaningful risks taken.
    \item Discovery of new truths or places.
    \item Advancement of character Bonds or Complications.
    \item Defining victories or defeats tied to Patrons.
\end{itemize}

\section{What's Next?}

You now have everything needed to run Fate's Edge with your friends. This Quickstart is intentionally light, focusing on the core loop and essential rules. For more depth:

\begin{itemize}
    \item Explore the \textbf{Full SRD} for expanded rules, advanced talents, and detailed GM advice.
    \item Try a short arc using the \textbf{Travel Framework} and let consequences ripple outward.
    \item Experiment with \textbf{Patron play}, integrating Rites, Obligation, and Symbol Paths into long-term campaigns.
\end{itemize}

\begin{center}
\textit{Remember: Every choice carries weight. Every roll is a step toward destiny or downfall.}
\end{center}

\section{Designer’s Notes}

Fate's Edge was built around these guiding ideas:
\begin{enumerate}
    \item \textbf{Fail Forward}: Failure is story fuel, not a dead end.
    \item \textbf{Fiction First}: The dice matter, but description always comes before resolution.
    \item \textbf{Consequences Drive Drama}: Risk creates tension, tension creates story.
    \item \textbf{Player Agency}: Characters reshape the world through their actions, bargains, and sacrifices.
\end{enumerate}

The Quickstart emphasizes a \emph{learn by play} approach. Rather than memorizing charts, lean on the Outcome Matrix, SB spends, and clocks as your core tools.

\section{License and Use}

This work is released under the \textbf{Creative Commons Attribution 4.0 International License (CC BY 4.0)}.  

You are free to:
\begin{itemize}
    \item Share — copy and redistribute the material in any medium or format
    \item Adapt — remix, transform, and build upon the material for any purpose
\end{itemize}

Under the following terms:
\begin{itemize}
    \item Attribution — You must give appropriate credit, provide a link to the license, and indicate if changes were made.
\end{itemize}

For full details, visit: \url{https://creativecommons.org/licenses/by/4.0/}

\begin{center}
\textbf{Fate's Edge Quickstart Guide}\\
© 2025 by Nick Gasper and collaborators.\\
Released under CC BY 4.0.\\[1em]
\textit{In Fate's Edge, nothing is free — and every choice shapes the world.}
\end{center}

\appendix

\section*{Quick GM Reference}

\begin{tabularx}{\linewidth}{|l|X|}
\hline
\textbf{Position} & Dominant (safe edge), Controlled (typical), Desperate (high risk / high reward) \\
\hline
\textbf{Effect} & Limited (partial impact), Standard (full), Great (above expectation) \\
\hline
\textbf{Fatigue} & Each Fatigue adds a \emph{re-roll} on successes. At 4+, start marking Harm. \\
\hline
\textbf{Harm} & Harm 1 = Minor (narrative drag). Harm 2 = Moderate (roll penalties). Harm 3 = Severe (loss of Action type). Harm 4 = Incapacitated. \\
\hline
\textbf{Story Beats (SB)} & GM spends 1–3 SB for Complications: clock advances, enemy reinforcements, resource drain, position shift. \\
\hline
\textbf{Deck of Consequences} & Spades = Harm, Hearts = Social, Clubs = Resource, Diamonds = Arcane. Draw on critical failure/Desperate fallout. \\
\hline
\end{tabularx}

\section*{Example Rites}

\subsubsection*{Ward of Ash (Runekeeper)}
\textbf{Cost:} 1 Fatigue.
\textbf{Effect:} Allies gain +1 Position against environmental hazards for the scene.
\textbf{Risk:} If disrupted (Harm 2+), the ward collapses.

\subsubsection*{Cantor’s Lament (Invoker)}
\textbf{Cost:} 2 Fatigue, 1 SB to GM.
\textbf{Effect:} Create a Desperate push: target must re-roll their highest die.
\textbf{Risk:} Adds corruption clock tick.

\subsubsection*{Flame-Bind (Invoker/Hybrid)}
\textbf{Cost:} 1 Fatigue.
\textbf{Effect:} Target restrained until they spend an Action to break free.
\textbf{Risk:} On 1, flame spreads—Deck draw (Clubs).

\section*{Regional Flavor (Quick d6)}

\subsubsection*{Duskway Caravan Route}
1. Lantern-lit rest point, but half the supplies are spoiled.
2. Patrol demands a toll “for your safety.”
3. Sky full of storm-hawks—ominous omen.
4. Collapsed bridge, reroute through marsh.
5. A rival caravan overtakes you, tensions rise.
6. Abandoned shrine still warm from use.

\subsubsection*{Sunken Archive Approaches}
1. Drowned staircases leading nowhere.
2. Statue’s eyes weep brine that whispers secrets.
3. Crabs dragging torn pages into holes.
4. Broken ward runes flash at random.
5. Submerged corridor hides skeletal guardians.
6. Vault door pulses with faint light.

\section*{Sample Characters (30 XP)}

\subsubsection*{Thane, the Shield-Bearer}
Body 4, heavy armor, Inspire 1, Harm soak.
\textbf{Edge:} Can absorb Harm to raise Position for allies.

\subsubsection*{Kestra, the Voice}
Body 2, Inspire 3, Boon support.
\textbf{Edge:} Generates Boons when allies fail nearby.

\subsubsection*{Lyra, the Scout}
Body 3, ranged specialist, Mobility 2.
\textbf{Edge:} +1 Effect when striking from new Position.

\subsubsection*{Cael, the Cantor}
Body 2, Invocation 3.
\textbf{Edge:} Gains a free Boon when pushing corruption clock.
