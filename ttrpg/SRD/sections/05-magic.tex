
% --- Fate's Edge SRD — Section 5: Magic System ---
% Include this file from your main .tex with: 
% --- Fate's Edge SRD — Section 5: Magic System ---
% Include this file from your main .tex with: 
% --- Fate's Edge SRD — Section 5: Magic System ---
% Include this file from your main .tex with: 
% --- Fate's Edge SRD — Section 5: Magic System ---
% Include this file from your main .tex with: \input{05-magic.tex}

\section{The Magic System}

Magic in Fate's Edge is expressed through three interconnected paths. 
You may specialize in one, or mix them at greater bookkeeping cost. 
All paths share the same dice engine and SB/Obligation economies, but their flavor and risks differ.

\subsection{Three Faces of Magic}
\begin{description}[leftmargin=1.5em, style=nextline]
  \item[Caster (Freeform):] Requires \textbf{Talent: Caster’s Gift (2 XP)}. Grants access to Weave \& Cast using the Eight Elements. Flexible, creative, and risky (Backlash on 1s). 
  \item[Rites User (Runekeeper):] Requires \textbf{Patron + Thiasos (Familiar)}. Grants access to a Patron’s Rites. Structured, powerful, but debt-driven through \textbf{Obligation}.
  \item[Invoker (Symbol Path):] Requires one or more \textbf{Patron’s Symbols (4 XP each)}. Grants access to that Patron’s Rites via rituals. Safe but slow; can \emph{Crack the Seal} to cast instantly at steep Obligation cost.
\end{description}

\subsection{Casting (Freeform)}
\paragraph{Weave \& Cast}
Casters describe the effect in terms of the Eight Elements (Earth, Fire, Air, Water, Fate, Life, Luck, Death). The GM sets DV and Effect based on scope. 

\begin{itemize}
  \item \textbf{Weave:} Player builds dice pool and rolls. On success, they stabilize the spell’s form.
  \item \textbf{Cast:} A second roll channels the effect into the world. 
  \item \textbf{Backlash:} Any 1 rolled may cause narrative backlash related to the Element.
\end{itemize}

\paragraph{Limits}
Casters can attempt any effect that can be described, but the larger the scope, the higher the DV. Improvisation is costly; reliable effects require repeated use and narrative justification.

\subsection{Rites Users (Runekeepers)}
\paragraph{Requirements}
A Patron bond, a Thiasos (Familiar; small spirit in the form of a creature/construct), and a Codex (4 XP) mark a character as a Runekeeper. 

\paragraph{Invocation}
\begin{itemize}
  \item \textbf{Action Cost:} Invoking a Rite requires 1 Action. 
  \item \textbf{Obligation:} Each Rite used marks Obligation on its clock. 
  \item \textbf{Push It:} Once per Rite, you may Push to increase its duration or potency by +1 step at the cost of +1 Obligation.
\end{itemize}

\paragraph{Obligation Clock}
Tracks the Patron’s claim. When full, the GM resolves the debt in-fiction. Obligation is reduced through service or downtime actions.

\subsection{Rites Difficulty Value}
\label{sec:rites-dv-expanded}
\index{Rites!Difficulty Value}
\index{DV}

The Difficulty Value (DV) to cast a Rite is:

\[
\text{DV} = \max\!\big(\text{Obligation Cost} - \text{Spirit}, \, \text{Tier}\big)
\]

\begin{description}
  \item[Obligation Cost:] The Rite’s listed cost in Obligation segments. This reflects the Patron’s toll for the magic.
  \item[Spirit:] The caster’s Spirit attribute. Each point reduces the effective weight of the Obligation, representing inner resilience and willpower.
  \item[Tier:] The Rite’s intrinsic difficulty based on scope or potency. DV can never fall below this floor.
\end{description}

\subsection{Invokers (Symbol Path)}
\paragraph{Patron’s Symbol}
\begin{itemize}
  \item \textbf{Minor Asset, 4 XP each.}
  \item Each Symbol is consecrated to one Patron and grants ritual access to that Patron’s Rites. 
  \item You may hold multiple Symbols, one per Patron.
\end{itemize}

\paragraph{Rite Invocation via Symbol}
\begin{itemize}
  \item \textbf{Time.} Invoking a Rite via Symbol takes \(\text{DV} + 1\) rounds.
  \item \textbf{Obligation.} On completion, mark +1 Obligation (in addition to any listed Rite costs, if applicable).
  \item \textbf{No Push.} Invoker Rites cannot use \emph{Push It} benefits.
  \item \textbf{Symbol Display.} The Symbol must remain visible throughout the invocation.
  \item \textbf{Materials.} Symbols replace any Thaisos and Codex requirements.
\end{itemize}

\paragraph{Crack the Seal (Instant Cast)}
As part of an Invoker Rite, you may immediately resolve the effect by setting the Symbol to \emph{Compromised} and marking +2 Obligation segments (+3 if High-Power). The GM may spend 1 SB on-theme. The Symbol remains but must be restored in downtime.

\paragraph{Restoring Symbols}
A Compromised Symbol is inert until repaired. Use a downtime action and test (DV 3 or fiction-appropriate). Success restores it; a shaky result leaves it Neglected (rituals work but cost +1 Obligation).

\paragraph{Invoker Path Limitations}
\begin{itemize}
  \item Cannot Push. 
  \item Max simultaneous rituals = Spirit. Starting a new ritual ends the oldest or adds +1 Obligation to it. 
  \item Carrying 4+ Symbols causes interference: the first ritual each scene marks +1 extra Obligation.
\end{itemize}

\subsection{Patron’s Gift (Imbuements)}
\textbf{Patron’s Gift (Free, Requires Thiasos)}\\
Duration: Scene; Range: Touch; Stacking: No.\\
Effect: Imbue one item with +1 Weapon (Melee) and +1 Thematic Skill (Patron domain) for the scene.\\
Activation: Requires 1 Action once per scene.\\
Push It: The item’s power persists for one additional scene but marks +1 Obligation.\\
Requires: Familiar (Invoke: 1 Boon).

\subsection{Mixing the Paths}
Players may combine Casting, Rites, and Invoking, but each path introduces its own bookkeeping:
\begin{itemize}
  \item Casters track Backlash. 
  \item Rites users track Obligation. 
  \item Invokers track Symbol states (Maintained, Neglected, Compromised). 
\end{itemize}
Mixing provides flexibility but less efficiency than specialization. Specialists gain stronger benefits, while mixers gain narrative breadth.

% !TEX root = srd_main.tex
% SRD Insert: Elemental Backlash (Condensed SRD Version)

\section{Elemental Backlash (Condensed)}\label{sec:backlash-condensed}
\index{Backlash}\index{Elements}\index{Story Beats}

Magic unsettles the weave. Each element (and its counterpart) carries a distinct backlash pattern. When a roll shows a 1 (generating a (SB)), or when a player accepts a (SB) to escalate, apply a \textbf{Minor} backlash. Players may opt to escalate to \textbf{Major} by taking +1 (SB).

\begin{table}[h]
\centering
\caption{Backlash at a Glance}
\label{tab:backlash-condensed}
\renewcommand{\arraystretch}{1.12}
\begin{tabularx}{\linewidth}{>{\bfseries}l >{\raggedright}X >{\raggedright}X}
\toprule
Element & Minor Backlash & Major Backlash \
\midrule
Earth / Fate & Slips, binds, encumbrance • –1 Position or \textsc{Encumbered}. & Fissure, entrapment • Clock +1 (Collapse) or \textsc{Pinned}. \
Fire / Life & Smoke, sparks, heat • –1 Effect or \textsc{Singed}. & Blaze, fever, ignition • Clock +1 (Fire) or 1 Harm. \
Air / Luck & Scatter, misheard words • –1 Position or Clock +1/2 (Attention). & Unlikely mishap • Lose a tool/use or (SB) +1. \
Water / Dreams (Obishaal) & Slippery tide, slow gear • –1 Effect or \textsc{Waterlogged}. & Undertow, veering path • Clock +1 (Flood) or intrusion from Ways Between. \
Fate / Earth & Probability resists • –1 Effect or Clock +1/2 (Inevitable). & Demand arrives • Immediate sacrifice or (SB) +1 & mark \textsc{Omen}. \
Life / Fire & Growth surge, vines tether • –1 Effect or \textsc{Overgrowth}. & Riot of life • Clock +1 (Biohazard) or convert healing to (SB) +1. \
Luck / Air & Odds flip • –1 Position or Clock +1/2 (Coincidence). & Catastrophic fluke • Force re-roll; if any 1, (SB) +1 and Minor repeats. \
Death / Water (Obishaal) & Whispers, chill • \textsc{Shaken} or Clock +1/2 (Haunting). & Threshold opens • Clock +1 (Crossing Due) or revenant intrusion. \
\bottomrule
\end{tabularx}
\end{table}

\begin{tcolorbox}[title={Cheatsheet},colback=gray!5,colframe=black]
\small Minor = wobble; Major = lurch. Apply once per cast. Offer players the option to escalate to Major by taking (SB) +1.\par\smallskip
Earth/Fate binds; Fire/Life burns or grows; Air/Luck scatters or flips; Water/Obishaal pulls or opens.\end{tcolorbox}
% !TEX root = srd_main.tex
% SRD Insert: Universal Rituals — Quick-Start Spread (SRD concise)
% Assumes: booktabs, tabularx, xcolor, tcolorbox, index set up

\section{Universal Rituals (Quick-Start)}\label{sec:universal-rituals}
\index{Rituals}\index{Story Beats}\index{Backlash}\index{Realms}\index{Obishaal@Obishaal}

These table-ready rituals are system-agnostic and available to any chassis that can perform rituals. Each lists \textbf{Cast Time}, \textbf{Setup/Components}, \textbf{Effect}, and explicit \textbf{Costs/(SB) hooks}. GMs should reskin names freely to match patrons, runes, symbols, or tag-sets.

\begin{tcolorbox}[title={Ritual Casting Basics},colback=gray!5,colframe=black]
\textbf{Triggering Risk.} On any ritual roll showing a 1, gain a (SB) and apply elemental Backlash (\S\ref{sec:backlash-condensed}). Players may accept +1 (SB) to push an effect one step (position/effect/scale) if fictionally supported.\index{Story Beats!rituals}
\end{tcolorbox}

% –––––––––– TABLE ––––––––––
\begin{table}[h]
\centering
\caption{Rituals at a Glance}
\label{tab:rituals-quick}
\renewcommand{\arraystretch}{1.12}
\begin{tabularx}{\linewidth}{>{\bfseries}l c l X X}
\toprule
Name & Tier & Cast Time & Setup & Components & Effect (with Costs/(SB) Hooks) \
\midrule
Wayfinder’s Thread & Low & 1 minute & Red cord knotted thrice; whisper a destination. & Create a faint tether toward the nearest safe path. \emph{Cost:} mark \textsc{Fatigue} if used more than once/scene. \emph{Push:} +1 (SB) to reveal a hidden shortcut (Clock –1/2 on Travel). \
Oath-Ward & Low & 5 minutes & Chalk circle; sworn phrase all participants repeat. & Ward a small area vs. intrusion (mundane/lesser). \emph{Cost:} requires sincere oath; breaking it triggers (SB) +1 and ends ward. \
Ember-Glass & Low & 1 minute & Hold an ember behind smoked glass. & Sense nearby heat sources/life signs through cover. \emph{Cost:} lose one use of a tinder/torch. \emph{Push:} +1 (SB) to pierce thin walls. \
Salt-Cut & Low & 1 minute & Salt line and bronze knife. & Sever a simple ongoing effect (rope-binds, minor charm). \emph{Cost:} consume 1 use of salt. \emph{Push:} +1 (SB) to cut a tougher link (Clock –1/2 on Restraint/Hex). \
River’s Memory & Med & 10 minutes & Bowl of water and a personal token. & Scry a recent passage/event tied to the token, brief and blurry. \emph{Cost:} token is waterlogged/ruined. \emph{Push:} +1 (SB) for a clearer second image. \
Bargain-Bead & Med & 10 minutes & Two carved beads; one is offered openly. & Invite a nearby power/spirit to parley. \emph{Cost:} give up a valuable concession now or take (SB) +1 when you refuse. \
Quiet Veil & Med & 5 minutes & Ash across lips; bell muted in cloth. & Muffle a group’s sound and scent for a scene. \emph{Cost:} \textsc{Muted} Condition (social checks –1) until scene ends. \emph{Backlash:} Air/Luck. \
Shadow-Loom & Med & 5 minutes & Three pins; weave ambient shadow between them. & Create light-obscuring cover or misdirection in a small zone. \emph{Cost:} dim your own vision (–1 precision) while maintained. \emph{Push:} +1 (SB) to mirror a decoy image briefly. \
Dream-Way Marker & Med & 10 minutes & Sleep mask inked with a circle; water drip cadence. & Mark a safe entrance to the Ways Between; next sleep at site allows short transit. \emph{Cost:} all participants mark \textsc{Shaken} on waking. \emph{Backlash:} Death/Obishaal. \
Purge & Med & 10 minutes & Smoke of bitter herbs; clean blade drawn across incense. & Cleanse taint/disease/curse one step. \emph{Cost:} cleanse passes a lesser echo to the caster (–1 to a related action next scene). \emph{Push:} +1 (SB) to remove two steps but take \textsc{Weakened}. \
Fortune-Braid & High & 15 minutes & Three strands (hair, thread, wire) braided tight. & Bank a single lucky break: replace one die with its highest result this scene. \emph{Cost:} immediately take (SB) +1 if used offensively. \emph{Backlash:} Air/Luck. \
Fate-Splice & High & 15 minutes & Knot two names written on vellum. & Temporarily link two fates: transfer a single consequence/boon between them. \emph{Cost:} both bear a subtle mark until dawn; \emph{Push:} +1 (SB) to redirect a Major consequence. \emph{Backlash:} Fate/Earth. \
Summoner’s Gate & High & 20 minutes & Circle inscribed with true-name sigil or emblem. & Call a known entity safely; on success it arrives bound by a simple charge. \emph{Cost:} occupies one concurrency slot; breaking terms creates (SB) +1 and Disruption. \emph{Backlash:} varies by entity. \
\bottomrule
\end{tabularx}
\end{table}

\subsection*{Usage Notes}\label{subsec:ritual-usage}
\begin{itemize}
\item \textbf{Scaling.} Effects scale by position/effect/area via explicit (SB) offers or extra time/components.\index{Rituals!scaling}
\item \textbf{Elements.} Choose the dominant element by fiction (Fire for Ember-Glass; Water/Obishaal for Dream-Way) and apply the condensed backlash table (\S\ref{sec:backlash-condensed}).\index{Backlash!elements}
\item \textbf{Teamwork.} Extra participants can donate narrative components to reduce cast time \emph{or} to accept (SB) on the caster’s behalf once per ritual.\index{Teamwork}
\end{itemize}

\begin{tcolorbox}[title={Design Intent},colback=gray!5,colframe=black]
Each ritual bakes in a crisp \emph{cost}, a tempting \emph{push}, and a likely \emph{backlash}. Keep it fiction-first: components are story handles the GM can threaten, not bookkeeping chores.\end{tcolorbox}

\section{Talent: Cantor's Path --- ``Songs of the Low Rites''}
\label{talent:cantors-path}

\begin{tcolorbox}[colback=black!3,colframe=black!40!white,title={Cantor's Path}]
You echo the liturgies of Patrons through breath and string. Not a sworn celebrant but a perilous mimic, you weave Low Rites into song. It is slower, riskier, and beautiful---but never free.
\end{tcolorbox}

\paragraph*{Type} Major Talent (15 XP) \quad
\paragraph*{Prerequisites} \textbf{Lore 1+}, \textbf{Performance 2+}, \textbf{Presence 2+} \quad
\paragraph*{Access} Any character (does not require Thiasos membership).

\subsection*{Effect}
You may learn and perform \textbf{Low Rites as Songs}. Each Song counts as knowing the associated Low Rite for performance purposes only.

\begin{itemize}
  \item \textbf{Casting Test:} \emph{Lore + Performance vs.\ DV} (default DV = 2--3).
  \item \textbf{Action Economy:} \emph{1 action to begin;} Song \emph{resolves at the start of your next turn} unless accelerated.
  \item \textbf{Scope:} \emph{Low Rites only.} Standard/High Rites remain exclusive to Patrons and Thiasos initiates.
  \item \textbf{Costs:} Pay any \emph{materials} listed. On success you do \emph{not} mark Obligation.
\end{itemize}

\subsection*{Corruption Clock}
\begin{itemize}
  \item You gain a personal \textbf{Corruption Clock} equal in segments to your \textbf{Body} rating.
  \item Each time you cast a Song or whenever the Keeper spends a Story Beat involving you, mark +1 segment.
  \item When the Clock fills:
    \begin{itemize}
      \item You immediately gain a trait of corruption from the \textbf{last Patron} whose Rite you performed.
      \item All of your followers, retainers, or familiars also gain a trait of the same corruption (NPCs manifest visibly unsettling traits).
      \item Reset the Clock to empty.
    \end{itemize}
  \item Corruption traits gained in this way fade at the next \indexterm{Downtime}, unless reinforced by further Patron influence.
\end{itemize}

\subsection*{Outcomes}
\begin{description}
\item[Success:] The Low Rite takes effect as written.
\item[Partial:] The Rite manifests with reduced effect (–1 step) or shortened duration. Mark \textbf{Fatigue 1}.
\item[Failure:] No effect; mark \textbf{Fatigue 1} and the Keeper gains \textbf{+1 SB (Hearts)}.
\item[Interrupted:] Harm, Silence, or disruption before resolution = treat as Failure.
\end{description}

\subsection*{Push It}
When you Push:
\begin{itemize}
  \item Song resolves immediately instead of next round.
  \item Mark \textbf{Fatigue 1}.
  \item Add +1 to your \textbf{Corruption Clock}.
  \item Keeper immediately triggers a \textbf{Story Beat}, representing fallout from a Patron, the Road, or social attention.
\end{itemize}

\subsection*{Limits \& Interactions}
\begin{itemize}
  \item \textbf{Stacking:} Cannot benefit from the same Rite twice.
  \item \textbf{Visibility:} Songs are inherently noticeable. On Failure or Push, assume observers take note.
  \item \textbf{Silence/Disruption:} Impose –1 to –3 dice at Keeper’s discretion.
\end{itemize}

\section{The Magic System}

Magic in Fate's Edge is expressed through three interconnected paths. 
You may specialize in one, or mix them at greater bookkeeping cost. 
All paths share the same dice engine and SB/Obligation economies, but their flavor and risks differ.

\subsection{Three Faces of Magic}
\begin{description}[leftmargin=1.5em, style=nextline]
  \item[Caster (Freeform):] Requires \textbf{Talent: Caster’s Gift (2 XP)}. Grants access to Weave \& Cast using the Eight Elements. Flexible, creative, and risky (Backlash on 1s). 
  \item[Rites User (Runekeeper):] Requires \textbf{Patron + Thiasos (Familiar)}. Grants access to a Patron’s Rites. Structured, powerful, but debt-driven through \textbf{Obligation}.
  \item[Invoker (Symbol Path):] Requires one or more \textbf{Patron’s Symbols (4 XP each)}. Grants access to that Patron’s Rites via rituals. Safe but slow; can \emph{Crack the Seal} to cast instantly at steep Obligation cost.
\end{description}

\subsection{Casting (Freeform)}
\paragraph{Weave \& Cast}
Casters describe the effect in terms of the Eight Elements (Earth, Fire, Air, Water, Fate, Life, Luck, Death). The GM sets DV and Effect based on scope. 

\begin{itemize}
  \item \textbf{Weave:} Player builds dice pool and rolls. On success, they stabilize the spell’s form.
  \item \textbf{Cast:} A second roll channels the effect into the world. 
  \item \textbf{Backlash:} Any 1 rolled may cause narrative backlash related to the Element.
\end{itemize}

\paragraph{Limits}
Casters can attempt any effect that can be described, but the larger the scope, the higher the DV. Improvisation is costly; reliable effects require repeated use and narrative justification.

\subsection{Rites Users (Runekeepers)}
\paragraph{Requirements}
A Patron bond, a Thiasos (Familiar; small spirit in the form of a creature/construct), and a Codex (4 XP) mark a character as a Runekeeper. 

\paragraph{Invocation}
\begin{itemize}
  \item \textbf{Action Cost:} Invoking a Rite requires 1 Action. 
  \item \textbf{Obligation:} Each Rite used marks Obligation on its clock. 
  \item \textbf{Push It:} Once per Rite, you may Push to increase its duration or potency by +1 step at the cost of +1 Obligation.
\end{itemize}

\paragraph{Obligation Clock}
Tracks the Patron’s claim. When full, the GM resolves the debt in-fiction. Obligation is reduced through service or downtime actions.

\subsection{Rites Difficulty Value}
\label{sec:rites-dv-expanded}
\index{Rites!Difficulty Value}
\index{DV}

The Difficulty Value (DV) to cast a Rite is:

\[
\text{DV} = \max\!\big(\text{Obligation Cost} - \text{Spirit}, \, \text{Tier}\big)
\]

\begin{description}
  \item[Obligation Cost:] The Rite’s listed cost in Obligation segments. This reflects the Patron’s toll for the magic.
  \item[Spirit:] The caster’s Spirit attribute. Each point reduces the effective weight of the Obligation, representing inner resilience and willpower.
  \item[Tier:] The Rite’s intrinsic difficulty based on scope or potency. DV can never fall below this floor.
\end{description}

\subsection{Invokers (Symbol Path)}
\paragraph{Patron’s Symbol}
\begin{itemize}
  \item \textbf{Minor Asset, 4 XP each.}
  \item Each Symbol is consecrated to one Patron and grants ritual access to that Patron’s Rites. 
  \item You may hold multiple Symbols, one per Patron.
\end{itemize}

\paragraph{Rite Invocation via Symbol}
\begin{itemize}
  \item \textbf{Time.} Invoking a Rite via Symbol takes \(\text{DV} + 1\) rounds.
  \item \textbf{Obligation.} On completion, mark +1 Obligation (in addition to any listed Rite costs, if applicable).
  \item \textbf{No Push.} Invoker Rites cannot use \emph{Push It} benefits.
  \item \textbf{Symbol Display.} The Symbol must remain visible throughout the invocation.
  \item \textbf{Materials.} Symbols replace any Thaisos and Codex requirements.
\end{itemize}

\paragraph{Crack the Seal (Instant Cast)}
As part of an Invoker Rite, you may immediately resolve the effect by setting the Symbol to \emph{Compromised} and marking +2 Obligation segments (+3 if High-Power). The GM may spend 1 SB on-theme. The Symbol remains but must be restored in downtime.

\paragraph{Restoring Symbols}
A Compromised Symbol is inert until repaired. Use a downtime action and test (DV 3 or fiction-appropriate). Success restores it; a shaky result leaves it Neglected (rituals work but cost +1 Obligation).

\paragraph{Invoker Path Limitations}
\begin{itemize}
  \item Cannot Push. 
  \item Max simultaneous rituals = Spirit. Starting a new ritual ends the oldest or adds +1 Obligation to it. 
  \item Carrying 4+ Symbols causes interference: the first ritual each scene marks +1 extra Obligation.
\end{itemize}

\subsection{Patron’s Gift (Imbuements)}
\textbf{Patron’s Gift (Free, Requires Thiasos)}\\
Duration: Scene; Range: Touch; Stacking: No.\\
Effect: Imbue one item with +1 Weapon (Melee) and +1 Thematic Skill (Patron domain) for the scene.\\
Activation: Requires 1 Action once per scene.\\
Push It: The item’s power persists for one additional scene but marks +1 Obligation.\\
Requires: Familiar (Invoke: 1 Boon).

\subsection{Mixing the Paths}
Players may combine Casting, Rites, and Invoking, but each path introduces its own bookkeeping:
\begin{itemize}
  \item Casters track Backlash. 
  \item Rites users track Obligation. 
  \item Invokers track Symbol states (Maintained, Neglected, Compromised). 
\end{itemize}
Mixing provides flexibility but less efficiency than specialization. Specialists gain stronger benefits, while mixers gain narrative breadth.

% !TEX root = srd_main.tex
% SRD Insert: Elemental Backlash (Condensed SRD Version)

\section{Elemental Backlash (Condensed)}\label{sec:backlash-condensed}
\index{Backlash}\index{Elements}\index{Story Beats}

Magic unsettles the weave. Each element (and its counterpart) carries a distinct backlash pattern. When a roll shows a 1 (generating a (SB)), or when a player accepts a (SB) to escalate, apply a \textbf{Minor} backlash. Players may opt to escalate to \textbf{Major} by taking +1 (SB).

\begin{table}[h]
\centering
\caption{Backlash at a Glance}
\label{tab:backlash-condensed}
\renewcommand{\arraystretch}{1.12}
\begin{tabularx}{\linewidth}{>{\bfseries}l >{\raggedright}X >{\raggedright}X}
\toprule
Element & Minor Backlash & Major Backlash \
\midrule
Earth / Fate & Slips, binds, encumbrance • –1 Position or \textsc{Encumbered}. & Fissure, entrapment • Clock +1 (Collapse) or \textsc{Pinned}. \
Fire / Life & Smoke, sparks, heat • –1 Effect or \textsc{Singed}. & Blaze, fever, ignition • Clock +1 (Fire) or 1 Harm. \
Air / Luck & Scatter, misheard words • –1 Position or Clock +1/2 (Attention). & Unlikely mishap • Lose a tool/use or (SB) +1. \
Water / Dreams (Obishaal) & Slippery tide, slow gear • –1 Effect or \textsc{Waterlogged}. & Undertow, veering path • Clock +1 (Flood) or intrusion from Ways Between. \
Fate / Earth & Probability resists • –1 Effect or Clock +1/2 (Inevitable). & Demand arrives • Immediate sacrifice or (SB) +1 & mark \textsc{Omen}. \
Life / Fire & Growth surge, vines tether • –1 Effect or \textsc{Overgrowth}. & Riot of life • Clock +1 (Biohazard) or convert healing to (SB) +1. \
Luck / Air & Odds flip • –1 Position or Clock +1/2 (Coincidence). & Catastrophic fluke • Force re-roll; if any 1, (SB) +1 and Minor repeats. \
Death / Water (Obishaal) & Whispers, chill • \textsc{Shaken} or Clock +1/2 (Haunting). & Threshold opens • Clock +1 (Crossing Due) or revenant intrusion. \
\bottomrule
\end{tabularx}
\end{table}

\begin{tcolorbox}[title={Cheatsheet},colback=gray!5,colframe=black]
\small Minor = wobble; Major = lurch. Apply once per cast. Offer players the option to escalate to Major by taking (SB) +1.\par\smallskip
Earth/Fate binds; Fire/Life burns or grows; Air/Luck scatters or flips; Water/Obishaal pulls or opens.\end{tcolorbox}
% !TEX root = srd_main.tex
% SRD Insert: Universal Rituals — Quick-Start Spread (SRD concise)
% Assumes: booktabs, tabularx, xcolor, tcolorbox, index set up

\section{Universal Rituals (Quick-Start)}\label{sec:universal-rituals}
\index{Rituals}\index{Story Beats}\index{Backlash}\index{Realms}\index{Obishaal@Obishaal}

These table-ready rituals are system-agnostic and available to any chassis that can perform rituals. Each lists \textbf{Cast Time}, \textbf{Setup/Components}, \textbf{Effect}, and explicit \textbf{Costs/(SB) hooks}. GMs should reskin names freely to match patrons, runes, symbols, or tag-sets.

\begin{tcolorbox}[title={Ritual Casting Basics},colback=gray!5,colframe=black]
\textbf{Triggering Risk.} On any ritual roll showing a 1, gain a (SB) and apply elemental Backlash (\S\ref{sec:backlash-condensed}). Players may accept +1 (SB) to push an effect one step (position/effect/scale) if fictionally supported.\index{Story Beats!rituals}
\end{tcolorbox}

% –––––––––– TABLE ––––––––––
\begin{table}[h]
\centering
\caption{Rituals at a Glance}
\label{tab:rituals-quick}
\renewcommand{\arraystretch}{1.12}
\begin{tabularx}{\linewidth}{>{\bfseries}l c l X X}
\toprule
Name & Tier & Cast Time & Setup & Components & Effect (with Costs/(SB) Hooks) \
\midrule
Wayfinder’s Thread & Low & 1 minute & Red cord knotted thrice; whisper a destination. & Create a faint tether toward the nearest safe path. \emph{Cost:} mark \textsc{Fatigue} if used more than once/scene. \emph{Push:} +1 (SB) to reveal a hidden shortcut (Clock –1/2 on Travel). \
Oath-Ward & Low & 5 minutes & Chalk circle; sworn phrase all participants repeat. & Ward a small area vs. intrusion (mundane/lesser). \emph{Cost:} requires sincere oath; breaking it triggers (SB) +1 and ends ward. \
Ember-Glass & Low & 1 minute & Hold an ember behind smoked glass. & Sense nearby heat sources/life signs through cover. \emph{Cost:} lose one use of a tinder/torch. \emph{Push:} +1 (SB) to pierce thin walls. \
Salt-Cut & Low & 1 minute & Salt line and bronze knife. & Sever a simple ongoing effect (rope-binds, minor charm). \emph{Cost:} consume 1 use of salt. \emph{Push:} +1 (SB) to cut a tougher link (Clock –1/2 on Restraint/Hex). \
River’s Memory & Med & 10 minutes & Bowl of water and a personal token. & Scry a recent passage/event tied to the token, brief and blurry. \emph{Cost:} token is waterlogged/ruined. \emph{Push:} +1 (SB) for a clearer second image. \
Bargain-Bead & Med & 10 minutes & Two carved beads; one is offered openly. & Invite a nearby power/spirit to parley. \emph{Cost:} give up a valuable concession now or take (SB) +1 when you refuse. \
Quiet Veil & Med & 5 minutes & Ash across lips; bell muted in cloth. & Muffle a group’s sound and scent for a scene. \emph{Cost:} \textsc{Muted} Condition (social checks –1) until scene ends. \emph{Backlash:} Air/Luck. \
Shadow-Loom & Med & 5 minutes & Three pins; weave ambient shadow between them. & Create light-obscuring cover or misdirection in a small zone. \emph{Cost:} dim your own vision (–1 precision) while maintained. \emph{Push:} +1 (SB) to mirror a decoy image briefly. \
Dream-Way Marker & Med & 10 minutes & Sleep mask inked with a circle; water drip cadence. & Mark a safe entrance to the Ways Between; next sleep at site allows short transit. \emph{Cost:} all participants mark \textsc{Shaken} on waking. \emph{Backlash:} Death/Obishaal. \
Purge & Med & 10 minutes & Smoke of bitter herbs; clean blade drawn across incense. & Cleanse taint/disease/curse one step. \emph{Cost:} cleanse passes a lesser echo to the caster (–1 to a related action next scene). \emph{Push:} +1 (SB) to remove two steps but take \textsc{Weakened}. \
Fortune-Braid & High & 15 minutes & Three strands (hair, thread, wire) braided tight. & Bank a single lucky break: replace one die with its highest result this scene. \emph{Cost:} immediately take (SB) +1 if used offensively. \emph{Backlash:} Air/Luck. \
Fate-Splice & High & 15 minutes & Knot two names written on vellum. & Temporarily link two fates: transfer a single consequence/boon between them. \emph{Cost:} both bear a subtle mark until dawn; \emph{Push:} +1 (SB) to redirect a Major consequence. \emph{Backlash:} Fate/Earth. \
Summoner’s Gate & High & 20 minutes & Circle inscribed with true-name sigil or emblem. & Call a known entity safely; on success it arrives bound by a simple charge. \emph{Cost:} occupies one concurrency slot; breaking terms creates (SB) +1 and Disruption. \emph{Backlash:} varies by entity. \
\bottomrule
\end{tabularx}
\end{table}

\subsection*{Usage Notes}\label{subsec:ritual-usage}
\begin{itemize}
\item \textbf{Scaling.} Effects scale by position/effect/area via explicit (SB) offers or extra time/components.\index{Rituals!scaling}
\item \textbf{Elements.} Choose the dominant element by fiction (Fire for Ember-Glass; Water/Obishaal for Dream-Way) and apply the condensed backlash table (\S\ref{sec:backlash-condensed}).\index{Backlash!elements}
\item \textbf{Teamwork.} Extra participants can donate narrative components to reduce cast time \emph{or} to accept (SB) on the caster’s behalf once per ritual.\index{Teamwork}
\end{itemize}

\begin{tcolorbox}[title={Design Intent},colback=gray!5,colframe=black]
Each ritual bakes in a crisp \emph{cost}, a tempting \emph{push}, and a likely \emph{backlash}. Keep it fiction-first: components are story handles the GM can threaten, not bookkeeping chores.\end{tcolorbox}

\section{Talent: Cantor's Path --- ``Songs of the Low Rites''}
\label{talent:cantors-path}

\begin{tcolorbox}[colback=black!3,colframe=black!40!white,title={Cantor's Path}]
You echo the liturgies of Patrons through breath and string. Not a sworn celebrant but a perilous mimic, you weave Low Rites into song. It is slower, riskier, and beautiful---but never free.
\end{tcolorbox}

\paragraph*{Type} Major Talent (15 XP) \quad
\paragraph*{Prerequisites} \textbf{Lore 1+}, \textbf{Performance 2+}, \textbf{Presence 2+} \quad
\paragraph*{Access} Any character (does not require Thiasos membership).

\subsection*{Effect}
You may learn and perform \textbf{Low Rites as Songs}. Each Song counts as knowing the associated Low Rite for performance purposes only.

\begin{itemize}
  \item \textbf{Casting Test:} \emph{Lore + Performance vs.\ DV} (default DV = 2--3).
  \item \textbf{Action Economy:} \emph{1 action to begin;} Song \emph{resolves at the start of your next turn} unless accelerated.
  \item \textbf{Scope:} \emph{Low Rites only.} Standard/High Rites remain exclusive to Patrons and Thiasos initiates.
  \item \textbf{Costs:} Pay any \emph{materials} listed. On success you do \emph{not} mark Obligation.
\end{itemize}

\subsection*{Corruption Clock}
\begin{itemize}
  \item You gain a personal \textbf{Corruption Clock} equal in segments to your \textbf{Body} rating.
  \item Each time you cast a Song or whenever the Keeper spends a Story Beat involving you, mark +1 segment.
  \item When the Clock fills:
    \begin{itemize}
      \item You immediately gain a trait of corruption from the \textbf{last Patron} whose Rite you performed.
      \item All of your followers, retainers, or familiars also gain a trait of the same corruption (NPCs manifest visibly unsettling traits).
      \item Reset the Clock to empty.
    \end{itemize}
  \item Corruption traits gained in this way fade at the next \indexterm{Downtime}, unless reinforced by further Patron influence.
\end{itemize}

\subsection*{Outcomes}
\begin{description}
\item[Success:] The Low Rite takes effect as written.
\item[Partial:] The Rite manifests with reduced effect (–1 step) or shortened duration. Mark \textbf{Fatigue 1}.
\item[Failure:] No effect; mark \textbf{Fatigue 1} and the Keeper gains \textbf{+1 SB (Hearts)}.
\item[Interrupted:] Harm, Silence, or disruption before resolution = treat as Failure.
\end{description}

\subsection*{Push It}
When you Push:
\begin{itemize}
  \item Song resolves immediately instead of next round.
  \item Mark \textbf{Fatigue 1}.
  \item Add +1 to your \textbf{Corruption Clock}.
  \item Keeper immediately triggers a \textbf{Story Beat}, representing fallout from a Patron, the Road, or social attention.
\end{itemize}

\subsection*{Limits \& Interactions}
\begin{itemize}
  \item \textbf{Stacking:} Cannot benefit from the same Rite twice.
  \item \textbf{Visibility:} Songs are inherently noticeable. On Failure or Push, assume observers take note.
  \item \textbf{Silence/Disruption:} Impose –1 to –3 dice at Keeper’s discretion.
\end{itemize}

\section{The Magic System}

Magic in Fate's Edge is expressed through three interconnected paths. 
You may specialize in one, or mix them at greater bookkeeping cost. 
All paths share the same dice engine and SB/Obligation economies, but their flavor and risks differ.

\subsection{Three Faces of Magic}
\begin{description}[leftmargin=1.5em, style=nextline]
  \item[Caster (Freeform):] Requires \textbf{Talent: Caster’s Gift (2 XP)}. Grants access to Weave \& Cast using the Eight Elements. Flexible, creative, and risky (Backlash on 1s). 
  \item[Rites User (Runekeeper):] Requires \textbf{Patron + Thiasos (Familiar)}. Grants access to a Patron’s Rites. Structured, powerful, but debt-driven through \textbf{Obligation}.
  \item[Invoker (Symbol Path):] Requires one or more \textbf{Patron’s Symbols (4 XP each)}. Grants access to that Patron’s Rites via rituals. Safe but slow; can \emph{Crack the Seal} to cast instantly at steep Obligation cost.
\end{description}

\subsection{Casting (Freeform)}
\paragraph{Weave \& Cast}
Casters describe the effect in terms of the Eight Elements (Earth, Fire, Air, Water, Fate, Life, Luck, Death). The GM sets DV and Effect based on scope. 

\begin{itemize}
  \item \textbf{Weave:} Player builds dice pool and rolls. On success, they stabilize the spell’s form.
  \item \textbf{Cast:} A second roll channels the effect into the world. 
  \item \textbf{Backlash:} Any 1 rolled may cause narrative backlash related to the Element.
\end{itemize}

\paragraph{Limits}
Casters can attempt any effect that can be described, but the larger the scope, the higher the DV. Improvisation is costly; reliable effects require repeated use and narrative justification.

\subsection{Rites Users (Runekeepers)}
\paragraph{Requirements}
A Patron bond, a Thiasos (Familiar; small spirit in the form of a creature/construct), and a Codex (4 XP) mark a character as a Runekeeper. 

\paragraph{Invocation}
\begin{itemize}
  \item \textbf{Action Cost:} Invoking a Rite requires 1 Action. 
  \item \textbf{Obligation:} Each Rite used marks Obligation on its clock. 
  \item \textbf{Push It:} Once per Rite, you may Push to increase its duration or potency by +1 step at the cost of +1 Obligation.
\end{itemize}

\paragraph{Obligation Clock}
Tracks the Patron’s claim. When full, the GM resolves the debt in-fiction. Obligation is reduced through service or downtime actions.

\subsection{Rites Difficulty Value}
\label{sec:rites-dv-expanded}
\index{Rites!Difficulty Value}
\index{DV}

The Difficulty Value (DV) to cast a Rite is:

\[
\text{DV} = \max\!\big(\text{Obligation Cost} - \text{Spirit}, \, \text{Tier}\big)
\]

\begin{description}
  \item[Obligation Cost:] The Rite’s listed cost in Obligation segments. This reflects the Patron’s toll for the magic.
  \item[Spirit:] The caster’s Spirit attribute. Each point reduces the effective weight of the Obligation, representing inner resilience and willpower.
  \item[Tier:] The Rite’s intrinsic difficulty based on scope or potency. DV can never fall below this floor.
\end{description}

\subsection{Invokers (Symbol Path)}
\paragraph{Patron’s Symbol}
\begin{itemize}
  \item \textbf{Minor Asset, 4 XP each.}
  \item Each Symbol is consecrated to one Patron and grants ritual access to that Patron’s Rites. 
  \item You may hold multiple Symbols, one per Patron.
\end{itemize}

\paragraph{Rite Invocation via Symbol}
\begin{itemize}
  \item \textbf{Time.} Invoking a Rite via Symbol takes \(\text{DV} + 1\) rounds.
  \item \textbf{Obligation.} On completion, mark +1 Obligation (in addition to any listed Rite costs, if applicable).
  \item \textbf{No Push.} Invoker Rites cannot use \emph{Push It} benefits.
  \item \textbf{Symbol Display.} The Symbol must remain visible throughout the invocation.
  \item \textbf{Materials.} Symbols replace any Thaisos and Codex requirements.
\end{itemize}

\paragraph{Crack the Seal (Instant Cast)}
As part of an Invoker Rite, you may immediately resolve the effect by setting the Symbol to \emph{Compromised} and marking +2 Obligation segments (+3 if High-Power). The GM may spend 1 SB on-theme. The Symbol remains but must be restored in downtime.

\paragraph{Restoring Symbols}
A Compromised Symbol is inert until repaired. Use a downtime action and test (DV 3 or fiction-appropriate). Success restores it; a shaky result leaves it Neglected (rituals work but cost +1 Obligation).

\paragraph{Invoker Path Limitations}
\begin{itemize}
  \item Cannot Push. 
  \item Max simultaneous rituals = Spirit. Starting a new ritual ends the oldest or adds +1 Obligation to it. 
  \item Carrying 4+ Symbols causes interference: the first ritual each scene marks +1 extra Obligation.
\end{itemize}

\subsection{Patron’s Gift (Imbuements)}
\textbf{Patron’s Gift (Free, Requires Thiasos)}\\
Duration: Scene; Range: Touch; Stacking: No.\\
Effect: Imbue one item with +1 Weapon (Melee) and +1 Thematic Skill (Patron domain) for the scene.\\
Activation: Requires 1 Action once per scene.\\
Push It: The item’s power persists for one additional scene but marks +1 Obligation.\\
Requires: Familiar (Invoke: 1 Boon).

\subsection{Mixing the Paths}
Players may combine Casting, Rites, and Invoking, but each path introduces its own bookkeeping:
\begin{itemize}
  \item Casters track Backlash. 
  \item Rites users track Obligation. 
  \item Invokers track Symbol states (Maintained, Neglected, Compromised). 
\end{itemize}
Mixing provides flexibility but less efficiency than specialization. Specialists gain stronger benefits, while mixers gain narrative breadth.

% !TEX root = srd_main.tex
% SRD Insert: Elemental Backlash (Condensed SRD Version)

\section{Elemental Backlash (Condensed)}\label{sec:backlash-condensed}
\index{Backlash}\index{Elements}\index{Story Beats}

Magic unsettles the weave. Each element (and its counterpart) carries a distinct backlash pattern. When a roll shows a 1 (generating a (SB)), or when a player accepts a (SB) to escalate, apply a \textbf{Minor} backlash. Players may opt to escalate to \textbf{Major} by taking +1 (SB).

\begin{table}[h]
\centering
\caption{Backlash at a Glance}
\label{tab:backlash-condensed}
\renewcommand{\arraystretch}{1.12}
\begin{tabularx}{\linewidth}{>{\bfseries}l >{\raggedright}X >{\raggedright}X}
\toprule
Element & Minor Backlash & Major Backlash \
\midrule
Earth / Fate & Slips, binds, encumbrance • –1 Position or \textsc{Encumbered}. & Fissure, entrapment • Clock +1 (Collapse) or \textsc{Pinned}. \
Fire / Life & Smoke, sparks, heat • –1 Effect or \textsc{Singed}. & Blaze, fever, ignition • Clock +1 (Fire) or 1 Harm. \
Air / Luck & Scatter, misheard words • –1 Position or Clock +1/2 (Attention). & Unlikely mishap • Lose a tool/use or (SB) +1. \
Water / Dreams (Obishaal) & Slippery tide, slow gear • –1 Effect or \textsc{Waterlogged}. & Undertow, veering path • Clock +1 (Flood) or intrusion from Ways Between. \
Fate / Earth & Probability resists • –1 Effect or Clock +1/2 (Inevitable). & Demand arrives • Immediate sacrifice or (SB) +1 & mark \textsc{Omen}. \
Life / Fire & Growth surge, vines tether • –1 Effect or \textsc{Overgrowth}. & Riot of life • Clock +1 (Biohazard) or convert healing to (SB) +1. \
Luck / Air & Odds flip • –1 Position or Clock +1/2 (Coincidence). & Catastrophic fluke • Force re-roll; if any 1, (SB) +1 and Minor repeats. \
Death / Water (Obishaal) & Whispers, chill • \textsc{Shaken} or Clock +1/2 (Haunting). & Threshold opens • Clock +1 (Crossing Due) or revenant intrusion. \
\bottomrule
\end{tabularx}
\end{table}

\begin{tcolorbox}[title={Cheatsheet},colback=gray!5,colframe=black]
\small Minor = wobble; Major = lurch. Apply once per cast. Offer players the option to escalate to Major by taking (SB) +1.\par\smallskip
Earth/Fate binds; Fire/Life burns or grows; Air/Luck scatters or flips; Water/Obishaal pulls or opens.\end{tcolorbox}
% !TEX root = srd_main.tex
% SRD Insert: Universal Rituals — Quick-Start Spread (SRD concise)
% Assumes: booktabs, tabularx, xcolor, tcolorbox, index set up

\section{Universal Rituals (Quick-Start)}\label{sec:universal-rituals}
\index{Rituals}\index{Story Beats}\index{Backlash}\index{Realms}\index{Obishaal@Obishaal}

These table-ready rituals are system-agnostic and available to any chassis that can perform rituals. Each lists \textbf{Cast Time}, \textbf{Setup/Components}, \textbf{Effect}, and explicit \textbf{Costs/(SB) hooks}. GMs should reskin names freely to match patrons, runes, symbols, or tag-sets.

\begin{tcolorbox}[title={Ritual Casting Basics},colback=gray!5,colframe=black]
\textbf{Triggering Risk.} On any ritual roll showing a 1, gain a (SB) and apply elemental Backlash (\S\ref{sec:backlash-condensed}). Players may accept +1 (SB) to push an effect one step (position/effect/scale) if fictionally supported.\index{Story Beats!rituals}
\end{tcolorbox}

% –––––––––– TABLE ––––––––––
\begin{table}[h]
\centering
\caption{Rituals at a Glance}
\label{tab:rituals-quick}
\renewcommand{\arraystretch}{1.12}
\begin{tabularx}{\linewidth}{>{\bfseries}l c l X X}
\toprule
Name & Tier & Cast Time & Setup & Components & Effect (with Costs/(SB) Hooks) \
\midrule
Wayfinder’s Thread & Low & 1 minute & Red cord knotted thrice; whisper a destination. & Create a faint tether toward the nearest safe path. \emph{Cost:} mark \textsc{Fatigue} if used more than once/scene. \emph{Push:} +1 (SB) to reveal a hidden shortcut (Clock –1/2 on Travel). \
Oath-Ward & Low & 5 minutes & Chalk circle; sworn phrase all participants repeat. & Ward a small area vs. intrusion (mundane/lesser). \emph{Cost:} requires sincere oath; breaking it triggers (SB) +1 and ends ward. \
Ember-Glass & Low & 1 minute & Hold an ember behind smoked glass. & Sense nearby heat sources/life signs through cover. \emph{Cost:} lose one use of a tinder/torch. \emph{Push:} +1 (SB) to pierce thin walls. \
Salt-Cut & Low & 1 minute & Salt line and bronze knife. & Sever a simple ongoing effect (rope-binds, minor charm). \emph{Cost:} consume 1 use of salt. \emph{Push:} +1 (SB) to cut a tougher link (Clock –1/2 on Restraint/Hex). \
River’s Memory & Med & 10 minutes & Bowl of water and a personal token. & Scry a recent passage/event tied to the token, brief and blurry. \emph{Cost:} token is waterlogged/ruined. \emph{Push:} +1 (SB) for a clearer second image. \
Bargain-Bead & Med & 10 minutes & Two carved beads; one is offered openly. & Invite a nearby power/spirit to parley. \emph{Cost:} give up a valuable concession now or take (SB) +1 when you refuse. \
Quiet Veil & Med & 5 minutes & Ash across lips; bell muted in cloth. & Muffle a group’s sound and scent for a scene. \emph{Cost:} \textsc{Muted} Condition (social checks –1) until scene ends. \emph{Backlash:} Air/Luck. \
Shadow-Loom & Med & 5 minutes & Three pins; weave ambient shadow between them. & Create light-obscuring cover or misdirection in a small zone. \emph{Cost:} dim your own vision (–1 precision) while maintained. \emph{Push:} +1 (SB) to mirror a decoy image briefly. \
Dream-Way Marker & Med & 10 minutes & Sleep mask inked with a circle; water drip cadence. & Mark a safe entrance to the Ways Between; next sleep at site allows short transit. \emph{Cost:} all participants mark \textsc{Shaken} on waking. \emph{Backlash:} Death/Obishaal. \
Purge & Med & 10 minutes & Smoke of bitter herbs; clean blade drawn across incense. & Cleanse taint/disease/curse one step. \emph{Cost:} cleanse passes a lesser echo to the caster (–1 to a related action next scene). \emph{Push:} +1 (SB) to remove two steps but take \textsc{Weakened}. \
Fortune-Braid & High & 15 minutes & Three strands (hair, thread, wire) braided tight. & Bank a single lucky break: replace one die with its highest result this scene. \emph{Cost:} immediately take (SB) +1 if used offensively. \emph{Backlash:} Air/Luck. \
Fate-Splice & High & 15 minutes & Knot two names written on vellum. & Temporarily link two fates: transfer a single consequence/boon between them. \emph{Cost:} both bear a subtle mark until dawn; \emph{Push:} +1 (SB) to redirect a Major consequence. \emph{Backlash:} Fate/Earth. \
Summoner’s Gate & High & 20 minutes & Circle inscribed with true-name sigil or emblem. & Call a known entity safely; on success it arrives bound by a simple charge. \emph{Cost:} occupies one concurrency slot; breaking terms creates (SB) +1 and Disruption. \emph{Backlash:} varies by entity. \
\bottomrule
\end{tabularx}
\end{table}

\subsection*{Usage Notes}\label{subsec:ritual-usage}
\begin{itemize}
\item \textbf{Scaling.} Effects scale by position/effect/area via explicit (SB) offers or extra time/components.\index{Rituals!scaling}
\item \textbf{Elements.} Choose the dominant element by fiction (Fire for Ember-Glass; Water/Obishaal for Dream-Way) and apply the condensed backlash table (\S\ref{sec:backlash-condensed}).\index{Backlash!elements}
\item \textbf{Teamwork.} Extra participants can donate narrative components to reduce cast time \emph{or} to accept (SB) on the caster’s behalf once per ritual.\index{Teamwork}
\end{itemize}

\begin{tcolorbox}[title={Design Intent},colback=gray!5,colframe=black]
Each ritual bakes in a crisp \emph{cost}, a tempting \emph{push}, and a likely \emph{backlash}. Keep it fiction-first: components are story handles the GM can threaten, not bookkeeping chores.\end{tcolorbox}

\section{Talent: Cantor's Path --- ``Songs of the Low Rites''}
\label{talent:cantors-path}

\begin{tcolorbox}[colback=black!3,colframe=black!40!white,title={Cantor's Path}]
You echo the liturgies of Patrons through breath and string. Not a sworn celebrant but a perilous mimic, you weave Low Rites into song. It is slower, riskier, and beautiful---but never free.
\end{tcolorbox}

\paragraph*{Type} Major Talent (15 XP) \quad
\paragraph*{Prerequisites} \textbf{Lore 1+}, \textbf{Performance 2+}, \textbf{Presence 2+} \quad
\paragraph*{Access} Any character (does not require Thiasos membership).

\subsection*{Effect}
You may learn and perform \textbf{Low Rites as Songs}. Each Song counts as knowing the associated Low Rite for performance purposes only.

\begin{itemize}
  \item \textbf{Casting Test:} \emph{Lore + Performance vs.\ DV} (default DV = 2--3).
  \item \textbf{Action Economy:} \emph{1 action to begin;} Song \emph{resolves at the start of your next turn} unless accelerated.
  \item \textbf{Scope:} \emph{Low Rites only.} Standard/High Rites remain exclusive to Patrons and Thiasos initiates.
  \item \textbf{Costs:} Pay any \emph{materials} listed. On success you do \emph{not} mark Obligation.
\end{itemize}

\subsection*{Corruption Clock}
\begin{itemize}
  \item You gain a personal \textbf{Corruption Clock} equal in segments to your \textbf{Body} rating.
  \item Each time you cast a Song or whenever the Keeper spends a Story Beat involving you, mark +1 segment.
  \item When the Clock fills:
    \begin{itemize}
      \item You immediately gain a trait of corruption from the \textbf{last Patron} whose Rite you performed.
      \item All of your followers, retainers, or familiars also gain a trait of the same corruption (NPCs manifest visibly unsettling traits).
      \item Reset the Clock to empty.
    \end{itemize}
  \item Corruption traits gained in this way fade at the next \indexterm{Downtime}, unless reinforced by further Patron influence.
\end{itemize}

\subsection*{Outcomes}
\begin{description}
\item[Success:] The Low Rite takes effect as written.
\item[Partial:] The Rite manifests with reduced effect (–1 step) or shortened duration. Mark \textbf{Fatigue 1}.
\item[Failure:] No effect; mark \textbf{Fatigue 1} and the Keeper gains \textbf{+1 SB (Hearts)}.
\item[Interrupted:] Harm, Silence, or disruption before resolution = treat as Failure.
\end{description}

\subsection*{Push It}
When you Push:
\begin{itemize}
  \item Song resolves immediately instead of next round.
  \item Mark \textbf{Fatigue 1}.
  \item Add +1 to your \textbf{Corruption Clock}.
  \item Keeper immediately triggers a \textbf{Story Beat}, representing fallout from a Patron, the Road, or social attention.
\end{itemize}

\subsection*{Limits \& Interactions}
\begin{itemize}
  \item \textbf{Stacking:} Cannot benefit from the same Rite twice.
  \item \textbf{Visibility:} Songs are inherently noticeable. On Failure or Push, assume observers take note.
  \item \textbf{Silence/Disruption:} Impose –1 to –3 dice at Keeper’s discretion.
\end{itemize}

\section{The Magic System}

Magic in Fate's Edge is expressed through three interconnected paths. 
You may specialize in one, or mix them at greater bookkeeping cost. 
All paths share the same dice engine and SB/Obligation economies, but their flavor and risks differ.

\subsection{Three Faces of Magic}
\begin{description}[leftmargin=1.5em, style=nextline]
  \item[Caster (Freeform):] Requires \textbf{Talent: Caster’s Gift (2 XP)}. Grants access to Weave \& Cast using the Eight Elements. Flexible, creative, and risky (Backlash on 1s). 
  \item[Rites User (Runekeeper):] Requires \textbf{Patron + Thiasos (Familiar)}. Grants access to a Patron’s Rites. Structured, powerful, but debt-driven through \textbf{Obligation}.
  \item[Invoker (Symbol Path):] Requires one or more \textbf{Patron’s Symbols (4 XP each)}. Grants access to that Patron’s Rites via rituals. Safe but slow; can \emph{Crack the Seal} to cast instantly at steep Obligation cost.
\end{description}

\subsection{Casting (Freeform)}
\paragraph{Weave \& Cast}
Casters describe the effect in terms of the Eight Elements (Earth, Fire, Air, Water, Fate, Life, Luck, Death). The GM sets DV and Effect based on scope. 

\begin{itemize}
  \item \textbf{Weave:} Player builds dice pool and rolls. On success, they stabilize the spell’s form.
  \item \textbf{Cast:} A second roll channels the effect into the world. 
  \item \textbf{Backlash:} Any 1 rolled may cause narrative backlash related to the Element.
\end{itemize}

\paragraph{Limits}
Casters can attempt any effect that can be described, but the larger the scope, the higher the DV. Improvisation is costly; reliable effects require repeated use and narrative justification.

\subsection{Rites Users (Runekeepers)}
\paragraph{Requirements}
A Patron bond, a Thiasos (Familiar), and a Codex (4 XP) mark a character as a Runekeeper. 

\paragraph{Invocation}
\begin{itemize}
  \item \textbf{Action Cost:} Invoking a Rite requires 1 Action. 
  \item \textbf{Obligation:} Each Rite used marks Obligation on its clock. 
  \item \textbf{Push It:} Once per Rite, you may Push to increase its duration or potency by +1 step at the cost of +1 Obligation.
\end{itemize}

\paragraph{Obligation Clock}
Tracks the Patron’s claim. When full, the GM resolves the debt in-fiction. Obligation is reduced through service or downtime actions.

\subsection{Rites Difficulty Value}
\label{sec:rites-dv-expanded}
\index{Rites!Difficulty Value}
\index{DV}

The Difficulty Value (DV) to cast a Rite is:

\[
\text{DV} = \max\!\big(\text{Obligation Cost} - \text{Spirit}, \, \text{Tier}\big)
\]

\begin{description}
  \item[Obligation Cost:] The Rite’s listed cost in Obligation segments. This reflects the Patron’s toll for the magic.
  \item[Spirit:] The caster’s Spirit attribute. Each point reduces the effective weight of the Obligation, representing inner resilience and willpower.
  \item[Tier:] The Rite’s intrinsic difficulty based on scope or potency. DV can never fall below this floor.
\end{description}

\subsection{Invokers (Symbol Path)}
\paragraph{Patron’s Symbol}
\begin{itemize}
  \item \textbf{Minor Asset, 4 XP each.}
  \item Each Symbol is consecrated to one Patron and grants ritual access to that Patron’s Rites. 
  \item You may hold multiple Symbols, one per Patron.
\end{itemize}

\paragraph{Rite Invocation via Symbol}
\begin{itemize}
  \item \textbf{Time.} Invoking a Rite via Symbol takes \(\text{DV} + 1\) rounds.
  \item \textbf{Obligation.} On completion, mark +1 Obligation (in addition to any listed Rite costs, if applicable).
  \item \textbf{No Push.} Invoker Rites cannot use \emph{Push It} benefits.
  \item \textbf{Symbol Display.} The Symbol must remain visible throughout the invocation.
  \item \textbf{Materials.} Symbols replace any Thaisos and Codex requirements.
\end{itemize}

\paragraph{Crack the Seal (Instant Cast)}
As part of an Invoker Rite, you may immediately resolve the effect by setting the Symbol to \emph{Compromised} and marking +2 Obligation segments (+3 if High-Power). The GM may spend 1 SB on-theme. The Symbol remains but must be restored in downtime.

\paragraph{Restoring Symbols}
A Compromised Symbol is inert until repaired. Use a downtime action and test (DV 3 or fiction-appropriate). Success restores it; a shaky result leaves it Neglected (rituals work but cost +1 Obligation).

\paragraph{Invoker Path Limitations}
\begin{itemize}
  \item Cannot Push. 
  \item Max simultaneous rituals = Spirit. Starting a new ritual ends the oldest or adds +1 Obligation to it. 
  \item Carrying 4+ Symbols causes interference: the first ritual each scene marks +1 extra Obligation.
\end{itemize}

\subsection{Patron’s Gift (Imbuements)}
\textbf{Patron’s Gift (Free, Requires Thiasos)}\\
Duration: Scene; Range: Touch; Stacking: No.\\
Effect: Imbue one item with +1 Weapon (Melee) and +1 Thematic Skill (Patron domain) for the scene.\\
Activation: Requires 1 Action once per scene.\\
Push It: The item’s power persists for one additional scene but marks +1 Obligation.\\
Requires: Familiar (Invoke: 1 Boon).

\subsection{Mixing the Paths}
Players may combine Casting, Rites, and Invoking, but each path introduces its own bookkeeping:
\begin{itemize}
  \item Casters track Backlash. 
  \item Rites users track Obligation. 
  \item Invokers track Symbol states (Maintained, Neglected, Compromised). 
\end{itemize}
Mixing provides flexibility but less efficiency than specialization. Specialists gain stronger benefits, while mixers gain narrative breadth.

% !TEX root = srd_main.tex
% SRD Insert: Elemental Backlash (Condensed SRD Version)

\section{Elemental Backlash (Condensed)}\label{sec:backlash-condensed}
\index{Backlash}\index{Elements}\index{Story Beats}

Magic unsettles the weave. Each element (and its counterpart) carries a distinct backlash pattern. When a roll shows a 1 (generating a (SB)), or when a player accepts a (SB) to escalate, apply a \textbf{Minor} backlash. Players may opt to escalate to \textbf{Major} by taking +1 (SB).

\begin{table}[h]
\centering
\caption{Backlash at a Glance}
\label{tab:backlash-condensed}
\renewcommand{\arraystretch}{1.12}
\begin{tabularx}{\linewidth}{>{\bfseries}l >{\raggedright}X >{\raggedright}X}
\toprule
Element & Minor Backlash & Major Backlash \
\midrule
Earth / Fate & Slips, binds, encumbrance • –1 Position or \textsc{Encumbered}. & Fissure, entrapment • Clock +1 (Collapse) or \textsc{Pinned}. \
Fire / Life & Smoke, sparks, heat • –1 Effect or \textsc{Singed}. & Blaze, fever, ignition • Clock +1 (Fire) or 1 Harm. \
Air / Luck & Scatter, misheard words • –1 Position or Clock +1/2 (Attention). & Unlikely mishap • Lose a tool/use or (SB) +1. \
Water / Dreams (Obishaal) & Slippery tide, slow gear • –1 Effect or \textsc{Waterlogged}. & Undertow, veering path • Clock +1 (Flood) or intrusion from Ways Between. \
Fate / Earth & Probability resists • –1 Effect or Clock +1/2 (Inevitable). & Demand arrives • Immediate sacrifice or (SB) +1 & mark \textsc{Omen}. \
Life / Fire & Growth surge, vines tether • –1 Effect or \textsc{Overgrowth}. & Riot of life • Clock +1 (Biohazard) or convert healing to (SB) +1. \
Luck / Air & Odds flip • –1 Position or Clock +1/2 (Coincidence). & Catastrophic fluke • Force re-roll; if any 1, (SB) +1 and Minor repeats. \
Death / Water (Obishaal) & Whispers, chill • \textsc{Shaken} or Clock +1/2 (Haunting). & Threshold opens • Clock +1 (Crossing Due) or revenant intrusion. \
\bottomrule
\end{tabularx}
\end{table}

\begin{tcolorbox}[title={Cheatsheet},colback=gray!5,colframe=black]
\small Minor = wobble; Major = lurch. Apply once per cast. Offer players the option to escalate to Major by taking (SB) +1.\par\smallskip
Earth/Fate binds; Fire/Life burns or grows; Air/Luck scatters or flips; Water/Obishaal pulls or opens.\end{tcolorbox}
% !TEX root = srd_main.tex
% SRD Insert: Universal Rituals — Quick-Start Spread (SRD concise)
% Assumes: booktabs, tabularx, xcolor, tcolorbox, index set up

\section{Universal Rituals (Quick-Start)}\label{sec:universal-rituals}
\index{Rituals}\index{Story Beats}\index{Backlash}\index{Realms}\index{Obishaal@Obishaal}

These table-ready rituals are system-agnostic and available to any chassis that can perform rituals. Each lists \textbf{Cast Time}, \textbf{Setup/Components}, \textbf{Effect}, and explicit \textbf{Costs/(SB) hooks}. GMs should reskin names freely to match patrons, runes, symbols, or tag-sets.

\begin{tcolorbox}[title={Ritual Casting Basics},colback=gray!5,colframe=black]
\textbf{Triggering Risk.} On any ritual roll showing a 1, gain a (SB) and apply elemental Backlash (\S\ref{sec:backlash-condensed}). Players may accept +1 (SB) to push an effect one step (position/effect/scale) if fictionally supported.\index{Story Beats!rituals}
\end{tcolorbox}

% –––––––––– TABLE ––––––––––
\begin{table}[h]
\centering
\caption{Rituals at a Glance}
\label{tab:rituals-quick}
\renewcommand{\arraystretch}{1.12}
\begin{tabularx}{\linewidth}{>{\bfseries}l c l X X}
\toprule
Name & Tier & Cast Time & Setup & Components & Effect (with Costs/(SB) Hooks) \
\midrule
Wayfinder’s Thread & Low & 1 minute & Red cord knotted thrice; whisper a destination. & Create a faint tether toward the nearest safe path. \emph{Cost:} mark \textsc{Fatigue} if used more than once/scene. \emph{Push:} +1 (SB) to reveal a hidden shortcut (Clock –1/2 on Travel). \
Oath-Ward & Low & 5 minutes & Chalk circle; sworn phrase all participants repeat. & Ward a small area vs. intrusion (mundane/lesser). \emph{Cost:} requires sincere oath; breaking it triggers (SB) +1 and ends ward. \
Ember-Glass & Low & 1 minute & Hold an ember behind smoked glass. & Sense nearby heat sources/life signs through cover. \emph{Cost:} lose one use of a tinder/torch. \emph{Push:} +1 (SB) to pierce thin walls. \
Salt-Cut & Low & 1 minute & Salt line and bronze knife. & Sever a simple ongoing effect (rope-binds, minor charm). \emph{Cost:} consume 1 use of salt. \emph{Push:} +1 (SB) to cut a tougher link (Clock –1/2 on Restraint/Hex). \
River’s Memory & Med & 10 minutes & Bowl of water and a personal token. & Scry a recent passage/event tied to the token, brief and blurry. \emph{Cost:} token is waterlogged/ruined. \emph{Push:} +1 (SB) for a clearer second image. \
Bargain-Bead & Med & 10 minutes & Two carved beads; one is offered openly. & Invite a nearby power/spirit to parley. \emph{Cost:} give up a valuable concession now or take (SB) +1 when you refuse. \
Quiet Veil & Med & 5 minutes & Ash across lips; bell muted in cloth. & Muffle a group’s sound and scent for a scene. \emph{Cost:} \textsc{Muted} Condition (social checks –1) until scene ends. \emph{Backlash:} Air/Luck. \
Shadow-Loom & Med & 5 minutes & Three pins; weave ambient shadow between them. & Create light-obscuring cover or misdirection in a small zone. \emph{Cost:} dim your own vision (–1 precision) while maintained. \emph{Push:} +1 (SB) to mirror a decoy image briefly. \
Dream-Way Marker & Med & 10 minutes & Sleep mask inked with a circle; water drip cadence. & Mark a safe entrance to the Ways Between; next sleep at site allows short transit. \emph{Cost:} all participants mark \textsc{Shaken} on waking. \emph{Backlash:} Death/Obishaal. \
Purge & Med & 10 minutes & Smoke of bitter herbs; clean blade drawn across incense. & Cleanse taint/disease/curse one step. \emph{Cost:} cleanse passes a lesser echo to the caster (–1 to a related action next scene). \emph{Push:} +1 (SB) to remove two steps but take \textsc{Weakened}. \
Fortune-Braid & High & 15 minutes & Three strands (hair, thread, wire) braided tight. & Bank a single lucky break: replace one die with its highest result this scene. \emph{Cost:} immediately take (SB) +1 if used offensively. \emph{Backlash:} Air/Luck. \
Fate-Splice & High & 15 minutes & Knot two names written on vellum. & Temporarily link two fates: transfer a single consequence/boon between them. \emph{Cost:} both bear a subtle mark until dawn; \emph{Push:} +1 (SB) to redirect a Major consequence. \emph{Backlash:} Fate/Earth. \
Summoner’s Gate & High & 20 minutes & Circle inscribed with true-name sigil or emblem. & Call a known entity safely; on success it arrives bound by a simple charge. \emph{Cost:} occupies one concurrency slot; breaking terms creates (SB) +1 and Disruption. \emph{Backlash:} varies by entity. \
\bottomrule
\end{tabularx}
\end{table}

\subsection*{Usage Notes}\label{subsec:ritual-usage}
\begin{itemize}
\item \textbf{Scaling.} Effects scale by position/effect/area via explicit (SB) offers or extra time/components.\index{Rituals!scaling}
\item \textbf{Elements.} Choose the dominant element by fiction (Fire for Ember-Glass; Water/Obishaal for Dream-Way) and apply the condensed backlash table (\S\ref{sec:backlash-condensed}).\index{Backlash!elements}
\item \textbf{Teamwork.} Extra participants can donate narrative components to reduce cast time \emph{or} to accept (SB) on the caster’s behalf once per ritual.\index{Teamwork}
\end{itemize}

\begin{tcolorbox}[title={Design Intent},colback=gray!5,colframe=black]
Each ritual bakes in a crisp \emph{cost}, a tempting \emph{push}, and a likely \emph{backlash}. Keep it fiction-first: components are story handles the GM can threaten, not bookkeeping chores.\end{tcolorbox}

\section{Talent: Cantor's Path --- ``Songs of the Low Rites''}
\label{talent:cantors-path}

\begin{tcolorbox}[colback=black!3,colframe=black!40!white,title={Cantor's Path}]
You echo the liturgies of Patrons through breath and string. Not a sworn celebrant but a perilous mimic, you weave Low Rites into song. It is slower, riskier, and beautiful---but never free.
\end{tcolorbox}

\paragraph*{Type} Minor Talent (10 XP) \quad
\paragraph*{Prerequisites} \textbf{Lore 1+}, \textbf{Performance 2+}, \textbf{Presence 2+} \quad
\paragraph*{Access} Any character (does not require Thiasos membership).

\subsection*{Effect}
You may learn and perform \textbf{Low Rites as Songs}. Each Song counts as knowing the associated Low Rite for performance purposes only.

\begin{itemize}
\item \textbf{Casting Test:} \emph{Lore + Performance vs.\ DV} (default DV = 2--3; see \S\ref{talent:cantors-path-dv}).
\item \textbf{Action Economy:} \emph{1 action to begin;} Song \emph{resolves at the start of your next turn} unless accelerated.
\item \textbf{Scope:} \emph{Low Rites only.} Standard/High Rites remain exclusive to Patrons and Thiasos initiates.
\item \textbf{Costs:} Pay any \emph{materials} listed. On success you do \emph{not} mark Obligation.
\end{itemize}

\subsection*{Outcomes}
\begin{description}
\item[Success:] The Low Rite takes effect as written.
\item[Partial:] The Rite manifests with \emph{reduced Effect} (–1 step) or \emph{shortened duration}. Mark \textbf{Fatigue 1}.
\item[Failure:] No effect; mark \textbf{Fatigue 1} and the Keeper gains \textbf{+1 SB (Hearts)}. You \emph{do not} mark Obligation.
\item[Interrupted:] Harm, Silence, or disruption before resolution = treat as Failure.
\end{description}

\subsection*{Push It}
You may Push \emph{when you begin casting}:
\begin{itemize}
\item Song resolves immediately instead of next round.
\item Mark \textbf{Fatigue 1}.
\item Keeper immediately triggers a \textbf{Story Beat}, representing fallout from a Patron, the Road, or social attention.
\end{itemize}

\subsection*{Limits \& Interactions}
\begin{itemize}
\item \textbf{Stacking:} Cannot benefit from the same Rite twice.
\item \textbf{Social Visibility:} Songs are inherently noticeable. On Failure or Push, assume observers take note.
\item \textbf{Silence/Disruption:} Impose \emph{–1 to –3 dice} at Keeper’s discretion.
\end{itemize}

\subsection*{DV Guidance}
\label{talent:cantors-path-dv}
\begin{description}
\item[DV 2:] Personal augments, simple glamours.
\item[DV 3:] Zone effects, multi-target edges, time-bending moods.
\item[+1 DV:] Hostile crowds, loud environments, warded spaces.
\end{description}

\subsection*{Examples}
\begin{itemize}
\item \textbf{Perfect Note (Low):} Song resolves next round (DV 3). Push it to resolve immediately, mark +1 Fatigue, and trigger a Story Beat.
\item \textbf{Endless Revel (Low):} Creates revel zone on next turn. Push it to start now with the same costs.
\end{itemize}