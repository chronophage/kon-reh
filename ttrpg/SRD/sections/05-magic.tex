
% --- Fate's Edge SRD — Section 5: Magic System ---
% Include this file from your main .tex with: 
% --- Fate's Edge SRD — Section 5: Magic System ---
% Include this file from your main .tex with: 
% --- Fate's Edge SRD — Section 5: Magic System ---
% Include this file from your main .tex with: 
% --- Fate's Edge SRD — Section 5: Magic System ---
% Include this file from your main .tex with: \input{05-magic.tex}

\section{The Magic System}

Magic in Fate's Edge is expressed through three interconnected paths. 
You may specialize in one, or mix them at greater bookkeeping cost. 
All paths share the same dice engine and SB/Obligation economies, but their flavor and risks differ.

\subsection{Three Faces of Magic}
\begin{description}[leftmargin=1.5em, style=nextline]
  \item[Caster (Freeform):] Requires \textbf{Talent: Caster’s Gift (2 XP)}. Grants access to Weave \& Cast using the Eight Elements. Flexible, creative, and risky (Backlash on 1s). 
  \item[Rites User (Runekeeper):] Requires \textbf{Patron + Thiasos (Familiar)}. Grants access to a Patron’s Rites. Structured, powerful, but debt-driven through \textbf{Obligation}.
  \item[Invoker (Symbol Path):] Requires one or more \textbf{Patron’s Symbols (4 XP each)}. Grants access to that Patron’s Rites via rituals. Safe but slow; can \emph{Crack the Seal} to cast instantly at steep Obligation cost.
\end{description}

\subsection{Casting (Freeform)}
\paragraph{Weave \& Cast}
Casters describe the effect in terms of the Eight Elements (Earth, Fire, Air, Water, Fate, Life, Luck, Death). The GM sets DV and Effect based on scope. 

\begin{itemize}
  \item \textbf{Weave:} Player builds dice pool and rolls. On success, they stabilize the spell’s form.
  \item \textbf{Cast:} A second roll channels the effect into the world. 
  \item \textbf{Backlash:} Any 1 rolled may cause narrative backlash related to the Element.
\end{itemize}

\paragraph{Limits}
Casters can attempt any effect that can be described, but the larger the scope, the higher the DV. Improvisation is costly; reliable effects require repeated use and narrative justification.

\subsection{Rites Users (Runekeepers)}
\paragraph{Requirements}
A Patron bond, a Thiasos (Familiar), and a Codex (4 XP) mark a character as a Runekeeper. 

\paragraph{Invocation}
\begin{itemize}
  \item \textbf{Action Cost:} Invoking a Rite requires 1 Action. 
  \item \textbf{Obligation:} Each Rite used marks Obligation on its clock. 
  \item \textbf{Push It:} Once per Rite, you may Push to increase its duration or potency by +1 step at the cost of +1 Obligation.
\end{itemize}

\paragraph{Obligation Clock}
Tracks the Patron’s claim. When full, the GM resolves the debt in-fiction. Obligation is reduced through service or downtime actions.

\subsection{Rites Difficulty Value}
\label{sec:rites-dv-expanded}
\index{Rites!Difficulty Value}
\index{DV}

The Difficulty Value (DV) to cast a Rite is:

\[
\text{DV} = \max\!\big(\text{Obligation Cost} - \text{Spirit}, \, \text{Tier}\big)
\]

\begin{description}
  \item[Obligation Cost:] The Rite’s listed cost in Obligation segments. This reflects the Patron’s toll for the magic.
  \item[Spirit:] The caster’s Spirit attribute. Each point reduces the effective weight of the Obligation, representing inner resilience and willpower.
  \item[Tier:] The Rite’s intrinsic difficulty based on scope or potency. DV can never fall below this floor.
\end{description}

\subsection{Invokers (Symbol Path)}
\paragraph{Patron’s Symbol}
\begin{itemize}
  \item \textbf{Minor Asset, 4 XP each.}
  \item Each Symbol is consecrated to one Patron and grants ritual access to that Patron’s Rites. 
  \item You may hold multiple Symbols, one per Patron.
\end{itemize}

\paragraph{Rite Invocation via Symbol}
\begin{itemize}
  \item \textbf{Time.} Invoking a Rite via Symbol takes \(\text{DV} + 1\) rounds.
  \item \textbf{Obligation.} On completion, mark +1 Obligation (in addition to any listed Rite costs, if applicable).
  \item \textbf{No Push.} Invoker Rites cannot use \emph{Push It} benefits.
  \item \textbf{Symbol Display.} The Symbol must remain visible throughout the invocation.
  \item \textbf{Materials.} Symbols replace any Thaisos and Codex requirements.
\end{itemize}

\paragraph{Crack the Seal (Instant Cast)}
As part of an Invoker Rite, you may immediately resolve the effect by setting the Symbol to \emph{Compromised} and marking +2 Obligation segments (+3 if High-Power). The GM may spend 1 SB on-theme. The Symbol remains but must be restored in downtime.

\paragraph{Restoring Symbols}
A Compromised Symbol is inert until repaired. Use a downtime action and test (DV 3 or fiction-appropriate). Success restores it; a shaky result leaves it Neglected (rituals work but cost +1 Obligation).

\paragraph{Invoker Path Limitations}
\begin{itemize}
  \item Cannot Push. 
  \item Max simultaneous rituals = Spirit. Starting a new ritual ends the oldest or adds +1 Obligation to it. 
  \item Carrying 4+ Symbols causes interference: the first ritual each scene marks +1 extra Obligation.
\end{itemize}

\subsection{Patron’s Gift (Imbuements)}
\textbf{Patron’s Gift (Free, Requires Thiasos)}\\
Duration: Scene; Range: Touch; Stacking: No.\\
Effect: Imbue one item with +1 Weapon (Melee) and +1 Thematic Skill (Patron domain) for the scene.\\
Activation: Requires 1 Action once per scene.\\
Push It: The item’s power persists for one additional scene but marks +1 Obligation.\\
Requires: Familiar (Invoke: 1 Boon).

\subsection{Mixing the Paths}
Players may combine Casting, Rites, and Invoking, but each path introduces its own bookkeeping:
\begin{itemize}
  \item Casters track Backlash. 
  \item Rites users track Obligation. 
  \item Invokers track Symbol states (Maintained, Neglected, Compromised). 
\end{itemize}
Mixing provides flexibility but less efficiency than specialization. Specialists gain stronger benefits, while mixers gain narrative breadth.

% !TEX root = srd_main.tex
% SRD Insert: Elemental Backlash (Condensed SRD Version)

\section{Elemental Backlash (Condensed)}\label{sec:backlash-condensed}
\index{Backlash}\index{Elements}\index{Story Beats}

Magic unsettles the weave. Each element (and its counterpart) carries a distinct backlash pattern. When a roll shows a 1 (generating a (SB)), or when a player accepts a (SB) to escalate, apply a \textbf{Minor} backlash. Players may opt to escalate to \textbf{Major} by taking +1 (SB).

\begin{table}[h]
\centering
\caption{Backlash at a Glance}
\label{tab:backlash-condensed}
\renewcommand{\arraystretch}{1.12}
\begin{tabularx}{\linewidth}{>{\bfseries}l >{\raggedright}X >{\raggedright}X}
\toprule
Element & Minor Backlash & Major Backlash \
\midrule
Earth / Fate & Slips, binds, encumbrance • –1 Position or \textsc{Encumbered}. & Fissure, entrapment • Clock +1 (Collapse) or \textsc{Pinned}. \
Fire / Life & Smoke, sparks, heat • –1 Effect or \textsc{Singed}. & Blaze, fever, ignition • Clock +1 (Fire) or 1 Harm. \
Air / Luck & Scatter, misheard words • –1 Position or Clock +1/2 (Attention). & Unlikely mishap • Lose a tool/use or (SB) +1. \
Water / Dreams (Obishaal) & Slippery tide, slow gear • –1 Effect or \textsc{Waterlogged}. & Undertow, veering path • Clock +1 (Flood) or intrusion from Ways Between. \
Fate / Earth & Probability resists • –1 Effect or Clock +1/2 (Inevitable). & Demand arrives • Immediate sacrifice or (SB) +1 & mark \textsc{Omen}. \
Life / Fire & Growth surge, vines tether • –1 Effect or \textsc{Overgrowth}. & Riot of life • Clock +1 (Biohazard) or convert healing to (SB) +1. \
Luck / Air & Odds flip • –1 Position or Clock +1/2 (Coincidence). & Catastrophic fluke • Force re-roll; if any 1, (SB) +1 and Minor repeats. \
Death / Water (Obishaal) & Whispers, chill • \textsc{Shaken} or Clock +1/2 (Haunting). & Threshold opens • Clock +1 (Crossing Due) or revenant intrusion. \
\bottomrule
\end{tabularx}
\end{table}

\begin{tcolorbox}[title={Cheatsheet},colback=gray!5,colframe=black]
\small Minor = wobble; Major = lurch. Apply once per cast. Offer players the option to escalate to Major by taking (SB) +1.\par\smallskip
Earth/Fate binds; Fire/Life burns or grows; Air/Luck scatters or flips; Water/Obishaal pulls or opens.\end{tcolorbox}
% !TEX root = srd_main.tex
% SRD Insert: Universal Rituals — Quick-Start Spread (SRD concise)
% Assumes: booktabs, tabularx, xcolor, tcolorbox, index set up

\section{Universal Rituals (Quick-Start)}\label{sec:universal-rituals}
\index{Rituals}\index{Story Beats}\index{Backlash}\index{Realms}\index{Obishaal@Obishaal}

These table-ready rituals are system-agnostic and available to any chassis that can perform rituals. Each lists \textbf{Cast Time}, \textbf{Setup/Components}, \textbf{Effect}, and explicit \textbf{Costs/(SB) hooks}. GMs should reskin names freely to match patrons, runes, symbols, or tag-sets.

\begin{tcolorbox}[title={Ritual Casting Basics},colback=gray!5,colframe=black]
\textbf{Triggering Risk.} On any ritual roll showing a 1, gain a (SB) and apply elemental Backlash (\S\ref{sec:backlash-condensed}). Players may accept +1 (SB) to push an effect one step (position/effect/scale) if fictionally supported.\index{Story Beats!rituals}
\end{tcolorbox}

% –––––––––– TABLE ––––––––––
\begin{table}[h]
\centering
\caption{Rituals at a Glance}
\label{tab:rituals-quick}
\renewcommand{\arraystretch}{1.12}
\begin{tabularx}{\linewidth}{>{\bfseries}l c l X X}
\toprule
Name & Tier & Cast Time & Setup & Components & Effect (with Costs/(SB) Hooks) \
\midrule
Wayfinder’s Thread & Low & 1 minute & Red cord knotted thrice; whisper a destination. & Create a faint tether toward the nearest safe path. \emph{Cost:} mark \textsc{Fatigue} if used more than once/scene. \emph{Push:} +1 (SB) to reveal a hidden shortcut (Clock –1/2 on Travel). \
Oath-Ward & Low & 5 minutes & Chalk circle; sworn phrase all participants repeat. & Ward a small area vs. intrusion (mundane/lesser). \emph{Cost:} requires sincere oath; breaking it triggers (SB) +1 and ends ward. \
Ember-Glass & Low & 1 minute & Hold an ember behind smoked glass. & Sense nearby heat sources/life signs through cover. \emph{Cost:} lose one use of a tinder/torch. \emph{Push:} +1 (SB) to pierce thin walls. \
Salt-Cut & Low & 1 minute & Salt line and bronze knife. & Sever a simple ongoing effect (rope-binds, minor charm). \emph{Cost:} consume 1 use of salt. \emph{Push:} +1 (SB) to cut a tougher link (Clock –1/2 on Restraint/Hex). \
River’s Memory & Med & 10 minutes & Bowl of water and a personal token. & Scry a recent passage/event tied to the token, brief and blurry. \emph{Cost:} token is waterlogged/ruined. \emph{Push:} +1 (SB) for a clearer second image. \
Bargain-Bead & Med & 10 minutes & Two carved beads; one is offered openly. & Invite a nearby power/spirit to parley. \emph{Cost:} give up a valuable concession now or take (SB) +1 when you refuse. \
Quiet Veil & Med & 5 minutes & Ash across lips; bell muted in cloth. & Muffle a group’s sound and scent for a scene. \emph{Cost:} \textsc{Muted} Condition (social checks –1) until scene ends. \emph{Backlash:} Air/Luck. \
Shadow-Loom & Med & 5 minutes & Three pins; weave ambient shadow between them. & Create light-obscuring cover or misdirection in a small zone. \emph{Cost:} dim your own vision (–1 precision) while maintained. \emph{Push:} +1 (SB) to mirror a decoy image briefly. \
Dream-Way Marker & Med & 10 minutes & Sleep mask inked with a circle; water drip cadence. & Mark a safe entrance to the Ways Between; next sleep at site allows short transit. \emph{Cost:} all participants mark \textsc{Shaken} on waking. \emph{Backlash:} Death/Obishaal. \
Purge & Med & 10 minutes & Smoke of bitter herbs; clean blade drawn across incense. & Cleanse taint/disease/curse one step. \emph{Cost:} cleanse passes a lesser echo to the caster (–1 to a related action next scene). \emph{Push:} +1 (SB) to remove two steps but take \textsc{Weakened}. \
Fortune-Braid & High & 15 minutes & Three strands (hair, thread, wire) braided tight. & Bank a single lucky break: replace one die with its highest result this scene. \emph{Cost:} immediately take (SB) +1 if used offensively. \emph{Backlash:} Air/Luck. \
Fate-Splice & High & 15 minutes & Knot two names written on vellum. & Temporarily link two fates: transfer a single consequence/boon between them. \emph{Cost:} both bear a subtle mark until dawn; \emph{Push:} +1 (SB) to redirect a Major consequence. \emph{Backlash:} Fate/Earth. \
Summoner’s Gate & High & 20 minutes & Circle inscribed with true-name sigil or emblem. & Call a known entity safely; on success it arrives bound by a simple charge. \emph{Cost:} occupies one concurrency slot; breaking terms creates (SB) +1 and Disruption. \emph{Backlash:} varies by entity. \
\bottomrule
\end{tabularx}
\end{table}

\subsection*{Usage Notes}\label{subsec:ritual-usage}
\begin{itemize}
\item \textbf{Scaling.} Effects scale by position/effect/area via explicit (SB) offers or extra time/components.\index{Rituals!scaling}
\item \textbf{Elements.} Choose the dominant element by fiction (Fire for Ember-Glass; Water/Obishaal for Dream-Way) and apply the condensed backlash table (\S\ref{sec:backlash-condensed}).\index{Backlash!elements}
\item \textbf{Teamwork.} Extra participants can donate narrative components to reduce cast time \emph{or} to accept (SB) on the caster’s behalf once per ritual.\index{Teamwork}
\end{itemize}

\begin{tcolorbox}[title={Design Intent},colback=gray!5,colframe=black]
Each ritual bakes in a crisp \emph{cost}, a tempting \emph{push}, and a likely \emph{backlash}. Keep it fiction-first: components are story handles the GM can threaten, not bookkeeping chores.\end{tcolorbox}


\section{The Magic System}

Magic in Fate's Edge is expressed through three interconnected paths. 
You may specialize in one, or mix them at greater bookkeeping cost. 
All paths share the same dice engine and SB/Obligation economies, but their flavor and risks differ.

\subsection{Three Faces of Magic}
\begin{description}[leftmargin=1.5em, style=nextline]
  \item[Caster (Freeform):] Requires \textbf{Talent: Caster’s Gift (2 XP)}. Grants access to Weave \& Cast using the Eight Elements. Flexible, creative, and risky (Backlash on 1s). 
  \item[Rites User (Runekeeper):] Requires \textbf{Patron + Thiasos (Familiar)}. Grants access to a Patron’s Rites. Structured, powerful, but debt-driven through \textbf{Obligation}.
  \item[Invoker (Symbol Path):] Requires one or more \textbf{Patron’s Symbols (4 XP each)}. Grants access to that Patron’s Rites via rituals. Safe but slow; can \emph{Crack the Seal} to cast instantly at steep Obligation cost.
\end{description}

\subsection{Casting (Freeform)}
\paragraph{Weave \& Cast}
Casters describe the effect in terms of the Eight Elements (Earth, Fire, Air, Water, Fate, Life, Luck, Death). The GM sets DV and Effect based on scope. 

\begin{itemize}
  \item \textbf{Weave:} Player builds dice pool and rolls. On success, they stabilize the spell’s form.
  \item \textbf{Cast:} A second roll channels the effect into the world. 
  \item \textbf{Backlash:} Any 1 rolled may cause narrative backlash related to the Element.
\end{itemize}

\paragraph{Limits}
Casters can attempt any effect that can be described, but the larger the scope, the higher the DV. Improvisation is costly; reliable effects require repeated use and narrative justification.

\subsection{Rites Users (Runekeepers)}
\paragraph{Requirements}
A Patron bond, a Thiasos (Familiar), and a Codex (4 XP) mark a character as a Runekeeper. 

\paragraph{Invocation}
\begin{itemize}
  \item \textbf{Action Cost:} Invoking a Rite requires 1 Action. 
  \item \textbf{Obligation:} Each Rite used marks Obligation on its clock. 
  \item \textbf{Push It:} Once per Rite, you may Push to increase its duration or potency by +1 step at the cost of +1 Obligation.
\end{itemize}

\paragraph{Obligation Clock}
Tracks the Patron’s claim. When full, the GM resolves the debt in-fiction. Obligation is reduced through service or downtime actions.

\subsection{Rites Difficulty Value}
\label{sec:rites-dv-expanded}
\index{Rites!Difficulty Value}
\index{DV}

The Difficulty Value (DV) to cast a Rite is:

\[
\text{DV} = \max\!\big(\text{Obligation Cost} - \text{Spirit}, \, \text{Tier}\big)
\]

\begin{description}
  \item[Obligation Cost:] The Rite’s listed cost in Obligation segments. This reflects the Patron’s toll for the magic.
  \item[Spirit:] The caster’s Spirit attribute. Each point reduces the effective weight of the Obligation, representing inner resilience and willpower.
  \item[Tier:] The Rite’s intrinsic difficulty based on scope or potency. DV can never fall below this floor.
\end{description}

\subsection{Invokers (Symbol Path)}
\paragraph{Patron’s Symbol}
\begin{itemize}
  \item \textbf{Minor Asset, 4 XP each.}
  \item Each Symbol is consecrated to one Patron and grants ritual access to that Patron’s Rites. 
  \item You may hold multiple Symbols, one per Patron.
\end{itemize}

\paragraph{Rite Invocation via Symbol}
\begin{itemize}
  \item \textbf{Time.} Invoking a Rite via Symbol takes \(\text{DV} + 1\) rounds.
  \item \textbf{Obligation.} On completion, mark +1 Obligation (in addition to any listed Rite costs, if applicable).
  \item \textbf{No Push.} Invoker Rites cannot use \emph{Push It} benefits.
  \item \textbf{Symbol Display.} The Symbol must remain visible throughout the invocation.
  \item \textbf{Materials.} Symbols replace any Thaisos and Codex requirements.
\end{itemize}

\paragraph{Crack the Seal (Instant Cast)}
As part of an Invoker Rite, you may immediately resolve the effect by setting the Symbol to \emph{Compromised} and marking +2 Obligation segments (+3 if High-Power). The GM may spend 1 SB on-theme. The Symbol remains but must be restored in downtime.

\paragraph{Restoring Symbols}
A Compromised Symbol is inert until repaired. Use a downtime action and test (DV 3 or fiction-appropriate). Success restores it; a shaky result leaves it Neglected (rituals work but cost +1 Obligation).

\paragraph{Invoker Path Limitations}
\begin{itemize}
  \item Cannot Push. 
  \item Max simultaneous rituals = Spirit. Starting a new ritual ends the oldest or adds +1 Obligation to it. 
  \item Carrying 4+ Symbols causes interference: the first ritual each scene marks +1 extra Obligation.
\end{itemize}

\subsection{Patron’s Gift (Imbuements)}
\textbf{Patron’s Gift (Free, Requires Thiasos)}\\
Duration: Scene; Range: Touch; Stacking: No.\\
Effect: Imbue one item with +1 Weapon (Melee) and +1 Thematic Skill (Patron domain) for the scene.\\
Activation: Requires 1 Action once per scene.\\
Push It: The item’s power persists for one additional scene but marks +1 Obligation.\\
Requires: Familiar (Invoke: 1 Boon).

\subsection{Mixing the Paths}
Players may combine Casting, Rites, and Invoking, but each path introduces its own bookkeeping:
\begin{itemize}
  \item Casters track Backlash. 
  \item Rites users track Obligation. 
  \item Invokers track Symbol states (Maintained, Neglected, Compromised). 
\end{itemize}
Mixing provides flexibility but less efficiency than specialization. Specialists gain stronger benefits, while mixers gain narrative breadth.

% !TEX root = srd_main.tex
% SRD Insert: Elemental Backlash (Condensed SRD Version)

\section{Elemental Backlash (Condensed)}\label{sec:backlash-condensed}
\index{Backlash}\index{Elements}\index{Story Beats}

Magic unsettles the weave. Each element (and its counterpart) carries a distinct backlash pattern. When a roll shows a 1 (generating a (SB)), or when a player accepts a (SB) to escalate, apply a \textbf{Minor} backlash. Players may opt to escalate to \textbf{Major} by taking +1 (SB).

\begin{table}[h]
\centering
\caption{Backlash at a Glance}
\label{tab:backlash-condensed}
\renewcommand{\arraystretch}{1.12}
\begin{tabularx}{\linewidth}{>{\bfseries}l >{\raggedright}X >{\raggedright}X}
\toprule
Element & Minor Backlash & Major Backlash \
\midrule
Earth / Fate & Slips, binds, encumbrance • –1 Position or \textsc{Encumbered}. & Fissure, entrapment • Clock +1 (Collapse) or \textsc{Pinned}. \
Fire / Life & Smoke, sparks, heat • –1 Effect or \textsc{Singed}. & Blaze, fever, ignition • Clock +1 (Fire) or 1 Harm. \
Air / Luck & Scatter, misheard words • –1 Position or Clock +1/2 (Attention). & Unlikely mishap • Lose a tool/use or (SB) +1. \
Water / Dreams (Obishaal) & Slippery tide, slow gear • –1 Effect or \textsc{Waterlogged}. & Undertow, veering path • Clock +1 (Flood) or intrusion from Ways Between. \
Fate / Earth & Probability resists • –1 Effect or Clock +1/2 (Inevitable). & Demand arrives • Immediate sacrifice or (SB) +1 & mark \textsc{Omen}. \
Life / Fire & Growth surge, vines tether • –1 Effect or \textsc{Overgrowth}. & Riot of life • Clock +1 (Biohazard) or convert healing to (SB) +1. \
Luck / Air & Odds flip • –1 Position or Clock +1/2 (Coincidence). & Catastrophic fluke • Force re-roll; if any 1, (SB) +1 and Minor repeats. \
Death / Water (Obishaal) & Whispers, chill • \textsc{Shaken} or Clock +1/2 (Haunting). & Threshold opens • Clock +1 (Crossing Due) or revenant intrusion. \
\bottomrule
\end{tabularx}
\end{table}

\begin{tcolorbox}[title={Cheatsheet},colback=gray!5,colframe=black]
\small Minor = wobble; Major = lurch. Apply once per cast. Offer players the option to escalate to Major by taking (SB) +1.\par\smallskip
Earth/Fate binds; Fire/Life burns or grows; Air/Luck scatters or flips; Water/Obishaal pulls or opens.\end{tcolorbox}
% !TEX root = srd_main.tex
% SRD Insert: Universal Rituals — Quick-Start Spread (SRD concise)
% Assumes: booktabs, tabularx, xcolor, tcolorbox, index set up

\section{Universal Rituals (Quick-Start)}\label{sec:universal-rituals}
\index{Rituals}\index{Story Beats}\index{Backlash}\index{Realms}\index{Obishaal@Obishaal}

These table-ready rituals are system-agnostic and available to any chassis that can perform rituals. Each lists \textbf{Cast Time}, \textbf{Setup/Components}, \textbf{Effect}, and explicit \textbf{Costs/(SB) hooks}. GMs should reskin names freely to match patrons, runes, symbols, or tag-sets.

\begin{tcolorbox}[title={Ritual Casting Basics},colback=gray!5,colframe=black]
\textbf{Triggering Risk.} On any ritual roll showing a 1, gain a (SB) and apply elemental Backlash (\S\ref{sec:backlash-condensed}). Players may accept +1 (SB) to push an effect one step (position/effect/scale) if fictionally supported.\index{Story Beats!rituals}
\end{tcolorbox}

% –––––––––– TABLE ––––––––––
\begin{table}[h]
\centering
\caption{Rituals at a Glance}
\label{tab:rituals-quick}
\renewcommand{\arraystretch}{1.12}
\begin{tabularx}{\linewidth}{>{\bfseries}l c l X X}
\toprule
Name & Tier & Cast Time & Setup & Components & Effect (with Costs/(SB) Hooks) \
\midrule
Wayfinder’s Thread & Low & 1 minute & Red cord knotted thrice; whisper a destination. & Create a faint tether toward the nearest safe path. \emph{Cost:} mark \textsc{Fatigue} if used more than once/scene. \emph{Push:} +1 (SB) to reveal a hidden shortcut (Clock –1/2 on Travel). \
Oath-Ward & Low & 5 minutes & Chalk circle; sworn phrase all participants repeat. & Ward a small area vs. intrusion (mundane/lesser). \emph{Cost:} requires sincere oath; breaking it triggers (SB) +1 and ends ward. \
Ember-Glass & Low & 1 minute & Hold an ember behind smoked glass. & Sense nearby heat sources/life signs through cover. \emph{Cost:} lose one use of a tinder/torch. \emph{Push:} +1 (SB) to pierce thin walls. \
Salt-Cut & Low & 1 minute & Salt line and bronze knife. & Sever a simple ongoing effect (rope-binds, minor charm). \emph{Cost:} consume 1 use of salt. \emph{Push:} +1 (SB) to cut a tougher link (Clock –1/2 on Restraint/Hex). \
River’s Memory & Med & 10 minutes & Bowl of water and a personal token. & Scry a recent passage/event tied to the token, brief and blurry. \emph{Cost:} token is waterlogged/ruined. \emph{Push:} +1 (SB) for a clearer second image. \
Bargain-Bead & Med & 10 minutes & Two carved beads; one is offered openly. & Invite a nearby power/spirit to parley. \emph{Cost:} give up a valuable concession now or take (SB) +1 when you refuse. \
Quiet Veil & Med & 5 minutes & Ash across lips; bell muted in cloth. & Muffle a group’s sound and scent for a scene. \emph{Cost:} \textsc{Muted} Condition (social checks –1) until scene ends. \emph{Backlash:} Air/Luck. \
Shadow-Loom & Med & 5 minutes & Three pins; weave ambient shadow between them. & Create light-obscuring cover or misdirection in a small zone. \emph{Cost:} dim your own vision (–1 precision) while maintained. \emph{Push:} +1 (SB) to mirror a decoy image briefly. \
Dream-Way Marker & Med & 10 minutes & Sleep mask inked with a circle; water drip cadence. & Mark a safe entrance to the Ways Between; next sleep at site allows short transit. \emph{Cost:} all participants mark \textsc{Shaken} on waking. \emph{Backlash:} Death/Obishaal. \
Purge & Med & 10 minutes & Smoke of bitter herbs; clean blade drawn across incense. & Cleanse taint/disease/curse one step. \emph{Cost:} cleanse passes a lesser echo to the caster (–1 to a related action next scene). \emph{Push:} +1 (SB) to remove two steps but take \textsc{Weakened}. \
Fortune-Braid & High & 15 minutes & Three strands (hair, thread, wire) braided tight. & Bank a single lucky break: replace one die with its highest result this scene. \emph{Cost:} immediately take (SB) +1 if used offensively. \emph{Backlash:} Air/Luck. \
Fate-Splice & High & 15 minutes & Knot two names written on vellum. & Temporarily link two fates: transfer a single consequence/boon between them. \emph{Cost:} both bear a subtle mark until dawn; \emph{Push:} +1 (SB) to redirect a Major consequence. \emph{Backlash:} Fate/Earth. \
Summoner’s Gate & High & 20 minutes & Circle inscribed with true-name sigil or emblem. & Call a known entity safely; on success it arrives bound by a simple charge. \emph{Cost:} occupies one concurrency slot; breaking terms creates (SB) +1 and Disruption. \emph{Backlash:} varies by entity. \
\bottomrule
\end{tabularx}
\end{table}

\subsection*{Usage Notes}\label{subsec:ritual-usage}
\begin{itemize}
\item \textbf{Scaling.} Effects scale by position/effect/area via explicit (SB) offers or extra time/components.\index{Rituals!scaling}
\item \textbf{Elements.} Choose the dominant element by fiction (Fire for Ember-Glass; Water/Obishaal for Dream-Way) and apply the condensed backlash table (\S\ref{sec:backlash-condensed}).\index{Backlash!elements}
\item \textbf{Teamwork.} Extra participants can donate narrative components to reduce cast time \emph{or} to accept (SB) on the caster’s behalf once per ritual.\index{Teamwork}
\end{itemize}

\begin{tcolorbox}[title={Design Intent},colback=gray!5,colframe=black]
Each ritual bakes in a crisp \emph{cost}, a tempting \emph{push}, and a likely \emph{backlash}. Keep it fiction-first: components are story handles the GM can threaten, not bookkeeping chores.\end{tcolorbox}


\section{The Magic System}

Magic in Fate's Edge is expressed through three interconnected paths. 
You may specialize in one, or mix them at greater bookkeeping cost. 
All paths share the same dice engine and SB/Obligation economies, but their flavor and risks differ.

\subsection{Three Faces of Magic}
\begin{description}[leftmargin=1.5em, style=nextline]
  \item[Caster (Freeform):] Requires \textbf{Talent: Caster’s Gift (2 XP)}. Grants access to Weave \& Cast using the Eight Elements. Flexible, creative, and risky (Backlash on 1s). 
  \item[Rites User (Runekeeper):] Requires \textbf{Patron + Thiasos (Familiar)}. Grants access to a Patron’s Rites. Structured, powerful, but debt-driven through \textbf{Obligation}.
  \item[Invoker (Symbol Path):] Requires one or more \textbf{Patron’s Symbols (4 XP each)}. Grants access to that Patron’s Rites via rituals. Safe but slow; can \emph{Crack the Seal} to cast instantly at steep Obligation cost.
\end{description}

\subsection{Casting (Freeform)}
\paragraph{Weave \& Cast}
Casters describe the effect in terms of the Eight Elements (Earth, Fire, Air, Water, Fate, Life, Luck, Death). The GM sets DV and Effect based on scope. 

\begin{itemize}
  \item \textbf{Weave:} Player builds dice pool and rolls. On success, they stabilize the spell’s form.
  \item \textbf{Cast:} A second roll channels the effect into the world. 
  \item \textbf{Backlash:} Any 1 rolled may cause narrative backlash related to the Element.
\end{itemize}

\paragraph{Limits}
Casters can attempt any effect that can be described, but the larger the scope, the higher the DV. Improvisation is costly; reliable effects require repeated use and narrative justification.

\subsection{Rites Users (Runekeepers)}
\paragraph{Requirements}
A Patron bond, a Thiasos (Familiar), and a Codex (4 XP) mark a character as a Runekeeper. 

\paragraph{Invocation}
\begin{itemize}
  \item \textbf{Action Cost:} Invoking a Rite requires 1 Action. 
  \item \textbf{Obligation:} Each Rite used marks Obligation on its clock. 
  \item \textbf{Push It:} Once per Rite, you may Push to increase its duration or potency by +1 step at the cost of +1 Obligation.
\end{itemize}

\paragraph{Obligation Clock}
Tracks the Patron’s claim. When full, the GM resolves the debt in-fiction. Obligation is reduced through service or downtime actions.

\subsection{Rites Difficulty Value}
\label{sec:rites-dv-expanded}
\index{Rites!Difficulty Value}
\index{DV}

The Difficulty Value (DV) to cast a Rite is:

\[
\text{DV} = \max\!\big(\text{Obligation Cost} - \text{Spirit}, \, \text{Tier}\big)
\]

\begin{description}
  \item[Obligation Cost:] The Rite’s listed cost in Obligation segments. This reflects the Patron’s toll for the magic.
  \item[Spirit:] The caster’s Spirit attribute. Each point reduces the effective weight of the Obligation, representing inner resilience and willpower.
  \item[Tier:] The Rite’s intrinsic difficulty based on scope or potency. DV can never fall below this floor.
\end{description}

\subsection{Invokers (Symbol Path)}
\paragraph{Patron’s Symbol}
\begin{itemize}
  \item \textbf{Minor Asset, 4 XP each.}
  \item Each Symbol is consecrated to one Patron and grants ritual access to that Patron’s Rites. 
  \item You may hold multiple Symbols, one per Patron.
\end{itemize}

\paragraph{Rite Invocation via Symbol}
\begin{itemize}
  \item \textbf{Time.} Invoking a Rite via Symbol takes \(\text{DV} + 1\) rounds.
  \item \textbf{Obligation.} On completion, mark +1 Obligation (in addition to any listed Rite costs, if applicable).
  \item \textbf{No Push.} Invoker Rites cannot use \emph{Push It} benefits.
  \item \textbf{Symbol Display.} The Symbol must remain visible throughout the invocation.
  \item \textbf{Materials.} Symbols replace any Thaisos and Codex requirements.
\end{itemize}

\paragraph{Crack the Seal (Instant Cast)}
As part of an Invoker Rite, you may immediately resolve the effect by setting the Symbol to \emph{Compromised} and marking +2 Obligation segments (+3 if High-Power). The GM may spend 1 SB on-theme. The Symbol remains but must be restored in downtime.

\paragraph{Restoring Symbols}
A Compromised Symbol is inert until repaired. Use a downtime action and test (DV 3 or fiction-appropriate). Success restores it; a shaky result leaves it Neglected (rituals work but cost +1 Obligation).

\paragraph{Invoker Path Limitations}
\begin{itemize}
  \item Cannot Push. 
  \item Max simultaneous rituals = Spirit. Starting a new ritual ends the oldest or adds +1 Obligation to it. 
  \item Carrying 4+ Symbols causes interference: the first ritual each scene marks +1 extra Obligation.
\end{itemize}

\subsection{Patron’s Gift (Imbuements)}
\textbf{Patron’s Gift (Free, Requires Thiasos)}\\
Duration: Scene; Range: Touch; Stacking: No.\\
Effect: Imbue one item with +1 Weapon (Melee) and +1 Thematic Skill (Patron domain) for the scene.\\
Activation: Requires 1 Action once per scene.\\
Push It: The item’s power persists for one additional scene but marks +1 Obligation.\\
Requires: Familiar (Invoke: 1 Boon).

\subsection{Mixing the Paths}
Players may combine Casting, Rites, and Invoking, but each path introduces its own bookkeeping:
\begin{itemize}
  \item Casters track Backlash. 
  \item Rites users track Obligation. 
  \item Invokers track Symbol states (Maintained, Neglected, Compromised). 
\end{itemize}
Mixing provides flexibility but less efficiency than specialization. Specialists gain stronger benefits, while mixers gain narrative breadth.

% !TEX root = srd_main.tex
% SRD Insert: Elemental Backlash (Condensed SRD Version)

\section{Elemental Backlash (Condensed)}\label{sec:backlash-condensed}
\index{Backlash}\index{Elements}\index{Story Beats}

Magic unsettles the weave. Each element (and its counterpart) carries a distinct backlash pattern. When a roll shows a 1 (generating a (SB)), or when a player accepts a (SB) to escalate, apply a \textbf{Minor} backlash. Players may opt to escalate to \textbf{Major} by taking +1 (SB).

\begin{table}[h]
\centering
\caption{Backlash at a Glance}
\label{tab:backlash-condensed}
\renewcommand{\arraystretch}{1.12}
\begin{tabularx}{\linewidth}{>{\bfseries}l >{\raggedright}X >{\raggedright}X}
\toprule
Element & Minor Backlash & Major Backlash \
\midrule
Earth / Fate & Slips, binds, encumbrance • –1 Position or \textsc{Encumbered}. & Fissure, entrapment • Clock +1 (Collapse) or \textsc{Pinned}. \
Fire / Life & Smoke, sparks, heat • –1 Effect or \textsc{Singed}. & Blaze, fever, ignition • Clock +1 (Fire) or 1 Harm. \
Air / Luck & Scatter, misheard words • –1 Position or Clock +1/2 (Attention). & Unlikely mishap • Lose a tool/use or (SB) +1. \
Water / Dreams (Obishaal) & Slippery tide, slow gear • –1 Effect or \textsc{Waterlogged}. & Undertow, veering path • Clock +1 (Flood) or intrusion from Ways Between. \
Fate / Earth & Probability resists • –1 Effect or Clock +1/2 (Inevitable). & Demand arrives • Immediate sacrifice or (SB) +1 & mark \textsc{Omen}. \
Life / Fire & Growth surge, vines tether • –1 Effect or \textsc{Overgrowth}. & Riot of life • Clock +1 (Biohazard) or convert healing to (SB) +1. \
Luck / Air & Odds flip • –1 Position or Clock +1/2 (Coincidence). & Catastrophic fluke • Force re-roll; if any 1, (SB) +1 and Minor repeats. \
Death / Water (Obishaal) & Whispers, chill • \textsc{Shaken} or Clock +1/2 (Haunting). & Threshold opens • Clock +1 (Crossing Due) or revenant intrusion. \
\bottomrule
\end{tabularx}
\end{table}

\begin{tcolorbox}[title={Cheatsheet},colback=gray!5,colframe=black]
\small Minor = wobble; Major = lurch. Apply once per cast. Offer players the option to escalate to Major by taking (SB) +1.\par\smallskip
Earth/Fate binds; Fire/Life burns or grows; Air/Luck scatters or flips; Water/Obishaal pulls or opens.\end{tcolorbox}
% !TEX root = srd_main.tex
% SRD Insert: Universal Rituals — Quick-Start Spread (SRD concise)
% Assumes: booktabs, tabularx, xcolor, tcolorbox, index set up

\section{Universal Rituals (Quick-Start)}\label{sec:universal-rituals}
\index{Rituals}\index{Story Beats}\index{Backlash}\index{Realms}\index{Obishaal@Obishaal}

These table-ready rituals are system-agnostic and available to any chassis that can perform rituals. Each lists \textbf{Cast Time}, \textbf{Setup/Components}, \textbf{Effect}, and explicit \textbf{Costs/(SB) hooks}. GMs should reskin names freely to match patrons, runes, symbols, or tag-sets.

\begin{tcolorbox}[title={Ritual Casting Basics},colback=gray!5,colframe=black]
\textbf{Triggering Risk.} On any ritual roll showing a 1, gain a (SB) and apply elemental Backlash (\S\ref{sec:backlash-condensed}). Players may accept +1 (SB) to push an effect one step (position/effect/scale) if fictionally supported.\index{Story Beats!rituals}
\end{tcolorbox}

% –––––––––– TABLE ––––––––––
\begin{table}[h]
\centering
\caption{Rituals at a Glance}
\label{tab:rituals-quick}
\renewcommand{\arraystretch}{1.12}
\begin{tabularx}{\linewidth}{>{\bfseries}l c l X X}
\toprule
Name & Tier & Cast Time & Setup & Components & Effect (with Costs/(SB) Hooks) \
\midrule
Wayfinder’s Thread & Low & 1 minute & Red cord knotted thrice; whisper a destination. & Create a faint tether toward the nearest safe path. \emph{Cost:} mark \textsc{Fatigue} if used more than once/scene. \emph{Push:} +1 (SB) to reveal a hidden shortcut (Clock –1/2 on Travel). \
Oath-Ward & Low & 5 minutes & Chalk circle; sworn phrase all participants repeat. & Ward a small area vs. intrusion (mundane/lesser). \emph{Cost:} requires sincere oath; breaking it triggers (SB) +1 and ends ward. \
Ember-Glass & Low & 1 minute & Hold an ember behind smoked glass. & Sense nearby heat sources/life signs through cover. \emph{Cost:} lose one use of a tinder/torch. \emph{Push:} +1 (SB) to pierce thin walls. \
Salt-Cut & Low & 1 minute & Salt line and bronze knife. & Sever a simple ongoing effect (rope-binds, minor charm). \emph{Cost:} consume 1 use of salt. \emph{Push:} +1 (SB) to cut a tougher link (Clock –1/2 on Restraint/Hex). \
River’s Memory & Med & 10 minutes & Bowl of water and a personal token. & Scry a recent passage/event tied to the token, brief and blurry. \emph{Cost:} token is waterlogged/ruined. \emph{Push:} +1 (SB) for a clearer second image. \
Bargain-Bead & Med & 10 minutes & Two carved beads; one is offered openly. & Invite a nearby power/spirit to parley. \emph{Cost:} give up a valuable concession now or take (SB) +1 when you refuse. \
Quiet Veil & Med & 5 minutes & Ash across lips; bell muted in cloth. & Muffle a group’s sound and scent for a scene. \emph{Cost:} \textsc{Muted} Condition (social checks –1) until scene ends. \emph{Backlash:} Air/Luck. \
Shadow-Loom & Med & 5 minutes & Three pins; weave ambient shadow between them. & Create light-obscuring cover or misdirection in a small zone. \emph{Cost:} dim your own vision (–1 precision) while maintained. \emph{Push:} +1 (SB) to mirror a decoy image briefly. \
Dream-Way Marker & Med & 10 minutes & Sleep mask inked with a circle; water drip cadence. & Mark a safe entrance to the Ways Between; next sleep at site allows short transit. \emph{Cost:} all participants mark \textsc{Shaken} on waking. \emph{Backlash:} Death/Obishaal. \
Purge & Med & 10 minutes & Smoke of bitter herbs; clean blade drawn across incense. & Cleanse taint/disease/curse one step. \emph{Cost:} cleanse passes a lesser echo to the caster (–1 to a related action next scene). \emph{Push:} +1 (SB) to remove two steps but take \textsc{Weakened}. \
Fortune-Braid & High & 15 minutes & Three strands (hair, thread, wire) braided tight. & Bank a single lucky break: replace one die with its highest result this scene. \emph{Cost:} immediately take (SB) +1 if used offensively. \emph{Backlash:} Air/Luck. \
Fate-Splice & High & 15 minutes & Knot two names written on vellum. & Temporarily link two fates: transfer a single consequence/boon between them. \emph{Cost:} both bear a subtle mark until dawn; \emph{Push:} +1 (SB) to redirect a Major consequence. \emph{Backlash:} Fate/Earth. \
Summoner’s Gate & High & 20 minutes & Circle inscribed with true-name sigil or emblem. & Call a known entity safely; on success it arrives bound by a simple charge. \emph{Cost:} occupies one concurrency slot; breaking terms creates (SB) +1 and Disruption. \emph{Backlash:} varies by entity. \
\bottomrule
\end{tabularx}
\end{table}

\subsection*{Usage Notes}\label{subsec:ritual-usage}
\begin{itemize}
\item \textbf{Scaling.} Effects scale by position/effect/area via explicit (SB) offers or extra time/components.\index{Rituals!scaling}
\item \textbf{Elements.} Choose the dominant element by fiction (Fire for Ember-Glass; Water/Obishaal for Dream-Way) and apply the condensed backlash table (\S\ref{sec:backlash-condensed}).\index{Backlash!elements}
\item \textbf{Teamwork.} Extra participants can donate narrative components to reduce cast time \emph{or} to accept (SB) on the caster’s behalf once per ritual.\index{Teamwork}
\end{itemize}

\begin{tcolorbox}[title={Design Intent},colback=gray!5,colframe=black]
Each ritual bakes in a crisp \emph{cost}, a tempting \emph{push}, and a likely \emph{backlash}. Keep it fiction-first: components are story handles the GM can threaten, not bookkeeping chores.\end{tcolorbox}


\section{The Magic System}

Magic in Fate's Edge is expressed through three interconnected paths. 
You may specialize in one, or mix them at greater bookkeeping cost. 
All paths share the same dice engine and SB/Obligation economies, but their flavor and risks differ.

\subsection{The Many Faces of Magic}
\begin{description}[leftmargin=1.5em, style=nextline]
  \item[Caster (Freeform):] Requires \textbf{Talent: Caster’s Gift (2 XP)}. Grants access to Weave \& Cast using the Eight Elements. Flexible, creative, and risky (Backlash on 1s). 
  \item[Rites User (Runekeeper):] Requires \textbf{Patron + Thiasos (Familiar)}. Grants access to a Patron’s Rites. Structured, powerful, but debt-driven through \textbf{Obligation}.
  \item[Invoker (Symbol Path):] Requires one or more \textbf{Patron’s Symbols (4 XP each)}. Grants access to that Patron’s Rites via rituals. Safe but slow; can \emph{Crack the Seal} to cast instantly at steep Obligation cost.
\end{description}

\subsection{Caster (Freeform/Weave \& Cast) - Balanced Mechanics}

\textbf{Core Concept:} Flexible, improvisational magic that shapes raw elemental forces. Power comes with inherent instability and risk.

\textbf{Mechanics:}

\begin{itemize}
  \item \textbf{Prerequisite:} Caster's Gift (2 XP Talent).
  \item \textbf{Process:} Two-Action casting loop: \textbf{Weave} (shape) $\rightarrow$ \textbf{Cast} (release).
  \item \textbf{Description:} Player narrates the desired effect, framing it through the \textbf{Eight Elements} (Earth, Air, Fire, Water, Fate, Luck, Life, Death/Dreams). The GM sets the \textbf{Difficulty Value (DV)} and the base \textbf{Effect} based on scope and complexity.
  \item \textbf{Elemental Focus:} The caster typically focuses on one or two primary elements for their Art (e.g., Fire/Destruction, Water/Healing, Air/Movement, Earth/Protection). Effects strongly aligned with these elements might gain a minor benefit (e.g., +1 Boon or reduced Backlash severity, as detailed below).
\end{itemize}

\textbf{Casting Loop Details:}

\begin{enumerate}
  \item \textbf{Weave (Action 1):}
    \begin{itemize}
      \item \textbf{Roll:} Attribute + Arcana (or relevant skill if justified by Art).
      \item \textbf{Outcome:}
        \begin{itemize}
          \item \textbf{Success:} The spell's form is stabilized. It is ready to be Cast.
          \item \textbf{Partial/Miss:} The shaping fails. No effect occurs, but Backlash (see below) still applies based on this roll.
          \item \textbf{Backlash:} If one or more 1s are rolled, the GM notes the number and the element(s) involved. Resolve Backlash \textit{after} the Cast roll.
        \end{itemize}
    \end{itemize}
    
  \item \textbf{Cast (Action 2 - Performed only if Weave Succeeds):}
    \begin{itemize}
      \item \textbf{Roll:} Attribute + Arcana (or relevant skill).
      \item \textbf{Outcome:}
        \begin{itemize}
          \item \textbf{Success:} The spell is successfully channeled into the world, producing the intended Effect.
          \item \textbf{Partial/Miss:} The spell partially manifests or misfires. Apply the intended Effect at a reduced level (e.g., -1 step) or with a significant drawback. The caster still suffers Backlash.
          \item \textbf{Backlash:} If one or more 1s are rolled, the GM notes the number and element(s). Combine these 1s with any 1s from the Weave roll for the Backlash Severity check.
        \end{itemize}
    \end{itemize}
\end{enumerate}

\textbf{Backlash System:}

\begin{itemize}
  \item \textbf{Trigger:} Any 1 rolled on either the \textbf{Weave} or \textbf{Cast} roll. Count the total number of 1s from both rolls for the current spell.
  \item \textbf{Severity Check:} Based on the total number of 1s and the primary Element(s) involved.
    \begin{itemize}
      \item \textbf{1 One (Minor Backlash):}
        \begin{itemize}
          \item \textbf{Effect:} Flavorful, minor inconvenience or environmental effect related to the element and its opposite.
          \item \textbf{Examples:}
            \begin{itemize}
              \item \textbf{Fire:} Smoke, sparks, sudden heatwave, minor burn (Fatigue).
              \item \textbf{Water:} Sudden dampness, minor slip, condensation forming.
              \item \textbf{Earth:} Dust cloud, minor tremor, instability (Position -1 if balancing).
              \item \textbf{Air:} Gust dispersing papers, sudden noise, dizziness.
              \item \textbf{Fate:} A nearby decision point feels ``wrong'' or ``forced.''
              \item \textbf{Luck:} A nearby ally suffers a minor stroke of bad luck (tripped, dropped item).
              \item \textbf{Life:} Brief fatigue, minor pain, plant nearby wilts slightly.
              \item \textbf{Death/Dreams:} A shiver, a fleeting dark vision, silence falls for a moment.
            \end{itemize}
        \end{itemize}
      \item \textbf{2-3 Ones (Moderate Backlash):}
        \begin{itemize}
          \item \textbf{Effect:} Clear mechanical penalty or harmful effect related to the element and its opposite.
          \item \textbf{Options (GM Choice or Player Narrative Fit):}
            \begin{itemize}
              \item Apply a \textbf{Condition} (Dazed, Shaken, Exposed, etc.).
              \item Impose \textbf{-1d} on the caster's next relevant roll.
              \item Inflict \textbf{Harm 1} (typically Minor, e.g., a burn, bruise, shock).
              \item Trigger a \textbf{Story Beat} spend by the GM (related to the element or situation).
              \item Cause a \textbf{minor environmental hazard} (fire catches a curtain, water spills, earth shifts, etc.).
            \end{itemize}
        \end{itemize}
      \item \textbf{4+ Ones (Severe Backlash):}
        \begin{itemize}
          \item \textbf{Effect:} Significant mechanical consequence or narrative disruption.
          \item \textbf{Options:}
            \begin{itemize}
              \item Inflict \textbf{Harm 2} (Moderate).
              \item Apply a \textbf{persistent Condition} or a \textbf{severe Condition} (Impaired, Panicked, etc.).
              \item Cause a \textbf{major environmental change} or hazard.
              \item Trigger a \textbf{significant Story Beat} spend by the GM (major complication).
              \item The spell catastrophically misfires, potentially harming allies or creating an unintended, powerful (but likely uncontrolled) magical effect related to the elements involved.
            \end{itemize}
        \end{itemize}
    \end{itemize}
  \item \textbf{Art Specialization Mitigation:} If the spell's effect is strongly aligned with the caster's defined Magical Art (elements/themes), they may reduce the Backlash severity by one step (Severe $\rightarrow$ Moderate, Moderate $\rightarrow$ Minor, Minor $\rightarrow$ \textit{Narrative Only} flavor). This represents mastery making the magic feel more ``natural'' and less likely to spiral out of control.
\end{itemize}

\textbf{Key Clarifications:}

\begin{itemize}
  \item \textbf{DV Setting:} The GM sets the DV based on the scope and complexity of the intended effect, informed by the elements used. A simple Fire bolt is DV 3. A complex Fate/Luck weave to ensure victory in a tournament might be DV 5.
  \item \textbf{Effect Definition:} The base Effect is also set by the GM based on the description. ``I blast him with fire'' is different from ``I create a wall of fire.'' The player should be clear about the desired outcome.
  \item \textbf{TAGS (Optional):} For games using the TAGS system, the GM and player can agree on relevant TAGS to define the spell's specific mechanical impacts, which can also inform DV and Backlash potential (e.g., a spell with [AREA] or [DISPEL] might have a higher DV or Backlash risk).
\end{itemize}

\subsection{Rites Users (Runekeepers)}
\paragraph{Requirements}
A Patron bond, a Thiasos (Familiar; small spirit in the form of a creature/construct), and a Codex (4 XP) mark a character as a Runekeeper. 

\paragraph{Invocation}
\begin{itemize}
  \item \textbf{Action Cost:} Invoking a Rite requires 1 Action. 
  \item \textbf{Obligation:} Each Rite used marks Obligation on its clock. 
  \item \textbf{Push It:} Once per Rite, you may Push to increase its duration or potency by +1 step at the cost of +1 Obligation.
\end{itemize}

\paragraph{Obligation Clock}
Tracks the Patron’s claim. When full, the GM resolves the debt in-fiction. Obligation is reduced through service or downtime actions.

\subsection{Rites Difficulty Value}
\label{sec:rites-dv-expanded}
\index{Rites!Difficulty Value}
\index{DV}

The Difficulty Value (DV) to cast a Rite is:

\[
\text{DV} = \max\!\big(\text{Obligation Cost} - \text{Spirit}, \, \text{Tier}\big)
\]

\begin{description}
  \item[Obligation Cost:] The Rite’s listed cost in Obligation segments. This reflects the Patron’s toll for the magic.
  \item[Spirit:] The caster’s Spirit attribute. Each point reduces the effective weight of the Obligation, representing inner resilience and willpower.
  \item[Tier:] The Rite’s intrinsic difficulty based on scope or potency. DV can never fall below this floor.
\end{description}

\begin{fatebox}[Invoker Path Features]
  \begin{tabularx}{\textwidth}{lX}
  \toprule
  \textbf{Feature} & \textbf{Description and Limitations} \\
  \midrule
  Invoker's Grimoire & Major Talent, 6 XP. Grants knowledge of Ritual Magic theory and access to perform a limited number of Rites. \\
  Ritual Repertoire & Start with knowledge of \textbf{2} Low or Standard Rites from any Patrons you research. Learn new Rites through Downtime study (see below). \\
  Ritual Invocation & Takes \(\text{DV}\) rounds (default 2--3 rounds). Requires specific components/materials. \\
  Base Cost & Mark \textbf{+1 Obligation} when you successfully resolve any known Rite (Low or Standard). \textit{(High-Power/High Rites are normally unavailable; if the Keeper permits, treat their \emph{base} Obligation as +2.)}\\
  Symbol Enhancement & Possessing the correct Patron's Symbol for a Rite you are casting reduces its \textbf{DV by 1} and its \textbf{Obligation cost by 1} (minimum 0). Only one Symbol may apply to a given Rite. \\
  \textbf{No Symbol (Explicit Penalties)} & You may attempt the Rite without the Patron's Symbol, but suffer: \textbf{+1 DV}, \textbf{+1 Obligation} (in addition to Base), and \textbf{+1 round} casting time. On \emph{Partial/Failure}, generate \textbf{+1 extra SB}. \\
  Symbol Display & The Symbol must be visible/active throughout the ritual. If it is concealed, disrupted, or removed mid-cast: immediately \textbf{+1 DV}; on Failure, apply \emph{Backlash} (see below). \\
  Crack the Seal & Desperate technique. Instantly cast any known Rite by setting the relevant Symbol to \textsc{Compromised}. Mark \textbf{+2 Obligation} (\textbf{+3} for High-Power Rites). Does not reduce Base Obligation below 0. \\
  Optional Push & Invokers may \emph{Push} a Rite: choose one (\(+2\) dice \emph{or} +1 Effect \emph{or} resolve one round faster). Always mark \textbf{+1 Obligation} \emph{and} generate \textbf{1 SB}, in addition to other costs. \\
  Cross-Resonance & If you cast Rites from \emph{different Patrons} in the same scene, each Patron after the first adds \textbf{+1 DV} to that Rite. \\
  \bottomrule
  \end{tabularx}
  \end{fatebox}
  
  \paragraph{Symbol States \& Repair}
  \begin{itemize}
    \item \textsc{Compromised:} A Symbol set to \textsc{Compromised} (e.g., via \emph{Crack the Seal}) provides \emph{no} DV/Obligation reduction until repaired. Casting with a \textsc{Compromised} Symbol imposes \(-1\) die on the Casting Test.
    \item \textsc{Shattered:} If you \emph{Crack the Seal} again while the Symbol is \textsc{Compromised}, it becomes \textsc{Shattered} and cannot be used until replaced (Asset lost).
    \item \textbf{Repair (Downtime):} 1 day of focused work and a \emph{Craft or Lore + Tinker} test vs.\ DV~3. Success: clear \textsc{Compromised}. Failure: no progress. Alternatively, spend \textbf{1 XP} to auto-repair.
  \end{itemize}
  
  \paragraph{Backlash \& Failure (Explicit)}
  \begin{itemize}
    \item \textbf{Success:} Rite resolves; apply Base/added Obligation and any SB from Push or No-Symbol clauses.
    \item \textbf{Partial:} Effect \(-1\) step \emph{or} shortened duration; mark \textbf{Fatigue 1}. If cast \emph{without} a Symbol, Keeper gains \textbf{+1 SB} (in addition to normal SB generation).
    \item \textbf{Failure:} No effect; mark \textbf{Fatigue 1}; Keeper gains \textbf{+1 SB}. Then test \emph{Spirit + Resolve} vs.\ DV~3:
      \begin{itemize}
        \item On Fail: suffer \textbf{Harm 1 (Shock)} or start \textbf{Backlash Static [4]} (Keeper's choice).
        \item If the Symbol was disrupted/hidden mid-cast \emph{or} you \emph{Cracked the Seal}: upgrade to \textbf{Harm 2 (Shock)}.
      \end{itemize}
    \item \textbf{Interrupted:} Harm, Silence, or disruption before resolution counts as \emph{Failure}.
  \end{itemize}
  
  \textbf{Example:} Magus Vex, bearing the \textbf{Invoker's Grimoire}, has studied the rites of Raéyn and the Sealed Gate. He knows Raéyn's \emph{Whispering Currents} (Low) and the Sealed Gate's \emph{Circle of Denial} (Standard). Faced with a collapsing tunnel, he attempts the Sealed Gate's ritual. It's a Standard Rite, so \textbf{DV 3}, taking \textbf{3 rounds}, and costs \textbf{+1 Obligation}. He has the Sealed Gate's Symbol, reducing the DV to \textbf{2} and the Obligation cost to \textbf{0}. When ambushed, he needs quick protection. He \textbf{Cracks the Seal} on the \emph{Circle of Denial}. The Symbol becomes \textsc{Compromised}, the Rite is instant, and he marks \textbf{+2 Obligation}. Later, needing to bind a particularly strong foe, he \textbf{Pushes} the Rite, marking an additional \textbf{+1 Obligation} and generating \textbf{1 SB}; the barrier strengthens. If he tried a Raéyn Rite afterwards in the same scene, \emph{Cross-Resonance} would add \textbf{+1 DV} to that casting.
  
  \subsubsection*{Learning New Rites}
  An Invoker can expand their \textbf{Ritual Repertoire} through dedicated study during \textbf{Downtime}.
  \begin{itemize}
      \item \textbf{Cost:} 1 week of Downtime + 2 XP.
      \item \textbf{Requirement:} Access to texts, a teacher, or direct observation of the Rite being performed by another adept.
      \item \textbf{Test:} \emph{Lore + Investigation} (or a relevant skill) vs.\ DV~3--5 (based on Rite rarity/complexity).
      \item \textbf{Success:} Add the Rite to your Ritual Repertoire.
      \item \textbf{Failure:} Cannot learn this specific Rite for a significant time (GM discretion). The Keeper may set a relevant Complication (e.g., \emph{Forbidden Knowledge Pursued}).
  \end{itemize}
  
  \subsubsection*{Symbols as Assets}
  \begin{itemize}
      \item A Patron's Symbol is a \textbf{Minor Asset (4 XP)} whose primary value is as a \textbf{ritual focus/component}.
      \item You \emph{can} attempt any ritual \textbf{without} the Symbol, but you incur these \textbf{No Symbol} penalties: \textbf{+1 DV} \emph{(and therefore +1 round to cast, since casting time = DV rounds)}, \textbf{+1 Obligation} \emph{(in addition to Base)}, and on \emph{Partial/Failure} the Keeper gains \textbf{+1 extra SB}.    \item Symbols can be \textbf{maintained/upgraded} like other Assets. Example upgrades: \emph{Hardened} (ignore the first application of \textsc{Compromised} per session), \emph{Bright} (treat as \emph{visible} for Symbol Display while concealed on your person).
  \end{itemize}
  
  \subsection*{Borrowed Grace}
  \label{talent:borrowed-grace}
  \index{Talents!Invoker}\index{Imbuement!Lesser}
  
  \textbf{Type:} Invoker Talent — \textit{Lesser Imbuement}
  
  \subsubsection*{Use}
  \begin{itemize}
    \item \textbf{Cost:} \textbf{1 Boon}, \textbf{1 action}.
    \item \textbf{Effect (pick one on use):} \textbf{+1 Melee} \emph{or} \textbf{+1 Thematic} (your table's signature/thematic Skill).
    \item \textbf{Duration:} \textit{Single action/attack} (instantaneous boost only).
    \item \textbf{Requirement:} Wield/display the relevant Patron's \textbf{Symbol}.
    \item \textbf{Obligation:} Immediately mark \textbf{+1 Obligation} to that Patron (see \S\ref{sec:obligation}).
    \item \textbf{Limits:} Cannot be extended, stacked, or \emph{Pushed} for duration. Using \emph{Borrowed Grace} while the Symbol is \textsc{Compromised} lowers your \textbf{Position} by one step \emph{(or imposes \(-1\) die if already \textbf{Desperate})}.)
  \end{itemize}

\subsection{Patron’s Gift (Imbuements)}
\textbf{Patron’s Gift (Free, Requires Thiasos)}\\
Duration: Scene; Range: Touch; Stacking: No.\\
Effect: Imbue one item with +1 Weapon (Melee) and +1 Thematic Skill (Patron domain) for the scene.\\
Activation: Requires 1 Action once per scene.\\
Push It: The item’s power persists for one additional scene but marks +1 Obligation.\\
Requires: Familiar (Invoke: 1 Boon).

\subsection{Mixing the Paths}
Players may combine Casting, Rites, and Invoking, but each path introduces its own bookkeeping:
\begin{itemize}
  \item Casters track Backlash. 
  \item Rites users track Obligation. 
  \item Invokers track Symbol states and Obligation (Maintained, Neglected, Compromised). 
\end{itemize}
Mixing provides flexibility but less efficiency than specialization. Specialists gain stronger benefits, while mixers gain narrative breadth.

% !TEX root = srd_main.tex
% SRD Insert: Elemental Backlash (Condensed SRD Version)

\section{Elemental Backlash (Condensed)}\label{sec:backlash-condensed}
\index{Backlash}\index{Elements}\index{Story Beats}

Magic unsettles the weave. Each element (and its counterpart) carries a distinct backlash pattern. When a roll shows a 1 (generating a (SB)), or when a player accepts a (SB) to escalate, apply a \textbf{Minor} backlash. Players may opt to escalate to \textbf{Major} by taking +1 (SB).

\begin{table}[h]
\centering
\caption{Backlash at a Glance}
\label{tab:backlash-condensed}
\renewcommand{\arraystretch}{1.12}
\begin{tabularx}{\linewidth}{>{\bfseries}l >{\raggedright}X >{\raggedright}X}
\toprule
Element & Minor Backlash & Major Backlash \
\midrule
Earth / Fate & Slips, binds, encumbrance • –1 Position or \textsc{Encumbered}. & Fissure, entrapment • Clock +1 (Collapse) or \textsc{Pinned}. \
Fire / Life & Smoke, sparks, heat • –1 Effect or \textsc{Singed}. & Blaze, fever, ignition • Clock +1 (Fire) or 1 Harm. \
Air / Luck & Scatter, misheard words • –1 Position or Clock +1/2 (Attention). & Unlikely mishap • Lose a tool/use or (SB) +1. \
Water / Dreams (Obishaal) & Slippery tide, slow gear • –1 Effect or \textsc{Waterlogged}. & Undertow, veering path • Clock +1 (Flood) or intrusion from Ways Between. \
Fate / Earth & Probability resists • –1 Effect or Clock +1/2 (Inevitable). & Demand arrives • Immediate sacrifice or (SB) +1 & mark \textsc{Omen}. \
Life / Fire & Growth surge, vines tether • –1 Effect or \textsc{Overgrowth}. & Riot of life • Clock +1 (Biohazard) or convert healing to (SB) +1. \
Luck / Air & Odds flip • –1 Position or Clock +1/2 (Coincidence). & Catastrophic fluke • Force re-roll; if any 1, (SB) +1 and Minor repeats. \
Death / Water (Obishaal) & Whispers, chill • \textsc{Shaken} or Clock +1/2 (Haunting). & Threshold opens • Clock +1 (Crossing Due) or revenant intrusion. \
\bottomrule
\end{tabularx}
\end{table}

\begin{tcolorbox}[title={Cheatsheet},colback=gray!5,colframe=black]
\small Minor = wobble; Major = lurch. Apply once per cast. Offer players the option to escalate to Major by taking (SB) +1.\par\smallskip
Earth/Fate binds; Fire/Life burns or grows; Air/Luck scatters or flips; Water/Obishaal pulls or opens.\end{tcolorbox}
% !TEX root = srd_main.tex
% SRD Insert: Universal Rituals — Quick-Start Spread (SRD concise)
% Assumes: booktabs, tabularx, xcolor, tcolorbox, index set up

\section{Universal Rituals (Quick-Start)}\label{sec:universal-rituals}
\index{Rituals}\index{Story Beats}\index{Backlash}\index{Realms}\index{Obishaal@Obishaal}

These table-ready rituals are system-agnostic and available to any chassis that can perform rituals. Each lists \textbf{Cast Time}, \textbf{Setup/Components}, \textbf{Effect}, and explicit \textbf{Costs/(SB) hooks}. GMs should reskin names freely to match patrons, runes, symbols, or tag-sets.

\begin{tcolorbox}[title={Ritual Casting Basics},colback=gray!5,colframe=black]
\textbf{Triggering Risk.} On any ritual roll showing a 1, gain a (SB) and apply elemental Backlash (\S\ref{sec:backlash-condensed}). Players may accept +1 (SB) to push an effect one step (position/effect/scale) if fictionally supported.\index{Story Beats!rituals}
\end{tcolorbox}

% –––––––––– TABLE ––––––––––
\begin{table}[h]
\centering
\caption{Rituals at a Glance}
\label{tab:rituals-quick}
\renewcommand{\arraystretch}{1.12}
\begin{tabularx}{\linewidth}{>{\bfseries}l c l X X}
\toprule
Name & Tier & Cast Time & Setup & Components & Effect (with Costs/(SB) Hooks) \
\midrule
Wayfinder’s Thread & Low & 1 minute & Red cord knotted thrice; whisper a destination. & Create a faint tether toward the nearest safe path. \emph{Cost:} mark \textsc{Fatigue} if used more than once/scene. \emph{Push:} +1 (SB) to reveal a hidden shortcut (Clock –1/2 on Travel). \
Oath-Ward & Low & 5 minutes & Chalk circle; sworn phrase all participants repeat. & Ward a small area vs. intrusion (mundane/lesser). \emph{Cost:} requires sincere oath; breaking it triggers (SB) +1 and ends ward. \
Ember-Glass & Low & 1 minute & Hold an ember behind smoked glass. & Sense nearby heat sources/life signs through cover. \emph{Cost:} lose one use of a tinder/torch. \emph{Push:} +1 (SB) to pierce thin walls. \
Salt-Cut & Low & 1 minute & Salt line and bronze knife. & Sever a simple ongoing effect (rope-binds, minor charm). \emph{Cost:} consume 1 use of salt. \emph{Push:} +1 (SB) to cut a tougher link (Clock –1/2 on Restraint/Hex). \
River’s Memory & Med & 10 minutes & Bowl of water and a personal token. & Scry a recent passage/event tied to the token, brief and blurry. \emph{Cost:} token is waterlogged/ruined. \emph{Push:} +1 (SB) for a clearer second image. \
Bargain-Bead & Med & 10 minutes & Two carved beads; one is offered openly. & Invite a nearby power/spirit to parley. \emph{Cost:} give up a valuable concession now or take (SB) +1 when you refuse. \
Quiet Veil & Med & 5 minutes & Ash across lips; bell muted in cloth. & Muffle a group’s sound and scent for a scene. \emph{Cost:} \textsc{Muted} Condition (social checks –1) until scene ends. \emph{Backlash:} Air/Luck. \
Shadow-Loom & Med & 5 minutes & Three pins; weave ambient shadow between them. & Create light-obscuring cover or misdirection in a small zone. \emph{Cost:} dim your own vision (–1 precision) while maintained. \emph{Push:} +1 (SB) to mirror a decoy image briefly. \
Dream-Way Marker & Med & 10 minutes & Sleep mask inked with a circle; water drip cadence. & Mark a safe entrance to the Ways Between; next sleep at site allows short transit. \emph{Cost:} all participants mark \textsc{Shaken} on waking. \emph{Backlash:} Death/Obishaal. \
Purge & Med & 10 minutes & Smoke of bitter herbs; clean blade drawn across incense. & Cleanse taint/disease/curse one step. \emph{Cost:} cleanse passes a lesser echo to the caster (–1 to a related action next scene). \emph{Push:} +1 (SB) to remove two steps but take \textsc{Weakened}. \
Fortune-Braid & High & 15 minutes & Three strands (hair, thread, wire) braided tight. & Bank a single lucky break: replace one die with its highest result this scene. \emph{Cost:} immediately take (SB) +1 if used offensively. \emph{Backlash:} Air/Luck. \
Fate-Splice & High & 15 minutes & Knot two names written on vellum. & Temporarily link two fates: transfer a single consequence/boon between them. \emph{Cost:} both bear a subtle mark until dawn; \emph{Push:} +1 (SB) to redirect a Major consequence. \emph{Backlash:} Fate/Earth. \
Summoner’s Gate & High & 20 minutes & Circle inscribed with true-name sigil or emblem. & Call a known entity safely; on success it arrives bound by a simple charge. \emph{Cost:} occupies one concurrency slot; breaking terms creates (SB) +1 and Disruption. \emph{Backlash:} varies by entity. \
\bottomrule
\end{tabularx}
\end{table}

\subsection*{Usage Notes}\label{subsec:ritual-usage}
\begin{itemize}
\item \textbf{Scaling.} Effects scale by position/effect/area via explicit (SB) offers or extra time/components.\index{Rituals!scaling}
\item \textbf{Elements.} Choose the dominant element by fiction (Fire for Ember-Glass; Water/Obishaal for Dream-Way) and apply the condensed backlash table (\S\ref{sec:backlash-condensed}).\index{Backlash!elements}
\item \textbf{Teamwork.} Extra participants can donate narrative components to reduce cast time \emph{or} to accept (SB) on the caster’s behalf once per ritual.\index{Teamwork}
\end{itemize}

\begin{tcolorbox}[title={Design Intent},colback=gray!5,colframe=black]
Each ritual bakes in a crisp \emph{cost}, a tempting \emph{push}, and a likely \emph{backlash}. Keep it fiction-first: components are story handles the GM can threaten, not bookkeeping chores.\end{tcolorbox}

%----------------------------------------
\section{Talent: Cantor's Path --- ``Songs of the Low Rites''}
\label{talent:cantors-path}

\begin{tcolorbox}[colback=black!3,colframe=black!40!white,title={Cantor's Path}]
You echo the liturgies of Patrons through breath and string. Not a sworn celebrant but a perilous mimic, you weave Low Rites into song. It is slower, riskier, and beautiful---but never free.
\end{tcolorbox}

\paragraph*{Type} Major Talent (8 XP) \quad
\paragraph*{Prerequisites} \textbf{Lore 1+}, \textbf{Performance 2+}, \textbf{Presence 2+} \quad
\paragraph*{Access} Any character (does not require Thiasos membership).

\subsection*{Effect}
You may learn and perform \textbf{Low Rites as Songs}. Each Song counts as knowing the associated Low Rite for performance purposes only.

\begin{itemize}
  \item \textbf{Casting Test:} \emph{Lore + Performance vs.\ DV} (default DV = 2--3).
  \item \textbf{Action Economy:} \emph{1 action to begin;} the Song \emph{resolves at the start of your next turn} unless accelerated.
  \item \textbf{Scope:} \emph{Low Rites only.} Standard/High Rites remain exclusive to Patrons and Thiasos initiates.
  \item \textbf{Costs:} Pay any \emph{materials} listed. On success you do \emph{not} mark Obligation.
\end{itemize}

\subsection*{Performance Integration}
Songs are most effective when performed as part of social performances:
\begin{itemize}
  \item \textbf{Audience Awareness:} Perform in front of 5+ observers for +1 die but +1 Corruption risk.
  \item \textbf{Cultural Context:} Appropriate venues/occasions grant +1 Effect.
  \item \textbf{Social Momentum:} Successful performances create opportunities for additional Songs in the same scene.
\end{itemize}

\subsection*{Song Repertoire Progression}
Develop a \textbf{Repertoire Clock [6]} to track learned Songs:
\begin{itemize}
  \item Mark a segment for each \emph{unique} Song learned through practice or exposure.
  \item At 2 segments: Reduce base DV of Songs by 1 (minimum 2).
  \item At 4 segments: Gain +1 die to Song performances.
  \item At 6 segments: Learn one \emph{Standard Rite as a Song} (temporary, requires ongoing practice).
\end{itemize}

\subsection*{Corruption Clock}
\begin{itemize}
  \item You gain a personal \textbf{Corruption Clock} with segments equal to your \textbf{Body} rating.
  \item \textbf{Mark Corruption when:}
    \begin{itemize}
        \item You \textbf{Push It} (Song resolves immediately).
        \item You perform a \textbf{Resonant Rite}.
        \item The Keeper spends a Story Beat involving your psionic/occult activities.
    \end{itemize}
  \item \textbf{Corruption Accumulation:} Multiple triggers may be required to mark a segment:
    \begin{itemize}
        \item \textbf{2 Push It uses} = +1 Corruption segment
        \item \textbf{1 Push It + 1 Resonant Rite} = +1 Corruption segment
        \item \textbf{3 GM SB spends} on occult activities = +1 Corruption segment
        \item \textbf{1 High Cantor Standard Rite} = +1 Corruption segment
    \end{itemize}
  \item When the Clock fills:
    \begin{itemize}
      \item You immediately gain a \textbf{thematic benefit} and \textbf{drawback} from the last Patron whose Rite you performed.
      \item All of your followers, retainers, or familiars also gain a trait of the same corruption.
      \item Reset the Clock, but it cannot go below your character's \textbf{Tier} (minimum corruption).
    \end{itemize}
  \item Corruption traits can be \textbf{Embraced} for permanent thematic advantages.
\end{itemize}

\subsection*{Thematic Corruption Benefits}
Instead of purely punitive effects, Corruption creates character-defining traits:
\begin{description}
  \item[Ikasha (Shadow):] +1 die to Stealth in shadows, but $-1$ die in bright light; always noticed by shadow-dwellers.
  \item[Inaea (Mercy):] +1 die to social manipulation, but $-1$ die when alone; compelled to offer aid to the helpless.
  \item[Isoka (Change):] +1 die to escape/transform actions, but $-1$ die to maintain consistency; physical changes become visible.
  \item[Raéyn (Sea):] +1 die to water/navigational tasks, but $-1$ die on land; attracts sea creatures.
  \item[Aveh (Freedom):] +1 die to escape/avoidance, but $-1$ die to commitments; leaves traces of passage.
\end{description}

\subsection*{Resonant Rites}
Some powerful or thematically significant Low Rites carry the weight of the Patron's direct influence. Performing these Rites is a conscious act of drawing deep power.
\begin{itemize}
    \item When learning a Song that mimics such a Rite, the GM or the rules text will designate it as \textbf{Resonant}.
    \item Performing a \textbf{Resonant Rite Song} successfully allows you to mark +1 segment on your Corruption Clock. This represents the lingering echo of power.
    \item \textbf{Choosing to Resonate} is optional. You can perform the Rite normally without marking Corruption.
    \item This choice adds a layer of strategy: is the Rite's power worth the potential long-term cost?
\end{itemize}

\subsection*{Song Synergy System}
Create combinations and interactions between Songs:
\begin{itemize}
  \item \textbf{Harmony:} Performing two compatible Songs grants +1 Effect to both.
  \item \textbf{Counterpoint:} Using opposing Songs can cancel negative effects.
  \item \textbf{Chorus:} With allies, combine Songs for amplified effects (+1 Effect per participant).
\end{itemize}

\subsection*{Outcomes}
\begin{description}
\item[Success:] The Low Rite takes effect as written.
\item[Partial:] The Rite manifests with reduced effect (one step) or shortened duration. Mark \textbf{Fatigue 1}.
\item[Failure:] No effect; mark \textbf{Fatigue 1} and the Keeper gains \textbf{+1 SB (Hearts)}.
\item[Interrupted:] Harm, Silence, or disruption before resolution = treat as Failure.
\end{description}

\subsection*{Push It}
When you Push:
\begin{itemize}
  \item The Song resolves immediately instead of next round.
  \item Mark \textbf{Fatigue 1}.
  \item \textbf{Mark toward Corruption accumulation} (see Corruption Clock).
  \item The Keeper immediately triggers a \textbf{Story Beat}, representing fallout from a Patron, the Road, or social attention.
\end{itemize}

\subsection*{Enhanced Departure Options}
\begin{itemize}
  \item \textbf{Graceful Coda:} End a Song early to gain +1 Boon and reduce Corruption accumulation progress by 1 (if any progress exists).
  \item \textbf{Lingering Verse:} Song effect continues for one round after ending, but mark +1 Fatigue.
  \item \textbf{Audience Impact:} A successful Song performance improves social Position +1 for the next interaction.
\end{itemize}

\subsection*{Limits \& Interactions}
\begin{itemize}
  \item \textbf{Stacking:} Cannot benefit from the same Rite twice.
  \item \textbf{Visibility:} Songs are inherently noticeable. On Failure or Push, assume observers take note.
  \item \textbf{Silence/Disruption:} Impose $-1$ to $-3$ dice at the Keeper's discretion.
  \item \textbf{Obligation Transference:} Whenever a Rite would normally increase Obligation, it instead increases Corruption accumulation progress.
\end{itemize}

\subsection*{Downtime Activities}
\begin{itemize}
  \item \textbf{Song Composition:} Practice and refine Songs, potentially reducing their DV or Corruption risk.
  \item \textbf{Performance Practice:} Improve Performance skill and social reputation.
  \item \textbf{Patron Study:} Research new Rites to add to your Repertoire.
  \item \textbf{Audience Building:} Cultivate followers who provide +1 die to future performances.
\end{itemize}

\subsection*{New Talents}

\subsection*{Talent: Resonant Performance (3 XP)}
\textbf{Requirements:} Cantor's Path, Performance 2+ \\
\textbf{Effect:} When performing a Song in front of an audience of 5+ people, reduce Corruption generation requirements by 1 (minimum 1 trigger) and gain +1 die to the performance.

\subsection*{Talent: Song Weaver (4 XP)}
\textbf{Requirements:} Cantor's Path, Repertoire Clock at 4+ segments \\
\textbf{Effect:} Combine two compatible Songs for +1 Effect to both. Once per scene, create Harmony between Songs for all participants.

\subsection*{Talent: Siren's Call (Major Talent - 8 XP)}
\textbf{Requirements:} Cantor's Path, Performance 3+, Repertoire 4+ \\
\textbf{Effect:} Your Songs can compel supernatural beings.
\begin{itemize}
  \item \texttt{[COMMAND]} effects work on Outsiders (Cap 3 or less)  
  \item Resistance is Spirit + Resolve vs. your Performance + Lore
  \item On success: outsider acts as commanded for one exchange
  \item On failure: generate 2 SB, outsider becomes hostile
\end{itemize}

\subsection*{Song Specialization Paths}
\begin{description}
  \item[Battle Cantor:] War Songs grant allies +1 Position in combat; Hymn of Fury converts 1 Harm to Fatigue for allies Near you; Anthem of the Fallen allows departed allies to return as spectral echoes (1/session).
  
  \item[Shadow Cantor:] Songs of Veiling create \texttt{[VEIL]} effects without ritual components; Melody of Misdirection imposes -1d to Notice rolls on enemies; Dirge of Passing enables communication with dead and scrying through recent deaths.
  
  \item[Healing Cantor:] Songs of Restoration heal +1 Harm; Chant of Purification removes poison/disease; Hymn of Vitality grants temporary +1 Body.
  
  \item[Knowledge Cantor:] Lore Songs reveal hidden knowledge; Chant of Understanding grants +2d to Investigation/Lore; Ode to Memory allows perfect recall of witnessed events.
\end{description}

\subsection*{Corruption Fading}
\label{subsec:corruption-fading}
\index{Corruption!Fading}

Corruption does not fade easily. It requires deliberate action and often, a price.
\begin{description}
  \item[\indexterm{Natural Fading}]  
  At the beginning of each Downtime, reduce a character's current \textbf{Corruption accumulation progress} by 1 step, and reduce the total \textbf{Corruption segments} by 1 (to a minimum of the character's Tier). Lingering effects persist until actively addressed.

  \medskip
  \item[\indexterm{Act of Contrition}]  
  Perform a genuine act that contradicts the Patron's influence or repairs its harm (GM/Player agreement on suitability). \textbf{Effect:} Remove 1 Corruption segment and clear one persistent effect. Costs the character something significant.

  \medskip
  \item[\indexterm{Ritual Purification}]  
  Undertake a significant act of cleansing (pilgrimage, service, seeking rival absolution). \textbf{Effect:} Remove 2 Corruption segments and clear all persistent effects. Likely requires marking Fatigue or temporary Obligation.

  \medskip
  \item[\indexterm{Embrace Corruption}] \label{talent:embrace-corruption}
  \textbf{Type:} Major Talent (6 XP) \quad
  \textbf{Prerequisite:} 2+ levels of Corruption. \\
  You accept the creeping decay, transforming it into a permanent Talent. \textbf{Embracing locks your Corruption at its current level---it reshapes it.} The deeper the corruption, the greater the power and the cost.
    \begin{itemize}
        \item Gain a \textbf{Minor} permanent thematic boon/condition related to the Patron (e.g., +1 die to Stealth in shadows for Ikasha, but $-1$ die in bright light).
        \item Your Corruption cannot naturally fade below the level at which you Embraced it.
        \item The Keeper gains +1 SB to spend against you related to that Patron's themes.
    \end{itemize}
    \textbf{Narrative Integration:} This Talent represents the Faustian bargain. Players gain agency over their corruption, ensuring that it always carries meaningful consequences.

  \medskip
  \item[\indexterm{Patron Bargain}]  
  Negotiate directly with the Patron. \textbf{Effect:} Remove 1--3 Corruption segments based on the exchange's gravity. Always comes with a narrative cost or condition set by the Keeper.

  \medskip
  \item[\indexterm{Persistence}]  
  Corruption effects do not clear through rest. They require deliberate narrative resolution or specific actions listed above. Every method is an opportunity for character development.
\end{description}

\paragraph*{High Cantor (18 XP Prestige Talent)}%
\textit{Prerequisite: Tier II+, Cantor's Path, Performance 3+}\\[3pt]
You have learned to weave the sacred tongue through breath and pulse rather than word or gesture. You may now learn and cast \textbf{Standard Rites}, as a \textbf{High Cant}.
\begin{itemize}
  \item The Rite resolves instantly.
  \item Gain +1 die to its primary effect.
  \item \textbf{Mark toward Corruption accumulation} (1 High Cantor Standard Rite = 1 Corruption trigger).
\end{itemize}

\noindent
\textbf{Special:}  
Each Patron's resonance colors the manifestation differently---flame halos for the Oath, rippling silence for the Choir, tolling harmonics for the Confessor. High Canting is recognizable to other adepts; it draws attention. Repeated use within a single scene risks moral fatigue: add +1 DV to all subsequent \emph{Resolve} rolls against fear, charm, or social pressure in that scene.

\subsection*{Divine Resonance (Major Talent - 15 XP)}
\textit{Prerequisite: High Cantor, Performance 4+, Tier III+}\\[3pt]
Your voice carries divine authority. Once per scene, spend 2 Boons:
\begin{itemize}
  \item \textbf{Command Effect:} Issue a \texttt{[COMMAND]} that affects up to (Presence) targets simultaneously
  \item \textbf{Miracle Effect:} Replicate any Low Rite without marking Corruption (but generate 1 SB)
  \item \textbf{Omen Effect:} Gain insight into a major threat - ask 3 questions about one enemy/faction
\end{itemize}
\textbf{Cost:} Mark +2 Corruption segments, immediately trigger Patron attention.

\begin{quote}
``The louder the hymn, the nearer the flame.''
\end{quote}


\subsection*{Bookkeeping Light (Table Guidance)}
To keep play fast, track at most \emph{two} clocks for a Cantor:
\begin{enumerate}
  \item \textbf{Corruption Clock} (segments = Body).
  \item \textbf{Repertoire Clock [6]} (optional; advances only when a new Song is learned).
\end{enumerate}
No per-Song timers are required beyond \emph{Push} and \emph{Outcome} handling. Harmony/Counterpoint/Chorus provide situational modifiers and never introduce new clocks.

\subsection*{Inspire Chorus}
\label{subsec:inspire-chorus}

While \emph{actively singing a Song} (from the action to begin until it resolves, or while a \emph{Lingering Verse} persists), the Cantor may \textbf{invoke Inspire Chorus}:

\begin{itemize}
  \item \textbf{Effect:} All \textbf{allies within Near} (including the Cantor) \textbf{shift Position +1} for \textbf{one exchange} (e.g., Desperate$\to$Risky, Risky$\to$Controlled). Position cannot exceed \textbf{Controlled}. This does not stack with other Position-shift auras; use the best single shift.
  \item \textbf{Use:} \textbf{Once per scene} at no cost. \textbf{Additional uses in the same scene} are allowed, but each \textbf{immediately marks +1 segment on the Cantor's Corruption Clock}.
  \item \textbf{Requirements:} The performance must be perceptible to recipients (line of hearing; \textit{Silence} or similar effects suppress it).
  \item \textbf{Timing:} Declare on starting the Song or at any time before it resolves; the shift lasts until the start of the Cantor's next turn.
  \item \textbf{Notes:} Using \emph{Inspire Chorus} does not change Song DV, Action cost, or outcomes. It respects \emph{Bookkeeping Light}: no new clocks are created.
\end{itemize}

\subsection*{Cantors as Cult Leaders (Chorus-Founders)}
Cantors gather crowds—and crowds gather debts. The Song’s Corruption stains the air, and listeners answer with vows, tithes, and favors. Many Cantors drift into leadership not by decree but by \emph{obligation}: their audience becomes a \textit{chorus} that expects guidance, protection, and more songs. In practice, the Cantor’s rising \textbf{Corruption} is mirrored by the flock’s growing \textbf{Obligation} to the Cantor (and the Patron behind the music).

\begin{fatebox}[Chorus Cult --- Quick Rule]
\begin{tabularx}{\textwidth}{lX}
\toprule
\textbf{Trigger} & After a public Song using \emph{Inspire Chorus} or a \textbf{Resonant Rite} before 10+ witnesses, the Cantor may found or deepen a \textit{Chorus} (cult). \\[2pt]
\textbf{Cost} & Immediately convert \textbf{+1 Corruption segment} into \textbf{+1 Obligation} (to the Patron or the Chorus, GM’s call). \\[2pt]
\textbf{Benefit} & Gain a \textbf{Minor Follower (Chorus)}: once/scene (if present or reachable), \textit{+1 die} to Performance/Sway \emph{or} establish a rumor/cover within the community. Scale \(\approx\) Cantor’s \textbf{Presence}. \\[2pt]
\textbf{Maintenance} & Each scene/session you leverage the Chorus, mark \textbf{+1 Obligation}. If neglected, start \textbf{Devotion Sours [4]}; on fill, the Chorus fractures into a Complication (rival sect, scandal, or betrayed devotee). \\[2pt]
\textbf{Safety Valve} & During Downtime, a \textit{Vigil} (public service, free performance, or restitution) clears \textbf{1 Obligation} to the Chorus and resets \textbf{Devotion Sours} by 1. \\
\bottomrule
\end{tabularx}
\end{fatebox}

\subsection{Paths of Magic: Complete Comparison}
\label{subsec:magic-comparison}
\index{Magic!Comparison}

Five distinct paths define supernatural power in \textsc{Fate's Edge}. Each carries a unique risk, cadence, and narrative flavor. These paths are intentionally \textit{asymmetric}—balanced through story consequences and tactical tradeoffs, not identical mechanics.

\begin{center}
\renewcommand{\arraystretch}{1.15}
\begin{tabularx}{\textwidth}{@{}l Y Y Y Y Y@{}}
\toprule
\textbf{Feature} &
\textbf{Summoner (Pact-Whisperer)} &
\textbf{Cantor's Path} &
\textbf{Caster (Freeform)} &
\textbf{Runekeeper (Rites)} &
\textbf{Invoker (Symbols)} \\
\midrule

\textbf{Core Identity} &
The \textit{Conjurer}: calls and commands spirits as allies &
The \textit{Bootlegger}: steals magic through song &
The \textit{Artist}: improvises magic via elemental will &
The \textit{Devotee}: channels a Patron's power &
The \textit{Ritualist}: works slow, precise magic via Symbols \\

\textbf{Access} &
\textit{Pact-Whisperer (2 XP)}, then Pactwright Talents &
\textit{Cantor's Path (15 XP)} &
\textit{Caster's Gift (2 XP)} &
\textit{Codex (4 XP)} + Familiar (2 XP) &
\textit{Patron's Symbol (4 XP each)} \\

\textbf{How It Works} &
Call (1 action) $\rightarrow$ Bind (Boon/Fatigue) $\rightarrow$ Command.  
Spirit acts each round, tied to a \textbf{Leash} clock &
Perform Song (1 action) $\rightarrow$ effect next beat.  
Mimics Low Rites &
Weave + Cast (2 actions). Highly flexible element magic &
Invoke Rite (1 action). Immediate supernatural effect &
Ritual Invocation (multiple rounds).
\textbf{Crack the Seal} for instant power \\

\textbf{Primary Risk} &
\textbf{Loss of Control}: fill the Leash, spirit acts independently &
\textbf{Corruption}: personal decay and aura effects &
\textbf{Backlash}: volatile elemental consequences &
\textbf{Obligation}: narrative debt owed to Patron &
\textbf{Ritual Cost}: Symbol damage or Obligation \\

\textbf{Power Source} &
Bound spirits and Outsiders &
Stolen resonance, no pact &
Personal discipline + elements &
Formal pact with a Patron &
Consecrated Symbol + precise lore \\

\textbf{Flexibility} &
\textbf{Extreme (via proxy)}: flight, phasing, stealth, combat, etc. &
Structured: mimic known Low Rites &
\textbf{Very high}: any describable effect &
Moderate: Patron Rite list &
Moderate: Symbols owned \\

\textbf{Speed} &
Fast: Spirit acts each round, but commands cost actions &
Moderate: 1 action to begin, effect next beat &
Moderate: 2 actions per spell &
\textbf{Very fast}: 1 action &
\textbf{Very slow}: multi-round rituals \\

\textbf{Key Mechanic} &
\textbf{The Leash} + Boon Finesse (clear ticks with Boons) &
Corruption Clock \& Push It &
GM-set DV \& Element choice &
Push It (gain Obligation) &
Crack the Seal (instant cast at high cost) \\

\textbf{Player Fantasy} &
\textit{The Tactician}: minion control, economy, versatility &
\textit{The Gambler}: risk-for-power, stolen magic &
\textit{The Improviser}: creative problem-solving &
\textit{The Dramatist}: pact, faith, narrative consequences &
\textit{The Planner}: preparation and precision \\
\bottomrule
\end{tabularx}
\end{center}

\paragraph{Balance by Asymmetry.}
\index{Balance}
These paths do not share identical mechanics. They are balanced narratively:
\begin{itemize}
  \item \textbf{Summoners} gain sustained power and versatility, but risk catastrophic loss of control.
  \item \textbf{Cantors} enjoy quick access to magic without a Patron, but corruption erodes them over time.
  \item \textbf{Casters} can attempt nearly anything, but risk explosive elemental backlash.
  \item \textbf{Runekeepers} unleash powerful effects instantly, but every use deepens Patron obligations.
  \item \textbf{Invokers} can safely reshape the world through ritual, but rarely in the heat of battle.
\end{itemize}

Collectively, they form a complete \textbf{pentarchy of power}—distinct, dramatic, and tactically meaningful. No path is universally superior; each shines in different challenges and story arcs.

\section*{Free Casting (TAGS System)}
Some casters do not prepare rote rites. They shape raw forces through shared arcane grammar known as \textbf{TAGS}. A spell is constructed at the table using a short phrase of TAGS. You only need the fiction, the TAG selection, and a casting roll.

\subsection*{Spell Structure}
\textbf{Intent} + \textbf{Target} + \textbf{Tags} = effect.

Example formula:
\begin{quote}
``I unleash Burning • Area • Force against the marauders.''
\end{quote}

The GM sets a Difficulty Value (DV) based on TAG complexity and danger.

\subsection*{Base Difficulty Value (DV)}
Start at DV 1 and add +1 for each TAG used.

\begin{center}
\textbf{DV = 1 + number of TAGS}
\end{center}

Adding powerful or perilous TAGS (Teleportation, Transformation, Dominate) adds +2 instead.

Mastery, focus, or appropriate tools may lower DV by 1.

\subsection*{Casting Roll}
Roll \textbf{Wits + Arcana} (or Ritual, Channeling, etc.).  
Success = spell goes off.  
Failure or 1 = Backlash (see below).

\subsection*{Backlash}
Whenever a Free Caster fails—or pushes power beyond safety—the magic pushes back. Choose one:
\begin{itemize}
\item Harm 2 (Arcane)
\item +2 Fatigue
\item Corruption +1
\item Catastrophic side effect (GM describes)
\end{itemize}

If the spell included a ``Dangerous'' TAG, Backlash triggers on \emph{mixed} results as well.

\newpage

\section*{TAG Library}
Pick 1–3 for minor spells.  
Pick 4–6 for heavy magic (very dangerous).  
More than 6 is suicidal.

\subsection*{Elemental TAGS}
\begin{itemize}[leftmargin=*]
\item \textbf{Burning}: flame, heat, combustion.
\item \textbf{Freezing}: ice, slowing, brittle shatter.
\item \textbf{Storm}: lightning, crackling arcs, thunder shock.
\item \textbf{Stone}: walls, spikes, tremors, armor.
\item \textbf{Wave}: crushing water, currents, pressure.
\item \textbf{Wind}: levitate, gusts, deflection.
\end{itemize}

\subsection*{Force TAGS}
\begin{itemize}[leftmargin=*]
\item \textbf{Force}: pure kinetic power, shields, blasts.
\item \textbf{Area}: cone, circle, corridor, zone.
\item \textbf{Strike}: single target precision.
\item \textbf{Wall}: barrier or blockade.
\item \textbf{Bind}: restrain, hold, suspend.
\item \textbf{Dispel}: suppress magic, unravel effects.
\end{itemize}

\subsection*{Mind \& Veil TAGS}
\begin{itemize}[leftmargin=*]
\item \textbf{Veil}: conceal, blur, illusion, silence.
\item \textbf{Scry}: reveal hidden, see distance, read traces.
\item \textbf{Memory}: erase, alter, restore.
\item \textbf{Command}: compel short action.
\item \textbf{Fear}: panic, flee, break morale.
\end{itemize}

\subsection*{Life \& Body TAGS}
\begin{itemize}[leftmargin=*]
\item \textbf{Mend}: close wounds, restore flesh, reduce Harm 1.
\item \textbf{Purify}: remove poison, corruption, disease.
\item \textbf{Strengthen}: enhance body, armor, senses.
\item \textbf{Waken}: counter sleep, paralysis, stun.
\item \textbf{Beast}: speak with or influence animals.
\end{itemize}

\subsection*{Space \& Motion TAGS (Always +2 DV Each)}
\begin{itemize}[leftmargin=*]
\item \textbf{Leap}: jump far, blink across short space.
\item \textbf{Fold}: short-range teleport, vanish–reappear.
\item \textbf{Gate}: long distance passage, open/close path.
\item \textbf{Gravity}: crush, lift, suspend, walk skyward.
\end{itemize}

\subsection*{Creation \& Transformation TAGS (Always +2 DV Each)}
\begin{itemize}[leftmargin=*]
\item \textbf{Create}: manifest matter briefly.
\item \textbf{Summon}: call a being or construct.
\item \textbf{Transmute}: turn one thing into another.
\item \textbf{Animate}: make objects act with intent.
\end{itemize}

