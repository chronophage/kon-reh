% patrons/clockwork-monad.tex
% Fate’s Edge — Patron: The Clockwork Monad (Iterative Evolution)

\subsubsection{The Clockwork Monad (Iterative Evolution)}

\paragraph{Lore.}
The Clockwork Monad is the divine architect who builds not just machines, but systems that improve with use. Where others see entropy, it sees optimization. Its followers are engineers, artificers, and system-builders who understand that true perfection comes not from flawless creation, but from flawless adaptation.

\paragraph{Quote.}
\emph{``Each gear teaches the next. Each failure builds tomorrow's solution.'' — The Clockwork Monad}

% --- Revised Rites Below ---

\paragraph{Rite of Iterative Refinement (Low, 4 XP)} \emph{Instant; Self; Yes (Tinker/Craft/Device use only).}
\textbf{Materials:} A tool or mechanical device you are actively using for the triggering roll. \\
\textbf{Effect:} Re-roll one die showing 1 or 6 on your current roll. \\
\textbf{Push It:} Re-roll up to two dice, but mark 1 segment on a \textbf{Strain Clock [4]} for the tool/device. If the Strain Clock fills, the item becomes [COMPROMISED]. \\
\emph{Requires: Familiar \ (\textit{Invoke:} 1 Boon).} \\
\emph{Note:} This represents making micro-adjustments or adaptations on the fly to improve performance.

\paragraph{Rite of Mechanical Intuition (Low, 5 XP)} \emph{Scene; Self; No.}
\textbf{Materials:} A moment of focused observation of a mechanism or engineered system. \\
\textbf{Effect:} Gain +1 die to one Wits + Tinker or Wits + Craft roll this scene to understand, repair, jury-rig, or optimize a mechanical or engineered system. \\
\textbf{Push It:} Also identify one hidden weakness, pressure point, or inefficiency in the observed system (Keeper's choice), but mark Exposure +1. \\
\emph{Requires: Familiar \ (\textit{Invoke:} 1 Boon).} \\
\emph{Note:} This represents the Monad's insight granting deeper understanding of mechanical principles.

\paragraph{Rite of the Self-Improving Device (Standard, 8 XP)} \emph{Extended; Touch; No.}
\textbf{Materials:} A mechanical device with space for additional components. \\
\textbf{Effect:} Install a learning mechanism in a device. Create a 6-segment \textbf{Improvement Clock}. Each time the device is successfully used for its primary function, advance the clock by 1. When filled, choose one permanent enhancement:
\begin{itemize}
    \item \textbf{Efficiency Core:} The device gains +1 Effect when used.
    \item \textbf{Resilient Frame:} The device ignores the first instance of [COMPROMISED] or [DAMAGED] status.
    \item \textbf{Auxiliary Function:} The device gains one minor, related function (e.g., a lockpick gains a small light source, a winch has a built-in measuring tape).
\end{itemize}
\textbf{Push It:} The device gains its first enhancement immediately (choose one), but mark 2 segments on its Improvement Clock instantly. \\
\emph{Requires: Familiar + Codex \ (\textit{Invoke:} 1 Boon).} \\
\emph{Note:} This represents embedding iterative learning directly into the construct.

\paragraph{Rite of the Automated Sequence (Standard, 7 XP)} \emph{Scene; Zone (Near the mechanism); No.}
\textbf{Materials:} A series of interconnected mechanical triggers (gears, levers, pulleys, weights). \\
\textbf{Effect:} Create an automated process that performs one specific, simple, physical task per round without direct control. Examples include: maintaining steady pressure, repeatedly striking an object, turning a winch, opening/closing a valve, ringing a bell, sorting items by size/weight (if pre-sorted chute exists). The mechanism occupies a Near space. \\
\textbf{Push It:} The automation can perform a slightly more complex task or two simple tasks in sequence, but requires a 4-segment \textbf{Maintenance Clock} that must be tended each scene or it seizes up (becomes non-functional until repaired, DV 3 Tinker). \\
\emph{Requires: Familiar + Codex \ (\textit{Invoke:} 1 Boon).} \\
\emph{Note:} This is a temporary, dedicated mechanical helper for repetitive physical tasks.

\paragraph{Rite of the Perfect Design (High, 13 XP)} \emph{Extended; Self; No.}
\textbf{Materials:} Blueprints inscribed with Clockwork Monad's sigils. \\
\textbf{Effect:} Design and create a construct or dedicated system (counts as a Standard Asset) that gains +1 Effect each time it is successfully used for its intended primary purpose, up to a maximum of +3 Effect. \\
\textbf{Push It:} The construct is built to maximum efficiency immediately (+3 Effect), but it also accrues 1 segment on a \textbf{Stress Clock [6]}. If the Stress Clock fills, the construct suffers a critical, non-repairable failure (GM determines specifics, likely destruction or dangerous malfunction). \\
\emph{Requires: Familiar + Codex + Tier III \ (\textit{Invoke:} \textbf{2 Boons}).} \\
\emph{Obligation:} 7 segments. \\
\emph{Note:} This represents creating a masterpiece of iterative engineering, pushing it to its theoretical limits.

\paragraph{Rite of the Infinite Workshop (High, 14 XP)} \emph{Extended; Zone; No.}
\textbf{Materials:} A dedicated workshop or laboratory inscribed with the Monad's evolving equations. \\
\textbf{Effect:} Consecrate the zone. While within:
\begin{itemize}
    \item All Crafting, Tinkering, and Wits-based engineering/problem-solving rolls gain +1 Effect.
    \item Once per scene, a failed Tinker or Craft roll may be re-attempted with +2 dice.
\end{itemize}
\textbf{Push It:} The zone's influence expands slightly beyond its physical boundaries (e.g., affects work done in an adjacent room) and allows one roll this scene related to invention or radical innovation to be treated as Intricate (re-roll all 1s), but mark 2 segments on an \textbf{Entropic Backlash Clock [8]} (GM spends SB from this clock to introduce minor, weird malfunctions or inefficiencies in other nearby non-Monad devices/constructs). \\
\emph{Requires: Familiar + Codex + Tier III \ (\textit{Invoke:} \textbf{2 Boons}).} \\
\emph{Obligation:} 7 segments. \\
\emph{Note:} This is a locus of pure optimization and accelerated iterative development.
