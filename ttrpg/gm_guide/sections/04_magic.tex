\chapter{Magic and Backlash}\index{magic}\index{backlash}

In \textbf{Fate's Edge}, magic is not a clean or safe art practiced in sterile towers. It is a \textbf{dangerous negotiation with forces beyond mortal comprehension}—a dance on the razor's edge between power and damnation. Every spell is a gamble where power weighs on one side of the scale and consequence balances on the other. As the GM, your role is to make magic \textbf{feel weighty}, \textbf{thematic}, and \textbf{alive with risk}—a force that reshapes reality at a terrible price.

\section*{The Metaphysical Foundation: Eight Elements}\index{Eight Elements}\index{metaphysics}

Magic in Fate's Edge flows from eight fundamental forces that shape existence itself. These elements are not just energy sources—they are \textbf{philosophical principles} that define how reality functions and how magic interacts with it. They represent the core tensions that drive the universe: permanence versus change, creation versus destruction, order versus chaos, life versus death.

\begin{fatebox}[The Eight Elements of Magic]
\begin{tabularx}{\textwidth}{lX}
\toprule
\textbf{Element} & \textbf{Domain and Philosophical Nature} \\
\midrule
Earth & Stability, foundation, permanence, material reality, mountains, bones, cities \\
Fire & Transformation, passion, destruction, creation, will, forge, revolution \\
Air & Freedom, thought, communication, wind, breath, ideas, movement, change \\
Water & Flow, emotion, adaptation, tides, blood, intuition, reflection, cycles \\
Fate & Destiny, patterns, inevitability, threads, prophecy, order, consequence \\
Life & Growth, vitality, healing, nature, birth, connection, ecosystem, renewal \\
Luck & Chance, opportunity, randomness, fortune, accident, serendipity, risk \\
Death & Endings, transition, memory, ghosts, sacrifice, silence, completion \\
\bottomrule
\end{tabularx}
\end{fatebox}

Each element has its opposite—Earth opposes Air's changeability, Fire counters Water's fluidity, Fate clashes with Luck's randomness, and Life battles Death's finality. This opposition shapes how backlash manifests when magic goes awry. When Earth magic fails, it might cause sudden shifts and instability (Air's domain); when Fire magic backfires, it might create unexpected flows or emotional turbulence (Water's domain).

\section*{Many Faces of Magic}\index{magic paths}

Magic in Fate's Edge is expressed through many interconnected paths. You may specialize in one, or mix them at greater bookkeeping cost. All paths share the same dice engine and SB/Obligation economies, but their flavor and risks differ.

\begin{tcolorbox}[colback=black!3,colframe=black!40!white,title={Sidebar: \texttt{[TAGS]} \& Casting},enhanced]
  \label{sidebar:tags-casting}
  \textbf{What are \texttt{[TAGS]}?} Effects in \textit{Fate's Edge} are communicated via \texttt{[TAGS]}. Each \texttt{[TAG]} is a discrete effect gated behind a Talent, Rite, spell, or asset—\emph{it cannot be invoked spontaneously} unless a rule grants access.
  
  \medskip
  \textbf{How they’re used.} \texttt{[TAGS]} provide a common language for describing effects, especially when players invent spells via \emph{Free Casting}. Many prewritten spells and abilities also list their \texttt{[TAGS]} for clarity.
  
  \medskip
  \textbf{Cross-reference.} For the canonical glossary and full list of available \texttt{[TAGS]}, see \S\ref{magic:tags}.
  \end{tcolorbox}

\subsection*{Casting (Freeform aka "Free Casting")}\index{Casting}\index{freeform magic}

Requires Talent: \textbf{Caster's Gift} (6 XP). Grants access to Weave & Cast using the Eight Elements. Flexible, creative, and risky (Backlash on 1s).

\textbf{Weave \& Cast}: Casters describe the effect in terms of the Eight Elements. The GM sets DV and Effect based on scope.
\begin{itemize}
    \item \textbf{Weave}: Player builds dice pool and rolls. On success, they stabilize the spell's form.
    \item \textbf{Cast}: A second roll channels the effect into the world.
    \item \textbf{Backlash}: Any 1 rolled may cause narrative backlash related to the Element.
\end{itemize}

\subsection*{Limits to Free Casting}

\paragraph*{Core Principle}
Casters may attempt any effect describable using the Eight Elements and \texttt{[TAGS]} system, but power and scope are strictly bounded by \textbf{cost}, \textbf{risk}, and \textbf{fictional logic}. Magic should feel wondrous yet grounded, with clear stakes for overreach.

% ============================
% TAG-BASED LIMITATIONS
% ============================
\subsubsection*{TAG-Based Limitations}

\paragraph*{Element Restrictions}
\begin{itemize}
  \item \textbf{Dangerous \texttt{[TAGS]}} (e.g.\ \texttt{[TELEPORT]}, \texttt{[TRANSFORM]}, \texttt{[DOMINATE]}): Always treated as \emph{high-risk} effects with increased cost and Backlash severity.
  \item \textbf{Creation/Summoning:} Magic cannot create matter from nothing; it must reshape, relocate, or awaken existing materials or energies.
  \item \textbf{Life/Death:} Magic cannot directly kill or truly resurrect. It may heal or inflict Harm within normal game limits, stabilize, or worsen conditions.
\end{itemize}

\paragraph*{Power Caps}
\begin{itemize}
  \item \textbf{Maximum \texttt{[TAGS]} per casting:} 6 total. Beyond this is considered \emph{suicidal} and generally off-limits outside of legendary or sacrificial scenes.
  \item \textbf{Dangerous Combinations:} If 4+ \emph{Dangerous} \texttt{[TAGS]} are combined, any mixed result (success with 1s, partial) \textbf{automatically triggers Backlash}.
  \item \textbf{Elemental Opposition:} Using directly opposing elements together (e.g.\ Fire/Water, Earth/Air) increases Backlash severity on a miss or partial.
\end{itemize}

% ============================
% NARRATIVE CONSTRAINTS
% ============================
\subsubsection*{Narrative and Fictional Constraints}

\paragraph*{Description Requirements}
\begin{itemize}
  \item \textbf{Clear Intent:} The caster must state what the magic does in concrete, table-understandable terms.
  \item \textbf{Logical Consistency:} Effects must align with chosen elements and \texttt{[TAGS]} (no ``fire that heals bones'' without a healing or life-related tag).
  \item \textbf{Environmental Integration:} Magic interacts with existing conditions; it does not ignore terrain, weather, or other ongoing effects.
\end{itemize}

\paragraph*{Reliability Through Practice}
\begin{itemize}
  \item \textbf{Repeated Effects:} After three successful uses of the same basic effect, the GM may treat it as more routine (easier Position, reduced risk, or simpler costs), down to a reasonable minimum.
  \item \textbf{Signature Spells:} Repeatedly used, thematically consistent effects can become personal techniques with reduced cost or risk at GM discretion.
  \item \textbf{Art Specialization:} Focusing on specific elements or techniques can grant +1 die, gentler Backlash, or other small advantages when staying in-theme.
\end{itemize}

% ============================
% COSTS AND CONSEQUENCES
% ============================
\subsubsection*{Resource and Consequence Management}

\paragraph*{Casting Costs}
\begin{itemize}
  \item \textbf{Mental Strain:} Each casting generates Story Beats (SB) roughly equal to the number of \texttt{[TAGS]} used (minimum 1). Bigger, more complex spells give the GM more SB.
  \item \textbf{Backlash Severity:} Stronger, riskier spells escalate potential Backlash: minor twists for small workings, serious harm or narrative fallout for large ones.
  \item \textbf{Fatigue Integration:} Extended, repeated, or especially powerful castings can cause Fatigue in addition to any Backlash.
\end{itemize}

\paragraph*{Recovery Limitations}
\begin{itemize}
  \item \textbf{Cooling Off:} The same complex effect should not be cast more than three times per scene without penalty; afterward, expect worsened Position, increased Backlash, or Fatigue.
  \item \textbf{Preparation:} Major effects require setup: components, positioning, time, or ritual support. If the fiction does not support this, scale the effect down.
  \item \textbf{Sustain Costs:} Ongoing maintained effects cost \emph{at least} 1 Mental Fatigue per scene, plus any narrative vulnerabilities (concentration, focus).
\end{itemize}

% ============================
% GM ADJUDICATION
% ============================
\subsubsection*{GM Adjudication Guidelines}

\paragraph*{When to Say No}
\begin{itemize}
  \item The effect breaks the campaign premise or unravels core mysteries.
  \item The power level is clearly beyond the current Tier of play.
  \item The effect clashes with established genre, tone, or setting logic.
  \item The player is using magic to bypass core game systems (social, travel, investigation) without paying meaningful costs.
\end{itemize}

\paragraph*{When to Say Yes (With Cost)}
\begin{itemize}
  \item \textbf{Creative Problem Solving:} Unconventional but logically-tagged uses of magic.
  \item \textbf{Dramatic Moments:} Big swings that clearly serve character or story arcs.
  \item \textbf{Thematic Consistency:} Effects that reinforce Patron, Domain, or regional magic flavor.
  \item \textbf{Risk/Reward Balance:} High-impact effects that come with visible consequences: clocks, Obligation, Fatigue, or lasting scars.
\end{itemize}

% ============================
% SAFETY VALVES
% ============================
\subsubsection*{Safety Valves}

\paragraph*{Player Options}
\begin{itemize}
  \item \textbf{Scale Back:} Reduce area, targets, or duration to bring the effect inside reasonable limits.
  \item \textbf{Collaborative Casting:} Combine efforts with other casters as a Ritual to share cost and risk.
  \item \textbf{Preparation Time:} Take longer, gather tools, and perform setup to gain better Position or reduce cost.
  \item \textbf{Tool Integration:} Proper components, symbols, or foci can improve Position or gently reduce difficulty.
\end{itemize}

\paragraph*{GM Tools}
\begin{itemize}
  \item \textbf{Compromise Solutions:} ``You cannot fly, but you can leap great distances for this scene.''
  \item \textbf{Temporary Effects:} ``It works, but only for this scene or until the clock ticks.''
  \item \textbf{Conditional Success:} ``It works, but leaves a glaring magical signature or debt.''
  \item \textbf{Narrative Integration:} ``It works, but attracts the attention of a relevant authority, Patron, or entity.''
\end{itemize}

% ============================
% LEARNING AND MASTERY
% ============================
\subsubsection*{Learning and Mastery}

\paragraph*{Progression Path}
\begin{enumerate}
  \item \textbf{Apprentice:} 1--2 \texttt{[TAGS]}, basic elements, small personal effects.
  \item \textbf{Journeyman:} 3--4 \texttt{[TAGS]}, combining elements, reliable scene-scale workings.
  \item \textbf{Master:} 5--6 \texttt{[TAGS]}, complex interactions, major effects with real risks.
  \item \textbf{Legendary:} Beyond normal limits, always with transformational costs, sacrifices, or permanent changes.
\end{enumerate}

\paragraph*{Specialization Benefits}
\begin{itemize}
  \item \textbf{Element Mastery:} +1 die and milder Backlash when working within a primary element or well-established style.
  \item \textbf{TAG Synergy:} Certain combinations become easier with practice; the GM may treat them as one ``compound'' tag instead of several.
  \item \textbf{Art Development:} Personal styles reduce cost or risk for signature techniques, but offer little help outside that niche.
  \item \textbf{Theoretical Advancement:} Deep understanding of principles can justify creative workarounds, with the GM assigning appropriate new costs rather than simple denial.
\end{itemize}

\medskip
\noindent\textit{Free casting remains as creative as the table allows, but power always carries proportional cost and risk.}

\subsection*{Rites User (Runekeeper)}\index{Rites}\index{Runekeeper}

Requires Patron + Thiasos (Familiar) + Codex (4 XP). Grants access to a Patron's Rites. Structured, powerful, but debt-driven through Obligation.

\textbf{Requirements}: A Patron bond, a Thiasos (Familiar), and a Codex (4 XP) mark a character as a Runekeeper.

The Difficulty Value (DV) to cast a Rite is:

\[
\text{DV} = \max\!\big(\text{Obligation Cost} - \text{Spirit}, \, \text{Tier)}\big)
\]

\textbf{Invocation}:
\begin{itemize}
    \item \textbf{Action Cost}: Invoking a Rite requires 1 Action.
    \item \textbf{Obligation}: Each Rite used marks Obligation on its clock.
    \item \textbf{Push It}: Once per Rite, you may Push to increase its duration or potency by +1 step at the cost of +1 Obligation.
\end{itemize}

\textbf{Obligation Clock}: Tracks the Patron's claim. When full, the GM resolves the debt in-fiction. Obligation is reduced through service or downtime actions.

\subsection*{Invoker (Ritualist's Path)}\index{Invoker}\index{Ritualist}\index{Grimoire}

Requires the \textbf{Invoker's Grimoire} talent (6 XP) and study of specific rites. Grants deep knowledge of ritual magic and the ability to perform Rites from multiple Patrons. Symbols are potent tools that enhance this knowledge.

\begin{fatebox}[Invoker Path Features]
\begin{tabularx}{\textwidth}{lX}
\toprule
\textbf{Feature} & \textbf{Description and Limitations} \\
\midrule
Invoker's Grimoire & Major Talent, 6 XP. Grants knowledge of Ritual Magic theory and access to perform a limited number of Rites. \\
Ritual Repertoire & Start with knowledge of \textbf{2} Low or Standard Rites from any Patrons you research. Learn new Rites through Downtime study (see below). \\
Ritual Invocation & Takes \(\text{DV}\) rounds (default 2--3 rounds). Requires specific components/materials. \\
Base Cost & Mark \textbf{+1 Obligation} when you successfully resolve any known Rite (Low or Standard). \textit{(High-Power/High Rites are normally unavailable; if the Keeper permits, treat their \emph{base} Obligation as +2.)}\\
Symbol Enhancement & Possessing the correct Patron's Symbol for a Rite you are casting reduces its \textbf{DV by 1} and its \textbf{Obligation cost by 1} (minimum 0). Only one Symbol may apply to a given Rite. \\
\textbf{No Symbol (Explicit Penalties)} & You may attempt the Rite without the Patron's Symbol, but suffer: \textbf{+1 DV}, \textbf{+1 Obligation} (in addition to Base), and \textbf{+1 round} casting time. On \emph{Partial/Failure}, generate \textbf{+1 extra SB}. \\
Symbol Display & The Symbol must be visible/active throughout the ritual. If it is concealed, disrupted, or removed mid-cast: immediately \textbf{+1 DV}; on Failure, apply \emph{Backlash} (see below). \\
Crack the Seal & Desperate technique. Instantly cast any known Rite by setting the relevant Symbol to \textsc{Compromised}. Mark \textbf{+2 Obligation} (\textbf{+3} for High-Power Rites). Does not reduce Base Obligation below 0. \\
Optional Push & Invokers may \emph{Push} a Rite: choose one (\(+2\) dice \emph{or} +1 Effect \emph{or} resolve one round faster). Always mark \textbf{+1 Obligation} \emph{and} generate \textbf{1 SB}, in addition to other costs. \\
Cross-Resonance & If you cast Rites from \emph{different Patrons} in the same scene, each Patron after the first adds \textbf{+1 DV} to that Rite. \\
\bottomrule
\end{tabularx}
\end{fatebox}

\paragraph{Symbol States \& Repair}
\begin{itemize}
  \item \textsc{Compromised:} A Symbol set to \textsc{Compromised} (e.g., via \emph{Crack the Seal}) provides \emph{no} DV/Obligation reduction until repaired. Casting with a \textsc{Compromised} Symbol imposes \(-1\) die on the Casting Test.
  \item \textsc{Shattered:} If you \emph{Crack the Seal} again while the Symbol is \textsc{Compromised}, it becomes \textsc{Shattered} and cannot be used until replaced (Asset lost).
  \item \textbf{Repair (Downtime):} 1 day of focused work and a \emph{Craft or Lore + Tinker} test vs.\ DV~3. Success: clear \textsc{Compromised}. Failure: no progress. Alternatively, spend \textbf{1 XP} to auto-repair.
\end{itemize}

\paragraph{Backlash \& Failure (Explicit)}
\begin{itemize}
  \item \textbf{Success:} Rite resolves; apply Base/added Obligation and any SB from Push or No-Symbol clauses.
  \item \textbf{Partial:} Effect \(-1\) step \emph{or} shortened duration; mark \textbf{Fatigue 1}. If cast \emph{without} a Symbol, Keeper gains \textbf{+1 SB} (in addition to normal SB generation).
  \item \textbf{Failure:} No effect; mark \textbf{Fatigue 1}; Keeper gains \textbf{+1 SB}. Then test \emph{Spirit + Resolve} vs.\ DV~3:
    \begin{itemize}
      \item On Fail: suffer \textbf{Harm 1 (Shock)} or start \textbf{Backlash Static [4]} (Keeper's choice).
      \item If the Symbol was disrupted/hidden mid-cast \emph{or} you \emph{Cracked the Seal}: upgrade to \textbf{Harm 2 (Shock)}.
    \end{itemize}
  \item \textbf{Interrupted:} Harm, Silence, or disruption before resolution counts as \emph{Failure}.
\end{itemize}

\textbf{Example:} Magus Vex, bearing the \textbf{Invoker's Grimoire}, has studied the rites of Raéyn and the Sealed Gate. He knows Raéyn's \emph{Whispering Currents} (Low) and the Sealed Gate's \emph{Circle of Denial} (Standard). Faced with a collapsing tunnel, he attempts the Sealed Gate's ritual. It's a Standard Rite, so \textbf{DV 3}, taking \textbf{3 rounds}, and costs \textbf{+1 Obligation}. He has the Sealed Gate's Symbol, reducing the DV to \textbf{2} and the Obligation cost to \textbf{0}. When ambushed, he needs quick protection. He \textbf{Cracks the Seal} on the \emph{Circle of Denial}. The Symbol becomes \textsc{Compromised}, the Rite is instant, and he marks \textbf{+2 Obligation}. Later, needing to bind a particularly strong foe, he \textbf{Pushes} the Rite, marking an additional \textbf{+1 Obligation} and generating \textbf{1 SB}; the barrier strengthens. If he tried a Raéyn Rite afterwards in the same scene, \emph{Cross-Resonance} would add \textbf{+1 DV} to that casting.

\subsubsection*{Learning New Rites}
An Invoker can expand their \textbf{Ritual Repertoire} through dedicated study during \textbf{Downtime}.
\begin{itemize}
    \item \textbf{Cost:} 1 week of Downtime + 2 XP.
    \item \textbf{Requirement:} Access to texts, a teacher, or direct observation of the Rite being performed by another adept.
    \item \textbf{Test:} \emph{Lore + Investigation} (or a relevant skill) vs.\ DV~3--5 (based on Rite rarity/complexity).
    \item \textbf{Success:} Add the Rite to your Ritual Repertoire.
    \item \textbf{Failure:} Cannot learn this specific Rite for a significant time (GM discretion). The Keeper may set a relevant Complication (e.g., \emph{Forbidden Knowledge Pursued}).
\end{itemize}

\subsubsection*{Symbols as Assets}
\begin{itemize}
    \item A Patron's Symbol is a \textbf{Minor Asset (4 XP)} whose primary value is as a \textbf{ritual focus/component}.
    \item You \emph{can} attempt any ritual \textbf{without} the Symbol, but you incur these \textbf{No Symbol} penalties: \textbf{+1 DV} \emph{(and therefore +1 round to cast, since casting time = DV rounds)}, \textbf{+1 Obligation} \emph{(in addition to Base)}, and on \emph{Partial/Failure} the Keeper gains \textbf{+1 extra SB}.    \item Symbols can be \textbf{maintained/upgraded} like other Assets. Example upgrades: \emph{Hardened} (ignore the first application of \textsc{Compromised} per session), \emph{Bright} (treat as \emph{visible} for Symbol Display while concealed on your person).
\end{itemize}

\subsection*{Borrowed Grace}
\label{talent:borrowed-grace}
\index{Talents!Invoker}\index{Imbuement!Lesser}

\textbf{Type:} Invoker Talent — \textit{Lesser Imbuement}

\subsubsection*{Use}
\begin{itemize}
  \item \textbf{Cost:} \textbf{1 Boon}, \textbf{1 action}.
  \item \textbf{Effect (pick one on use):} \textbf{+1 Melee} \emph{or} \textbf{+1 Thematic} (your table's signature/thematic Skill).
  \item \textbf{Duration:} \textit{Single action/attack} (instantaneous boost only).
  \item \textbf{Requirement:} Wield/display the relevant Patron's \textbf{Symbol}.
  \item \textbf{Obligation:} Immediately mark \textbf{+1 Obligation} to that Patron (see \S\ref{sec:obligation}).
  \item \textbf{Limits:} Cannot be extended, stacked, or \emph{Pushed} for duration. Using \emph{Borrowed Grace} while the Symbol is \textsc{Compromised} lowers your \textbf{Position} by one step \emph{(or imposes \(-1\) die if already \textbf{Desperate})}.)
\end{itemize}

\section{Obligation Capacity}

A character’s \textbf{Obligation Capacity} equals Spirit + Presence.
Track total Obligation segments across all Patrons (or Symbols, for Invokers).

\begin{itemize}
  \item \textbf{Exceeding Capacity:} For each segment above Capacity, mark 1 Fatigue. The character cannot Invoke Rites or perform rituals until Obligation is reduced below Capacity.
  \item \textbf{Overload (≥ 2x Capacity):} Clear all Fatigue, take +1 Harm, and suffer immediate Patron intrusion (Claim, demand, or narrative cost). Downtime cannot reduce Obligation until Harm is addressed.
  \item \textbf{Resolution:} Reduce Obligation through Downtime service, Patron tasks, ritual cleansing, or story resolution.
\end{itemize}

\textbf{Example:} Spirit~2 + Presence~3 = Capacity 5.
6 segments → Fatigue~1.
7 segments → Fatigue~2.
10 segments → Harm~1.
11 segments → Harm~2.


\centering
\caption{Universal Push It Costs}
\begin{longtable}{|l|l|}
\hline
\textbf{Cost Component} & \textbf{Effect} \\
\hline
+1 SB & Escalate effect immediately \\
+1 Fatigue & Immediate physical/mental strain \\
+1 Corruption Clock Segment & Long-term Patron influence (unless otherwise specified) \\
GM spends 1 SB & Thematic complication (unless otherwise specified) \\
\hline
\end{longtable}


Note: Some talents, Rites, or magical paths may specify alternative corruption costs or additional consequences for Push It actions. When explicitly stated, those specific rules override the universal costs.

\paragraph{Clearing Corruption}
Corruption may be reduced through \textit{purging rituals}, such as exorcisms, sacred songs, or rites of contrition. 
These require a test (typically \textbf{Lore + Spirit}) against a DV equal to the character’s current corruption level.  
On success, reduce corruption by 1. On failure, the corruption manifests violently, imposing a temporary Condition or advancing its narrative expression.  

Optional: A \textbf{Story Beat} may also be spent to attempt such a ritual, representing the personal cost of atonement. Patrons may demand specific acts of service, sacrifice, or obligation as part of the purging process.
\section*{Summoning: Binding Outsider Forces}\index{Summoning}\index{Pact-Whisperer}

%----------------------------------------
\section{Summoning (Pact-Whisperer) — GM Mechanics}
\label{subsec:summoning-gm}

\begin{fatebox}[Core]
\begin{tabularx}{\textwidth}{lX}
\toprule
\textbf{Step} & \textbf{Rule (GM-Facing)} \\
\midrule
\textbf{Access} & Requires \textbf{Pact-Whisperer} (2 XP). Talents gate capacity: \textbf{Lesser Pactwright} (Cap~1), \textbf{Greater Pactwright} (Cap~3). With both, one of each may be maintained. \\
\textbf{Call} & \emph{1 Action}. Manifest at \textit{Near}. Choose a fitting Spirit Template (scene/Patron aligned). \\
\textbf{Bind} & Pay 1 Boon \emph{or} mark 1 Fatigue. \\
\textbf{Leash Capacity} & Set \(\textbf{Leash} = \textit{Cap} + \textit{Spirit}\) \emph{segments}. (Cap~1 = Lesser, Cap~3 = Greater.) \\
\textbf{Tick Triggers} & Tick on: spirit takes Harm; command vs.\ nature; summoner splits focus; rival contests; rapid \textit{Close}\(\rightarrow\)\textit{Far} reposition; crossing a \texttt{[WARD]} (test \(\mathrm{DV}=\mathrm{Cap}\)). \\
\textbf{Act/Order} & A meaningful new order uses the summoner's Action; \emph{Quick Commands} (attack nearest, hold doorway, relocate within \textit{Near}, fetch and return) do not. \\
\textbf{Departure} & When Leash fills, spirit acts to its nature once, then departs (or turns hostile at GM discretion). \\
\bottomrule
\end{tabularx}
\end{fatebox}

\paragraph*{Spirit Bond (Light Progress).}
Track a \textbf{Spirit Bond Clock [4]} per recurring spirit type. Mark on shared victories, good handling, or mutual aid. At 2: +1 die to communicate. At 4: on natural departure, +1 Boon and spirit becomes \textbf{Favored} (its Leash $-1$). \textit{Near-Miss:} once/session per type, mark +1 on a meaningful failed Call/Bind.

\paragraph*{Specializations (Pick When Relevant).}
Combat Specialist (+1 Harm melee; ignore first Harm on attacks) \quad
Scout Form (stealth/range; carry up to Cap~1: 2\,kg, Cap~3: 10\,kg; dragging $\le 3\times$; overburden = tick) \quad
Utility Spirit (simple tasks) \quad
Shield Guardian (interpose; convert Harm\(\rightarrow\)Fatigue).

\paragraph*{Economy \& Limits.}
\begin{itemize}
  \item \textbf{Boon Finesse:} Once/round, spend 1 Boon to clear 1 tick (before fill).
  \item \textbf{Order of Action:} Spirit acts immediately after summoner.
  \item \textbf{Concurrency:} One active spirit unless a Talent states otherwise; excess costs 1 Fatigue per extra Cap point.
  \item \textbf{End-State:} All summons end at Downtime unless sustained by Rite/Asset.
\end{itemize}

\paragraph*{GM Reminders (No New Clocks).}
Use only the \textbf{Leash} and optional \textbf{Spirit Bond} clocks. \emph{Quick Commands} should be crisp rulings; tick when the player overreaches the spirit's nature or fiction.

%----------------------------------------
\section{Summoning (Pact-Whisperer) — GM Mechanics}
\label{subsec:summoning-gm}

\begin{fatebox}[Core]
\begin{tabularx}{\textwidth}{lX}
\toprule
\textbf{Step} & \textbf{Rule (GM-Facing)} \\
\midrule
\textbf{Access} & Requires \textbf{Pact-Whisperer} (2 XP). Talents gate capacity: \textbf{Lesser Pactwright} (Cap~1), \textbf{Greater Pactwright} (Cap~3). With both, one of each may be maintained. \\
\textbf{Call} & \emph{1 Action}. Manifest at \textit{Near}. Choose a fitting Spirit Template (scene/Patron aligned). \\
\textbf{Bind} & Pay 1 Boon \emph{or} mark 1 Fatigue. \\
\textbf{Leash Capacity} & Set \(\textbf{Leash} = \textit{Cap} + \textit{Spirit}\) \emph{segments}. (Cap~1 = Lesser, Cap~3 = Greater.) \\
\textbf{Tick Triggers} & Tick on: spirit takes Harm; command vs.\ nature; summoner performs separate concentration-requiring action while commanding; rival contests; crossing a \texttt{[WARD]} successfully (test \(\mathrm{DV}=\mathrm{Cap}\)). \\
\textbf{Act/Order} & A meaningful new order uses the summoner's Action; \emph{Quick Commands} (attack nearest, hold doorway, relocate within \textit{Near}, fetch and return) do not. \\
\textbf{Departure} & When Leash fills, spirit acts to its nature once, then departs (or turns hostile at GM discretion). \\
\bottomrule
\end{tabularx}
\end{fatebox}

\paragraph*{Spirit Bond (Light Progress).}
Track a \textbf{Spirit Bond Clock [4]} per recurring spirit type. Mark on shared victories, good handling, or mutual aid. At 2: +1 die to communicate. At 4: on natural departure, +1 Boon and spirit becomes \textbf{Favored} (its Leash $-1$). \textit{Near-Miss:} once/session per type, mark +1 on a meaningful failed Call/Bind.

\paragraph*{Specializations (Pick When Relevant).}
Combat Specialist (+1 Harm melee; ignore first Harm on attacks) \quad
Scout Form (stealth/range; carry up to Cap~1: 2\,kg, Cap~3: 10\,kg; dragging $\le 3\times$; overburden = tick) \quad
Utility Spirit (simple tasks) \quad
Shield Guardian (interpose; convert Harm\(\rightarrow\)Fatigue).

\paragraph*{Economy \& Limits.}
\begin{itemize}
  \item \textbf{Boon Finesse:} Once/round, spend 1 Boon to clear 1 tick (before fill).
  \item \textbf{Order of Action:} Spirit acts immediately after the command is given.
  \item \textbf{Concurrency:} One active spirit unless a Talent states otherwise; excess costs 1 Fatigue per extra Cap point.
  \item \textbf{End-State:} All summons end at Downtime unless sustained by Rite/Asset.
\end{itemize}

\paragraph*{GM Reminders (No New Clocks).}
Use only the \textbf{Leash} and optional \textbf{Spirit Bond} clocks. \emph{Quick Commands} should be crisp rulings; tick when the player overreaches the spirit's nature or fiction. Clarify that "splitting focus" means performing a separate, concentration-requiring action while actively directing the spirit.

%----------------------------------------
\section{Cantor's Path (Songs) -- GM Mechanics}
\label{talent:cantors-path-gm}

\begin{fatebox}[Core]
\begin{tabularx}{\textwidth}{lX}
\toprule
\textbf{Element} & \textbf{Rule (GM-Facing)} \\
\midrule
\textbf{Access} & Talent: \textbf{Cantor's Path} (8 XP). Prereqs: Lore~1+, Performance~2+, Presence~2+. \\
\textbf{Scope} & \textbf{Low Rites as Songs} only (counts as knowing for performance use). \\
\textbf{Cast Test} & \emph{Lore + Performance vs.\ DV} (typical DV~2--3). \\
\textbf{Timing} & Start with 1 Action; resolves at \emph{start of next turn} unless \emph{Pushed}. \\
\textbf{Cost} & Pay listed materials. \emph{No Obligation} on success. \\
\textbf{Visibility} & Songs are noticeable; on Failure or Push, assume observers take note. \\
\bottomrule
\end{tabularx}
\end{fatebox}

\paragraph*{Corruption (Light).}
Track a \textbf{Corruption Clock} with segments equal to \textbf{Body}. Mark toward accumulation when: \emph{Push}, perform a \emph{Resonant Rite}, or the Keeper spends an SB tied to psionic/occult activity.
On fill: apply the last-Patron \textbf{benefit \& drawback} (and echo to followers/retainers); then reset to character's \textbf{Tier} minimum.

\paragraph*{Corruption Accumulation Triggers.}
Multiple triggers required for +1 Corruption segment:
\begin{itemize}
    \item \textbf{2 Push It uses} = +1 Corruption segment
    \item \textbf{1 Push It + 1 Resonant Rite} = +1 Corruption segment
    \item \textbf{3 GM SB spends} on occult activities = +1 Corruption segment
    \item \textbf{1 High Cantor Standard Rite} = +1 Corruption segment
\end{itemize}

\paragraph*{Resonant Rites.}
Designated Low Rites may \emph{optionally} mark toward Corruption accumulation on success for added weight. Player chooses to resonate or not.

\paragraph*{Results.}
\textbf{Success:} Rite as written. \quad
\textbf{Partial:} $-1$ step or shorter duration; mark Fatigue~1. \quad
\textbf{Failure/Interrupted:} No effect; mark Fatigue~1; GM gains +1 SB (Hearts).

\paragraph*{Push.}
Resolve now; mark Fatigue~1; mark toward Corruption accumulation; trigger a GM \textbf{Story Beat} (Patron/Road/social fallout).

\paragraph*{Song Synergy (Tight Rulings).}
\begin{itemize}
  \item \textbf{Compatible Songs} = same Patron \emph{or} clearly similar thematic purpose.
  \item \textbf{Harmony:} Two compatible Songs: +1 Effect to both.
  \item \textbf{Counterpoint:} Opposed Songs may cancel a drawback (GM adjudicates).
  \item \textbf{Chorus:} Multiple singers amplify (+1 Effect per participant), but cap coordinated \emph{Song Weaver} style combos at \textbf{3 participants}.
\end{itemize}

\paragraph*{Repertoire (Optional, Light).}
Single \textbf{Repertoire Clock [6]} for breadth: 2~seg = base DV $-1$ (min 2); 4~seg = +1 die to Song rolls; 6~seg = one \emph{temporary} Standard Rite as Song (practice-dependent).

\paragraph*{Song Specialization Paths.}
\begin{itemize}
  \item \textbf{Battle Cantor:} War Songs grant allies +1 Position in combat; Hymn of Fury converts 1 Harm to Fatigue for allies Near you; Anthem of the Fallen allows departed allies to return as spectral echoes (1/session).
  \item \textbf{Shadow Cantor:} Songs of Veiling create \texttt{[VEIL]} effects without ritual components; Melody of Misdirection imposes -1d to Notice rolls on enemies; Dirge of Passing enables communication with dead and scrying through recent deaths.
  \item \textbf{Healing Cantor:} Songs of Restoration heal +1 Harm; Chant of Purification removes poison/disease; Hymn of Vitality grants temporary +1 Body.
  \item \textbf{Knowledge Cantor:} Lore Songs reveal hidden knowledge; Chant of Understanding grants +2d to Investigation/Lore; Ode to Memory allows perfect recall of witnessed events.
\end{itemize}

\paragraph*{High Cantor (Prestige, Fast Ruling).}
Standard Rites as \textbf{High Cant}: instant; +1 die to primary effect; mark toward Corruption accumulation (1 High Cantor Standard Rite = 1 Corruption trigger). Recognizably flashy; repeated uses in a scene add +1 DV to subsequent \emph{Resolve} saves (fear/charm/social pressure).

\paragraph*{Divine Resonance (Major Talent - 15 XP).}
Your voice carries divine authority. Once per scene, spend 2 Boons:
\begin{itemize}
  \item \textbf{Command Effect:} Issue a \texttt{[COMMAND]} that affects up to (Presence) targets simultaneously
  \item \textbf{Miracle Effect:} Replicate any Low Rite without marking Corruption (but generate 1 SB)
  \item \textbf{Omen Effect:} Gain insight into a major threat - ask 3 questions about one enemy/faction
\end{itemize}
\textbf{Cost:} Mark +2 Corruption segments, immediately trigger Patron attention.

\paragraph*{Bookkeeping Cap.}
At the table, track \emph{only}: \textbf{Corruption Clock} and (optionally) the single \textbf{Repertoire Clock}. Do not add per-Song timers; use outcomes and Push to pace.

\subsection*{Inspire Chorus}
\label{subsec:inspire-chorus}

While \emph{actively singing a Song} (from the action to begin until it resolves, or while a \emph{Lingering Verse} persists), the Cantor may \textbf{invoke Inspire Chorus}:

\begin{itemize}
  \item \textbf{Effect:} All \textbf{allies within Near} (including the Cantor) \textbf{shift Position +1} for \textbf{one exchange} (e.g., Desperate$\to$Controlled, Controlled$\to$Dominant). Position cannot exceed \textbf{Controlled}. This does not stack with other Position-shift auras; use the best single shift.
  \item \textbf{Use:} \textbf{Once per scene} at no cost. \textbf{Additional uses in the same scene} are allowed, but each \textbf{immediately marks toward Corruption accumulation} (see Corruption rules).
  \item \textbf{Requirements:} The performance must be perceptible to recipients (line of hearing; \textit{Silence} or similar effects suppress it).
  \item \textbf{Timing:} Declare on starting the Song or at any time before it resolves; the shift lasts until the start of the Cantor's next turn.
  \item \textbf{Notes:} Using \emph{Inspire Chorus} does not change Song DV, Action cost, or outcomes. It respects \emph{Bookkeeping Light}: no new clocks are created.
\end{itemize}

\subsection*{Cantors as Cult Leaders (Chorus-Founders)}
Cantors gather crowds—and crowds gather debts. The Song's Corruption stains the air, and listeners answer with vows, tithes, and favors. Many Cantors drift into leadership not by decree but by \emph{obligation}: their audience becomes a \textit{chorus} that expects guidance, protection, and more songs. In practice, the Cantor's rising \textbf{Corruption} is mirrored by the flock's growing \textbf{Obligation} to the Cantor (and the Patron behind the music).

\begin{fatebox}[Chorus Cult --- Quick Rule]
\begin{tabularx}{\textwidth}{lX}
\toprule
\textbf{Trigger} & After a public Song using \emph{Inspire Chorus} or a \textbf{Resonant Rite} before 10+ witnesses, the Cantor may found or deepen a \textit{Chorus} (cult). \\[2pt]
\textbf{Cost} & Immediately convert \textbf{+1 Corruption segment} into \textbf{+1 Obligation} (to the Patron or the Chorus, GM's call). \\[2pt]
\textbf{Benefit} & Gain a \textbf{Minor Follower (Chorus)}: once/scene (if present or reachable), \textit{+1 die} to Performance/Sway \emph{or} establish a rumor/cover within the community. Scale $\approx$ Cantor's \textbf{Presence}. \\[2pt]
\textbf{Maintenance} & Each scene/session you leverage the Chorus, mark \textbf{+1 Obligation}. If neglected, start \textbf{Devotion Sours [4]}; on fill, the Chorus fractures into a Complication (rival sect, scandal, or betrayed devotee). \\[2pt]
\textbf{Safety Valve} & During Downtime, a \textit{Vigil} (public service, free performance, or restitution) clears \textbf{1 Obligation} to the Chorus and resets \textbf{Devotion Sours} by 1. \\
\bottomrule
\end{tabularx}
\end{fatebox}

\section*{Magical Arts and Specialization}\index{Magical Arts}

A character's Art represents their personal approach to magic—the techniques, tools, and philosophies that define their craft. When a character gains magical capability, they define their Art with specific parameters.

\begin{fatebox}[Defining Your Magical Art]
\begin{tabularx}{\textwidth}{lX}
\toprule
\textbf{Component} & \textbf{Description and Examples} \\
\midrule
Gesture \& Medium & Ink sigils, sung names, lantern-light, bone charms, legal contracts, salt-threads \\
Elemental Alignment & Choose 2 primary Elements the Art typically engages with (Fire+Earth, Air+Water, etc.) \\
Thematic Focus & Destruction, protection, revelation, transformation, communication, healing \\
Cultural Roots & High Elf crystal-song, Ykrul blood-runes, Aeler spirit-whispers, Human alchemy \\
\bottomrule
\end{tabularx}
\end{fatebox}

\subsection*{Art in Play}\index{Magical Arts!in play}

The fictional positioning of a character's Art matters significantly:

\begin{itemize}
    \item \textbf{Spotlight Bump (1/scene)}: If the Art is clearly honored in fiction (right tools, time, setting), gain +1 die on the Cast roll
    \item \textbf{Off-Style Strain}: If forced to work against the Art's nature (no tools, hostile environment), suffer worse Position or accept extra Backlash
    \item \textbf{Art-Based Backlash}: Consequences should reflect the Art's themes and elements
\end{itemize}

\section*{Tags: The Language of Magical Effects}\index{Tags}\index{magical effects}
\label{magic:tags}

Tags provide a common language for describing magical effects and their limitations. They only function when printed on a Talent, Ability, or Spell result.

\begin{fatebox}[Common Magical Tags and Effects]
\begin{tabularx}{\textwidth}{lX}
\toprule
\textbf{Tag} & \textbf{Effect and Usage Guidelines} \\
\midrule
[DISPEL] & End an ongoing magical effect/construct. DV by fiction. \\
[COUNTER] & Interrupt a cast/rite in progress. DV by fiction. \\
[BARRIER] & Create cover/obstruction. DV by fiction. \\
[SEAL]/[UNSEAL] & Lock or unlock a container/door/portal. DV by fiction. \\
[VEIL] & Obscure a person/thing/zone. DV by fiction. \\
[REVEAL] & Expose illusions, disguises, hidden clauses. DV by fiction. \\
[MARK] & Tag a target for tracking or leverage. DV by fiction. \\
[CURSE] & Inflict a sticky hindrance with a clear release. DV by fiction. \\
[CLEANSE] & Remove/suppress a condition. DV by fiction. \\
[FORTIFY] & Harden against a vector. DV by fiction. \\
[COMMAND] & Issue a clear order to a sapient target. DV by fiction. \\
[OATH] & Bind parties to terms; breaking has teeth. DV by fiction. \\
[SANCTIFY] & Consecrate a zone to a code/patron. DV by fiction. \\
[PASSAGE] & Declare a route as permitted/easy. DV by fiction. \\
[TRANSPORT] & Move a target across an obstacle. DV by fiction. \\
[CONJURE] & Create a useful object/cover/hazard. DV by fiction. \\
[WARD] & Challenge Outsiders crossing a warded edge/zone. DV = target Cap. \\
[BANISH] & Drive a visible Outsider toward departure. DV = target Cap. \\
[UNWARD] & Unmake/suppress a [WARD]. DV by fiction. \\
\bottomrule
\end{tabularx}
\end{fatebox}

Tags work within consistent parameters:
\begin{itemize}
    \item \textbf{DV by Fiction}: Potency, preparation, and opposition set difficulty
    \item \textbf{Duration}: Typically "Scene" unless specified otherwise
    \item \textbf{Stacking}: No same-source stacking; identical tags use strongest instance
\end{itemize}

\section*{Backlash: The Price of Power}\index{Backlash}

Backlash represents magic escaping control—the inevitable consequence of wielding forces beyond mortal comprehension. It's never arbitrary; backlash always reflects the elements involved and their philosophical oppositions.

\subsection*{Backlash Triggers and Severity}

Backlash occurs when magic goes awry:
\begin{itemize}
    \item \textbf{Primary Trigger}: Partial or Miss on either the Weave or Cast roll
    \item \textbf{Secondary Trigger}: Hit showing two or more 1s (minor backlash rides success)
    \item \textbf{SB Integration}: Backlash does not generate extra SB—it's how GM spends SB from rolled 1s
\end{itemize}

Backlash colors the cost of magic and is always expressed through fiction first.

\begin{fatebox}[Backlash Menu]
\begin{tabularx}{\textwidth}{lX}
\toprule
\textbf{Backlash Type} & \textbf{Effect} \\
\midrule
Position Shift & Worsen Position by 1 step for current or next action \\
Fleeting Harm/Condition & Sear, vertigo, chill that matters for this scene \\
Exposure/Noise & Draws notice or complicates stealth \\
Resource Drain & Time, focus, or component damaged \\
Collateral Spark & Threatens ally or fragile thing nearby \\
\bottomrule
\end{tabularx}
\end{fatebox}

\subsection*{Elemental Backlash Coloring}\index{Backlash!elemental}

On Partial/Miss (or double-1s on a Hit), color consequences by Element:

\begin{fatebox}[Elemental Backlash Coloring]
\begin{tabularx}{\textwidth}{lX}
\toprule
\textbf{Element Pair} & \textbf{Minor Backlash} \\
\midrule
Earth / Fate & Slips, binds, encumbrance \\
Fire / Life & Smoke, sparks, heat \\
Air / Luck & Scatter, misheard words \\
Water / Dreams & Slippery tide, slow gear \\
Fate / Earth & Probability resists \\
Life / Fire & Growth surge, vines tether \\
Luck / Air & Odds flip \\
Death / Water & Whispers, chill \\
\bottomrule
\end{tabularx}
\end{fatebox}

Backlash should always feel thematic to the magic employed:
\begin{itemize}
    \item \textbf{Fire Magic}: Burns, flares, smoke, heat exhaustion, uncontrolled fires
    \item \textbf{Water Magic}: Flooding, slick surfaces, damp-related rot, emotional turbulence
    \item \textbf{Earth Magic}: Tremors, collapsing structures, immobilization, heavy burdens
    \item \textbf{Air Magic}: Unexpected winds, carried sounds, vertigo, scattered plans
    \item \textbf{Fate Magic}: Closed options, inevitable consequences, prophetic nightmares
    \item \textbf{Luck Magic}: Allied misfortunes, fragile successes, random complications
    \item \textbf{Life Magic}: Overgrowth, sympathetic pain, unnatural hunger, fertility curses
    \item \textbf{Death Magic}: Ghostly echoes, premature aging, silence, memory loss
\end{itemize}

\section*{Ritual Casting: Collective Magic}\index{ritual casting}

Some workings require multiple casters pooling their strength. Rituals allow for greater effects but multiply risks.

\subsection*{Ritual Procedure}

\begin{enumerate}
    \item \textbf{Declaration}: Primary caster states intent and gathers participants
    \item \textbf{Channel Together}: All participants contribute (Scene-long action)
    \item \textbf{Weave}: Primary caster shapes combined Potential (Scene-long action)  
    \item \textbf{Backlash}: Consequences affect all participants based on their contribution
\end{enumerate}

\subsection*{Ritual Mechanics}

\begin{itemize}
    \item \textbf{Helper Cap}: Primary caster can draw on ceil(Arcana/2) helpers (max 3)
    \item \textbf{Skill Flexibility}: Helpers may use different relevant skills if fictionally distinct
    \item \textbf{Risk Distribution}: SB from Channel affects individual rollers; SB from Weave affects primary caster
\end{itemize}

\section*{Magic in Combat}\index{magic combat}

Spellcasting in combat follows the same principles but with heightened stakes and immediate consequences.

\subsection*{Combat Casting Considerations}

\begin{fatebox}[Magic in Combat: Position and Effect]
\begin{tabularx}{\textwidth}{lX}
\toprule
\textbf{Position} & \textbf{Effect on Magical Actions} \\
\midrule
Dominant & +1 die to Channel; reduced Backlash risk; can maintain subtle effects \\
Controlled & Standard casting conditions; typical risk/reward balance \\
Desperate & -1 die to Channel; increased Backlash severity; may attract unwanted attention \\
\bottomrule
\end{tabularx}
\end{fatebox}

\subsection*{Tactical Magic Applications}

Magic can reshape combat dynamics:
\begin{itemize}
    \item \textbf{Position Warfare}: Spells that create cover, elevate positions, or restrict movement
    \item \textbf{Morale Effects}: Magic that inspires allies or terrifies enemies
    \item \textbf{Environmental Control}: Creating hazards, altering terrain, manipulating weather
    \item \textbf{Resource Denial}: Destroying enemy equipment, exhausting their supplies
\end{itemize}

\section*{Prestige Magical Abilities}\index{Prestige Abilities!magical}

High-level magical talents represent profound mastery or unique cultural inheritances.

\begin{fatebox}[Example Prestige Magical Abilities]
\begin{tabularx}{\textwidth}{lX}
\toprule
\textbf{Ability} & \textbf{Description and Requirements} \\
\midrule
Ways-Walker's Step & Observe perfect echo of past event (1/arc); GM banks +2 SB; reveals hidden truths (Req: Wits 5, Arcana 4) \\
Warglord & Unify scattered warbands into host for season; track Logistics and Grudge clocks (Req: Body 5, Command 3) \\
Spirit-Shield & Erase up to 3 SB from ally's roll (1/session); caster takes Fatigue +1 and GM banks +1 SB (Req: Spirit 4, Insight 3) \\
Elemental Mastery & Choose one Element; gain +2 dice when using it, but backlash from opposite element is doubled \\
\bottomrule
\end{tabularx}
\end{fatebox}

\section*{Free Casting (TAGS System)}
Some casters do not prepare rote rites. They shape raw forces through shared arcane grammar known as \textbf{TAGS}. A spell is constructed at the table using a short phrase of TAGS. You only need the fiction, the TAG selection, and a casting roll.

\subsection*{Spell Structure}
\textbf{Intent} + \textbf{Target} + \textbf{Tags} = effect.

Example formula:
\begin{quote}
``I unleash Burning • Area • Force against the marauders.''
\end{quote}

The GM sets a Difficulty Value (DV) based on TAG complexity and danger.

\subsection*{Base Difficulty Value (DV)}
Start at DV 1 and add +1 for each TAG used.

\begin{center}
\textbf{DV = 1 + number of TAGS}
\end{center}

Adding powerful or perilous TAGS (Teleportation, Transformation, Dominate) adds +2 instead.

Mastery, focus, or appropriate tools may lower DV by 1.

\subsection*{Casting Roll}
Roll \textbf{Wits + Arcana} (or Ritual, Channeling, etc.).  
Success = spell goes off.  
Failure or 1 = Backlash (see below).

\subsection*{Backlash}
Whenever a Free Caster fails—or pushes power beyond safety—the magic pushes back. Choose one:
\begin{itemize}
\item Harm 2 (Arcane)
\item +2 Fatigue
\item Corruption +1
\item Catastrophic side effect (GM describes)
\end{itemize}

If the spell included a ``Dangerous'' TAG, Backlash triggers on \emph{mixed} results as well.

\newpage

\section*{TAG Library}
Pick 1–3 for minor spells.  
Pick 4–6 for heavy magic (very dangerous).  
More than 6 is suicidal.

\subsection*{Elemental TAGS}
\begin{itemize}[leftmargin=*]
\item \textbf{Burning}: flame, heat, combustion.
\item \textbf{Freezing}: ice, slowing, brittle shatter.
\item \textbf{Storm}: lightning, crackling arcs, thunder shock.
\item \textbf{Stone}: walls, spikes, tremors, armor.
\item \textbf{Wave}: crushing water, currents, pressure.
\item \textbf{Wind}: levitate, gusts, deflection.
\end{itemize}

\subsection*{Force TAGS}
\begin{itemize}[leftmargin=*]
\item \textbf{Force}: pure kinetic power, shields, blasts.
\item \textbf{Area}: cone, circle, corridor, zone.
\item \textbf{Strike}: single target precision.
\item \textbf{Wall}: barrier or blockade.
\item \textbf{Bind}: restrain, hold, suspend.
\item \textbf{Dispel}: suppress magic, unravel effects.
\end{itemize}

\subsection*{Mind \& Veil TAGS}
\begin{itemize}[leftmargin=*]
\item \textbf{Veil}: conceal, blur, illusion, silence.
\item \textbf{Scry}: reveal hidden, see distance, read traces.
\item \textbf{Memory}: erase, alter, restore.
\item \textbf{Command}: compel short action.
\item \textbf{Fear}: panic, flee, break morale.
\end{itemize}

\subsection*{Life \& Body TAGS}
\begin{itemize}[leftmargin=*]
\item \textbf{Mend}: close wounds, restore flesh, reduce Harm 1.
\item \textbf{Purify}: remove poison, corruption, disease.
\item \textbf{Strengthen}: enhance body, armor, senses.
\item \textbf{Waken}: counter sleep, paralysis, stun.
\item \textbf{Beast}: speak with or influence animals.
\end{itemize}

\subsection*{Space \& Motion TAGS (Always +2 DV Each)}
\begin{itemize}[leftmargin=*]
\item \textbf{Leap}: jump far, blink across short space.
\item \textbf{Fold}: short-range teleport, vanish–reappear.
\item \textbf{Gate}: long distance passage, open/close path.
\item \textbf{Gravity}: crush, lift, suspend, walk skyward.
\end{itemize}

\subsection*{Creation \& Transformation TAGS (Always +2 DV Each)}
\begin{itemize}[leftmargin=*]
\item \textbf{Create}: manifest matter briefly.
\item \textbf{Summon}: call a being or construct.
\item \textbf{Transmute}: turn one thing into another.
\item \textbf{Animate}: make objects act with intent.
\end{itemize}

\section*{Design Philosophy: Magic as Narrative Engine}\index{design intent!magic}

Magic in Fate's Edge serves specific design goals:

\begin{itemize}
    \item \textbf{Risk-Reward Balance}: Every magical act should feel consequential
    \item \textbf{Thematic Consistency}: Magic should reflect the world's metaphysics
    \item \textbf{Narrative Primacy}: Mechanics exist to serve interesting stories
    \item \textbf{Player Agency}: Magic should offer creative solutions, not bypass challenges
    \item \textbf{World Reactivity}: The setting should respond meaningfully to magical use
\end{itemize}

\subsection*{GM Guidance: Making Magic Feel Magical}

\begin{itemize}
    \item \textbf{Describe the Unseen}: When magic is cast, describe how the world reacts—air crackles, shadows deepen, spirits stir
    \item \textbf{Follow the Consequences}: Magical actions should have lasting effects on the narrative
    \item \textbf{Respect the Elements}: Backlash should feel philosophically appropriate
    \item \textbf{Highlight the Cost}: Make players feel the weight of their magical choices
    \item \textbf{Encourage Creativity}: Reward inventive uses of magic that enhance the story
\end{itemize}

\textbf{Remember}: In Fate's Edge, magic is never a shortcut. It's a pathway filled with wonders and dangers—a tool that changes both the world and the wielder. The dice are not your enemy; they're your collaborator in crafting a world where \textbf{true power always demands an equal price}.

