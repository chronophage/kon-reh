\chapter{Magic and Backlash}\index{magic}\index{backlash}

In \textbf{Fate's Edge}, magic is not a clean or safe art practiced in sterile towers. It is a \textbf{dangerous negotiation with forces beyond mortal comprehension}—a dance on the razor's edge between power and damnation. Every spell is a gamble where power weighs on one side of the scale and consequence balances on the other. As the GM, your role is to make magic \textbf{feel weighty}, \textbf{thematic}, and \textbf{alive with risk}—a force that reshapes reality at a terrible price.

\section*{The Metaphysical Foundation: Eight Elements}\index{Eight Elements}\index{metaphysics}

Magic in Fate's Edge flows from eight fundamental forces that shape existence itself. These elements are not just energy sources—they are \textbf{philosophical principles} that define how reality functions and how magic interacts with it. They represent the core tensions that drive the universe: permanence versus change, creation versus destruction, order versus chaos, life versus death.

\begin{fatebox}[The Eight Elements of Magic]
\begin{tabularx}{\textwidth}{lX}
\toprule
\textbf{Element} & \textbf{Domain and Philosophical Nature} \\
\midrule
Earth & Stability, foundation, permanence, material reality, mountains, bones, cities \\
Fire & Transformation, passion, destruction, creation, will, forge, revolution \\
Air & Freedom, thought, communication, wind, breath, ideas, movement, change \\
Water & Flow, emotion, adaptation, tides, blood, intuition, reflection, cycles \\
Fate & Destiny, patterns, inevitability, threads, prophecy, order, consequence \\
Life & Growth, vitality, healing, nature, birth, connection, ecosystem, renewal \\
Luck & Chance, opportunity, randomness, fortune, accident, serendipity, risk \\
Death & Endings, transition, memory, ghosts, sacrifice, silence, completion \\
\bottomrule
\end{tabularx}
\end{fatebox}

Each element has its opposite—Earth opposes Air's changeability, Fire counters Water's fluidity, Fate clashes with Luck's randomness, and Life battles Death's finality. This opposition shapes how backlash manifests when magic goes awry. When Earth magic fails, it might cause sudden shifts and instability (Air's domain); when Fire magic backfires, it might create unexpected flows or emotional turbulence (Water's domain).

\section*{Three Faces of Magic}\index{magic paths}

Magic in Fate's Edge is expressed through three interconnected paths. You may specialize in one, or mix them at greater bookkeeping cost. All paths share the same dice engine and SB/Obligation economies, but their flavor and risks differ.

\subsection*{Casting (Freeform)}\index{Casting}\index{freeform magic}

Requires Talent: \textbf{Caster's Gift} (2 XP). Grants access to Weave & Cast using the Eight Elements. Flexible, creative, and risky (Backlash on 1s).

\textbf{Weave \& Cast}: Casters describe the effect in terms of the Eight Elements. The GM sets DV and Effect based on scope.
\begin{itemize}
    \item \textbf{Weave}: Player builds dice pool and rolls. On success, they stabilize the spell's form.
    \item \textbf{Cast}: A second roll channels the effect into the world.
    \item \textbf{Backlash}: Any 1 rolled may cause narrative backlash related to the Element.
\end{itemize}

Limits: Casters can attempt any effect that can be described, but the larger the scope, the higher the DV. Improvisation is costly; reliable effects require repeated use and narrative justification.

\subsection*{Rites User (Runekeeper)}\index{Rites}\index{Runekeeper}

Requires Patron + Thiasos (Familiar) + Codex (4 XP). Grants access to a Patron's Rites. Structured, powerful, but debt-driven through Obligation.

\textbf{Requirements}: A Patron bond, a Thiasos (Familiar), and a Codex (4 XP) mark a character as a Runekeeper.

\textbf{Invocation}:
\begin{itemize}
    \item \textbf{Action Cost}: Invoking a Rite requires 1 Action.
    \item \textbf{Obligation}: Each Rite used marks Obligation on its clock.
    \item \textbf{Push It}: Once per Rite, you may Push to increase its duration or potency by +1 step at the cost of +1 Obligation.
\end{itemize}

\textbf{Obligation Clock}: Tracks the Patron's claim. When full, the GM resolves the debt in-fiction. Obligation is reduced through service or downtime actions.

\subsection*{Invoker (Ritualist's Path)}\index{Invoker}\index{Ritualist}\index{Grimoire}

Requires the \textbf{Invoker's Grimoire} talent (6 XP) and study of specific rites. Grants deep knowledge of ritual magic and the ability to perform Rites from multiple Patrons. Symbols are potent tools that enhance this knowledge.

\begin{fatebox}[Invoker Path Features]
\begin{tabularx}{\textwidth}{lX}
\toprule
\textbf{Feature} & \textbf{Description and Limitations} \\
\midrule
Invoker's Grimoire & Major Talent, 6 XP. Grants knowledge of Ritual Magic theory and access to perform a limited number of Rites. \\
Ritual Repertoire & Start with knowledge of \textbf{2} Low or Standard Rites from any Patrons you research. Learn new Rites through Downtime study (see below). \\
Ritual Invocation & Takes \(\text{DV}\) rounds (default 2--3 rounds). Requires specific components/materials. \\
Base Cost & Mark \textbf{+1 Obligation} when you successfully resolve any known Rite (Low or Standard). \textit{(High-Power/High Rites are normally unavailable; if the Keeper permits, treat their \emph{base} Obligation as +2.)}\\
Symbol Enhancement & Possessing the correct Patron's Symbol for a Rite you are casting reduces its \textbf{DV by 1} and its \textbf{Obligation cost by 1} (minimum 0). Only one Symbol may apply to a given Rite. \\
\textbf{No Symbol (Explicit Penalties)} & You may attempt the Rite without the Patron's Symbol, but suffer: \textbf{+1 DV}, \textbf{+1 Obligation} (in addition to Base), and \textbf{+1 round} casting time. On \emph{Partial/Failure}, generate \textbf{+1 extra SB}. \\
Symbol Display & The Symbol must be visible/active throughout the ritual. If it is concealed, disrupted, or removed mid-cast: immediately \textbf{+1 DV}; on Failure, apply \emph{Backlash} (see below). \\
Crack the Seal & Desperate technique. Instantly cast any known Rite by setting the relevant Symbol to \textsc{Compromised}. Mark \textbf{+2 Obligation} (\textbf{+3} for High-Power Rites). Does not reduce Base Obligation below 0. \\
Optional Push & Invokers may \emph{Push} a Rite: choose one (\(+2\) dice \emph{or} +1 Effect \emph{or} resolve one round faster). Always mark \textbf{+1 Obligation} \emph{and} generate \textbf{1 SB}, in addition to other costs. \\
Cross-Resonance & If you cast Rites from \emph{different Patrons} in the same scene, each Patron after the first adds \textbf{+1 DV} to that Rite. \\
\bottomrule
\end{tabularx}
\end{fatebox}

\paragraph{Symbol States \& Repair}
\begin{itemize}
  \item \textsc{Compromised:} A Symbol set to \textsc{Compromised} (e.g., via \emph{Crack the Seal}) provides \emph{no} DV/Obligation reduction until repaired. Casting with a \textsc{Compromised} Symbol imposes \(-1\) die on the Casting Test.
  \item \textsc{Shattered:} If you \emph{Crack the Seal} again while the Symbol is \textsc{Compromised}, it becomes \textsc{Shattered} and cannot be used until replaced (Asset lost).
  \item \textbf{Repair (Downtime):} 1 day of focused work and a \emph{Craft or Lore + Tinker} test vs.\ DV~3. Success: clear \textsc{Compromised}. Failure: no progress. Alternatively, spend \textbf{1 XP} to auto-repair.
\end{itemize}

\paragraph{Backlash \& Failure (Explicit)}
\begin{itemize}
  \item \textbf{Success:} Rite resolves; apply Base/added Obligation and any SB from Push or No-Symbol clauses.
  \item \textbf{Partial:} Effect \(-1\) step \emph{or} shortened duration; mark \textbf{Fatigue 1}. If cast \emph{without} a Symbol, Keeper gains \textbf{+1 SB} (in addition to normal SB generation).
  \item \textbf{Failure:} No effect; mark \textbf{Fatigue 1}; Keeper gains \textbf{+1 SB}. Then test \emph{Spirit + Resolve} vs.\ DV~3:
    \begin{itemize}
      \item On Fail: suffer \textbf{Harm 1 (Shock)} or start \textbf{Backlash Static [4]} (Keeper's choice).
      \item If the Symbol was disrupted/hidden mid-cast \emph{or} you \emph{Cracked the Seal}: upgrade to \textbf{Harm 2 (Shock)}.
    \end{itemize}
  \item \textbf{Interrupted:} Harm, Silence, or disruption before resolution counts as \emph{Failure}.
\end{itemize}

\textbf{Example:} Magus Vex, bearing the \textbf{Invoker's Grimoire}, has studied the rites of Raéyn and the Sealed Gate. He knows Raéyn's \emph{Whispering Currents} (Low) and the Sealed Gate's \emph{Circle of Denial} (Standard). Faced with a collapsing tunnel, he attempts the Sealed Gate's ritual. It's a Standard Rite, so \textbf{DV 3}, taking \textbf{3 rounds}, and costs \textbf{+1 Obligation}. He has the Sealed Gate's Symbol, reducing the DV to \textbf{2} and the Obligation cost to \textbf{0}. When ambushed, he needs quick protection. He \textbf{Cracks the Seal} on the \emph{Circle of Denial}. The Symbol becomes \textsc{Compromised}, the Rite is instant, and he marks \textbf{+2 Obligation}. Later, needing to bind a particularly strong foe, he \textbf{Pushes} the Rite, marking an additional \textbf{+1 Obligation} and generating \textbf{1 SB}; the barrier strengthens. If he tried a Raéyn Rite afterwards in the same scene, \emph{Cross-Resonance} would add \textbf{+1 DV} to that casting.

\subsubsection*{Learning New Rites}
An Invoker can expand their \textbf{Ritual Repertoire} through dedicated study during \textbf{Downtime}.
\begin{itemize}
    \item \textbf{Cost:} 1 week of Downtime + 2 XP.
    \item \textbf{Requirement:} Access to texts, a teacher, or direct observation of the Rite being performed by another adept.
    \item \textbf{Test:} \emph{Lore + Investigation} (or a relevant skill) vs.\ DV~3--5 (based on Rite rarity/complexity).
    \item \textbf{Success:} Add the Rite to your Ritual Repertoire.
    \item \textbf{Failure:} Cannot learn this specific Rite for a significant time (GM discretion). The Keeper may set a relevant Complication (e.g., \emph{Forbidden Knowledge Pursued}).
\end{itemize}

\subsubsection*{Symbols as Assets}
\begin{itemize}
    \item A Patron's Symbol is a \textbf{Minor Asset (4 XP)} whose primary value is as a \textbf{ritual focus/component}.
    \item You \emph{can} attempt any ritual \textbf{without} the Symbol, but you incur these \textbf{No Symbol} penalties: \textbf{+1 DV} \emph{(and therefore +1 round to cast, since casting time = DV rounds)}, \textbf{+1 Obligation} \emph{(in addition to Base)}, and on \emph{Partial/Failure} the Keeper gains \textbf{+1 extra SB}.    \item Symbols can be \textbf{maintained/upgraded} like other Assets. Example upgrades: \emph{Hardened} (ignore the first application of \textsc{Compromised} per session), \emph{Bright} (treat as \emph{visible} for Symbol Display while concealed on your person).
\end{itemize}

\subsection*{Borrowed Grace}
\label{talent:borrowed-grace}
\index{Talents!Invoker}\index{Imbuement!Lesser}

\textbf{Type:} Invoker Talent — \textit{Lesser Imbuement}

\subsubsection*{Use}
\begin{itemize}
  \item \textbf{Cost:} \textbf{1 Boon}, \textbf{1 action}.
  \item \textbf{Effect (pick one on use):} \textbf{+1 Melee} \emph{or} \textbf{+1 Thematic} (your table's signature/thematic Skill).
  \item \textbf{Duration:} \textit{Single action/attack} (instantaneous boost only).
  \item \textbf{Requirement:} Wield/display the relevant Patron's \textbf{Symbol}.
  \item \textbf{Obligation:} Immediately mark \textbf{+1 Obligation} to that Patron (see \S\ref{sec:obligation}).
  \item \textbf{Limits:} Cannot be extended, stacked, or \emph{Pushed} for duration. Using \emph{Borrowed Grace} while the Symbol is \textsc{Compromised} lowers your \textbf{Position} by one step \emph{(or imposes \(-1\) die if already \textbf{Desperate})}.)
\end{itemize}

\section{Obligation Capacity}

A character’s \textbf{Obligation Capacity} equals Spirit + Presence.
Track total Obligation segments across all Patrons (or Symbols, for Invokers).

\begin{itemize}
  \item \textbf{Exceeding Capacity:} For each segment above Capacity, mark 1 Fatigue. The character cannot Invoke Rites or perform rituals until Obligation is reduced below Capacity.
  \item \textbf{Overload (≥ 2x Capacity):} Clear all Fatigue, take +1 Harm, and suffer immediate Patron intrusion (Claim, demand, or narrative cost). Downtime cannot reduce Obligation until Harm is addressed.
  \item \textbf{Resolution:} Reduce Obligation through Downtime service, Patron tasks, ritual cleansing, or story resolution.
\end{itemize}

\textbf{Example:} Spirit~2 + Presence~3 = Capacity 5.
6 segments → Fatigue~1.
7 segments → Fatigue~2.
10 segments → Harm~1.
11 segments → Harm~2.

\begin{table}[h]
\centering
\caption{Universal Push It Costs}
\begin{tabular}{|l|l|}
\hline
\textbf{Cost Component} & \textbf{Effect} \\
\hline
+1 SB & Escalate effect immediately \\
+1 Fatigue & Immediate physical/mental strain \\
+1 Corruption Clock Segment & Long-term Patron influence (unless otherwise specified) \\
GM spends 1 SB & Thematic complication (unless otherwise specified) \\
\hline
\end{tabular}
\end{table}

Note: Some talents, Rites, or magical paths may specify alternative corruption costs or additional consequences for Push It actions. When explicitly stated, those specific rules override the universal costs.

\paragraph{Clearing Corruption}
Corruption may be reduced through \textit{purging rituals}, such as exorcisms, sacred songs, or rites of contrition. 
These require a test (typically \textbf{Lore + Spirit}) against a DV equal to the character’s current corruption level.  
On success, reduce corruption by 1. On failure, the corruption manifests violently, imposing a temporary Condition or advancing its narrative expression.  

Optional: A \textbf{Story Beat} may also be spent to attempt such a ritual, representing the personal cost of atonement. Patrons may demand specific acts of service, sacrifice, or obligation as part of the purging process.
\section*{Summoning: Binding Outsider Forces}\index{Summoning}\index{Pact-Whisperer}

%----------------------------------------
%----------------------------------------
\%----------------------------------------
\section{Summoning (Pact-Whisperer)}
\label{subsec:summoning}

Summoning is the disciplined art of calling and binding Outsiders for temporary aid.  
This path requires the \textbf{Pact-Whisperer} Talent (2 XP).  
Each summoned being is restrained by a metaphysical tether called a \textit{Leash}, representing the summoner's control and the strain of sustaining the bond.

\paragraph*{Talents \& Access.}
\begin{itemize}
  \item \textbf{Lesser Pactwright:} You may \emph{Call} spirits of \textbf{Cap~1}.
  \item \textbf{Greater Pactwright:} You may also \emph{Call} spirits of \textbf{Cap~3}.
  \item \textbf{Dual Pactwright:} With both Lesser and Greater Pactwright, you may maintain one spirit of each Cap simultaneously.
\end{itemize}

\begin{fatebox}[Summoning Core Mechanics]
\begin{tabularx}{\textwidth}{lX}
\toprule
\textbf{Mechanic} & \textbf{Description and Requirements} \\
\midrule
\textbf{Call} & \emph{1 Action} to manifest the spirit at \textit{Near} range; choose a Spirit Template aligned to fiction or Patron domain. \\
\textbf{Bind} & Spend 1 Boon \emph{or} mark 1 Fatigue to establish initial control. \\
\textbf{Leash Capacity} & Set Leash Capacity $=$ \textbf{Cap $+$ Spirit} \emph{segments}. \\
& (\textit{Cap} is the Outsider's tier: Cap~1 for Lesser, Cap~3 for Greater.) \\
\textbf{Tick Leash} & Whenever the spirit takes Harm, you command it against its nature, you split focus, a rival contests it, it moves \textit{Close} $\rightarrow$ \textit{Far} rapidly, or crosses a \texttt{[WARD]} \big($\mathrm{DV}=\mathrm{Cap}$\big). \\
\textbf{Departure} & When the Leash fills, the spirit acts to its nature once, then departs (or turns hostile at GM discretion). \\
\bottomrule
\end{tabularx}
\end{fatebox}

\paragraph*{Spirit Bond Progression.}
Each spirit you summon regularly can develop a \textbf{Spirit Bond Clock [4]}:
\begin{itemize}
  \item Mark segments for successful commands, shared victories, or acts of mutual aid.
  \item At 2 segments: +1 die to communicate with this spirit type.
  \item At 4 segments: Spirit grants +1 Boon when departing naturally and becomes \textbf{Favored} (Leash reduced by 1).
  \item Reset: Spirit departs as ally and may return in future scenes with +1 Effect.
\end{itemize}

\paragraph*{Spirit Specialization Paths.}
Spirits can develop specialized capabilities through repeated summoning:
\begin{itemize}
  \item \textbf{Combat Specialist:} +1 Harm in melee; ignore first Harm when attacking.
  \item \textbf{Scout Form:} Extended range, stealth bonuses, can carry small items.
  \item \textbf{Utility Spirit:} Perform simple tasks (lockpicking, carrying, environmental interaction).
  \item \textbf{Shield Guardian:} Interpose to protect allies; convert Harm to Fatigue.
\end{itemize}

\paragraph*{Procedure.}
\begin{enumerate}
  \item \textbf{Call (1 Action):} A spirit manifests at \textit{Near}. Choose a Spirit Template appropriate to the scene or Patron.
  \item \textbf{Bind:} Spend 1 Boon \emph{or} mark 1 Fatigue to anchor the connection.
  \item \textbf{Leash Capacity:} Record Leash Capacity $=$ \textbf{Cap $+$ Spirit} \emph{segments}. Draw a clock to track strain (the Leash).
  \item \textbf{Command:} Each round, issuing a meaningful order uses your Action. Commands contrary to the spirit's nature tick the Leash.
  \item \textbf{Maintain:} If you split focus or perform other significant actions while it acts on your order, tick the Leash.
  \item \textbf{Departure:} When the Leash fills, the spirit acts to its nature once, then departs. Use this to escalate or reveal consequences.
\end{enumerate}

\paragraph*{Enhanced Action Economy.}
\begin{itemize}
  \item \textbf{Spirit Assist:} Once per scene, the spirit can grant +2 dice to an ally's roll instead of acting.
  \item \textbf{Quick Command:} Simple commands (attack, move, defend) do not require a full Action.
  \item \textbf{Spirit Resonance:} When commanding multiple spirits of the same type, +1 Effect.
  \item \textbf{Honorable Departure:} Voluntarily end a summon early to gain +1 Boon and reduce Leash by 2.
\end{itemize}

\paragraph*{Economy \& Limits.}
\begin{itemize}
  \item \textbf{Boon Finesse:} Once per round, spend 1 Boon to clear 1 Leash tick (before it fills). Represents appeasement or renewed focus.
  \item \textbf{Action Economy:} Issuing commands uses your Action; most spirits act immediately after their summoner.
  \item \textbf{Concurrency:} Only one active summoned spirit at a time unless a Talent states otherwise. Exceeding this limit inflicts 1 Fatigue per extra Cap point.
  \item \textbf{Downtime:} All summons end at Downtime unless explicitly sustained by a Rite or Asset.
\end{itemize}

\paragraph*{New Talent: Spirit Synergy (4 XP).}\\
\textbf{Requirements:} Pact-Whisperer, Lesser Pactwright.\\
\textbf{Effect:} When commanding two or more spirits simultaneously, reduce each Leash by 1 segment and gain +1 die to Command rolls.

\paragraph*{New Talent: Bonded Summoner (3 XP).}\\
\textbf{Requirements:} Pact-Whisperer, Spirit Bond Clock at 2+ segments with any spirit type.\\
\textbf{Effect:} Favored spirits reduce their Leash cost by 2 (minimum 3). Once per session, recall a departed Favored spirit by spending 2 Boons.

\paragraph*{Example.}
\textit{Kestra calls a Cap~3 fire elemental to aid in battle. She spends 1 Boon to Bind it.  
The elemental's Leash Capacity is 7 segments (\textit{Cap}~3 $+$ \textit{Spirit}~4). When it takes Harm, the GM ticks the Leash. Later, Kestra splits focus to issue orders while attacking, ticking again.  
Careful management and Boon Finesse keep the bond stable---until the elemental's fury tests her will. After the battle, she marks her Spirit Bond Clock +1 for the shared victory.}

%----------------------------------------
\section{Talent: Cantor's Path --- ``Songs of the Low Rites''}
\label{talent:cantors-path}

\begin{tcolorbox}[colback=black!3,colframe=black!40!white,title={Cantor's Path}]
You echo the liturgies of Patrons through breath and string. Not a sworn celebrant but a perilous mimic, you weave Low Rites into song. It is slower, riskier, and beautiful---but never free.
\end{tcolorbox}

\paragraph*{Type} Major Talent (8 XP) \quad
\paragraph*{Prerequisites} \textbf{Lore 1+}, \textbf{Performance 2+}, \textbf{Presence 2+} \quad
\paragraph*{Access} Any character (does not require Thiasos membership).

\subsection*{Effect}
You may learn and perform \textbf{Low Rites as Songs}. Each Song counts as knowing the associated Low Rite for performance purposes only.

\begin{itemize}
  \item \textbf{Casting Test:} \emph{Lore + Performance vs.\ DV} (default DV = 2--3).
  \item \textbf{Action Economy:} \emph{1 action to begin;} the Song \emph{resolves at the start of your next turn} unless accelerated.
  \item \textbf{Scope:} \emph{Low Rites only.} Standard/High Rites remain exclusive to Patrons and Thiasos initiates.
  \item \textbf{Costs:} Pay any \emph{materials} listed. On success you do \emph{not} mark Obligation.
\end{itemize}

\subsection*{Performance Integration}
Songs are most effective when performed as part of social performances:
\begin{itemize}
  \item \textbf{Audience Awareness:} Perform in front of 5+ observers for +1 die but +1 Corruption risk.
  \item \textbf{Cultural Context:} Appropriate venues/occasions grant +1 Effect.
  \item \textbf{Social Momentum:} Successful performances create opportunities for additional Songs in the same scene.
\end{itemize}

\subsection*{Song Repertoire Progression}
Develop a \textbf{Repertoire Clock [6]} to track learned Songs:
\begin{itemize}
  \item Mark a segment for each \emph{unique} Song learned through practice or exposure.
  \item At 2 segments: Reduce base DV of Songs by 1 (minimum 2).
  \item At 4 segments: Gain +1 die to Song performances.
  \item At 6 segments: Learn one \emph{Standard Rite as a Song} (temporary, requires ongoing practice).
\end{itemize}

\subsection*{Corruption Clock}
\begin{itemize}
  \item You gain a personal \textbf{Corruption Clock} with segments equal to your \textbf{Body} rating.
  \item \textbf{Mark Corruption when:}
    \begin{itemize}
        \item You \textbf{Push It} (Song resolves immediately).
        \item You perform a \textbf{Resonant Rite}.
        \item The Keeper spends a Story Beat involving your psionic/occult activities.
    \end{itemize}
  \item When the Clock fills:
    \begin{itemize}
      \item You immediately gain a \textbf{thematic benefit} and \textbf{drawback} from the last Patron whose Rite you performed.
      \item All of your followers, retainers, or familiars also gain a trait of the same corruption.
      \item Reset the Clock to empty.
    \end{itemize}
  \item Corruption traits can be \textbf{Embraced} for permanent thematic advantages (see below).
\end{itemize}

\subsection*{Thematic Corruption Benefits}
Instead of purely punitive effects, Corruption creates character-defining traits:
\begin{description}
  \item[Ikasha (Shadow):] +1 die to Stealth in shadows, but $-1$ die in bright light; always noticed by shadow-dwellers.
  \item[Inaea (Mercy):] +1 die to social manipulation, but $-1$ die when alone; compelled to offer aid to the helpless.
  \item[Isoka (Change):] +1 die to escape/transform actions, but $-1$ die to maintain consistency; physical changes become visible.
  \item[Raéyn (Sea):] +1 die to water/navigational tasks, but $-1$ die on land; attracts sea creatures.
  \item[Aveh (Freedom):] +1 die to escape/avoidance, but $-1$ die to commitments; leaves traces of passage.
\end{description}

\subsection*{Resonant Rites}
Some powerful or thematically significant Low Rites carry the weight of the Patron's direct influence. Performing these Rites is a conscious act of drawing deep power.
\begin{itemize}
    \item When learning a Song that mimics such a Rite, the GM or the rules text will designate it as \textbf{Resonant}.
    \item Performing a \textbf{Resonant Rite Song} successfully allows you to mark +1 segment on your Corruption Clock. This represents the lingering echo of power.
    \item \textbf{Choosing to Resonate} is optional. You can perform the Rite normally without marking Corruption.
    \item This choice adds a layer of strategy: is the Rite's power worth the potential long-term cost?
\end{itemize}

\subsection*{Song Synergy System}
Create combinations and interactions between Songs:
\begin{itemize}
  \item \textbf{Harmony:} Performing two compatible Songs grants +1 Effect to both.
  \item \textbf{Counterpoint:} Using opposing Songs can cancel negative effects.
  \item \textbf{Chorus:} With allies, combine Songs for amplified effects (+1 Effect per participant).
\end{itemize}

\subsection*{Outcomes}
\begin{description}
\item[Success:] The Low Rite takes effect as written.
\item[Partial:] The Rite manifests with reduced effect (one step) or shortened duration. Mark \textbf{Fatigue 1}.
\item[Failure:] No effect; mark \textbf{Fatigue 1} and the Keeper gains \textbf{+1 SB (Hearts)}.
\item[Interrupted:] Harm, Silence, or disruption before resolution = treat as Failure.
\end{description}

\subsection*{Push It}
When you Push:
\begin{itemize}
  \item The Song resolves immediately instead of next round.
  \item Mark \textbf{Fatigue 1}.
  \item \textbf{Mark +1 segment on your Corruption Clock.}
  \item The Keeper immediately triggers a \textbf{Story Beat}, representing fallout from a Patron, the Road, or social attention.
\end{itemize}

\subsection*{Enhanced Departure Options}
\begin{itemize}
  \item \textbf{Graceful Coda:} End a Song early to gain +1 Boon and reduce Corruption by 1 (if any is marked).
  \item \textbf{Lingering Verse:} Song effect continues for one round after ending, but mark +1 Fatigue.
  \item \textbf{Audience Impact:} A successful Song performance improves social Position +1 for the next interaction.
\end{itemize}

\subsection*{Limits \& Interactions}
\begin{itemize}
  \item \textbf{Stacking:} Cannot benefit from the same Rite twice.
  \item \textbf{Visibility:} Songs are inherently noticeable. On Failure or Push, assume observers take note.
  \item \textbf{Silence/Disruption:} Impose $-1$ to $-3$ dice at the Keeper's discretion.
  \item \textbf{Obligation Transference:} Whenever a Rite would normally increase Obligation, it instead increases Corruption by that amount.
\end{itemize}

\subsection*{Downtime Activities}
\begin{itemize}
  \item \textbf{Song Composition:} Practice and refine Songs, potentially reducing their DV or Corruption risk.
  \item \textbf{Performance Practice:} Improve Performance skill and social reputation.
  \item \textbf{Patron Study:} Research new Rites to add to your Repertoire.
  \item \textbf{Audience Building:} Cultivate followers who provide +1 die to future performances.
\end{itemize}

\subsection*{New Talent: Resonant Performance (3 XP)}
\textbf{Requirements:} Cantor's Path, Performance 2+ \\
\textbf{Effect:} When performing a Song in front of an audience of 5+ people, reduce Corruption generation by 1 (minimum 1) and gain +1 die to the performance.

\subsection*{New Talent: Song Weaver (4 XP)}
\textbf{Requirements:} Cantor's Path, Repertoire Clock at 4+ segments \\
\textbf{Effect:} Combine two compatible Songs for +1 Effect to both. Once per scene, create Harmony between Songs for all participants.

\subsection*{Corruption Fading}
\label{subsec:corruption-fading}
\index{Corruption!Fading}

Corruption does not fade easily. It requires deliberate action and often, a price.
\begin{description}
  \item[\indexterm{Natural Fading}]  
  At the beginning of each Downtime, reduce a character's current \textbf{Corruption} by 1 segment, \emph{if no new segments were added during the last session/arc}. Lingering effects persist until actively addressed.

  \medskip
  \item[\indexterm{Act of Contrition}]  
  Perform a genuine act that contradicts the Patron's influence or repairs its harm (GM/Player agreement on suitability). \textbf{Effect:} Remove 1 Corruption segment and clear one persistent effect. Costs the character something significant.

  \medskip
  \item[\indexterm{Ritual Purification}]  
  Undertake a significant act of cleansing (pilgrimage, service, seeking rival absolution). \textbf{Effect:} Remove 2 Corruption segments and clear all persistent effects. Likely requires marking Fatigue or temporary Obligation.

  \medskip
  \item[\indexterm{Embrace Corruption}] \label{talent:embrace-corruption}
  \textbf{Type:} Major Talent (6 XP) \quad
  \textbf{Prerequisite:} 2+ levels of Corruption. \\
  You accept the creeping decay, transforming it into a permanent Talent. \textbf{Embracing never reduces Corruption---it reshapes it.} The deeper the corruption, the greater the power and the cost.
    \begin{itemize}
        \item Gain a \textbf{Minor} permanent thematic boon/condition related to the Patron (e.g., +1 die to Stealth in shadows for Ikasha, but $-1$ die in bright light).
        \item Your Corruption cannot naturally fade below the level at which you Embraced it.
        \item The Keeper gains +1 SB to spend against you related to that Patron's themes.
    \end{itemize}
    \textbf{Narrative Integration:} This Talent represents the Faustian bargain. Players gain agency over their corruption, ensuring that it always carries meaningful consequences.

  \medskip
  \item[\indexterm{Patron Bargain}]  
  Negotiate directly with the Patron. \textbf{Effect:} Remove 1--3 Corruption segments based on the exchange's gravity. Always comes with a narrative cost or condition set by the Keeper.

  \medskip
  \item[\indexterm{Persistence}]  
  Corruption effects do not clear through rest. They require deliberate narrative resolution or specific actions listed above. Every method is an opportunity for character development.
\end{description}

\paragraph*{High Cantor (18 XP Prestige Talent)}%
\textit{Prerequisite: Tier II+, Cantor's Path, Performance 3+}\\[3pt]
You have learned to weave the sacred tongue through breath and pulse rather than word or gesture. You may now learn and cast \textbf{Standard Rites}, as a \textbf{High Cant}.
\begin{itemize}
  \item The Rite resolves instantly.
  \item Gain +1 die to its primary effect.
  \item \textbf{Mark +1 segment on your Corruption Clock.}
\end{itemize}

\noindent
\textbf{Special:}  
Each Patron's resonance colors the manifestation differently---flame halos for the Oath, rippling silence for the Choir, tolling harmonics for the Confessor. High Canting is recognizable to other adepts; it draws attention. Repeated use within a single scene risks moral fatigue: add +1 DV to all subsequent \emph{Resolve} rolls against fear, charm, or social pressure in that scene.

\begin{quote}
``The louder the hymn, the nearer the flame.''
\end{quote}

\subsection*{Bookkeeping Light (Table Guidance)}
To keep play fast, track at most \emph{two} clocks for a Cantor:
\begin{enumerate}
  \item \textbf{Corruption Clock} (segments = Body).
  \item \textbf{Repertoire Clock [6]} (optional; advances only when a new Song is learned).
\end{enumerate}
No per-Song timers are required beyond \emph{Push} and \emph{Outcome} handling. Harmony/Counterpoint/Chorus provide situational modifiers and never introduce new clocks.


\section*{Magical Arts and Specialization}\index{Magical Arts}

A character's Art represents their personal approach to magic—the techniques, tools, and philosophies that define their craft. When a character gains magical capability, they define their Art with specific parameters.

\begin{fatebox}[Defining Your Magical Art]
\begin{tabularx}{\textwidth}{lX}
\toprule
\textbf{Component} & \textbf{Description and Examples} \\
\midrule
Gesture \& Medium & Ink sigils, sung names, lantern-light, bone charms, legal contracts, salt-threads \\
Elemental Alignment & Choose 2 primary Elements the Art typically engages with (Fire+Earth, Air+Water, etc.) \\
Thematic Focus & Destruction, protection, revelation, transformation, communication, healing \\
Cultural Roots & High Elf crystal-song, Ykrul blood-runes, Aeler spirit-whispers, Human alchemy \\
\bottomrule
\end{tabularx}
\end{fatebox}

\subsection*{Art in Play}\index{Magical Arts!in play}

The fictional positioning of a character's Art matters significantly:

\begin{itemize}
    \item \textbf{Spotlight Bump (1/scene)}: If the Art is clearly honored in fiction (right tools, time, setting), gain +1 die on the Cast roll
    \item \textbf{Off-Style Strain}: If forced to work against the Art's nature (no tools, hostile environment), suffer worse Position or accept extra Backlash
    \item \textbf{Art-Based Backlash}: Consequences should reflect the Art's themes and elements
\end{itemize}

\section*{Tags: The Language of Magical Effects}\index{Tags}\index{magical effects}

Tags provide a common language for describing magical effects and their limitations. They only function when printed on a Talent, Ability, or Spell result.

\begin{fatebox}[Common Magical Tags and Effects]
\begin{tabularx}{\textwidth}{lX}
\toprule
\textbf{Tag} & \textbf{Effect and Usage Guidelines} \\
\midrule
[DISPEL] & End an ongoing magical effect/construct. DV by fiction. \\
[COUNTER] & Interrupt a cast/rite in progress. DV by fiction. \\
[BARRIER] & Create cover/obstruction. DV by fiction. \\
[SEAL]/[UNSEAL] & Lock or unlock a container/door/portal. DV by fiction. \\
[VEIL] & Obscure a person/thing/zone. DV by fiction. \\
[REVEAL] & Expose illusions, disguises, hidden clauses. DV by fiction. \\
[MARK] & Tag a target for tracking or leverage. DV by fiction. \\
[CURSE] & Inflict a sticky hindrance with a clear release. DV by fiction. \\
[CLEANSE] & Remove/suppress a condition. DV by fiction. \\
[FORTIFY] & Harden against a vector. DV by fiction. \\
[COMMAND] & Issue a clear order to a sapient target. DV by fiction. \\
[OATH] & Bind parties to terms; breaking has teeth. DV by fiction. \\
[SANCTIFY] & Consecrate a zone to a code/patron. DV by fiction. \\
[PASSAGE] & Declare a route as permitted/easy. DV by fiction. \\
[TRANSPORT] & Move a target across an obstacle. DV by fiction. \\
[CONJURE] & Create a useful object/cover/hazard. DV by fiction. \\
[WARD] & Challenge Outsiders crossing a warded edge/zone. DV = target Cap. \\
[BANISH] & Drive a visible Outsider toward departure. DV = target Cap. \\
[UNWARD] & Unmake/suppress a [WARD]. DV by fiction. \\
\bottomrule
\end{tabularx}
\end{fatebox}

Tags work within consistent parameters:
\begin{itemize}
    \item \textbf{DV by Fiction}: Potency, preparation, and opposition set difficulty
    \item \textbf{Duration}: Typically "Scene" unless specified otherwise
    \item \textbf{Stacking}: No same-source stacking; identical tags use strongest instance
\end{itemize}

\section*{Backlash: The Price of Power}\index{Backlash}

Backlash represents magic escaping control—the inevitable consequence of wielding forces beyond mortal comprehension. It's never arbitrary; backlash always reflects the elements involved and their philosophical oppositions.

\subsection*{Backlash Triggers and Severity}

Backlash occurs when magic goes awry:
\begin{itemize}
    \item \textbf{Primary Trigger}: Partial or Miss on either the Weave or Cast roll
    \item \textbf{Secondary Trigger}: Hit showing two or more 1s (minor backlash rides success)
    \item \textbf{SB Integration}: Backlash does not generate extra SB—it's how GM spends SB from rolled 1s
\end{itemize}

Backlash colors the cost of magic and is always expressed through fiction first.

\begin{fatebox}[Backlash Menu]
\begin{tabularx}{\textwidth}{lX}
\toprule
\textbf{Backlash Type} & \textbf{Effect} \\
\midrule
Position Shift & Worsen Position by 1 step for current or next action \\
Fleeting Harm/Condition & Sear, vertigo, chill that matters for this scene \\
Exposure/Noise & Draws notice or complicates stealth \\
Resource Drain & Time, focus, or component damaged \\
Collateral Spark & Threatens ally or fragile thing nearby \\
\bottomrule
\end{tabularx}
\end{fatebox}

\subsection*{Elemental Backlash Coloring}\index{Backlash!elemental}

On Partial/Miss (or double-1s on a Hit), color consequences by Element:

\begin{fatebox}[Elemental Backlash Coloring]
\begin{tabularx}{\textwidth}{lX}
\toprule
\textbf{Element Pair} & \textbf{Minor Backlash} \\
\midrule
Earth / Fate & Slips, binds, encumbrance \\
Fire / Life & Smoke, sparks, heat \\
Air / Luck & Scatter, misheard words \\
Water / Dreams & Slippery tide, slow gear \\
Fate / Earth & Probability resists \\
Life / Fire & Growth surge, vines tether \\
Luck / Air & Odds flip \\
Death / Water & Whispers, chill \\
\bottomrule
\end{tabularx}
\end{fatebox}

Backlash should always feel thematic to the magic employed:
\begin{itemize}
    \item \textbf{Fire Magic}: Burns, flares, smoke, heat exhaustion, uncontrolled fires
    \item \textbf{Water Magic}: Flooding, slick surfaces, damp-related rot, emotional turbulence
    \item \textbf{Earth Magic}: Tremors, collapsing structures, immobilization, heavy burdens
    \item \textbf{Air Magic}: Unexpected winds, carried sounds, vertigo, scattered plans
    \item \textbf{Fate Magic}: Closed options, inevitable consequences, prophetic nightmares
    \item \textbf{Luck Magic}: Allied misfortunes, fragile successes, random complications
    \item \textbf{Life Magic}: Overgrowth, sympathetic pain, unnatural hunger, fertility curses
    \item \textbf{Death Magic}: Ghostly echoes, premature aging, silence, memory loss
\end{itemize}

\section*{Ritual Casting: Collective Magic}\index{ritual casting}

Some workings require multiple casters pooling their strength. Rituals allow for greater effects but multiply risks.

\subsection*{Ritual Procedure}

\begin{enumerate}
    \item \textbf{Declaration}: Primary caster states intent and gathers participants
    \item \textbf{Channel Together}: All participants contribute (Scene-long action)
    \item \textbf{Weave}: Primary caster shapes combined Potential (Scene-long action)  
    \item \textbf{Backlash}: Consequences affect all participants based on their contribution
\end{enumerate}

\subsection*{Ritual Mechanics}

\begin{itemize}
    \item \textbf{Helper Cap}: Primary caster can draw on ceil(Arcana/2) helpers (max 3)
    \item \textbf{Skill Flexibility}: Helpers may use different relevant skills if fictionally distinct
    \item \textbf{Risk Distribution}: SB from Channel affects individual rollers; SB from Weave affects primary caster
\end{itemize}

\section*{Magic in Combat}\index{magic combat}

Spellcasting in combat follows the same principles but with heightened stakes and immediate consequences.

\subsection*{Combat Casting Considerations}

\begin{fatebox}[Magic in Combat: Position and Effect]
\begin{tabularx}{\textwidth}{lX}
\toprule
\textbf{Position} & \textbf{Effect on Magical Actions} \\
\midrule
Dominant & +1 die to Channel; reduced Backlash risk; can maintain subtle effects \\
Controlled & Standard casting conditions; typical risk/reward balance \\
Desperate & -1 die to Channel; increased Backlash severity; may attract unwanted attention \\
\bottomrule
\end{tabularx}
\end{fatebox}

\subsection*{Tactical Magic Applications}

Magic can reshape combat dynamics:
\begin{itemize}
    \item \textbf{Position Warfare}: Spells that create cover, elevate positions, or restrict movement
    \item \textbf{Morale Effects}: Magic that inspires allies or terrifies enemies
    \item \textbf{Environmental Control}: Creating hazards, altering terrain, manipulating weather
    \item \textbf{Resource Denial}: Destroying enemy equipment, exhausting their supplies
\end{itemize}

\section*{Prestige Magical Abilities}\index{Prestige Abilities!magical}

High-level magical talents represent profound mastery or unique cultural inheritances.

\begin{fatebox}[Example Prestige Magical Abilities]
\begin{tabularx}{\textwidth}{lX}
\toprule
\textbf{Ability} & \textbf{Description and Requirements} \\
\midrule
Ways-Walker's Step & Observe perfect echo of past event (1/arc); GM banks +2 SB; reveals hidden truths (Req: Wits 5, Arcana 4) \\
Warglord & Unify scattered warbands into host for season; track Logistics and Grudge clocks (Req: Body 5, Command 3) \\
Spirit-Shield & Erase up to 3 SB from ally's roll (1/session); caster takes Fatigue +1 and GM banks +1 SB (Req: Spirit 4, Insight 3) \\
Elemental Mastery & Choose one Element; gain +2 dice when using it, but backlash from opposite element is doubled \\
\bottomrule
\end{tabularx}
\end{fatebox}

\section*{Free Casting (TAGS System)}
Some casters do not prepare rote rites. They shape raw forces through shared arcane grammar known as \textbf{TAGS}. A spell is constructed at the table using a short phrase of TAGS. You only need the fiction, the TAG selection, and a casting roll.

\subsection*{Spell Structure}
\textbf{Intent} + \textbf{Target} + \textbf{Tags} = effect.

Example formula:
\begin{quote}
``I unleash Burning • Area • Force against the marauders.''
\end{quote}

The GM sets a Difficulty Value (DV) based on TAG complexity and danger.

\subsection*{Base Difficulty Value (DV)}
Start at DV 1 and add +1 for each TAG used.

\begin{center}
\textbf{DV = 1 + number of TAGS}
\end{center}

Adding powerful or perilous TAGS (Teleportation, Transformation, Dominate) adds +2 instead.

Mastery, focus, or appropriate tools may lower DV by 1.

\subsection*{Casting Roll}
Roll \textbf{Wits + Arcana} (or Ritual, Channeling, etc.).  
Success = spell goes off.  
Failure or 1 = Backlash (see below).

\subsection*{Backlash}
Whenever a Free Caster fails—or pushes power beyond safety—the magic pushes back. Choose one:
\begin{itemize}
\item Harm 2 (Arcane)
\item +2 Fatigue
\item Corruption +1
\item Catastrophic side effect (GM describes)
\end{itemize}

If the spell included a ``Dangerous'' TAG, Backlash triggers on \emph{mixed} results as well.

\newpage

\section*{TAG Library}
Pick 1–3 for minor spells.  
Pick 4–6 for heavy magic (very dangerous).  
More than 6 is suicidal.

\subsection*{Elemental TAGS}
\begin{itemize}[leftmargin=*]
\item \textbf{Burning}: flame, heat, combustion.
\item \textbf{Freezing}: ice, slowing, brittle shatter.
\item \textbf{Storm}: lightning, crackling arcs, thunder shock.
\item \textbf{Stone}: walls, spikes, tremors, armor.
\item \textbf{Wave}: crushing water, currents, pressure.
\item \textbf{Wind}: levitate, gusts, deflection.
\end{itemize}

\subsection*{Force TAGS}
\begin{itemize}[leftmargin=*]
\item \textbf{Force}: pure kinetic power, shields, blasts.
\item \textbf{Area}: cone, circle, corridor, zone.
\item \textbf{Strike}: single target precision.
\item \textbf{Wall}: barrier or blockade.
\item \textbf{Bind}: restrain, hold, suspend.
\item \textbf{Dispel}: suppress magic, unravel effects.
\end{itemize}

\subsection*{Mind \& Veil TAGS}
\begin{itemize}[leftmargin=*]
\item \textbf{Veil}: conceal, blur, illusion, silence.
\item \textbf{Scry}: reveal hidden, see distance, read traces.
\item \textbf{Memory}: erase, alter, restore.
\item \textbf{Command}: compel short action.
\item \textbf{Fear}: panic, flee, break morale.
\end{itemize}

\subsection*{Life \& Body TAGS}
\begin{itemize}[leftmargin=*]
\item \textbf{Mend}: close wounds, restore flesh, reduce Harm 1.
\item \textbf{Purify}: remove poison, corruption, disease.
\item \textbf{Strengthen}: enhance body, armor, senses.
\item \textbf{Waken}: counter sleep, paralysis, stun.
\item \textbf{Beast}: speak with or influence animals.
\end{itemize}

\subsection*{Space \& Motion TAGS (Always +2 DV Each)}
\begin{itemize}[leftmargin=*]
\item \textbf{Leap}: jump far, blink across short space.
\item \textbf{Fold}: short-range teleport, vanish–reappear.
\item \textbf{Gate}: long distance passage, open/close path.
\item \textbf{Gravity}: crush, lift, suspend, walk skyward.
\end{itemize}

\subsection*{Creation \& Transformation TAGS (Always +2 DV Each)}
\begin{itemize}[leftmargin=*]
\item \textbf{Create}: manifest matter briefly.
\item \textbf{Summon}: call a being or construct.
\item \textbf{Transmute}: turn one thing into another.
\item \textbf{Animate}: make objects act with intent.
\end{itemize}

\section*{Design Philosophy: Magic as Narrative Engine}\index{design intent!magic}

Magic in Fate's Edge serves specific design goals:

\begin{itemize}
    \item \textbf{Risk-Reward Balance}: Every magical act should feel consequential
    \item \textbf{Thematic Consistency}: Magic should reflect the world's metaphysics
    \item \textbf{Narrative Primacy}: Mechanics exist to serve interesting stories
    \item \textbf{Player Agency}: Magic should offer creative solutions, not bypass challenges
    \item \textbf{World Reactivity}: The setting should respond meaningfully to magical use
\end{itemize}

\subsection*{GM Guidance: Making Magic Feel Magical}

\begin{itemize}
    \item \textbf{Describe the Unseen}: When magic is cast, describe how the world reacts—air crackles, shadows deepen, spirits stir
    \item \textbf{Follow the Consequences}: Magical actions should have lasting effects on the narrative
    \item \textbf{Respect the Elements}: Backlash should feel philosophically appropriate
    \item \textbf{Highlight the Cost}: Make players feel the weight of their magical choices
    \item \textbf{Encourage Creativity}: Reward inventive uses of magic that enhance the story
\end{itemize}

\textbf{Remember}: In Fate's Edge, magic is never a shortcut. It's a pathway filled with wonders and dangers—a tool that changes both the world and the wielder. The dice are not your enemy; they're your collaborator in crafting a world where \textbf{true power always demands an equal price}.

