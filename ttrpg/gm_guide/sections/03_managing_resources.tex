\chapter{Managing Resources}\index{resources}

In \textbf{Fate's Edge}, resources are not mere numbers etched on parchment---they are \textbf{living, breathing elements of the fiction} that pulse with the same vitality as the characters who wield them. From the last precious sip of water in the sun-scorched sands of Akilan to the fragile loyalty of a Ykrul war-band chanting beneath blood-red banners, every resource tells a story, and every story demands its price. As the GM, you are the keeper of these vital threads, the weaver of scarcity and abundance. This chapter illuminates how to manage and narrate the systems that fuel both mortal ambitions and epic campaigns.

\section*{Supply Clock: The Pulse of Survival}\index{Supply Clock}

The \textbf{Supply Clock} beats as a shared heartbeat for the entire party, tracking their access to life's essentials---food that sustains, water that quenches, gear that endures, and the logistical support that separates civilization from chaos. This is no sterile inventory system; it is a \textbf{narrative lever} that tightens tension when the party finds themselves isolated in whispering forests, pressed by pursuing foes, or cut off from the comforting glow of hearth fires.

\subsection*{Supply Clock States}

\begin{fatebox}[Supply Clock Conditions]\index{Supply Clock!conditions}
\begin{tabularx}{\textwidth}{lX}
\toprule
\textbf{Segments Filled} & \textbf{Narrative Effects} \\
\midrule
0 (Full) & The party moves with confidence, well-equipped and prepared for the journey ahead \\
2 (Low) & Minor complications emerge: bland rations, damaged arrows, thinning waterskins, fraying ropes \\
3 (Dangerous) & Each character gains Fatigue as exhaustion and scarcity take their toll \\
4 (Empty) & Severe penalties manifest—starvation, dehydration, equipment failure become imminent threats \\
\bottomrule
\end{tabularx}
\end{fatebox}

\subsection*{Filling the Clock}

The Supply Clock fills when the world turns against the party's preparations:

\begin{itemize}
    \item Extended travel through hostile lands without proper provisioning
    \item The GM spends 2+ SB on logistics failures or environmental hardships
    \item The party chooses to travel light for speed or stealth advantage
    \item Failed Survival or Craft rolls related to hunting, foraging, or repair
\end{itemize}

\subsection*{Emptying the Clock}

Hope returns when the party finds respite:

\begin{itemize}
    \item Reaching civilization resets the clock to Full—the comfort of inns and markets
    \item Group Survival check (Wits + Survival, DV 2) under favorable conditions clears 1 segment
    \item Downtime spent in relative safety removes 1 segment through rest and recovery
    \item Successful provisioning actions—a good hunt, discovered cache—can reduce segments
\end{itemize}

\textbf{Example}: A week-long sea passage across the Dolmis Straits with uncertain winds that whisper of storms. A failed Navigation roll causes the GM to spend 2 SB---filling two segments as supplies spoil in the damp hold. The party is now at Low Supply, tasting the bitterness of hardtack and warm water. A second failed roll against contrary winds fills another segment---Dangerously Low. Fatigue sets in like a creeping frost. The sea, once a path to glory, now gnaws at their endurance with salt-crusted lips.

\section*{Fatigue: The Weight of the World}\index{Fatigue}

Fatigue represents the cumulative toll of journeying through a world that rarely offers comfort---the \textbf{exhaustion that seeps into bones, the hunger that hollows cheeks, the strain that clouds judgment}. Each level of Fatigue forces the character to \textbf{re-roll one success} on their next action, as weariness undermines their competence.

\subsection*{Fatigue Effects}

\begin{fatebox}[Fatigue Progression]\index{Fatigue!progression}
\begin{tabularx}{\textwidth}{lX}
\toprule
\textbf{Fatigue Level} & \textbf{Physical and Narrative Manifestations} \\
\midrule
1 Fatigue & Re-roll one success: Minor exhaustion, distractedness, slight impairment \\
2 Fatigue & Re-roll one success cumulative: Noticeable weariness, slower reactions, aching muscles \\
3 Fatigue & Re-roll two successes: Significant exhaustion, labored movement, mental fog settling in \\
4 Fatigue & Collapse, KO, or spiritual break: Character falls unconscious or becomes incapacitated \\
\bottomrule
\end{tabularx}
\end{fatebox}

\subsection*{Clearing Fatigue}

Recovery requires genuine respite:

\begin{itemize}
    \item A night's rest with adequate Supply removes 1 Fatigue—the healing power of true rest
    \item Fatigue cannot be removed while the party is Dangerously Low or Out of Supply—exhaustion compounds scarcity
    \item Medical attention (Presence + Heal, DV 2) during downtime can remove 1 Fatigue through proper care
\end{itemize}

\textbf{Narrative Note}: Fatigue is not just physical---it can reflect the mental strain of constant vigilance, the grief of lost companions, or the spiritual exhaustion from battling unnatural forces. A failed ritual might leave a caster \textbf{Fatigue 2} from the metaphysical backlash alone, their soul bruised by unseen energies.

\section*{Followers and Assets: Power Beyond the Self}\index{Followers}\index{Assets}

In Fate's Edge, players can invest XP into \textbf{Followers} and \textbf{Assets}---tools that extend their reach beyond personal skill. These are not mere stat blocks---they are \textbf{story agents} with their own motivations, risks, and narrative arcs that intertwine with the player's destiny.

\subsection*{Followers: On-Scene Allies}\index{Followers!on-scene}

Followers are \textbf{on-screen allies} who stand beside you in danger—loyal swords, cunning scouts, faithful apprentices. They are bought with XP and tracked by a \textbf{Cap}\index{Cap} (their maximum assist bonus), representing their competence and dedication.

Cost: A follower with Specialty Cap C costs C² XP. Downtime = 1--3 days to recruit, train, and build trust.

\subsection*{Assisting in Scenes}

Followers assist by adding their expertise to your endeavors:

\begin{itemize}
    \item Assist dice come from the helper's capabilities, not the leader's pool
    \item Total Assist on any roll (from any sources) remains hard-capped at +3, representing practical limits of coordination
    \item Exception: The "Exceptional Coordination"\index{Exceptional Coordination} Talent allows one follower to provide +4 assist dice through preternatural synergy
    \item When applicable, the follower adds help dice equal to \textbf{min(C, the helper's relevant Skill)}, capped at +3 dice
    \item Slot Limit: Only one follower may assist a given action—too many cooks spoil the broth
\end{itemize}

\subsection*{Follower Initiative Actions}\index{Followers!initiative actions}

Once per scene (across the party), one on-screen follower may take a small independent action that demonstrates their initiative:

\begin{itemize}
    \item Scout \& Signal --- Change an ally's next action position to Controlled through timely warning
    \item Distract \& Draw --- Reduce a kinetic rail (Hunt/Escape/Hazard) by –1 tick through clever diversion
    \item Fetch \& Carry --- Move a small object through danger when moments count
\end{itemize}

\textbf{Cost:} Mark Exposure +1 or Harm 1 on that follower—bravery risks consequences.

\subsection*{Follower Upkeep}\index{Followers!upkeep}

Relationships require maintenance:

\begin{itemize}
    \item Each Downtime, pay XP equal to Cap or spend a Scene tending the relationship—neglect erodes loyalty
    \item Risk: If the GM spends 2+ Story Beats\index{Story Beat (SB)} on an action you take with assistance, they may mark Exposure or Harm on the follower instead of applying other consequences, if fictionally appropriate
    \item Off-Screen Capability: Once per downtime, a follower with Cap 3 or higher can solve one significant problem but generates 1 SB for party. The GM must describe how the follower's action creates story consequences for the SB generated
\end{itemize}

\subsection*{Follower Condition}\index{Followers!condition}

Followers track their own trials through \textbf{Exposure}\index{Exposure} and \textbf{Harm}\index{Harm}:

\begin{description}
    \item[Exposure] --- Heat, attention, stress, or narrative pressure placed upon the follower—the cost of being noticed
    \item[Harm] --- Injury, trauma, fatigue, or direct damage to the follower—the price of involvement
\end{description}

\textbf{States:}
\begin{itemize}
    \item \textbf{Maintained} --- Reliable and ready, their loyalty reinforced by attention and care
    \item \textbf{Neglected} --- Needs downtime or care. Impose a -1 die penalty to their assistance—distance grows
    \item \textbf{Compromised} --- Captured, defected, lost, or incapacitated. Cannot assist until recovered—the bond frays or breaks
\end{itemize}

\subsection*{Assets: Off-Scene Influence}\index{Assets!off-scene}

Assets are \textbf{off-screen resources}---titles that open doors, safehouses that offer sanctuary, spy rings that gather secrets, charters that grant authority. They do not act in scenes directly, but they change the fiction and provide leverage when you return to the table, like chess pieces moved between matches.

\begin{fatebox}[Asset Tiers and Descriptions]\index{Assets!tiers}
\begin{tabularx}{\textwidth}{lX}
\toprule
\textbf{Asset Tier} & \textbf{Narrative Examples and Scope} \\
\midrule
Minor (4 XP, 1 day) & Safehouse in the docks, small shop in the market, petty title granting minor respect, local contact network with limited reach \\
Standard (8 XP, 1 week) & Noble title with actual influence, guild section with specialized resources, spy ring covering a district, workshop producing quality goods \\
Major (12 XP, 1 month) & City license for major operations, regional network spanning cities, fortress lease with strategic value, trading company with international reach \\
\bottomrule
\end{tabularx}
\end{fatebox}

\subsection*{Using Assets}\index{Assets!usage}

Assets provide subtle but powerful benefits:

\begin{itemize}
    \item \textbf{Off-Screen Effect:} Use each Asset's listed Off-Screen effect once per session for free—the quiet work between adventures
    \item \textbf{On-Screen Activation:} To reshape the current scene dramatically, spend 1 Boon\index{Boons}—calling in favors at crucial moments
    \item \textbf{Downtime Activation:} A player may activate an off-screen asset at the very start of a campaign or during Downtime. It costs 2 XP or 1 Boon to activate—investment paying dividends
    \item The Asset must have scope and reach for the intended effect—a local shop can't influence royal politics
\end{itemize}

\subsection*{Asset Condition}\index{Assets!condition}

All Assets have a \textbf{Condition Track}\index{Condition Track} reflecting their maintenance and standing:

\begin{description}
    \item[Maintained] --- Full capability. Functions normally—the asset thrives under care
    \item[Neglected] --- Impaired. Impose a -1 die penalty when used; requires attention—dust gathers, contacts cool
    \item[Compromised] --- Unavailable. Cannot be used until repaired or recovered—sealed by authorities, burned by rivals, lost to misfortune
\end{description}

\section*{Boons: The Currency of Resilience}\index{Boons}

Boons are \textbf{narrative tokens} earned by embracing risk and moving the story forward against the tide of misfortune. They reward \textbf{failure with texture and opportunity}, not failure with emptiness—the silver lining in clouds of defeat.

\subsection*{Earning Boons}

Boons flow to those who engage deeply with the world:

\begin{itemize}
    \item On a failed roll with meaningful Complications (see Fail Forward, Chapter 2)
    \item Through clever or risky roleplay that drives the story into new territory
    \item Via bond-driven actions with intricate descriptions that deepen relationships
    \item Through GM discretion for exceptional collaborative play that enhances everyone's experience
\end{itemize}

\subsection*{Boon Economy}\index{Boon economy}

Boons follow natural rhythms:

\begin{itemize}
    \item \textbf{Holding cap:} You can hold at most 5 Boons—fortune favors preparation but not hoarding
    \item \textbf{Carryover Limit:} At the end of each scene, reduce held Boons to a maximum of 2. Excess Boons are lost—opportunities fade if not seized
    \item \textbf{Conversion:} Once per session, in downtime, you may convert 2 Boons → 1 XP (max 2 XP via conversion per session)—lessons hard-learned become permanent growth
\end{itemize}

\subsection*{Using Boons}

Boons empower moments of exceptional effort:

\begin{itemize}
    \item \textbf{Re-roll one die} after seeing the pool—a second chance when it matters most
    \item \textbf{Activate an Asset} for on-screen effect—calling upon resources at critical junctures
    \item \textbf{Power Rites} that require Boon expenditure—channeling energy into mystical workings
    \item \textbf{Improve Position} by one step (1 Boon)—turning desperation into opportunity
    \item \textbf{Clear 1 tick} from a spirit's Leash (Pact-Whisperer, 1 Boon per round)—reinforcing supernatural bonds
\end{itemize}

\subsection*{Anti-Fishing Measures}\index{Anti-Fishing}

To maintain healthy game flow and prevent exploitation:

\begin{itemize}
    \item \textbf{Once/Scene Cap:} At most \textbf{2 Boons from failures} per character per scene—diminishing returns on repeated failure
    \item \textbf{Repetition Rule:} Same approach + same stakes in the same scene cannot award another Boon—innovation required for continued reward
    \item \textbf{Position Gate:} Controlled tests with trivial fallout do not award Boons—no reward without risk
\end{itemize}

\textbf{Design Note}: Boons are not a "get out of jail free" card. They are earned by \textbf{leaning into the fiction with courage and creativity}, not by fishing for failure. Reward players who take narrative risks, not those who roll badly on purpose.

\section*{XP Awards: Growth Through Choice}\index{XP awards}\index{Experience Points!awards}

XP in Fate's Edge is \textbf{meaningful currency} that represents genuine growth through experience. It is not handed out for mere attendance---it is earned through \textbf{active engagement, meaningful risk, and tangible narrative impact}.

\subsection*{Session Awards}

\begin{fatebox}[XP Award Guidelines]\index{XP!award guidelines}
\begin{tabularx}{\textwidth}{lX}
\toprule
\textbf{Award Type} & \textbf{Description and Examples} \\
\midrule
Table Attendance & +2 XP for participating in the shared story \\
Major Objective & +2--4 XP for achieving significant story goals \\
Discovery/Lore & +1--2 XP for uncovering important information or secrets \\
Hard Choice & +1--2 XP for making difficult decisions with consequences \\
Complication Spotlight & +1--3 XP for engaging meaningfully with complications \\
Bond/Flag Driven Play & +1--2 XP for roleplaying that emphasizes relationships \\
GM Curveball & +0--3 XP for adapting well to unexpected developments \\
\bottomrule
\end{tabularx}
\end{fatebox}

\subsection*{Milestones}\index{milestones}

Major achievements bring significant growth:

\begin{itemize}
    \item +8--12 XP to all players at the conclusion of a major story arc—the reward for epic endeavors
    \item +2 XP bonus to one player for a signature moment of the arc—recognition for exceptional contribution
\end{itemize}

\section*{Campaign Resources: Mandate and Crisis}\index{Campaign resources}

At the campaign level, two great clocks track the party's rising influence and the world's gathering resistance—the tide of fortune that lifts or drowns ambitions.

\subsection*{Mandate Clock (0--6)}\index{Mandate}\index{Campaign Clocks!Mandate}

Tracks the party's public legitimacy and the world's willingness to support their cause:
\begin{itemize}
    \item High Mandate: Allies seek them out, resources flow freely, doors open without force
    \item Low Mandate: Suspicion dogs their steps, bureaucratic obstacles multiply, support withers
\end{itemize}

\subsection*{Crisis Clock (0--6)}\index{Crisis}\index{Campaign Clocks!Crisis}

Tracks the opposition's growing strength and the world's mounting troubles:
\begin{itemize}
    \item Rising Crisis: Complications escalate relentlessly, enemies grow bolder, disasters loom
    \item Managed Crisis: Breathing room emerges, opportunities to strike back appear, pressure relents
\end{itemize}

\section*{Combat Resource Management}\index{combat resources}

In combat, resource management takes on desperate urgency. The same systems that govern exploration and downtime now operate under the sword's edge, with consequences that echo immediately through the clash of steel.

\subsection*{Supply in Combat}\index{Supply Clock!combat}

Extended combat encounters drain resources with alarming speed:

\begin{itemize}
    \item \textbf{Intense Combat}: GM may spend 1 SB to fill 1 Supply segment as arrows break and waterskins puncture
    \item \textbf{Prolonged Engagement}: Each hour of sustained combat adds 1 Supply segment—the slow drain of endurance
    \item \textbf{Ammunition Depletion}: Ranged weapons may run low, requiring scavenging actions amidst danger
\end{itemize}

\subsection*{Fatigue in Combat}\index{Fatigue!combat}

Combat fatigue compounds existing strain with brutal efficiency:

\begin{itemize}
    \item \textbf{Each Round}: Characters with existing Fatigue re-roll additional successes equal to their Fatigue level—exhaustion undermines skill
    \item \textbf{Critical Exhaustion}: Reaching 4 Fatigue during combat causes immediate collapse—the body's final surrender
    \item \textbf{Recovery}: Cannot clear Fatigue during active combat—no rest while blades flash
\end{itemize}

\subsection*{Follower Combat Integration}\index{Followers!combat}

Followers in combat face unique risks and opportunities that test their loyalty and competence:

\begin{itemize}
    \item \textbf{Combat Assistance}: Followers can assist in combat rolls using their Cap—standing together against danger
    \item \textbf{Follower Risk}: 2+ SB spent in combat can endanger assisting followers (mark Exposure or Harm)—bravery has its price
    \item \textbf{Initiative Actions}\index{Initiative Actions}: Followers can take combat-relevant independent actions (cost: Exposure +1 or Harm 1)—moments of individual valor
    \item \textbf{Combat Exposure}: Each time a follower acts on-screen in high-risk combat, mark Exposure +1 after the second such beat this scene—the attention they attract
\end{itemize}

\subsection*{Tactical Clocks as Resources}\index{Tactical Clocks}\index{Tactical Clocks!resources}

Tactical clocks represent persistent combat conditions that drain party resources like sieges drain garrisons:

\begin{fatebox}[Tactical Clock Effects]\index{Tactical Clocks!effects}
\begin{tabularx}{\textwidth}{lX}
\toprule
\textbf{Clock Type} & \textbf{Resource Drain and Narrative Impact} \\
\midrule
Mob Overwhelm [6] & Enemy numbers become advantage—forces Supply depletion and accelerates Fatigue through relentless pressure \\
Fatigue Spiral [4] & Exhaustion affects performance—accelerates existing Fatigue, making each action more difficult than the last \\
Morale Collapse [6] & Fear undermines effectiveness—generates SB and reduces effectiveness as confidence shatters \\
Environmental Collapse [8] & Terrain/fire/building failure—creates new Supply and safety concerns as the battlefield turns against you \\
\bottomrule
\end{tabularx}
\end{fatebox}

\section*{Narrative First: The Fiction Is the Ledger}\index{narrative first}

In Fate's Edge, arrows, rations, and waterskins are tracked only in the fiction that surrounds them. Mechanics engage only when those resources become scarce enough to matter. The focus remains always on \textbf{narrative tension}—the gnawing hunger, the fading light, the last arrow—not sterile bookkeeping.

Let the world breathe with its own needs and abundances. Let the fiction lead through scarcity and surplus. And when the dice say the world pushes back against mortal plans—\textbf{listen to what they tell you about the price of ambition.}

\end{chapter}
