\chapter{Tier IV and V Play}\index{Tier IV}\index{Tier V}\index{epic tier}

As characters reach Tier IV and V levels, the scope of play expands dramatically. What once were local concerns become matters of regional, national, or even world-shaking importance. This chapter provides guidance for managing the unique challenges and opportunities that come with high-tier play.

\section{The Nature of High-Tier Play}\index{high-tier play}

At Tier IV and V, characters are no longer operating on the margins---they are movers and shakers. Their actions have visible, lasting impacts on the world. This shift requires the Game Master to think bigger, plan longer, and embrace the cascading consequences of player choices.

\subsection*{Key Characteristics}\index{high-tier play!characteristics}

\begin{itemize}
    \item \textbf{Wider Scope}: Actions affect cities, regions, or nations
    \item \textbf{Longer Timelines}: Consequences unfold over weeks, months, or years
    \item \textbf{Greater Stakes}: Failure means more than personal loss
    \item \textbf{Complex Alliances}: Multiple factions with competing interests
    \item \textbf{Legacy Impact}: Choices create lasting changes to the world
\end{itemize}

\section{Deck-Based Campaign Management}\index{deck-based management}

High-tier play benefits from structured campaign management using the Game Deck and other tools to track large-scale developments.

\subsection*{Campaign Clock Expansion}\index{Campaign Clocks!expansion}

Expand beyond core campaign clocks to include:

\begin{itemize}
    \item \textbf{Faction Influence} (6): Track major faction relationships
    \item \textbf{Public Opinion} (8): Regional perception of the group
    \item \textbf{Resource Network} (6): Economic and logistical reach
    \item \textbf{Legacy Projects} (10): Long-term initiatives with lasting impact
\end{itemize}

\subsection*{Using Cards for World Events}\index{Game Deck!world events}

Draw cards periodically to introduce world events:

\begin{itemize}
    \item \textbf{Suit 1 (Swords)}\index{Swords (suit)}: Geographic/political changes
    \item \textbf{Suit 2 (Crowns)}\index{Crowns (suit)}: Social/cultural shifts
    \item \textbf{Suit 3 (Anchors)}\index{Anchors (suit)}: Economic/resource developments
    \item \textbf{Suit 4 (Glyphs)}\index{Glyphs (suit)}: Opportunities/leverage points
\end{itemize}

\section{Managing Multiple Holdings and Allies}\index{Holdings!management}\index{Allies!management}

Tier IV+ characters often command extensive networks. Use these techniques to keep management manageable:

\subsection*{Holding Clustering}\index{Holdings!clustering}

Group related holdings into portfolios:
\begin{itemize}
    \item \textbf{Economic}: Trade routes, businesses, investments
    \item \textbf{Political}: Titles, contacts, influence networks
    \item \textbf{Military}: Retainers, fortifications, strategic positions
    \item \textbf{Intelligence}: Informants, research facilities
\end{itemize}

\subsection*{Ally Hierarchies}\index{Allies!hierarchies}

Create chains of command:
\begin{itemize}
    \item \textbf{Lieutenants} (Expertise 4-5): Direct reports who manage others
    \item \textbf{Commanders} (Expertise 3): Mid-level managers of specific portfolios
    \item \textbf{Agents} (Expertise 2): Field operatives and specialists
\end{itemize}

\section{High-Stakes Consequences}\index{high-stakes consequences}

Setback Points at high tiers should reflect the expanded scope of play:

\subsection*{High-Tier Setback Sinks}\index{Setback Points!high-tier sinks}

\begin{itemize}
    \item \textbf{3-4 SP}: Regional setback, major holding compromised
    \item \textbf{5-6 SP}: Faction relationship damaged, public scandal
    \item \textbf{7-8 SP}: Strategic position lost, major ally turned
    \item \textbf{9+ SP}: Paradigm shift, fundamental world change
\end{itemize}

\begin{tcolorbox}[title=\textbf{Tier VI — Mythic Play Guidelines},
colback=white!97!gray,colframe=black!60!gray,boxrule=0.4pt]
\textbf{Scope.} At Tier VI, actions reshape continents and concepts.
Characters gain auto-successes equal to Tier (6) but remain bound by
\textit{Obligation, Corruption, and Harm}—now existential.

\textbf{DV Scaling:} $\mathrm{DV}=8+n_{\text{active clocks}}+\text{Opposition mod}$
\textbf{Clocks:} 8–10 segments model nations, gods, or cosmic forces.
\textbf{Resource Cap:} Obligation 12, Harm 3 (severe), Boons replaced by Mythic Tags.
\textbf{Mythic Tags:} [OMEN], [REALITY], [SOVEREIGN], each grants +1 Effect and adds +1 Obligation when invoked.

\textbf{Endgame Principle:}
Power demands metamorphosis—advancement changes what a character \emph{is}, not what they \emph{can do}.
\end{tcolorbox}

\subsection*{Deck-Driven Consequences}\index{Game Deck!high-tier}

Use the Game Deck for major setbacks:
\begin{itemize}
    \item \textbf{Court Cards}: Major faction leaders or institutions affected
    \item \textbf{Aces}: Foundational assumptions challenged
    \item \textbf{Multiple Cards}: Cascade effects across multiple domains
\end{itemize}

\section{Running Epic Campaigns}\index{epic campaigns}

High-tier play often involves extended campaigns with multiple acts and lasting consequences.

\subsection*{Act Structure}\index{campaign structure}

\begin{itemize}
    \item \textbf{Act I - Establishment} (Sessions 1-3): Set the stage, establish stakes
    \item \textbf{Act II - Escalation} (Sessions 4-8): Complications multiply, alliances shift
    \item \textbf{Act III - Resolution} (Sessions 9-12): Climactic confrontations, lasting changes
    \item \textbf{Epilogue} (Session 13+): Legacy assessment, new beginnings
\end{itemize}

\subsection*{Campaign Seeds}\index{campaign seeds}

Use the full 4-card draw for major campaign hooks:
\begin{itemize}
    \item \textbf{Suit 1 (Swords)}: Primary location/region of conflict
    \item \textbf{Suit 2 (Crowns)}: Key faction/leader driving events
    \item \textbf{Suit 3 (Anchors)}: Major complication/threat
    \item \textbf{Suit 4 (Glyphs)}: Opportunity/resource to exploit
\end{itemize}

\section{Mass Combat and Warfare}\index{mass combat}\index{warfare}

Tier IV+ characters often find themselves commanding armies or influencing wars.

\subsection*{Army Scale Combat}\index{combat!army scale}

Simplify large-scale battles:
\begin{itemize}
    \item Treat armies as powerful allies with specialized skills
    \item Use clocks to track morale, supply, and strategic position
    \item Focus rolls on leadership and tactical decisions, not individual combat
\end{itemize}

\subsection*{Advanced Subsystem: Mass Combat}\index{advanced subsystem!mass combat}

For a more detailed warfare system, use the following framework:

\begin{description}
    \item[Army as an Entity] Create a character sheet for the army with Approaches (e.g., Aggressive, Disciplined, Cunning) and a "Morale \& Supply" clock.
    \item[The Battle Clock] Each significant battle is a 4-6 segment clock. Characters can contribute by using their skills to create advantages or by leading from the front.
    \item[Strategic Rolls] Commanders make skill checks against a target number. Success fills segments on the Battle Clock; failure fills segments on the army's "Morale \& Supply" clock or introduces a complication via the Game Deck.
\end{description}

\subsection*{War Campaigns}\index{war campaigns}

Structure extended conflicts:
\begin{itemize}
    \item \textbf{Strategic Phase}: Resource management, alliance building
    \item \textbf{Tactical Phase}: Key battles, covert operations
    \item \textbf{Political Phase}: Negotiations, aftermath management
\end{itemize}

\section{Mythic Challenges}\index{mythic challenges}

At Tier V, characters approach legendary status. Create challenges that match their stature:

\subsection*{Existential Threats}\index{existential threats}

\begin{itemize}
    \item Cosmic entities beyond normal understanding
    \item Reality-altering phenomena
    \item Threats to entire civilizations or ways of life
\end{itemize}

\subsection*{Legacy Missions}\index{legacy missions}

Missions that will be remembered for generations:
\begin{itemize}
    \item Founding or destroying nations
    \item Ending or beginning ages
    \item Reshaping fundamental aspects of the world
\end{itemize}

\subsection*{Advanced Subsystem: Legacy Projects}\index{advanced subsystem!legacy projects}

A Legacy Project is a long-term goal that extends beyond a single adventure. To run one:

\begin{description}
    \item[Define the Project] The players state their goal (e.g., "Build a Mage University," "Forge an Alliance of Kingdoms").
    \item[Create the Project Clock] This is a large clock, typically 8-12 segments.
    \item[Determine Prerequisites] The project may require specific resources, allies, or completed quests to even begin.
    \item[Milestone Advances] Instead of filling the clock with single rolls, each major story arc or significant achievement fills 2-3 segments. Setbacks from the Game Deck can remove segments.
\end{description}

\section{Managing Player Agency}\index{player agency}

With great power comes the need for great Game Master flexibility:

\subsection*{Player-Driven Narratives}\index{player-driven narratives}

\begin{itemize}
    \item Let player choices genuinely reshape the world
    \item Honor long-term commitments and consequences
    \item Provide meaningful opposition that matches their scale
\end{itemize}

\subsection*{World Reactivity}\index{world reactivity}

\begin{itemize}
    \item Factions respond realistically to player actions
    \item Economic and political systems show cause-and-effect
    \item Non-player characters remember and react to past interactions
\end{itemize}

\section{Rivals and Counterpoints}\index{rivals}

High-tier characters attract attention---both positive and negative:

\subsection*{Creating Worthy Opponents}\index{rivals!creating}

\begin{itemize}
    \item Mirror player capabilities and resources
    \item Give them their own networks and influence
    \item Create personal connections and history with the group
\end{itemize}

\subsection*{Dynamic Rivalry}\index{rivals!dynamic}

\begin{itemize}
    \item Rivals evolve based on player actions
    \item Competition across multiple domains (political, economic, social)
    \item Occasional cooperation against greater threats
\end{itemize}

\section{Campaign Legacy}\index{campaign legacy}

Help players see the lasting impact of their choices:

\subsection*{Legacy Tracking}\index{legacy tracking}

\begin{itemize}
    \item Document major world changes initiated by the group
    \item Track faction relationships and their evolution
    \item Record personal legacies and how they're remembered
\end{itemize}

\subsection*{Epilogue Framework}\index{epilogue}

Use cards to determine long-term outcomes:
\begin{itemize}
    \item Draw 2-3 cards from each suit
    \item Interpret results as 5-10 year outcomes
    \item Let players narrate their characters' final fates
\end{itemize}

\section{Game Master Preparation Tips}\index{Game Master preparation}

\subsection*{Think in Campaign Arcs}\index{campaign arcs}

\begin{itemize}
    \item Plan 3-5 major story arcs per tier
    \item Each arc should have lasting world impact
    \item Connect arcs through recurring themes or non-player characters
\end{itemize}

\subsection*{Prepare Flexible Frameworks}\index{flexible frameworks}

\begin{itemize}
    \item Create faction relationship matrices
    \item Develop economic and political systems that respond to actions
    \item Build modular locations that can evolve
\end{itemize}

\subsection*{Embrace Player Creativity}\index{player creativity}

\begin{itemize}
    \item Let player holdings genuinely solve problems
    \item Reward creative use of influence and resources
    \item Say "yes" to ambitious player plans, then make them interesting
\end{itemize}

\section{Sample High-Tier Scenario}\index{sample scenario}

\textbf{The Shattered Crown Crisis}

A Tier IV campaign seed:
\begin{itemize}
    \item \textbf{Suit 1 (The Royal Crypts)}: Ancient tombs beneath the capital
    \item \textbf{Suit 2 (The Usurper)}: A noble house claiming the vacant throne
    \item \textbf{Suit 3 (Fractured Loyalties)}: Regional lords choosing sides
    \item \textbf{Suit 4 (The Crown's Secret)}: Hidden royal treasures and alliances
\end{itemize}

Clocks: Succession Crisis (8), Noble Conspiracy (6), Public Unrest (6)

This scenario can evolve based on player choices---supporting the usurper, finding a true heir, or establishing a new form of government.

\section{Tools of the Game Master}\index{tools}

This section summarizes the key procedures for running epic-tier play.

\subsection*{Core Procedures}

\begin{description}
    \item[Starting an Epic Arc] Draw 4 cards (one per suit) to generate the core elements of a major story: Location, Faction, Threat, and Opportunity.
    \item[Managing the World] At the start of each session or after a major event, draw a card from the Game Deck to see how the world changes. Use the suit to determine the domain (Political, Social, Economic, Opportunistic).
    \item[Handling Major Setbacks] When players accumulate 3+ Setback Points, consider spending them for a high-tier consequence. Use the Game Deck to determine the nature of the setback, with Court Cards and Aces indicating severe, world-altering events.
    \item[Tracking Progress] Maintain the expanded set of Campaign Clocks (Faction Influence, Public Opinion, etc.) to mechanically represent the group's impact on the world.
\end{description}

\subsection*{Running Key Scenes}

\begin{description}
    \item[Mass Combat] Use the Battle Clock subsystem. Focus on the characters' leadership actions and their consequences for the army's morale.
    \item[Legacy Projects] Use the Legacy Project Clock. Advance it through milestone achievements, not individual rolls.
    \item[Epilogue] After the final session, use a multi-card draw from the Game Deck to inspire the narration of the world's and characters' long-term futures.
\end{description}

\section{Boss Generator}

\subsection{Core Concept}

A deck-based tool to quickly create compelling, thematically rich boss encounters with built-in mechanical scaling and narrative hooks. The GM uses the draws as a foundation and then applies their knowledge of the setting, party, and desired challenge level to finalize the boss.

\subsection{Deck Structure}

\subsubsection{Standard Deck (52 Cards)}

\textbf{Suits Define Core Aspects:}
\begin{itemize}
    \item \textbf{♠ (Spades - Structure):} The boss's physical form, defenses, and core mechanics.
    \item \textbf{♥ (Hearts - Drive):} The boss's motivations, goals, and psychological core.
    \item \textbf{♣ (Clubs - Complication):} The boss's signature hazards, environmental effects, and unique challenges it introduces.
    \item \textbf{♦ (Diamonds - Reward):} The boss's unique loot, knowledge, or narrative currency the players gain by defeating it (can also be a ``Twist'' reward that changes the story).
\end{itemize}

\textbf{Ranks Define Scale/Intensity (Modified for Bosses):}
\begin{itemize}
    \item \textbf{2-5 (Minor):} A challenging elite enemy or minor boss. 4-segment ``Phase'' clock.
    \item \textbf{6-10 (Standard):} A significant boss encounter. 6-segment ``Phase'' clock.
    \item \textbf{J, Q, K (Major):} A major set-piece boss. 8-segment ``Phase'' clock.
    \item \textbf{A (Pivotal):} An epic, campaign-defining boss. 10-segment ``Phase'' clock.
\end{itemize}

\textbf{Color Influence:}
\begin{itemize}
    \item \textbf{Black Suits (♠, ♣):} Physical, tangible threats and defenses.
    \item \textbf{Red Suits (♥, ♢):} Psychological, social, or intangible aspects.
\end{itemize}

\subsection{Deck Categories \& Examples}

\subsubsection{♠ Spades - Structure (Form, Defenses, Core Mechanics)}

\begin{itemize}
    \item \textbf{2-5:} Augmented Body (Cybernetics, armor plating), Swarm Core (Controls lesser units), Fragile Shell (Weak physical form, relies on other defenses).
    \item \textbf{6-10:} Massive Construct (High Body, area attacks), Adaptive Core (Changes tactics/defenses), Phased Form (Intangible/invulnerable at certain times).
    \item \textbf{J/Q/K:} Living Weapon (Its body IS its weapon), Reality Anchor (Negates certain magic/effects in its zone), Hive Mind (Shares health/pool with minions).
    \item \textbf{A:} Titan (Massive scale, environmental effects just by existing), Conceptual Entity (Exists partially outside normal reality), World-Soul (Bound to the location itself).
\end{itemize}

\subsubsection{♥ Hearts - Drive (Motivation, Goals, Psychology)}

\begin{itemize}
    \item \textbf{2-5:} Greed (Wants treasure/resources), Survival (Will do anything to stay alive), Guarding (Protecting something/someone).
    \item \textbf{6-10:} Domination (Seeks control/power over others), Corruption (Spreads decay/evil), Restoration (Trying to fix/revive something, even destructively).
    \item \textbf{J/Q/K:} Vengeance (Driven by a specific past wrong), Ascension (Seeks to transcend current form/state), Preservation (Wants to prevent change or end the world).
    \item \textbf{A:} Cosmic Hunger (Consumes to fuel its existence), Paradox Incarnate (Embodies a fundamental contradiction), The Inevitable (Its goal is preordained, unstoppable).
\end{itemize}

\subsubsection{♣ Clubs - Complication (Hazards, Environment, Unique Challenges)}

\begin{itemize}
    \item \textbf{2-5:} Overheating Systems (Condition clock that worsens attacks if filled), Unstable Terrain (Difficult/unsafe ground), Reactive Defenses (Traps triggered by player actions).
    \item \textbf{6-10:} Environmental Collapse (Clock ticking towards a disaster), Phased Attacks (Must be attacked in a specific sequence), Debilitating Field (Ongoing condition for players).
    \item \textbf{J/Q/K:} Minion Control (Commands powerful followers), Reality Distortion (Rules of physics/magic are bent), Soul Drain (Attacks also sap resolve/resources).
    \item \textbf{A:} Apotheosis Trigger (Defeating it the ``wrong'' way makes it stronger), Causality Loop (Actions have delayed, paradoxical effects), Domain Authority (The battlefield itself is hostile).
\end{itemize}

\subsubsection{♢ Diamonds - Reward/Twist (Loot, Knowledge, Narrative Shift)}

\begin{itemize}
    \item \textbf{2-5:} Valuables (Riches, rare materials), Useful Tool (Minor artifact, key, helpful item), Tactical Knowledge (Insight into a related threat).
    \item \textbf{6-10:} Powerful Artifact (Significant magic item), Forbidden Lore (Dangerous but valuable information), Faction Favor (Gain status with a group).
    \item \textbf{J/Q/K:} Soul Bargain (Power at a cost), Command Obedience (Gain control over something related), Rewriting Fate (Undo a past failure or gain a major advantage).
    \item \textbf{A:} Worldly Truth (Reveals a major plot point), Shifting Balance (Fundamentally alters the power structure), Divine Spark (A step towards mythic status for a PC).
\end{itemize}

\subsection{GM Usage Procedure}

\begin{enumerate}
    \item \textbf{Define Scope:} Decide the general tier/impact of the boss (Minor encounter to Pivotal climax).
    \item \textbf{Draw Cards:} Draw one card from each suit. The highest rank determines the base \textbf{Phase Clock Size} (4/6/8/10 segments).
    \item \textbf{Interpret Core:} Read the four cards as a cohesive whole. What kind of boss does this combination suggest?
    \begin{itemize}
        \item \emph{Example: ♠6 Massive Construct, ♥J Vengeance, ♣Q Reality Distortion, ♢A Worldly Truth.}
        \item \emph{Interpretation: A vengeful, colossal war construct whose very presence warps reality. Defeating it reveals a crucial truth.}
    \end{itemize}
    \item \textbf{Theme \& Flavour:} Use the core concept to tie the boss to the setting, the party's story, or the current location. What \emph{is} this Massive Construct? Who is it seeking Vengeance against?
    \item \textbf{Set Base Stats:} Use the party's Tier and the boss's scale (rank) to determine a base dice pool.
    \begin{itemize}
        \item \emph{Guideline:} Tier I (Rookie/Seasoned): Boss Base ~6-7 dice. Tier II (Veteran): ~7-8 dice. Tier III (Paragon): ~8-9 dice. Tier IV/V (Mythic): 9+ dice.
    \end{itemize}
    \item \textbf{Mechanize the Cards:}
    \begin{itemize}
        \item \textbf{♠ Structure:} Defines base form, resistances, and primary attack modes. \emph{Massive Construct} = High Body, area attacks, maybe [COMPROMISED] resistance.
        \item \textbf{♥ Drive:} Influences behavior and special actions. \emph{Vengeance} = Targeted attacks on specific PC/ally, bonus against those who ``wronged'' it.
        \item \textbf{♣ Complication:} Create a named clock or ongoing effect. \emph{Reality Distortion} = ``Warp Field'' clock [6]. When filled, the laws of physics in the zone shift dramatically for a round.
        \item \textbf{♢ Reward/Twist:} Plan the narrative outcome. \emph{Worldly Truth} = Defeating it reveals the location of a hidden vault or the true identity of a patron.
    \end{itemize}
    \item \textbf{Define Phases:} Based on the clock size, break the fight into 2-3 phases. As the main ``Phase Clock'' fills, the boss gains +1 die per phase (or other escalating effects) and may trigger its Complication clock or introduce new elements.
    \item \textbf{Tie to Story Beats:} Remember, the boss generates Story Beats (SB) on 1s. Use the ♥ Drive and ♣ Complication to guide how SB are spent. \emph{Vengeance} SB might target the PC it's after. \emph{Reality Distortion} SB might trigger environmental weirdness.
    \item \textbf{Run the Encounter:} Use the established framework, but narrate freely. Let player actions and SB spends influence the specific details within the established parameters.
\end{enumerate}

\subsection{GM Guidance \& Theming Advice}

\begin{itemize}
    \item \textbf{Start Simple:} For a first boss, use fewer complications or a straightforward phase structure. Add complexity as you get comfortable.
    \item \textbf{Tie to the Party:} Use the ♥ Drive to connect the boss to the PCs. A boss driven by \emph{Vengeance} is more impactful if it's specifically targeting one of the players or their homeland.
    \item \textbf{Make the Environment a Character:} Use the ♣ Complication to make the fight location dynamic. The boss doesn't just exist \emph{in} the environment; it \emph{is} part of the environment or actively manipulates it.
    \item \textbf{Reward Narrative Investment:} The ♢ Reward/Twist is crucial. It shouldn't just be loot; it should advance the story or give players a meaningful choice.
    \item \textbf{Use Clocks Liberally:} The boss's Phase Clock, its Complication Clock, and environmental clocks (like Self-Destruct) are fantastic tools for pacing and adding tension. Name them evocatively.
    \item \textbf{Embrace Failures:} A ``Miss'' for the boss or a player isn't a dead end. It's an opportunity for a complication (SB spend) that makes the story more interesting.
    \item \textbf{Scale the Fight:} Don't be afraid to adjust on the fly. If the boss is too easy, spend SB to make it nastier. If it's too hard, let a fortunate player action create an opening.
\end{itemize}

\subsection{Example: The Tyrant-Engine}

\begin{itemize}
    \item \textbf{Draw:} ♠8 Adaptive Core, ♥K Domination, ♣Q Reality Distortion, ♢K Rewriting Fate. Highest rank K (Major) -> \textbf{8-segment Phase Clock}.
    \item \textbf{Theme:} A war machine fused with a corrupted intelligence, driven to impose order through force, capable of warping the battlefield.
    \item \textbf{Stats:} Tier II party base (~8 dice).
    \item \textbf{Mechanics:}
    \begin{itemize}
        \item \textbf{♠ Adaptive Core:} Gains resistance tags or minor condition immunities. Changes primary attack mode (melee/ranged/area) based on who is most threatening.
        \item \textbf{♥ Domination:} Focuses attacks on the perceived leader or most defiant PC. Gains bonuses when enemies are impaired/frightened.
        \item \textbf{♣ Reality Distortion:} ``Warp Field'' clock [6]. Fills via SB spends or when boss takes significant damage. When full: Range bands shift, gravity flickers, or a zone becomes [WARD] against certain actions for one round.
        \item \textbf{♢ Rewriting Fate:} Defeating it reveals a command code or core logic that can be used to control other similar constructs, or rewrite the narrative of how this fortress fell (ally survived, different outcome).
    \end{itemize}
    \item \textbf{Phases:} 3 phases (0-2/3-5/6-8 segments on Phase Clock) granting +0/+1/+2 dice respectively.
    \item \textbf{SB Spends:} Tie to ♥ (targeting defiant PCs) and ♣ (triggering minor warp effects, environmental hazards).
\end{itemize}

This system provides a structured spark for creativity, ensuring bosses are not just stat blocks but integral, dynamic parts of the narrative, perfectly aligned with Fate's Edge's core principles.

\section{Beyond the Combat Monster: Bosses as Systemic Challenges}
\label{sec:noncombat-bosses}

A \emph{boss} in \textit{Fate's Edge} does not need to be a creature to fight. It can be any significant, \emph{active} challenge that requires multiple scenes---often multiple sessions---to overcome.

\subsection*{Types of Non-Combat Bosses}

\paragraph{1) The Institutional Boss}
\textit{Examples:} a corrupt bureaucracy, a rigged legal system, an entrenched guild. \\
\textit{Mechanics:} multiple related clocks representing facets such as red tape, key officials, and public opinion. \\
\textit{Defeat:} reform the institution, circumvent it entirely, or replace it.

\paragraph{2) The Scheming Mastermind}
\textit{Examples:} a political figure with ongoing plans, a criminal kingpin with operations. \\
\textit{Mechanics:} a central \emph{Scheme} clock plus subsidiary \emph{Asset} clocks (followers, resources, safe houses). \\
\textit{Attacks:} pre-planned moves that trigger in response to player actions. \\
\textit{Defeat:} expose their plans, remove their power base, or turn their schemes against them.

\paragraph{3) The Environmental Boss}
\textit{Examples:} a spreading curse, an economic collapse, a natural disaster. \\
\textit{Mechanics:} a growing \emph{Threat} clock that spawns complications and subsidiary problems. \\
\textit{Attacks:} worsening conditions, resource depletion, cascading crises. \\
\textit{Defeat:} contain the threat, find its source, or adapt the community to survive it.

\paragraph{4) The Social Movement Boss}
\textit{Examples:} a popular uprising, a religious revival, a trade embargo. \\
\textit{Mechanics:} momentum clocks, faction support clocks, public opinion tracks. \\
\textit{Attacks:} shifting social pressure, mob actions, volatile alliances. \\
\textit{Defeat:} co-opt the movement, address root causes, or decisively crush it (with consequences).

\subsection*{Key Principles for Non-Combat Bosses}
\begin{itemize}
  \item \textbf{Structure:} Provide a clear mechanical representation (usually clocks) that shows the boss's \emph{health} or progress toward its goal.
  \item \textbf{Drive:} Give the boss explicit motivations and behaviors. What does it want? How does it respond to pressure?
  \item \textbf{Complications:} Define signature ways the boss creates problems beyond direct confrontation.
  \item \textbf{Reward/Twist:} Decide what happens when the boss is \emph{defeated}; aim for meaningful, potentially transformative outcomes.
  \item \textbf{Mini-Campaign Nature:} These bosses should require multiple scenes/encounters to resolve, not a single roll.
\end{itemize}

\subsection*{Using the Boss Generator for Non-Combat Bosses}
\begin{itemize}
  \item \textbf{Spades (Structure):} the boss's organization, resources, or foundational power.
  \item \textbf{Hearts (Drive):} the boss's core motivation and psychological profile.
  \item \textbf{Clubs (Complication):} the boss's signature methods for creating problems.
  \item \textbf{Diamonds (Reward):} what is gained by overcoming the boss (may be narrative currency or lasting leverage).
\end{itemize}

\subsection*{Design Note}
In \textit{Fate's Edge}, the greatest threats are often not monsters to slay but problems to solve, systems to reform, or schemes to unravel. The \emph{boss fight} frequently plays out as an extended conflict across investigation, social maneuvering, and strategic decision-making.
\subsection{Mythic Ascension: The Paragon's Path}
\label{sec:mythic_martial}

At Tier IV and beyond, the greatest warriors transcend the merely physical, becoming living legends whose prowess reshapes the battlefield itself. These Paragon Warriors are no longer simply fighters; they are \textit{forces of nature}, their willpower and skill manifesting in ways that defy mortal limits.

\subsubsection*{Mythic Tags for Martial Prowess}

Characters who embody the pinnacle of martial evolution may earn Mythic Tags that reflect their legendary status. These are not merely mechanical bonuses, but fundamental aspects of their heroic identity:

\begin{description}
    \item [\sovereign] The warrior commands the field. Allies within Near gain +1 Effect on combat actions. Enemies must test Resolve (DV 3) or suffer -1 die when directly engaging the Paragon.
    \item [\omen] The hero's fate is intertwined with destiny. Once per scene, when reduced to Harm 3 (Incapacitated), the Paragon may declare this Tag to instead stabilize at Harm 2 and gain +2 dice on their next action as fate intervenes.
    \item [\reality] The warrior's blows reshape truth. May ignore one instance of armor conversion or damage resistance. May sunder one non-artifact weapon or shield per scene.
\end{description}

\vspace{1em}
\textbf{Acquiring Mythic Tags:} These are granted by the GM for moments of truly legendary action that shift the campaign's narrative—defeating a dragon that threatened a kingdom, holding a pass against an army, or sacrificing greatly for allies. They represent the character's ascension beyond mortal bounds.

\subsubsection*{Legendary Actions and Domain Authority}

At Tier V, Paragon Warriors may perform actions that transcend normal physical laws:

\begin{itemize}
    \item \textbf{Momentum Cascade (1 \sovereign):} A single attack triggers a chain reaction, striking all enemies in Close range with -2 dice for secondary targets.
    \item \textbf{Unbroken Will (1 \omen):} Ignore all Conditions (Fear, Charm, etc.) for one full exchange.
    \item \textbf{Warrior's Dominion (1 \reality):} Declare a zone (e.g., "This Bridge is Mine"). All enemies in the zone suffer -1 die to attack rolls, while allies gain +1 Effect on defensive actions.
\end{itemize}

\subsubsection*{The Price of Legend}

Epic martial power demands transformation. As warriors ascend, they must choose \textit{Legendary Flaws} that reflect their mythic nature:
\begin{itemize}
    \item \textbf{The Burden of Honor:} Must always accept a fair duel. Refusing marks 2 SB.
    \item \textbf{The Last Stand:} When allies fall, gain +1 Effect but suffer +1 Harm automatically each round.
    \item \textbf{Destiny's Magnet:} Major villains and cosmic threats are drawn to the Paragon's presence (+1 to relevant encounter clocks).
\end{itemize}

The greatest warriors understand that true legend is not just about power gained, but about the humanity sacrificed to achieve it. Their epic tales echo through generations not just for their victories, but for the price eternally paid for those triumphs.

\section{Conclusion}

Tier IV and V play represents the pinnacle of storytelling in this system. Embrace the epic scope, honor player agency, and let the world truly respond to their legendary actions. Remember: these characters don't just participate in history---they make it.

The dice still matter, consequences still flow, and every choice still carries weight. But now, those choices echo across nations and generations.

Make it legendary.
