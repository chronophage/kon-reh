\chapter{Tier IV and V Play}\index{Tier IV}\index{Tier V}\index{epic tier}

As characters reach Tier IV and V levels, the scope of play expands dramatically. What once were local concerns become matters of regional, national, or even world-shaking importance. This chapter provides guidance for managing the unique challenges and opportunities that come with high-tier play.

\section{The Nature of High-Tier Play}\index{high-tier play}

At Tier IV and V, characters are no longer operating on the margins---they are movers and shakers. Their actions have visible, lasting impacts on the world. This shift requires the Game Master to think bigger, plan longer, and embrace the cascading consequences of player choices.

\subsection*{Key Characteristics}\index{high-tier play!characteristics}

\begin{itemize}
    \item \textbf{Wider Scope}: Actions affect cities, regions, or nations
    \item \textbf{Longer Timelines}: Consequences unfold over weeks, months, or years
    \item \textbf{Greater Stakes}: Failure means more than personal loss
    \item \textbf{Complex Alliances}: Multiple factions with competing interests
    \item \textbf{Legacy Impact}: Choices create lasting changes to the world
\end{itemize}

\section{Deck-Based Campaign Management}\index{deck-based management}

High-tier play benefits from structured campaign management using the Game Deck and other tools to track large-scale developments.

\subsection*{Campaign Clock Expansion}\index{Campaign Clocks!expansion}

Expand beyond core campaign clocks to include:

\begin{itemize}
    \item \textbf{Faction Influence} (6): Track major faction relationships
    \item \textbf{Public Opinion} (8): Regional perception of the group
    \item \textbf{Resource Network} (6): Economic and logistical reach
    \item \textbf{Legacy Projects} (10): Long-term initiatives with lasting impact
\end{itemize}

\subsection*{Using Cards for World Events}\index{Game Deck!world events}

Draw cards periodically to introduce world events:

\begin{itemize}
    \item \textbf{Suit 1 (Swords)}\index{Swords (suit)}: Geographic/political changes
    \item \textbf{Suit 2 (Crowns)}\index{Crowns (suit)}: Social/cultural shifts
    \item \textbf{Suit 3 (Anchors)}\index{Anchors (suit)}: Economic/resource developments
    \item \textbf{Suit 4 (Glyphs)}\index{Glyphs (suit)}: Opportunities/leverage points
\end{itemize}

\section{Managing Multiple Holdings and Allies}\index{Holdings!management}\index{Allies!management}

Tier IV+ characters often command extensive networks. Use these techniques to keep management manageable:

\subsection*{Holding Clustering}\index{Holdings!clustering}

Group related holdings into portfolios:
\begin{itemize}
    \item \textbf{Economic}: Trade routes, businesses, investments
    \item \textbf{Political}: Titles, contacts, influence networks
    \item \textbf{Military}: Retainers, fortifications, strategic positions
    \item \textbf{Intelligence}: Informants, research facilities
\end{itemize}

\subsection*{Ally Hierarchies}\index{Allies!hierarchies}

Create chains of command:
\begin{itemize}
    \item \textbf{Lieutenants} (Expertise 4-5): Direct reports who manage others
    \item \textbf{Commanders} (Expertise 3): Mid-level managers of specific portfolios
    \item \textbf{Agents} (Expertise 2): Field operatives and specialists
\end{itemize}

\section{High-Stakes Consequences}\index{high-stakes consequences}

Setback Points at high tiers should reflect the expanded scope of play:

\subsection*{High-Tier Setback Sinks}\index{Setback Points!high-tier sinks}

\begin{itemize}
    \item \textbf{3-4 SP}: Regional setback, major holding compromised
    \item \textbf{5-6 SP}: Faction relationship damaged, public scandal
    \item \textbf{7-8 SP}: Strategic position lost, major ally turned
    \item \textbf{9+ SP}: Paradigm shift, fundamental world change
\end{itemize}

\subsection*{Deck-Driven Consequences}\index{Game Deck!high-tier}

Use the Game Deck for major setbacks:
\begin{itemize}
    \item \textbf{Court Cards}: Major faction leaders or institutions affected
    \item \textbf{Aces}: Foundational assumptions challenged
    \item \textbf{Multiple Cards}: Cascade effects across multiple domains
\end{itemize}

\section{Running Epic Campaigns}\index{epic campaigns}

High-tier play often involves extended campaigns with multiple acts and lasting consequences.

\subsection*{Act Structure}\index{campaign structure}

\begin{itemize}
    \item \textbf{Act I - Establishment} (Sessions 1-3): Set the stage, establish stakes
    \item \textbf{Act II - Escalation} (Sessions 4-8): Complications multiply, alliances shift
    \item \textbf{Act III - Resolution} (Sessions 9-12): Climactic confrontations, lasting changes
    \item \textbf{Epilogue} (Session 13+): Legacy assessment, new beginnings
\end{itemize}

\subsection*{Campaign Seeds}\index{campaign seeds}

Use the full 4-card draw for major campaign hooks:
\begin{itemize}
    \item \textbf{Suit 1 (Swords)}: Primary location/region of conflict
    \item \textbf{Suit 2 (Crowns)}: Key faction/leader driving events
    \item \textbf{Suit 3 (Anchors)}: Major complication/threat
    \item \textbf{Suit 4 (Glyphs)}: Opportunity/resource to exploit
\end{itemize}

\section{Mass Combat and Warfare}\index{mass combat}\index{warfare}

Tier IV+ characters often find themselves commanding armies or influencing wars.

\subsection*{Army Scale Combat}\index{combat!army scale}

Simplify large-scale battles:
\begin{itemize}
    \item Treat armies as powerful allies with specialized skills
    \item Use clocks to track morale, supply, and strategic position
    \item Focus rolls on leadership and tactical decisions, not individual combat
\end{itemize}

\subsection*{Advanced Subsystem: Mass Combat}\index{advanced subsystem!mass combat}

For a more detailed warfare system, use the following framework:

\begin{description}
    \item[Army as an Entity] Create a character sheet for the army with Approaches (e.g., Aggressive, Disciplined, Cunning) and a "Morale \& Supply" clock.
    \item[The Battle Clock] Each significant battle is a 4-6 segment clock. Characters can contribute by using their skills to create advantages or by leading from the front.
    \item[Strategic Rolls] Commanders make skill checks against a target number. Success fills segments on the Battle Clock; failure fills segments on the army's "Morale \& Supply" clock or introduces a complication via the Game Deck.
\end{description}

\subsection*{War Campaigns}\index{war campaigns}

Structure extended conflicts:
\begin{itemize}
    \item \textbf{Strategic Phase}: Resource management, alliance building
    \item \textbf{Tactical Phase}: Key battles, covert operations
    \item \textbf{Political Phase}: Negotiations, aftermath management
\end{itemize}

\section{Mythic Challenges}\index{mythic challenges}

At Tier V, characters approach legendary status. Create challenges that match their stature:

\subsection*{Existential Threats}\index{existential threats}

\begin{itemize}
    \item Cosmic entities beyond normal understanding
    \item Reality-altering phenomena
    \item Threats to entire civilizations or ways of life
\end{itemize}

\subsection*{Legacy Missions}\index{legacy missions}

Missions that will be remembered for generations:
\begin{itemize}
    \item Founding or destroying nations
    \item Ending or beginning ages
    \item Reshaping fundamental aspects of the world
\end{itemize}

\subsection*{Advanced Subsystem: Legacy Projects}\index{advanced subsystem!legacy projects}

A Legacy Project is a long-term goal that extends beyond a single adventure. To run one:

\begin{description}
    \item[Define the Project] The players state their goal (e.g., "Build a Mage University," "Forge an Alliance of Kingdoms").
    \item[Create the Project Clock] This is a large clock, typically 8-12 segments.
    \item[Determine Prerequisites] The project may require specific resources, allies, or completed quests to even begin.
    \item[Milestone Advances] Instead of filling the clock with single rolls, each major story arc or significant achievement fills 2-3 segments. Setbacks from the Game Deck can remove segments.
\end{description}

\section{Managing Player Agency}\index{player agency}

With great power comes the need for great Game Master flexibility:

\subsection*{Player-Driven Narratives}\index{player-driven narratives}

\begin{itemize}
    \item Let player choices genuinely reshape the world
    \item Honor long-term commitments and consequences
    \item Provide meaningful opposition that matches their scale
\end{itemize}

\subsection*{World Reactivity}\index{world reactivity}

\begin{itemize}
    \item Factions respond realistically to player actions
    \item Economic and political systems show cause-and-effect
    \item Non-player characters remember and react to past interactions
\end{itemize}

\section{Rivals and Counterpoints}\index{rivals}

High-tier characters attract attention---both positive and negative:

\subsection*{Creating Worthy Opponents}\index{rivals!creating}

\begin{itemize}
    \item Mirror player capabilities and resources
    \item Give them their own networks and influence
    \item Create personal connections and history with the group
\end{itemize}

\subsection*{Dynamic Rivalry}\index{rivals!dynamic}

\begin{itemize}
    \item Rivals evolve based on player actions
    \item Competition across multiple domains (political, economic, social)
    \item Occasional cooperation against greater threats
\end{itemize}

\section{Campaign Legacy}\index{campaign legacy}

Help players see the lasting impact of their choices:

\subsection*{Legacy Tracking}\index{legacy tracking}

\begin{itemize}
    \item Document major world changes initiated by the group
    \item Track faction relationships and their evolution
    \item Record personal legacies and how they're remembered
\end{itemize}

\subsection*{Epilogue Framework}\index{epilogue}

Use cards to determine long-term outcomes:
\begin{itemize}
    \item Draw 2-3 cards from each suit
    \item Interpret results as 5-10 year outcomes
    \item Let players narrate their characters' final fates
\end{itemize}

\section{Game Master Preparation Tips}\index{Game Master preparation}

\subsection*{Think in Campaign Arcs}\index{campaign arcs}

\begin{itemize}
    \item Plan 3-5 major story arcs per tier
    \item Each arc should have lasting world impact
    \item Connect arcs through recurring themes or non-player characters
\end{itemize}

\subsection*{Prepare Flexible Frameworks}\index{flexible frameworks}

\begin{itemize}
    \item Create faction relationship matrices
    \item Develop economic and political systems that respond to actions
    \item Build modular locations that can evolve
\end{itemize}

\subsection*{Embrace Player Creativity}\index{player creativity}

\begin{itemize}
    \item Let player holdings genuinely solve problems
    \item Reward creative use of influence and resources
    \item Say "yes" to ambitious player plans, then make them interesting
\end{itemize}

\section{Sample High-Tier Scenario}\index{sample scenario}

\textbf{The Shattered Crown Crisis}

A Tier IV campaign seed:
\begin{itemize}
    \item \textbf{Suit 1 (The Royal Crypts)}: Ancient tombs beneath the capital
    \item \textbf{Suit 2 (The Usurper)}: A noble house claiming the vacant throne
    \item \textbf{Suit 3 (Fractured Loyalties)}: Regional lords choosing sides
    \item \textbf{Suit 4 (The Crown's Secret)}: Hidden royal treasures and alliances
\end{itemize}

Clocks: Succession Crisis (8), Noble Conspiracy (6), Public Unrest (6)

This scenario can evolve based on player choices---supporting the usurper, finding a true heir, or establishing a new form of government.

\section{Tools of the Game Master}\index{tools}

This section summarizes the key procedures for running epic-tier play.

\subsection*{Core Procedures}

\begin{description}
    \item[Starting an Epic Arc] Draw 4 cards (one per suit) to generate the core elements of a major story: Location, Faction, Threat, and Opportunity.
    \item[Managing the World] At the start of each session or after a major event, draw a card from the Game Deck to see how the world changes. Use the suit to determine the domain (Political, Social, Economic, Opportunistic).
    \item[Handling Major Setbacks] When players accumulate 3+ Setback Points, consider spending them for a high-tier consequence. Use the Game Deck to determine the nature of the setback, with Court Cards and Aces indicating severe, world-altering events.
    \item[Tracking Progress] Maintain the expanded set of Campaign Clocks (Faction Influence, Public Opinion, etc.) to mechanically represent the group's impact on the world.
\end{description}

\subsection*{Running Key Scenes}

\begin{description}
    \item[Mass Combat] Use the Battle Clock subsystem. Focus on the characters' leadership actions and their consequences for the army's morale.
    \item[Legacy Projects] Use the Legacy Project Clock. Advance it through milestone achievements, not individual rolls.
    \item[Epilogue] After the final session, use a multi-card draw from the Game Deck to inspire the narration of the world's and characters' long-term futures.
\end{description}

\section{Conclusion}

Tier IV and V play represents the pinnacle of storytelling in this system. Embrace the epic scope, honor player agency, and let the world truly respond to their legendary actions. Remember: these characters don't just participate in history---they make it.

The dice still matter, consequences still flow, and every choice still carries weight. But now, those choices echo across nations and generations.

Make it legendary.
\end{chapter}