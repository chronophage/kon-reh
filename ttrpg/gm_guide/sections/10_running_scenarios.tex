\chapter{Running the Game: A Practical Guide}
\index{running the game}\index{gm guidance}\index{session zero}

Running \textbf{Fate’s Edge} is not about mastering every rule—it is about managing \textbf{pressure, pacing, and care at the table}.  
The system is designed to support you, not replace your judgment.

This chapter focuses on the \emph{human side} of running the game: how to set expectations, structure sessions, call for breaks, frame scenes, and use the mechanics to support tension without burnout.

\section*{Session Zero: Setting the Ground Rules}

Before dice are rolled, establish shared expectations. Fate’s Edge thrives when everyone understands the tone, boundaries, and responsibilities at the table.

\subsection*{What Kind of Story Are We Playing?}
Discuss:
\begin{itemize}
  \item Tone (grim, hopeful, political, mythic, intimate)
  \item Themes that matter to the group
  \item What kinds of consequences feel exciting vs.\ exhausting
\end{itemize}

Clarity here prevents friction later.

\subsection*{Safety, Boundaries, and Trust}
Use clear safety practices:
\begin{itemize}
  \item Establish lines (hard limits) and veils (fade-to-black topics)
  \item Normalize check-ins during play
  \item Make it explicit that anyone may call for a pause or rewind
\end{itemize}

\textbf{Calling a break is never disruptive.}  
Emotional intensity is a feature of Fate’s Edge—not something to push through at all costs.

\section*{Calling Breaks and Managing Energy}

Fate’s Edge scenes can become emotionally dense. As GM:
\begin{itemize}
  \item Call breaks proactively every 60--90 minutes
  \item Pause after major consequences or revelations
  \item Watch for cognitive overload, not just time elapsed
\end{itemize}

A short pause preserves tension better than pushing through exhaustion.

\section*{Framing Scenes}

Every scene should answer one question:
\begin{quote}
\emph{What pressure are the characters under right now?}
\end{quote}

When framing a scene:
\begin{enumerate}
  \item State where the characters are
  \item State what is immediately at stake
  \item State what will happen if nothing changes
\end{enumerate}

If you cannot answer all three, the scene is not ready yet.

\subsection*{Zooming In and Zooming Out}

Not every moment deserves dice.
\begin{itemize}
  \item \textbf{Zoom in} when choices matter and pressure is high
  \item \textbf{Zoom out} when actions are routine or foregone
\end{itemize}

Use Clocks and Assets to resolve larger efforts without grinding.

\section*{Using Position, Clocks, and Story Beats}

These tools are not punishments—they are communication.

\subsection*{Position Sets Expectations}
Position tells players:
\begin{itemize}
  \item How much control they have
  \item How bad things can get
  \item What kind of consequences make sense
\end{itemize}

State Position aloud. Let players react.

\subsection*{Clocks Make Pressure Visible}
Use clocks to:
\begin{itemize}
  \item Track escalation
  \item Signal looming consequences
  \item Give players something to push against
\end{itemize}

Most scenes only need \textbf{one to three clocks}.  
If players can’t name the clocks, you’re using too many.

\subsection*{Story Beats Are Momentum}
Spend Story Beats to:
\begin{itemize}
  \item Introduce complications
  \item Escalate existing threats
  \item Shift the situation sideways, not just worse
\end{itemize}

A good SB spend changes the question the players are answering.

\section*{Let Failure Work for You}

Failure in Fate’s Edge is productive.
\begin{itemize}
  \item Misses generate Boons
  \item Partials move the story forward
  \item Consequences create new angles, not dead ends
\end{itemize}

If a failure stalls play, it was framed too narrowly.

\section*{Ending Scenes Cleanly}

End a scene when:
\begin{itemize}
  \item The central question is answered
  \item The pressure has shifted
  \item Continuing would dilute the impact
\end{itemize}

Cut hard. Let the consequences breathe.

\section*{Your Role as GM}

You are:
\begin{itemize}
  \item A fan of the characters
  \item A steward of pressure
  \item A facilitator of difficult choices
\end{itemize}

You are not:
\begin{itemize}
  \item The enemy
  \item The sole storyteller
  \item Responsible for carrying tension alone
\end{itemize}

\begin{tcolorbox}[title=The GM’s Touchstones, colback=gray!5, colframe=gray!60, fonttitle=\bfseries]
\begin{itemize}
  \item Show the pressure
  \item Let players choose where it lands
  \item Respect breaks
  \item Trust the mechanics
  \item End scenes sooner than you think
\end{itemize}
\end{tcolorbox}

If the table feels engaged, tense, and safe—you’re running Fate’s Edge correctly.

\newpage
\section{Illustrative Session Walkthrough}
\index{session walkthrough}\index{example play}

This section presents a full example session of \textbf{Fate’s Edge}, illustrating how scenes are framed, how pressure escalates, and how mechanics support narrative flow.  
The purpose is not to show a “perfect” outcome, but to demonstrate how success, failure, Boons, and Story Beats work together across an evening of play.

\subsection*{The Table}

\begin{itemize}
  \item \textbf{Valerius} — Ecktorian ex-legionary (Body 3, Warfare 2)
  \item \textbf{Elara} — Vhasian agent and negotiator (Wits 3, Sway 2)
  \item \textbf{Kael} — Dwarven geomancer and archivist (Lore 3, Geomancy 2)
\end{itemize}

The GM has prepared no fixed plot—only factions, motives, and clocks.

\subsection*{Campaign Context}

The group seeks a sealed charter proving their patron’s claim to a trade route.  
The charter is held by \textbf{Lord Silas}, a rival merchant prince.

\textbf{Known Pressure:}
\begin{itemize}
  \item Silas is politically insulated
  \item His manor is guarded but discreet
  \item Exposure would be as dangerous as failure
\end{itemize}

\subsection*{Scene 1: Information Gathering}

\textbf{GM Frames the Scene:}  
“You are in Silkstrand’s lower markets at dusk. Silas’s manor rises above you. Guards are visible, but relaxed.”

\textbf{Stakes Declared:}
\begin{itemize}
  \item Success: Identify an entry vector
  \item Failure: Guard suspicion escalates
\end{itemize}

\textbf{Clock Introduced:}
\begin{itemize}
  \item \textbf{Manor Alert} [4]
\end{itemize}

\paragraph{Action.}
Elara scouts using \textbf{Wits + Subterfuge}.

\textbf{Position:} Controlled

She rolls 2 Successes against DV 3 → \textbf{Partial Success}.

\paragraph{Resolution.}
The GM rules:
\begin{itemize}
  \item Elara identifies a servant gate (progress achieved)
  \item GM advances \textbf{Manor Alert} by 1 (cost)
  \item Elara gains \textbf{1 Boon}
\end{itemize}

The fiction advances; pressure is visible.

\subsection*{Scene 2: Entry}

The group decides to scale the outer wall at night.

\textbf{Position:} Desperate  
\textbf{Stakes:}
\begin{itemize}
  \item Success: Inside undetected
  \item Failure: Immediate pursuit
\end{itemize}

\paragraph{Action.}
Valerius rolls \textbf{Body + Athletics}.

He rolls 4 Successes including a 10 → \textbf{Clean Success with flourish}.

\paragraph{Resolution.}
\begin{itemize}
  \item The group enters cleanly
  \item GM reduces future Position penalties inside the manor
  \item No Story Beat is spent
\end{itemize}

This demonstrates how strong rolls relieve pressure without removing future danger.

\subsection*{Scene 3: Complication Inside}

Kael attunes to the stone to locate the vault.

\textbf{Position:} Controlled  
\textbf{Hidden Clock:} \emph{None} (the GM keeps pressure visible)

Kael rolls 0 Successes → \textbf{Miss}.

\paragraph{Resolution.}
\begin{itemize}
  \item Kael gains \textbf{2 Boons}
  \item GM spends 1 SB to introduce a servant unexpectedly entering
  \item GM advances \textbf{Manor Alert} by 1
\end{itemize}

The failure creates momentum rather than stalling play.

\subsection*{Scene 4: Social Pivot}

Elara intervenes, attempting to calm the servant.

\textbf{Position:} Risky  
\textbf{Clock Introduced:}
\begin{itemize}
  \item \textbf{Servant Panic} [4]
\end{itemize}

She rolls a \textbf{Success with Cost}.

\paragraph{Resolution.}
\begin{itemize}
  \item The servant is convinced
  \item GM spends SB to advance \textbf{Servant Panic} by 2
  \item The group now operates under time pressure
\end{itemize}

\subsection*{Scene Cut}

The GM cuts immediately.

\begin{quote}
“You have the vault location. You also hear boots on marble.”
\end{quote}

This prevents overplaying the scene and preserves tension.

\subsection*{Scene 5: The Vault}

The vault is magically reinforced.

\textbf{Clock:}
\begin{itemize}
  \item \textbf{Vault Integrity} [6]
\end{itemize}

Kael attempts a geomantic bypass.

He rolls a Partial (2 Successes vs DV 4).

\paragraph{Resolution.}
The GM chooses proportional progress:
\begin{itemize}
  \item \textbf{Vault Integrity} advances by 2
  \item GM takes a Story Beat instead of increasing Alert
  \item Kael gains 1 Boon
\end{itemize}

This demonstrates partials as \emph{GM-shaped outcomes}.

\subsection*{Scene 6: Extraction}

The vault opens just as guards arrive.

\textbf{Final Clock Check:}
\begin{itemize}
  \item Manor Alert at 3/4
\end{itemize}

Valerius creates a distraction while Elara escapes with the charter.

A final roll succeeds—but the GM fills \textbf{Manor Alert} as the cost.

\paragraph{Outcome.}
\begin{itemize}
  \item The charter is secured
  \item Silas knows he was robbed
  \item A new faction clock begins: \textbf{Silas’s Retaliation} [6]
\end{itemize}

\subsection*{Session Close}

\textbf{End-of-Session Review:}
\begin{itemize}
  \item Boons generated primarily from Partials and Misses
  \item Story Beats spent to escalate, not negate success
  \item New clocks seeded future conflict
\end{itemize}

The session ends with:
\begin{itemize}
  \item A clear win
  \item Lasting consequences
  \item Visible momentum into the next session
\end{itemize}

\paragraph{Takeaway.}
Fate’s Edge sessions are not about avoiding failure—they are about choosing where pressure lands, and watching the world respond.

\section{High-Tier Play Walkthrough: Tier IV/V and Boon Scarcity}
\index{high-tier play}\index{boons!scarcity}\index{example play}

This walkthrough illustrates how \textbf{Tier IV and Tier V} play feels at the table.  
At high tiers, characters roll more dice and wield powerful talents, but \textbf{Boons are scarcer and more precious}. Success increasingly comes from preparation, bonds, assets, and choosing where to accept cost.

\subsection*{The Characters (Tier IV)}

\begin{itemize}
  \item \textbf{Valerius, the Iron Standard} — Tier IV commander (Body 4, Warfare 3)
  \item \textbf{Elara, Voice of Silkstrand} — Tier IV political operator (Wits 4, Sway 3)
  \item \textbf{Kael, Deep-Delver of Stone} — Tier IV geomancer (Lore 4, Geomancy 3)
\end{itemize}

Each character begins the session with:
\begin{itemize}
  \item \textbf{0--1 Boons} (carryover from last session)
  \item Multiple Tier III–IV talents that \emph{consume} Boons
  \item At least one major Asset
\end{itemize}

\subsection*{Campaign Context}

The group now leads a regional faction.  
Their enemy, \textbf{House Morrenn}, has launched a coordinated political and military strike.

\textbf{Active Campaign Clocks:}
\begin{itemize}
  \item \textbf{Morrenn Ascendancy} [6]
  \item \textbf{Faction Stability} [6]
\end{itemize}

Failure no longer means inconvenience—it reshapes the campaign map.

\subsection*{Scene 1: The Council Confrontation}

House Morrenn publicly challenges the legitimacy of the PCs’ faction.

\textbf{Stakes:}
\begin{itemize}
  \item Success: Delay or fracture Morrenn’s coalition
  \item Failure: Advance Morrenn Ascendancy
\end{itemize}

\textbf{Position:} Risky  
\textbf{Clock:} \textbf{Council Opinion} [4]

\paragraph{Action.}
Elara invokes a Tier IV talent, \emph{Voice That Cannot Be Ignored}, costing \textbf{1 Boon}.

She rolls well—5 Successes—but has \textbf{no Boons left}.

\paragraph{Resolution.}
\begin{itemize}
  \item Council Opinion advances by 3
  \item GM spends a Story Beat: Morrenn marks Elara as a personal enemy
  \item A new clock appears: \textbf{Targeted Retaliation} [4]
\end{itemize}

\textbf{High-Tier Note:}  
The roll succeeds strongly, but the cost is \emph{strategic exposure}, not momentary danger.

\subsection*{Scene 2: Asset Pressure}

That night, assassins strike the PCs’ Safehouse (Major Asset).

\textbf{Clock:}
\begin{itemize}
  \item \textbf{Safehouse Compromised} [4]
\end{itemize}

Valerius could activate the Safehouse dramatically—but that costs a Boon.

\textbf{Problem:} He has none.

\paragraph{Decision.}
Instead of asset activation, Valerius:
\begin{itemize}
  \item Accepts a worse Position
  \item Takes Harm to protect key NPCs
\end{itemize}

\paragraph{Outcome.}
\begin{itemize}
  \item The Safehouse survives
  \item Valerius marks Harm 2
  \item Faction Stability advances by 1 (people saw leadership under fire)
\end{itemize}

\textbf{High-Tier Note:}  
At this tier, players often choose \emph{endurance over optimization}.

\subsection*{Scene 3: The Risky Play}

Kael attempts a ritual geomantic collapse to delay Morrenn troops.

\textbf{Position:} Desperate  
\textbf{DV:} High

He rolls a \textbf{Miss}.

\paragraph{Resolution.}
\begin{itemize}
  \item Kael gains \textbf{2 Boons}
  \item GM advances \textbf{Morrenn Ascendancy} by 1
  \item GM introduces long-term fallout: destabilized ley fault
\end{itemize}

\textbf{High-Tier Pattern:}  
Misses become \emph{intentional fuel}.  
Players sometimes accept failure to regain leverage.

\subsection*{Scene 4: Spending the Scarcity}

With 2 Boons regained, the group must decide:
\begin{itemize}
  \item Spend Boons to blunt the immediate threat
  \item Or save them for a coming decisive confrontation
\end{itemize}

They choose restraint.

\paragraph{Result.}
Morrenn advances—but the PCs are positioned for a future, decisive strike.

\subsection*{End-of-Session State}

\begin{itemize}
  \item Boons: 1–2 across the table
  \item Assets strained but intact
  \item Multiple clocks near resolution
  \item No inflated DVs were required
\end{itemize}

\paragraph{Takeaway.}
At Tier IV/V:
\begin{itemize}
  \item Power comes from \textbf{reach}, not safety
  \item Boons are \textbf{strategic currency}, not roll-fixers
  \item Failure is often chosen, not avoided
  \item The game shifts from tactical survival to \emph{campaign-scale consequence}
\end{itemize}

High-tier play is not easier—it is heavier.  
Every Boon spent should feel like a deliberate handoff of power, not a reflex.