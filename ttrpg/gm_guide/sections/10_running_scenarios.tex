\chapter{Running the Game: A Practical Guide}\index{running the game}\index{scenarios}

Reading the rules is one thing; feeling the flow of the game is another. This chapter provides a practical, illustrative walkthrough of how \textbf{Fate's Edge} operates at the table. We will follow a hypothetical group through several common scenarios, showing how the core procedures---Position, Rails, Clocks, and the Deck of Consequences---create a dynamic, responsive narrative. The goal is to see the rules not as restrictions, but as natural tools for collaborative storytelling.

\section*{The Setup: Our Intrepid Band10}

To illustrate, we'll follow a specific group:
\begin{itemize}
    \item \textbf{Valerius}: An Ecktorian ex-legionary (Body/Resolve), the group's protector.
    \item \textbf{Elara}: A Vhasian spy and infiltrator (Wits/Skulduggery), the group's face and trickster.
    \item \textbf{Kael}: A dwarven stonemason and lore-keeper (Lore/Geomancy), the group's scholar and planner.
\end{itemize}

They are in the city of \textbf{Silkstrand10}, Acasia, and have learned that a rival merchant, \textbf{Lord Silas}, possesses a sealed charter that proves their patron's rightful claim to a lucrative trade route. Their goal: acquire the charter from Silas's heavily guarded manor.

\section*{Scenario 1: The Heist - Infiltrating Silas's Manor}\index{heists}

A heist is a classic test of planning, improvisation, and dealing with cascading complications. Let's see how it unfolds.

\subsection*{Phase 1: The Approach - Gathering Information}

The players decide to case the manor before attempting entry. Elara suggests scouting the perimeter and socializing with the household staff at a nearby tavern.

\textbf{The Action}: Elara will use \textbf{Wits + Skulduggery} to identify patrol patterns and a weak point in the security.

\begin{itemize}
    \item \textbf{Position}: \textbf{Controlled}. The streets are watched, but the evening crowd provides some cover.
    \item \textbf{Rails}: The GM sets a \textbf{Hunt Rail} (4 segments) representing the alertness of Silas's guards. A complication might fill this clock.
\end{itemize}

Elara rolls: \textbf{2d10} (Wits 2 + Skulduggery 0). She gets a \textbf{5} and a \textbf{3} → a \textbf{Partial Success}.

\textbf{The Outcome}: She successfully identifies a side gate used by kitchen staff that is less frequently watched. \textbf{However}, the GM spends 1 Story Beat. A patrolling guard spots her loitering and becomes suspicious, advancing the \textbf{Hunt Rail} by 1 segment. The guard doesn't raise an alarm yet but will remember her face.

\subsection*{Phase 2: The Infiltration - A Desperate Climb}

With the side gate identified but now under increased scrutiny, Valerius proposes a different approach: scaling the outer wall in a blind spot under cover of darkness.

\textbf{The Action}: Valerius will use \textbf{Body + Athletics} to scale the wall.

\begin{itemize}
    \item \textbf{Position}: \textbf{Desperate}. The wall is high and slick with dew. A fall would be serious, and he's exposed.
    \item \textbf{Rails}: The \textbf{Hunt Rail} is now at 1/4. A complication here could be dire.
\end{itemize}

Valerius rolls: \textbf{3d10} (Body 3 + Athletics 0). He gets a \textbf{6}, a \textbf{2}, and a \textbf{1} → a \textbf{Full Success}! He scrambles silently over the wall and drops into a deserted herb garden.

\textbf{The Outcome}: No complication. He's inside. He secures a rope for the others. The GM notes that the \textbf{Desperate} position was overcome by a great roll, avoiding what could have been a nasty fall or immediate discovery.

\subsection*{Phase 3: The Complication - An Unlocked Door}

Inside, Kael uses his \textbf{Stone-Sense} to try and feel the layout of the manor's lower levels, hoping to locate the vault.

\textbf{The Action}: Kael uses \textbf{Lore + Geomancy} to attune to the stonework.

\begin{itemize}
    \item \textbf{Position}: \textbf{Controlled10}. He's in a quiet, stable area and can focus.
\end{itemize}

Kael rolls: \textbf{2d10} (Lore 2 + Geomancy 0). He gets a \textbf{1} and a \textbf{3} → a \textbf{Complication}.

\textbf{The Outcome}: He gets a vague sense of a reinforced room to the east, but the GM now has 2 SB to spend. The GM decides to introduce a new element: the door to the kitchen swing opens, and a young, nervous apprentice carrying a tray of wine steps out, freezing when he sees the intruders. The scene immediately shifts to a social encounter.

\subsection*{Phase 4: Improvisation - Swaying the Apprentice}

Elara quickly steps forward, putting herself between the apprentice and the armed Valerius.

\textbf{The Action}: Elara will use \textbf{Presence + Sway} to convince the apprentice he saw nothing, perhaps with a bribe.

\begin{itemize}
    \item \textbf{Position}: \textbf{Controlled}. He's scared and could easily scream.
    \item \textbf{Rails}: The GM invokes a \textbf{Curfew Rail} (6 segments)---how long until the master of the kitchen comes looking for the late wine?
\end{itemize}

Elara rolls: \textbf{3d10} (Presence 2 + Sway 1). She gets a \textbf{4}, a \textbf{5}, and a \textbf{2} → a \textbf{Partial Success}.

\textbf{The Outcome}: The apprentice is swayed by the coin and doesn't scream, but he whispers, "The master's steward makes his rounds in five minutes! You must be gone!" The GM advances the \textbf{Curfew Rail} by 2 segments, creating immediate time pressure. The heist continues, but the clock is ticking loudly.

\begin{tcolorbox}[title=Heist Flow Summary, colback=blue!5!white, colframe=blue!75!black, fonttitle=\bfseries]
This sequence shows the core loop:
\begin{enumerate}
    \item \textbf{Player declares goal and approach.}
    \item \textbf{GM sets Position and relevant Rails/Clocks.}
    \item \textbf{Roll determines outcome:} Success moves the plan forward; Partial Success does so with a cost (SB or Clock advance); Complication introduces a new problem (spending SB).
    \item \textbf{The fiction changes,} and the loop repeats. The game naturally oscillates between controlled planning and chaotic improvisation.
\end{enumerate}
\end{tcolorbox}

\section*{Scenario 2: The Aftermath - Social Fallout}\index{social encounters}

The group successfully retrieves the charter (though not without further close calls). However, Lord Silas knows he was robbed and suspects their patron. A few days later, Elara is invited to a high-society party at Silas's manor---a clear trap, but one she cannot refuse without admitting guilt.

\subsection*{The Scene: A Gilded Trap}

The party is in full swing. Silas corners Elara, his tone friendly but his eyes cold.

\textbf{The Action}: Elara needs to navigate this conversation without giving anything away, using \textbf{Wits + Sway} to maintain her cover story.

\begin{itemize}
    \item \textbf{Position}: \textbf{Desperate}. She's on his turf, surrounded by his allies.
    \item \textbf{Rails}: The GM sets a \textbf{Crowd Rail} (8 segments) representing the social pressure and potential for a public scandal that could ruin her patron.
\end{itemize}

Elara rolls: \textbf{3d10} (Wits 2 + Sway 1). She gets a \textbf{1}, a \textbf{1}, and a \textbf{4} → a \textbf{Complication}.

\textbf{The Outcome}: Disaster. Her story has holes. Silas smiles thinly and says, loud enough for others to hear, "A curious tale. It seems the rats in this city are growing bold." The GM spends the SB for a major social setback: the \textbf{Crowd Rail} is filled instantly. Whispers spread, and her patron's reputation takes a significant hit. The GM also draws from the Deck of Consequences for a long-term effect: the \textbf{Queen of Spades}---a major political figure (perhaps the Matron of Silkstrand herself) takes note of the scandal, creating a new, powerful rival.

\section*{Scenario 3: The Journey - A Chase through the Mistlands}\index{travel}\index{chases}

With heat increasing in Silkstrand, the group decides to flee north into the Mistlands to deliver the charter to a safe ally. Lord Silas has hired a band of mercenaries to pursue them.

This is a perfect opportunity to use the \textbf{Travel Deck} and abstract a chase sequence.

\subsection*{The Chase as a Series of Clocks}

The GM sets up two opposing clocks:
\begin{itemize}
    \item \textbf{PCs' Escape Clock} (6 segments): They need to lose their pursuers or reach the safety of the dwarven holds.
    \item \textbf{Pursuers' Hunt Clock} (6 segments): The mercenaries are closing in.
\end{itemize}

Each leg of the journey is resolved with a skill check, with the outcome affecting both clocks.

\textbf{Leg 1: Navigating the Fog}. Kael uses \textbf{Lore + Survival} to guide them.

\begin{itemize}
    \item \textbf{Position}: \textbf{Controlled}. The mist is thick and disorienting.
\end{itemize}

Kael rolls a \textbf{Partial Success}. The GM rules: The PCs advance their \textbf{Escape Clock} by 1 segment, but the pursuers also advance their \textbf{Hunt Clock} by 1 segment---the mercenaries are doggedly following their trail.

\textbf{Leg 2: Crossing the Charnel Bog}. Valerius uses \textbf{Body + Athletics} to find a safe path.

\begin{itemize}
    \item \textbf{Position}: \textbf{Desperate}. The bog is treacherous and slow-going.
\end{itemize}

Valerius rolls a \textbf{Full Success}! The PCs find a swift, hidden path, advancing their \textbf{Escape Clock} by 2 segments. The pursuers are stymied, and their \textbf{Hunt Clock} does not advance.

\textbf{Leg 3: The Ambush}. The Hunt Clock is at 4/6. The mercenaries catch up! This triggers a \textbf{Skirmish} as a discrete scene (see below), which will decisively impact the chase clocks.

\section*{Scenario 4: The Skirmish - A Fight in the Fog}\index{combat!examples}

The mercenaries emerge from the mist, blades drawn. The GM frames the conflict not as a round-by-round tactical simulation, but as a high-stakes action scene with a clear objective: \textbf{break through the ambush and escape}.

\subsection*{Setting the Stakes}

\begin{itemize}
    \item \textbf{Objective}: The PCs need to create an opening to flee.
    \item \textbf{Position}: \textbf{Desperate}. They are ambushed and outnumbered.
    \item \textbf{Clocks}: The GM creates a \textbf{Mob Overwhelm Clock} (4 segments). If it fills, the PCs are surrounded and captured.
\end{itemize}

\textbf{Valerius's Action}: He decides to charge the leader, hoping to break the mercenaries' morale with a show of force. He uses \textbf{Body + Warfare}.

He rolls a \textbf{Partial Success}. He clashes with the leader, holding him off, but the GM spends a SB: a lesser mercenary gets a lucky strike. Valerius takes \textbf{Harm 1} (a gash on his arm). The \textbf{Mob Overwhelm Clock} advances by 1 segment.

\textbf{Elara's Action}: Seeing Valerius in trouble, she throws a smoke pellet (a temporary asset) and uses \textbf{Wits + Skulduggery} to create a diversion.

She rolls a \textbf{Full Success}! The smoke and her shouts confuse the mercenaries, creating the needed opening. The \textbf{Mob Overwhelm Clock} is reduced by 2 segments as the enemy formation breaks.

\textbf{The Outcome}: With the opening created, Kael shouts for a retreat. The group disengages. The skirmish is over. They have survived but are wounded. The \textbf{Hunt Clock} is reset---the mercenaries are temporarily scattered. The PCs can now continue their journey, with the consequences of Valerius's harm applying to his future actions.

\section*{Conclusion: The Rhythm of Play}

As these examples show, \textbf{Fate's Edge} is not about rigidly following a script. It's about a conversation---a rhythm between the players' ambitions and the world's reactions.

\begin{tcolorbox}[title=The GM's Mantra, colback=green!5!white, colframe=green!75!black, fonttitle=\bfseries]
\begin{itemize}
    \item \textbf{Fiction First}: Always start with the fictional situation. What is happening? What makes sense?
    \item \textbf{Set Position and Rails}: Use these tools to define the stakes and pressure of a scene.
    \item \textbf{Let the Dice Decide}: Embrace the results. A complication is not a failure; it's a twist.
    \item \textbf{Spend Story Beats}: Make the world feel alive and reactive. Consequences should flow naturally from the fiction.
    \item \textbf{Think in Arcs}: Connect scenes. The social fallout from the party leads to the chase, which leads to new opportunities or threats in the Mistlands.
\end{itemize}
\end{tcolorbox}

Your role as the GM is to be a fan of the characters, a fair judge of the rules, and an enthusiastic architect of a world that responds. Let the players drive the story, and use the mechanics to make their choices feel meaningful and consequential. The dice will guide you to a story that neither you nor your players could have predicted, and that is the greatest strength of this game.

