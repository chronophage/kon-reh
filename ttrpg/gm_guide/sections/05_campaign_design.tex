\chapter{Campaigns, Clocks, and Consequences}\index{campaigns}\index{clocks}\index{consequences}

In \textbf{Fate's Edge}, campaigns are not just a string of adventures---they are \textbf{living narratives} shaped by player choices, faction dynamics, and the slow accumulation of influence that echoes through the ages. As the GM, you are the architect of long-term tension, guiding the story from its first spark to its final reckoning. This chapter introduces the tools that help you build and sustain that tension: the \textbf{Campaign Clocks}, the \textbf{Crown Spread}, and the art of managing consequences that ripple across entire seasons.

\section*{Campaign Clocks: Tracking Influence and Pressure}\index{Campaign Clocks}

The \textbf{Campaign Clocks} are two great dials that track the ebb and flow of player power and opposition over the course of a campaign. They are not mechanical scoreboards---they are \textbf{narrative thermometers}, showing how the world reacts to the PCs' actions and how the weight of their choices accumulates like stones in a riverbed.

\subsection*{Mandate (0--6)}\index{Campaign Clocks!Mandate}

\textbf{Mandate} represents the table's \textbf{public legitimacy and buy-in}—the measure of how much the world accepts the PCs' authority, influence, or sacred mission. It is the currency of reputation in a world that remembers both triumphs and failures.

\begin{fatebox}[Mandate Conditions and Effects]
\begin{tabularx}{\textwidth}{lX}
\toprule
\textbf{Mandate Level} & \textbf{Narrative Manifestations} \\
\midrule
0-2 (Low) & Suspicion dogs every step, doors remain closed, allies hesitate, every favor comes with strings attached \\
3-4 (Medium) & Respect is earned but conditional, some doors open while others require persuasion, trust must be continually maintained \\
5-6 (High) & Recognition precedes arrival, gates swing open unbidden, allies rally without question, reputation becomes a tangible asset \\
\bottomrule
\end{tabularx}
\end{fatebox}

\subsection*{Crisis (0--6)}\index{Campaign Clocks!Crisis}

\textbf{Crisis} tracks the \textbf{opposition engine}—the rising tide of rivals, mounting pressures, and accumulating attrition that defines a challenging campaign. It shows how much the world pushes back against ambition and change.

\begin{fatebox}[Crisis Conditions and Effects]
\begin{tabularx}{\textwidth}{lX}
\toprule
\textbf{Crisis Level} & \textbf{Narrative Manifestations} \\
\midrule
0-2 (Low) & Breathing room emerges, opportunities bloom like spring flowers, enemies regroup rather than attack \\
3-4 (Medium) & Pressure builds like gathering storm clouds, rivals make cautious moves, consequences become more immediate \\
5-6 (High) & Enemies strike with boldness born of desperation, clocks tick with alarming speed, the world tightens like a noose \\
\bottomrule
\end{tabularx}
\end{fatebox}

\subsection*{Advancing the Clocks}\index{Campaign Clocks!advancement}

At the end of each major scene, you may advance one or both clocks based on significant narrative developments:

\begin{itemize}
    \item \textbf{Clean Loss}: A rival codifies their position or escapes with leverage that threatens the party's standing
    \item \textbf{Public Cost Paid}: Extravagant feasts, declared holidays, or public penance that shifts perception
    \item \textbf{Asset Neglect}: Major assets degrade from inattention, signaling unreliability to potential allies
    \item \textbf{Evidence Shifts}: Immaculate reputations become scorched by scandal or revelation
\end{itemize}

\section*{Calling or Forcing the Crown}\index{Crown}

The campaign reaches its crescendo when one of two thresholds is met—the moment when accumulated influence and mounting pressure collide in a final reckoning.

\begin{fatebox}[Finale Triggers and Conditions]
\begin{tabularx}{\textwidth}{lX}
\toprule
\textbf{Finale Type} & \textbf{Conditions and Narrative Implications} \\
\midrule
Player-Called & Mandate ≥ 6 and Crisis ≤ 3—the party has earned the right to choose their moment of triumph \\
Forced Finale & Crisis ≥ 6 regardless of Mandate—the world forces a confrontation that can no longer be avoided \\
Balanced Finale & Both dials at 4-5—a tense equilibrium where victory and defeat hang in perfect balance \\
\bottomrule
\end{tabularx}
\end{fatebox}

\section*{The Crown Spread: Seeding the Campaign}\index{Crown Spread}

At \textbf{Session 0}, draw the \textbf{Crown Spread}—a five-card ritual that seeds the campaign's deepest themes, most dangerous rivals, and the very conditions of its ultimate resolution. This is not random chance but \textbf{oracular guidance} for the story to come.

\subsection*{Drawing the Spread}\index{Crown Spread!drawing}

Draw one card for each of the five fundamental aspects of your campaign:

\begin{itemize}
    \item \textbf{Spade}: The Crown Site—where destiny will be decided
    \item \textbf{Heart}: The Crown Rival—who stands between the party and their goals
    \item \textbf{Club}: The Crown Pressure—the relentless force that prevents complacency
    \item \textbf{Diamond}: The Crown Leverage—the advantage that can turn the tide
    \item \textbf{Wild}: The Hidden Force—the unknown element that will emerge when least expected
\end{itemize}

\subsection*{Interpreting the Spread}\index{Crown Spread!interpreting}

\begin{fatebox}[Crown Spread Interpretation Guide]
\begin{tabularx}{\textwidth}{lX}
\toprule
\textbf{Card Position} & \textbf{Interpretation Guidelines and Examples} \\
\midrule
Spade (Site) & A fortress shrouded in mist? A shrine built on forgotten truths? A battlefield where history repeats? \\
Heart (Rival) & A noble with hidden motives? A cult leader with apocalyptic visions? A spirit with ancient grievances? \\
Club (Pressure) & An escalating curfew? A spreading plague? A resource shortage that turns allies into competitors? \\
Diamond (Leverage) & Seasonal endorsement from powerful factions? A city license that grants unusual authority? \\
Wild (Hidden) & Face card: hidden patron steps from shadows; Ace: the site itself becomes a 10-clock challenge \\
\bottomrule
\end{tabularx}
\end{fatebox}

\textbf{Example Spread}: Spade = High-Mist Pass (Aeler territory); Heart = Margrave of Acasia (Face card—ambitious ruler); Club = Curfew (restricting movement); Diamond = Seasonal Endorsement (temporary authority); Wild = Hidden Patron (Face card—mysterious benefactor or foe).

\section*{The Finale Procedure}\index{Finale}

When the Crown is called, run the three-beat finale that brings the campaign to its narrative climax:

\begin{enumerate}
    \item \textbf{Reckoning}: Defend or sanctify the record of accomplishments. Draw upon the Rival's established motives. Place the Pressure rail that will drive the scene forward.
    \item \textbf{Crossing}: Stage the kinetic rail (Escape/Hunt/Hazard) that threatens to end the scene prematurely if not managed carefully.
    \item \textbf{Coronation}: Use the Diamond Leverage to sign, seal, or swear the oath that cements the campaign's legacy.
\end{enumerate}

\subsection*{Twist Collision (Finale Clause)}\index{Finale!Twist Collision}

Exactly once, when the Rival's Spade Twist contradicts their Club Belief, the table chooses:
\begin{itemize}
    \item GM gains +1 SB to complicate matters, or
    \item Players reduce two ticks total across the active rails, gaining breathing room.
\end{itemize}

\section*{Legacy Conversion: Epilogue}\index{Legacy Conversion}\index{epilogue}

After the Finale, each PC draws 2 cards and answers epilogue prompts by suit. Then convert campaign elements into lasting legacy:

\begin{itemize}
    \item \textbf{Major Asset → Institution} (12 XP): A safehouse becomes a school, a spy ring becomes an intelligence service
    \item \textbf{Seasonal Endorsement → Doctrine Rider} (4 XP): Temporary support becomes permanent policy
    \item \textbf{Follower (Cap 3+) → Stationed NPC} (0 XP): Loyal companions become custodians of the new order
    \item \textbf{Rival → Fixture}: Surviving adversaries become recurring elements of the setting's fabric
\end{itemize}

\section*{The Clockwork Engine: Tracking Tension}\index{clocks!introduction}

Clocks are the pulsating heart of tension in Fate's Edge. They represent ongoing conditions, threats, or progress toward objectives in a visible, tangible way that everyone can track and anticipate.

\subsection*{Types of Clocks}\index{clocks!types}

\begin{fatebox}[Clock Types and Their Purposes]
\begin{tabularx}{\textwidth}{lX}
\toprule
\textbf{Clock Type} & \textbf{Purpose and Typical Segment Count} \\
\midrule
Travel Clocks & Track progress through dangerous journey legs (4-10 segments) \\
Tactical Clocks & Represent ongoing combat conditions like morale or fatigue (4-8 segments) \\
Campaign Clocks & Track long-term pressure and influence (Mandate 0-6, Crisis 0-6) \\
Scene Clocks & Specific to immediate situations like chases or disasters (4-6 segments) \\
War Clocks & Large-scale conflict tracking like supply lines or morale (6-10 segments) \\
\bottomrule
\end{tabularx}
\end{fatebox}

\subsection*{Clock Creation Guidelines}\index{clocks!creation}

Creating effective clocks requires thoughtful design:

\begin{enumerate}
    \item \textbf{Announce Clearly}: Always tell players what each clock represents and what fictional events cause it to advance.
    \item \textbf{Logical Triggers}: Clock advancement should follow naturally from player actions and world events.
    \item \textbf{Visible Progression}: Use physical tokens or visual aids so everyone can see tension building.
    \item \textbf{Meaningful Consequences}: When clocks fill, the consequences should change the story in significant ways.
\end{enumerate}

\subsection*{Clock Advancement Rules}\index{clocks!advancement}

Story Beats drive clock progression in measurable ways:

\begin{itemize}
    \item \textbf{1 SB}: Minor advancement (1 segment)—a small but noticeable step forward
    \item \textbf{2-3 SB}: Moderate advancement (2 segments)—significant progress or escalation
    \item \textbf{4+ SB}: Major advancement (3+ segments) or filling smaller clocks entirely
    \item \textbf{Multiple Clocks}: Distribute SB across relevant clocks rather than overfilling one
\end{itemize}

\section*{Campaign Combat Integration}\index{campaign combat}

Extended conflicts and war-level events require special handling to maintain narrative tension while scaling the mechanical scope appropriately.

\subsection*{War Clocks}\index{War Clocks}

Large-scale conflicts are tracked through persistent war-level clocks that represent strategic realities:

\begin{fatebox}[War Clock Examples]
\begin{tabularx}{\textwidth}{lX}
\toprule
\textbf{War Clock} & \textbf{Strategic Implications and Triggers} \\
\midrule
Supply Lines (8) & Logistics and reinforcement flow; advances when routes are cut or resources dwindle \\
Morale (6) & Troop effectiveness and desertion risk; advances after defeats or poor conditions \\
Political Support (6) & Civilian and noble backing; advances when scandals emerge or costs mount \\
Strategic Position (8) & Control of key locations and routes; advances when territory is lost or gained \\
\bottomrule
\end{tabularx}
\end{fatebox}

\subsection*{Faction Combat}\index{faction combat}

When player factions engage in large-scale conflict, the rules adapt to maintain both narrative coherence and mechanical consistency:

\begin{itemize}
    \item \textbf{Follower Armies}: Cap 5 followers can represent military units with distinct capabilities
    \item \textbf{Asset Leverage}: Off-screen assets provide strategic advantages like intelligence or supply
    \item \textbf{Campaign Clock Impact}: Major battles significantly shift Mandate and Crisis dials
\end{itemize}

\section*{Between Sessions: The GM's Sacred Trust}\index{GM responsibilities}

Between game sessions, the Game Master undertakes crucial preparation that transforms good games into unforgettable campaigns. This quiet work is the foundation upon which epic stories are built.

\subsection*{Mandatory Preparation}\index{GM responsibilities!preparation}

\begin{fatebox}[Between-Session Checklist]
\begin{tabularx}{\textwidth}{lX}
\toprule
\textbf{Task} & \textbf{Description and Guidelines} \\
\midrule
Campaign Clock Updates & Advance Mandate/Crisis based on session outcomes. Track developments that affect long-term trajectory \\
Complication Debt & Calculate starting SB: banked SB (max 2) + character complications + asset complications \\
Thread Management & Review active complication threads. Ensure no more than (Tier + 1) active threads per scene \\
Resource Tracking & Update NPC statuses, faction relationships, and world conditions based on player actions \\
\bottomrule
\end{tabularx}
\end{fatebox}

\subsection*{Session Planning}\index{GM responsibilities!session planning}

Prepare the following elements with an eye toward pacing and player engagement:

\begin{itemize}
    \item \textbf{Scene Preparation}: Design scenes with appropriate SB spending budgets (standard: 12 SB max, climactic: 16 SB max, session: 20 SB total)
    \item \textbf{Complication Hooks}: Develop 3-5 potential complications connecting to player backgrounds and campaign themes
    \item \textbf{Tactical Considerations}: Prepare combat, social, and exploration challenges with appropriate difficulties
    \item \textbf{Deck Preparation}: Ensure Consequences Deck is ready with cards appropriate for expected complication types
\end{itemize}

\subsection*{XP Award Calculation}\index{XP awards}

Between sessions, calculate XP awards that reflect both accomplishment and engagement:

\begin{fatebox}[XP Award Guidelines]
\begin{tabularx}{\textwidth}{lX}
\toprule
\textbf{Award Type} & \textbf{Description and Typical Value} \\
\midrule
Table Attendance & +2 XP for participating in the shared story experience \\
Major Objectives & +2-4 XP for achieving significant story goals that advance the campaign \\
Discoveries & +1-2 XP for uncovering important information or hidden truths \\
Hard Choices & +1-2 XP for making difficult decisions with meaningful consequences \\
Complication Spotlight & +1-3 XP for engaging meaningfully with complications and setbacks \\
Bond/Flag Play & +1-2 XP for roleplaying that emphasizes relationships and character depth \\
GM Curveball & +0-3 XP for adapting well to unexpected developments and surprises \\
\bottomrule
\end{tabularx}
\end{fatebox}

\section*{Narrative First: The World Remembers}\index{narrative first}

Campaign design in Fate's Edge is not about railroading players along predetermined paths---it's about \textbf{responding to player choices} with consequences that accumulate like stones in a river, gradually shaping the flow of the narrative itself. Let the world shift in response to their actions. Let factions rise and fall based on their allegiances. Let the dice sing the song of a universe that reacts.

And when the Crown is finally crowned---when the last card is played and the final clock ticks to completion---let the echo of that moment be heard across the entire Amaranthine, a testament to stories well-lived and consequences fully earned.

Remember: Your preparation between sessions is the quiet magic that transforms random encounters into meaningful episodes and mechanical challenges into memorable stories. The investment in this sacred trust pays dividends in player engagement, narrative coherence, and the creation of campaigns that will be remembered long after the final dice have been rolled.

\end{chapter}

