\chapter{Appendix: Tools, Tables, and Optional Rules}\index{appendix}

This appendix provides quick-reference tools, sample content, and optional rules to support your game. Whether you're running a quick session or a long campaign, these tables and tips will help you keep the world alive and the tension high.

\section*{Quick Reference Sheets}\index{quick reference}

\subsection*{Outcome Matrix}\index{Outcome Matrix}

\begin{center}
\begin{tabular}{lll}
\toprule
\textbf{Case} & \textbf{Name} & \textbf{Guidance} \\
\midrule
$S \geq DV$ and $C = 0$ & Clean Success\index{Clean Success} & Deliver the intent crisply. \\
$S \geq DV$ and $C > 0$ & Success \& Cost\index{Success \& Cost} & Grant the intent; spend/bank SB for complications. \\
$0 < S < DV$ & Partial\index{Partial} & Progress with a fork. Award a boon. \\
$S = 0$ & Miss\index{Miss} & No progress. Cash/bank SB. Award two boons. \\
\bottomrule
\end{tabular}
\end{center}

\subsection*{Story Beat (SB) Spend Menu}\index{SB spend menu}

\begin{itemize}
    \item \textbf{1 SB}: Minor pressure: noise, trace, +1 Supply segment\index{Supply segment}.
    \item \textbf{2 SB}: Moderate setback: alarm raised, lose position/cover, lesser foe or lock.
    \item \textbf{3 SB}: Serious trouble: reinforcements, key gear breaks, rail tick\index{rail tick}.
    \item \textbf{4+ SB}: Major turn: trap springs, authority arrives, scene shifts.
\end{itemize}

\subsection*{Position Descriptions}\index{position}

\begin{itemize}
    \item \textbf{Controlled}\index{position!Controlled}: You act on your terms. Consequences of failure are manageable.
    \item \textbf{Risky}\index{position!Risky}: You act under pressure. Failure carries a real cost.
    \item \textbf{Desperate}\index{position!Desperate}: The odds are stacked against you. Failure could be catastrophic.
\end{itemize}

\subsection*{Difficulty Ladder (Set Before the Roll)}\index{Difficulty Value}

\begin{center}
\begin{tabular}{cll}
\toprule
\textbf{DV} & \textbf{Name} & \textbf{When to Use} \\
\midrule
2 & Routine\index{Difficulty Value!Routine} & Clear intent, modest stakes, controlled environment. \\
3 & Pressured\index{Difficulty Value!Pressured} & Time pressure, mild resistance, partial info. \\
4 & Hard\index{Difficulty Value!Hard} & Hostile conditions, active opposition, precise timing. \\
5+ & Extreme\index{Difficulty Value!Extreme} & Multiple constraints, high precision, dramatic failure. \\
\bottomrule
\end{tabular}
\end{center}

\section*{Optional Rule: Grid-Based Combat}\index{combat!grid-based}\index{optional rules}

While \textbf{Fate's Edge} is designed for theater-of-the-mind play, some groups prefer the tactical clarity of a grid. This optional rule provides a framework for using miniatures or tokens without adding excessive complexity.

\subsection*{Core Concepts}

\begin{itemize}
    \item \textbf{Zones of Control (ZoC)}: Each character exerts control over the squares immediately adjacent to them (typically the 8 surrounding squares). An enemy cannot move \textit{through} a square in another creature's ZoC without first engaging that creature or using a special ability. They can move \textit{around} it.
    \item \textbf{Ranges}: The GM sets ranges based on the narrative and the battlemap size.
        \begin{itemize}
            \item \textbf{Engaged}: In the same square or an adjacent square. For melee combat.
            \item \textbf{Near}: Within a short move (e.g., 5-6 squares). For thrown weapons, short charges.
            \item \textbf{Far}: Requires a full action to move into \textbf{Near} range. For bows, crossbows.
            \item \textbf{Distant}: Beyond \textbf{Far} range, requiring multiple moves or special effort to engage.
        \end{itemize}
    \item \textbf{Movement}: On their turn, a character can typically move a number of squares equal to 5 + their Body rating. Moving through difficult terrain (rubble, thick mud) may halve this distance or require an Athletics roll.
\end{itemize}

\subsection*{Actions on the Grid}

The core action resolution remains the same. The grid simply provides spatial context.
\begin{itemize}
    \item \textbf{Engage}: Move into an enemy's ZoC to enter melee.
    \item \textbf{Attack}: Make a combat roll as normal. Position is determined by the tactical situation (e.g., flanking an enemy might be \textbf{Risky} for you but \textbf{Desperate} for them).
    \item \textbf{Take Cover}: Move behind a terrain feature to improve position (e.g., from \textbf{Desperate} to \textbf{Risky}) against ranged attacks.
    \item \textbf{Flank}: By positioning on opposite sides of an enemy, allies may grant each other assistance dice on attacks.
\end{itemize}

\subsection*{Example Grid Combat}

Valerius is battling two bandits in a ruined temple. The GM sets up a map.
\begin{itemize}
    \item Valerius is \textbf{Engaged} with Bandit A. Bandit B is \textbf{Near} (4 squares away), behind a broken pillar.
    \item Valerius wants to charge Bandit B. He must first disengage from Bandit A's ZoC. The GM rules this is a \textbf{Risky} \textbf{Body + Athletics} roll. He succeeds, avoids an attack of opportunity, and moves into Bandit B's ZoC.
    \item Now engaged with Bandit B, Valerius attacks. The GM rules his position is \textbf{Risky} as Bandit A is now moving up behind him.
\end{itemize}

\section*{Optional Rule: Detailed Warfare}\index{warfare!detailed rules}\index{optional rules}

For campaigns where large-scale battles are a focus, this subsystem provides more granularity for mass combat.

\subsection*{The Army as a Character}

Treat an army as a powerful Follower with its own attributes and clocks.
\begin{itemize}
    \item \textbf{Scale}: The army's size and reach. Adds dice to strategic rolls (e.g., logistics, intimidation).
    \item \textbf{Discipline}: The army's training and morale. Used to resist routing and maintain formation.
    \item \textbf{Supply Clock} (6-8 segments): Tracks food, ammunition, and medical supplies. If filled, the army suffers penalties (e.g., -1 die to all rolls) and risks disintegration.
    \item \textbf{Morale Clock} (6-8 segments): Tracks the army's will to fight. Major defeats, poor conditions, or enemy terror tactics fill this clock. If filled, the army routs.
\end{itemize}

\subsection*{Battlefield Actions}

Instead of individual attacks, characters leading armies make command rolls to achieve objectives. Each objective is represented by a clock.
\begin{itemize}
    \item \textbf{Break Their Line} (6-segment clock): Use \textbf{Spirit + Command}. Success fills segments. Complications might fill the army's Morale Clock or allow an enemy counter-attack.
    \item \textbf{Flank the Enemy} (4-segment clock): Use \textbf{Wits + Skulduggery}. Requires a successful maneuver roll first.
    \item \textbf{Hold the Line} (Ongoing): Use \textbf{Body + Resolve} to withstand an enemy assault. Failure advances the enemy's objective clocks.
\end{itemize}

\subsection*{The Battle's Edge}

Warfare uses a modified Story Beat system called \textbf{The Battle's Edge}. SB generated from command rolls can be spent by the GM to represent the fog of war and battlefield chaos:
\begin{itemize}
    \item \textbf{1-2 SB}: A unit is out of position. A key piece of intelligence is wrong.
    \item \textbf{3-4 SB}: A trusted officer falls. A supply wagon is lost.
    \item \textbf{5+ SB}: The enemy unveils a secret weapon. The terrain turns against you (e.g., a dam breaks).
\end{itemize}

\section*{Sample NPCs}\index{NPCs}

\subsubsection*{Encounters}\index{NPCs!encounters}

\begin{itemize}
    \item \textbf{Bandit Skirmisher}: Body 2, Wits 2. Melee 2, Stealth 1. Light armor, opportunistic.
    \item \textbf{Ykrul Rider}: Body 4, Wits 3. Riding 3, Melee 3. Mobile, brutal.
    \item \textbf{Street Bravo}: Presence 3, Body 2. Dueling 3. Quick to anger.
\end{itemize}

\subsubsection*{Foils \& Rivals}\index{NPCs!foils}

\begin{itemize}
    \item \textbf{Ambitious Scribe}: Wits 3, Presence 3. Intrigue 3, Lore 2. Always knows a rumor.
    \item \textbf{Mercenary Captain}: Body 4, Spirit 3. Command 3, Melee 4. Pragmatic, dangerous ally.
    \item \textbf{Flame Preacher}: Presence 4, Spirit 3. Oratory 4, Faith 3. Incites mobs.
\end{itemize}

\subsubsection*{Prestige NPCs}\index{NPCs!prestige}

\begin{itemize}
    \item \textbf{High Elf Loremaster}: Wits 5, Spirit 4. Lore 5, Arcana 4. Knows secrets older than nations.
    \item \textbf{Dwarven Forge-Patriarch}: Body 5, Spirit 4. Craft 5, Command 4. Commands stone and steel.
    \item \textbf{Ykrul Warglord}: Body 5, Presence 4. Command 4, Melee 5. Unites clans with blood and will.
\end{itemize}

\section*{Deck of Consequences Interpretation Guide}\index{Deck of Consequences}

\subsection*{Two Deck Systems}\index{Deck of Consequences!two deck systems}

\paragraph{Deck of Consequences (scene drama).}
\emph{Hearts}\index{Hearts (suit)}=social fallout, \emph{Spades}\index{Spades (suit)}=harm/escalation, \emph{Clubs}\index{Clubs (suit)}=material cost, \emph{Diamonds}\index{Diamonds (suit)}=magical/spiritual disturbance.

\paragraph{Travel Decks (regional, 52-card).}\index{Travel Decks}
\emph{Spade}\index{Spades (suit)}=Place, \emph{Heart}\index{Hearts (suit)}=Actor, \emph{Club}\index{Clubs (suit)}=Pressure, \emph{Diamond}\index{Diamonds (suit)}=Leverage.

\subsection*{Hearts (Emotional/Social)}\index{Hearts (suit)}

\begin{itemize}
    \item \textbf{Ace--3}: Minor offense, awkward moment.
    \item \textbf{4--6}: Relationship strain, public embarrassment.
    \item \textbf{7--9}: Betrayal, scandal, loss of trust.
    \item \textbf{10--King}: Heartbreak, exile, shattered alliance.
\end{itemize}

\subsection*{Spades (Harm/Escalation)}\index{Spades (suit)}

\begin{itemize}
    \item \textbf{Ace--3}: Bruise, scrape, fatigue.
    \item \textbf{4--6}: Wound, gear damaged, position lost.
    \item \textbf{7--9}: Severe injury, ally down, structural collapse.
    \item \textbf{10--King}: Death, dismemberment, permanent loss.
\end{itemize}

\subsection*{Clubs (Material/Cost)}\index{Clubs (suit)}

\begin{itemize}
    \item \textbf{Ace--3}: Minor loss, delayed payment.
    \item \textbf{4--6}: Gear failure, debt incurred.
    \item \textbf{7--9}: Major asset lost, bankruptcy.
    \item \textbf{10--King}: Total ruin, legacy debt.
\end{itemize}

\subsection*{Diamonds (Magical/Spiritual)}\index{Diamonds (suit)}

\begin{itemize}
    \item \textbf{Ace--3}: Omen, whisper, strange coincidence.
    \item \textbf{4--6}: Curse triggered, spirit appears, past returns.
    \item \textbf{7--9}: Arcane backlash, forbidden knowledge revealed.
    \item \textbf{10--King}: Reality bends, godlike force intervenes.
\end{itemize}

\section*{Campaign Clock Examples}\index{Campaign Clocks}

\subsection*{Mandate Advancement Triggers}\index{Campaign Clocks!Mandate}

\begin{itemize}
    \item Public victory in battle or debate.
    \item Successful resolution of a major crisis.
    \item Recognition by a powerful faction or ruler.
\end{itemize}

\subsection*{Crisis Advancement Triggers}\index{Campaign Clocks!Crisis}

\begin{itemize}
    \item Rival faction gains influence or territory.
    \item Asset neglect or betrayal.
    \item Scandal or public loss of trust.
\end{itemize}

\section*{Travel Clock Sizes}\index{clocks!Travel}

\begin{itemize}
    \item \textbf{2--5}: 4 segments (short leg, low risk).
    \item \textbf{6--10}: 6 segments (standard journey).
    \item \textbf{J/Q/K}: 8 segments (long or dangerous route).
    \item \textbf{Ace}: 10 segments (epic or supernatural travel).
\end{itemize}

\section*{Follower and Asset Condition States}\index{Followers!condition}\index{Assets!condition}

\begin{itemize}
    \item \textbf{Maintained}: Full capability.
    \item \textbf{Neglected}: -1 die penalty; narrative wear.
    \item \textbf{Compromised}: Unavailable until repaired or recovered.
\end{itemize}

\section*{Boon Economy Quick Guide}\index{Boon economy}

\begin{itemize}
    \item \textbf{Holding cap}: You can hold at most 5 Boons\index{Boons}.
    \item \textbf{Conversion}: Once per session, in downtime, you may convert 2 Boons → 1 XP (max 2 XP via conversion per session).
    \item \textbf{Using Boons}: Re-roll one die after seeing the pool; Activate an Off-Screen Asset.
\end{itemize}

\section*{Mechanical Constraints}\index{mechanical constraints}

\begin{itemize}
    \item \textbf{ASSIST MAX}: +3 dice total per roll, regardless of helpers. Exception: The "Exceptional Coordination" Talent allows one follower to provide +4 assist dice.
    \item \textbf{BOON MAX}: 5 total, 2→1 XP conversion once/session (max 2 XP via conversion per session).
    \item \textbf{INITIATIVE}\index{Initiative Actions}: 1 Follower Action per scene party-wide.
    \item \textbf{OVER-STACK}\index{Over-Stack}: 2+ structural advantages = start rails +1 OR GM banks +1 SB.
    \item \textbf{POSITION}: Controlled | Risky | Desperate (affects success/failure texture).
\end{itemize}

\section*{Optional Rule: Hex-Based Exploration}\index{exploration!hex-based}\index{optional rules}

For a more structured exploration phase, the GM can map a region using a hex grid.

\begin{itemize}
    \item \textbf{Hex Size}: Typically 6 miles (a half-day's travel in clear terrain).
    \item \textbf{Travel}: Moving into a new hex requires a \textbf{Wits + Survival} roll. The DV is set by the terrain (DV 2 for plains, DV 4 for dense forest or mountains).
    \item \textbf{Discovery}: On a Clean Success, the group discovers any points of interest in the hex automatically. On a Success with Cost or Partial, they might stumble upon a danger first or only get a hint of the interest. On a Miss, they become lost, wasting time and resources.
    \item \textbf{Points of Interest}: Each hex can have a pre-planned location or one generated on the fly using the Travel Deck (Spade=Location, Heart=Encounter, etc.).
\end{itemize}

\section*{Let the Tools Serve You}

These tools and optional rules are not meant to constrain your game---they are meant to \textbf{support your vision}. Use them to keep tension high, consequences real, and the story moving forward. Choose the rules that fit your table's style, and don't be afraid to adapt them on the fly.

The ultimate goal is a collaborative, exciting story. These are just the brushes and paints.