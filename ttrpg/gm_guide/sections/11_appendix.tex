\chapter{Appendix: Tools, Tables, and Optional Rules}\index{appendix}

This appendix provides quick-reference tools, sample content, and optional rules to support your game. Whether you're running a quick session or a long campaign, these tables and tips will help you keep the world alive and the tension high.

\subsection{GM Cheat Sheets}

\subsubsection{Core Resolution Quick Reference}

\begin{tabular}{|l|l|l|}
\hline
\textbf{Position} & \textbf{Mechanical Effect} & \textbf{Narrative Impact} \\
\hline
Dominant & Re-roll one failure & Advantageous circumstances \\
Controlled & Standard resolution & Normal risk/reward balance \\
Desperate & Re-roll one success & High-stakes situation \\
\hline
\end{tabular}

\textbf{Dice Pool Calculation}: Attribute + Skill dice
\textbf{Success Threshold}: 6+ on d10
\textbf{Story Beat Generation}: 1s on any die

\subsubsection{Complication Management}

\textbf{Story Beat Economy}:
\begin{itemize}
\item 1 SB: Minor inconvenience or flavor complication
\item 2 SB: Moderate setback with narrative impact
\item 3 SB: Significant consequence altering the scene
\item 4+ SB: Major fallout introducing new problems
\end{itemize}

\textbf{Common Complication Types}:
\begin{itemize}
\item \textbf{Social}: Relationship strain, reputation damage, alliance shifts
\item \textbf{Physical}: Injury, equipment damage, environmental hazards
\item \textbf{Temporal}: Time pressure, missed opportunities, delayed consequences
\item \textbf{Resource}: Supply depletion, financial strain, loss of support
\end{itemize}

\section*{Quick Reference Sheets}\index{quick reference}

\subsection*{Outcome Matrix}\index{Outcome Matrix}

\begin{center}
\begin{tabular}{lll}
\toprule
\textbf{Case} & \textbf{Name} & \textbf{Guidance} \\
\midrule
$S \geq DV$ and $C = 0$ & Clean Success\index{Clean Success} & Deliver the intent crisply. \\
$S \geq DV$ and $C > 0$ & Success \& Cost\index{Success \& Cost} & Grant the intent; spend/bank SB for complications. \\
$0 < S < DV$ & Partial\index{Partial} & Progress with a fork. Award a boon. \\
$S = 0$ & Miss\index{Miss} & No progress. Cash/bank SB. Award two boons. \\
\bottomrule
\end{tabular}
\end{center}

\subsection*{Story Beat (SB) Spend Menu}\index{SB spend menu}

\begin{itemize}
    \item \textbf{1 SB}: Minor pressure: noise, trace, +1 Supply segment\index{Supply segment}.
    \item \textbf{2 SB}: Moderate setback: alarm raised, lose position/cover, lesser foe or lock.
    \item \textbf{3 SB}: Serious trouble: reinforcements, key gear breaks, rail tick\index{rail tick}.
    \item \textbf{4+ SB}: Major turn: trap springs, authority arrives, scene shifts.
\end{itemize}

\subsection*{Position Descriptions}\index{position}

\begin{itemize}
    \item \textbf{Dominant}\index{position!Dominant}: You act on your terms. Consequences of failure are manageable.
    \item \textbf{Controlled}\index{position!Controlled}: You act under pressure. Failure carries a real cost.
    \item \textbf{Desperate}\index{position!Desperate}: The odds are stacked against you. Failure could be catastrophic.
\end{itemize}

\subsection*{Difficulty Ladder (Set Before the Roll)}\index{Difficulty Value}

\begin{center}
\begin{tabular}{cll}
\toprule
\textbf{DV} & \textbf{Name} & \textbf{When to Use} \\
\midrule
2 & Routine\index{Difficulty Value!Routine} & Clear intent, modest stakes, controlled environment. \\
3 & Pressured\index{Difficulty Value!Pressured} & Time pressure, mild resistance, partial info. \\
4 & Hard\index{Difficulty Value!Hard} & Hostile conditions, active opposition, precise timing. \\
5+ & Extreme\index{Difficulty Value!Extreme} & Multiple constraints, high precision, dramatic failure. \\
\bottomrule
\end{tabular}
\end{center}

\section*{Optional Rule: Grid-Based Combat}\index{combat!grid-based}\index{optional rules}

While \textbf{Fate's Edge} is designed for theater-of-the-mind play, some groups prefer the tactical clarity of a grid. This optional rule provides a framework for using miniatures or tokens without adding excessive complexity.

\subsection*{Core Concepts}

\begin{itemize}
    \item \textbf{Zones of Control (ZoC)}: Each character exerts control over the squares immediately adjacent to them (typically the 8 surrounding squares). An enemy cannot move \textit{through} a square in another creature's ZoC without first engaging that creature or using a special ability. They can move \textit{around} it.
    \item \textbf{Ranges}: The GM sets ranges based on the narrative and the battlemap size.
        \begin{itemize}
            \item \textbf{Engaged}: In the same square or an adjacent square. For melee combat.
            \item \textbf{Near}: Within a short move (e.g., 5-6 squares). For thrown weapons, short charges.
            \item \textbf{Far}: Requires a full action to move into \textbf{Near} range. For bows, crossbows.
            \item \textbf{Distant}: Beyond \textbf{Far} range, requiring multiple moves or special effort to engage.
        \end{itemize}
    \item \textbf{Movement}: On their turn, a character can typically move a number of squares equal to 5 + their Body rating. Moving through difficult terrain (rubble, thick mud) may halve this distance or require an Athletics roll.
\end{itemize}

\subsection*{Actions on the Grid}

The core action resolution remains the same. The grid simply provides spatial context.
\begin{itemize}
    \item \textbf{Engage}: Move into an enemy's ZoC to enter melee.
    \item \textbf{Attack}: Make a combat roll as normal. Position is determined by the tactical situation (e.g., flanking an enemy might be \textbf{Controlled} for you but \textbf{Desperate} for them).
    \item \textbf{Take Cover}: Move behind a terrain feature to improve position (e.g., from \textbf{Desperate} to \textbf{Controlled}) against ranged attacks.
    \item \textbf{Flank}: By positioning on opposite sides of an enemy, allies may grant each other assistance dice on attacks.
\end{itemize}

\subsection*{Example Grid Combat}

Valerius is battling two bandits in a ruined temple. The GM sets up a map.
\begin{itemize}
    \item Valerius is \textbf{Engaged} with Bandit A. Bandit B is \textbf{Near} (4 squares away), behind a broken pillar.
    \item Valerius wants to charge Bandit B. He must first disengage from Bandit A's ZoC. The GM rules this is a \textbf{Controlled} \textbf{Body + Athletics} roll. He succeeds, avoids an attack of opportunity, and moves into Bandit B's ZoC.
    \item Now engaged with Bandit B, Valerius attacks. The GM rules his position is \textbf{Controlled} as Bandit A is now moving up behind him.
\end{itemize}

\section*{Optional Rule: Detailed Warfare}\index{warfare!detailed rules}\index{optional rules}

For campaigns where large-scale battles are a focus, this subsystem provides more granularity for mass combat.

\subsection*{The Army as a Character}

Treat an army as a powerful Follower with its own attributes and clocks.
\begin{itemize}
    \item \textbf{Scale}: The army's size and reach. Adds dice to strategic rolls (e.g., logistics, intimidation).
    \item \textbf{Discipline}: The army's training and morale. Used to resist routing and maintain formation.
    \item \textbf{Supply Clock} (6-8 segments): Tracks food, ammunition, and medical supplies. If filled, the army suffers penalties (e.g., -1 die to all rolls) and risks disintegration.
    \item \textbf{Morale Clock} (6-8 segments): Tracks the army's will to fight. Major defeats, poor conditions, or enemy terror tactics fill this clock. If filled, the army routs.
\end{itemize}

\subsection*{Battlefield Actions}

Instead of individual attacks, characters leading armies make command rolls to achieve objectives. Each objective is represented by a clock.
\begin{itemize}
    \item \textbf{Break Their Line} (6-segment clock): Use \textbf{Spirit + Command}. Success fills segments. Complications might fill the army's Morale Clock or allow an enemy counter-attack.
    \item \textbf{Flank the Enemy} (4-segment clock): Use \textbf{Wits + Subterfuge}. Requires a successful maneuver roll first.
    \item \textbf{Hold the Line} (Ongoing): Use \textbf{Body + Resolve} to withstand an enemy assault. Failure advances the enemy's objective clocks.
\end{itemize}

\subsection*{The Battle's Edge}

Warfare uses a modified Story Beat system called \textbf{The Battle's Edge}. SB generated from command rolls can be spent by the GM to represent the fog of war and battlefield chaos:
\begin{itemize}
    \item \textbf{1-2 SB}: A unit is out of position. A key piece of intelligence is wrong.
    \item \textbf{3-4 SB}: A trusted officer falls. A supply wagon is lost.
    \item \textbf{5+ SB}: The enemy unveils a secret weapon. The terrain turns against you (e.g., a dam breaks).
\end{itemize}

\section*{Sample NPCs}\index{NPCs}

\subsubsection*{Encounters}\index{NPCs!encounters}

\begin{itemize}
    \item \textbf{Bandit Skirmisher}: Body 2, Wits 2. Melee 2, Stealth 1. Light armor, opportunistic.
    \item \textbf{Ykrul Rider}: Body 4, Wits 3. Riding 3, Melee 3. Mobile, brutal.
    \item \textbf{Street Bravo}: Presence 3, Body 2. Dueling 3. Quick to anger.
\end{itemize}

\subsubsection*{Foils \& Rivals}\index{NPCs!foils}

\begin{itemize}
    \item \textbf{Ambitious Scribe}: Wits 3, Presence 3. Intrigue 3, Lore 2. Always knows a rumor.
    \item \textbf{Mercenary Captain}: Body 4, Spirit 3. Command 3, Melee 4. Pragmatic, dangerous ally.
    \item \textbf{Flame Preacher}: Presence 4, Spirit 3. Oratory 4, Faith 3. Incites mobs.
\end{itemize}

\subsubsection*{Prestige NPCs}\index{NPCs!prestige}

\begin{itemize}
    \item \textbf{High Elf Loremaster}: Wits 5, Spirit 4. Lore 5, Arcana 4. Knows secrets older than nations.
    \item \textbf{Dwarven Forge-Patriarch}: Body 5, Spirit 4. Craft 5, Command 4. Commands stone and steel.
    \item \textbf{Ykrul Warglord}: Body 5, Presence 4. Command 4, Melee 5. Unites clans with blood and will.
\end{itemize}

\section*{Deck of Consequences Interpretation Guide}\index{Deck of Consequences}

\subsection*{Two Deck Systems}\index{Deck of Consequences!two deck systems}

\paragraph{Deck of Consequences (scene drama).}
\emph{Hearts}\index{Hearts (suit)}=social fallout, \emph{Spades}\index{Spades (suit)}=harm/escalation, \emph{Clubs}\index{Clubs (suit)}=material cost, \emph{Diamonds}\index{Diamonds (suit)}=magical/spiritual disturbance.

\paragraph{Travel Decks (regional, 52-card).}\index{Travel Decks}
\emph{Spade}\index{Spades (suit)}=Place, \emph{Heart}\index{Hearts (suit)}=Actor, \emph{Club}\index{Clubs (suit)}=Pressure, \emph{Diamond}\index{Diamonds (suit)}=Leverage.

\subsection*{Hearts (Emotional/Social)}\index{Hearts (suit)}

\begin{itemize}
    \item \textbf{Ace--3}: Minor offense, awkward moment.
    \item \textbf{4--6}: Relationship strain, public embarrassment.
    \item \textbf{7--9}: Betrayal, scandal, loss of trust.
    \item \textbf{10--King}: Heartbreak, exile, shattered alliance.
\end{itemize}

\subsection*{Spades (Harm/Escalation)}\index{Spades (suit)}

\begin{itemize}
    \item \textbf{Ace--3}: Bruise, scrape, fatigue.
    \item \textbf{4--6}: Wound, gear damaged, position lost.
    \item \textbf{7--9}: Severe injury, ally down, structural collapse.
    \item \textbf{10--King}: Death, dismemberment, permanent loss.
\end{itemize}

\subsection*{Clubs (Material/Cost)}\index{Clubs (suit)}

\begin{itemize}
    \item \textbf{Ace--3}: Minor loss, delayed payment.
    \item \textbf{4--6}: Gear failure, debt incurred.
    \item \textbf{7--9}: Major asset lost, bankruptcy.
    \item \textbf{10--King}: Total ruin, legacy debt.
\end{itemize}

\subsection*{Diamonds (Magical/Spiritual)}\index{Diamonds (suit)}

\begin{itemize}
    \item \textbf{Ace--3}: Omen, whisper, strange coincidence.
    \item \textbf{4--6}: Curse triggered, spirit appears, past returns.
    \item \textbf{7--9}: Arcane backlash, forbidden knowledge revealed.
    \item \textbf{10--King}: Reality bends, godlike force intervenes.
\end{itemize}

\section*{Campaign Clock Examples}\index{Campaign Clocks}

\subsection*{Mandate Advancement Triggers}\index{Campaign Clocks!Mandate}

\begin{itemize}
    \item Public victory in battle or debate.
    \item Successful resolution of a major crisis.
    \item Recognition by a powerful faction or ruler.
\end{itemize}

\subsection*{Crisis Advancement Triggers}\index{Campaign Clocks!Crisis}

\begin{itemize}
    \item Rival faction gains influence or territory.
    \item Asset neglect or betrayal.
    \item Scandal or public loss of trust.
\end{itemize}

\section*{Travel Clock Sizes}\index{clocks!Travel}

\begin{itemize}
    \item \textbf{2--5}: 4 segments (short leg, low risk).
    \item \textbf{6--10}: 6 segments (standard journey).
    \item \textbf{J/Q/K}: 8 segments (long or dangerous route).
    \item \textbf{Ace}: 10 segments (epic or supernatural travel).
\end{itemize}

\section*{Follower and Asset Condition States}\index{Followers!condition}\index{Assets!condition}

\begin{itemize}
    \item \textbf{Maintained}: Full capability.
    \item \textbf{Neglected}: -1 die penalty; narrative wear.
    \item \textbf{Compromised}: Unavailable until repaired or recovered.
\end{itemize}

\section*{Boon Economy Quick Guide}\index{Boon economy}

\begin{itemize}
    \item \textbf{Holding cap}: You can hold at most 5 Boons\index{Boons}.
    \item \textbf{Conversion}: Once per session, in downtime, you may convert 2 Boons → 1 XP (max 2 XP via conversion per session).
    \item \textbf{Using Boons}: Re-roll one die after seeing the pool; Activate an Off-Screen Asset.
\end{itemize}

\section*{Mechanical Constraints}\index{mechanical constraints}

\begin{itemize}
    \item \textbf{ASSIST MAX}: +3 dice total per roll, regardless of helpers. Exception: The "Exceptional Coordination" Talent allows one follower to provide +4 assist dice.
    \item \textbf{BOON MAX}: 5 total, 2→1 XP conversion once/session (max 2 XP via conversion per session).
    \item \textbf{INITIATIVE}\index{Initiative Actions}: 1 Follower Action per scene party-wide.
    \item \textbf{OVER-STACK}\index{Over-Stack}: 2+ structural advantages = start rails +1 OR GM banks +1 SB.
    \item \textbf{POSITION}: Dominant/Controlled/Desperate (affects success/failure texture).
\end{itemize}

\section*{Optional Rule: Hex-Based Exploration}\index{exploration!hex-based}\index{optional rules}

For a more structured exploration phase, the GM can map a region using a hex grid.

\begin{itemize}
    \item \textbf{Hex Size}: Typically 6 miles (a half-day's travel in clear terrain).
    \item \textbf{Travel}: Moving into a new hex requires a \textbf{Wits + Survival} roll. The DV is set by the terrain (DV 2 for plains, DV 4 for dense forest or mountains).
    \item \textbf{Discovery}: On a Clean Success, the group discovers any points of interest in the hex automatically. On a Success with Cost or Partial, they might stumble upon a danger first or only get a hint of the interest. On a Miss, they become lost, wasting time and resources.
    \item \textbf{Points of Interest}: Each hex can have a pre-planned location or one generated on the fly using the Travel Deck (Spade=Location, Heart=Encounter, etc.).
\end{itemize}

\section*{Let the Tools Serve You}

These tools and optional rules are not meant to constrain your game---they are meant to \textbf{support your vision}. Use them to keep tension high, consequences real, and the story moving forward. Choose the rules that fit your table's style, and don't be afraid to adapt them on the fly.

The ultimate goal is a collaborative, exciting story. These are just the brushes and paints.

\section{Miniatures and Tactical Layer}
\label{sec:miniatures}

\subsection{Core Concepts}
\begin{itemize}
  \item Works on square or hex grids; declare grid type at setup.
  \item Units have base sizes (Small, Medium, Large, Huge) and a facing.
  \item Actions per turn: Move and Act (attack, cast, interact, etc.), in either order.
  \item All checks use normal SRD roll + DV system.
\end{itemize}

\subsection{Turn Structure}
\begin{enumerate}
  \item Start: resolve ongoing effects.
  \item Move: up to Speed; obey Zones of Control (ZOC).
  \item Act: attack, test, assist, cast, rally, shove, guard, etc.
  \item End: resolve end effects and reactions.
\end{enumerate}

\subsection{Zones of Control (ZOC)}
\begin{itemize}
  \item \textbf{Squares:} 4 orthogonal adjacents (optional: 8). 
  \item \textbf{Hexes:} 6 adjacents.
  \item Large/Huge project ZOC from edges; Reach may extend ZOC by +1 ring.
  \item \textbf{Rules:} 
    \begin{itemize}
      \item Entering enemy ZOC ends movement (you are engaged).
      \item Cannot move through enemy ZOC.
      \item Leaving requires Disengage (DV 4–6) or spend 1 Boon.
      \item Multiple ZOCs increase DV by +1 per extra controller.
    \end{itemize}
\end{itemize}

\subsection{Facing and Flanking}
\begin{itemize}
  \item Choose a facing at end of movement.
  \item Flank: +1 die if attacked from opposite arcs; Rear: +1 die and +1 Effect.
\end{itemize}

\subsection{Special Actions}
\begin{itemize}
  \item \textbf{Guard:} Ready a strike when enemy leaves ZOC.
  \item \textbf{Dash:} +2 movement this turn.
  \item \textbf{Brace:} Resist Shoves/Pulls and extend ZOC (opportunity only).
  \item \textbf{Tackle:} Knock target prone (DV 4–6).
\end{itemize}

\subsection{Magic Integration}
\begin{itemize}
  \item Magic uses \textbf{[TAGS]} (e.g., [WARD], [BANISH], [CONJURE]) tied to ZOC, range, and LoS.
  \item Casting while engaged worsens Position unless [INSTANT] or aided by Talent.
  \item Rituals require clear space and visible Symbols; disrupted rituals fail or require a test.
\end{itemize}

\subsection{Quick Reference}
\begin{itemize}
  \item Entering enemy ZOC ends movement; leaving requires Disengage.
  \item Flank = +1 die; Rear = +1 die and +1 Effect.
  \item Difficult terrain +1 cost; moving up elevation +1.
  \item Boons may break ZOC rules: auto-Disengage, change facing, or Heroic Rush.
\end{itemize}

\begin{tcolorbox}[title=\textbf{Miniatures Mode — Speed Defaults},colback=white!98!gray,colframe=black!50!gray,boxrule=0.4pt]
\textbf{DV:} $\mathrm{DV}=\mathrm{Tier}+2+\text{Keywords}$ \quad(Elevation +1, Altar[WARD] +1, Disengage=4).\\
\textbf{Crit:} Bump Position one step; if already Dominant, Push/Pull 1 hex \emph{or} gain +1 Success.\\
\textbf{ZOC:} Enter/leave an adjacent hex provokes 1 \emph{Reaction} (Free Strike \emph{or} Shove 1 hex). Each unit has 1 Reaction/round.\\
\textbf{Tags:} Max 2 active tags per unit. [WARD] = -1 die vs target; attacker may accept 2 Fatigue to ignore once.\\
\textbf{Terrain:} Difficult=2 MP/hex. Elevation=+1 DV from below.\\
\textbf{Heat:} On any Crit, GM immediately spends 1 Heat to degrade Position or trigger terrain.
\end{tcolorbox}

\begin{tcolorbox}[title=\textbf{Hex Keywords},colback=white!98!gray,colframe=black!50!gray,boxrule=0.4pt]
\textbf{Difficult:} 2 MP/hex \quad \textbf{Elevation:} +1 DV from below \quad \textbf{ZOC:} Reaction on cross\\
\textbf{Altar [WARD]:} -1 die to target (or attacker takes 2 Fatigue to ignore)\\
\textbf{Incorporeal:} Ignore Difficult; may pass through occupied hexes; cannot end there\\
\textbf{Assist (mini):} +1 Effect (not dice); max 1 helper
\end{tcolorbox}


x% ===== GM SCREEN: COMMON ROLLS QUICK REFERENCE =====
\section*{Common Rolls (GM Screen)}
\small
% Tip: print this on a panel; pairs give you fiction-first handles.
\begin{multicols}{2}
\paragraph{Athletics}
Climb rough wall (\emph{Body+Athletics}); sprint a gap (\emph{Body+Athletics}); time a leap to a moving cart (\emph{Wits+Athletics}).

\paragraph{Stealth}
Shadow a patrol (\emph{Wits+Stealth}); cross a lit balcony silently (\emph{Body+Stealth}); hold still under lantern sweep (\emph{Spirit+Stealth}).

\paragraph{Endurance}
Resist cold night march (\emph{Spirit+Endurance}); push through pain (\emph{Spirit+Endurance}); carry wounded comrade (\emph{Body+Endurance}).

\paragraph{Craft}
Blueprint a fix (\emph{Wits+Craft}); brace a broken door (\emph{Body+Craft}); restore a relic carefully (\emph{Spirit+Craft}).

\paragraph{Melee}
Break guard’s stance (\emph{Body+Melee}); bind blade to set up ally (\emph{Wits+Melee}); press the advantage while bleeding (\emph{Spirit+Melee}).

\paragraph{Ranged}
Leading shot at sprinting target (\emph{Wits+Ranged}); loose in a squall (\emph{Spirit+Ranged}); snap throw in close quarters (\emph{Body+Ranged}).

\paragraph{Brawl}
Grapple and pin (\emph{Body+Brawl}); feint to open a clinch (\emph{Wits+Brawl}); fight on dazed (\emph{Spirit+Brawl}).

\paragraph{Tactics}
Set an ambush lane (\emph{Wits+Tactics}); coordinate fighting retreat (\emph{Presence+Tactics}); read enemy morale at a glance (\emph{Wits+Tactics}).

\paragraph{Diplomacy}
Formal audience etiquette (\emph{Presence+Diplomacy}); draft terms both sides can live with (\emph{Wits+Diplomacy}); keep decorum under insult (\emph{Spirit+Diplomacy}).

\paragraph{Sway}
Haggle fast for a better price (\emph{Presence+Sway}); sell a risky plan to allies (\emph{Presence+Sway}); change a mind mid-argument (\emph{Wits+Sway}).

\paragraph{Deception}
Tell a clean lie under scrutiny (\emph{Presence+Deception}); misdirect with half-truths (\emph{Wits+Deception}); hold a lie when cornered (\emph{Spirit+Deception}).

\paragraph{Performance}
Captivate a restless crowd (\emph{Presence+Performance}); mimic accent and posture (\emph{Wits+Performance}); steady stage nerves (\emph{Spirit+Performance}).

\paragraph{Subterfuge}
Talk past a checkpoint in a borrowed coat (\emph{Presence+Subterfuge}); case staff routines over one drink (\emph{Wits+Subterfuge}); palm/plant during a handshake (\emph{Body+Subterfuge}); keep a cover through interrogation (\emph{Spirit+Subterfuge}).

\paragraph{Streetwise}
Find a fence by sundown (\emph{Presence+Streetwise}); sift rumor from bait (\emph{Wits+Streetwise}); walk a bad block without flashing fear (\emph{Spirit+Streetwise}).

\paragraph{Arcana}
Read a ward’s anchor (\emph{Wits+Arcana}); hold a rite steady in chaos (\emph{Spirit+Arcana}); countermark a seal (\emph{Wits+Arcana}).

\paragraph{Mechanics}
Diagnose a jammed lock (\emph{Wits+Mechanics}); disarm a sprung trap without firing it (\emph{Wits+Mechanics}); field-rig a pump with scrap (\emph{Body+Mechanics}).

\paragraph{Investigation}
Reconstruct a scene’s timeline (\emph{Wits+Investigation}); follow a paper trail (\emph{Wits+Investigation}); interview to fill a gap (\emph{Presence+Investigation}).

\paragraph{Lore}
Cite a custom that grants passage (\emph{Presence+Lore}); recall taboo at an old shrine (\emph{Wits+Lore}); perform a rite correctly over hours (\emph{Spirit+Lore}).

\paragraph{Nature}
Read tomorrow’s weather from sky-signs (\emph{Wits+Nature}); track a limping stag over stone (\emph{Wits+Nature}); calm a spooked mount (\emph{Presence+Nature}).

\paragraph{Medicine}
Stabilize in the field (\emph{Wits+Medicine}); cut out rot cleanly (\emph{Body+Medicine}); talk a patient through the pain (\emph{Presence+Medicine}).

\paragraph{Command}
Rally shaken allies (\emph{Presence+Command}); issue clear orders in chaos (\emph{Wits+Command}); hold the line when it ought to break (\emph{Spirit+Command}).
\end{multicols}

\medskip
\noindent\textbf{Fast Boundaries}
\begin{itemize}[leftmargin=*]
  \item \textbf{Locks \& Traps:} \emph{Mechanical} = \textbf{Mechanics + Attribute}; \emph{Arcane} = \textbf{Arcana + Attribute}.
  \item \textbf{People vs. Mechanisms:} \textbf{Subterfuge} gets you past \emph{people} (papers, covers, diversions); \textbf{Stealth} keeps you unseen; \textbf{Mechanics/Arcana} open the thing.
  \item \textbf{Formal vs. Informal:} \textbf{Diplomacy} (courts, treaties, protocol) vs. \textbf{Sway} (informal persuasion, bargaining).
\end{itemize}

\textbf{Subterfuge — Common Rolls}
Wits + Subterfuge: case venue; map guard habits. \\
Presence + Subterfuge: talk past checkpoint; play the official. \\
Body + Subterfuge: palm/plant during a handshake.

\subsection{Experience Point Costs}
\label{subsec:xp-costs-ref}
\index{Appendices!XP Costs}

\begin{center}
\feTableStart
\begin{tabularx}{\linewidth}{@{}l l l @{}}
\toprule
\textbf{Improvement} & \textbf{Cost} & \textbf{Downtime} \\
\midrule
Attribute increase & New rating $\times$ 3 XP & New rating days \\
Skill increase & New level $\times$ 2 XP & New level days \\
On-Screen Follower & Cap$^2$ XP & 1--3 days \\
Minor Asset & 4 XP & 1 day \\
Standard Asset & 8 XP & 1 week \\
Major Asset & 12 XP & 1 month \\
\bottomrule
\end{tabularx}
\feTableEnd
\end{center}

% !TEX root = srd_main.tex
% SRD Insert: Upkeep (Condensed)

% !TEX root = srd_main.tex
% SRD Insert: Upkeep (Condensed)

\subsection*{Upkeep}\label{sec:upkeep-srd}
\index{Upkeep}\index{Downtime}\index{Followers}\index{Assets}

\textbf{Frequency.} Pay upkeep once per Downtime period.

\begin{itemize}
\item \textbf{Efficient (Higher XP, Less Time).} \emph{Cost:} Upkeep XP $= \max\big(1, \tfrac{\text{XP Acquisition Cost}}{3}\big)$. \emph{Time:} Minimal; delegation/check-in.
\item \textbf{Intensive (Lower XP, More Time).} \emph{Cost:} 1 XP. \emph{Time:} One dedicated Downtime action of significant personal attention.
\end{itemize}

\textbf{Failure.} If upkeep is not paid this Downtime, the resource degrades:
\begin{itemize}
\item \emph{Follower:} becomes \textbf{Wary} (or \textbf{Seized} if already Wary).
\item \emph{Asset:} becomes \textbf{Neglected} (or \textbf{Compromised} if already Neglected).
\end{itemize}

\paragraph{Notes.} Each follower/asset checks upkeep separately; a single Intensive action may cover a cohesive group if fiction supports it. Tie upkeep scenes to Patron themes for flavor, not discounts.

\subsection{Difficulty Value (DV) Reference}
\label{subsec:dv-reference}
\index{Appendices!DV Reference}

\begin{center}
\feTableStart
\begin{tabularx}{\linewidth}{@{}>{\centering\arraybackslash}p{1.2cm} l Y @{}}
\toprule
\textbf{DV} & \textbf{Difficulty} & \textbf{Typical Situations} \\
\midrule
2 & Routine   & Clear intent, modest stakes, controlled environment \\
3 & Pressured & Time pressure, mild resistance, partial information \\
4 & Hard      & Hostile conditions, active opposition, precise timing \\
5+ & Extreme  & Multiple constraints, high precision, dramatic failure risk \\
\bottomrule
\end{tabularx}
\feTableEnd
\end{center}

\section{Deck Usage Reference}
\label{sec:deck-reference}
\index{Appendices!Deck Reference}

\subsection{Deck Types and Meanings}
\label{subsec:deck-types}
\index{Appendices!Deck Types}

\begin{description}
\item[\textbf{Travel Decks} (regional, 52-card)] Used for journey content and location-based adventures.\index{Decks!Travel}
\begin{itemize}
\item Spade $=$ Place/Location
\item Heart $=$ Actor/Faction
\item Club $=$ Pressure/Complication
\item Diamond $=$ Leverage/Reward
\end{itemize}

\item[\textbf{Deck of Consequences} (scene drama)] Used for immediate complications and narrative twists.\index{Decks!Consequences}
\begin{itemize}
\item Hearts $=$ Social/Emotional fallout
\item Spades $=$ Harm/Escalation
\item Clubs $=$ Material cost/Resource drain
\item Diamonds $=$ Magical/Spiritual disturbance
\end{itemize}
\end{description}

\textbf{Important:} Never mix suit meanings across decks. Travel deck suits differ from Consequences deck suits.

\subsection{Deck Usage Procedure}
\label{subsec:deck-procedure}
\index{Appendices!Deck Procedure}

After a roll generating Story Beats:
\begin{enumerate}
\item \textbf{Direct Spend}: Translate SB into immediate consequences or clock ticks.
\item \textbf{Deck Draw}: Draw up to $\min(\text{SB},\,3)$ cards and synthesize a single twist.
\item Interpret cards based on suit meanings and highest rank.
\end{enumerate}

\subsection{Rank Severity Guide}
\label{subsec:rank-severity}
\index{Appendices!Rank Severity}

\begin{description}
\item[Ace--3] Minor inconvenience or flavor complication.\index{Decks!Ranks!Minor}
\item[4--6] Moderate setback with narrative impact.\index{Decks!Ranks!Moderate}
\item[7--9] Significant consequence altering the scene.\index{Decks!Ranks!Significant}
\item[10--King] Major fallout introducing new problems or lasting effects.\index{Decks!Ranks!Major}
\end{description}

\section{Magic System Quick Reference}
\label{sec:magic-reference}
\index{Appendices!Magic Reference}

\subsection{Magic Paths Comparison}
\label{subsec:magic-paths-ref}
\index{Appendices!Magic Paths}

\begin{center}
\feTableStart
\begin{tabularx}{\linewidth}{@{}l l l l @{}}
\toprule
\textbf{Path} & \textbf{Requirements} & \textbf{Key Feature} & \textbf{Risk Type} \\
\midrule
Caster (Freeform) & Caster's Gift (6 XP) & Flexible improvisation & Backlash \\
Runekeeper (Rites) & Thiasos + Codex (6 XP) & Structured Rites & Obligation \\
Invoker (Symbols) & Patron's Symbol (4 XP) & Ritual precision & Symbol compromise \\
\bottomrule
\end{tabularx}
\feTableEnd
\end{center}

\subsection*{DV Reference Table}
\label{tab:dv-reference}
\index{Rites!DV Table}

The following table shows the resulting DV for common Obligation Costs across Spirit scores and Rite Tiers.
DV is always calculated as $\max(\text{Obligation} - \text{Spirit}, \, \text{Tier})$.

\begin{longtable}{c|cccc|c|c|c}
\toprule
\textbf{Obligation Cost} & \textbf{Spirit 0} & \textbf{Spirit 1} & \textbf{Spirit 2} & \textbf{Spirit 3--4} & \textbf{Tier 1} & \textbf{Tier 2} & \textbf{Tier 3} \\
\midrule
\endfirsthead
\toprule
\textbf{Obligation Cost} & \textbf{Spirit 0} & \textbf{Spirit 1} & \textbf{Spirit 2} & \textbf{Spirit 3--4} & \textbf{Tier 1} & \textbf{Tier 2} & \textbf{Tier 3} \\
\midrule
\endhead
\bottomrule
\endfoot
\bottomrule
\endlastfoot
1 & 1 & 1 & 1 & 1 & 1 & 2 & 3 \\
2 & 2 & 1 & 1 & 1 & 1 & 2 & 3 \\
3 & 3 & 2 & 1 & 1 & 1 & 2 & 3 \\
4 & 4 & 3 & 2 & 1 & 1 & 2 & 3 \\
5 & 5 & 4 & 3 & 2 & 1 & 2 & 3 \\
6 & 6 & 5 & 4 & 3 & 1 & 2 & 3 \\
7 & 7 & 6 & 5 & 4 & 1 & 2 & 3 \\
\end{longtable}

\paragraph{How to Read.}
\begin{itemize}
  \item Left block: DV before applying the Tier floor.
  \item Right block: the minimum DV once Tier is considered.
  \item Example: A Rite with Obligation 4, Spirit 2, Tier 2 → Base DV = 2, but Tier floor raises it to 2.
\end{itemize}

\subsection{Casting Loop Summary}
\label{subsec:casting-loop-ref}
\index{Appendices!Casting Loop}

\begin{enumerate}
\item \textbf{Channel}: Wits + Arcana roll to gather Potential.\index{Magic!Channel}
\item \textbf{Weave}: Wits + Art roll to shape spell effect.\index{Magic!Weave}
\item \textbf{Backlash}: SB spent through thematic consequences.\index{Magic!Backlash}
\end{enumerate}

\subsection{Eight Elements of Magic}
\label{subsec:elements-ref}
\index{Appendices!Magic Elements}

\begin{description}
\item[Earth] Solidity, stability, foundation.\index{Magic!Elements!Earth}
\item[Fire] Energy, transformation, destruction.\index{Magic!Elements!Fire}
\item[Air] Movement, speed, freedom.\index{Magic!Elements!Air}
\item[Water] Fluidity, healing, adaptability.\index{Magic!Elements!Water}
\item[Fate] Destiny, inevitability, causality.\index{Magic!Elements!Fate}
\item[Life] Vitality, creation, growth.\index{Magic!Elements!Life}
\item[Luck] Chance, unpredictability, probability.\index{Magic!Elements!Luck}
\item[Death/Dreams] Endings, thresholds, subconscious.\index{Magic!Elements!Death}
\end{description}

\section{Combat and Conflict Reference}
\label{sec:combat-reference}
\index{Appendices!Combat Reference}

\subsection{Position States}
\label{subsec:position-ref}
\index{Appendices!Position States}

\begin{description}
\item[Dominant] Advantageous position, minor consequences.\index{Combat!Controlled}
\item[Controlled] Standard situation, moderate consequences.\index{Combat!Controlled}
\item[Desperate] Disadvantaged, severe consequences.\index{Combat!Desperate}
\end{description}

\subsection{Harm Levels and Effects}
\label{subsec:harm-ref}
\index{Appendices!Harm Reference}

\begin{center}
\feTableStart
\begin{tabularx}{\linewidth}{@{}l l l l @{}}
\toprule
\textbf{Harm Level} & \textbf{SB Generation} & \textbf{Penalty} & \textbf{Recovery} \\
\midrule
Minor & 1 SB on next 2 rolls & $-1$ die to related actions & Rest or basic care \\
Moderate & 1 SB on next roll & $-1$ die to most actions & Medical treatment \\
Severe & 2 SB on next roll & $-2$ dice to most actions & Extended care \\
Critical & 3 SB on next roll & Incapacitated & Major intervention \\
\bottomrule
\end{tabularx}
\feTableEnd
\end{center}

\subsection{Range Bands}
\label{subsec:range-bands-ref}
\index{Appendices!Range Bands}

\begin{description}
\item[Close] Arm's length, grappling distance.\index{Combat!Close Range}
\item[Near] Same room or immediate area.\index{Combat!Near Range}
\item[Far] Visible but not immediately reachable.\index{Combat!Far Range}
\item[Absent] Off-screen or out of current scene.\index{Combat!Absent Range}
\end{description}

\subsection{Movement Actions}
\label{subsec:movement-ref}
\index{Appendices!Movement}

\begin{itemize}
\item \textbf{1 Move}: Shift one range band (Close$\leftrightarrow$Near or Near$\leftrightarrow$Far).
\item \textbf{Dash Action}: Shift two bands in one action.
\item \textbf{Disengage}: Test to leave Close range when threatened.
\item \textbf{Sprint}: Rapid movement across the battlefield.
\end{itemize}

\section{Resource Management Reference}
\label{sec:resource-reference}
\index{Appendices!Resource Reference}

\subsection{Story Beat Economy}
\label{subsec:sb-economy-ref}
\index{Appendices!Story Beats}

\begin{center}
\feTableStart
\begin{tabularx}{\linewidth}{@{}>{\centering\arraybackslash}p{1.8cm} l Y @{}}
\toprule
\textbf{SB Cost} & \textbf{Effect Scale} & \textbf{Typical Effects} \\
\midrule
1 SB & Minor pressure & Noise, trace, time loss, +1 Supply segment \\
2 SB & Moderate setback & Alarm, lose position/cover, lesser foe appears \\
3 SB & Serious trouble & Reinforcements, key gear breaks, major complication \\
4+ SB & Major turn & Trap springs, authority arrives, scene shifts dramatically \\
\bottomrule
\end{tabularx}
\feTableEnd
\end{center}

\subsection{Boon Usage Guide}
\label{subsec:boon-usage-ref}
\index{Appendices!Boons}

\begin{center}
\feTableStart
\begin{tabularx}{\linewidth}{@{}l l Y @{}}
\toprule
\textbf{Boon Cost} & \textbf{Effect} & \textbf{Limitations} \\
\midrule
1 Boon & Re-roll one die            & Once per action \\
1 Boon & Activate on-screen Asset   & Plausibility test required \\
1 Boon & Improve Position by 1 step & One step maximum per action \\
2 Boons & Convert to 1 XP           & Once per session; max 2 XP \\
Variable & Power Rites/Abilities    & As specified \\
\bottomrule
\end{tabularx}
\feTableEnd
\end{center}

\textbf{Boon Limits:}
\begin{itemize}
\item Hold maximum of 5 Boons at any time.
\item Trim to 2 Boons at scene endings.
\item Maximum 2 Boons from failures per character per scene.
\item Conversion: 2 Boons $=$ 1 XP (max 2 XP per session).
\end{itemize}

\subsection{Supply Clock States}
\label{subsec:supply-ref}
\index{Appendices!Supply Clock}

\begin{description}
\item[\textbf{Full Supply} (0)] No penalties; well-equipped.\index{Supply!Full}
\item[\textbf{Low Supply} (2)] Minor narrative complications.\index{Supply!Low}
\item[\textbf{Dangerously Low} (3)] Each character gains 1 Fatigue.\index{Supply!Dangerously Low}
\item[\textbf{Out of Supply} (4)] Severe penalties; starvation risk.\index{Supply!Out of Supply}
\end{description}

\section{Travel and Exploration Reference}
\label{sec:travel-reference}
\index{Appendices!Travel Reference}

\subsection{Travel Clock Sizes}
\label{subsec:travel-clocks-ref}
\index{Appendices!Travel Clocks}

\begin{description}
\item[4 segments] Short, straightforward journeys.\index{Travel!Clocks!4-segment}
\item[6 segments] Standard travel legs.\index{Travel!Clocks!6-segment}
\item[8 segments] Extended or complex journeys.\index{Travel!Clocks!8-segment}
\item[10 segments] Epic or highly dangerous travel.\index{Travel!Clocks!10-segment}
\end{description}

\subsection{Card Draw Procedures}
\label{subsec:card-draw-ref}
\index{Appendices!Card Draw}

\textbf{Quick Hook (2 cards):}
\begin{itemize}
\item Draw one Spade (place) and one Heart (actor).
\item Use higher rank to set clock size.
\end{itemize}

\textbf{Full Seed (4 cards):}
\begin{itemize}
\item Draw until one card of each suit appears.
\item Spade $=$ location, Heart $=$ faction, Club $=$ pressure, Diamond $=$ leverage.
\item Highest rank sets main clock size.
\end{itemize}

\section{Character Advancement Guide}
\label{sec:advancement-reference}
\index{Appendices!Advancement}

\subsection{Reputation Tiers}
\label{subsec:reputation-tiers-ref}
\index{Appendices!Reputation Tiers}

\begin{description}
\item[Tier I -- Rookie (0--40 XP)] Local reputation; prestige locked.\index{Reputation Tiers!Rookie}
\item[Tier II -- Seasoned (41--90 XP)] Regional notice; prestige may unlock.\index{Reputation Tiers!Seasoned}
\item[Tier III -- Veteran (91--150 XP)] National influence; second follower suggested.\index{Reputation Tiers!Veteran}
\item[Tier IV -- Paragon (151--220 XP)] Movers and shakers; rivals emerge.\index{Reputation Tiers!Paragon}
\item[Tier V -- Mythic (221+ XP)] Legendary status; kingdoms respond.\index{Reputation Tiers!Mythic}
\end{description}

\subsection{Player Archetypes}
\label{subsec:archetypes-ref}
\index{Appendices!Player Archetypes}

\begin{description}
\item[Solo] 70--90\% self investment; minimal followers/assets.\index{Character Build!Solo}
\item[Mixed] 50--65\% self; balanced with followers/assets.\index{Character Build!Mixed Player}
\item[Mastermind] 25--40\% self; focuses on networks and followers.\index{Character Build!Mastermind}
\end{description}

\section{Safety and Inclusivity}

\subsection{Content Warnings System}

\subsubsection{X-Card Implementation}

\textbf{Basic Functionality}:
\begin{itemize}
\item \textbf{X-Card}: "I'm not comfortable with this content. Let's change direction."
\item \textbf{O-Card}: "I'm excited about this content. Let's lean into it."
\item \textbf{N-Card}: "I need a break from this content. Let's pause."
\end{itemize}

\textbf{Session Integration}:
\begin{enumerate}
\item Introduce X-Card during Session Zero with practice scenarios
\item Keep X-Cards visible and accessible during all sessions
\item Model appropriate X-Card usage as the GM
\item Address any misuse or abuse of the system immediately
\end{enumerate}

\subsubsection{Script Change Integration}

\textbf{Core Questions}:
\begin{itemize}
\item \textbf{Pause}: "Are you okay with the content right now?"
\item \textbf{Rewind}: "Would you like to go back and change something?"
\item \textbf{Fast Forward}: "Would you like to skip ahead to something else?"
\end{itemize}

\textbf{Application Timing}:
\begin{itemize}
\item Before potentially sensitive content is introduced
\item During scenes that might trigger discomfort
\item After sessions to gather feedback on content experience
\end{itemize}

\subsection{Inclusive Worldbuilding}

\subsubsection{Representation Guidelines}

\textbf{Cultural Sensitivity}:
\begin{itemize}
\item Research real-world cultures thoroughly before incorporating elements
\item Avoid stereotypes and oversimplifications of cultural practices
\item Consult with individuals from relevant backgrounds when possible
\item Create fictional cultures inspired by but distinct from real-world sources
\end{itemize}

\textbf{Diverse Character Creation}:
\begin{itemize}
\item Provide character options that don't rely on real-world cultural markers
\item Include mechanical benefits that reflect diverse backgrounds and experiences
\item Create backstory elements that celebrate different life experiences
\item Avoid mechanics that reinforce harmful stereotypes
\end{itemize}

\subsubsection{Accessibility Considerations}

\textbf{Physical Accessibility}:
\begin{itemize}
\item Provide digital versions of all materials for screen readers
\item Use high-contrast colors and clear fonts in visual materials
\item Offer alternative formats for dice-based mechanics
\item Minimize requirements for fine motor skills in gameplay
\end{itemize}

\textbf{Cognitive Accessibility (continued)}:
\begin{itemize}
\item Offer simplified versions of complex mechanics
\item Break down rules into step-by-step processes
\item Provide examples for abstract concepts
\item Create quick-reference charts for common actions
\end{itemize}

\subsection{Conflict Resolution}

\subsubsection{Preventive Measures}

\textbf{Clear Communication Protocols}:
\begin{itemize}
\item Establish regular check-ins for player satisfaction
\item Create anonymous feedback channels for concerns
\item Define GM authority boundaries and player agency limits
\item Set expectations for group decision-making processes
\end{itemize}

\textbf{Session Structure for Harmony}:
\begin{itemize}
\item Begin sessions with brief mood check-ins
\item Include structured time for character interaction
\item Balance individual spotlight time with group activities
\item End sessions with reflection on positive moments
\end{itemize}

\subsubsection{Active Mediation Techniques}

\textbf{When Conflicts Arise}:
\begin{enumerate}
\item \textbf{Immediate Pause}: Stop gameplay to address the issue
\item \textbf{Private Discussion}: Separate conflicting parties if necessary
\item \textbf{Neutral Facilitation}: Focus on understanding rather than determining fault
\item \textbf{Collaborative Solution}: Work together to find mutually acceptable resolutions
\item \textbf{Documentation}: Record agreements to prevent future misunderstandings
\end{enumerate}

\textbf{Common Conflict Types and Solutions}:
\begin{itemize}
\item \textbf{Rules Disputes}: Establish GM ruling as final for session, research and discuss later
\item \textbf{Character Interference}: Create clear boundaries and consequences for unwanted intrusions
\item \textbf{Play Style Mismatches}: Find compromise activities that satisfy different preferences
\item \textbf{Narrative Control Conflicts}: Establish clear division between player agency and GM narrative authority
\end{itemize}