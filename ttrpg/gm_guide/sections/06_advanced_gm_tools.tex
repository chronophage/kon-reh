\chapter{Advanced GM Techniques}\index{advanced techniques}

In \textbf{Fate's Edge}, as the campaign deepens and the stakes rise, the GM must evolve from storyteller to \textbf{architect of tension}. This chapter explores advanced techniques for managing complex scenes, faction interplay, and custom content creation. These tools will help you keep the world dynamic, the choices meaningful, and the consequences \textbf{echoing through the ages}.

\section*{Story Beat Management}\index{Story Beats!management}

The GM should manage Story Beat (SB) spending to maintain dramatic tension while preserving player agency and game flow. SB spending scales with character tier but is subject to hard limits to ensure playability and prevent narrative overload.

\subsection*{Core Principles}\index{Story Beats!core principles}
\begin{itemize}
    \item \textbf{Narrative Coherence}: All SB spends within a scene should connect thematically, creating a cohesive story rather than random setbacks
    \item \textbf{Player Agency}: Complications create interesting choices, not insurmountable obstacles---always provide resolution paths
    \item \textbf{Progressive Escalation}: Higher tier characters naturally attract greater consequences that match their growing influence
    \item \textbf{Resolution Paths}: Every complication thread should have potential resolution through player action and creativity
\end{itemize}

\subsection*{Spending Formula}\index{Story Beats!spending formula}
\textbf{Base SB = 4 + Character Tier}\index{Character Tier}
\begin{itemize}
    \item \textbf{Tier I (Rookie 0-40 XP)}: 5 SB base---local consequences, manageable threats
    \item \textbf{Tier II (Seasoned 41-90 XP)}: 6 SB base---regional impact, organized opposition
    \item \textbf{Tier III (Veteran 91-150 XP)}: 7 SB base---national consequences, strategic rivals
    \item \textbf{Tier IV (Paragon 151-220 XP)}: 8 SB base---continental scale, legendary challenges
    \item \textbf{Tier V (Mythic 221+ XP)}: 9 SB base---world-shaping events, mythic adversaries
\end{itemize}

\subsection*{Hard Limits}\index{Story Beats!hard limits}
\begin{itemize}
    \item \textbf{Standard Scenes}: Maximum 12 SB spending---maintains pace without overwhelming
    \item \textbf{Climactic Scenes}: Maximum 16 SB spending---allows for dramatic tension peaks
    \item \textbf{Active Threads}: Maximum (Tier + 1) concurrent threads---prevents narrative sprawl
    \item \textbf{Session Budget}: Maximum 20 SB total per session---ensures sustainable challenge
\end{itemize}

\subsection*{Banked SB Integration}\index{Story Beats!banked integration}
Banked SB from character complications count toward scene spending limits rather than adding to available SB. This prevents exponential complication stacking while honoring narrative debt from previous choices and established character backgrounds.

\subsection*{Thread Management}\index{threads!management}
Complication threads follow a natural escalation pattern that creates satisfying narrative arcs:

\begin{fatebox}[Complication Thread Escalation Pattern]
\begin{tabularx}{\textwidth}{lX}
\toprule
\textbf{Thread Level} & \textbf{SB Escalation and Narrative Impact} \\
\midrule
First Exposure & 1-2 SB (Minor inconvenience that introduces the complication) \\
Second Occurrence & 2-4 SB (Moderate setback that develops the thread) \\
Third Strike & 3-6 SB (Major consequence that brings the thread to climax) \\
Resolution & Thread concludes with narrative payoff and character growth \\
\bottomrule
\end{tabularx}
\end{fatebox}

\begin{fatebox}[Story Beat Management by Tier]
\begin{tabularx}{\textwidth}{lXXXXX}
\toprule
\textbf{Tier} & \textbf{Base SB} & \textbf{Max Threads} & \textbf{Scene Cap} & \textbf{Climax Cap} & \textbf{Session Budget} \\
\midrule
I (0-40 XP) & 5 SB & 2 threads & 12 SB & 16 SB & 20 SB \\
II (41-90 XP) & 6 SB & 3 threads & 12 SB & 16 SB & 20 SB \\
III (91-150 XP) & 7 SB & 4 threads & 12 SB & 16 SB & 20 SB \\
IV (151-220 XP) & 8 SB & 5 threads & 12 SB & 16 SB & 20 SB \\
V (221+ XP) & 9 SB & 6 threads & 12 SB & 16 SB & 20 SB \\
\bottomrule
\end{tabularx}
\end{fatebox}

\begin{fatebox}[Complication Spending Safety Guidelines]
\begin{tabularx}{\textwidth}{lX}
\toprule
\textbf{Scenario} & \textbf{Guidance and Best Practices} \\
\midrule
Standard Scenes & Spend 50-75\% of available SB budget; preserve some for escalation and player adaptation \\
Climactic Scenes & May use full SB allocation; ensure resolution opportunities and meaningful choices remain \\
Teaching Moments & Explicit player consent required; time-limited complications; thorough debrief afterward \\
New Players & Reduce SB spending by 25-50\%; focus on clear, actionable complications with obvious solutions \\
Grimdark Mode & Reserved for veteran groups; requires explicit session zero discussion; safety tools always active \\
\bottomrule
\end{tabularx}
\end{fatebox}

\section*{Mastering the Deck of Consequences}\index{Deck of Consequences}

The \textbf{Deck of Consequences} is more than a randomizer---it is a \textbf{thematic engine} that externalizes risk and ensures that setbacks feel consistent, fair, and deeply integrated with your campaign's unique atmosphere.

\subsection*{Two Deck Systems (Compatibility)}\index{Deck of Consequences!two deck systems}

Fate's Edge uses two distinct card tools that serve different narrative purposes:

\begin{fatebox}[Deck System Comparison]
\begin{tabularx}{\textwidth}{lX}
\toprule
\textbf{Deck Type} & \textbf{Purpose and Suit Meanings} \\
\midrule
Travel Decks (regional, 52-card) & Spade=Place, Heart=Actor, Club=Pressure, Diamond=Leverage. Powers journeys and geographic challenges \\
Deck of Consequences (scene drama) & Hearts=social fallout, Spades=harm/escalation, Clubs=material cost, Diamonds=magical/spiritual disturbance \\
\bottomrule
\end{tabularx}
\end{fatebox}

\textit{Critical Guidance:} Never mix suit meanings across decks. When a rule references ``Spade/Club/Diamond,'' it means \emph{Travel Deck}. When it says ``Hearts/Spades/Clubs/Diamonds,'' it means \emph{Consequences Deck}.

\subsection*{When to Draw}\index{Deck of Consequences!when to draw}

After a roll generates Story Beats, the GM faces a strategic choice:

\begin{itemize}
    \item \textbf{Direct Spend}: Translate SB into immediate consequences/rail ticks---fast, reliable, maintains pacing
    \item \textbf{Deck Draw}: Draw up to \textbf{min(SB, 3)} cards and \textbf{synthesize a single twist} guided by suit and highest rank---rich, thematic, introduces novelty
\end{itemize}

Never do both for the same roll. If a drawn card contradicts established fiction, reinterpret creatively or redraw to maintain thematic consistency.

\subsection*{Structure of the Deck}\index{Deck of Consequences!structure}

\begin{fatebox}[Deck of Consequences Structure Guide]
\begin{tabularx}{\textwidth}{lX}
\toprule
\textbf{Component} & \textbf{Description and Application} \\
\midrule
Hearts Suit & Emotional, social, or relational fallout---betrayals, misunderstandings, emotional wounds \\
Spades Suit & Harm, danger, or escalation of conflict---injuries, reinforcements, tactical disadvantages \\
Clubs Suit & Resource strain, economic or material cost---broken gear, lost supplies, financial setbacks \\
Diamonds Suit & Magical, spiritual, or cosmic disturbances---backlash, omens, supernatural complications \\
Ace-3 (Minor) & Inconvenience or flavor complication that adds texture without major impact \\
4-6 (Moderate) & Setback with narrative teeth that requires player attention and response \\
7-9 (Significant) & Consequence altering the course of action with lasting implications \\
10-King (Major) & Major fallout introducing new problems or lasting scars that change the story \\
\bottomrule
\end{tabularx}
\end{fatebox}

\begin{fatebox}[Complication Application Methods: GM Decision Guide]
\begin{tabularx}{\textwidth}{lXXX}
\toprule
\textbf{Method} & \textbf{When to Use} & \textbf{Benefits} & \textbf{Typical Session Use} \\
\midrule
Direct Spend (70\%) & Routine actions, combat, quick resolution & Fast resolution, consistent pacing, reliable complications & Investigation, travel, standard challenges \\
Deck Draw (30\%) & Major revelations, character moments, climaxes & Thematic richness, unique complications, player surprise & Plot twists, discoveries, emotional scenes \\
\bottomrule
\end{tabularx}
\end{fatebox}

\begin{fatebox}[Session Phase Application Guidelines]
\begin{tabularx}{\textwidth}{lXXX}
\toprule
\textbf{Campaign Phase} & \textbf{Recommended Ratio} & \textbf{Resolution Time} & \textbf{Rationale} \\
\midrule
Early Game (Exploration) & 80/20 & 3 seconds & Quick resolution maintains investigation flow and momentum \\
Mid Game (Development) & 70/30 & 5 seconds & Balanced approach supports rising tension with meaningful pivots \\
Late Game (Climax) & 50/50 to 40/60 & 8 seconds & Maximum impact complications for story resolution \\
\bottomrule
\end{tabularx}
\end{fatebox}

\section*{Advanced Travel and Exploration}\index{travel}\index{exploration}

Travel in Fate's Edge is not a downtime skip---it is a \textbf{narrative layer} filled with tension, discovery, and risk that reveals the world's character through every mile crossed.

\subsection*{Core Travel Procedure}\index{travel!core procedure}

For each leg of a journey, draw 3--4 cards using the decks for your destination and controlling authority:

\begin{itemize}
    \item \textbf{Spade} from the destination deck: sets the scene (place)---the physical and cultural landscape
    \item \textbf{Heart} from the destination deck: introduces the local actor or faction---who they meet and why it matters
    \item \textbf{Club} from the Wilds or destination: brings pressure---what challenges the journey itself
    \item \textbf{Diamond} from the authority that gates the route: papers, escorts, rights, or exceptions---the bureaucratic landscape
\end{itemize}

Set a travel clock by the highest rank:
\begin{itemize}
    \item \textbf{2--5} → 4 segments (brief, intense journeys)
    \item \textbf{6--10} → 6 segments (standard expedition length)
    \item \textbf{J/Q/K} → 8 segments (extended, epic travels)
    \item \textbf{Ace} → 10 segments (campaign-defining voyages)
\end{itemize}

\textbf{Example}: Traveling the \textbf{Aelerian Passes} in deep winter, the PCs draw: Spade (Avalanche gallery---treacherous narrow path), Heart (Geometer---mapmaker with secret knowledge), Club (Engineer requisition---military demands), Diamond (Underway Pass---ancient right of passage). Clock: 8 segments. On a failed navigation roll, the GM spends SB to trigger a rockslide---Hazard +2 that threatens to bury the path entirely.

\section*{Faction Dynamics and Grand Strategy}\index{factions}

Factions are \textbf{living entities} with goals, rivals, and shifting loyalties. They are not static backdrops---they are \textbf{active players in the story} whose movements shape the campaign's grand narrative.

\subsection*{Creating Memorable Factions}\index{factions!creating}

Each faction should have distinct personality and concrete capabilities:

\begin{fatebox}[Faction Creation Template]
\begin{tabularx}{\textwidth}{lX}
\toprule
\textbf{Element} & \textbf{Development Guidelines} \\
\midrule
Core Motive & What they fundamentally want---territory, ideology, survival, power, knowledge \\
Key Figures & Who leads or represents them---names, personalities, ambitions, vulnerabilities \\
Resources & What they can bring to bear---military, economic, social, magical assets \\
Weaknesses & What makes them vulnerable---internal divisions, external pressures, resource limitations \\
Relationship Map & How they connect to other factions---allies, rivals, neutrals, complicated histories \\
\bottomrule
\end{tabularx}
\end{fatebox}

\subsection*{Faction Clocks and Grand Strategy}\index{factions!clocks}

Track factional pressure with clocks that represent their changing fortunes:

\begin{itemize}
    \item \textbf{Rising Influence} (6): Gaining power, allies, or territory---momentum is building
    \item \textbf{Internal Strife} (6): Leadership challenged, morale low---fractures appear
    \item \textbf{Public Scandal} (4): Reputation damaged, support wanes---trust evaporates
    \item \textbf{Strategic Initiative} (8): Controlling the narrative and setting terms of engagement
\end{itemize}

\textbf{Example}: The \textbf{Viterra Dawn Knights} gain Rising Influence as they rally to the new Queen's banner---but suffer Internal Strife as old commanders resist her modernizing reforms, creating tension between tradition and progress.

\section*{Creating Custom Content and House Rules}\index{custom content}

Fate's Edge thrives on \textbf{player agency} and \textbf{world customization}. When designing new Talents, Assets, or Prestige Abilities, follow these principles to maintain balance while encouraging creativity.

\subsection*{Designing Balanced Talents}\index{Talents!designing}

\begin{fatebox}[Talent Design Guidelines by Tier]
\begin{tabularx}{\textwidth}{lX}
\toprule
\textbf{Tier} & \textbf{Design Principles and XP Cost Guidelines} \\
\midrule
General Talents (2-4 XP) & Simple benefits that enhance core capabilities without complexity \\
Cultural Talents (4-6 XP) & Thematic abilities tied to specific backgrounds or training \\
Prestige Abilities (6+ XP) & Campaign-defining powers requiring significant investment and narrative milestones \\
\bottomrule
\end{tabularx}
\end{fatebox}

\subsection*{Example Talent Designs}

\begin{itemize}
    \item \textbf{Battle Instincts} (6 XP): Once per scene, re-roll a failed defense roll---honed reflexes saving from certain disaster
    \item \textbf{Silver Tongue} (4 XP): Gain +1 die when persuading or deceiving through speech---words that charm and manipulate
    \item \textbf{Stone-Sense} (Dwarves, 5 XP): Detect flaws in stone or earth; gain +1 die on Engineering or Craft rolls underground---ancestral connection to the deep places
    \item \textbf{Blood Memory} (Ykrul, 5 XP): After a battle, meditate to gain one temporary Skill die reflecting a foe's tactics---learning through spilled blood
\end{itemize}

\subsection*{Designing Meaningful Assets}\index{Assets!designing}

\begin{fatebox}[Asset Design Framework]
\begin{tabularx}{\textwidth}{lX}
\toprule
\textbf{Asset Tier} & \textbf{Scope and Narrative Impact} \\
\midrule
Minor (4 XP) & Local influence---safehouse, petty title, small shop with limited reach \\
Standard (8 XP) & Regional impact---spy ring, charter, workshop with measurable influence \\
Major (12 XP) & National scale---fortress lease, mercantile network, institution with lasting presence \\
\bottomrule
\end{tabularx}
\end{fatebox}

Each Asset should include:
\begin{itemize}
    \item \textbf{Activation Cost}: Typically 1 Boon for on-screen effect
    \item \textbf{Scope}: Clear boundaries on what it can plausibly accomplish
    \item \textbf{Fictional Hook}: Why it exists in the world and how it was obtained
    \item \textbf{Condition Track}: How it degrades or improves with use and attention
\end{itemize}

\section*{Running Complex Scenarios with Confidence}\index{complex scenarios}

\subsection*{Heists and Infiltration}\index{heists}

\begin{fatebox}[Heist Scenario Framework]
\begin{tabularx}{\textwidth}{lX}
\toprule
\textbf{Element} & \textbf{Implementation Guidelines} \\
\midrule
Positioning & Controlled entries through planning, distractions creating opportunities, asset use for specialized access \\
Social Rails & Curfew (time pressure), Crowd (witness management), Sanctity (cultural restrictions) \\
Physical Rails & Hazard (environmental dangers), Hunt (pursuit escalation), Escape (exit strategy) \\
GM Philosophy & Let players plan thoroughly but make the world react realistically---guards change, nobles arrive early, systems update \\
\bottomrule
\end{tabularx}
\end{fatebox}

\subsection*{Mass Combat and Warfare}\index{combat!mass}

\begin{fatebox}[Mass Combat Management System]
\begin{tabularx}{\textwidth}{lX}
\toprule
\textbf{Component} & \textbf{Handling Method} \\
\midrule
Follower Units & Cap 5 followers represent military forces with distinct capabilities and morale \\
War Clocks & Supply Lines (8), Morale (6), Strategic Position (8)---track strategic realities \\
Command Actions & Leaders coordinate multiple units through decisive action and tactical insight \\
Environmental Factors & Weather, terrain, and time of day significantly impact large-scale engagements \\
\bottomrule
\end{tabularx}
\end{fatebox}

\subsection*{Political Intrigue and Social Conflict}\index{political intrigue}

\begin{itemize}
    \item \textbf{Leverage}: Diamonds and social rails determine influence in courtly settings
    \item \textbf{Allies and Rivals}: Represented by Assets and Followers with their own agendas
    \item \textbf{Public Image}: Tied directly to Mandate and Crisis clocks---reputation is currency
    \item \textbf{Information Economy}: Secrets become tangible assets with measurable value
\end{itemize}

\section*{Advanced Magic and Supernatural Challenges}\index{magic!advanced}

\subsection*{Magic Duels and Arcane Confrontations}\index{combat!magic duels}

High-stakes magical combat requires special considerations that honor both the power and the peril of arcane arts:

\begin{fatebox}[Magic Duel Framework]
\begin{tabularx}{\textwidth}{lX}
\toprule
\textbf{Element} & \textbf{Special Considerations} \\
\midrule
Counterspelling & Interrupting opponent's Casting Loop requires precise timing and significant risk \\
Backlash Cascade & Multiple casters generate SB that can create compound complications \\
Environmental Magic & Terrain-altering spells change the battlefield with lasting consequences \\
Elemental Opposition & Fire vs Water, Earth vs Air, Fate vs Luck---opposites create dramatic tension \\
\bottomrule
\end{tabularx}
\end{fatebox}

\subsection*{Supernatural Investigations}\index{supernatural investigations}

When the party confronts mysteries beyond mortal understanding:

\begin{itemize}
    \item \textbf{Clue-Based Progression}: Information becomes the primary resource
    \item \textbf{Sanity and Corruption}: Exposure to the unnatural has measurable effects
    \item \textbf{Ritual Timelines}: Some threats operate on schedules beyond human comprehension
    \item \textbf{Reality Bleed}: The supernatural leaks into the mundane world with subtle signs
\end{itemize}

% !TEX root = resource_guide_main.tex
% Resource Guide Subsection: Disabling Magical Traps (Wards)

\subsection{Disabling Magical Traps (Wards)}\label{subsec:disabling-wards}
\index{Wards}\index{Traps}\index{DV}\index{Position}\index{Clocks}\index{Story Beats}

Wards are traps encoded in magic: a binding, sensor, or anchored effect. Treat them like any obstacle: \textbf{state the fiction}, \textbf{set Position}, and \textbf{resolve against DV}. This subsection expands the SRD example with GM-facing tools: how to set DV from fiction, what good/bad results look like, and how to keep the scene moving.

\paragraph{Structure (at the table).}
\begin{enumerate}[label=\textbf{Step~\arabic*:}, leftmargin=2.1em]
\item \textbf{Name the Ward.} What does it sense (heat, oath-breaking, Patron presence)? What does it do (alarm, bind, blast, invert gravity)? Where is the \emph{anchor} (sigil, keystone, idol, threshold)?
\item \textbf{Reveal Handles.} Offer at least two fictional levers: the arcane pattern (\emph{Arcana}), the mechanism/anchor (\emph{Tinker}), and the approach route (\emph{Agility/Move}). Tools, components, or Patron symbols count as handles.
\item \textbf{Set Position \textrightarrow{} DV.} Use the ladder as defaults; shift for scale/urgency.
\begin{itemize}
\item \textbf{Controlled} (time, tools, clear access): \textbf{DV2}.
\item \textbf{Risky} (pressure, partial access): \textbf{DV3}.
\item \textbf{Desperate} (activating, hostile field): \textbf{DV~4–5+}.
\end{itemize}
\item \textbf{Declare Stakes.} What happens on Partial/Miss? Name the \emph{Ward Clock} and the \emph{Alert Clock} up front if relevant.
\end{enumerate}

\paragraph{Common Ward Types & Anchors.}
\begin{tabularx}{\linewidth}{>{\bfseries}l >{\raggedright}X >{\raggedright}X}
\toprule
Type & Trigger & Anchor Examples \
\midrule
Sentry & Crossing a threshold, line of sight, sound cadence & Lintel sigil, eye-gem, bell filament \
Binding & Handling an item, touching a seal, oath breach & Wax seal, chain knot, iron brand \
Punitive & Trespass + time window, named rival, taboo action & Idol tooth, ash sigil under rug, curse knot \
Displacement & Path inversion, gravity twist, dream-step & Floor mandala, mirrored post, sleep pin \
Anti-Magic & Casting in zone, symbol presence, boon use & Null plate, cold-iron frame, void ampoule \
\bottomrule
\end{tabularx}

\paragraph{Approaches & Aids (examples).}
\begin{itemize}
\item \textbf{Wits + Arcana:} analyze glyph grammar; ground the loop with salt/ink; counter-frequency hum.
\item \textbf{Wits + Tinker:} wedge the keystone; snip hair-thin filament; re-route power to a safe sink.
\item \textbf{Body + Agility:} thread the safe arc; pin the pressure plate; time the chime cadence.
\item \textbf{Teamwork:} one character steadies Position with tools while another rolls; assistants may accept \textbf{1~SB} to grant +1d once.
\item \textbf{Talents/Tools:} lockpicks, ward-ink, tuning forks, mirror-shards; Patron symbols might improve Position if thematically aligned.
\end{itemize}

\paragraph{Clocks that Matter.}
\begin{description}[leftmargin=1.6em]
\item[\textsc{Ward Integrity} (4):] how close the binding is to disarming \emph{or} rupturing. On Partials, tick 1; on Misses, tick 2.
\item[\textsc{Alerted Attention} (4):] nearby guardians, spirits, or owners notice tampering.
\item[\textsc{Escalation} (4/6):] punitive effects ramp (more Harm, broader area, additional tags).
\end{description}

\paragraph{Outcomes Palette (use or adapt).}
\begin{itemize}
\item \textbf{Strong Hit:} Ward disabled/redirected cleanly; optionally \emph{bank a tell} (learn its signature for later).
\item \textbf{Mixed:} Suppressed but \emph{unstable}; pick one: tick \textsc{Alerted Attention}~+1, apply minor Backlash by element/Patron, or leave a trace (future complication).
\item \textbf{Miss:} Ward holds or triggers; GM may (a) advance \textsc{Ward Integrity} toward rupture, (b) trigger a reduced effect now, (c) spend SB to introduce a hard complication.
\end{itemize}

\paragraph{Counter-Ward (Arcana Focus; optional).}
On a Success, choose one: (1) \textbf{Suppress} for a scene; (2) \textbf{Disarm} permanently (mark +1 Obligation or expend rare material); (3) \textbf{Invert} (redirect first trigger to a designated sink; start \textsc{Unintended Echo}~1/4).

\paragraph{Difficulty Dials.}
\begin{itemize}
\item \textbf{Scale:} Large areas or ancient wards add +1 DV or an extra clock.
\item \textbf{Haste:} Rushed attempts worsen Position one step.
\item \textbf{Right of Names:} Knowing the warder’s true name improves Position one step or grants +1d.
\item \textbf{Wrong Patron:} Symbols from an opposed Patron impose –1d or +1 DV (GM call).
\end{itemize}

\paragraph{Fiction First Hooks.}
\begin{itemize}
\item Show sensory tells (ozone, cold breath, hair standing) before rolls.
\item On Partials, change the room: new cover lines, heat shimmer, dead zones for boons.
\item On Miss, give a choice: suffer a lesser effect now to avoid full trigger later, or press on and risk rupture.
\end{itemize}

\paragraph{Worked Mini-Example.}
\emph{Vault Sentry (DV 3, Risky).} Anchor: eye-gem in an idol. The crew uses Wits+Tinker with ward-ink and a mirror shard. Mixed result: \textsc{Ward Integrity}+1, \textsc{Alerted Attention}+1, but the beam is shunted for this scene. Later a rival recognizes the mirror signature—payoff for the trace left behind.

% !TEX root = srd_main.tex
% SRD Insert: Optional Player-Character Death Rules

\section{Optional Player-Character Death}\label{sec:pc-death-optional}
\index{Death}\index{Last Stand}\index{Death Clock}\index{Patron’s Claim}\index{Inheritance}\index{Resurrection}

By default, Fate’s Edge treats death as rare and dramatic. Use any of the following modules (singly or in combination) to tune how lethal your table feels. All options respect \textbf{Story Beats (SB)} and \textbf{Obligation} as core currencies.

\subsection*{Baseline (Default)}\label{subsec:death-baseline}
PCs do not die on ordinary failures. Instead, they suffer \textbf{Harm}, Conditions, lost opportunities, or narrative costs. Death only occurs when a rule below is in play or the table agrees a scene warrants it.

\subsection*{Option A — Severe Harm Death}\label{subsec:death-severe}
\textbf{Trigger.} When a PC would take a \textbf{third} instance of Severe Harm (or escalate past the top of your harm track), they instead face death.\par
\textbf{Stave It Off.} The player may avoid death by choosing one: (1) mark \textbf{2 SB} \emph{and} take a \textsc{Maimed} permanent Condition, (2) accept a \textbf{Patron’s Claim} (see OptionD), or (3) convert the blow into a \textbf{Last Stand} (OptionC).

\subsection*{Option B — Death Clock}\label{subsec:death-clock}
\textbf{Trigger.} Catastrophic consequences (falls, crushes, mortal wounds) fill a named \textbf{Death Clock} (4 or 6 ticks).\par
\textbf{While Ticking.} Actions that stabilize reduce the clock; taking further punishment advances it.\par
\textbf{When Full.} The character dies unless one of the following occurs immediately: spend \textbf{2 SB} to hold at full (buy a single action), accept a \textbf{Patron’s Claim}, or another PC succeeds at \emph{Pull From the Brink} (risky, effect = clock –2 on success).

\subsection*{Option C — Last Stand}\label{subsec:last-stand}
\textbf{Trigger.} On lethal harm or a full Death Clock, the player may declare a \textbf{Last Stand}.\par
\textbf{Effect.} For the remainder of the scene, the PC acts with \textbf{+1 effect} and ignores new Harm. Each action automatically creates \textbf{1 SB}. When the scene ends, the character \textbf{dies} unless a \textbf{miracle} is secured (Patron rite, relic, or equivalent).

\subsection*{Option D — Patron’s Claim}\label{subsec:patron-claim}
\textbf{Trigger.} On death, the PC’s Patron (or a circling power) intervenes.\par
\textbf{Bargain.} The GM offers 1–3 \non-negotiable terms (e.g., \emph{Obligation +2}, \emph{forfeit a Gift}, \emph{become a vessel for a season}). If accepted, the PC lives; mark the costs immediately and record the \textbf{Claim} as an ongoing front. Refusal means the death proceeds.\par
\textbf{Note.} Claims should change the campaign; use sparingly.

\subsection*{Option E — Dramatic Exit & Inheritance}\label{subsec:inheritance}
\textbf{Dramatic Exit.} The player may choose a meaningful death that resolves a question or saves others.\par
\textbf{Inheritance.} The next PC created by that player inherits one of: (a) a \textbf{Relationship} (bond, contact, or rival), (b) a \textbf{Tool} (asset degraded one step), or (c) a \textbf{Lesson} (start with +1 XP toward a Talent used in the exit). Record how the world remembers them.

\subsection*{Option F — Return From Beyond}\label{subsec:return}
\textbf{Trigger.} A body, a name, and a path (rite, gate, bargain).\par
\textbf{Cost.} Treat as a High Rite with Obligation appropriate to the transgression. Set DV using \S\ref{sec:rites-dv-expanded}. On success, the PC returns \textbf{changed}: apply a \textsc{Scar} (permanent Condition) and \textbf{1 SB} to the rescuer. On failure, choose: lose the body, or return with a Patron’s Claim.

\subsection*{Table Safety}\label{subsec:death-safety}
Always confirm that \textbf{PC death is on the table} before a session begins. Use lines/veils or your safety tools; offer \textbf{Dramatic Exit} in place of surprise death if needed.

\subsection*{GM Guidance}\label{subsec:death-gm}
\begin{itemize}
\item \textbf{Name the Stakes.} Before lethal rolls, state clearly which option(s) are available.
\item \textbf{Tie to Themes.} Death, Claims, and Returns should echo the Patron’s domain (e.g., Ikasha’s shadow vigil, Mykkiel’s writ of reprieve).
\item \textbf{One Big Choice.} Offer a single fork at the brink (Last Stand, Claim, or Exit) rather than all at once to preserve drama.
\item \textbf{Make It Matter.} Death or survival should change clocks, factions, and future scenes.
\end{itemize}

\begin{tcolorbox}[title={Quick Picks},colback=gray!5,colframe=black]
\small \textbf{Low lethality:} Baseline + Death Clock only.\quad \textbf{Heroic tragedy:} Severe Harm Death + Last Stand + Inheritance.\quad \textbf{Dark bargains:} Any combo with Patron’s Claim.\end{tcolorbox}

\section*{Narrative First: The World Remembers}\index{narrative first}

In Fate's Edge, the world is not a puzzle to be solved---it is a \textbf{living system} that responds to player choices with consequences that ripple across time and space. Let factions shift their allegiances based on player actions. Let consequences accumulate like stones in a riverbed, gradually shaping the flow of history itself. And above all---let the story unfold organically from the collision of player ambition and world reaction.

Because in the end, it is not the GM who writes the legend that will echo through the ages.

It is the players, through their choices, sacrifices, and triumphs.

You simply hold the quill that records their epic.

And what an honor that is.

% !TEX root = resource_guide_main.tex
% Resource Guide Insert: Poison = +Fatigue as Story Beats (Narrative ↔ Mechanics)

\section{Poison as Story Beats (Narrative \texorpdfstring{$\leftrightarrow$}{↔} Mechanics)}\label{sec:poison-sb}
\index{Poison}\index{Story Beats}\index{Fatigue}\index{Harm}\index{Clocks}\index{Conditions}

Poison is best when it \emph{pushes scenes forward} rather than only dealing damage. This module frames poison as \textbf{Fatigue-forward}, with \textbf{Story Beats (SB)} powering escalation.

\subsection*{Core Model}
\begin{itemize}
\item \textbf{Primary Effect:} Poison applies \textbf{Fatigue} (not Harm) unless it explicitly says otherwise.
\item \textbf{Escalation:} The GM may \textbf{spend 1 SB} created in the scene to upgrade a poison instance: \emph{Fatigue} (\to) \emph{Condition} (\to) \emph{Harm}. Name the fiction of the escalation (spasms, fever spike, airway closing).
\item \textbf{Clocks:} Use a named clock per poison (e.g., \textsc{Neurotoxin Load} 1/4). Ticks convert into Fatigue/Conditions at thresholds.
\item \textbf{Resist:} Players may Resist poison normally; strong fiction (antidote, mask, purge rite) can reduce Position or cancel a tick.
\end{itemize}

\subsection*{Delivery \& Position}
\begin{tabularx}{\linewidth}{>{\bfseries}l l X}
\toprule
Route & Base DV & Position Cues \
\midrule
Contact & 2 & Sweat, numbness, rashes; Controlled with gloves/time. \
Ingested & 3 & Bitter note, wine film; Risky at table, Controlled in lab. \
Inhaled & 3--4 & Perfume cloud, dust motes; Desperate in confined spaces. \
Injury/Venom & 3--4 & Fanged strike, tainted blade; Risky unless braced/prepared. \
\bottomrule
\end{tabularx}

\subsection*{Outcomes Palette}
\begin{description}[leftmargin=1.6em]
\item[\textbf{Strong Hit.}] Avoid or purge: clear \textsc{Poison Clock} by 1 and ignore the current instance.
\item[\textbf{Mixed.}] Mark \textbf{Fatigue +1} \emph{or} take a minor \textbf{Condition} (\textsc{Dizzy}, \textsc{Shaking}). GM may tick the \textsc{Poison Clock} +1.
\item[\textbf{Miss.}] Mark \textbf{Fatigue +1} and tick \textsc{Poison Clock} +1–2. The GM may spend SB to apply a stronger Condition \emph{or} escalate to \textbf{Harm 1}.
\end{description}

\subsection*{Conversion \& Interactions}
\begin{itemize}
\item \textbf{Fatigue → Harm.} Normal Fatigue overflow still converts per your health rules (see \S\ref{sec:health-fatigue-harm-rev}).
\item \textbf{Armor.} Armor rarely helps vs. contact/ingested poisons; vs. \emph{injury} delivery, apply armor before poison unless the blade is \textbf{Barbed/Venomous} (tag), which bypasses one step of conversion.
\item \textbf{Rites.} \emph{Purge} (\S\ref{sec:universal-rituals}) clears 1–2 ticks or downgrades Condition \textrightarrow{} Fatigue (GM sets Obligation).
\item \textbf{Patrons.} Poison from a Patron-aligned source may trigger thematic backlash or Claims when resisted.
\end{itemize}

\subsection*{SB Dials for GMs}
\begin{itemize}
\item \textbf{Pressure Valve (soft).} Spend SB to \emph{delay} catastrophic onset: tick clock instead of immediate Harm, but widen the threat.
\item \textbf{Spike (hard).} Spend SB to \emph{frontload} impact: apply a named Condition now (\textsc{Blind}, \textsc{Cramping}) or force a risky reposition.
\item \textbf{Echo.} Bank 1 SB to trigger a delayed complication (e.g., relapse during negotiations).
\end{itemize}

\subsection*{Sample Poisons}
\begin{tabularx}{\linewidth}{>{\bfseries}l X X}
\toprule
Name & Primary Effect (on hit) & SB-Driven Escalation \
\midrule
Bitterglass & Fatigue +1; breath tastes like copper. & 1 SB: \textsc{Dizzy}. 2 SB: Harm 1 (air hunger). \
Widowroot & Fatigue +1; numb grip. & 1 SB: drop held item. 2 SB: \textsc{Shaking}. \
Dream-Marrow & Fatigue +1; slow recall. & 1 SB: \textsc{Fogged} (–1 to memory). 2 SB: \textsc{Asleep} unless shaken. \
Serpent’s Borrow & Fatigue +1; swelling puncture. & 1 SB: \textsc{Cramping}. 2 SB: Harm 1 unless pressure bandaged. \
Courtier’s Kiss & Fatigue +1; sweet aftertaste. & 1 SB: \textsc{Swayed} (–1 vs. social manipulation). 2 SB: \textsc{Blurred Vision}. \
\bottomrule
\end{tabularx}

\subsection*{Worked Micro-Example}
\emph{Gala Balcony.} Inhaled perfume trap (DV~3, Risky). Mixed: PC marks Fatigue +1; GM ticks \textsc{Poison Clock}+1. Later, during the toast, the GM spends 1 SB to apply \textsc{Dizzy}; the PC chooses to lean on a friend (creating a bond) rather than push through and risk Harm.

\begin{tcolorbox}[title={Summary},colback=gray!5,colframe=black]
\small Poison leans on \textbf{Fatigue} first. \textbf{SB} lets the GM pace escalation as the scene demands. Name the clock, show the tells, and let fiction lead the effect.
\end{tcolorbox}
