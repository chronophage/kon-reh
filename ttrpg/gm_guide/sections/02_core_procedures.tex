\chapter{Running the Game: Core Procedures}

% === BEGINNER PATHWAY SIDEBAR ===
\begin{tcolorbox}[title=\textbf{The Core Loop: Your First 10 Games},colback=blue!5,colframe=blue!75!black,fonttitle=\bfseries]
\textbf{Start here for your first session! Ignore advanced systems until comfortable.}

\medskip
1. Player states Goal \& Action (Attribute + Skill)\\
2. GM sets simple DV (2-5)\\
3. Player rolls. Count Successes (6+)\\
4. GM consults Outcome Matrix. \textbf{Ignore Boons for now. Use simple complications on Partial/Miss.}

\medskip
\footnotesize
\textit{Once comfortable, add:} \hyperref[sec:boons]{\textbf{Boons}} \textbar\ 
\hyperref[sec:story-beats]{\textbf{Story Beats}} \textbar\ 
\hyperref[sec:clocks]{\textbf{Clocks}}
\end{tcolorbox}

% =========================================================
% SECTION: FIRST GAME MINI-MODULE
% =========================================================
\section*{First Game Scenario: The Sunstone Tower}
\addcontentsline{toc}{section}{First Game Scenario: The Sunstone Tower}

\textbf{Premise:} The party is hired to infiltrate the ruined Sunstone Tower and retrieve a magical sunstone before rival treasure hunters do. This scenario uses a limited rules subset perfect for beginners.

\begin{itemize}
\item \textbf{Ignore for this scenario:} Boons, Corruption, intricate magic subsystems, detailed Follower/Asset upkeep.
\item \textbf{Focus on:} Core dice pool, Success/Partial/Miss outcomes, Position, Clocks.
\item \textbf{Character Setup:} Use pregenerated characters or create simple ones with 20 XP and 2--3 clear hooks.
\end{itemize}

\subsection*{GM Prep in 10 Minutes}

\begin{fatebox}[Sunstone Tower Prep Checklist]
\begin{itemize}
\item Name 3 NPCs: hirer, rival delver, tower spirit (or echo).
\item Write 1 sentence for each scene: Approach, Interior, Sunstone Chamber.
\item Create 2 clocks: \textbf{Guardian Alert [4]} and \textbf{Tower Collapse [6]}.
\item Decide 1 twist: rival arrives early, tower shifts, or sunstone is not what it seems.
\end{itemize}
\end{fatebox}

\subsection*{Scene 1: The Approach}
\begin{tcolorbox}[colback=gray!5,colframe=gray!75!black,title=\textbf{Lane Marker: Skill Challenge with a Clock}]
\textbf{GM Focus:} Practice calls for DV and Partial outcomes. For this intro, tick the \textbf{Guardian Alert [4]} clock on a Partial or Miss.
\end{tcolorbox}

The tower stands on a cliffside. Players must navigate three challenges:
\begin{itemize}
\item Cross the crumbling bridge (Athletics) -- DV 3
\item Scale the cliff face (Athletics/Strength) -- DV 4  
\item Sneak past the stone guardians (Stealth) -- DV 3
\end{itemize}

\textbf{Guardian Alert Clock [4]:} Each Partial or Miss advances the clock by 1. If filled, guardians activate and pursue; treat them as a single \emph{Tower Guardian} threat with a simple [4] Harm track.

\subsection*{Scene 2: The Tower Interior}
This scene teaches \textbf{Position} and \textbf{environmental stakes}.

\begin{itemize}
\item \textbf{Dominant:} ``You have the high ground on the crumbling staircase, looking down on the patrol.''
\item \textbf{Controlled:} ``The hallway is dark but quiet. You can take your time, but there might be traps.''
\item \textbf{Desperate:} ``The floor gives way beneath you as you lunge for the door!''
\end{itemize}

\begin{fatebox}[How to Call Position in Play]
\begin{itemize}
\item \textbf{Ask:} ``How are you doing this? Cautious? Bold? Rushed?''
\item \textbf{Map to Position:} Cautious $\rightarrow$ Controlled; clever leverage or clear edge $\rightarrow$ Dominant; rushed, cornered, or outnumbered $\rightarrow$ Desperate.
\item \textbf{State It Out Loud:} ``This is Desperate. Big reward, but real danger if it goes wrong.''
\end{itemize}
\end{fatebox}

\subsection*{Scene 3: The Sunstone Chamber}
The chamber is collapsing as rival delvers close in. Use a simple [6] clock for the collapse. Each round, the clock ticks up by 1. On each Miss, advance it by an extra 1 segment.

\begin{itemize}
\item \textbf{Goal:} Retrieve the sunstone and escape before the clock fills.
\item \textbf{On Fill:} The chamber seals. Escape requires a final, Desperate group action.
\end{itemize}

\begin{tcolorbox}[colback=gray!5,colframe=gray!75!black,title=\textbf{Lane Marker: One New Idea at a Time}]
\footnotesize
Scene 1: Clocks \emph{only}.\\
Scene 2: Add \emph{Position}.\\
Scene 3: Combine \emph{Clocks + Position}. Reserve Boons, Corruption, and advanced magic for later sessions.
\end{tcolorbox}

% =========================================================
% SECTION: LEARNING THROUGH EXAMPLES
% =========================================================
\section{Learning Through Examples}

This section walks the same situation through multiple passes, adding one rule at a time. Use it as a mental template when you improvise.

\subsection*{Example 1: Core Loop Only}

\begin{tcolorbox}[colback=gray!3,colframe=gray!60,title=\textbf{Example 1 -- No SB, No Boons}]
Lyra tries to sneak past the guardian in a narrow hallway. (Agility + Stealth = 5d10).\\
The GM says it's a normal risk, DV 3, Controlled Position.\\
She rolls and gets: \texttt{3, 5, 6, 7, 9} $\Rightarrow$ 3 successes (6, 7, 9).\\[0.3em]
DV 3 with 3 successes is a \textbf{Success}.\\[0.3em]
\textbf{GM:} ``You slip past the guardian quietly. You're in position behind it if you want to act.''
\end{tcolorbox}

\subsection*{Example 2: Add Story Beats Only}
\label{sec:ex-sb-only}

\begin{tcolorbox}[colback=gray!3,colframe=gray!60,title=\textbf{Example 2 -- Adding Story Beats}]
Same situation. Lyra rolls 5d10 and gets: \texttt{1, 4, 6, 6, 8}.\\
Successes: 3 (6, 6, 8). She also rolled a \textbf{1}, which makes 1 Story Beat (SB) for the GM.\\[0.3em]
DV 3 with 3 successes is still a \textbf{Success}, but the GM now has 1 SB.\\[0.3em]
\textbf{GM Spend (1 SB = Minor Complication):}\\
\textit{``You get past the guardian, but your cloak snags and tears. The sound echoes -- the next room has a -1 Position penalty for stealth checks.''}
\end{tcolorbox}

\subsection*{Example 3: Add Boons Only}

\begin{tcolorbox}[colback=gray!3,colframe=gray!60,title=\textbf{Example 3 -- Adding Boons}]
Later, Lyra attempts to scale a loose stone wall (Athletics + Agility = 4d10). DV 3.\\
She rolls: \texttt{2, 3, 4, 5}. Zero successes = \textbf{Miss}.\\[0.3em]
Instead of only consequence, the GM awards \textbf{2 Boons}.\\
\textbf{GM:} ``You slip and land hard, taking 1 Harm. Mark 2 Boons -- you can use them later to turn the tide.''\\[0.3em]
On her next attempt, Lyra spends 1 Boon to re-roll a die, turning a failure into a success. The player feels the sting of failure \emph{and} the promise of payoff.
\end{tcolorbox}

\subsection*{Example 4: Putting It Together}

\begin{tcolorbox}[colback=gray!3,colframe=gray!60,title=\textbf{Example 4 -- SB + Boons + Position}]
The party is fleeing as the tower begins to collapse. A crumbling staircase stands between them and the exit.\\
\textbf{GM:} ``This is Desperate. DV 4 to get everyone across safely.''\\
Lyra leads, rolling 6d10 (Agility + Athletics + help). She gets: \texttt{1, 2, 6, 7, 7, 9}.\\[0.3em]
\textbf{Readout:}
\begin{itemize}
\item 4 successes (6, 7, 7, 9) vs DV 4 = Success.
\item 1 rolled \textbf{1} = 1 SB to the GM.
\item Desperate Position means high risk if there had been fewer successes.
\end{itemize}
\textbf{GM:} ``You all make it across, but the last step collapses behind you. I spend 1 SB: your exit path is gone. You'll need a new way out.''\\[0.3em]
\textbf{Lesson:} Even on a success, SB let you bend the fiction toward drama without erasing the player's win.
\end{tcolorbox}

% =========================================================
% SECTION: CORE RESOLUTION CYCLE
% =========================================================
\section{The Core Resolution Cycle}

When a player rolls, they engage the world through risk, consequence, and discovery. Follow these steps:

\begin{enumerate}
\item \textbf{Declare Action \& Approach:} Player states intent, Attribute + Skill.
\item \textbf{Set Difficulty Value (DV):} Based on narrative stakes.
\item \textbf{Establish Position:} Dominant, Controlled, or Desperate.
\item \textbf{Roll Pool of d10s.}
\item \textbf{Count Successes (6+) and Story Beats (1s).}
\item \textbf{Check Against DV} using Outcome Matrix.
\item \textbf{Apply Outcome:} Success, Partial, or Miss.
\item \textbf{Spend SB} for complications and twists.
\end{enumerate}

\begin{fatebox}[Position Effects]
\begin{tabularx}{\textwidth}{lX}
\toprule
\textbf{Position} & \textbf{Mechanical Effect} \\
\midrule
Dominant & May re-roll one failure (die < 6). \\
Controlled & Default state; no re-rolls. \\
Desperate & Must re-roll one success (die 6+), keeping the second result. \\
\bottomrule
\end{tabularx}
\end{fatebox}

\begin{fatebox}[Difficulty Ladder -- Beginner Focus]
\begin{tabularx}{\textwidth}{lX}
\toprule
\textbf{DV} & \textbf{When to Use} \\
\midrule
2 & Routine: Clear path, no pressure, almost guaranteed. \\
3 & Default: Mild challenge, some risk, use this most often. \\
4 & Hard: Serious opposition, bad angle, or strong resistance. \\
5+ & Extreme: Save for major boss encounters or dramatic gambles. \\
\bottomrule
\end{tabularx}
\end{fatebox}

% =========================================================
% SECTION: STORY BEATS
% =========================================================
\section{Story Beats: The Engine of Drama}\label{sec:story-beats}

Every time a player rolls a 1, you gain Story Beats (SB). These are narrative tools, not punishments.

\begin{fatebox}[SB Spend Menu (Beginner Version)]
\begin{tabularx}{\textwidth}{lX}
\toprule
\textbf{SB Cost} & \textbf{Simple Complications} \\
\midrule
1 SB & Minor: Noise, distraction, small setback, lost gear. \\
2 SB & Moderate: Alarm raised, new threat enters, lose advantage. \\
3+ SB & Major: Reinforcements, scene shifts, stakes escalate. \\
\bottomrule
\end{tabularx}
\end{fatebox}

\subsection*{Failing Forward with SB}

\begin{itemize}
\item \textbf{On Success:} Use SB to add texture. The hero wins, but something changes.
\item \textbf{On Partial:} The hero gets what they wanted \emph{and} you spend SB to add cost or threat.
\item \textbf{On Miss:} You can spend SB to make the setback sharper, broader, or longer-lasting.
\end{itemize}

\begin{tcolorbox}[colback=gray!5,colframe=gray!75!black,title=\textbf{Lane Marker: One Complication per Roll}]
When in doubt, spend SB on \emph{one} clear complication instead of many small ones. Name it: ``Guard Alerted'', ``Floor Cracked'', ``Oath Owed'' and move on.
\end{tcolorbox}

% =========================================================
% SECTION: COMBAT SIMPLIFIED
% =========================================================
\section{Combat Made Simple}

Combat uses the same core loop. The only difference is that the stakes are higher and usually more immediate.

\begin{fatebox}[Combat Quick Reference]
\begin{itemize}
\item \textbf{Initiative:} No fixed order. Ask: ``Who acts next?'' Follow the fiction.
\item \textbf{Actions:} Each turn, a character can move and take one meaningful action.
\item \textbf{Position Matters:} Flanking or ambush $\rightarrow$ Dominant; surrounded or exposed $\rightarrow$ Desperate.
\item \textbf{Use Clocks:} Track enemy morale [6], environmental dangers [4], and boss thresholds.
\end{itemize}
\end{fatebox}

\subsection*{Three-Beat Combat Structure}

When you feel lost in a fight, fall back on three beats:

\begin{enumerate}
\item \textbf{Opening Exchange:} Establish where everyone is and what they want.
\item \textbf{Turning Point:} Spend SB or tick clocks to change the situation (reinforcements, terrain shifts).
\item \textbf{Final Gamble:} Make the last few rolls matter -- higher DV, Desperate Position, or big rewards.
\end{enumerate}

\begin{tcolorbox}[colback=gray!5,colframe=gray!75!black,title=\textbf{Example: SB in Combat}]
\footnotesize
The party fights bandits on a collapsing bridge.\\
On a Partial, the GM spends 2 SB: ``You push them back, but the bridge loses another support. The \emph{Bridge Collapse [4]} clock advances.''\\
Now every action is also about whether they escape in time.
\end{tcolorbox}

% =========================================================
% SECTION: CLOCKS & FRONTS
% =========================================================
\section{Clocks \& Fronts}\label{sec:clocks}

Clocks are visual trackers for threats, progress, and looming changes. They keep pressure visible and shared.

\subsection*{Basic Clock Usage}

\begin{itemize}
\item \textbf{Size:} Use [4] for small scenes, [6] for bigger problems, [8] for arcs.
\item \textbf{Advance On:} Misses, certain Partials, or SB spends.
\item \textbf{Name It:} ``Alarm Raised [4]'' or ``Floodwaters Rise [6]'' is better than a blank circle.
\end{itemize}

\begin{fatebox}[Beginner Clock Limits]
\begin{itemize}
\item No more than 3 active clocks per scene.
\item Only 1--2 clocks should matter to the current roll.
\item Say out loud when you tick a clock and why.
\end{itemize}
\end{fatebox}

\subsection*{Mini-Fronts for Short Arcs}

A \emph{Front} is a cluster of related clocks and threats.

\begin{itemize}
\item \textbf{Example Front:} ``Rival Delvers of the Sunstone Tower''
\item Clocks: ``Rivals Close In [6]'', ``Tower Integrity [6]'', ``Rival Reputation [4]''.
\item Each time you spend SB, consider ticking one of these clocks.
\end{itemize}

% =========================================================
% SECTION: MODULAR PATHWAYS
% =========================================================
\section{Ready for More? Add These Systems}

Once you're comfortable with the core loop, introduce these modules one at a time. Treat each as a new lane marker.

\subsection*{Boons}\label{sec:boons}
Boons reward players for pushing their luck.

\begin{itemize}
\item On \textbf{Miss:} Gain 2 Boons.
\item On \textbf{Partial:} Gain 1 Boon.
\item \textbf{Spend:} Re-roll a die, add +1 Effect, or seize a small advantage (GM-approved).
\end{itemize}

Start with \emph{re-roll only}. Once players are comfortable, add the other options.

\subsection*{Advanced Magic}
\begin{itemize}
\item \textbf{Basic:} Use simple Tag-based magic (single roll, single tag).
\item \textbf{Intermediate:} Add Runekeepers after 2--3 sessions, when everyone understands SB and Clocks.
\item \textbf{Advanced:} Introduce Invokers and ritual magic for groups that enjoy planning and complex payoffs.
\end{itemize}

\subsection*{Followers, Assets, and Domains}
Add persistent resources when the group cares about territory, organizations, or long-term projects.

\begin{itemize}
\item Start with 1 simple Follower or Asset each.
\item Tie them to clocks: ``Caravan Reputation [6]'', ``Temple Influence [4]''.
\item Spend SB to threaten or complicate these resources.
\end{itemize}

% =========================================================
% SECTION: SESSION FLOW & CHECKLISTS
% =========================================================
\section{Session Flow: A GM Cognitive Checklist}

Use this as a quiet mental script. You do not need to say any of it out loud.

\subsection*{Opening 10 Minutes}

\begin{itemize}
\item Ask: ``What does everyone want out of tonight?'' (goal, vibe, focus).
\item Recap 2--3 key facts and 1 unresolved clock.
\item Ask each player: ``What is your character worried about right now?''
\end{itemize}

\subsection*{During Play}

\begin{itemize}
\item Before a roll: Name DV, Position, and stakes.
\item After a roll: Say the outcome type (Success/Partial/Miss) first, then the fiction.
\item Between scenes: Look at clocks. Ask, ``Which one should move? What changes in the world?''
\end{itemize}

\subsection*{Closing 10 Minutes}

\begin{itemize}
\item Ask: ``What was your favorite moment?'' (signals what to do more of).
\item Note any clocks that reached halfway or full.
\item Jot 2 bullet points: ``Next time on Fate's Edge...'' as hooks.
\end{itemize}

\begin{tcolorbox}[title=\textbf{Quick GM Cheat Sheet},colback=green!5,colframe=green!75!black]
\begin{itemize}
\item \textbf{DV Default:} 3 (adjust $\pm1$ for ease/difficulty).
\item \textbf{Position:} Controlled (normal), Dominant (advantage), Desperate (risk).
\item \textbf{SB Spend:} 1=minor, 2=moderate, 3+=major complication.
\item \textbf{Max Clocks:} 3 per scene to avoid overwhelm.
\item \textbf{Golden Rule:} When in doubt, make a ruling that keeps the story moving.
\end{itemize}
\end{tcolorbox}

\vspace{1em}
\noindent\textbf{Remember:} The goal is to tell an exciting story together. These rules are tools, not tests. Start simple, add complexity when you're ready, and keep your lane markers clear: one new idea at a time, one clear consequence per roll, and one or two clocks that really matter.