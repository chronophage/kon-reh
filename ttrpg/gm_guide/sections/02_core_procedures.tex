\chapter{Running the Game: Core Procedures}\index{core procedures}

In \textbf{Fate's Edge}, the game flows through a series of \textbf{actions, consequences, and escalating stakes}. As the GM, your role is to guide this flow---not by dictating outcomes, but by \textbf{framing scenes, interpreting rolls, and spending Story Beats}\index{Story Beats} to keep tension alive. This chapter walks you through the core procedures that define play, from the moment a player declares an action to the fallout that follows.

\section{Scene Framing: Start with Stakes}\index{scene framing}

Every scene begins with a question: \textbf{What's at risk?} Not just for the characters, but for the world, the mission, or the fragile alliances they've built. As the GM, you frame the scene by establishing:

\begin{itemize}
    \item \textbf{Position}\index{position}: Is the action \textit{Controlled}\index{position!Controlled}, \textit{Risky}\index{position!Risky}, or \textit{Desperate}\index{position!Desperate}?
    \item \textbf{Effect}: What happens on a success? What changes?
    \item \textbf{Stakes}: What is gained---or lost---if things go wrong?
\end{itemize}

A scene in the \textbf{Mistlands} might begin with the PCs crossing a flooded causeway at dusk. The bell-line hums with tension. The GM sets the position as Risky---slippery stones, rising mist, and the distant echo of a wraith-call. A failure here could mean separation, exposure, or worse.

\subsection{Position Descriptions}

\begin{itemize}
    \item \textbf{Controlled}: You act on your terms. Complications are minor, setbacks are rare.
    \item \textbf{Risky}: You act under pressure. Success is possible, but failure brings a cost.
    \item \textbf{Desperate}: The odds are stacked against you. Success is hard-won, and failure is dramatic.
\end{itemize}

Use position to guide the fiction. A controlled entry into a noble salon in \textbf{Vhasia} might allow the PCs to charm or intimidate without resistance. A desperate one---perhaps after triggering an alarm---means blades are drawn before words.

\section{Adjudicating Rolls: The Core Resolution Cycle}\index{roll adjudication}

When a player rolls, they are not simply trying to "beat" a number. They are engaging with the world. This section guides you through the complete resolution process.

\subsection{Step-by-Step Roll Resolution}

\begin{enumerate}
    \item \textbf{Player declares action and approach} (Attribute + Skill).
    \item \textbf{GM sets Difficulty Value (DV)}\index{Difficulty Value (DV)}\index{DV|see{Difficulty Value}} based on stakes and fiction.
    \item \textbf{Player rolls pool of d10s.}
    \item \textbf{Count successes (6+)} and \textbf{Story Beats (1s)}\index{Story Beats}.
    \item \textbf{Compare successes to DV} and apply Outcome Matrix.
    \item \textbf{GM spends SB} or draws from the Deck of Consequences\index{Deck of Consequences}.
\end{enumerate}

\begin{fatebox}[Difficulty Ladder (Set Before the Roll)]\index{Difficulty Ladder}
\begin{tabularx}{\textwidth}{lX}
\toprule
\textbf{DV} & \textbf{When to Use} \\
\midrule
2 & Routine: Clear intent, modest stakes, controlled environment \\
3 & Pressured: Time pressure, mild resistance, partial information \\
4 & Hard: Hostile conditions, active opposition, precise timing required \\
5+ & Extreme: Multiple constraints, high precision, dramatic failure possible \\
\bottomrule
\end{tabularx}
\end{fatebox}

A DV should reflect not just mechanical difficulty, but narrative weight. Climbing a wall? That's routine. Climbing it while pursued by Aeler vault-wardens? That's pressured---or worse.

\begin{fatebox}[Outcome Matrix]\index{Outcome Matrix}
\begin{tabularx}{\textwidth}{lX}
\toprule
\textbf{Result} & \textbf{GM Guidance} \\
\midrule
$S \geq DV$ and $C = 0$ & Clean Success: Deliver the intent crisply \\
$S \geq DV$ and $C > 0$ & Success \& Cost: Grant the intent; spend/bank SB for complications \\
$0 < S < DV$ & Partial: Progress with a fork or complication. Award one boon to the player. \\
$S = 0$ & Miss: No progress. Cash/bank SB, award two boons to player \\
\bottomrule
\end{tabularx}
\end{fatebox}

\section{Fail Forward: Every Roll Matters}\index{Fail Forward}

When you \textbf{MISS} on a \emph{significant action}, you gain \textbf{2 Boons} and one on a \textbf{PARTIAL} success. A meaningful action is one that:
\begin{enumerate}
  \item follows the standard procedure (declared intent and approach, DV set, then roll), 
  \item has \textbf{stated stakes} before the roll (what changes on success; what bites on failure), and
  \item lands a \textbf{real consequence} now (the GM spends or banks SB, applies a condition, or advances a thread).
\end{enumerate}

\subsection{Important Notes}
\begin{itemize}
  \item Rolling a \textbf{1} always creates \textbf{SB} for the GM. Re-rolling \textbf{1}s does not remove SB already generated.
  \item \textbf{No Boon} for rehearsal, null-risk probes, or repeated identical attempts without a new approach, position, or stakes.
  \item \textbf{Controlled} tests with trivial fallout do not award Boons; they're for information or positioning, not currency.
\end{itemize}

\subsection{Anti-Fishing Measures}
If Boons are spiking, use one or more:
\begin{itemize}
  \item \textbf{Once/Scene Cap:} At most \textbf{2 Boons from failures} per character per scene (further misses still create SB).
  \item \textbf{Repetition Rule:} Same approach + same stakes in the same scene cannot award another Boon.
\end{itemize}

\subsection{Example}
Picking a lock under watch (\emph{Risky}, DV 3). Stakes are set: success opens the door; a miss trips the alarm. The roll \textbf{MISS}es; the GM spends 2 SB to start the alarm and tick \emph{Guards Incoming}. The player gains \textbf{2 Boons} from meaningful failure.

\subsection{Boon Sharing}

Players may gift \textbf{1 Boon per scene} to an ally with a brief narrative justification.  
\begin{itemize}
  \item \textbf{Bonded Allies:} If characters share a bond, they may gift \textbf{2 Boons per scene}.  
  \item \textbf{Assistance:} Boons may be spent to enhance an ally’s roll (counts as assistance).  
  \item \textbf{Campaign Events:} Major victories or setbacks may generate shared Boons for the party.  
\end{itemize}

\textbf{Table Use:} Require a short story beat for each gift. Normal Boon limits apply. Track shared Boons openly.  
\textbf{GM Notes:} Reward generosity with extra opportunities, introduce occasional complications from dependence, and balance group vs.\ individual needs.

\section{Story Beats: The Engine of Drama}\index{Story Beats}

Every time a player rolls a \textbf{1}, a Story Beat is generated. These are not mere penalties---they are narrative levers. Spend them to:

\begin{itemize}
    \item Escalate a threat (drawing more enemies, raising the stakes).
    \item Drain resources (time, gear, positioning).
    \item Reveal hidden dangers or betrayals.
    \item Cause collateral damage or unintended consequences.
\end{itemize}

Story Beats should \textbf{push the story forward}, not grind it to a halt. Use them to add pressure, not to punish.

\begin{fatebox}[SB Spend Menu]\index{SB Spend Menu}
\begin{tabularx}{\textwidth}{lX}
\toprule
\textbf{SB Cost} & \textbf{Example Complications} \\
\midrule
1 SB & Minor pressure: noise, trace, +1 Supply segment, brief distraction \\
2 SB & Moderate setback: alarm raised, lose position/cover, lesser foe appears \\
3 SB & Serious trouble: reinforcements arrive, key gear breaks, tactical disadvantage \\
4+ SB & Major turn: trap springs, authority intervenes, scene shifts dramatically \\
\bottomrule
\end{tabularx}
\end{fatebox}

\subsection{When to Draw from the Deck of Consequences}\index{Deck of Consequences}

The Deck of Consequences is a powerful tool for \textbf{thematic consistency}. When a player generates SB, you may choose to:

\begin{itemize}
    \item \textbf{Direct Spend}: Translate SB into consequences/rail ticks immediately.
    \item \textbf{Deck Draw}: Draw up to \textbf{min(SB, 3)} cards and \textbf{synthesize a single twist} guided by suit and highest rank.
\end{itemize}

Never do both for the same roll. If the drawn card contradicts established fiction, reinterpret or redraw to fit the suit and tone.

\subsection{High-Tier SB Sinks}\index{High-Tier SB Sinks}
For 3--6+ SB spends that move the world (reputation cascades, faction instability, resonance, prophecy), see the stand-alone \emph{High SB Sinks} handout. A good default: at end of leg, \textbf{3 SB → tick 1 Front}\index{Front}.

\subsection{Banking \& Cashing SB}\index{banking SB}\index{cashing SB}

\begin{itemize}
    \item Banked SB should pay off within the same scene or arc.
    \item Avoid nickel-and-diming. Prefer one memorable complication over many petty penalties.
\end{itemize}

\section{Scene Management Tools}

\subsection{Scene Starters and Hooks}\index{scene starters}\index{hooks}

To keep the game moving, always open a scene with a strong hook:

\begin{itemize}
    \item "The alarm bells ring as you step into the courtyard."
    \item "A courier collapses at your feet, clutching a sealed scroll."
    \item "The tide is turning---the ghost-ferry won't wait."
\end{itemize}

Let the players react. Let the world respond. And always---\textbf{follow the consequences.}

\subsection{Setting Stakes Fast (Cheat Prompts)}\index{stakes setting}

\begin{itemize}
    \item If this goes right, what changes?
    \item If this goes wrong, what bites back?
\end{itemize}

\section{Bond-Driven Resource Generation}\index{bond-driven resource generation}\index{bonds!resource generation}

Players may earn boons by taking significant actions to aid bonded allies while providing intricate descriptions of how their bonds motivate their actions.

\subsection{Adjudication Guidelines}
\begin{itemize}
    \item \textbf{Mutual Bond:} Verify the player and ally share a defined bond
    \item \textbf{Intricate Description:} The description must meaningfully reference the bond's nature
    \item \textbf{Significant Aid:} The assistance must be substantial, not routine help
    \item \textbf{Fiction First:} The bond must genuinely explain the character's motivation
\end{itemize}

\subsection{GM Discretion}
\begin{itemize}
    \item Deny the boon if the action is trivial or the bond reference is superficial
    \item Encourage creative bond references that deepen character relationships
    \item Consider allowing this even when the aiding action fails, if the bond motivation was genuine
\end{itemize}

This mechanic reinforces collaborative play and character relationship development while providing meaningful mechanical rewards for roleplaying.

\section{Integrated Combat Procedures}\index{combat procedures}

Combat in \textbf{Fate's Edge} follows the same core procedures as all other actions, but with specific applications for violent conflict. Every combat action generates potential for both triumph and complication, with consequences that cascade through the same economy as all other challenges.

\subsection{Combat Resolution Procedure}\index{combat resolution}

\begin{enumerate}
    \item \textbf{Declare Action}: Player states intent and approach (Attribute + Skill)
    \item \textbf{Set Position}: GM sets Controlled, Risky, or Desperate based on tactical situation
    \item \textbf{Roll Dice}: Roll pool = Attribute + Skill (takes 1 Player Turn)
    \item \textbf{Count Results}: 6+ = Success, 1 = Story Beat (SB)
    \item \textbf{Apply Outcome}: Use standard Outcome Matrix
    \item \textbf{Manage Consequences}: GM spends SB or draws from Consequences Deck
\end{enumerate}

\subsection{Combat-Specific Position Applications}

\begin{itemize}
    \item \textbf{Controlled}: Advantageous position, minor consequences (flanking, higher ground, surprised foe)
    \item \textbf{Risky}: Even odds, moderate consequences (evenly matched, contested terrain)
    \item \textbf{Desperate}: Disadvantaged, severe consequences (outnumbered, wounded, poor positioning)
\end{itemize}

\begin{fatebox}[Combat Consequence Types by Suit]\index{combat consequences}\index{consequence types!combat}
\begin{tabularx}{\textwidth}{lX}
\toprule
\textbf{Suit} & \textbf{Complication Themes} \\
\midrule
Hearts & Morale, fear, command breakdown, psychological pressure, loyalty tests \\
Spades & Physical harm, positioning changes, weapon status, tactical wounds, cover loss \\
Clubs & Resource depletion, gear damage, fatigue, ammunition issues, supply problems \\
Diamonds & Environmental hazards, reinforcements, terrain changes, unexpected events \\
\bottomrule
\end{tabularx}
\end{fatebox}

\begin{fatebox}[Harm Integration with SB Economy]\index{harm integration}\index{SB economy!harm}
\begin{tabularx}{\textwidth}{lX}
\toprule
\textbf{Harm Level} & \textbf{Effects \& SB Generation} \\
\midrule
Harm 1 & -1 die to related actions; generate 1 SB on next 2 rolls \\
Harm 2 & -1 die to most actions; generate 1 SB on next roll until treated \\
Harm 3 & -2 dice to relevant actions; generate 2 SB on next roll; incapacitation risk \\
\bottomrule
\end{tabularx}
\end{fatebox}

\begin{fatebox}[Tactical Clocks]\index{Tactical Clocks}
\begin{tabularx}{\textwidth}{lX}
\toprule
\textbf{Clock Type} & \textbf{Purpose \& Triggers} \\
\midrule
Mob Overwhelm [6] & Enemy numbers become advantage; tick when outflanked or surrounded \\
Fatigue Spiral [4] & Exhaustion affects performance; tick after strenuous actions or wounds \\
Morale Collapse [6] & Fear undermines effectiveness; tick after leader falls or shocking events \\
Environmental [8] & Terrain/fire/building failure; tick after destructive actions or time pressure \\
\bottomrule
\end{tabularx}
\end{fatebox}

\subsection{Position Dynamics in Combat}\index{position dynamics}

Position can shift during combat based on SB spending and narrative triggers:

\begin{itemize}
    \item \textbf{GM Spend (1 SB)}: Shift position one step worse for current action
    \item \textbf{Player Spend (1 Boon)}: Shift position one step better or cancel GM shift
    \item \textbf{Narrative Triggers}: Flanking, reinforcement arrival, environmental changes, superior leverage
\end{itemize}

\subsection{Magic Combat Integration}\index{magic combat}

Spellcasting in combat feeds the same consequence economy:

\begin{itemize}
    \item Channel/Weave Backlash SB applies to tactical situation
    \item Spells can shift position, create tactical clocks, or generate combat consequences
    \item Magic consequences cascade through existing combat systems
\end{itemize}

\subsection{Asset/Follower Combat Integration}\index{asset combat}\index{follower combat}

\begin{itemize}
    \item \textbf{Follower Risk}: 2+ SB spent in combat can endanger assisting followers
    \item \textbf{Asset Compromise}: Combat in certain locations can damage relevant assets  
    \item \textbf{Offensive Activation}: 1 Boon activates asset for combat advantage
    \item \textbf{Initiative Actions}: Followers can take combat-relevant independent actions
\end{itemize}

\subsection{Combat Outcome Matrix Application}

Same as standard resolution, but consequences are combat-specific:

\begin{itemize}
    \item \textbf{Clean Success}: Intent achieved with no tactical complications
    \item \textbf{Success \& Cost}: Intent achieved, but GM spends SB for combat consequences
    \item \textbf{Partial}: Progress with tactical fork (accept cost OR concede ground). Award one Boon.
    \item \textbf{Miss}: No progress; GM spends SB for combat consequences. Award two Boons.
\end{itemize}

\section{30-Second Adjudication Loop}\index{adjudication loop}

Use this fast procedure to keep the game flowing:

\begin{enumerate}
    \item \textbf{Clarify}: "What do you want, and how?"
    \item \textbf{Set Stakes}: "If it works, what changes? If it fails, what bites?" Start Risky/Standard unless fiction says otherwise.
    \item \textbf{Roll \& Read}: Count successes (6+) and SB (1s). Compare to DV.
    \item \textbf{Spend One Beat Well}: Cash SB on one memorable twist or tick a relevant Clock.
    \item \textbf{Push Forward}: Describe how the fiction changes; ask "Who moves next?"
\end{enumerate}

\section{Position + Effect in Action}

A player declares a \textbf{Risky} action to \textbf{pick a lock} while guards patrol above. They roll \textbf{Wits + Skullduggery}, get 3 successes and 1 Story Beat.

The GM consults the Outcome Matrix: \textbf{Success \& Cost}. The lock clicks---but a guard's bootstep halts above. The GM spends 1 SB to add tension: the patrol changes direction, heading toward the PCs' position.

\section{Common Pitfalls and Solutions}\index{pitfalls}\index{GM tips}

\begin{fatebox}[Troubleshooting Common Issues]
\begin{tabularx}{\textwidth}{lX}
\toprule
\textbf{Issue} & \textbf{Solution} \\
\midrule
Over-cranking SB & Halve SB spends temporarily or convert to visible Clocks \\
Clock Sprawl & Merge redundant Clocks; scenes need only 2-3 active Clocks \\
Tag Paralysis & Paraphrase: "Sounds like [TAG]. DV 3. Want to roll?" \\
Rules Drift & Pick ruling that keeps flow, note TODO for post-session reconciliation \\
Boon Inflation & Enforce 2 Boon/scene cap from failures; use Repetition Rule \\
\bottomrule
\end{tabularx}
\end{fatebox}

\section{Miniatures and Tactical Layer}
\label{sec:miniatures}

\subsection{Core Concepts}
\begin{itemize}
  \item Works on square or hex grids; declare grid type at setup.
  \item Units have base sizes (Small, Medium, Large, Huge) and a facing.
  \item Actions per turn: Move and Act (attack, cast, interact, etc.), in either order.
  \item All checks use normal SRD roll + DV system.
\end{itemize}

\subsection{Turn Structure}
\begin{enumerate}
  \item Start: resolve ongoing effects.
  \item Move: up to Speed; obey Zones of Control (ZOC).
  \item Act: attack, test, assist, cast, rally, shove, guard, etc.
  \item End: resolve end effects and reactions.
\end{enumerate}

\subsection{Zones of Control (ZOC)}
\begin{itemize}
  \item \textbf{Squares:} 4 orthogonal adjacents (optional: 8). 
  \item \textbf{Hexes:} 6 adjacents.
  \item Large/Huge project ZOC from edges; Reach may extend ZOC by +1 ring.
  \item \textbf{Rules:} 
    \begin{itemize}
      \item Entering enemy ZOC ends movement (you are engaged).
      \item Cannot move through enemy ZOC.
      \item Leaving requires Disengage (DV 4–6) or spend 1 Boon.
      \item Multiple ZOCs increase DV by +1 per extra controller.
    \end{itemize}
\end{itemize}

\subsection{Facing and Flanking}
\begin{itemize}
  \item Choose a facing at end of movement.
  \item Flank: +1 die if attacked from opposite arcs; Rear: +1 die and +1 Effect.
\end{itemize}

\subsection{Special Actions}
\begin{itemize}
  \item \textbf{Guard:} Ready a strike when enemy leaves ZOC.
  \item \textbf{Dash:} +2 movement this turn.
  \item \textbf{Brace:} Resist Shoves/Pulls and extend ZOC (opportunity only).
  \item \textbf{Tackle:} Knock target prone (DV 4–6).
\end{itemize}

\subsection{Magic Integration}
\begin{itemize}
  \item Magic uses \textbf{[TAGS]} (e.g., [WARD], [BANISH], [CONJURE]) tied to ZOC, range, and LoS.
  \item Casting while engaged worsens Position unless [INSTANT] or aided by Talent.
  \item Rituals require clear space and visible Symbols; disrupted rituals fail or require a test.
\end{itemize}

\subsection{Quick Reference}
\begin{itemize}
  \item Entering enemy ZOC ends movement; leaving requires Disengage.
  \item Flank = +1 die; Rear = +1 die and +1 Effect.
  \item Difficult terrain +1 cost; moving up elevation +1.
  \item Boons may break ZOC rules: auto-Disengage, change facing, or Heroic Rush.
\end{itemize}

\subsection{Running Social Clocks (GM Guidance)}
\index{Social!Clocks}\index{Persuasion}\index{Bonds}\index{Story Beats}

\paragraph{Framing the Ask.}
State a concrete outcome (“grant passage tonight,” “drop the bounty,” “fund our expedition”). If it’s strictly binary and low-stakes, use one roll. Otherwise, build a clock.

\paragraph{Set Two Clocks.}
\begin{itemize}
  \item \textbf{Persuasion} (4/6/8): your progress.
  \item \textbf{Opposition} (4/6): their resistance (ego, risk, duty, rival’s whisper).
\end{itemize}
Name them (“\textsc{Council Swayed},” “\textsc{Captain’s Doubt}”) so the table sees the story move.

\paragraph{Position → DV.}
Use setting, leverage, and stakes:
\begin{itemize}
  \item \textbf{Controlled (DV 2):} private audience, proof in hand, shared values.
  \item \textbf{Risky (DV 3):} time pressure, partial access, mixed reception.
  \item \textbf{Desperate (DV 4–5+):} public scrutiny, scandal risk, hostile crowd.
\end{itemize}

\paragraph{Distinct Approaches (examples).}
\begin{itemize}
  \item \textbf{Wits+Sway:} reframe incentives; offer face-saving out.
  \item \textbf{Wits+Lore:} cite precedent, produce documents or testimony.
  \item \textbf{Body+Presence:} command presence, ritual authority, oath.
  \item \textbf{Wits+Tinker:} demonstrate a device/proof-of-concept on the spot.
  \item \textbf{Bonds/Boons:} an ally vouches (assist) or gifts a Boon; NPC bonds can reduce DV by 1 if genuinely invoked.
\end{itemize}

\paragraph{Outcomes Palette.}
\begin{itemize}
  \item \textbf{Strong Hit:} Tick Persuasion +2; optionally bank a \emph{Concession} (you can waive a future cost).
  \item \textbf{Mixed:} Tick +1 and choose a cost: start a small \textsc{Rebuttal 1/4}, owe a minor favor, or the GM spends 1 SB to introduce a new stakeholder.
  \item \textbf{Miss:} No progress. GM may (a) reduce Persuasion −1, (b) advance \textsc{Opposition} +1–2, or (c) worsen Position one step.
\end{itemize}

\paragraph{When Opposition Fills First.}
The target hardens or reframes the negotiation: narrow the ask, accept a condition, or escalate proof (pay a Boon, reveal evidence, bring a witness) to keep going.

\paragraph{Scaling & Dials.}
\begin{itemize}
  \item \textbf{Crowd Scenes:} Add \textsc{Hecklers}/\textsc{Applause} 1/4 that swing Position when they fill.
  \item \textbf{Stake Weight:} For life-or-death asks, require a \emph{Concession} on success (named cost that must be paid during wrap-up).
  \item \textbf{Truth vs. Bluff:} Real proof improves Position; blatant lies risk a hidden \textsc{Caught Out} 1/4 that explodes later.
  \item \textbf{Patron Color:} Invoking a Patron symbol can shift Position if on-theme—or add +1 DV if it antagonizes the audience’s loyalties.
\end{itemize}

\paragraph{End States.}
On fill, summarize the agreement and record any \emph{Concessions}, debts, or clocks that carry forward. If not filled by scene end, bank current ticks and reopen later if fiction supports it.

\subsection{Recommended Session Order (GM Checklist)}

\paragraph{1) Off-Screen (Downtime, 10–20 min)}
\begin{itemize}
  \item Upkeep: choose Efficient/Intensive; apply Neglected/Compromised if missed.
  \item Obligation: clear via Acts of Service; note Claims/overflow risk.
  \item Projects: tick long-term clocks; resolve Gather Info; prep assets.
  \item Intent: each player states one on-screen goal; GM surfaces 1–2 front pressures.
\end{itemize}

\paragraph{2) On-Screen (Scenes)}
\begin{itemize}
  \item Frame hard: where/what’s at stake; set Position $\to$ DV.
  \item Run spotlight: rotate beats; fold in bonds and Boon sharing.
  \item Advance: move faction/Patron clocks openly when triggered.
\end{itemize}

\paragraph{3) Wrap-Up (5–10 min)}
\begin{itemize}
  \item XP \& Talents: award, mark progress; note any Gifts gained/forfeit.
  \item SB \& Harm: convert Fatigue$\to$Harm if full; apply recoveries.
  \item Fronts: advance unresolved clocks; note consequences.
\end{itemize}

\paragraph{4) Off-Screen Hooks (2–5 min)}
\begin{itemize}
  \item Log next Downtime intents, service opportunities, upkeep deadlines.
  \item Capture cliffhangers and Patron Largess seeds for next session open.
\end{itemize}

\emph{Optional:} Add a cold open flash-cut before Step 2 to spotlight a rival or Patron omen.

% GM's Guide: Session 0 → Session 1 Onboarding
\section{Session 0 \texorpdfstring{$\rightarrow$}{→} Session 1 (GM Onboarding)}
\label{sec:gm-session0-1}
\index{Onboarding}\index{Session 0}\index{Session 1}\index{Story Beats}\index{Boons}\index{DV}\index{Position}

\subsection*{Goal}
Reach \textbf{informed readiness} in one Session~0 (3–4 hours): the table understands the core loop and has functional characters, then roll into a tutorial-style Session~1 that practices the basics.

\subsection*{Session 0 Agenda (3–4 hours)}
\begin{tcolorbox}[title={Overview},colback=gray!5,colframe=black]
\textbf{Outcome:} Shared vocabulary, finished character sheets, party bonds, and a primed first scene. Mastery comes during Session~1.
\end{tcolorbox}

\paragraph{Hour 1 — Core Principles \& The Central Question (Why).}
\begin{itemize}
  \item Philosophy: \emph{What are you willing to risk?} Narrative-First.
  \item Core loop: \textbf{Approach} $\rightarrow$ \textbf{Roll} (d10; 6+ success; 1s = SB) $\rightarrow$ \textbf{Outcome} (GM spends SB).
  \item Currencies: \textbf{Story Beats (SB)} for GM, \textbf{Boons} for players.
  \item Tools: DV Ladder; Position (\textit{Controlled / Risky / Desperate}).
\end{itemize}

\paragraph{Hour 2 — Character Creation (Who).}
\begin{itemize}
  \item Choose \emph{Ancestry/Culture} (Affinity).
  \item Allocate starting XP to \emph{Attributes} \& \emph{Skills}.
  \item Pick 1–2 starting \emph{Talents}.
  \item Choose an initial \emph{Complication} hook.
  \item \textit{Mastery expectation:} Functional sheets; intuitive timing for Boons comes later.
\end{itemize}

\paragraph{Hour 3 — Setting \& Party Fit (Where).}
\begin{itemize}
  \item Quick Hook (2-card draw or starter prompt) to frame the opening situation.
  \item Establish \emph{party bonds} and one shared near-term goal.
  \item Demonstrate \emph{Position} $\rightarrow$ \emph{DV} with the hook (one quick example).
\end{itemize}

\subsection*{Outputs \& Handshakes (End of Session 0)}
\begin{itemize}
  \item \textbf{Character sheets} complete; bonds recorded.
  \item \textbf{Opening scene} sketched (location, stake, immediate pressure).
  \item \textbf{Table tools} named: SB spend menu, DV ladder reference, Boon tracking.
\end{itemize}

\subsection*{Session 1: The Tutorial Level (90–120 min focus core)}
\begin{tcolorbox}[title={Crucial Advice},colback=gray!5,colframe=black]
\textbf{Defer subsystems.} Teach the core loop first; layer complexity later.
\end{tcolorbox}

\paragraph{Use Now.}
\begin{itemize}
  \item Core resolution (Approach $\rightarrow$ Roll $\rightarrow$ Outcome).
  \item \textbf{SB/Boons} earn/spend in play.
  \item One simple \textbf{Combat} or \textbf{Social} set-piece to practice \emph{Position} and \emph{DV}.
\end{itemize}

\paragraph{Defer For Later Sessions.}
\begin{itemize}
  \item Travel procedures and extended exploration.
  \item Deep/complex magic modules; advanced Rites; asset activation webs.
  \item Faction game and complex multi-front clocks.
\end{itemize}

\paragraph{GM Safety Nets (have these at hand).}
\begin{itemize}
  \item \textbf{DV Ladder}: Controlled=DV~2, Risky=DV~3, Desperate=DV~4–5+.
  \item \textbf{SB Spend Menu}: soft/hard complications, clocks, position shifts.
  \item \textbf{Quick Hook}: 1–2 prompts to cut into a scene without prep.
\end{itemize}

\subsection*{Recommended Flow (Session 1)}
\begin{enumerate}
  \item \textbf{Cold Open (2–3 min):} Re-state the hook and stakes.
  \item \textbf{Scene A (15–25 min):} Low-risk challenge to practice SB/Boons.
  \item \textbf{Scene B (25–35 min):} One focused conflict (Combat \emph{or} Social clock).
  \item \textbf{Wrap (5–10 min):} Award XP, note SB used, log 1–2 evolving clocks.
\end{enumerate}

\subsection*{Do/Don’t (Quick Coach Cards)}
\begin{description}[leftmargin=1.6em]
  \item[\textbf{Do}] Ask for a one-sentence fiction beat for every assist/Boon gift.
  \item[\textbf{Do}] Favor \emph{Position} shifts over raw dice creep when help is excellent.
  \item[\textbf{Do}] Name costs before rolls when spending SB for consequences.
  \item[\textbf{Don’t}] Introduce every subsystem in Session~1—save it for Session~3+.
  \item[\textbf{Don’t}] Stall on rules lookups; rule, note, move on.
\end{description}

\subsection*{Staged Onboarding Plan}
\begin{itemize}
  \item \textbf{Session 0:} Vocabulary \& build. Success = “We know what the buttons do.”
  \item \textbf{Session 1:} Core loop in a contained scenario. Success = “Fun scene, minimal lookups.”
  \item \textbf{Session 2–3:} Add \emph{one} subsystem per session (e.g., Social clocks, Travel, then Rites).
\end{itemize}

\subsection*{Monitoring (GM Notes)}
Track over the next 3–5 sessions:
\begin{itemize}
  \item \textbf{SB usage:} Are you spending 1–2 per scene for visible consequence?
  \item \textbf{Boon economy:} Are players earning/spending, or hoarding for one talent?
  \item \textbf{Friction points:} Note FAQs to tighten your table’s cheat sheets.
\end{itemize}

\subsection*{Mini Checklists}
\begin{tcolorbox}[title={Session 0 Done When},colback=gray!5,colframe=black]
\begin{itemize}
  \item All sheets complete; bonds and one party goal set.
  \item Opening scene framed; Position/DV example demonstrated.
  \item SB/Boons tracking agreed (cards, tokens, or sheet boxes).
\end{itemize}
\end{tcolorbox}

\begin{tcolorbox}[title={Session 1 Done When},colback=gray!5,colframe=black]
\begin{itemize}
  \item Each player earned and spent at least one Boon.
  \item You spent at least one soft and one hard SB consequence.
  \item One clock advanced (any size), and stakes for next time are noted.
\end{itemize}
\end{tcolorbox}