\chapter{Running the Game: Core Procedures}

\section*{How to Think in Fate’s Edge}
\index{Core Procedures!GM mindset}

Running \textbf{Fate’s Edge} is less about adjudicating difficulty and more about managing momentum.  
Your primary job is not to block success, protect outcomes, or simulate realism—it is to \textbf{frame pressure}, \textbf{track consequences}, and \textbf{let the fiction move forward}.

If you find yourself asking, “How hard should this be?” you are already one step too late.

Instead, ask:

\begin{itemize}
\item What is at stake right now?
\item What happens if this goes wrong?
\item What pressure is already in play?
\end{itemize}

The rules exist to answer those questions—not to replace them.

\subsection*{The Core Loop (In Practice)}
Every meaningful action follows the same simple loop:

\begin{enumerate}
\item \textbf{Player declares intent and approach.}
\item \textbf{You state Position and set any relevant Clocks.}
\item \textbf{The roll resolves the moment.}
\item \textbf{You apply consequences and update the fiction.}
\end{enumerate}

Do not search for the “right roll.”  
If the action matters, the procedure already applies.



\subsection*{Your Defaults}
Your first rule: When in doubt, trust the fiction. The system will follow. When unsure, fall back on these defaults:

\begin{itemize}
\item \textbf{DV}: 3
\item \textbf{Position}: Controlled
\item \textbf{Consequence}: Fatigue, time pressure, or exposure
\item \textbf{Clocks}: One short clock tied to the scene
\end{itemize}

Escalate only when the fiction demands it.



\subsection*{What You Are Not Doing}
You are \emph{not}:
\begin{itemize}
\item calling for rolls to gate progress
\item inflating difficulty to “challenge” the players
\item protecting NPC plans from disruption
\item hiding information behind unnecessary checks
\end{itemize}

If the players act decisively, the world must respond decisively.



\subsection*{What You Are Doing}
You \emph{are}:
\begin{itemize}
\item stating Position clearly and confidently
\item letting partials move the story forward
\item spending Story Beats to make failure interesting
\item allowing success to create new problems
\end{itemize}

The game works best when nothing stalls.



\subsection*{A Useful Reframe}
Think of each scene as a \textbf{pressure engine}:

\begin{itemize}
\item Dice create outcomes
\item Clocks track fallout
\item Story Beats fuel escalation
\item Boons reward risk
\end{itemize}

Your job is to keep that engine turning—not to decide where it ends.

Let the procedures do the work.  
Your attention belongs on the fiction.

\section{Standard Rule: Player-Managed Modules}
\label{sec:player-managed-modules}

This rule makes each player the primary steward of their character-facing trackers (\emph{modules}). It keeps table pace high, reduces hidden bookkeeping, and clarifies when mechanical thresholds trigger. The GM retains authority over world-facing clocks, faction fronts, and all major narrative consequences.

\subsection{Scope (\emph{What Counts as a Module})}
Player-managed modules are any \textbf{character-facing} clocks, counters, or discrete states that sit on a single character sheet:
\begin{itemize}
  \item \textbf{Obligation} (per Patron or Symbol).
  \item \textbf{Corruption Clock} (e.g., Cantor).
  \item \textbf{Leash} (Summoned spirit strain) and \textbf{Spirit Bond Clock}.
  \item \textbf{Repertoire Clock} (Cantor) or similar progression clocks.
  \item \textbf{Asset States} (e.g., Symbol: Maintained / Neglected / \textsc{Compromised} / \textsc{Shattered}).
  \item \textbf{Scene Counters} explicitly tied to a PC (e.g., Exposure on that PC, personal Buff/Debuff durations).
\end{itemize}
\textit{Not included:} GM story resources (global \textbf{Story Beats}), location/faction clocks, and mystery/doom fronts.

\begin{tcolorbox}[colback=black!3,colframe=black!40!white,title={What Players Track (at a Glance)}]
\begin{tabularx}{\textwidth}{l l X}
\toprule
\textbf{Module} & \textbf{Owner} & \textbf{Tick / Change Triggers (examples)} \\
\midrule
Obligation (by Patron) & Player & Invoke/Push/ritual text; Invoker \emph{Borrowed Grace}; cracking a Symbol; bargain costs. \\
Corruption Clock & Player & Cantor Push; Resonant Rite; GM spends a Beat tied to the PC’s occult actions. \\
Leash (Summoning) & Player & Harm to spirit; commands against nature; split focus; crossing \texttt{[WARD]} (DV = Cap). \\
Spirit Bond [4] & Player & Shared victories, mutual aid, meaningful attempts (\emph{near-miss progress} once/session/type). \\
Repertoire [6] & Player & Learn a new unique Song/rite-as-song; practice milestones. \\
Asset State (Symbol) & Player & Maintenance/downtime checks; \emph{Crack the Seal} \(\rightarrow\) \textsc{Compromised}; breakage \(\rightarrow\) \textsc{Shattered}. \\
\bottomrule
\end{tabularx}
\end{tcolorbox}

\subsection{Core Principle}
Players \textbf{immediately} mark their own modules when a rule says ``mark $+X$'' or a trigger fires. Threshold effects resolve as soon as they are reached.

\subsection{Player Duties}
\begin{enumerate}
  \item \textbf{Mark Increases/Decreases on Cue.} When you Invoke a Rite, Push, spend/clear per rules text, or a trigger fires, update your module \emph{now}, not later.
  \item \textbf{Declare Thresholds.} If marking fills a clock or crosses capacity, say so aloud; thresholds resolve before the scene proceeds.
  \item \textbf{State Ownership.} Keep per-Patron Obligation tallies distinct; track each Symbol’s state if you use Symbols.
  \item \textbf{Keep It Visible.} Use a tracker the GM and table can see (sheet boxes, index cards, or shared digital).
\end{enumerate}

\subsection{GM Duties}
\begin{enumerate}
  \item \textbf{Spot-Check.} At need, ask any player: current Obligation by Patron, Corruption segments, Leash state, Asset states.
  \item \textbf{Enforce Thresholds.} When a player reports a threshold, apply the standard effects below \emph{immediately}.
  \item \textbf{Own the Fallout.} Patron intrusions, faction reactions, front clocks, and major narrative consequences remain GM authority.
\end{enumerate}

\subsection{Standard Thresholds \& Effects}
\paragraph{Obligation Capacity}
\label{sec:obligation-capacity}
\[
\textbf{Obligation Capacity} \;=\; \textit{Spirit} + \textit{Presence}
\]
\begin{itemize}
  \item \textbf{Over Capacity:} Immediately mark \textbf{+1 Fatigue} per segment over capacity.
  \item \textbf{Over \(\mathbf{2\times}\) Capacity:} Immediately clear all Fatigue, mark \textbf{+1 Harm (Stress)}, and a \textbf{Patron Intrusion} occurs (GM frames on-theme demand/complication).
\end{itemize}

\paragraph{Corruption Full}
When a \textbf{Corruption Clock} fills:
\begin{itemize}
  \item Apply the last-Patron \textbf{benefit \& burden} (per Patron table or setting guidance) to the PC (and any listed followers/retainers).
  \item \textbf{Reset} the Corruption Clock to empty.
  \item If the player chooses \textbf{Embrace Corruption}, convert the current Patron theme into a permanent boon/curse per \S\ref{subsec:corruption-fading}.
\end{itemize}

\paragraph{Leash Full (Summoning)}
When the \textbf{Leash} fills:
\begin{itemize}
  \item The spirit acts once to its nature, then \textbf{departs} (or turns hostile at GM discretion and fiction).
\end{itemize}
\textbf{Leash Capacity:} \(\textit{Cap} + \textit{Spirit}\) segments. (\textit{Cap} is the outsider’s tier: Cap~1 for Lesser, Cap~3 for Greater.)

\paragraph{Symbol State (Invoker)}
\begin{itemize}
  \item \textbf{Maintained} \(\rightarrow\) normal function. \quad
        \textbf{Neglected} \(\rightarrow\) GM may impose $+1$ DV to related rites.
  \item \textbf{\textsc{Compromised}} (e.g., \emph{Crack the Seal}) \(\rightarrow\) instant resolution per rules; mark extra Obligation; repair in Downtime or pay 1 XP.
  \item \textbf{\textsc{Shattered}} \(\rightarrow\) unusable until replaced or ritually restored per fiction.
\end{itemize}

\subsection{Table Procedure (90-Second Loop)}
\paragraph{Start of Session}
Players read out: per-Patron \textbf{Obligation} totals, \textbf{Corruption} segments, standing \textbf{Asset States}, and any personal clocks at 3+.

\paragraph{End of Scene}
Quick pass: ``\emph{Any marks?}'' Players tick modules from scene events. If a threshold triggers, resolve now.

\paragraph{Downtime}
Players apply clears (service, contrition, purification, study) to their own modules. GM verifies any costs or fiction.

\subsection{Disputes \& Order of Operations}
If two marks would land simultaneously, apply them in the \textbf{least advantageous order for the acting character}, unless a rule specifies otherwise. The GM is final arbiter.

\subsection{Accessibility \& Tools}
Use highly visible trackers: bold boxes on sheets, poker chips for segments, or a shared table of per-Patron Obligation. Keep modules at-a-glance to minimize interruption.

\subsection{Worked Micro-Examples}
\begin{itemize}
  \item \textbf{Invoker Rites Twice:} Vessa Invokes two different Patrons. She marks each Patron’s \textbf{Obligation} separately. Hitting capacity with Patron A causes Fatigue; Patron B remains below capacity.
  \item \textbf{Cantor Pushes:} Jorel Pushes a Song (mark +1 Corruption). That fill triggers the last-Patron boon/burden immediately; then he resets to 0.
  \item \textbf{Summoner Clash:} Kestra’s Cap~3 elemental takes Harm and crosses a \texttt{[WARD]}; she ticks her \textbf{Leash} twice. On fill, the elemental flares once and departs.
\end{itemize}

% === BEGINNER PATHWAY SIDEBAR ===
\begin{tcolorbox}[title=\textbf{The Core Loop: Your First 10 Games},colback=blue!5,colframe=blue!75!black,fonttitle=\bfseries]
\textbf{Start here for your first session! Ignore advanced systems until comfortable.}

\medskip
1. Player states Goal \& Action (Attribute + Skill)\\
2. GM sets simple DV (2-5)\\
3. Player rolls. Count Successes (6+)\\
4. GM consults Outcome Matrix. \textbf{Ignore Boons for now. Use simple complications on Partial/Miss.}

\medskip
\footnotesize
\textit{Once comfortable, add:} \hyperref[sec:boons]{\textbf{Boons}} \textbar\ 
\hyperref[sec:story-beats]{\textbf{Story Beats}} \textbar\ 
\hyperref[sec:clocks]{\textbf{Clocks}}
\end{tcolorbox}

% =========================================================
% SECTION: FIRST GAME MINI-MODULE
% =========================================================
\section*{First Game Scenario: The Sunstone Tower}
\addcontentsline{toc}{section}{First Game Scenario: The Sunstone Tower}

\textbf{Premise:} The party is hired to infiltrate the ruined Sunstone Tower and retrieve a magical sunstone before rival treasure hunters do. This scenario uses a limited rules subset perfect for beginners.

\begin{itemize}
\item \textbf{Ignore for this scenario:} Boons, Corruption, intricate magic subsystems, detailed Follower/Asset upkeep.
\item \textbf{Focus on:} Core dice pool, Success/Partial/Miss outcomes, Position, Clocks.
\item \textbf{Character Setup:} Use pregenerated characters or create simple ones with 20 XP and 2--3 clear hooks.
\end{itemize}

\subsection*{GM Prep in 10 Minutes}

\begin{fatebox}[Sunstone Tower Prep Checklist]
\begin{itemize}
\item Name 3 NPCs: hirer, rival delver, tower spirit (or echo).
\item Write 1 sentence for each scene: Approach, Interior, Sunstone Chamber.
\item Create 2 clocks: \textbf{Guardian Alert [4]} and \textbf{Tower Collapse [6]}.
\item Decide 1 twist: rival arrives early, tower shifts, or sunstone is not what it seems.
\end{itemize}
\end{fatebox}

\subsection*{Scene 1: The Approach}
\begin{tcolorbox}[colback=gray!5,colframe=gray!75!black,title=\textbf{Lane Marker: Skill Challenge with a Clock}]
\textbf{GM Focus:} Practice calls for DV and Partial outcomes. For this intro, tick the \textbf{Guardian Alert [4]} clock on a Partial or Miss.
\end{tcolorbox}

The tower stands on a cliffside. Players must navigate three challenges:
\begin{itemize}
\item Cross the crumbling bridge (Athletics) -- DV 3
\item Scale the cliff face (Athletics/Strength) -- DV 4  
\item Sneak past the stone guardians (Stealth) -- DV 3
\end{itemize}

\textbf{Guardian Alert Clock [4]:} Each Partial or Miss advances the clock by 1. If filled, guardians activate and pursue; treat them as a single \emph{Tower Guardian} threat with a simple [4] Harm track.

\subsection*{Scene 2: The Tower Interior}
This scene teaches \textbf{Position} and \textbf{environmental stakes}.

\begin{itemize}
\item \textbf{Dominant:} ``You have the high ground on the crumbling staircase, looking down on the patrol.''
\item \textbf{Controlled:} ``The hallway is dark but quiet. You can take your time, but there might be traps.''
\item \textbf{Desperate:} ``The floor gives way beneath you as you lunge for the door!''
\end{itemize}

\begin{fatebox}[How to Set Position in Play]
  \index{Position!GM guidance}
  
  \begin{itemize}
  \item \textbf{Listen First:} Hear the player’s intent and approach. What are they actually doing in the fiction?
  
  \item \textbf{Assess the Situation:} Consider leverage, opposition, time pressure, exposure, and consequences. Ask yourself: \emph{How forgiving is the world right now?}
  
  \item \textbf{Assign the Position:}
  \begin{itemize}
    \item \textbf{Dominant}: The character has clear leverage, surprise, or overwhelming advantage.
    \item \textbf{Controlled}: The situation is stable; mistakes are recoverable.
    \item \textbf{Desperate}: The character is exposed, rushed, cornered, or facing serious danger.
  \end{itemize}
  
  \item \textbf{State It Clearly:}  
  Say it out loud before the roll.  
  \emph{“This is Desperate. If it goes wrong, the consequences will be serious.”}
  
  \end{itemize}
  
  \textbf{Reminder:}  
  Position is a statement of risk and consequence, not a reward or a penalty.  
  It frames what failure \emph{means}.
  \end{fatebox}

\subsection*{Scene 3: The Sunstone Chamber}
The chamber is collapsing as rival delvers close in. Use a simple [6] clock for the collapse. Each round, the clock ticks up by 1. On each Miss, advance it by an extra 1 segment.

\begin{itemize}
\item \textbf{Goal:} Retrieve the sunstone and escape before the clock fills.
\item \textbf{On Fill:} The chamber seals. Escape requires a final, Desperate group action.
\end{itemize}

\begin{tcolorbox}[colback=gray!5,colframe=gray!75!black,title=\textbf{Lane Marker: One New Idea at a Time}]
\footnotesize
Scene 1: Clocks \emph{only}.\\
Scene 2: Add \emph{Position}.\\
Scene 3: Combine \emph{Clocks + Position}. Reserve Boons, Corruption, and advanced magic for later sessions.
\end{tcolorbox}

% =========================================================
% SECTION: LEARNING THROUGH EXAMPLES
% =========================================================
\section{Learning Through Examples}

This section walks the same situation through multiple passes, adding one rule at a time. Use it as a mental template when you improvise.

\subsection*{Example 1: Core Loop Only}

\begin{tcolorbox}[colback=gray!3,colframe=gray!60,title=\textbf{Example 1 -- No SB, No Boons}]
Lyra tries to sneak past the guardian in a narrow hallway. (Agility + Stealth = 5d10).\\
The GM says it's a normal risk, DV 3, Controlled Position.\\
She rolls and gets: \texttt{3, 5, 6, 7, 9} $\Rightarrow$ 3 successes (6, 7, 9).\\[0.3em]
DV 3 with 3 successes is a \textbf{Success}.\\[0.3em]
\textbf{GM:} ``You slip past the guardian quietly. You're in position behind it if you want to act.''
\end{tcolorbox}

\subsection*{Example 2: Add Story Beats Only}
\label{sec:ex-sb-only}

\begin{tcolorbox}[colback=gray!3,colframe=gray!60,title=\textbf{Example 2 -- Adding Story Beats}]
Same situation. Lyra rolls 5d10 and gets: \texttt{1, 4, 6, 6, 8}.\\
Successes: 3 (6, 6, 8). She also rolled a \textbf{1}, which makes 1 Story Beat (SB) for the GM.\\[0.3em]
DV 3 with 3 successes is still a \textbf{Success}, but the GM now has 1 SB.\\[0.3em]
\textbf{GM Spend (1 SB = Minor Complication):}\\
\textit{``You get past the guardian, but your cloak snags and tears. The sound echoes -- the next room has a -1 Position penalty for stealth checks.''}
\end{tcolorbox}

\subsection*{Example 3: Add Boons Only}

\begin{tcolorbox}[colback=gray!3,colframe=gray!60,title=\textbf{Example 3 -- Adding Boons}]
Later, Lyra attempts to scale a loose stone wall (Athletics + Agility = 4d10). DV 3.\\
She rolls: \texttt{2, 3, 4, 5}. Zero successes = \textbf{Miss}.\\[0.3em]
Instead of only consequence, the GM awards \textbf{2 Boons}.\\
\textbf{GM:} ``You slip and land hard, taking 1 Harm. Mark 2 Boons -- you can use them later to turn the tide.''\\[0.3em]
On her next attempt, Lyra spends 1 Boon to re-roll a die, turning a failure into a success. The player feels the sting of failure \emph{and} the promise of payoff.
\end{tcolorbox}

\subsection*{Example 4: Putting It Together}

\begin{tcolorbox}[colback=gray!3,colframe=gray!60,title=\textbf{Example 4 -- SB + Boons + Position}]
The party is fleeing as the tower begins to collapse. A crumbling staircase stands between them and the exit.\\
\textbf{GM:} ``This is Desperate. DV 4 to get everyone across safely.''\\
Lyra leads, rolling 6d10 (Agility + Athletics + help). She gets: \texttt{1, 2, 6, 7, 7, 9}.\\[0.3em]
\textbf{Readout:}
\begin{itemize}
\item 4 successes (6, 7, 7, 9) vs DV 4 = Success.
\item 1 rolled \textbf{1} = 1 SB to the GM.
\item Desperate Position means high risk if there had been fewer successes.
\end{itemize}
\textbf{GM:} ``You all make it across, but the last step collapses behind you. I spend 1 SB: your exit path is gone. You'll need a new way out.''\\[0.3em]
\textbf{Lesson:} Even on a success, SB let you bend the fiction toward drama without erasing the player's win.
\end{tcolorbox}

% =========================================================
% SECTION: CORE RESOLUTION CYCLE
% =========================================================
\section{The Core Resolution Cycle}

When a player rolls, they engage the world through risk, consequence, and discovery. Follow these steps:

\begin{enumerate}
\item \textbf{Declare Action \& Approach:} Player states intent, Attribute + Skill.
\item \textbf{Set Difficulty Value (DV):} Based on narrative stakes.
\item \textbf{Establish Position:} Dominant, Controlled, or Desperate.
\item \textbf{Roll Pool of d10s.}
\item \textbf{Count Successes (6+) and Story Beats (1s).}
\item \textbf{Check Against DV} using Outcome Matrix.
\item \textbf{Apply Outcome:} Success, Partial, or Miss.
\item \textbf{Spend SB} for complications and twists.
\end{enumerate}

\paragraph{\textbf{Success.}}
The action succeeds as intended.
No Story Beats are generated unless specified by the roll or effect.
The fiction advances cleanly in the player’s favor.

\paragraph{\textbf{Success with Cost.}}
The action succeeds, but generates \textbf{Story Beats}.
The success stands; the cost is represented by the GM spending those Story Beats,
ideally applied to the inciting action or its immediate consequences.

\paragraph{\textbf{Miss.}}
The action fails to achieve its intended effect and generates \textbf{Story Beats}.
The GM uses those Story Beats to introduce complications, advance opposition,
or shift the situation against the character.

\paragraph{\textbf{Partial Success} (GM Adjudication).}
A Partial indicates that the action makes progress, but does not fully resolve the situation.
How that progress manifests is determined by the GM, based on context, scale, and narrative stakes.

Partial results may:
\begin{itemize}
  \item Improve the character’s \textbf{Position} on a follow-up attempt
  \item Reduce the effective \textbf{DV} of the next roll
  \item Achieve the goal at reduced scale or limited duration
  \item Succeed fully, but generate a \textbf{Story Beat} as cost or complication
  \item Reveal information or create a new opportunity instead of immediate resolution
\end{itemize}

\paragraph{Example: Lockpicking.}
A character attempts to pick a DV~3 lock.

\begin{itemize}
  \item With \textbf{1 success} (Partial), the GM rules the lock is loosened but not opened.
        The character gains \emph{Controlled Position} on the next attempt.
  \item With \textbf{2 successes} (Strong Partial), the lock opens — 
        but the GM chooses to spend the generated \textbf{Story Beat},
        upgrading the result to a \emph{Success with Cost}
        (the lock jams behind them, or someone hears the click).
\end{itemize}

\paragraph{Guiding Principle.}
Partials represent \emph{meaningful progress}. 
The GM may convert progress into advantage, resolution, or complication,
depending on what best serves the fiction.

\begin{fatebox}[Position Effects]
\begin{tabularx}{\textwidth}{lX}
\toprule
\textbf{Position} & \textbf{Mechanical Effect} \\
\midrule
Dominant & May re-roll one failure (die < 6). \\
Controlled & Default state; no re-rolls. \\
Desperate & Must re-roll one success (die 6+), keeping the second result. \\
\bottomrule
\end{tabularx}
\end{fatebox}

\begin{fatebox}[Difficulty Ladder -- Beginner Focus]
\begin{tabularx}{\textwidth}{lX}
\toprule
\textbf{DV} & \textbf{When to Use} \\
\midrule
2 & Routine: Clear path, no pressure, almost guaranteed. \\
3 & Default: Mild challenge, some risk, use this most often. \\
4 & Hard: Serious opposition, bad angle, or strong resistance. \\
5+ & Extreme: Save for major boss encounters or dramatic gambles. \\
\bottomrule
\end{tabularx}
\end{fatebox}

% =========================================================
% SECTION: STORY BEATS
% =========================================================
\section{Story Beats: The Engine of Drama}\label{sec:story-beats}

Every time a player rolls a 1, you gain Story Beats (SB). These are narrative tools, not punishments.

\begin{fatebox}[SB Spend Menu]
\begin{tabularx}{\textwidth}{lX}
\toprule
\textbf{SB Cost} & \textbf{Simple Complications} \\
\midrule
1 SB & Minor: Noise, distraction, small setback, lost gear. \\
2 SB & Moderate: Alarm raised, new threat enters, lose advantage. \\
3+ SB & Major: Reinforcements, scene shifts, stakes escalate. \\
\bottomrule
\end{tabularx}
\end{fatebox}

\subsection*{Failing Forward with SB}

\begin{itemize}
\item \textbf{On Success:} Use SB to add texture. The hero wins, but something changes.
\item \textbf{On Partial:} The hero gets what they wanted \emph{and} you spend SB to add cost or threat.
\item \textbf{On Miss:} You can spend SB to make the setback sharper, broader, or longer-lasting.
\end{itemize}
\begin{tcolorbox}[colback=gray!5,colframe=gray!75!black,title=\textbf{Lane Marker: One Complication per Roll}]
  When in doubt, spend SB on \emph{one} clear complication instead of many small ones.  
  Name it plainly—``Guard Alerted'', ``Floor Cracked'', ``Oath Owed''—and move on.
  
  \medskip
  Be \textbf{aggressive} with Story Beats, but never arbitrary:
  \begin{itemize}[noitemsep]
  \item Do not hoard SB—unused pressure is wasted momentum.
  \item Do not invent contrived consequences that ignore the fiction.
  \item Spend SB to \emph{sharpen what already matters}, not to add noise.
  \end{itemize}
  
  A single, well-chosen complication should change the situation, not just decorate it.
  \end{tcolorbox}

% =========================================================
% SECTION: COMBAT SIMPLIFIED
% =========================================================
\section{Combat Made Simple}

Combat uses the same core loop. The only difference is that the stakes are higher and usually more immediate.

\begin{fatebox}[Combat Quick Reference]
\begin{itemize}
\item \textbf{Initiative:} No fixed order. Ask: ``Who acts next?'' Follow the fiction.
\item \textbf{Actions:} Each turn, a character can move and take one meaningful action.
\item \textbf{Position Matters:} Flanking or ambush $\rightarrow$ Dominant; surrounded or exposed $\rightarrow$ Desperate.
\item \textbf{Use Clocks:} Track enemy morale [6], environmental dangers [4], and boss thresholds.
\end{itemize}
\end{fatebox}

\subsection*{Three-Beat Combat Structure}

When you feel lost in a fight, fall back on three beats:

\begin{enumerate}
\item \textbf{Opening Exchange:} Establish where everyone is and what they want.
\item \textbf{Turning Point:} Spend SB or tick clocks to change the situation (reinforcements, terrain shifts).
\item \textbf{Final Gamble:} Make the last few rolls matter -- higher DV, Desperate Position, or big rewards.
\end{enumerate}

\begin{tcolorbox}[colback=gray!5,colframe=gray!75!black,title=\textbf{Example: SB in Combat}]
\footnotesize
The party fights bandits on a collapsing bridge.\\
On a Partial, the GM spends 2 SB: ``You push them back, but the bridge loses another support. The \emph{Bridge Collapse [4]} clock advances.''\\
Now every action is also about whether they escape in time.
\end{tcolorbox}



% =========================================================
% SECTION: CLOCKS & FRONTS
% =========================================================
\section{Clocks \& Fronts}\label{sec:clocks}

Clocks are visual trackers for threats, progress, and looming changes. They keep pressure visible and shared.

\subsection*{Basic Clock Usage}

\begin{itemize}
\item \textbf{Size:} Use [4] for small scenes, [6] for bigger problems, [8] for arcs.
\item \textbf{Advance On:} Misses, certain Partials, or SB spends.
\item \textbf{Name It:} ``Alarm Raised [4]'' or ``Floodwaters Rise [6]'' is better than a blank circle.
\end{itemize}

\begin{fatebox}[Beginner Clock Limits]
\begin{itemize}
\item No more than 3 active clocks per scene.
\item Only 1--2 clocks should matter to the current roll.
\item Say out loud when you tick a clock and why.
\end{itemize}
\end{fatebox}

\subsection*{Mini-Fronts for Short Arcs}

A \emph{Front} is a cluster of related clocks and threats.

\begin{itemize}
\item \textbf{Example Front:} ``Rival Delvers of the Sunstone Tower''
\item Clocks: ``Rivals Close In [6]'', ``Tower Integrity [6]'', ``Rival Reputation [4]''.
\item Each time you spend SB, consider ticking one of these clocks.
\end{itemize}

% =========================================================
% SECTION: MODULAR PATHWAYS
% =========================================================
\section{Ready for More? Add These Systems}

Once you're comfortable with the core loop, introduce these modules one at a time. Treat each as a new lane marker.

\subsection*{Boons}\label{sec:boons}
Boons reward players for pushing their luck.

\begin{itemize}
\item On \textbf{Miss:} Gain 2 Boons.
\item On \textbf{Partial:} Gain 1 Boon.
\item \textbf{Spend:} Re-roll a die, add +1 Effect, or seize a small advantage (GM-approved).
\end{itemize}

Start with \emph{re-roll only}. Once players are comfortable, add the other options.

\subsection*{Advanced Magic}
\begin{itemize}
\item \textbf{Basic:} Use simple Tag-based magic (single roll, single tag).
\item \textbf{Intermediate:} Add Runekeepers after 2--3 sessions, when everyone understands SB and Clocks.
\item \textbf{Advanced:} Introduce Invokers and ritual magic for groups that enjoy planning and complex payoffs.
\end{itemize}

\subsection*{Followers, Assets, and Domains}
Add persistent resources when the group cares about territory, organizations, or long-term projects.

\begin{itemize}
\item Start with 1 simple Follower or Asset each.
\item Tie them to clocks: ``Caravan Reputation [6]'', ``Temple Influence [4]''.
\item Spend SB to threaten or complicate these resources.
\end{itemize}

% =========================================================
% SECTION: SESSION FLOW & CHECKLISTS
% =========================================================
\section{Session Flow: A GM Cognitive Checklist}

Use this as a quiet mental script. You do not need to say any of it out loud.

\subsection*{Opening 10 Minutes}

\begin{itemize}
\item Ask: ``What does everyone want out of tonight?'' (goal, vibe, focus).
\item Recap 2--3 key facts and 1 unresolved clock.
\item Ask each player: ``What is your character worried about right now?''
\end{itemize}

\subsection*{During Play}

\begin{itemize}
\item Before a roll: Name DV, Position, and stakes.
\item After a roll: Say the outcome type (Success/Partial/Miss) first, then the fiction.
\item Between scenes: Look at clocks. Ask, ``Which one should move? What changes in the world?''
\end{itemize}

\subsection*{Closing 10 Minutes}

\begin{itemize}
\item Ask: ``What was your favorite moment?'' (signals what to do more of).
\item Note any clocks that reached halfway or full.
\item Jot 2 bullet points: ``Next time on Fate's Edge...'' as hooks.
\end{itemize}

\begin{tcolorbox}[title=\textbf{Quick GM Cheat Sheet},colback=green!5,colframe=green!75!black]
\begin{itemize}
\item \textbf{DV Default:} 3 (adjust $\pm1$ for ease/difficulty).
\item \textbf{Position:} Controlled (normal), Dominant (advantage), Desperate (risk).
\item \textbf{SB Spend:} 1=minor, 2=moderate, 3+=major complication.
\item \textbf{Max Clocks:} 3 per scene to avoid overwhelm.
\item \textbf{Golden Rule:} When in doubt, make a ruling that keeps the story moving.
\end{itemize}
\end{tcolorbox}

\vspace{1em}
\noindent\textbf{Remember:} The goal is to tell an exciting story together. These rules are tools, not tests. Start simple, add complexity when you're ready, and keep your lane markers clear: one new idea at a time, one clear consequence per roll, and one or two clocks that really matter.

\newpage
% === Fate's Edge Systems Map (GM Guide size) ===
% Requires: \usepackage{tikz}
% and: \usetikzlibrary{arrows.meta,positioning}

\begin{figure*}[t]
  \begin{tikzpicture}[
      font=\small,
      every node/.style={align=center},
      core/.style={circle,draw=black,fill=black!85,text=white,thick,minimum size=2.7cm,inner sep=2pt},
      primary/.style={circle,draw=blue!70,fill=blue!15,thick,minimum size=2.2cm,inner sep=2pt},
      secondary/.style={circle,draw=blue!45,fill=blue!7,minimum size=1.6cm,inner sep=1pt},
      narrative/.style={rounded corners,draw=black,fill=yellow!18,thick,minimum width=3.1cm,minimum height=0.85cm,inner sep=3pt},
      arrow/.style={-Latex,thick},
      dottedarrow/.style={-Latex,thick,dashed},
  ]
  
  % --- Core ---
  \node[core] (core) {\textbf{Core Loop}\\[-0.2em]\footnotesize
  Action\\
  Position \& DV\\
  Roll d10\\
  Successes \& SB\\
  Outcome};
  
  % --- First ring ---
  \node[primary,above=2.4cm of core] (sb) {\textbf{Story}\\\textbf{Beats}\\[-0.2em]\footnotesize
  1s $\rightarrow$ SB\\
  Spend SB};
  
  \node[primary,right=2.9cm of core] (clocks) {\textbf{Clocks}\\[-0.2em]\footnotesize
  Scene \&\\
  Campaign};
  
  \node[primary,below=2.4cm of core] (modules) {\textbf{Player}\\\textbf{Modules}\\[-0.2em]\footnotesize
  O/C/L};
  
  \node[primary,left=2.9cm of core] (boons) {\textbf{Boons}\\[-0.2em]\footnotesize
  Miss=2\\
  Partial=1};
  
  % Core ↔ First ring
  \draw[arrow] (core) -- (sb);
  \draw[arrow] (core) -- (clocks);
  \draw[arrow] (core) -- (modules);
  \draw[arrow] (core) -- (boons);
  
  \draw[arrow] (sb) -- (core);
  \draw[arrow] (clocks) -- (core);
  \draw[arrow] (modules) -- (core);
  \draw[arrow] (boons) -- (core);
  
  % --- Second ring (specialized) ---
  \node[secondary,above right=1.7cm and 1.2cm of sb] (ob) {\textbf{Obl.}};
  \node[secondary,below right=1.7cm and 1.2cm of modules] (co) {\textbf{Corr.}};
  \node[secondary,below left=1.7cm and 1.2cm of modules] (le) {\textbf{Leash}};
  \node[secondary,above left=1.7cm and 1.2cm of sb] (su) {\textbf{Supply}};
  
  \node[secondary,above=1.55cm of core] (fa) {\textbf{Fatigue}\\\textbf{/Harm}};
  \node[secondary,right=2.0cm of core] (re) {\textbf{Res.}};
  \node[secondary,below=1.55cm of core] (ma) {\textbf{Mand.}};
  \node[secondary,left=2.0cm of core] (cr) {\textbf{Crisis}};
  
  % Links (keep minimal to avoid spaghetti)
  \draw[arrow] (modules) -- (ob);
  \draw[arrow] (modules) -- (co);
  \draw[arrow] (modules) -- (le);
  
  \draw[arrow] (clocks) -- (su);
  \draw[arrow] (clocks) -- (ma);
  \draw[arrow] (clocks) -- (cr);
  
  \draw[arrow] (su) -- (fa);
  \draw[arrow] (su) -- (re);
  
  \draw[arrow] (re) -- (boons);
  \draw[arrow] (re) -- (clocks);
  
  \draw[dottedarrow] (ob) -- (sb);
  \draw[dottedarrow] (co) -- (sb);
  \draw[dottedarrow] (le) -- (sb);
  
  \draw[arrow] (fa) -- (core);
  
  % --- Narrative layer (compact) ---
  \node[narrative,above=5.1cm of core] (na) {\textbf{Narrative Consequences}};
  \node[narrative,left=4.9cm of core] (ca) {\textbf{Campaign Evolves}};
  \node[narrative,right=4.9cm of core] (ag) {\textbf{Player Agency}};
  \node[narrative,below=5.1cm of core] (ch) {\textbf{Character Arcs}};
  
  \draw[arrow] (sb) -- (na);
  \draw[arrow] (clocks) -- (ca);
  \draw[arrow] (modules) -- (ag);
  \draw[arrow] (boons) -- (ag);
  
  \draw[arrow] (na) -- (core);
  \draw[arrow] (ca) -- (core);
  \draw[arrow] (ag) -- (core);
  \draw[arrow] (ch) -- (core);
  
  \end{tikzpicture}
  \caption{Fate’s Edge systems: the Core Loop drives SB/Clocks/Modules/Boons; pressure systems store consequences; narrative outcomes feed back into play.}
  \label{fig:systems-map}
  \end{figure*}

\clearpage

\subsection*{The Lazy DV Table (GM Guidance)}
\index{DV!lazy table}
\index{GM guidance!DV}

Use this table when you need a quick ruling.  
DV represents the \emph{significance of the challenge}, not the intrinsic difficulty of the task.

\begin{center}
\small
\begin{longtable}{clp{7.2cm}}
\toprule
\textbf{DV} & \textbf{Label} & \textbf{When to Use It} \\
\midrule
1 & Trivial & Routine actions; roll only if failure would be interesting. \\
2 & Easy & Competent effort; setbacks cause inconvenience or exposure. \\
3 & Standard & Meaningful challenge with real stakes. \\
4 & Hard & Time pressure, opposition, danger, or contested effort. \\
5 & Extreme & High-risk actions where failure escalates the situation. \\
6 & Legendary & Apex-tier challenges, climactic scenes, or mythic feats. \\
\bottomrule
\end{longtable}
\end{center}

\paragraph{Default Rule:}
For a meaningful challenge, assume \textbf{DV = Tier + 2}.  
If the obstacle is beneath the characters’ tier, do not roll.

Difficulty comes from \textbf{Position}, \textbf{Clocks}, and \textbf{Consequences}—not higher DVs.

\subsection*{Lazy Consequence Mapping}
\index{Consequences!lazy mapping}

Consequences should scale with \emph{pressure and tier}, not punish routine failure.

\begin{center}
\small
\begin{longtable}{p{3cm}p{9cm}}
\toprule
\textbf{Pressure} & \textbf{Default Consequence} \\
\midrule
Low & Fictional complication, lost time, new exposure \\
Moderate & \textbf{Fatigue (1)}, Clock advance, Position downgrade \\
High & \textbf{Fatigue (2)}, Harm~1, major Clock advance \\
Severe & Harm~2+, capture, separation, or irreversible loss \\
\bottomrule
\end{longtable}
\end{center}

\paragraph{Scaling Guidance:}
\begin{itemize}
\item At higher Tiers, prefer \textbf{Fatigue}, \textbf{Clock pressure}, and \textbf{collateral fallout} over raw Harm.
\item Harm represents failure to manage risk—not routine opposition.
\item Equipment loss, asset damage, or narrative loss should follow clear fictional cause.
\end{itemize}

\textbf{Harm is a consequence of escalation, not a default penalty.}

\subsection*{Why Fatigue Comes Before Harm}
\index{Fatigue!design philosophy}
\index{Harm!design philosophy}

\paragraph{Design Intent.}
Fatigue represents strain, stress, and narrowing options.  
Harm represents injury, trauma, or lasting damage.

Fatigue:
\begin{itemize}
\item Pressures future rolls
\item Forces resource decisions (Boons, Talents, retreat)
\item Encourages pacing and teamwork
\item Keeps characters active in the scene
\end{itemize}

Harm:
\begin{itemize}
\item Removes options
\item Shortens scenes
\item Risks sidelining characters
\item Escalates toward recovery or downtime
\end{itemize}

\paragraph{GM Guidance.}
Use Fatigue to say:  
\emph{“You can keep going, but it will cost you.”}

Use Harm to say:  
\emph{“You cannot ignore this.”}

If a consequence does not meaningfully change player decisions, it should not be Harm.



\subsection*{On Equipment Damage (Read This!)}
\index{Equipment!damage}
\index{GM guidance!equipment}

\paragraph{Default Rule:}
Do \textbf{not} compromise equipment trivially.

Equipment damage should be:
\begin{itemize}
\item Rare
\item Telegraph\-ed
\item Thematic
\item Tied to meaningful stakes
\end{itemize}

\paragraph{Why.}
Trivial equipment loss:
\begin{itemize}
\item Feels punitive rather than dramatic
\item Creates bookkeeping without tension
\item Undermines player investment
\item Disproportionately punishes martials
\end{itemize}

\paragraph{Better Alternatives.}
Before damaging gear, consider:
\begin{itemize}
\item Temporary disadvantage (\emph{jammed}, \emph{misaligned}, \emph{unstable})
\item Increased Fatigue to compensate for strain
\item Clock advancement representing wear or attention
\item Forced repositioning or exposure
\end{itemize}

\paragraph{When to Damage Equipment.}
Equipment damage is appropriate when:
\begin{itemize}
\item The fiction clearly targets the item
\item A major Story Beat is spent
\item The moment is climactic or symbolic
\item The player knowingly risked it
\end{itemize}

\emph{Example:}  
A shattered shield during a last stand is meaningful.  
A snapped sword on a routine miss is not.

\subsection*{DV Is Not Difficulty — It Is Commitment}
\index{DV!philosophy}

A higher DV does \emph{not} mean:
\begin{itemize}
\item harder math
\item better challenge
\item greater realism
\end{itemize}

A higher DV means:
\begin{itemize}
\item fewer partials
\item fewer Boons
\item faster escalation
\item less room to recover
\end{itemize}

\paragraph{GM Heuristic.}
If you want:
\begin{itemize}
\item more tension → add Clocks
\item more danger → spend Story Beats
\item more drama → worsen Position
\item more consequence → choose harsher costs
\end{itemize}

Raise DV only when the \emph{fiction itself} demands commitment.

\subsection*{High-Tier Reminder}
\index{High Tier play!DV}

At Tier IV–V:
\begin{itemize}
\item dice pools are larger
\item Boons are scarcer
\item consequences carry greater narrative weight
\end{itemize}

Do \textbf{not} inflate DV to compensate.

High-tier difficulty emerges naturally from:
\begin{itemize}
\item Boon scarcity
\item asset activation costs
\item multi-threaded Clocks
\item long-tail consequences
\end{itemize}

Let power feel earned, strained, and visible—never trivial.

\paragraph{GM Mantra.}
\begin{quote}
\emph{Set the DV low.  
Make the world react honestly.  
Let the pressure do the work.}
\end{quote}
\clearpage

\section*{First Session Cheat Sheet: The Pressure Engine in Action}
\index{Core Procedures!Checklist}
\index{GM guidance!First Session}

\begin{fatebox}[The Pressure Engine at a Glance]

This table distills the core mindset into actionable steps for your first session.  
It complements---not repeats---the principles in \emph{How to Think in Fate’s Edge} by providing concrete actions at critical moments.

\begin{center}
\small
\begin{longtable}{p{3.5cm}p{4cm}p{4.5cm}}
\toprule
\textbf{Critical Moment} & \textbf{What to Do} & \textbf{Why It Works} \\
\midrule
\textbf{Before the Session} &
Write the GM Mantra on a notecard:  
\emph{``Set the DV low. Make the world react honestly. Let the pressure do the work.''}  
Place it where you’ll see it. &
\textbf{Mindset matters first.}  
This primes you to trust the system before rules. \\
\midrule
\textbf{Introducing the Game} &
Say:  
\emph{``Your failures will create opportunities. When you roll a 1, you gain power to shape the story.''} &
\textbf{Reframes failure as narrative fuel.}  
Stops players from fearing rolls. \\
\midrule
\textbf{Setting the Scene} &
Before any roll:
\begin{enumerate}[noitemsep,leftmargin=*]
\item Listen: ``What are you actually doing?''
\item Assign Position
\item State it aloud: ``This is \emph{[Position]}.'' 
\end{enumerate}
&
\textbf{Makes Position the anchor.}  
This is the single most important step---the engine starts here. \\
\midrule
\textbf{When Players Roll} &
Note every \textbf{1} immediately.  
Ask: ``What does this mean for the story?'' &
\textbf{1s are the narrative engine.}  
Never skip this step—they create momentum. \\
\midrule
\textbf{On Partial Success} &
Don’t say ``You succeed, but\ldots''  
Say: ``You succeed at a cost. What does this success cost?'' &
\textbf{Makes the scene dynamic.}  
This is where the story grows. \\
\midrule
\textbf{When Setting DV} &
Start with \textbf{DV 3} as your default.  
Only adjust if the fiction demands it. &
\textbf{Prevents overcomplication.}  
90\% of scenes thrive at DV~3. \\
\midrule
\textbf{When Stuck} &
Ask: ``What’s at stake right now?''  
Not: ``How hard should this be?'' &
\textbf{Aligns with the core system logic.}  
This replaces difficulty math with pressure framing. \\
\midrule
\textbf{After the Session} &
Write one sentence:  
\emph{``What I learned: [your insight].''} &
\textbf{Cements the mindset.}  
Focus on what matters, not what went ``wrong.'' \\
\midrule
\textbf{When in Doubt} &
Set Position to \textbf{Controlled}, DV to \textbf{3}, and ask:  
``What happens if this goes wrong?'' &
\textbf{The golden rule.}  
This solves 95\% of first-session dilemmas. \\
\bottomrule
\end{longtable}
\end{center}

\paragraph{Final Note for GMs}
This table is not a rulebook.  
It is a compass.  
Use it until the rhythm of play becomes second nature.

\end{fatebox}
\clearpage
\section*{Your First Combat: A GM Cheat Sheet}
\index{Combat!First Session}
\index{GM Guidance!Combat}

\begin{center}
\small
\begin{longtable}{>{\bfseries}p{3cm} p{10cm}}
\toprule
\textbf{Stage} & \textbf{What to Do (Simplified)} \\
\midrule
\endfirsthead

\toprule
\textbf{Stage} & \textbf{What to Do (Simplified)} \\
\midrule
\endhead

\midrule
\endfoot

\bottomrule
\endlastfoot

Setup &
Describe the immediate situation clearly:  
``You’re in a narrow hallway, back to the wall. Three bandits advance.''  

Skip initiative. Follow narrative flow—who acts first should be obvious.  

Start at \textbf{Controlled Position} unless the fiction strongly says otherwise. \\

Position &
\textbf{Dominant}: Clear advantage (surprise, leverage, flanking).  

\textbf{Controlled}: Normal fighting (\emph{most rolls}).  

\textbf{Desperate}: Cornered, outnumbered, or everything at stake.  

\textbf{The GM always states Position before the roll.}  
``This is Desperate. If it goes wrong, you’ll be disarmed.'' \\

Dice Pool &
Roll \textbf{Attribute + Skill} (e.g., Body + Melee).  

No extra math. No stacking modifiers.  

Player states approach in fiction:  
``I charge in with my sword.'' \\

Outcomes &
\textbf{Clean Success}: Goal achieved cleanly.  

\textbf{Success with Cost}: Goal achieved; spend 1 SB for a complication.  

\textbf{Partial}: Progress at reduced scale, improved position, or a hard choice.  

\textbf{Miss}: No progress; spend SB to escalate the situation. \\

Story Beats &
Spend SB aggressively but cleanly—never arbitrarily:  

1 SB — Minor complication  
\emph{``Steel sparks; you gain ground, but the noise carries.''}  

2 SB — Moderate setback  
\emph{``Your ally is knocked off-balance; their next roll is Desperate.''}  

3+ SB — Major shift  
\emph{``Reinforcements arrive. New Clock: Escalation [4].''}  

Always spend on \emph{one clear consequence per roll}. \\

Clocks &
Use \textbf{one clock maximum} in the first combat  
(e.g., ``Enemy Harm [4]'').  

Advance clocks on Miss or Success-with-Cost.  

All clocks should be visible to players.  

Ignore layered or tactical clocks for Session One. \\

Multiple Enemies &
Treat a group as \textbf{one threat} unless individuals act differently.  

One clock covers the group.  
If one foe is distinct, give them their own clock. \\

Boons &
Players may spend 1 Boon to:  

• Re-roll a single die  
• Improve Position by one step  
• Activate an on-scene Asset  

Players gain \textbf{2 Boons on a Miss}, \textbf{1 Boon on a Partial}. \\

What to Ignore &
No miniatures or grids — theater of the mind only.  

No advanced maneuvers or combos.  

No fatigue-position interactions yet.  

No multi-clock combat frameworks. \\

Next Session &
Add \emph{one} new element:  

• A second clock, or  
• Environmental pressure, or  
• A structured combat arc  

Introduce the Three-Beat Combat structure later. \\

\end{longtable}
\end{center}

\paragraph{First Combat Rule of Thumb.}
Describe the situation → State Position → Roll → Apply outcome → Spend SB once → Move forward.

If the scene is moving, you’re doing it right.
\clearpage