\chapter{Running the Game: Core Procedures}\index{core procedures}

In \textbf{Fate's Edge}, the game flows through a series of \textbf{actions, consequences, and escalating stakes}. As the GM, your role is to guide this flow---not by dictating outcomes, but by \textbf{framing scenes, interpreting rolls, and spending Story Beats}\index{Story Beats} to keep tension alive. This chapter walks you through the core procedures that define play, from the moment a player declares an action to the fallout that follows.

\section{Scene Framing: Start with Stakes}\index{scene framing}

Every scene begins with a question: \textbf{What's at risk?} Not just for the characters, but for the world, the mission, or the fragile alliances they've built. As the GM, you frame the scene by establishing:

\begin{itemize}
    \item \textbf{Position}\index{position}: Is the action \textit{Dominant}\index{position!Dominant}, \textit{Controlled}\index{position!Controlled}, or \textit{Desperate}\index{position!Desperate}?
    \item \textbf{Effect}: What happens on a success? What changes?
    \item \textbf{Stakes}: What is gained---or lost---if things go wrong?
\end{itemize}

A scene in the \textbf{Mistlands} might begin with the PCs crossing a flooded causeway at dusk. The bell-line hums with tension. The GM sets the position as Controlled---slippery stones, rising mist, and the distant echo of a wraith-call. A failure here could mean separation, exposure, or worse.

\subsection{Position Descriptions}

\begin{itemize}
    \item \textbf{Dominant}: You act on your terms. Complications are minor, setbacks are rare.
    \item \textbf{Controlled}: You act under pressure. Success is possible, but failure brings a cost.
    \item \textbf{Desperate}: The odds are stacked against you. Success is hard-won, and failure is dramatic.
\end{itemize}

Use position to guide the fiction. A controlled entry into a noble salon in \textbf{Vhasia} might allow the PCs to charm or intimidate without resistance. A desperate one---perhaps after triggering an alarm---means blades are drawn before words.

\section{Adjudicating Rolls: The Core Resolution Cycle}\index{roll adjudication}

When a player rolls, they are not simply trying to \emph{beat a number}. They are engaging the world through risk, consequence, and discovery. This section walks through the full cycle.

\subsection{Step-by-Step Roll Resolution}

\begin{enumerate}
    \item \textbf{Declare Action \& Approach:} Player states intent, Attribute + Skill.
    \item \textbf{Set Difficulty Value (DV):}\index{Difficulty Value (DV)} Based on narrative stakes, not just mechanics.
    \item \textbf{Establish Position:}\index{Position} GM sets whether the action is \textbf{Dominant}, \textbf{Dominant}, or \textbf{Desperate}.
    \item \textbf{Roll Pool of d10s.}
    \item \textbf{Count:} \textbf{Successes (6+)} and \textbf{Story Beats (1s)}.\index{Story Beats}
    \item \textbf{Check Against DV:} Apply the Outcome Matrix. Note: \textbf{each 10 counts as 2 successes}.
    \item \textbf{Spend SB:} GM spends/banks Story Beats or draws from the Deck of Consequences.\index{Deck of Consequences}
\end{enumerate}

\begin{fatebox}[Position Effects]\index{Position}
\begin{tabularx}{\textwidth}{lX}
\toprule
\textbf{Position} & \textbf{Effect} \\
\midrule
Dominant & May re-roll one \textbf{failure} (die below 6). \\
Dominant & Normal roll; no rerolls. \\
Desperate & Must re-roll one \textbf{success} (6+), keeping the second result. \\
\bottomrule
\end{tabularx}
\end{fatebox}

\begin{fatebox}[Difficulty Ladder]\index{Difficulty Ladder}
\begin{tabularx}{\textwidth}{lX}
\toprule
\textbf{DV} & \textbf{Typical Case} \\
\midrule
2 & Routine: clear intent, modest stakes, controlled setting \\
3 & Pressured: time limits, mild resistance, partial info \\
4 & Hard: hostile conditions, active opposition, precision required \\
5+ & Extreme: stacked constraints, dangerous failure, high drama \\
\bottomrule
\end{tabularx}
\end{fatebox}

\begin{fatebox}[Outcome Matrix]\index{Outcome Matrix}
\begin{tabularx}{\textwidth}{lX}
\toprule
\textbf{Result} & \textbf{GM Guidance} \\
\midrule
$S \geq DV$, $C = 0$ & Clean Success: Grant intent, no added friction. \\
$S \geq DV$, $C > 0$ & Success \& Cost: Intent achieved; GM spends SB for complications. \\
$0 < S < DV$ & Partial: Progress \emph{proportional} to hits; intent advances but with gaps or risk. Player gains 1 Boon. \\
$S = 0$ & Miss: No progress. GM escalates with SB/Clocks. Player gains 2 Boons. \\
\bottomrule
\end{tabularx}
\end{fatebox}

\begin{tcolorbox}[title=\textbf{Critical Success \& GM Heat},colback=white!97!gray,colframe=black!80!gray,sharp corners,boxrule=0.4pt]
\textbf{Rule Summary.} A \textit{Critical Success} represents a decisive shift in control, while \textit{GM Heat} measures the world's reaction to the heroes' growing dominance.

\begin{description}
  \item[\textbf{Crit Effect.}] On a Critical Success, immediately \textbf{raise your Position} by one step:
  \[
  \text{Desperate} \rightarrow \text{Risky} \rightarrow \text{Controlled} \rightarrow \textbf{Dominant}.
  \]
  If already at \textit{Dominant}, gain \textbf{+1 Success} instead.\\
  Each Crit also adds \textbf{+1 Heat} to the GM’s pool.
  
  \item[\textbf{Dominant Position.}] Dominant is the highest attainable Position, representing decisive advantage. While Dominant:
  \begin{itemize}
    \item All actions begin with one automatic success.
    \item Further Crits add $+1$ Success per excess tier.
    \item Any Complication, Consequence, or GM Heat spend immediately lowers Position by one step.
  \end{itemize}
  
  \item[\textbf{GM Heat.}] 
  Heat reflects rising narrative tension. It resets to 0 at the end of each scene.
  \begin{itemize}
    \item \textbf{Gain 1 Heat:} Whenever any player scores a Crit.
    \item \textbf{Spend 1 Heat:} 
      \begin{itemize}
        \item Degrade a PC’s Position by one tier.
        \item Introduce a Complication or Clock tick.
        \item Mirror or escalate a previous Story Beat.
      \end{itemize}
    \item \textbf{Spend 2 Heat:} Manifest a major twist, environmental hazard, or factional response.
  \end{itemize}
  
  \item[\textbf{Balance and Flow.}]
  \begin{itemize}
    \item Spending \textit{Momentum} or invoking a Rite while Dominant consumes your edge, returning Position to Controlled.
    \item GM Heat ensures rising power is met with proportional world response, maintaining narrative tension.
  \end{itemize}
\end{description}

\textbf{Result Bands.}
\begin{center}
\renewcommand{\arraystretch}{1.15}
\begin{tabular}{l l l}
\toprule
\textbf{Result} & \textbf{Effect} & \textbf{Position Shift}\\
\midrule
Miss (1--3) & Failure + Cost & $\downarrow$ 1 tier\\
Weak Hit (4--6) & Success + Consequence & ---\\
Strong Hit (7--9) & Clean Success & ---\\
\textbf{Crit (10 or 2 at 8+)} & Position Bump + +1 Heat (GM) & $\uparrow$ 1 tier\\
\textbf{Crit while Dominant} & +1 Success + +1 Heat (GM) & ---\\
\bottomrule
\end{tabular}
\end{center}

\textbf{Design Intent.} 
This rule transforms Crits from “bigger wins” into shifts in control and tension. 
Players’ momentum raises their Position, while GM Heat keeps the narrative world responsive, ensuring ebb and flow without trivializing challenge.
\end{tcolorbox}

\section{Story Beats: The Engine of Drama}\index{Story Beats}

Every time a player rolls a \textbf{1}, a Story Beat is generated. These are not mere penalties---they are narrative levers. Spend them to:

\begin{itemize}
    \item Escalate a threat (drawing more enemies, raising the stakes).
    \item Drain resources (time, gear, positioning).
    \item Reveal hidden dangers or betrayals.
    \item Cause collateral damage or unintended consequences.
\end{itemize}

Story Beats should \textbf{push the story forward}, not grind it to a halt. Use them to add pressure, not to punish.

\begin{fatebox}[SB Spend Menu]\index{SB Spend Menu}
\begin{tabularx}{\textwidth}{lX}
\toprule
\textbf{SB Cost} & \textbf{Example Complications} \\
\midrule
1 SB & Minor pressure: noise, trace, +1 Supply segment, brief distraction \\
2 SB & Moderate setback: alarm raised, lose position/cover, lesser foe appears \\
3 SB & Serious trouble: reinforcements arrive, key gear breaks, tactical disadvantage \\
4+ SB & Major turn: trap springs, authority intervenes, scene shifts dramatically \\
\bottomrule
\end{tabularx}
\end{fatebox}

\subsection{When to Draw from the Deck of Consequences}\index{Deck of Consequences}

The Deck of Consequences is a powerful tool for \textbf{thematic consistency}. When a player generates SB, you may choose to:

\begin{itemize}
    \item \textbf{Direct Spend}: Translate SB into consequences/rail ticks immediately.
    \item \textbf{Deck Draw}: Draw up to \textbf{min(SB, 3)} cards and \textbf{synthesize a single twist} guided by suit and highest rank.
\end{itemize}

Never do both for the same roll. If the drawn card contradicts established fiction, reinterpret or redraw to fit the suit and tone.

\subsection{High-Tier SB Sinks}\index{High-Tier SB Sinks}
For 3--6+ SB spends that move the world (reputation cascades, faction instability, resonance, prophecy), see the stand-alone \emph{High SB Sinks} handout. A good default: at end of leg, \textbf{3 SB → tick 1 Front}\index{Front}.

\subsection{Banking \& Cashing SB}\index{banking SB}\index{cashing SB}

\begin{itemize}
    \item Banked SB should pay off within the same scene or arc.
    \item Avoid nickel-and-diming. Prefer one memorable complication over many petty penalties.
\end{itemize}

\section{Scene Management Tools}

\subsection{Scene Starters and Hooks}\index{scene starters}\index{hooks}

To keep the game moving, always open a scene with a strong hook:

\begin{itemize}
    \item "The alarm bells ring as you step into the courtyard."
    \item "A courier collapses at your feet, clutching a sealed scroll."
    \item "The tide is turning---the ghost-ferry won't wait."
\end{itemize}

Let the players react. Let the world respond. And always---\textbf{follow the consequences.}

\subsection{Setting Stakes Fast (Cheat Prompts)}\index{stakes setting}

\begin{itemize}
    \item If this goes right, what changes?
    \item If this goes wrong, what bites back?
\end{itemize}

\section{Bond-Driven Resource Generation}\index{bond-driven resource generation}\index{bonds!resource generation}

Players may earn boons by taking significant actions to aid bonded allies while providing intricate descriptions of how their bonds motivate their actions.

\subsection{Adjudication Guidelines}
\begin{itemize}
    \item \textbf{Mutual Bond:} Verify the player and ally share a defined bond
    \item \textbf{Intricate Description:} The description must meaningfully reference the bond's nature
    \item \textbf{Significant Aid:} The assistance must be substantial, not routine help
    \item \textbf{Fiction First:} The bond must genuinely explain the character's motivation
\end{itemize}

\subsection{GM Discretion}
\begin{itemize}
    \item Deny the boon if the action is trivial or the bond reference is superficial
    \item Encourage creative bond references that deepen character relationships
    \item Consider allowing this even when the aiding action fails, if the bond motivation was genuine
\end{itemize}

This mechanic reinforces collaborative play and character relationship development while providing meaningful mechanical rewards for roleplaying.

\subsection{Turn Economy (Quick Rules)}
\label{subsec:turn-economy-quick}

\paragraph{Two Actions.}
Each character takes \emph{1 Action and 1 Move} on their turn. Actions and Moves may be taken in any order; repeating the same Action is not allowed unless noted.

\paragraph{Move.}
Traverse up to your normal movement. \emph{Disengage:} move without provoking; your next offensive action is \textbf{Controlled}. \emph{Dash:} move again this turn; your next defense is \textbf{Desperate}.

\paragraph{Attack.}
Make a melee or ranged attack versus DV set by the GM and fiction. Teamwork/Assist costs 1 Boon.

\paragraph{Observe / Change Position (+1).}
Take a beat to read the field or set angles; gain \textbf{+1 Position} for one action this turn (e.g., Controlled$\to$Dominant). Limit: once/turn; cannot exceed \textbf{Dominant}.

\paragraph{Activate an Asset.}
Use gear, symbol, tool, or feature per its text/tags (e.g., torch, grapnel, smoke vial, rune focus). Items with \texttt{[Action]} consume one Action; \texttt{[Free]} do not.

\paragraph{Setup (Teamwork).}
Create advantage for an ally; on success, grant their next action \textbf{+1 Position} or step up Effect (GM’s call).

\paragraph{Assist (Teamwork).}
Spend \emph{1 Boon} to give an ally \emph{+1 die} on their current roll; you share appropriate risk/consequence.

\paragraph{Defend / Protect.}
Adopt a guarding stance or body-block. Choose a nearby ally; until your next turn you may intercept one hit on them and roll to resist it. On success, reduce/negate Harm; you take any fallout the GM assigns.

\paragraph{Channel / Weave.}
Runekeeper/ritual flow: \emph{Channel} (prime power) then \emph{Weave} (shape/release). Disruption or engagement may worsen Position; if \emph{Interrupted}, the casting fails.

\paragraph{Cast Rite / Song (Cantor).}
Perform a Rite/Song per its write-up. You may \emph{Push} to accelerate or empower at the cost of Fatigue/Corruption per class rules.

\paragraph{Interact.}
Lift, pull, flip a lever, shove a foe, break an object, apply a poultice, reload, draw/stow, etc. GM sets DV/Effect.

\paragraph{Free Items.}
Short shouts, dropping an item, quick glance. Longer or tactical assessments require \emph{Observe / Change Position} or \emph{Interact}.

\paragraph{Reactions (Out of Turn).}
\emph{Protection} may trigger when an ally is hit and you are in position. Class/Asset reactions fire as written (e.g., counter-runes, ripostes).

\paragraph{Position Caps.}
Bonuses cannot raise Position above \textbf{Dominant}; penalties cannot drop below \textbf{Desperate}. Beyond these caps, adjust DV or Effect instead.


\section{Integrated Combat Procedures}\index{combat procedures}

Combat in \textbf{Fate's Edge} follows the same core procedures as all other actions, but with specific applications for violent conflict. Every combat action generates potential for both triumph and complication, with consequences that cascade through the same economy as all other challenges.

\subsection{Combat Resolution Procedure}\index{combat resolution}

\begin{enumerate}
    \item \textbf{Declare Action}: Player states intent and approach (Attribute + Skill)
    \item \textbf{Set Position}: GM sets Dominant, Controlled, or Desperate based on tactical situation
    \item \textbf{Roll Dice}: Roll pool = Attribute + Skill (takes 1 Player Turn)
    \item \textbf{Count Results}: 6+ = Success, 1 = Story Beat (SB)
    \item \textbf{Apply Outcome}: Use standard Outcome Matrix
    \item \textbf{Manage Consequences}: GM spends SB or draws from Consequences Deck
\end{enumerate}

\subsection{Combat-Specific Position Applications}

\begin{itemize}
    \item \textbf{Dominant}: Advantageous position, minor consequences (flanking, higher ground, surprised foe)
    \item \textbf{Controlled}: Even odds, moderate consequences (evenly matched, contested terrain)
    \item \textbf{Desperate}: Disadvantaged, severe consequences (outnumbered, wounded, poor positioning)
\end{itemize}

\begin{fatebox}[Combat Consequence Types by Suit]\index{combat consequences}\index{consequence types!combat}
\begin{tabularx}{\textwidth}{lX}
\toprule
\textbf{Suit} & \textbf{Complication Themes} \\
\midrule
Hearts & Morale, fear, command breakdown, psychological pressure, loyalty tests \\
Spades & Physical harm, positioning changes, weapon status, tactical wounds, cover loss \\
Clubs & Resource depletion, gear damage, fatigue, ammunition issues, supply problems \\
Diamonds & Environmental hazards, reinforcements, terrain changes, unexpected events \\
\bottomrule
\end{tabularx}
\end{fatebox}

\begin{fatebox}[Harm Integration with SB Economy]\index{harm integration}\index{SB economy!harm}
\begin{tabularx}{\textwidth}{lX}
\toprule
\textbf{Harm Level} & \textbf{Effects \& SB Generation} \\
\midrule
Harm 1 & -1 die to related actions; generate 1 SB on next 2 rolls \\
Harm 2 & -1 die to most actions; generate 1 SB on next roll until treated \\
Harm 3 & -2 dice to relevant actions; generate 2 SB on next roll; incapacitation risk \\
\bottomrule
\end{tabularx}
\end{fatebox}

\begin{fatebox}[Tactical Clocks]\index{Tactical Clocks}
\begin{tabularx}{\textwidth}{lX}
\toprule
\textbf{Clock Type} & \textbf{Purpose \& Triggers} \\
\midrule
Mob Overwhelm [6] & Enemy numbers become advantage; tick when outflanked or surrounded \\
Morale Collapse [6] & Fear undermines effectiveness; tick after leader falls or shocking events \\
Environmental [8] & Terrain/fire/building failure; tick after destructive actions or time pressure \\
\bottomrule
\end{tabularx}
\end{fatebox}

\subsection{Initiative and Turn Order}

Fate's Edge does not use fixed initiative. 
Turn order follows the fiction and the GM's facilitation:
\begin{itemize}
    \item \textbf{Narrative Fiat:} The GM frames spotlight order based on circumstances, tension, and narrative flow.
    \item \textbf{Player Input:} Players may suggest acting when it makes sense in the fiction. 
    \item \textbf{Surprise:} Ambushers act first; targets respond after the opening exchange.
    \item \textbf{Flexibility:} Spotlight may shift mid-scene if fictionally appropriate (e.g., reacting to a falling ceiling, seizing a moment).
\end{itemize}

This ensures pacing and drama guide the sequence of actions, not rigid turn structures.

\subsection{Position Dynamics in Combat}\index{position dynamics}

Position can shift during combat based on SB spending and narrative triggers:

\begin{itemize}
    \item \textbf{GM Spend (1 SB)}: Shift position one step worse for current action
    \item \textbf{Player Spend (1 Boon)}: Shift position one step better or cancel GM shift
    \item \textbf{Narrative Triggers}: Flanking, reinforcement arrival, environmental changes, superior leverage
\end{itemize}

\begin{fatebox}[Tracking NPC Mechanics]
  Not every meter needs to be tracked for NPCs. 
  
  \begin{itemize}
      \item \textbf{Spotlight First:} NPCs only carry Obligation, Corruption, or similar mechanics if these traits matter to the current story.
      \item \textbf{Skip the Bookkeeping:} Do not track every enemy’s resource pool. If it’s not driving narrative tension, it can be abstracted away.
      \item \textbf{Focus on Impact:} Apply NPC Obligation or Corruption only when it changes how the party experiences them — e.g., a Patron visibly twisting a rival’s fate, or a recurring villain consumed by corruption.
      \item \textbf{Player-Facing First:} Keep full mechanics for PCs, since their journey is the story’s core.
  \end{itemize}
  
  This principle keeps GM effort focused where it matters: driving story beats and consequences, not filling ledgers.
  \end{fatebox}
  
\subsection{Magic Combat Integration}\index{magic combat}

Spellcasting in combat feeds the same consequence economy:

\begin{itemize}
    \item Channel/Weave Backlash SB applies to tactical situation
    \item Spells can shift position, create tactical clocks, or generate combat consequences
    \item Magic consequences cascade through existing combat systems
\end{itemize}

\subsection{Asset/Follower Combat Integration}\index{asset combat}\index{follower combat}

\begin{itemize}
    \item \textbf{Follower Risk}: 2+ SB spent in combat can endanger assisting followers
    \item \textbf{Asset Compromise}: Combat in certain locations can damage relevant assets  
    \item \textbf{Offensive Activation}: 1 Boon activates asset for combat advantage
    \item \textbf{Initiative Actions}: Followers can take combat-relevant independent actions
\end{itemize}

begin{fatebox}[Fatigue & Harm Reference]\index{Fatigue}\index{Harm}
\begin{tabularx}{\textwidth}{lX}
\toprule
\textbf{Fatigue Level} & \textbf{Effect} \
\midrule
0 & Fresh: no effect. \
1 & Winded: downgrade one Dominant roll to Controlled per scene. \
2 & Strained: Controlled rolls add +1 SB on 1s. \
3 & Exhausted: downgrade one Controlled roll to Desperate per scene. \
4 (Max) & Collapse: further exertion = DV 3 Body test or Severe Harm. \
\bottomrule
\end{tabularx}

\begin{tabularx}{\textwidth}{lX}
\toprule
\textbf{Harm Level} & \textbf{Casting Impact} \
\midrule
Minor & Concentration taxed; channeled effects DV +1. \
Moderate & Risk of disruption: roll DV 2 to maintain channels. \
Severe & Channel breaks; freeform casting auto-generates +1 SB. \
Critical & No further casting; Patron may impose terms for reprieve. \
\bottomrule
\end{tabularx}
\end{fatebox}

\subsection{Combat Outcome Matrix Application}

Same as standard resolution, but consequences are combat-specific:

\begin{itemize}
    \item \textbf{Clean Success}: Intent achieved with no tactical complications
    \item \textbf{Success \& Cost}: Intent achieved, but GM spends SB for combat consequences
    \item \textbf{Partial}: Progress with tactical fork (accept cost OR concede ground). Award one Boon.
    \item \textbf{Miss}: No progress; GM spends SB for combat consequences. Award two Boons.
\end{itemize}

\section{30-Second Adjudication Loop}\index{adjudication loop}

Use this fast procedure to keep the game flowing:

\begin{enumerate}
    \item \textbf{Clarify}: "What do you want, and how?"
    \item \textbf{Set Stakes}: "If it works, what changes? If it fails, what bites?" Start Controlled/Standard unless fiction says otherwise.
    \item \textbf{Roll \& Read}: Count successes (6+) and SB (1s). Compare to DV.
    \item \textbf{Spend One Beat Well}: Cash SB on one memorable twist or tick a relevant Clock.
    \item \textbf{Push Forward}: Describe how the fiction changes; ask "Who moves next?"
\end{enumerate}

\section{Position + Effect in Action}

A player declares a \textbf{Controlled} action to \textbf{pick a lock} while guards patrol above. They roll \textbf{Wits + Skullduggery}, get 3 successes and 1 Story Beat.

The GM consults the Outcome Matrix: \textbf{Success \& Cost}. The lock clicks---but a guard's bootstep halts above. The GM spends 1 SB to add tension: the patrol changes direction, heading toward the PCs' position.

\subsection{Fatigue}
\label{subsec:fatigue}
\index{Fatigue}

\textbf{Track:} Each character has a Fatigue track equal to \textbf{Body}. Mark Fatigue for exertion, strain, or backlash.
s
\textbf{In Play:} Each Fatigue step worsens your \textbf{Position} by one level 
(Controlled $\rightarrow$ Risky $\rightarrow$ Desperate). 
If you are already \textbf{Desperate}, instead apply a \textbf{--1 die} penalty per Fatigue to that roll.

\textbf{Overflow:} When your Fatigue track fills, immediately increase \textbf{Harm by 1 step} and clear all Fatigue to 0. 
If this raises Harm to a level that incapacitates you, you fall out of the scene as normal for Harm.

\textbf{Recovery:} Short rest clears 1--2 Fatigue; a full night's rest clears all Fatigue.

\section{Common Pitfalls and Solutions}\index{pitfalls}\index{GM tips}

\begin{fatebox}[Troubleshooting Common Issues]
\begin{tabularx}{\textwidth}{lX}
\toprule
\textbf{Issue} & \textbf{Solution} \\
\midrule
Over-cranking SB & Halve SB spends temporarily or convert to visible Clocks \\
Clock Sprawl & Merge redundant Clocks; scenes need only 2-3 active Clocks \\
Tag Paralysis & Paraphrase: "Sounds like [TAG]. DV 3. Want to roll?" \\
Rules Drift & Pick ruling that keeps flow, note TODO for post-session reconciliation \\
Boon Inflation & Enforce 2 Boon/scene cap from failures; use Repetition Rule \\
\bottomrule
\end{tabularx}
\end{fatebox}

\section{Miniatures and Tactical Layer}
\label{sec:miniatures}

\subsection{Core Concepts}
\begin{itemize}
  \item Works on square or hex grids; declare grid type at setup.
  \item Units have base sizes (Small, Medium, Large, Huge) and a facing.
  \item Actions per turn: Move and Act (attack, cast, interact, etc.), in either order.
  \item All checks use normal SRD roll + DV system.
\end{itemize}

\subsection{Turn Structure}
\begin{enumerate}
  \item Start: resolve ongoing effects.
  \item Move: up to Speed; obey Zones of Control (ZOC).
  \item Act: attack, test, assist, cast, rally, shove, guard, etc.
  \item End: resolve end effects and reactions.
\end{enumerate}

\subsection{Zones of Control (ZOC)}
\begin{itemize}
  \item \textbf{Squares:} 4 orthogonal adjacents (optional: 8). 
  \item \textbf{Hexes:} 6 adjacents.
  \item Large/Huge project ZOC from edges; Reach may extend ZOC by +1 ring.
  \item \textbf{Rules:} 
    \begin{itemize}
      \item Entering enemy ZOC ends movement (you are engaged).
      \item Cannot move through enemy ZOC.
      \item Leaving requires Disengage (DV 4–6) or spend 1 Boon.
      \item Multiple ZOCs increase DV by +1 per extra controller.
    \end{itemize}
\end{itemize}

\subsection{Facing and Flanking}
\begin{itemize}
  \item Choose a facing at end of movement.
  \item Flank: +1 die if attacked from opposite arcs; Rear: +1 die and +1 Effect.
\end{itemize}

\subsection{Special Actions}
\begin{itemize}
  \item \textbf{Guard:} Ready a strike when enemy leaves ZOC.
  \item \textbf{Dash:} +2 movement this turn.
  \item \textbf{Brace:} Resist Shoves/Pulls and extend ZOC (opportunity only).
  \item \textbf{Tackle:} Knock target prone (DV 4–6).
\end{itemize}

\subsection{Magic Integration}
\begin{itemize}
  \item Magic uses \textbf{[TAGS]} (e.g., [WARD], [BANISH], [CONJURE]) tied to ZOC, range, and LoS.
  \item Casting while engaged worsens Position unless [INSTANT] or aided by Talent.
  \item Rituals require clear space and visible Symbols; disrupted rituals fail or require a test.
\end{itemize}

\subsection{Quick Reference}
\begin{itemize}
  \item Entering enemy ZOC ends movement; leaving requires Disengage.
  \item Flank = +1 die; Rear = +1 die and +1 Effect.
  \item Difficult terrain +1 cost; moving up elevation +1.
  \item Boons may break ZOC rules: auto-Disengage, change facing, or Heroic Rush.
\end{itemize}

\subsection{Running Social Clocks (GM Guidance)}
\index{Social!Clocks}\index{Persuasion}\index{Bonds}\index{Story Beats}

\paragraph{Framing the Ask.}
State a concrete outcome (“grant passage tonight,” “drop the bounty,” “fund our expedition”). If it’s strictly binary and low-stakes, use one roll. Otherwise, build a clock.

\paragraph{Set Two Clocks.}
\begin{itemize}
  \item \textbf{Persuasion} (4/6/8): your progress.
  \item \textbf{Opposition} (4/6): their resistance (ego, risk, duty, rival’s whisper).
\end{itemize}
Name them (“\textsc{Council Swayed},” “\textsc{Captain’s Doubt}”) so the table sees the story move.

\paragraph{Position → DV.}
Use setting, leverage, and stakes:
\begin{itemize}
  \item \textbf{Dominant (DV 2):} private audience, proof in hand, shared values.
  \item \textbf{Controlled (DV 3):} time pressure, partial access, mixed reception.
  \item \textbf{Desperate (DV 4–5+):} public scrutiny, scandal risk, hostile crowd.
\end{itemize}

\paragraph{Distinct Approaches (examples).}
\begin{itemize}
  \item \textbf{Wits+Sway:} reframe incentives; offer face-saving out.
  \item \textbf{Wits+Lore:} cite precedent, produce documents or testimony.
  \item \textbf{Body+Presence:} command presence, ritual authority, oath.
  \item \textbf{Wits+Tinker:} demonstrate a device/proof-of-concept on the spot.
  \item \textbf{Bonds/Boons:} an ally vouches (assist) or gifts a Boon; NPC bonds can reduce DV by 1 if genuinely invoked.
\end{itemize}

\paragraph{Outcomes Palette.}
\begin{itemize}
  \item \textbf{Strong Hit:} Tick Persuasion +2; optionally bank a \emph{Concession} (you can waive a future cost).
  \item \textbf{Mixed:} Tick +1 and choose a cost: start a small \textsc{Rebuttal 1/4}, owe a minor favor, or the GM spends 1 SB to introduce a new stakeholder.
  \item \textbf{Miss:} No progress. GM may (a) reduce Persuasion −1, (b) advance \textsc{Opposition} +1–2, or (c) worsen Position one step.
\end{itemize}

\paragraph{When Opposition Fills First.}
The target hardens or reframes the negotiation: narrow the ask, accept a condition, or escalate proof (pay a Boon, reveal evidence, bring a witness) to keep going.

\paragraph{Scaling & Dials.}
\begin{itemize}
  \item \textbf{Crowd Scenes:} Add \textsc{Hecklers}/\textsc{Applause} 1/4 that swing Position when they fill.
  \item \textbf{Stake Weight:} For life-or-death asks, require a \emph{Concession} on success (named cost that must be paid during wrap-up).
  \item \textbf{Truth vs. Bluff:} Real proof improves Position; blatant lies risk a hidden \textsc{Caught Out} 1/4 that explodes later.
  \item \textbf{Patron Color:} Invoking a Patron symbol can shift Position if on-theme—or add +1 DV if it antagonizes the audience’s loyalties.
\end{itemize}

\paragraph{End States.}
On fill, summarize the agreement and record any \emph{Concessions}, debts, or clocks that carry forward. If not filled by scene end, bank current ticks and reopen later if fiction supports it.

\subsection{Recommended Session Order (GM Checklist)}

\paragraph{1) Off-Screen (Downtime, 10–20 min)}
\begin{itemize}
  \item Upkeep: choose Efficient/Intensive; apply Neglected/Compromised if missed.
  \item Obligation: clear via Acts of Service; note Claims/overflow risk.
  \item Projects: tick long-term clocks; resolve Gather Info; prep assets.
  \item Intent: each player states one on-screen goal; GM surfaces 1–2 front pressures.
\end{itemize}

\paragraph{2) On-Screen (Scenes)}
\begin{itemize}
  \item Frame hard: where/what’s at stake; set Position $\to$ DV.
  \item Run spotlight: rotate beats; fold in bonds and Boon sharing.
  \item Advance: move faction/Patron clocks openly when triggered.
\end{itemize}

\paragraph{3) Wrap-Up (5–10 min)}
\begin{itemize}
  \item XP \& Talents: award, mark progress; note any Gifts gained/forfeit.
  \item SB \& Harm: convert Fatigue$\to$Harm if full; apply recoveries.
  \item Fronts: advance unresolved clocks; note consequences.
\end{itemize}

\paragraph{4) Off-Screen Hooks (2–5 min)}
\begin{itemize}
  \item Log next Downtime intents, service opportunities, upkeep deadlines.
  \item Capture cliffhangers and Patron Largess seeds for next session open.
\end{itemize}

\emph{Optional:} Add a cold open flash-cut before Step 2 to spotlight a rival or Patron omen.

\paragraph{Maximum die pool}

An individual can have a max die pool of 10d10. All extra are converted to auto-successes. 
% GM's Guide: Session 0 → Session 1 Onboarding
\section{Session 0 \texorpdfstring{$\rightarrow$}{→} Session 1 (GM Onboarding)}
\label{sec:gm-session0-1}
\index{Onboarding}\index{Session 0}\index{Session 1}\index{Story Beats}\index{Boons}\index{DV}\index{Position}

\subsection*{Goal}
Reach \textbf{informed readiness} in one Session~0 (3–4 hours): the table understands the core loop and has functional characters, then roll into a tutorial-style Session~1 that practices the basics.

\subsection*{Session 0 Agenda (3–4 hours)}
\begin{tcolorbox}[title={Overview},colback=gray!5,colframe=black]
\textbf{Outcome:} Shared vocabulary, finished character sheets, party bonds, and a primed first scene. Mastery comes during Session~1.
\end{tcolorbox}

\paragraph{Hour 1 — Core Principles \& The Central Question (Why).}
\begin{itemize}
  \item Philosophy: \emph{What are you willing to risk?} Narrative-First.
  \item Core loop: \textbf{Approach} $\rightarrow$ \textbf{Roll} (d10; 6+ success; 1s = SB) $\rightarrow$ \textbf{Outcome} (GM spends SB).
  \item Currencies: \textbf{Story Beats (SB)} for GM, \textbf{Boons} for players.
  \item Tools: DV Ladder; Position (\textit{Dominant / Controlled / Desperate}).
\end{itemize}

\paragraph{Hour 2 — Character Creation (Who).}
\begin{itemize}
  \item Choose \emph{Ancestry/Culture} (Affinity).
  \item Allocate starting XP to \emph{Attributes} \& \emph{Skills}.
  \item Pick 1–2 starting \emph{Talents}.
  \item Choose an initial \emph{Complication} hook.
  \item \textit{Mastery expectation:} Functional sheets; intuitive timing for Boons comes later.
\end{itemize}

\paragraph{Hour 3 — Setting \& Party Fit (Where).}
\begin{itemize}
  \item Quick Hook (2-card draw or starter prompt) to frame the opening situation.
  \item Establish \emph{party bonds} and one shared near-term goal.
  \item Demonstrate \emph{Position} $\rightarrow$ \emph{DV} with the hook (one quick example).
\end{itemize}

\subsection*{Outputs \& Handshakes (End of Session 0)}
\begin{itemize}
  \item \textbf{Character sheets} complete; bonds recorded.
  \item \textbf{Opening scene} sketched (location, stake, immediate pressure).
  \item \textbf{Table tools} named: SB spend menu, DV ladder reference, Boon tracking.
\end{itemize}

\subsection*{Session 1: The Tutorial Level (90–120 min focus core)}
\begin{tcolorbox}[title={Crucial Advice},colback=gray!5,colframe=black]
\textbf{Defer subsystems.} Teach the core loop first; layer complexity later.
\end{tcolorbox}

\paragraph{Use Now.}
\begin{itemize}
  \item Core resolution (Approach $\rightarrow$ Roll $\rightarrow$ Outcome).
  \item \textbf{SB/Boons} earn/spend in play.
  \item One simple \textbf{Combat} or \textbf{Social} set-piece to practice \emph{Position} and \emph{DV}.
\end{itemize}

\paragraph{Defer For Later Sessions.}
\begin{itemize}
  \item Travel procedures and extended exploration.
  \item Deep/complex magic modules; advanced Rites; asset activation webs.
  \item Faction game and complex multi-front clocks.
\end{itemize}

\paragraph{GM Safety Nets (have these at hand).}
\begin{itemize}
  \item \textbf{DV Ladder}: Dominant=DV~2, Controlled=DV~3, Desperate=DV~4–5+.
  \item \textbf{SB Spend Menu}: soft/hard complications, clocks, position shifts.
  \item \textbf{Quick Hook}: 1–2 prompts to cut into a scene without prep.
\end{itemize}

\subsection*{Recommended Flow (Session 1)}
\begin{enumerate}
  \item \textbf{Cold Open (2–3 min):} Re-state the hook and stakes.
  \item \textbf{Scene A (15–25 min):} Low-risk challenge to practice SB/Boons.
  \item \textbf{Scene B (25–35 min):} One focused conflict (Combat \emph{or} Social clock).
  \item \textbf{Wrap (5–10 min):} Award XP, note SB used, log 1–2 evolving clocks.
\end{enumerate}

\subsection*{Do/Don’t (Quick Coach Cards)}
\begin{description}[leftmargin=1.6em]
  \item[\textbf{Do}] Ask for a one-sentence fiction beat for every assist/Boon gift.
  \item[\textbf{Do}] Favor \emph{Position} shifts over raw dice creep when help is excellent.
  \item[\textbf{Do}] Name costs before rolls when spending SB for consequences.
  \item[\textbf{Don’t}] Introduce every subsystem in Session~1—save it for Session~3+.
  \item[\textbf{Don’t}] Stall on rules lookups; rule, note, move on.
\end{description}

\subsection*{Staged Onboarding Plan}
\begin{itemize}
  \item \textbf{Session 0:} Vocabulary \& build. Success = “We know what the buttons do.”
  \item \textbf{Session 1:} Core loop in a contained scenario. Success = “Fun scene, minimal lookups.”
  \item \textbf{Session 2–3:} Add \emph{one} subsystem per session (e.g., Social clocks, Travel, then Rites).
\end{itemize}

\subsection*{Monitoring (GM Notes)}
Track over the next 3–5 sessions:
\begin{itemize}
  \item \textbf{SB usage:} Are you spending 1–2 per scene for visible consequence?
  \item \textbf{Boon economy:} Are players earning/spending, or hoarding for one talent?
  \item \textbf{Friction points:} Note FAQs to tighten your table’s cheat sheets.
\end{itemize}

\subsection*{Mini Checklists}
\begin{tcolorbox}[title={Session 0 Done When},colback=gray!5,colframe=black]
\begin{itemize}
  \item All sheets complete; bonds and one party goal set.
  \item Opening scene framed; Position/DV example demonstrated.
  \item SB/Boons tracking agreed (cards, tokens, or sheet boxes).
\end{itemize}
\end{tcolorbox}

\begin{tcolorbox}[title={Session 1 Done When},colback=gray!5,colframe=black]
\begin{itemize}
  \item Each player earned and spent at least one Boon.
  \item You spent at least one soft and one hard SB consequence.
  \item One clock advanced (any size), and stakes for next time are noted.
\end{itemize}
\end{tcolorbox}

\subsection{Fear Effects Table}
\label{subsec:fear-table}

When a character escalates on the Fear Track (Shaken $\rightarrow$ Frightened $\rightarrow$ Panicked), roll on the following table or choose an appropriate effect. These results apply primarily to NPCs, though PCs may adopt them as narrative guidance.

\begin{center}
\begin{tabular}{>{\bfseries}r l l}
\toprule
d10 & Effect & Magic Tags \\
\midrule
1 & \textbf{Freeze}: Cannot act this round, staring or trembling. & Silence, Stasis \\
2 & \textbf{Flee}: Must move at full speed away from the source of Fear. & Movement, Wind \\
3 & \textbf{Drop}: Character drops what they are holding. & Disarm, Break \\
4 & \textbf{Beg}: Character pleads or bargains incoherently. & Compulsion, Voice \\
5 & \textbf{Hide}: Seeks cover, concealment, or allies to cling to. & Shadow, Illusion \\
6 & \textbf{Attack in Panic}: Lashes out wildly at the nearest target. & Rage, Fire \\
7 & \textbf{Blunder}: Stumbles into danger (trap, hazard, off balance). & Chaos, Trickery \\
8 & \textbf{Obey}: Instinctively follows a simple command from the fear-causer. & Command, Charm \\
9 & \textbf{Break Down}: Sobs, prays, or becomes useless until aided. & Curse, Despair \\
10 & \textbf{Catatonia}: Becomes unresponsive, requiring intervention. & Sleep, Dream \\
\bottomrule
\end{tabular}
\end{center}

\paragraph{Note.}  
At GM discretion, results may escalate with each step of the Fear Track:  
- \emph{Shaken}: Apply minor versions (hesitation, lost die, startled).  
- \emph{Frightened}: Roll normally.  
- \emph{Panicked}: Apply severe or exaggerated results (e.g., 2 = reckless flight, 6 = attack allies).  
