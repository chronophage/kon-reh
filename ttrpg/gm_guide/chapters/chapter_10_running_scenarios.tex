\chapter{Running Specific Scenarios}

In \textbf{Fate’s Edge}, no two scenes play out the same way. The dice shift. The world reacts. And the players shape the story through bold choices and desperate gambits. This chapter offers guidance for running the most common—and most crucial—types of scenes in the game: \textbf{heists, battles, political intrigue, exploration}, and \textbf{mysteries}. Each is a lens into the world’s tension, and each rewards the GM who leans into narrative and consequence.

\section*{Heists and Infiltration}

A heist is not about perfect plans—it’s about \textbf{controlled chaos}. The PCs are not ghosts—they are \textbf{agents of disruption}, and the world will respond.

\subsection*{Scene Setup}

\begin{itemize}
    \item \textbf{Entry Position}: Controlled, Risky, or Desperate?
    \item \textbf{Social Rails}: Curfew, Crowd, Sanctity.
    \item \textbf{Physical Rails}: Hazard, Hunt, Escape.
    \item \textbf{Assets and Followers}: Can they create distractions or bypass security?
\end{itemize}

\subsection*{Example Scene: Infiltrating the Vhasian Château}

The PCs seek to steal a sealed charter from a noble’s vault. The GM frames the scene:

\begin{itemize}
    \item \textbf{Position}: Risky (guards patrol, windows shuttered).
    \item \textbf{Social Rail}: Curfew (the lord has ordered all gates barred by dusk).
    \item \textbf{Physical Rail}: Hunt (bloodhounds circle the grounds).
\end{itemize}

The PCs split up: one charms a servant for key access (Presence + Sway), another distracts the guards with a fake alarm (Wits + Skullduggery), and a third scales the wall (Body + Athletics). Each roll adds tension—successes advance the plan, but Complication Points trigger new dangers.

\subsection*{GM Tips}

\begin{itemize}
    \item \textbf{Let the world respond}: A guard changes shift. A noble returns early.
    \item \textbf{Use clocks to escalate}: Hunt +1 when an alarm sounds. Curfew tightens as bells ring.
    \item \textbf{Offer forks}: Partial success means progress—but at a cost. Let players choose.
\end{itemize}

\section*{Battles and Skirmishes}

Combat in Fate's Edge follows the same core procedures as all other actions, but with specific applications for violent conflict. Every combat action generates potential for both triumph and complication, with consequences that cascade through the same economy as all other challenges.

\subsection*{Scene Setup}

\begin{itemize}
    \item \textbf{Position}: Set Controlled, Risky, or Desperate based on tactical situation.
    \item \textbf{Group Actions}: Use the Lead system to coordinate.
    \item \textbf{Follower Risk}: 2+ CP spent in combat can endanger assisting followers.
    \item \textbf{Tactical Clocks}: Mob Overwhelm, Fatigue Spiral, Morale Collapse, Environmental Collapse.
\end{itemize}

\subsection*{Example Scene: Clash in the Mistlands}

A reaver band ambushes the PCs on a foggy causeway. The GM sets:

\begin{itemize}
    \item \textbf{Position}: Desperate (fog limits vision, reavers surround).
    \item \textbf{Mob Overwhelm Clock}: 6 segments (enemy numbers become advantage).
    \item \textbf{Environmental Collapse Clock}: 8 segments (terrain/fire/building failure).
\end{itemize}

The PCs fight, rally, and retreat—but not without cost. A follower takes Harm 1. The mist hides them—for now.

\subsection*{GM Tips}

\begin{itemize}
    \item \textbf{Focus on stakes}: What happens if the PCs lose? What if they win ugly?
    \item \textbf{Use tactical clocks to escalate tension}: Mob Overwhelm advances as reinforcements arrive.
    \item \textbf{Let followers matter}: They are not stat blocks—they are story agents. Let them act, suffer, and grow.
    \item \textbf{Apply harm integration}: Minor harm generates 1 CP on next 2 rolls; Moderate harm generates 1 CP next roll with -1 die.
\end{itemize}


\section*{Political Intrigue}

Intrigue is a \textbf{dance of leverage, lies, and legacy}. It rewards patience, perception, and the courage to \textbf{burn bridges} for greater gains.

\subsection*{Scene Setup}

\begin{itemize}
    \item \textbf{Social Rails}: Crowd (public opinion), Curfew (timing), Sanctity (reputation).
    \item \textbf{Leverage**: Diamonds and Assets shape influence.
    \item \textbf{Allies and Rivals**: Represented by Followers and Factions.
\end{itemize}

\subsection*{Example Scene: Council of the Three Greens}

The PCs seek to sway the moot in their favor. The GM sets:

\begin{itemize}
    \item \textbf{Crowd Rail}: 6 segments (tempers flare, factions shout).
    \item \textbf{Sanctity Rail}: 4 segments (accusations of heresy fly).
    \item \textbf{Diamonds**: A sealed charter, a noble’s favor, a scandalous letter.
\end{itemize}

The PCs must navigate shifting loyalties, whispered betrayals, and the ever-present threat of exile.

\subsection*{GM Tips}

\begin{itemize}
    \item \textbf{Let words carry weight**: A well-timed insult can shift a rail as fast as a blade.
    \item \textbf{Use the Deck of Consequences**: Hearts and Diamonds bring social fallout and leverage traps.
    \item \textbf{Offer moral choices**: Who will you betray? Who will you save?
\end{itemize}

\section*{Exploration and Mysteries}

Exploration is not just about maps—it’s about \textbf{discovery}, \textbf{danger}, and the \textbf{unknown}. Mysteries reward curiosity, caution, and the willingness to \textbf{dig deeper}.

\subsection*{Scene Setup}

\begin{itemize}
    \item \textbf{Environmental Pressure**: Weather, terrain, supernatural forces.
    \item \textbf{Lore Rewards**: Boons, Assets, or narrative hooks.
    \item \textbf{Clocks**: Supply, Fatigue, Hazard.
\end{itemize}

\subsection*{Example Scene: The Root Gallery Beneath the Oak Hill}

The PCs descend into a fae-haunted ruin. The GM sets:

\begin{itemize}
    \item \textbf{Hazard Clock}: 6 segments (roots shift, light fails).
    \item \textbf{Supply**: Low—no food, strange water.
    \item \textbf{Mystical Pressure**: Umbramancy twists the path.
\end{itemize}

Each roll reveals a new danger—or a hidden truth. A failed roll might trigger a fae encounter. A success might uncover a lost relic.

\subsection*{GM Tips}

\begin{itemize}
    \item \textbf{Let the environment tell the story**: Roots whisper. Stones bleed. Light lies.
    \item \textbf{Tie lore to player choices**: A glyph only activates if the PCs speak the old tongue.
    \item \textbf{Offer legacy hooks**: What will the PCs do with what they find?
\end{itemize}

\section*{Let the Dice Guide You}

In Fate’s Edge, every scene is a chance to \textbf{push the story forward}. Let the dice sing. Let the world respond. And above all—let the players \textbf{own the consequences}.

Because in the end, it is not the GM who writes the legend.

It is the players.

You simply hold the quill.

\end{chapter}
