\chapter{Enhanced GM Play}\index{enhanced GM play}

The Crown system becomes truly collaborative when the GM manages resources just like players, creating shared stakes and mutual investment in the narrative outcome.

\section{Resource Management}\index{resource management}

Track these key resources to enhance your GM experience and create more balanced gameplay.

\subsection{GM Relationship Management}\index{relationship management}

Just as players track relationship dice with NPCs, you should track relationship dice with major factions and key NPCs. This creates bidirectional engagement where both sides have stakes in interactions.

\subsubsection{Starting GM Relationships}\index{relationship management!starting}

Begin each campaign with 1-2 relationship dice per major faction:
\begin{itemize}
\item Political factions (nobility, merchants, clergy)
\item Criminal organizations (guilds, syndicates, pirates)
\item Military/civic authorities (guard, military, bureaucracy)
\item Supernatural entities (fae, undead, spirits)
\end{itemize}

\subsubsection{Relationship Shifts}\index{relationship management!shifts}

GM relationship dice change based on player actions:
\begin{itemize}
\item Successful player interaction with faction: GM may gain/lose relationship dice
\item Player betrayal of faction: GM gains relationship dice with antagonistic factions
\item Player aid to faction: GM may lose relationship dice with enemy factions
\end{itemize}

\subsubsection{Bidirectional Rolls}\index{relationship management!bidirectional rolls}

When players interact with NPCs, both sides roll:
\begin{itemize}
\item Player rolls their relationship dice with the NPC
\item GM rolls their relationship dice with that faction
\item Results determine the \textbf{quality} of interaction, not just success/failure
\end{itemize}

\subsection{Shared Leverage Pool}\index{leverage pool}

Create a collaborative economy where helping each other becomes strategic.

\subsubsection{Pool Management}\index{leverage pool!management}

\begin{itemize}
\item Players contribute 1 leverage each to shared pool at session start
\item GM can spend from pool to enhance player successes or create interesting complications
\item Players can spend to bypass GM complications or enhance their own actions
\item Pool refreshes each session
\end{itemize}

\subsubsection{Spending Options}\index{leverage pool!spending}

\textbf{GM Spending:}
\begin{itemize}
\item 1 leverage: Add interesting detail to player success
\item 2 leverage: Create beneficial coincidence
\item 3+ leverage: Introduce major plot hook
\end{itemize}

\textbf{Player Spending:}
\begin{itemize}
\item 1 leverage: Avoid minor complication
\item 2 leverage: Gain advantage on next roll
\item 3+ leverage: Rewrite recent unfavorable outcome
\end{itemize}

\section{Campaign Tracking Systems}\index{campaign tracking}

Simple tracking mechanisms that enhance long-term play without complex bookkeeping.

\subsection{Faction Loyalty Tracker}\index{faction loyalty}

Track persistent world state through faction relationships.

\subsubsection{Loyalty Scale}\index{faction loyalty!scale}

Use a simple -3 to +3 scale for each major faction:
\begin{description}
\item[-3 Enemy:] Actively working against player interests
\item[-2 Hostile:] Will cause trouble when possible
\item[-1 Unfriendly:] Suspicious, unhelpful
\item[0 Neutral:] Indifferent to player actions
\item[+1 Friendly:] Helpful when convenient
\item[+2 Supportive:] Actively assist player goals
\item[+3 Ally:] Will sacrifice for player interests
\end{description}

\subsubsection{Loyalty Shifts}\index{faction loyalty!shifts}

Player actions shift faction loyalty:
\begin{itemize}
\item Major help: +1 to +2 loyalty
\item Minor help: +1 loyalty
\item Neutral actions: No change
\item Minor harm: -1 loyalty
\item Major harm: -1 to -2 loyalty
\item Betrayal: -2 to -3 loyalty
\end{itemize}

\subsection{Revelation Economy}\index{revelation economy}

Control information flow through mechanical budgeting.

\subsubsection{Budget Management}\index{revelation economy!budget}

\begin{itemize}
\item Each clock segment = 1 revelation point
\item Major discoveries cost 1-3 revelation points
\item Players can "bank" unused revelation for future sessions
\item GM can "save" revelation for climax moments
\end{itemize}

\subsubsection{Revelation Costs}\index{revelation economy!costs}

\begin{description}
\item[1 Point:] Basic facts, surface details
\item[2 Points:] Strategic insights, tactical advantages
\item[3 Points:] Major revelations, plot-critical information
\end{description}

\subsection{Escalation Economy}\index{escalation economy}

Make tension management a player choice rather than imposed obstacle.

\subsubsection{Point System}\index{escalation economy!points}

\begin{itemize}
\item Start with 3 escalation points per conflict
\item Each escalation costs 1 point:
  \begin{itemize}
  \item Add +1 dice to opposition
  \item Introduce new threat
  \item Complicate existing situation
  \end{itemize}
\item Players can spend to de-escalate or redirect
\item Points refresh per new conflict
\end{itemize}

\section{Collaborative Mechanics}\index{collaborative mechanics}

Mechanics that make players active participants in narrative creation.

\subsection{Complication Trading}\index{complication trading}

Allow players to request specific challenge types, making them active participants in narrative creation.

\subsection{Relationship Investment Tracking}\index{relationship investment}
Track how Bond-Driven Resource Generation affects player engagement with relationship mechanics. Players who frequently use this mechanic demonstrate investment in collaborative storytelling and character connections.

\subsubsection{Player Challenge Requests}\index{complication trading!challenge requests}

Players can request specific complication types:
\begin{itemize}
\item Social complications (feuds, negotiations, diplomacy)
\item Physical challenges (combat, exploration, survival)
\item Mystery elements (investigation, puzzles, hidden information)
\item Moral dilemmas (ethical conflicts, difficult choices)
\end{itemize}

\subsubsection{Bargaining Process}\index{complication trading!bargaining}

\begin{enumerate}
\item Player declares desired complication type and spends leverage (1-2)
\item GM draws from appropriate deck but allows player modification
\item GM can spend relationship dice to enhance complications
\item Both sides benefit from engaging, invested complications
\end{enumerate}

\subsection{Cross-Deck Synergy}\index{cross-deck synergy}

Encourage creative cross-cultural storytelling through mechanical rewards.

\subsubsection{Synergy Bonuses}\index{cross-deck synergy!bonuses}

\begin{itemize}
\item Using elements from 2+ decks in same scene = +1 to relevant rolls
\item "Perfect match" (e.g., maritime + criminal) = bonus leverage or relationship die
\item Track "deck diversity" for session bonus
\end{itemize}

\subsection{Momentum Banking}\index{momentum banking}

Reward efficient play and strategic pacing through saved resources.

\subsubsection{Banking Rules}\index{momentum banking!rules}

\begin{itemize}
\item Each segment under "standard" resolution = 1 momentum point
\item Banked momentum can be spent for:
  \begin{itemize}
  \item +1 to any relationship roll
  \item 1 free leverage
  \item Reroll one diamond draw
  \end{itemize}
\item Momentum decays if not used within 3 sessions
\end{itemize}

\section{Session Management}\index{session management}

Procedures for managing enhanced gameplay elements during sessions.

\subsection{Pre-Session Setup}\index{session management!pre-session}

\begin{enumerate}
\item Review active decks for session
\item Check familiarity points for each deck
\item Set GM relationship dice for major factions
\item Note any momentum carryover from previous sessions
\item Refresh shared leverage pool
\end{enumerate}

\subsection{During Session Management}\index{session management!during session}

\begin{itemize}
\item Track relationship shifts through player actions
\item Monitor shared leverage pool spending
\item Facilitate information trading through leveraged negotiation
\item Manage familiarity points for deck optimization
\item Track faction loyalty changes
\item Monitor revelation economy spending
\end{itemize}

\subsection{Post-Session Wrap-up}\index{session management!post-session}

\begin{enumerate}
\item Adjust momentum based on clock resolution
\item Update familiarity points for used decks
\item Note relationship changes for next session
\item Bank unused revelation points
\item Track session investment ratings
\item Plan any carryover elements
\end{enumerate}

\section{Implementation Timeline}\index{implementation timeline}

Gradual implementation to avoid overwhelming players or yourself.

\subsection{Quick Start (Sessions 1-3)}\index{implementation timeline!quick start}

\begin{itemize}
\item Introduce Shared Leverage Pool (1 leverage each)
\item Start Faction Loyalty Tracker (simple -3 to +3 scale)
\item Use Complication Trading ("Want to make this more interesting?")
\item Track basic relationship shifts
\end{itemize}

\subsection{Intermediate (Sessions 4-6)}\index{implementation timeline!intermediate}

\begin{itemize}
\item Add Momentum Banking (track under/over segments)
\item Implement Revelation Economy (clock segments = discovery budget)
\item Introduce Cross-Deck Synergy tracking
\item Begin Escalation Economy for conflicts
\end{itemize}

\subsection{Advanced (Sessions 7+)}\index{implementation timeline!advanced}

\begin{itemize}
\item Full Cultural Familiarity system
\item Complete Session Investment tracking
\item Cultural Immersion Bonus system
\item Player-GM Relationship Mirror
\end{itemize}

These enhancements transform the Crown system from a tool for scenario generation into a complete collaborative storytelling framework where everyone at the table has meaningful stakes and resources to manage.

\end{chapter}
