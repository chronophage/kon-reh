%chapter_03_managing_resources.tex
\chapter{Managing Resources}

In \textbf{Fate’s Edge}, resources are not just numbers on a sheet—they are \textbf{living elements of the fiction}. From the last sip of water in the desert sands of Akilan to the loyalty of a Ykrul war-band, every resource has a story, and every story has a cost. As the GM, you are the keeper of these threads. This chapter outlines how to manage and narrate the systems that fuel both the characters and the campaign.

\section*{Supply Clock: The Pulse of Survival}

The \textbf{Supply Clock} is a shared condition for the entire party. It tracks access to food, water, basic gear, and logistical support. It is not a strict inventory system—it is a \textbf{narrative lever} that adds tension when the party is isolated, pressed, or cut off.

\subsection*{Supply Clock States}

\begin{itemize}
    \item \textbf{Full Supply (0/4)}: The party is well-equipped. No penalties.
    \item \textbf{Low Supply (2/4)}: Minor narrative complications—bland food, damaged arrows, thinning waterskins.
    \item \textbf{Dangerously Low (3/4)}: Each character gains \textbf{Fatigue 1}.
    \item \textbf{Out of Supply (4/4)}: Severe penalties—starvation, dehydration, failing gear. Actions become Risky or Desperate.
\end{itemize}

\subsection*{Filling the Clock}

The Supply Clock fills when:

\begin{itemize}
    \item Extended travel without provisioning.
    \item GM spends 2+ CP on logistics failures.
    \item The party chooses to travel light for advantage.
\end{itemize}

\subsection*{Emptying the Clock}

\begin{itemize}
    \item Reaching civilization resets to Full.
    \item Group Survival check clears 1 segment.
    \item Downtime in safety removes 1 segment.
\end{itemize}

\textbf{Example}: A week-long sea passage across the Dolmis with uncertain winds. A failed Navigation roll causes the GM to spend 2 CP—filling two segments. The party is now at Low Supply. A second failed roll fills another segment—Dangerously Low. Fatigue sets in. The sea, once a path, now gnaws at their endurance.

\section*{Fatigue: The Weight of the World}

Fatigue represents \textbf{exhaustion, hunger, and strain}. It is cumulative and persistent. Each level of Fatigue forces the character to \textbf{re-roll one success} on their next action.

\subsection*{Fatigue Effects}

\begin{itemize}
    \item \textbf{1 Fatigue}: Re-roll one success.
    \item \textbf{2 Fatigue}: Re-roll one success (cumulative).
    \item \textbf{3 Fatigue}: Re-roll two successes.
    \item \textbf{4 Fatigue}: Collapse, KO, or spiritual break. Out of the scene until treated.
\end{itemize}

\subsection*{Clearing Fatigue}

\begin{itemize}
    \item A night’s rest with adequate Supply removes 1 Fatigue.
    \item Fatigue cannot be removed while the party is Dangerously Low or Out of Supply.
\end{itemize}

\textbf{Narrative Note}: Fatigue is not just physical—it can reflect mental strain, grief, or spiritual exhaustion. A failed ritual might leave a caster \textbf{Fatigue 2} from the backlash alone.

\section*{Followers and Assets: Power Beyond the Self}

In Fate’s Edge, players can invest XP into \textbf{Followers} and \textbf{Assets}—tools that extend their reach beyond personal skill. These are not mere stat blocks—they are \textbf{story agents} with their own motivations, risks, and narrative arcs.

\subsection*{Followers: On-Scene Allies}

Followers are \textbf{on-screen allies} who can assist in their specialty. They are bought with XP and tracked by a \textbf{Cap} (their maximum assist bonus).

\begin{itemize}
    \item \textbf{Cap 1} (3 XP): Porter, Squire.
    \item \textbf{Cap 2} (5 XP): Scout, Bodyguard.
    \item \textbf{Cap 3} (8 XP): Veteran operative.
    \item \textbf{Cap 4} (12 XP): Elite aide.
    \item \textbf{Cap 5} (17 XP): Exceptional lieutenant.
\end{itemize}

\textbf{Assist Dice}: When relevant, a follower adds up to their Cap in dice to a roll (max +3 total assist).

\textbf{Follower Risks}: Any roll involving a follower exposes them to Complications. The GM may spend CP to harm, separate, or compromise them.

\subsection*{Assets: Off-Scene Influence}

Assets are \textbf{off-screen resources}—titles, safehouses, spy rings, charters. They solve problems between scenes but \textbf{cannot intervene mid-adventure}.

\begin{itemize}
    \item \textbf{Minor} (4 XP): Safehouse, Petty Title.
    \item \textbf{Standard} (8 XP): Guild Section, Spy Ring.
    \item \textbf{Major} (12 XP): City Charter, Mercantile Fleet.
\end{itemize}

\textbf{Activation Cost}: 1 Boon or 2 XP. Each activation accomplishes one clear, plausible outcome.

\textbf{Condition Tracks}: All Assets and Followers have a \textbf{Condition Track}:

\begin{itemize}
    \item \textbf{Maintained} → \textbf{Neglected} → \textbf{Compromised}
\end{itemize}

Neglected Assets impose a -1 die penalty. Compromised Assets are unavailable until repaired or recovered.

\section*{Boons: The Currency of Resilience}

Boons are \textbf{narrative tokens} earned by embracing risk and moving the story forward. They reward \textbf{failure with texture}, not failure with nothing.

\subsection*{Earning Boons}

\begin{itemize}
    \item On a failed roll with meaningful Complications.
    \item Through clever or risky roleplay.
    \item Via backstory ties with other players.
\end{itemize}

\subsection*{Using Boons}

\begin{itemize}
    \item \textbf{Re-roll one die} after seeing the roll.
    \item \textbf{Activate an Off-Screen Asset} (preferred cost).
    \item \textbf{Convert 2 Boons → 1 XP} during downtime (max 1 XP/session).
\end{itemize}

\textbf{Boon Cap}: A character may hold a maximum of 5 Boons. Overflow converts to XP (2→1).

\textbf{Design Note}: Boons are not a “get out of jail free” card. They are earned by \textbf{leaning into the fiction}, not by fishing for failure. Reward players who take risks, not those who roll badly on purpose.

\section*{XP Awards: Growth Through Choice}

XP in Fate’s Edge is \textbf{meaningful currency}. It is not handed out for showing up—it is earned through \textbf{engagement, risk, and narrative impact}.

\subsection*{Session Awards}

\begin{itemize}
    \item \textbf{Attendance}: +2 XP
    \item \textbf{Major Objective Reached}: +2–4 XP
    \item \textbf{Discovery or Lore Unlocked}: +1–2 XP
    \item \textbf{Hard Choice Embraced}: +1–2 XP
    \item \textbf{Complication Spotlight}: +1–3 XP
    \item \textbf{Bond/Flag Driven Play}: +1–2 XP
    \item \textbf{GM Curveball Award}: +0–3 XP
\end{itemize}

\subsection*{Milestone Awards}

At the end of a major arc:

\begin{itemize}
    \item \textbf{+8–12 XP} to all players.
    \item \textbf{+2 XP bonus} for a signature moment.
\end{itemize}

\textbf{Complication Dividend}: If a player accepts a high Complication card without mitigation:

\begin{itemize}
    \item \textbf{Face Card}: +1 XP
    \item \textbf{Ace}: +2 XP
\end{itemize}

\section*{Narrative First: The Fiction Is the Ledger}

In Fate’s Edge, arrows, rations, and waterskins are tracked only in the fiction. Mechanics engage only when those resources become scarce. The focus is always on \textbf{narrative tension}, not bookkeeping.

Let the world breathe. Let the fiction lead. And when the dice say the world pushes back—\textbf{listen.}

\end{chapter}

