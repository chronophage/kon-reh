% !TEX root = ../fates_edge_players_guide.tex

\chapter{Managing Resources}

In \textbf{Fate's Edge}, resources are not just numbers on a sheet—they are \textbf{living elements of the fiction}. From the last sip of water in the desert sands of Akilan to the loyalty of a Ykrul war-band, every resource has a story, and every story has a cost. As the GM, you are the keeper of these threads. This chapter outlines how to manage and narrate the systems that fuel both the characters and the campaign.

\section*{Supply Clock: The Pulse of Survival}

The \textbf{Supply Clock} is a shared condition for the entire party. It tracks access to food, water, basic gear, and logistical support. It is not a strict inventory system—it is a \textbf{narrative lever} that adds tension when the party is isolated, pressed, or cut off.

\subsection*{Supply Clock States}

\begin{center}
\begin{tabular}{cl}
\toprule
\textbf{Segments Filled} & \textbf{Effect} \\
\midrule
0 (Full) & The party is well-equipped. \\
2 (Low) & Minor narrative complications (bland food, damaged arrows, thinning waterskins). \\
3 (Dangerous) & Each character gains Fatigue. \\
4 (Empty) & Severe penalties. \\
\bottomrule
\end{tabular}
\end{center}

\subsection*{Filling the Clock}

The Supply Clock fills when:

\begin{itemize}
    \item Extended travel without provisioning.
    \item GM spends 2+ CP on logistics failures.
    \item The party chooses to travel light for advantage.
\end{itemize}

\subsection*{Emptying the Clock}

\begin{itemize}
    \item Reaching civilization resets to Full.
    \item Group Survival check clears 1 segment.
    \item Downtime in safety removes 1 segment.
\end{itemize}

\textbf{Example}: A week-long sea passage across the Dolmis with uncertain winds. A failed Navigation roll causes the GM to spend 2 CP—filling two segments. The party is now at Low Supply. A second failed roll fills another segment—Dangerously Low. Fatigue sets in. The sea, once a path, now gnaws at their endurance.

\section*{Fatigue: The Weight of the World}

Fatigue represents \textbf{exhaustion, hunger, and strain}. It is cumulative and persistent. Each level of Fatigue forces the character to \textbf{re-roll one success} on their next action.

\subsection*{Fatigue Effects}

\begin{itemize}
    \item \textbf{1 Fatigue}: Re-roll one success.
    \item \textbf{2 Fatigue}: Re-roll one success (cumulative).
    \item \textbf{3 Fatigue}: Re-roll two successes.
    \item \textbf{4 Fatigue}: Collapse, KO, or spiritual break. Out of the scene until treated.
\end{itemize}

\subsection*{Clearing Fatigue}

\begin{itemize}
    \item A night's rest with adequate Supply removes 1 Fatigue.
    \item Fatigue cannot be removed while the party is Dangerously Low or Out of Supply.
\end{itemize}

\textbf{Narrative Note}: Fatigue is not just physical—it can reflect mental strain, grief, or spiritual exhaustion. A failed ritual might leave a caster \textbf{Fatigue 2} from the backlash alone.

\section*{Followers and Assets: Power Beyond the Self}

In Fate's Edge, players can invest XP into \textbf{Followers} and \textbf{Assets}—tools that extend their reach beyond personal skill. These are not mere stat blocks—they are \textbf{story agents} with their own motivations, risks, and narrative arcs.

\subsection*{Followers: On-Scene Allies}

Followers are \textbf{on-screen allies} who can assist in their specialty. They are bought with XP and tracked by a \textbf{Cap} (their maximum assist bonus).

Cost: A follower with Specialty Cap C costs C² XP.

\subsection*{Assisting in Scenes}

Followers assist by adding dice to your rolls:

\begin{itemize}
    \item Assist dice come from the helper, not the leader.
    \item Total Assist on any roll (from any sources) remains hard-capped at +3. Exception: The "Exceptional Coordination" Talent allows one follower to provide +4 assist dice.
    \item When applicable, the follower adds help dice equal to \textbf{min(C, the helper's relevant Skill)}, capped at +3 dice.
    \item Slot Limit: Only one follower may assist a given action.
\end{itemize}

\subsection*{Follower Upkeep}

\begin{itemize}
    \item Each Downtime, pay Coin equal to C or spend a Scene tending the relationship.
    \item Risk: If the GM spends 2+ Complication Points on an action you take with assistance, they may endanger, injure, or separate the follower instead of you if fictionally appropriate.
    \item Off-Screen Capability: Once per downtime, a Cap 5 follower can solve one significant problem but generates 1 CP for crew. The GM must describe how the follower's action creates story consequences for the CP generated.
\end{itemize}

\subsection*{Follower Condition}

Followers track \textbf{Exposure} and \textbf{Harm}:

\begin{description}
    \item[Exposure] — Heat, attention, or narrative stress.
    \item[Harm] — Injury or trauma.
\end{description}

\textbf{States:}
\begin{itemize}
    \item \textbf{Maintained} — Reliable and ready.
    \item \textbf{Neglected} — Needs downtime or care.
    \item \textbf{Compromised} — Captured, defected, or lost.
\end{itemize}

\subsection*{Assets: Off-Scene Influence}

Assets are \textbf{off-screen resources}—titles, safehouses, spy rings, charters. They do not act in scenes directly, but they change the fiction and provide leverage when you return to the table.

\begin{description}
    \item[Minor (4 XP)] — Safehouse, small shop, petty title.
    \item[Standard (8 XP)] — Noble title, guild section, spy ring.
    \item[Major (12 XP)] — City license, regional network, fortress lease.
\end{description}

\subsection*{Using Assets}

Assets provide off-screen benefits:

\begin{itemize}
    \item Use each Asset's listed Off-Screen effect once per session for free.
    \item To reshape the current scene, spend 1 Boon.
    \item The Asset must have scope and reach.
\end{itemize}

\subsection*{Asset Condition}

All Assets have a \textbf{Condition Track}:

\begin{description}
    \item[Maintained] — Full capability.
    \item[Neglected] — -1 die when used; requires attention.
    \item[Compromised] — Unavailable until repaired or recovered.
\end{description}

\section*{Boons: The Currency of Resilience}

Boons are \textbf{narrative tokens} earned by embracing risk and moving the story forward. They reward \textbf{failure with texture}, not failure with nothing.

\subsection*{Earning Boons}

\begin{itemize}
    \item On a failed roll with meaningful Complications.
    \item Through clever or risky roleplay.
    \item Via backstory ties with other players.
\end{itemize}

\subsection*{Boon Economy}

\begin{itemize}
    \item Holding cap: You can hold at most 5 Boons.
    \item Conversion: Once per session, in downtime, you may convert 2 Boons → 1 XP (max 2 XP via conversion per session).
\end{itemize}

\subsection*{Using Boons}

\begin{itemize}
    \item \textbf{Re-roll one die} after seeing the pool.
    \item \textbf{Activate an Off-Screen Asset}.
\end{itemize}

\textbf{Design Note}: Boons are not a "get out of jail free" card. They are earned by \textbf{leaning into the fiction}, not by fishing for failure. Reward players who take risks, not those who roll badly on purpose.

\section*{XP Awards: Growth Through Choice}

XP in Fate's Edge is \textbf{meaningful currency}. It is not handed out for showing up—it is earned through \textbf{engagement, risk, and narrative impact}.

\subsection*{Session Awards}

\begin{itemize}
    \item \textbf{Table Attendance}: +2 XP
    \item \textbf{Major Objective Reached}: +2–4 XP
    \item \textbf{Discovery or Lore Unlocked}: +1–2 XP
    \item \textbf{Hard Choice Embraced}: +1–2 XP
    \item \textbf{Complication Spotlight}: +1–3 XP
    \item \textbf{Bond/Flag Driven Play}: +1–2 XP
    \item \textbf{GM Curveball Award}: +0–3 XP
\end{itemize}

\subsection*{Milestones}

\begin{itemize}
    \item +8–12 XP to all players at the conclusion of a major story arc.
    \item +2 XP bonus to one player for a signature moment of the arc.
\end{itemize}

\subsection*{Complication Dividend}

\begin{itemize}
    \item \textbf{Face Card}: +1 XP
    \item \textbf{Ace}: +2 XP
\end{itemize}

\section*{Combat Resource Management}

In combat, resource management takes on new urgency. The same systems that govern exploration and downtime now operate under pressure, with immediate consequences.

\subsection*{Supply in Combat}

Extended combat encounters can drain resources rapidly:

\begin{itemize}
    \item \textbf{Intense Combat}: GM may spend 1 CP to fill 1 Supply segment
    \item \textbf{Prolonged Engagement}: Each hour of sustained combat adds 1 Supply segment
    \item \textbf{Ammunition Depletion}: Ranged weapons may run low, requiring scavenging actions
\end{itemize}

\subsection*{Fatigue in Combat}

Combat fatigue compounds existing strain:

\begin{itemize}
    \item \textbf{Each Round}: Characters with existing Fatigue re-roll additional successes equal to their Fatigue level
    \item \textbf{Critical Exhaustion}: Reaching 4 Fatigue during combat causes immediate collapse
    \item \textbf{Recovery}: Cannot clear Fatigue during active combat
\end{itemize}

\subsection*{Follower Combat Integration}

Followers in combat face unique risks and opportunities:

\begin{itemize}
    \item \textbf{Combat Assistance}: Followers can assist in combat rolls using their Cap
    \item \textbf{Follower Risk}: 2+ CP spent in combat can endanger assisting followers
    \item \textbf{Initiative Actions}: Followers can take combat-relevant independent actions
    \item \textbf{Combat Exposure}: Each time a follower acts on-screen in high-risk combat, mark Exposure +1
\end{itemize}

\subsection*{Asset Combat Activation}

Assets can be activated for immediate combat advantage:

\begin{itemize}
    \item \textbf{1 Boon}: Activate asset for combat advantage
    \item \textbf{Environmental Assets}: Terrain features, fortifications, magical wards
    \item \textbf{Compromise Risk}: Combat in certain locations can damage relevant assets
\end{itemize}

\subsection*{Tactical Clocks as Resources}

Tactical clocks represent persistent combat conditions that drain party resources:

\begin{itemize}
    \item \textbf{Mob Overwhelm} (6): Enemy numbers become advantage—forces Supply depletion and Fatigue
    \item \textbf{Fatigue Spiral} (4): Exhaustion affects performance—accelerates existing Fatigue
    \item \textbf{Morale Collapse} (6): Fear undermines effectiveness—generates CP and reduces effectiveness
    \item \textbf{Environmental Collapse} (8): Terrain/fire/building failure—creates new Supply and safety concerns
\end{itemize}

\section*{Narrative First: The Fiction Is the Ledger}

In Fate's Edge, arrows, rations, and waterskins are tracked only in the fiction. Mechanics engage only when those resources become scarce. The focus is always on \textbf{narrative tension}, not bookkeeping.

Let the world breathe. Let the fiction lead. And when the dice say the world pushes back—\textbf{listen.}

\end{chapter}