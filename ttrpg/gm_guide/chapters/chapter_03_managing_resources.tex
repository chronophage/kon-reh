\chapter{Managing Resources}\index{resources}

In \textbf{Fate's Edge}, resources are not just numbers on a sheet---they are \textbf{living elements of the fiction}. From the last sip of water in the desert sands of Akilan to the loyalty of a Ykrul war-band, every resource has a story, and every story has a cost. As the GM, you are the keeper of these threads. This chapter outlines how to manage and narrate the systems that fuel both the characters and the campaign.

\section*{Supply Clock: The Pulse of Survival}\index{Supply Clock}

The \textbf{Supply Clock} is a shared condition for the entire party. It tracks access to food, water, basic gear, and logistical support. It is not a strict inventory system---it is a \textbf{narrative lever} that adds tension when the party is isolated, pressed, or cut off.

\subsection*{Supply Clock States}

\begin{center}
\begin{tabular}{cl}
\toprule
\textbf{Segments Filled} & \textbf{Effect} \\
\midrule
0 (Full) & The party is well-equipped. \\
2 (Low) & Minor narrative complications (bland food, damaged arrows, thinning waterskins). \\
3 (Dangerous) & Each character gains Fatigue\index{Fatigue}. \\
4 (Empty) & Severe penalties. \\
\bottomrule
\end{tabular}
\end{center}

\subsection*{Filling the Clock}

The Supply Clock fills when:

\begin{itemize}
    \item Extended travel without provisioning.
    \item GM spends 2+ CP on logistics failures.
    \item The party chooses to travel light for advantage.
    \item Failed Survival or Craft rolls related to provisioning.
\end{itemize}

\subsection*{Emptying the Clock}

\begin{itemize}
    \item Reaching civilization resets to Full.
    \item Group Survival check (Wits + Survival, DV 2) clears 1 segment.
    \item Downtime in safety removes 1 segment.
    \item Successful provisioning actions can reduce segments.
\end{itemize}

\textbf{Example}: A week-long sea passage across the Dolmis with uncertain winds. A failed Navigation roll causes the GM to spend 2 CP---filling two segments. The party is now at Low Supply. A second failed roll fills another segment---Dangerously Low. Fatigue sets in. The sea, once a path, now gnaws at their endurance.

\section*{Fatigue: The Weight of the World}\index{Fatigue}

Fatigue represents \textbf{exhaustion, hunger, and strain}. It is cumulative and persistent. Each level of Fatigue forces the character to \textbf{re-roll one success} on their next action.

\subsection*{Fatigue Effects}

\begin{itemize}
    \item \textbf{1 Fatigue}: Re-roll one success.
    \item \textbf{2 Fatigue}: Re-roll one success (cumulative).
    \item \textbf{3 Fatigue}: Re-roll two successes.
    \item \textbf{4 Fatigue}: Collapse, KO, or spiritual break. Out of the scene until treated.
\end{itemize}

\subsection*{Clearing Fatigue}

\begin{itemize}
    \item A night's rest with adequate Supply removes 1 Fatigue.
    \item Fatigue cannot be removed while the party is Dangerously Low or Out of Supply.
    \item Medical attention (Presence + Heal, DV 2) can remove 1 Fatigue during downtime.
\end{itemize}

\textbf{Narrative Note}: Fatigue is not just physical---it can reflect mental strain, grief, or spiritual exhaustion. A failed ritual might leave a caster \textbf{Fatigue 2} from the backlash alone.

\section*{Followers and Assets: Power Beyond the Self}\index{Followers}\index{Assets}

In Fate's Edge, players can invest XP into \textbf{Followers} and \textbf{Assets}---tools that extend their reach beyond personal skill. These are not mere stat blocks---they are \textbf{story agents} with their own motivations, risks, and narrative arcs.

\subsection*{Followers: On-Scene Allies}\index{Followers!on-scene}

Followers are \textbf{on-screen allies} who can assist in their specialty. They are bought with XP and tracked by a \textbf{Cap}\index{Cap} (their maximum assist bonus).

Cost: A follower with Specialty Cap C costs C² XP. Downtime = 1--3 days to recruit and brief.

\subsection*{Assisting in Scenes}

Followers assist by adding dice to your rolls:

\begin{itemize}
    \item Assist dice come from the helper, not the leader.
    \item Total Assist on any roll (from any sources) remains hard-capped at +3. Exception: The "Exceptional Coordination"\index{Exceptional Coordination} Talent allows one follower to provide +4 assist dice.
    \item When applicable, the follower adds help dice equal to \textbf{min(C, the helper's relevant Skill)}, capped at +3 dice.
    \item Slot Limit: Only one follower may assist a given action.
\end{itemize}

\subsection*{Follower Initiative Actions}\index{Followers!initiative actions}

Once per scene (across the party), one on-screen follower may take a small independent action:

\begin{itemize}
    \item Scout \& Signal --- Change an ally's next action position to Controlled.
    \item Distract \& Draw --- Reduce a kinetic rail (Hunt/Escape/Hazard) by –1 tick.
    \item Fetch \& Carry --- Move a small object through danger.
\end{itemize}

\textbf{Cost:} Mark Exposure +1 or Harm 1 on that follower.

\subsection*{Follower Upkeep}\index{Followers!upkeep}

\begin{itemize}
    \item Each Downtime, pay XP equal to Cap or spend a Scene tending the relationship.
    \item Risk: If the GM spends 2+ Complication Points\index{Complication Point (CP)} on an action you take with assistance, they may mark Exposure or Harm on the follower instead of applying other consequences, if fictionally appropriate.
    \item Off-Screen Capability: Once per downtime, a follower with Cap 3 or higher can solve one significant problem but generates 1 CP for party. The GM must describe how the follower's action creates story consequences for the CP generated.
\end{itemize}

\subsection*{Follower Condition}\index{Followers!condition}

Followers track \textbf{Exposure}\index{Exposure} and \textbf{Harm}\index{Harm}:

\begin{description}
    \item[Exposure] --- Heat, attention, stress, or narrative pressure placed upon the follower.
    \item[Harm] --- Injury, trauma, fatigue, or direct damage to the follower.
\end{description}

\textbf{States:}
\begin{itemize}
    \item \textbf{Maintained} --- Reliable and ready.
    \item \textbf{Neglected} --- Needs downtime or care. Impose a -1 die penalty to their assistance.
    \item \textbf{Compromised} --- Captured, defected, lost, or incapacitated. Cannot assist until recovered.
\end{itemize}

\subsection*{Assets: Off-Scene Influence}\index{Assets!off-scene}

Assets are \textbf{off-screen resources}---titles, safehouses, spy rings, charters. They do not act in scenes directly, but they change the fiction and provide leverage when you return to the table.

\begin{description}
    \item[Minor (4 XP, 1 day)] --- Safehouse, small shop, petty title, local contact network.
    \item[Standard (8 XP, 1 week)] --- Noble title, guild section, spy ring, workshop.
    \item[Major (12 XP, 1 month)] --- City license, regional network, fortress lease.
\end{description}

\subsection*{Using Assets}\index{Assets!usage}

Assets provide benefits:

\begin{itemize}
    \item \textbf{Off-Screen Effect:} Use each Asset's listed Off-Screen effect once per session for free.
    \item \textbf{On-Screen Activation:} To reshape the current scene, spend 1 Boon\index{Boons}.
    \item \textbf{Downtime Activation:} A player may activate an off-screen asset at the very start of a campaign or during Downtime. It costs 2 XP or 1 Boon to activate.
    \item The Asset must have scope and reach for the intended effect.
\end{itemize}

\subsection*{Asset Condition}\index{Assets!condition}

All Assets have a \textbf{Condition Track}\index{Condition Track}:

\begin{description}
    \item[Maintained] --- Full capability. Functions normally.
    \item[Neglected] --- Impaired. Impose a -1 die penalty when used; requires attention.
    \item[Compromised] --- Unavailable. Cannot be used until repaired or recovered.
\end{description}

\section*{Boons: The Currency of Resilience}\index{Boons}

Boons are \textbf{narrative tokens} earned by embracing risk and moving the story forward. They reward \textbf{failure with texture}, not failure with nothing.

\subsection*{Earning Boons}

\begin{itemize}
    \item On a failed roll with meaningful Complications (see Fail Forward, Chapter \ref{ch:core-procedures}).
    \item Through clever or risky roleplay that drives the story.
    \item Via bond-driven actions with intricate descriptions.
    \item Through GM discretion for exceptional collaborative play.
\end{itemize}

\subsection*{Boon Economy}\index{Boon economy}

\begin{itemize}
    \item \textbf{Holding cap:} You can hold at most 5 Boons.
    \item \textbf{Carryover Limit:} At the end of each scene, reduce held Boons to a maximum of 2. Excess Boons are lost.
    \item \textbf{Conversion:} Once per session, in downtime, you may convert 2 Boons → 1 XP (max 2 XP via conversion per session).
\end{itemize}

\subsection*{Using Boons}

\begin{itemize}
    \item \textbf{Re-roll one die} after seeing the pool.
    \item \textbf{Activate an Asset} for on-screen effect.
    \item \textbf{Power Rites} that require Boon expenditure.
    \item \textbf{Improve Position} by one step (1 Boon).
    \item \textbf{Clear 1 tick} from a spirit's Leash (Pact-Whisperer, 1 Boon per round).
\end{itemize}

\subsection*{Anti-Fishing Measures}\index{Anti-Fishing}

To maintain healthy game flow:

\begin{itemize}
    \item \textbf{Once/Scene Cap:} At most \textbf{2 Boons from failures} per character per scene.
    \item \textbf{Repetition Rule:} Same approach + same stakes in the same scene cannot award another Boon.
    \item \textbf{Position Gate:} Controlled tests with trivial fallout do not award Boons.
\end{itemize}

\textbf{Design Note}: Boons are not a "get out of jail free" card. They are earned by \textbf{leaning into the fiction}, not by fishing for failure. Reward players who take risks, not those who roll badly on purpose.

\section*{XP Awards: Growth Through Choice}\index{XP awards}\index{Experience Points!awards}

XP in Fate's Edge is \textbf{meaningful currency}. It is not handed out for showing up---it is earned through \textbf{engagement, risk, and narrative impact}.

\subsection*{Session Awards}

\begin{itemize}
    \item \textbf{Table Attendance}: +2 XP
    \item \textbf{Major Objective Reached}: +2--4 XP
    \item \textbf{Discovery or Lore Unlocked}: +1--2 XP
    \item \textbf{Hard Choice Embraced}: +1--2 XP
    \item \textbf{Complication Spotlight}: +1--3 XP
    \item \textbf{Bond/Flag Driven Play}: +1--2 XP
    \item \textbf{GM Curveball Award}: +0--3 XP
\end{itemize}

\subsection*{Milestones}\index{milestones}

\begin{itemize}
    \item +8--12 XP to all players at the conclusion of a major story arc.
    \item +2 XP bonus to one player for a signature moment of the arc.
\end{itemize}

\subsection*{Complication Dividend}\index{Complication Dividend}

\begin{itemize}
    \item \textbf{Face Card (J/Q/K)}: +1 XP
    \item \textbf{Ace}: +2 XP
\end{itemize}

\section*{Campaign Resources: Mandate and Crisis}\index{Campaign resources}

At the campaign level, two clocks track the party's influence and the world's resistance:

\subsection*{Mandate Clock (0--6)}\index{Mandate}\index{Campaign Clocks!Mandate}

Tracks the party's public legitimacy and buy-in:
\begin{itemize}
    \item High Mandate: Allies are easier to find, resources more available.
    \item Low Mandate: Suspicion, bureaucratic obstacles, reduced support.
\end{itemize}

\subsection*{Crisis Clock (0--6)}\index{Crisis}\index{Campaign Clocks!Crisis}

Tracks the opposition engine (rivals, pressure rails, attrition):
\begin{itemize}
    \item Rising Crisis: Complications escalate, enemies grow bolder.
    \item Managed Crisis: Breathing room, opportunities to strike back.
\end{itemize}

\section*{Combat Resource Management}\index{combat resources}

In combat, resource management takes on new urgency. The same systems that govern exploration and downtime now operate under pressure, with immediate consequences.

\subsection*{Supply in Combat}\index{Supply Clock!combat}

Extended combat encounters can drain resources rapidly:

\begin{itemize}
    \item \textbf{Intense Combat}: GM may spend 1 CP to fill 1 Supply segment.
    \item \textbf{Prolonged Engagement}: Each hour of sustained combat adds 1 Supply segment.
    \item \textbf{Ammunition Depletion}: Ranged weapons may run low, requiring scavenging actions.
\end{itemize}

\subsection*{Fatigue in Combat}\index{Fatigue!combat}

Combat fatigue compounds existing strain:

\begin{itemize}
    \item \textbf{Each Round}: Characters with existing Fatigue re-roll additional successes equal to their Fatigue level.
    \item \textbf{Critical Exhaustion}: Reaching 4 Fatigue during combat causes immediate collapse.
    \item \textbf{Recovery}: Cannot clear Fatigue during active combat.
\end{itemize}

\subsection*{Follower Combat Integration}\index{Followers!combat}

Followers in combat face unique risks and opportunities:

\begin{itemize}
    \item \textbf{Combat Assistance}: Followers can assist in combat rolls using their Cap.
    \item \textbf{Follower Risk}: 2+ CP spent in combat can endanger assisting followers (mark Exposure or Harm).
    \item \textbf{Initiative Actions}\index{Initiative Actions}: Followers can take combat-relevant independent actions (cost: Exposure +1 or Harm 1).
    \item \textbf{Combat Exposure}: Each time a follower acts on-screen in high-risk combat, mark Exposure +1 after the second such beat this scene.
\end{itemize}

\subsection*{Asset Combat Activation}\index{Assets!combat}

Assets can be activated for immediate combat advantage:

\begin{itemize}
    \item \textbf{1 Boon}: Activate asset for combat advantage.
    \item \textbf{Environmental Assets}: Terrain features, fortifications, magical wards.
    \item \textbf{Compromise Risk}: Combat in certain locations can damage relevant assets.
\end{itemize}

\subsection*{Tactical Clocks as Resources}\index{Tactical Clocks}\index{Tactical Clocks!resources}

Tactical clocks represent persistent combat conditions that drain party resources:

\begin{itemize}
    \item \textbf{Mob Overwhelm} (6)\index{Tactical Clocks!Mob Overwhelm}: Enemy numbers become advantage---forces Supply depletion and Fatigue.
    \item \textbf{Fatigue Spiral} (4)\index{Tactical Clocks!Fatigue Spiral}: Exhaustion affects performance---accelerates existing Fatigue.
    \item \textbf{Morale Collapse} (6)\index{Tactical Clocks!Morale Collapse}: Fear undermines effectiveness---generates CP and reduces effectiveness.
    \item \textbf{Environmental Collapse} (8)\index{Tactical Clocks!Environmental Collapse}: Terrain/fire/building failure---creates new Supply and safety concerns.
\end{itemize}

\section*{Narrative First: The Fiction Is the Ledger}\index{narrative first}

In Fate's Edge, arrows, rations, and waterskins are tracked only in the fiction. Mechanics engage only when those resources become scarce. The focus is always on \textbf{narrative tension}, not bookkeeping.

Let the world breathe. Let the fiction lead. And when the dice say the world pushes back---\textbf{listen.}

\end{chapter}
