\chapter{Magic and Backlash}

In \textbf{Fate’s Edge}, magic is not a clean or safe art. It is a \textbf{dangerous negotiation with forces beyond mortal comprehension}. Every spell is a gamble—power on one side of the scale, consequence on the other. As the GM, your role is to make magic \textbf{feel weighty}, \textbf{thematic}, and \textbf{alive with risk}.

\section*{Philosophy of Magic}

Magic in Fate’s Edge is not about optimization or damage output. It is about \textbf{shaping the world through will, risk, and resonance}. The dice never merely ask, “Does it work?”—they whisper, “\textbf{What is the cost?}”

\begin{itemize}
    \item \textbf{Volatility}: Magic is not fully understood. Every casting pushes against boundaries that resist being bent.
    \item \textbf{Risk}: Every spell generates \textbf{Complication Points}. These do not vanish—they manifest as \textbf{Backlash}.
    \item \textbf{Narrative Weight}: Casting is always a story moment. Even a “successful” spell alters the scene in ways the caster did not intend.
    \item \textbf{Thematic Consequence}: Backlash aligns with the opposing or uncontrolled element of the Art invoked.
\end{itemize}

\section*{The Casting Loop}

All spellcasting in Fate’s Edge follows a structured sequence called the \textbf{Casting Loop}. It unfolds across two phases of play: gathering strength, then weaving it into form.

\subsection*{1. Channel}

The caster focuses, rolling \textbf{Wits + Arcana} to gather \textbf{Potential}. Each success becomes fuel for shaping the spell. Each \textbf{1 rolled adds Complication Points immediately}.

\textbf{Example}: Kestra the Arcanist rolls to \textbf{Channel} a firebolt. She gets 4 successes and 2 CP. The GM spends 1 CP: a spark leaps from her fingers, scorching her sleeve.

\subsection*{2. Weave}

On the following turn, the caster rolls \textbf{Wits + (Art)} to shape Potential into a defined effect. The \textbf{Description Ladder} applies:

\begin{itemize}
    \item \textbf{Basic}: Roll as-is.
    \item \textbf{Detailed}: Re-roll one 1.
    \item \textbf{Intricate}: Re-roll all 1s and add one small narrative flourish if successful.
\end{itemize}

\textbf{Example}: Kestra Weaves the firebolt with an Intricate description—she calls the flame in the shape of a hawk. The GM allows the reroll and grants a small flourish: the fire-hawk circles once before striking, distracting an enemy.

\subsection*{3. Backlash}

Complication Points spent by the GM manifest as \textbf{uncontrolled consequences}. These are \textbf{thematic} to the Art and scale with the number of points spent.

\section*{Backlash Severity Table}

\begin{tabular}{|c|l|}
\hline
\textbf{CP Spent} & \textbf{Typical Consequence} \\
\hline
1–2 & Minor nuisance or tell; short-lived cost, noise, or reveal. \\
3–4 & Noticeable setback: a real hazard, condition, or new pressure clock. \\
5+ & Major turn: scene shifts, a new foe/clock enters, or severe condition. \\
\hline
\end{tabular}

\textbf{Example}: Kestra Weaves her firebolt but rolls two 1s. The GM spends 3 CP for Backlash: the flames flare too wide, catching a tapestry and starting a small Hazard clock.

\section*{Thematic Backlash by Art}

\begin{itemize}
    \item \textbf{Pyromancy}: Flames leap to unattended surfaces; smoke blinds allies; heat weakens structures.
    \item \textbf{Umbramancy}: Illusions persist too long; unseen things whisper truths best left hidden; morale crumbles.
    \item \textbf{Geomancy}: Stone shifts unexpectedly; tremors crack foundations; runes pulse with old warnings.
    \item \textbf{Hydromancy}: Tides turn too fast; water becomes too mirror-still; currents pull in the unwary.
    \item \textbf{Vitalism}: Life is drawn from surroundings; plants wither; healing leaves scars.
\end{itemize}

\section*{Ritual Casting (Optional Rule)}

Some workings are too great for a single will. A \textbf{ritual} allows multiple characters to join forces, pooling their dice and narrative effort—but the risk of Backlash rises with every participant.

\subsection*{Ritual Helper Cap}

You may draw on \textbf{ceil(Arcana/2)} helpers (max 3). Each helper contributes as per the ritual rules and adds their own Complication risk.

\subsection*{Ritual Procedure}

\begin{enumerate}
    \item \textbf{Declare the Ritual}: The primary caster names the effect and how others can help.
    \item \textbf{Channel Together}:
        \begin{itemize}
            \item Primary caster rolls \textbf{Wits + Arcana}.
            \item Each assistant rolls \textbf{Spirit + Relevant Skill}.
            \item Each success adds +1 Potential.
            \item Each 1 contributes a Complication Point (CP) to the shared pool.
        \end{itemize}
    \item \textbf{Weave}: The primary caster rolls \textbf{Wits + (Art)}. Assistants may add dice if their contributions are narrated in play.
    \item \textbf{Backlash}: Total all CPs. Apply normal Complication spending rules. Increase severity by +1 tier per assistant beyond the first.
\end{enumerate}

\textbf{Why Use Rituals?}

\begin{itemize}
    \item \textbf{Higher Ceiling}: Achieve effects impossible through normal spellcasting.
    \item \textbf{Shared Spotlight}: Every participant has narrative agency in the casting.
    \item \textbf{Bigger Risk}: More dice mean more 1s. Consequences can spread across the entire party.
\end{itemize}

\section*{Design Intent}

Magic should \textbf{feel dangerous}, \textbf{thematic}, and \textbf{alive}. It should never be a shortcut. Every magical act alters not just the world, but the flow of the narrative itself. The dice are not your enemy—they are your collaborator in crafting a world where \textbf{power always demands a price}.

\textbf{GM Tip}: When a player channels magic, describe the air shifting, the runes flaring, the tension in the weave. Make the world \textbf{react} to their casting. Let magic feel \textbf{alive}.

\end{chapter}
