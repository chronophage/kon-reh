% !TEX root = ../fates_edge_players_guide.tex

\chapter{Magic and Backlash}

In \textbf{Fate's Edge}, magic is not a clean or safe art. It is a \textbf{dangerous negotiation with forces beyond mortal comprehension}. Every spell is a gamble—power on one side of the scale, consequence on the other. As the GM, your role is to make magic \textbf{feel weighty}, \textbf{thematic}, and \textbf{alive with risk}.

\section*{Philosophy of Magic}

Magic in Fate's Edge is not about optimization or damage output. It is about \textbf{shaping the world through will, risk, and resonance}. The dice never merely ask, "Does it work?"—they whisper, "\textbf{What is the cost?}"

\begin{itemize}
    \item \textbf{Volatile by Design}: Magic is not fully understood.
    \item \textbf{Risk Embodied}: Each spell generates Complication Points.
    \item \textbf{Narrative Weight}: Casting is always a story moment.
    \item \textbf{Thematic Consequence}: Backlash is not arbitrary; it aligns with the opposing or uncontrolled element of the Art invoked.
\end{itemize}

\section*{The Caster's Burden}

Magicians are defined not by what they can do, but by what they are willing to risk.

\section*{The Casting Loop}

All spellcasting in Fate's Edge follows a structured sequence called the \textbf{Casting Loop}. It unfolds across two phases of play: gathering strength, then weaving it into form.

\subsection*{1. Channel}

The caster focuses, rolling \textbf{Wits + Arcana} to gather \textbf{Potential}. Each success becomes fuel for shaping the spell. Each \textbf{1 rolled adds Complication Points immediately}.

\textbf{Example}: Kestra the Arcanist rolls to \textbf{Channel} a firebolt. She gets 4 successes and 2 CP. The GM spends 1 CP: a spark leaps from her fingers, scorching her sleeve.

\subsection*{2. Weave}

On the following turn, the caster rolls \textbf{Wits + (Art)} to shape Potential into a defined effect. The \textbf{Description Ladder} applies:

\begin{itemize}
    \item \textbf{Basic Action}: Roll as-is. All 1s remain as CP.
    \item \textbf{Detailed Action}: A clear, descriptive flourish allows the player to re-roll one die showing 1.
    \item \textbf{Intricate Action}: A richly described, multi-sensory action allows the player to re-roll all dice showing 1, and add one positive narrative flourish to the scene if they succeed.
\end{itemize}

\textbf{Example}: Kestra Weaves the firebolt with an Intricate description—she calls the flame in the shape of a hawk. The GM allows the reroll and grants a small flourish: the fire-hawk circles once before striking, distracting an enemy.

\subsection*{3. Backlash}

Complication Points spent by the GM manifest as \textbf{uncontrolled consequences}. These are \textbf{thematic} to the Art and scale with the number of points spent.

\textbf{Mitigation}: Boons do not reduce CP unless a Talent/Asset explicitly says "Mitigate CP."

\section*{Backlash Severity Table}

\begin{center}
\begin{tabular}{cl}
\toprule
\textbf{CP Spent} & \textbf{Typical Consequence} \\
\midrule
1–2 & Minor nuisance or tell (noise, fatigue, brief distraction) \\
3–4 & Noticeable setback (hazard clock, condition, new pressure) \\
5+ & Major turn (scene shift, new foe, severe condition) \\
\bottomrule
\end{tabular}
\end{center}

\textbf{Example}: Kestra Weaves her firebolt but rolls two 1s. The GM spends 3 CP for Backlash: the flames flare too wide, catching a tapestry and starting a small Hazard clock.

\section*{Common Magical Arts}

Each Art has its own flavor and risk. Below are examples:

\begin{description}
    \item[Pyromancy] — Fire and heat. Backlash: Flames leap to unattended surfaces, smoke blinds allies, or the heat weakens structures.
    \item[Umbramancy] — Shadow and silence. Backlash: Illusions persist too long, unseen things whisper truths best left hidden, morale crumbles.
    \item[Stormcraft] — Wind and lightning. Backlash: Winds scatter allies' plans, lightning arcs toward unintended targets, storms linger beyond the caster's will.
    \item[Geomancy] — Stone and structure. Backlash: rigidity, slow movement, guardians awaken.
    \item[Hydromancy] — Water and flow. Backlash: stagnation, flooding, pests drawn.
    \item[Vitalism] — Life and healing. Backlash: overgrowth, exhaustion, sympathetic drain.
    \item[Thaumaturgy] — Divine or holy magic. Backlash: flickering sanctity, beacon effects, spiritual fatigue.
\end{description}

\section*{Ritual Casting (Optional Rule)}

Some workings are too great for a single will. A \textbf{ritual} allows multiple characters to join forces, pooling their dice and narrative effort—but the risk of Backlash rises with every participant.

\subsection*{Ritual Helper Cap}

You may draw on \textbf{ceil(Arcana/2)} helpers (max 3).

\subsection*{Ritual Procedure}

\begin{enumerate}
    \item \textbf{Declare the Ritual}.
    \item \textbf{Channel Together}.
    \item \textbf{Weave}.
    \item \textbf{Backlash}.
\end{enumerate}

\subsection*{Ritual Mechanics}

\begin{itemize}
    \item Helpers may use different relevant skills if their procedure is fictionally distinct.
    \item CP from Channel resolves on that roller. CP from Weave is assigned to the primary caster.
\end{itemize}

\textbf{Why Use Rituals?}

\begin{itemize}
    \item \textbf{Higher Ceiling}: Achieve effects impossible through normal spellcasting.
    \item \textbf{Shared Spotlight}: Every participant has narrative agency in the casting.
    \item \textbf{Bigger Risk}: More dice mean more 1s. Consequences can spread across the entire party.
\end{itemize}

\section*{Magic in Combat}

Spellcasting in combat feeds the same consequence economy as all other actions, but with unique tactical applications and risks.

\subsection*{Combat Casting Loop}

The Casting Loop operates in combat with specific tactical implications:

\begin{itemize}
    \item \textbf{Channel}: Can be done as an action during combat turns
    \item \textbf{Weave}: Requires the following turn, making casters vulnerable
    \item \textbf{Backlash}: Channel/Weave Backlash CP applies to tactical situation
\end{itemize}

\subsection*{Combat Position Effects}

Position affects magical casting in combat:

\begin{itemize}
    \item \textbf{Controlled}: +1 die to Channel, reduced Backlash risk
    \item \textbf{Risky}: Standard casting conditions
    \item \textbf{Desperate}: -1 die to Channel, increased Backlash severity
\end{itemize}

\subsection*{Tactical Applications}

Spells can shift the tactical situation:

\begin{itemize}
    \item \textbf{Position Shifting}: Spells can improve or worsen combat position
    \item \textbf{Clock Creation}: Magic can create or advance tactical clocks
    \item \textbf{Consequence Generation}: Spells generate combat-specific consequences
\end{itemize}

\subsection*{Magic Consequence Cascade}

Magic consequences cascade through existing combat systems:

\begin{itemize}
    \item \textbf{Spades}: Physical harm, weapon effects, area control
    \item \textbf{Hearts}: Morale effects, fear, command disruption
    \item \textbf{Clubs}: Resource depletion, gear damage, fatigue
    \item \textbf{Diamonds}: Environmental hazards, reinforcements, tactical setbacks
\end{itemize}

\section*{Prestige Magical Abilities}

\begin{itemize}
    \item \textbf{Echo-Walker's Step} (High Elf, Cost: 20 XP; Req: Wits 5, Arcana 4): 
1/arc, \emph{observe} a perfect echo of a past event at your location (no retconning). 
GM immediately banks +2 CP; scenes touching that memory carry an omen. Grants DV −1 on one action that uses the revealed truth.
    \item \textbf{Warglord} (Ykrul, Cost: 18 XP; Req: Body 5, Command 3): 
Once per campaign, unify scattered warbands into a single host for a season. Start a \emph{Logistics} clock and a \emph{Grudge} clock; either one filling fractures the host.
    \item \textbf{Spirit-Shield} (Aeler, Cost: 15 XP; Req: Spirit 4, Insight 3): 
1/session, erase up to 3 CP from an ally's \emph{current} roll; you immediately mark Fatigue +1 and the GM banks +1 CP as backlash.
\end{itemize}

\section*{Design Intent}

Magic should \textbf{feel dangerous}, \textbf{thematic}, and \textbf{alive}. It should never be a shortcut. Every magical act alters not just the world, but the flow of the narrative itself. The dice are not your enemy—they are your collaborator in crafting a world where \textbf{power always demands a price}.

\textbf{GM Tip}: When a player channels magic, describe the air shifting, the runes flaring, the tension in the weave. Make the world \textbf{react} to their casting. Let magic feel \textbf{alive}.

\end{chapter}

