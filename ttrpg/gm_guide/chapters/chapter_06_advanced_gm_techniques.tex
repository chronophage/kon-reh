\chapter{Advanced GM Techniques}

As the campaign deepens and the stakes rise, the GM must evolve from storyteller to \textbf{architect of tension}. This chapter explores advanced techniques for managing complex scenes, faction interplay, and custom content creation. These tools will help you keep the world dynamic, the choices meaningful, and the consequences \textbf{echoing}.

\section*{Using the Deck of Consequences}

The \textbf{Deck of Consequences} is more than a randomizer—it is a \textbf{thematic engine} that externalizes risk and ensures that setbacks feel consistent and fair.

\subsection*{When to Draw}

After a roll generates Complication Points, the GM may choose to:

\begin{itemize}
    \item \textbf{Spend CP directly} for tailored consequences.
    \item \textbf{Draw from the Deck} for a thematically fitting twist.
\end{itemize}

Never do both for the same roll. If a drawn card contradicts established fiction, reinterpret or redraw to fit the suit and tone.

\subsection*{Suit Meanings}

\begin{itemize}
    \item \textbf{Hearts (♠)}: Emotional, social, or relational fallout.
    \item \textbf{Diamonds (♢)}: Resource strain, economic or material cost.
    \item \textbf{Clubs (♣)}: Physical harm, environmental danger, or escalation.
    \item \textbf{Spades (♠)}: Mystical, narrative, or positional twists.
\end{itemize}

\subsection*{Rank Severity}

\begin{itemize}
    \item \textbf{2–5}: Minor inconvenience or flavor complication.
    \item \textbf{6–9}: Moderate setback with some narrative teeth.
    \item \textbf{10–King}: Severe twist; alters stakes of the scene.
    \item \textbf{Ace}: Catastrophic turn; reshapes narrative or mission goal.
\end{itemize}

\textbf{Example}: A failed lockpick generates 2 CP. The GM draws: \textbf{Clubs 7} — “A guard rounds the corner just as the lock clicks.” The PCs are not caught—but they are seen. The alarm clock begins to tick.

\section*{Travel and Exploration}

Travel in Fate’s Edge is not a downtime skip—it is a \textbf{narrative layer} filled with tension, discovery, and risk. Use the card-based travel system to seed each leg with place, people, pressure, and leverage.

\subsection*{Drawing Route Legs}

For each travel leg, draw:

\begin{itemize}
    \item \textbf{Spade}: Sets the scene (place).
    \item \textbf{Heart}: Introduces the local actor or faction.
    \item \textbf{Club}: Brings pressure (from Wilds or destination deck).
    \item \textbf{Diamond}: Codified outcome (papers, escorts, rights).
\end{itemize}

Set a travel clock by the highest rank:
\begin{itemize}
    \item \textbf{2–5} → 4 segments
    \item \textbf{6–10} → 6 segments
    \item \textbf{J/Q/K} → 8 segments
    \item \textbf{Ace} → 10 segments
\end{itemize}

\textbf{Example}: Traveling the \textbf{Aelerian Passes}, the PCs draw: Spade (Avalanche gallery), Heart (Geometer), Club (Engineer requisition), Diamond (Underway Pass). Clock: 8. On failure, the GM spends CP to trigger a rockslide—Hazard +2.

\section*{Faction Dynamics}

Factions are \textbf{living entities} with goals, rivals, and shifting loyalties. They are not static backdrops—they are \textbf{players in the story}.

\subsection*{Creating Factions}

Each faction should have:

\begin{itemize}
    \item \textbf{Core Motive}: What do they want?
    \item \textbf{Key Figures}: Who leads or represents them?
    \item \textbf{Resources}: What can they bring to bear?
    \item \textbf{Weaknesses}: What makes them vulnerable?
\end{itemize}

\subsection*{Faction Clocks}

Track factional pressure with clocks:

\begin{itemize}
    \item \textbf{Rising Influence} (6): Gaining power, allies, or territory.
    \item \textbf{Internal Strife} (6): Leadership challenged, morale low.
    \item \textbf{Public Scandal} (4): Reputation damaged, support wanes.
\end{itemize}

\textbf{Example}: The \textbf{Viterra Dawn Knights} gain Rising Influence as they rally to the new Queen—but suffer Internal Strife as old commanders resist her reforms.

\section*{Creating Custom Content}

Fate’s Edge thrives on \textbf{player agency} and \textbf{world customization}. When designing new Talents, Assets, or Prestige Abilities, follow these principles:

\begin{itemize}
    \item \textbf{Narrative First}: Does it reinforce a theme or culture?
    \item \textbf{Mechanical Balance}: Does it fit within the XP economy?
    \item \textbf{Fictional Integration}: Can it be explained in-world?
\end{itemize}

\subsection*{Designing Talents}

\begin{itemize}
    \item \textbf{Early (3–5 XP)}: Once/scene ability, +1 die, or minor narrative edge.
    \item \textbf{Mid (6–10 XP)}: Once/session ability, significant edge, or codified outcome.
    \item \textbf{Prestige (12+ XP)}: Once/arc ability, scene-reshaping, or faction-level impact.
\end{itemize}

\textbf{Example}: \textbf{Backlash Soothing} (Wood Elf, 6 XP) — Once per session, reduce a magical Backlash by 2 points in natural terrain.

\subsection*{Designing Assets}

\begin{itemize}
    \item \textbf{Minor (4 XP)}: Safehouse, Petty Title.
    \item \textbf{Standard (8 XP)}: Spy Ring, Charter.
    \item \textbf{Major (12 XP)}: Fortress Lease, Mercantile Network.
\end{itemize}

Each Asset should have:
\begin{itemize}
    \item \textbf{Activation Cost}: 1 Boon or 2 XP.
    \item \textbf{Scope}: What can it plausibly do?
    \item \textbf{Fictional Hook}: Why does it exist in the world?
\end{itemize}

\section*{Running Complex Scenarios}

\subsection*{Heists and Infiltration}

\begin{itemize}
    \item \textbf{Positioning}: Controlled entries, distractions, asset use.
    \item \textbf{Social Rails}: Curfew, Crowd, Sanctity.
    \item \textbf{Physical Rails}: Hazard, Hunt, Escape.
\end{itemize}

\textbf{GM Tip}: Let the PCs plan—but make the world react. A guard changes shift. A noble arrives early. The lock is newer than expected.

\subsection*{Battles and Skirmishes}

\begin{itemize}
    \item \textbf{Group Actions}: Use the Lead system to coordinate.
    \item \textbf{Follower Risk}: Helpers can be endangered on 2+ CP spends.
    \item \textbf{Clocks}: Hazard (terrain), Hunt (enemy approach), Escape (retreat).
\end{itemize}

\subsection*{Political Intrigue}

\begin{itemize}
    \item \textbf{Leverage}: Diamonds and social rails determine influence.
    \item \textbf{Allies and Rivals}: Represented by Assets and Followers.
    \item \textbf{Public Image}: Tied to Mandate and Crisis clocks.
\end{itemize}

\section*{Narrative First: The World Reacts}

In Fate’s Edge, the world is not a puzzle to be solved—it is a \textbf{living system} that responds to player choices. Let factions shift. Let consequences ripple. And above all—let the story unfold.

Because in the end, it is not the GM who writes the legend.

It is the players.

You simply hold the quill.

\end{chapter}
