\chapter{Advanced GM Techniques}\index{advanced techniques}

In \textbf{Fate's Edge}, as the campaign deepens and the stakes rise, the GM must evolve from storyteller to \textbf{architect of tension}. This chapter explores advanced techniques for managing complex scenes, faction interplay, and custom content creation. These tools will help you keep the world dynamic, the choices meaningful, and the consequences \textbf{echoing}.

\section{Complication Point Management}\index{Complication Points!management}

The GM should manage Complication Point (CP) spending to maintain dramatic tension while preserving player agency and game flow. CP spending scales with character tier but is subject to hard limits to ensure playability.

\subsection{Core Principles}\index{Complication Points!core principles}
\begin{itemize}
    \item \textbf{Narrative Coherence}: All CP spends within a scene should connect thematically
    \item \textbf{Player Agency}: Complications create interesting choices, not insurmountable obstacles
    \item \textbf{Progressive Escalation}: Higher tier characters naturally attract greater consequences
    \item \textbf{Resolution Paths}: Every complication thread should have potential resolution
\end{itemize}

\subsection{Spending Formula}\index{Complication Points!spending formula}
\textbf{Base CP = 4 + Character Tier}\index{Character Tier}
\begin{itemize}
    \item \textbf{Tier I (Rookie)}: 5 CP base
    \item \textbf{Tier II (Seasoned)}: 6 CP base
    \item \textbf{Tier III (Veteran)}: 7 CP base
    \item \textbf{Tier IV (Paragon)}: 8 CP base
    \item \textbf{Tier V (Mythic)}: 9 CP base
\end{itemize}

\subsection{Hard Limits}\index{Complication Points!hard limits}
\begin{itemize}
    \item \textbf{Standard Scenes}: Maximum 12 CP spending
    \item \textbf{Climactic Scenes}: Maximum 16 CP spending
    \item \textbf{Active Threads}: Maximum (Tier + 1) concurrent threads
    \item \textbf{Session Budget}: Maximum 20 CP total per session
\end{itemize}

\subsection{Banked CP Integration}\index{Complication Points!banked integration}
Banked CP from character complications count toward scene spending limits rather than adding to available CP. This prevents exponential complication stacking while honoring narrative debt.

\subsection{Thread Management}\index{threads!management}
Complication threads follow a natural escalation pattern:
\begin{itemize}
    \item \textbf{First Exposure}: 1-2 CP (Minor inconvenience)
    \item \textbf{Second Occurrence}: 2-4 CP (Moderate setback)
    \item \textbf{Third Strike}: 3-6 CP (Major consequence)
    \item \textbf{Resolution}: Thread concludes with narrative payoff
\end{itemize}

\begin{table}[htbp]
\centering
\caption{Complication Point Management by Tier}\index{Complication Points!management table}
\begin{tabular}{|c|c|c|c|c|c|}
\hline
\textbf{Tier} & \textbf{Base CP} & \textbf{Max Threads} & \textbf{Scene Cap} & \textbf{Climax Cap} & \textbf{Session Budget} \\
\hline
I (0-40 XP) & 5 CP & 2 threads & 12 CP & 16 CP & 20 CP \\
II (41-90 XP) & 6 CP & 3 threads & 12 CP & 16 CP & 20 CP \\
III (91-150 XP) & 7 CP & 4 threads & 12 CP & 16 CP & 20 CP \\
IV (151-220 XP) & 8 CP & 5 threads & 12 CP & 16 CP & 20 CP \\
V (221+ XP) & 9 CP & 6 threads & 12 CP & 16 CP & 20 CP \\
\hline
\end{tabular}
\end{table}

\begin{table}[htbp]
\centering
\caption{Complication Spending Safety Guidelines}\index{Complication Points!safety guidelines}
\begin{tabular}{|p{3.5cm}|p{5.5cm}|}
\hline
\textbf{Scenario} & \textbf{Guidance} \\
\hline
Standard Scenes & Spend 50-75\% of available CP budget; preserve some for escalation \\
Climactic Scenes & May use full CP allocation; ensure resolution opportunities \\
Teaching Moments & Explicit player consent required; time-limited; debrief afterward \\
New Players & Reduce CP spending by 25-50\%; focus on clear, actionable complications \\
Grimdark Mode & Reserved for veteran groups; explicit session zero discussion; safety tools active \\
\hline
\end{tabular}
\end{table}

\begin{table}[htbp]
\centering
\caption{Complication Thread Management}\index{threads!complication management}
\begin{tabular}{|p{3cm}|p{6cm}|}
\hline
\textbf{Thread Level} & \textbf{CP Escalation} \\
\hline
First Exposure & 1-2 CP (Minor inconvenience) \\
Second Occurrence & 2-4 CP (Moderate setback) \\
Third Strike & 3-6 CP (Major consequence) \\
Resolution & Thread concludes; narrative payoff provided \\
\hline
\end{tabular}
\end{table}

\section*{Using the Deck of Consequences}\index{Deck of Consequences}

The \textbf{Deck of Consequences} is more than a randomizer---it is a \textbf{thematic engine} that externalizes risk and ensures that setbacks feel consistent and fair.

\subsection*{Two Deck Systems (Compatibility)}\index{Deck of Consequences!two deck systems}

Fate's Edge uses two distinct card tools:

\paragraph{Travel Decks (regional, 52-card).}\index{Travel Decks}
\emph{Spade}=Place, \emph{Heart}=Actor, \emph{Club}=Pressure, \emph{Diamond}=Leverage. These power journeys and gates.

\paragraph{Deck of Consequences (scene drama).}
\emph{Hearts}=social fallout, \emph{Spades}=harm/escalation, \emph{Clubs}=material cost, \emph{Diamonds}=magical/spiritual disturbance.

\textit{Guidance:} Never mix suit meanings across decks. When a rule references ``Spade/Club/Diamond,'' it means \emph{Travel}. When it says ``Hearts/Spades/Clubs/Diamonds,'' it means \emph{Consequences}.

\subsection*{When to Draw}\index{Deck of Consequences!when to draw}

After a roll generates Complication Points, the GM may choose to:

\begin{itemize}
    \item \textbf{Direct Spend}: Translate CP into consequences/rail ticks immediately.
    \item \textbf{Deck Draw}: Draw up to \textbf{min(CP, 3)} cards and \textbf{synthesize a single twist} guided by suit and highest rank.
\end{itemize}

Never do both for the same roll. If a drawn card contradicts established fiction, reinterpret or redraw to fit the suit and tone.

\subsection*{Structure of the Deck}\index{Deck of Consequences!structure}

\begin{itemize}
    \item \textbf{Suits} = Domains of Complications
    \begin{itemize}
        \item Hearts\index{Hearts (suit)}: Emotional, social, or relational fallout.
        \item Spades\index{Spades (suit)}: Harm, danger, or escalation of conflict.
        \item Clubs\index{Clubs (suit)}: Resource strain, economic or material cost.
        \item Diamonds\index{Diamonds (suit)}: Magical, spiritual, or cosmic disturbances.
    \end{itemize}
    \item \textbf{Ranks} = Severity of Complications
    \begin{itemize}
        \item Ace--3: Minor inconvenience or flavor complication.
        \item 4--6: Moderate setback with some narrative teeth.
        \item 7--9: Significant consequence altering the course of action.
        \item 10--King: Major fallout, introducing new problems or lasting scars.
    \end{itemize}
\end{itemize}

\section*{Travel and Exploration}\index{travel}\index{exploration}

Travel in Fate's Edge is not a downtime skip---it is a \textbf{narrative layer} filled with tension, discovery, and risk. Use the card-based travel system to seed each leg with place, people, pressure, and leverage.

\subsection*{Core Travel Procedure}\index{travel!core procedure}

For each leg of a journey, draw 3--4 cards using the decks for your destination and controlling authority:

\begin{itemize}
    \item Spade from the destination deck: sets the scene (place).
    \item Heart from the destination deck: introduces the local actor or faction.
    \item Club from the Wilds (general hazards) or destination (if strongly policed): brings pressure.
    \item Diamond from the authority that gates the route: papers, escorts, rights, or exceptions.
\end{itemize}

Set a travel clock by the highest rank:
\begin{itemize}
    \item \textbf{2--5} → 4 segments
    \item \textbf{6--10} → 6 segments
    \item \textbf{J/Q/K} → 8 segments
    \item \textbf{Ace} → 10 segments
\end{itemize}

\textbf{Example}: Traveling the \textbf{Aelerian Passes}, the PCs draw: Spade (Avalanche gallery), Heart (Geometer), Club (Engineer requisition), Diamond (Underway Pass). Clock: 8. On failure, the GM spends CP to trigger a rockslide---Hazard +2.

\section*{Faction Dynamics}\index{factions}

Factions are \textbf{living entities} with goals, rivals, and shifting loyalties. They are not static backdrops---they are \textbf{players in the story}.

\subsection*{Creating Factions}\index{factions!creating}

Each faction should have:

\begin{itemize}
    \item \textbf{Core Motive}: What do they want?
    \item \textbf{Key Figures}: Who leads or represents them?
    \item \textbf{Resources}: What can they bring to bear?
    \item \textbf{Weaknesses}: What makes them vulnerable?
\end{itemize}

\subsection*{Faction Clocks}\index{factions!clocks}

Track factional pressure with clocks:

\begin{itemize}
    \item \textbf{Rising Influence} (6): Gaining power, allies, or territory.
    \item \textbf{Internal Strife} (6): Leadership challenged, morale low.
    \item \textbf{Public Scandal} (4): Reputation damaged, support wanes.
\end{itemize}

\textbf{Example}: The \textbf{Viterra Dawn Knights} gain Rising Influence as they rally to the new Queen---but suffer Internal Strife as old commanders resist her reforms.

\section*{Creating Custom Content}\index{custom content}

Fate's Edge thrives on \textbf{player agency} and \textbf{world customization}. When designing new Talents, Assets, or Prestige Abilities, follow these principles:

\begin{itemize}
    \item \textbf{Narrative First}: Does it reinforce a theme or culture?
    \item \textbf{Mechanical Balance}: Does it fit within the XP economy?
    \item \textbf{Fictional Integration}: Can it be explained in-world?
\end{itemize}

\subsection*{Designing Talents}\index{Talents!designing}

\begin{itemize}
    \item \textbf{General Talents}:
    \begin{itemize}
        \item Battle Instincts (Cost: 6 XP): Once per scene, re-roll a failed defense roll.
        \item Silver Tongue (Cost: 4 XP): Gain +1 die when persuading or deceiving through speech.
        \item Iron Stomach (Cost: 3 XP): Immune to mundane poisons and spoiled food; halve Complications from toxic sources.
        \item Exceptional Coordination (Cost: 8 XP): One follower can provide +4 assist dice.
    \end{itemize}
    \item \textbf{Racial or Cultural Talents}:
    \begin{itemize}
        \item Stone-Sense (Dwarves, Cost: 5 XP): Detect flaws in stone or earth; gain +1 die on Engineering or Craft rolls underground.
        \item Backlash Soothing (Wood Elves, Cost: 6 XP): Once per session, reduce a magical Backlash Complication by 2 points when in natural terrain.
        \item Blood Memory (Ykrul, Cost: 5 XP): After a battle, meditate to gain one temporary Skill die reflecting a foe's tactics for the next scene.
    \end{itemize}
    \item \textbf{Prestige Abilities}\index{Prestige Abilities}:
    \begin{itemize}
        \item Echo-Walker's Step (High Elf, Cost: 20 XP; Req: Wits 5, Arcana 4): 
1/arc, \emph{observe} a perfect echo of a past event at your location (no retconning). 
GM immediately banks +2 CP; scenes touching that memory carry an omen. Grants DV −1 on one action that uses the revealed truth.
        \item Warglord (Ykrul, Cost: 18 XP; Req: Body 5, Command 3): 
Once per campaign, unify scattered warbands into a single host for a season. Start a \emph{Logistics} clock and a \emph{Grudge} clock; either one filling fractures the host.
        \item Spirit-Shield (Aeler, Cost: 15 XP; Req: Spirit 4, Insight 3): 
1/session, erase up to 3 CP from an ally's \emph{current} roll; you immediately mark Fatigue +1 and the GM banks +1 CP as backlash.
    \end{itemize}
\end{itemize}

\subsection*{Designing Assets}\index{Assets!designing}

\begin{itemize}
    \item \textbf{Minor (4 XP)}: Safehouse, Petty Title.
    \item \textbf{Standard (8 XP)}: Spy Ring, Charter.
    \item \textbf{Major (12 XP)}: Fortress Lease, Mercantile Network.
\end{itemize}

Each Asset should have:
\begin{itemize}
    \item \textbf{Activation Cost}: 1 Boon\index{Boon}.
    \item \textbf{Scope}: What can it plausibly do?
    \item \textbf{Fictional Hook}: Why does it exist in the world?
\end{itemize}

\section*{Running Complex Scenarios}\index{complex scenarios}

\subsection*{Heists and Infiltration}\index{heists}

\begin{itemize}
    \item \textbf{Positioning}: Controlled entries, distractions, asset use.
    \item \textbf{Social Rails}: Curfew, Crowd, Sanctity.
    \item \textbf{Physical Rails}: Hazard, Hunt, Escape.
\end{itemize}

\textbf{GM Tip}: Let the PCs plan---but make the world react. A guard changes shift. A noble arrives early. The lock is newer than expected.

\subsection*{Battles and Skirmishes}\index{battles}

\begin{itemize}
    \item \textbf{Group Actions}: Use the Lead system to coordinate.
    \item \textbf{Follower Risk}: Helpers can be endangered on 2+ CP spends.
    \item \textbf{Clocks}: Hazard (terrain), Hunt (enemy approach), Escape (retreat).
\end{itemize}

\subsection*{Political Intrigue}\index{political intrigue}

\begin{itemize}
    \item \textbf{Leverage}: Diamonds and social rails determine influence.
    \item \textbf{Allies and Rivals}: Represented by Assets and Followers.
    \item \textbf{Public Image}: Tied to Mandate\index{Campaign Clocks!Mandate} and Crisis\index{Campaign Clocks!Crisis} clocks.
\end{itemize}

\section*{Advanced Combat Techniques}\index{combat!advanced techniques}

As campaigns progress, combat encounters become more complex and stakes higher. These advanced techniques help manage sophisticated tactical situations.

\subsection*{Environmental Combat}\index{combat!environmental}

Terrain and environmental factors create dynamic combat scenarios:

\begin{itemize}
    \item \textbf{Environmental Clocks}\index{clocks!Environmental}: Building collapse, fire spread, flooding
    \item \textbf{Positional Advantages}: High ground, cover, choke points
    \item \textbf{Hazard Integration}: Environmental dangers that generate CP
\end{itemize}

\subsection*{Mass Combat}\index{combat!mass}

Large-scale battles require special handling:

\begin{itemize}
    \item \textbf{Follower Units}: Cap 5 followers represent military forces
    \item \textbf{War Clocks}\index{War Clocks}: Supply Lines, Morale, Strategic Position
    \item \textbf{Command Actions}: Leaders can coordinate multiple units
\end{itemize}

\subsection*{Siege Warfare}\index{combat!siege}

Extended combat scenarios create persistent conditions:

\begin{itemize}
    \item \textbf{Resource Management}: Rapid Supply clock filling
    \item \textbf{Fatigue Accumulation}: Characters gain Fatigue each day
    \item \textbf{Morale Effects}: Hearts consequences affect entire forces
\end{itemize}

\subsection*{Magic Duels}\index{combat!magic duels}\index{magic combat}

High-stakes magical combat requires special considerations:

\begin{itemize}
    \item \textbf{Counterspelling}: Interrupting opponent's Casting Loop\index{Casting Loop}
    \item \textbf{Backlash Cascade}: Multiple sources of CP generation
    \item \textbf{Environmental Magic}: Terrain-altering spells
\end{itemize}

\section*{Narrative First: The World Reacts}\index{narrative first}

In Fate's Edge, the world is not a puzzle to be solved---it is a \textbf{living system} that responds to player choices. Let factions shift. Let consequences ripple. And above all---let the story unfold.

Because in the end, it is not the GM who writes the legend.

It is the players.

You simply hold the quill.

\end{chapter}
