\chapter{Clocks \& Campaigns}

\section*{Introduction to Clocks in Fate's Edge}

\gm{Clocks are one of the most important tools in Fate's Edge. They represent ongoing conditions, threats, or progress toward objectives. Think of them as visual progress bars that help everyone track tension and stakes.}

\subsection*{Types of Clocks}

\begin{itemize}[leftmargin=*]
\item \textbf{Travel Clocks}: Track progress through journey legs (4-10 segments)
\item \textbf{Tactical Clocks}: Represent ongoing combat conditions (Mob Overwhelm, Fatigue Spiral, etc.)
\item \textbf{Campaign Clocks}: Track long-term pressure (Mandate 0-6, Crisis 0-6)
\item \textbf{Scene Clocks}: Specific to current situations (Building Collapse, Pursuit, etc.)
\end{itemize}

\section*{Tutorial Session: Setting Up Clocks}

\player{GM} Today we'll run a session focused on teaching clock management. Our party consists of Elena the Scholar (Wits 4, Lore 3), Marcus the Warrior (Body 4, Melee 3), and Sariel the Scout (Wits 3, Stealth 3). They're investigating disturbances at an ancient Aeler ruin.

\subsection*{Step 1: Initial Scene Setup}

\player{GM} First, I'll establish the scene using the Aeler generator. Drawing cards until I have all suits:

\gm{Spade (8 - Gloam Cistern), Heart (6 - Key-Sister), Club (7 - Settling crack), Diamond (5 - Underway Pass)}

\gm{Highest rank is 8, so we have an 8-segment travel clock. But for this tutorial, let's focus on the immediate scene clocks.}

\subsection*{Step 2: Creating Scene Clocks}

\player{GM} As you enter the Gloam Cistern, I'm establishing three key clocks:

\begin{center}
\textbf{Environmental Collapse Clock} (8 segments)\\
\fbox{\clocksegment\clocksegment\clocksegment\clocksegment\clocksegment\clocksegment\clocksegment\clocksegment} 0/8\\
\textit{Represents the ancient structure's stability}
\end{center}

\begin{center}
\textbf{Warden Search Clock} (6 segments)\\
\fbox{\clocksegment\clocksegment\clocksegment\clocksegment\clocksegment\clocksegment} 0/6\\
\textit{How long before the Key-Sister finds you}
\end{center}

\begin{center}
\textbf{Forbidden Knowledge Clock} (4 segments)\\
\fbox{\clocksegment\clocksegment\clocksegment\clocksegment} 0/4\\
\textit{Progress in uncovering dangerous secrets}
\end{center}

\gm{I always announce clocks to players so they understand the stakes. These represent: the building might collapse, you're being hunted, and you're meddling with something dangerous.}

\section*{Scene 1: The Investigation Begins}

\player{Elena} I want to examine the ancient inscriptions on the walls to understand what happened here. Wits + Lore.

\gm{Risky position, DV 2. The inscriptions are faded and in an old dialect.}

\player{Elena} Rolling Wits 4 + Lore 3 = 7 dice. \roll{7}{9, 7, 6, 4, 3, 1, 1} \success{4} \cp{2}

\outcome{Success \& Cost}

\gm{You decipher enough to understand this was a prison for something the dwarves considered extremely dangerous. But \cp{1}: Your scrutiny of the walls causes some loose stonework to shift. \cp{2}: The sound attracts attention - you hear footsteps approaching.}

\gm{I'll tick the Environmental Collapse clock by 1 (loose stonework) and the Warden Search clock by 1 (attracted attention).}

\begin{center}
\textbf{Environmental Collapse Clock} (8 segments)\\
\fbox{\textcolor{clockcolor}{\clocksegment}\clocksegment\clocksegment\clocksegment\clocksegment\clocksegment\clocksegment\clocksegment} 1/8
\end{center}

\begin{center}
\textbf{Warden Search Clock} (6 segments)\\
\fbox{\textcolor{clockcolor}{\clocksegment}\clocksegment\clocksegment\clocksegment\clocksegment\clocksegment} 1/6
\end{center}

\player{Sariel} I'll scout ahead to see what's making those footsteps. Wits + Stealth.

\gm{Controlled position (you're choosing when to engage), DV 1.}

\player{Sariel} Rolling Wits 3 + Stealth 3 = 6 dice. \roll{6}{8, 6, 5, 4, 2, 1} \success{3} \cp{1}

\outcome{Success \& Cost}

\gm{You slip into a shadowy alcove and see the Key-Sister approaching with a lantern, muttering prayers under her breath. She's definitely looking for intruders. \cp{1}: As you watch, more stone dust sifts down from the ceiling near your position.}

\gm{Environmental Collapse clock ticks up by 1.}

\begin{center}
\textbf{Environmental Collapse Clock} (8 segments)\\
\fbox{\textcolor{clockcolor}{\clocksegment\clocksegment}\clocksegment\clocksegment\clocksegment\clocksegment\clocksegment\clocksegment} 2/8
\end{center}

\section*{Scene 2: Escalation}

\player{Marcus} I say we confront the Key-Sister directly. Maybe we can explain ourselves. Body + Command.

\gm{Risky position, DV 2. You're approaching someone who's clearly hostile to intruders.}

\player{Marcus} Rolling Body 4 + Command 1 = 5 dice. \roll{5}{7, 6, 4, 3, 1} \success{3} \cp{1}

\outcome{Success \& Cost}

\gm{You step out confidently and call out to the dwarf. She spins, drawing a ritual dagger, and begins chanting in a language that makes your teeth ache. The air grows thick and oppressive. \cp{1}: Your bold approach has definitely marked you as hostile rather than lost travelers.}

\gm{Warden Search clock jumps by 2 (now she knows exactly where you are and is actively hostile).}

\begin{center}
\textbf{Warden Search Clock} (6 segments)\\
\fbox{\textcolor{clockcolor}{\clocksegment\clocksegment\clocksegment}\clocksegment\clocksegment\clocksegment} 3/6
\end{center}

\player{Elena} Seeing her ritual, I'll try to counter it with my knowledge of ancient Aelerian practices. Wits + Arcana.

\gm{Desperate position, DV 3. You're trying to counter an active ritual you barely understand.}

\player{Elena} Rolling Wits 4 + Arcana 1 = 5 dice. \roll{5}{9, 6, 3, 2, 1} \success{2} \cp{1}

\outcome{Success \& Cost}

\gm{You manage to disrupt her chant by shouting the traditional Aelerian counter-phrase. She stumbles back, surprised, but her ritual dagger begins to glow ominously. \cp{1}: The magical backlash leaves you momentarily dizzy.}

\gm{But here's where Forbidden Knowledge comes in - you've engaged with the dangerous secrets, so I'll tick that clock by 2.}

\begin{center}
\textbf{Forbidden Knowledge Clock} (4 segments)\\
\fbox{\textcolor{clockcolor}{\clocksegment\clocksegment}\clocksegment\clocksegment} 2/4
\end{center}

\section*{Scene 3: Consequences and Resolution}

\player{GM} Let's check our clock status:

\begin{center}
\textbf{Environmental Collapse Clock} (8 segments)\\
\fbox{\textcolor{clockcolor}{\clocksegment\clocksegment}\clocksegment\clocksegment\clocksegment\clocksegment\clocksegment\clocksegment} 2/8
\end{center}

\begin{center}
\textbf{Warden Search Clock} (6 segments)\\
\fbox{\textcolor{clockcolor}{\clocksegment\clocksegment\clocksegment}\clocksegment\clocksegment\clocksegment} 3/6
\end{center}

\begin{center}
\textbf{Forbidden Knowledge Clock} (4 segments)\\
\fbox{\textcolor{clockcolor}{\clocksegment\clocksegment}\clocksegment\clocksegment} 2/4
\end{center}

\gm{None are filled yet, but they're all advancing. The tension is building.}

\player{Sariel} I think we should retreat and find another way in. I'll scout for a safer route while the others create a distraction. Wits + Survival.

\gm{Risky position, DV 2. You're trying to navigate unstable ancient stonework while being hunted.}

\player{Sariel} Rolling Wits 3 + Survival 3 = 6 dice. \roll{6}{10, 7, 5, 4, 2, 1} \success{4} \cp{1}

\outcome{Success \& Cost}

\gm{You spot a partially collapsed side passage that might lead around to the back of the cistern. But \cp{1}: As you point it out, more stones shift ominously above you.}

\gm{Environmental Collapse clock ticks up by 1.}

\begin{center}
\textbf{Environmental Collapse Clock} (8 segments)\\
\fbox{\textcolor{clockcolor}{\clocksegment\clocksegment\clocksegment}\clocksegment\clocksegment\clocksegment\clocksegment\clocksegment} 3/8
\end{center}

\player{Marcus} I'll create that distraction - charge the Key-Sister to give Sariel time. Body + Melee.

\gm{Desperate position, DV 3. You're charging a ritualist in close quarters in an unstable structure.}

\player{Marcus} Rolling Body 4 + Melee 3 = 7 dice. \roll{7}{8, 6, 5, 4, 3, 1, 1} \success{4} \cp{2}

\outcome{Success \& Cost}

\gm{You leap forward with a fierce yell, and the Key-Sister barely gets her glowing dagger up in time. Your blade clashes against hers with a shower of sparks. \cp{1}: The impact sends more debris raining down. \cp{2}: Your attack has fully committed you to this fight - there's no easy exit now.}

\gm{Environmental Collapse jumps by 2, and Warden Search by 1 (she's now fully engaged with you).}

\begin{center}
\textbf{Environmental Collapse Clock} (8 segments)\\
\fbox{\textcolor{clockcolor}{\clocksegment\clocksegment\clocksegment\clocksegment\clocksegment}\clocksegment\clocksegment\clocksegment} 5/8
\end{center}

\begin{center}
\textbf{Warden Search Clock} (6 segments)\\
\fbox{\textcolor{clockcolor}{\clocksegment\clocksegment\clocksegment\clocksegment}\clocksegment\clocksegment} 4/6
\end{center}

\section*{Scene 4: Clock Resolution}

\player{GM} We're getting close to clock resolution. Let's see what happens next:

\player{Elena} The structure is coming down! I'll try to use my knowledge to stabilize the area with geomantic principles. Wits + Arcana.

\gm{Pressured position, DV 3. You're trying complex magic while dodging falling masonry.}

\player{Elena} Rolling Wits 4 + Arcana 1 = 5 dice. \roll{5}{7, 5, 4, 3, 1} \success{3} \cp{1}

\outcome{Success \& Cost}

\gm{You manage to channel some stabilizing energy into the walls, and several dangerous-looking cracks stop spreading. But \cp{1}: The magical effort exhausts you, and you feel drained.}

\gm{I'll reduce the Environmental Collapse clock by 2 (your stabilization) but tick the Forbidden Knowledge clock by 1 (using advanced geomancy).}

\begin{center}
\textbf{Environmental Collapse Clock} (8 segments)\\
\fbox{\textcolor{clockcolor}{\clocksegment\clocksegment\clocksegment}\clocksegment\clocksegment\clocksegment\clocksegment\clocksegment} 3/8
\end{center}

\begin{center}
\textbf{Forbidden Knowledge Clock} (4 segments)\\
\fbox{\textcolor{clockcolor}{\clocksegment\clocksegment\clocksegment}\clocksegment} 3/4
\end{center}

\player{Sariel} While they're distracted, I'll slip through that side passage I found. Wits + Stealth.

\gm{Controlled position (the others are creating a perfect distraction), DV 1.}

\player{Sariel} Rolling Wits 3 + Stealth 3 = 6 dice. \roll{6}{9, 7, 6, 4, 2, 1} \success{4} \cp{1}

\outcome{Success \& Cost}

\gm{You slip away unnoticed and find the side passage leads to a hidden chamber filled with ancient scrolls and artifacts. But \cp{1}: As you enter, you accidentally trigger an old alarm system - a low gong sounds in the distance.}

\gm{The Warden Search clock jumps by 2! She now knows exactly where you've gone.}

\begin{center}
\textbf{Warden Search Clock} (6 segments)\\
\fbox{\textcolor{clockcolor}{\clocksegment\clocksegment\clocksegment\clocksegment\clocksegment\clocksegment}} 6/6 \textbf{FILLED!}
\end{center}

\section*{Clock Resolution: Warden Search Filled}

\gm{The Warden Search clock is filled! This triggers a major consequence. The Key-Sister shouts in dwarven and you hear her running toward the hidden chamber. She's bringing reinforcements and her ritual dagger is now blazing with eldritch fire.}

\player{Marcus} I'm right behind her - I'll pursue through the passage to back up Sariel. Body + Athletics.

\gm{Risky position, DV 2. You're chasing someone through unknown passages while the structure groans around you.}

\player{Marcus} Rolling Body 4 + Athletics 2 = 6 dice. \roll{6}{8, 6, 5, 4, 2, 1} \success{3} \cp{1}

\outcome{Success \& Cost}

\gm{You reach the hidden chamber just as the Key-Sister arrives, but she's not alone - two stone guardians have awakened and are blocking the entrance. \cp{1}: The chase has winded you slightly.}

\gm{Environmental Collapse clock ticks up by 1 due to the magical awakening.}

\begin{center}
\textbf{Environmental Collapse Clock} (8 segments)\\
\fbox{\textcolor{clockcolor}{\clocksegment\clocksegment\clocksegment\clocksegment}\clocksegment\clocksegment\clocksegment\clocksegment} 4/8
\end{center}

\section*{Final Scene: Multiple Clock Management}

\player{GM} Now we have multiple active threats. The Warden Search has reset but the situation is worse. The Forbidden Knowledge clock continues to tick as you're now in the heart of the ancient archive. The Environmental Collapse is advancing due to the awakened guardians.

\player{Elena} I need to understand what these guardians are protecting. I'll examine the central artifact in the chamber. Wits + Lore.

\gm{Pressured position, DV 3. You're researching while being hunted by stone constructs.}

\player{Elena} Rolling Wits 4 + Lore 3 = 7 dice. \roll{7}{10, 8, 6, 4, 3, 1, 1} \success{4} \cp{2}

\outcome{Success \& Cost}

\gm{You recognize the artifact as a prison seal - the entire cistern was built to contain something that's been trying to escape for centuries. The guardians exist to keep it bound. \cp{1}: Your research reveals that breaking the seal would be catastrophic. \cp{2}: But the knowledge also shows you how to strengthen it - at great personal risk.}

\gm{Forbidden Knowledge clock fills completely!}

\begin{center}
\textbf{Forbidden Knowledge Clock} (4 segments)\\
\fbox{\textcolor{clockcolor}{\clocksegment\clocksegment\clocksegment\clocksegment}} 4/4 \textbf{FILLED!}
\end{center}

\section*{Clock Resolution: Forbidden Knowledge Filled}

\gm{The Forbidden Knowledge clock fills! You now understand too much - the ancient entity is aware of your presence and is trying to influence you. You must make a choice: flee and leave the seal weakened, or risk everything to strengthen it.}

\player{Sariel} I say we strengthen it. What's the worst that could happen? Wits + Sway.

\gm{Desperate position, DV 3. You're trying to convince allies to take a massive risk while stone guardians approach.}

\player{Sariel} Rolling Wits 3 + Sway 2 = 5 dice. \roll{5}{7, 5, 4, 2, 1} \success{2} \cp{1}

\outcome{Success \& Cost}

\gm{You manage to convince Elena that strengthening the seal is the right choice, but Marcus remains unconvinced. \cp{1}: The debate has cost you precious seconds - the guardians are almost upon you.}

\section*{Tutorial Conclusion: Clock Management Lessons}

\player{GM} Let's pause here to discuss what we've learned about clock management:

\subsection*{Clock Creation Guidelines}

\begin{enumerate}[leftmargin=*]
\item \textbf{Announce Clearly}: Always tell players what each clock represents and how it advances.
\item \textbf{Logical Triggers}: Clock advancement should follow from player actions and fictional events.
\item \textbf{Visible Progression}: Use visual tracking so everyone can see tension building.
\item \textbf{Meaningful Consequences}: When clocks fill, the consequences should change the story significantly.
\end{enumerate}

\subsection*{Clock Advancement Rules}

\begin{itemize}[leftmargin=*]
\item \textbf{1 CP}: Minor advancement (1 segment)
\item \textbf{2-3 CP}: Moderate advancement (2 segments)  
\item \textbf{4+ CP}: Major advancement (3+ segments) or fill smaller clocks
\item \textbf{Multiple Clocks}: Distribute CP across relevant clocks rather than overfilling one
\end{itemize}

\subsection*{Clock Resolution Strategies}

\begin{enumerate}[leftmargin=*]
\item \textbf{Fill for Consequences}: When a clock fills, introduce a significant story turn.
\item \textbf{Reset with Changes}: After resolution, reset clocks but change the situation.
\item \textbf{Cascade Effects}: Filled clocks can trigger advancement in other clocks.
\item \textbf{Player Agency}: Give players meaningful choices in how they deal with filling clocks.
\end{enumerate}

\subsection*{Common Clock Types and Uses}

\begin{description}[leftmargin=*]
\item[Environmental Clocks] (4-8 segments): Building collapse, weather, fire, flood
\item[Social Clocks] (4-6 segments): Escalating tensions, public opinion, scandal  
\item[Pursuit Clocks] (6 segments): Chase scenes, investigations, hunts
\item[Preparation Clocks] (4-6 segments): Ritual casting, crafting, planning
\item[Corruption Clocks] (4-6 segments): Moral decay, magical taint, addiction
\end{description}

\subsection*{GM Tips for Clock Management}

\begin{itemize}[leftmargin=*]
\item Start scenes with 1-2 clocks to establish tension
\item Advance clocks through CP spends rather than arbitrary GM fiat
\item Let players see the connection between their actions and clock advancement
\item Use clocks to telegraph rising stakes and consequences
\item Don't be afraid to let clocks fill - that's when interesting things happen
\item Reset clocks when situations fundamentally change, rather than just emptying them
\end{itemize}

\gm{In our session, we saw how clocks create escalating tension: Environmental Collapse made the location dangerous, Warden Search brought active opposition, and Forbidden Knowledge raised the stakes of what the PCs were dealing with. When Forbidden Knowledge filled, it fundamentally changed the situation from a simple exploration to a moral dilemma with world-shaking implications.}

\gm{The key is that clocks don't just track time - they track the accumulation of tension, stakes, and consequences. Every tick should feel earned and meaningful.}

\section{Campaign Design and Long-Term Play}

In \textbf{Fate's Edge}, campaigns are not just a string of adventures—they are \textbf{living narratives} shaped by player choices, faction dynamics, and the slow accumulation of influence. As the GM, you are the architect of long-term tension, guiding the story from its first spark to its final reckoning. This chapter introduces the tools that help you build and sustain that tension: the \textbf{Campaign Clocks}, the \textbf{Crown Spread}, and how to scale play for mixed-tier parties.

\section*{Campaign Clocks: Tracking Influence and Pressure}

The \textbf{Campaign Clocks} are two dials that track the ebb and flow of player power and opposition over the course of a campaign. They are not mechanical scoreboards—they are \textbf{narrative thermometers}, showing how the world reacts to the PCs' actions.

\subsection*{Mandate (0–6)}

\textbf{Mandate} represents the table's \textbf{public legitimacy and buy-in}. It tracks how much the world accepts the PCs' authority, influence, or mission.

\begin{itemize}
    \item High Mandate: The PCs are recognized, respected, or feared. Doors open. Allies rally.
    \item Low Mandate: The PCs are ignored, questioned, or hunted. Every step is harder.
\end{itemize}

\subsection*{Crisis (0–6)}

\textbf{Crisis} tracks the \textbf{opposition engine}—rivals, pressure rails, attrition. It shows how much the world pushes back.

\begin{itemize}
    \item High Crisis: Enemies rise. Clocks tick. The world tightens around the PCs.
    \item Low Crisis: The PCs have breathing room. Opportunities bloom.
\end{itemize}

\subsection*{Advancing the Clocks}

At the end of each major scene, you may advance one or both clocks based on:

\begin{itemize}
    \item \textbf{Clean loss}: Rival codifies or escapes with leverage.
    \item \textbf{Public cost paid}: Feast, free day, penance.
    \item \textbf{Asset neglect}: Flagged Major degrades.
    \item \textbf{Evidence shifts}: Immaculate → Scorched.
\end{itemize}

\section*{Calling or Forcing the Crown}

The campaign reaches its crescendo when one of two thresholds is met:

\begin{itemize}
    \item \textbf{Player-Called Finale}: When \textbf{Mandate ≥ 6} and \textbf{Crisis ≤ 3}, the table may schedule the Finale at the next opportune site.
    \item \textbf{Forced Finale}: When \textbf{Crisis ≥ 6} (regardless of Mandate), the Rival forces a decision next arc.
\end{itemize}

A \textbf{Balanced Finale} occurs when both dials sit in the mid-band (4–5). Start both rails at +1; CP budget as normal.

\section*{The Crown Spread: Seeding the Campaign}

At \textbf{Session 0}, draw the \textbf{Crown Spread}—a five-card ritual that seeds the campaign's themes, rivals, and finale conditions.

\subsection*{Drawing the Spread}

Draw one card each of:

\begin{itemize}
    \item \textbf{Spade}: Crown Site (where the monument is decided).
    \item \textbf{Heart}: Crown Rival (who can still stop it).
    \item \textbf{Club}: Crown Pressure (the rail that will bite if the table turtles).
    \item \textbf{Diamond}: Crown Leverage (the payoff that can be codified).
    \item \textbf{Wild}: Reveal last—Face = hidden patron steps out; Ace = the site becomes a 10-clock.
\end{itemize}

\subsection*{Interpreting the Spread}

\begin{itemize}
    \item \textbf{Spade (Site)}: A fortress? A shrine? A battlefield? The setting of the finale.
    \item \textbf{Heart (Rival)}: A noble? A cult? A spirit? Generate full motives for them (♡, ♣, ♢, ♠).
    \item \textbf{Club (Pressure)}: Crowd, Hazard, Escape Net—pick one and name it now.
    \item \textbf{Diamond (Leverage)}: Seasonal endorsement, city license, doctrinal clause—never rolls, only changes position.
    \item \textbf{Wild (Hidden Force)}: A wildcard element—ally, enemy, or omen.
\end{itemize}

\textbf{Example}: Spade = High-Mist Pass (Aeler); Heart = Margrave of Acasia (Face); Club = Curfew; Diamond = Seasonal Endorsement; Wild = Hidden Patron (Face).

\section*{The Finale Procedure}

When the Crown is called, run the three-beat finale:

\begin{enumerate}
    \item \textbf{Reckoning}: Defend or sanctify the record. Draw the Rival's motives. Place the Pressure rail.
    \item \textbf{Crossing}: Stage the kinetic rail (Escape/Hunt/Hazard) that threatens to end the scene.
    \item \textbf{Coronation}: Use the Diamond Leverage to sign, seal, or oath the monument.
\end{enumerate}

\subsection*{Twist Collision (Finale Clause)}

Exactly once, when the Rival's ♠ Twist contradicts their ♣ Belief, the table chooses:

\begin{itemize}
    \item GM +1 CP, or
    \item Players reduce two ticks total across the rails.
\end{itemize}

\section*{Legacy Conversion: Epilogue}

After the Finale, each PC draws 2 cards and answers epilogue prompts by suit. Then convert:

\begin{itemize}
    \item \textbf{Major Asset → Institution} (12 XP): Permanent setting change.
    \item \textbf{Seasonal Endorsement → Doctrine Rider} (4 XP): Fold into the base Accord.
    \item \textbf{Follower (Cap 3+) → Stationed NPC} (0 XP): Promote to Custodian/Deputy Chair.
    \item \textbf{Rival → Fixture}: If they survive, they auto-tick the relevant rail whenever your style shows.
\end{itemize}

\section*{Scaling for Mixed-Tier Parties}

As characters grow, their investments may diverge. One may be a blade-master, another a network architect. Keep scenes tense with these tools:

\begin{itemize}
    \item \textbf{Structural Advantages}: Active buff, venue pennant, Follower Initiative unused, etc.
    \item \textbf{Over-Stack Rule}: If the crew enters with 2+ advantages, start rails at +1 OR GM banks +1 CP.
    \item \textbf{CP Floor}: Set minimum CP based on ΔTier = Obstacle − Highest PC Tier.
\end{itemize}

\textbf{GM Tip}: Let lanes matter. Enforce one assistant max, +3 dice cap. Target consequences fairly—endangering a follower should escalate stakes, not punish creativity.

\section*{Campaign Combat Integration}

Extended conflicts and war-level events require special handling to maintain narrative tension while scaling the mechanical scope.

\subsection*{War Clocks}

Large-scale conflicts are tracked through persistent war-level clocks:

\begin{itemize}
    \item \textbf{Supply Lines} (8): Logistics and reinforcement flow
    \item \textbf{Morale} (6): Troop effectiveness and desertion risk
    \item \textbf{Political Support} (6): Civilian and noble backing
    \item \textbf{Strategic Position} (8): Control of key locations and routes
\end{itemize}

\subsection*{Faction Combat}

When player factions engage in large-scale conflict:

\begin{itemize}
    \item \textbf{Follower Armies}: Cap 5 followers can represent military units
    \item \textbf{Asset Leverage}: Off-screen assets provide strategic advantages
    \item \textbf{Campaign Clock Impact}: Major battles shift Mandate and Crisis dials
\end{itemize}

\subsection*{Siege Conditions}

Extended combat scenarios create persistent conditions:

\begin{itemize}
    \item \textbf{Resource Depletion}: Supply clock fills rapidly
    \item \textbf{Fatigue Accumulation}: Characters gain Fatigue each day
    \item \textbf{Environmental Hazards}: Weather, disease, or magical effects
\end{itemize}

\section*{Narrative First: Let the World React}

Campaign design in Fate's Edge is not about railroading—it's about \textbf{responding to player choices} with escalating consequences. Let the world shift. Let factions rise. Let the dice sing.

And when the Crown is crowned—let the echo be heard across the Amaranthine.

\section{Between Sessions: GM Responsibilities}

Between game sessions, the Game Master has crucial preparation and administrative tasks to ensure continuity, challenge, and narrative coherence. This downtime is essential for maintaining the campaign's momentum and preparing meaningful challenges for the players.
\subsection{Mandatory Preparation}

The GM must complete the following essential tasks before the next session:

\textbf{Campaign Clock Updates}: Advance Mandate and Crisis clocks based on session outcomes and player actions. Track significant developments that affect the overall campaign trajectory.

\textbf{Complication Debt Management}: Calculate starting CP for the session based on:
\begin{itemize}
    \item Banked CP from previous sessions (maximum 2 CP carryover)
    \item Active character complications (+1 CP per character with complications)
    \item Asset-generated complications
    \item Epic Hooks spending
\end{itemize}

\textbf{Thread Management}: Review active complication threads and determine which are escalating, resolving, or requiring attention. Ensure no more than (Tier + 1) active threads per scene.

\textbf{Resource Tracking}: Update NPC statuses, faction relationships, and world conditions based on player actions.

\subsection{Session Planning}

Prepare the following elements for the upcoming session:

\textbf{Scene Preparation}: Design scenes with appropriate CP spending budgets:
\begin{itemize}
    \item Standard Scenes: Maximum 12 CP spending
    \item Climactic Scenes: Maximum 16 CP spending
    \item Session Budget: Maximum 20 CP total
\end{itemize}

\textbf{Complication Hooks}: Develop 3-5 potential complications that connect to player backgrounds, current threads, and campaign themes.

\textbf{Tactical Considerations}: Prepare for expected combat encounters, social challenges, and exploration scenes with appropriate difficulties and positioning.

\textbf{Deck Preparation}: Ensure Consequences Deck is ready with appropriate cards for expected complication types.

\subsection{Narrative Development}

Advance the overarching story elements:

\textbf{Character Integration}: Review each player's complications, bonds, and flags to create personalized challenges and opportunities.

\textbf{World Response}: Determine how the game world reacts to player actions. Allies may offer new opportunities; enemies may escalate their responses.

\textbf{Plot Thread Advancement}: Move major campaign arcs forward, introducing new elements or escalating existing tensions.

\textbf{Foreshadowing**: Plant seeds for future complications and story developments that will pay off in later sessions.

\subsection{Administrative Tasks}

Complete necessary record-keeping:

\textbf{XP Award Calculation**: Tally session awards for each player:
\begin{itemize}
    \item Table Attendance: +2 XP
    \item Major Objectives: +2-4 XP
    \item Discoveries: +1-2 XP
    \item Hard Choices: +1-2 XP
    \item Complication Spotlight: +1-3 XP
    \item Bond/Flag Play: +1-2 XP
    \item GM Curveballs: +0-3 XP
    \item Complication Dividends: Face cards (+1 XP), Aces (+2 XP)
\end{itemize}

\textbf{Asset and Follower Updates**: Track any changes to player assets, followers, and off-screen resources based on session events.

\textbf{Session Zero Integration**: If applicable, incorporate elements from the Crown Spread and campaign seeds into upcoming scenes.

\subsection{Strategic Considerations}

Plan with the following factors in mind:

\textbf{Pacing Management**: Balance high-tension scenes with recovery opportunities. Avoid spending maximum CP every scene to maintain dramatic impact.

\textbf{Player Agency**: Ensure complications create interesting choices rather than simple obstacles. Every CP spend should offer meaningful player responses.

\textbf{Group Dynamics**: Consider how complications affect different players and maintain equitable challenge distribution.

\textbf{Teaching Moments**: For new groups or experimental rules (like high complication debt), prepare appropriate scaffolding and support.

Remember: Your preparation directly impacts player enjoyment and engagement. Thorough session planning allows you to focus on collaborative storytelling during the game rather than scrambling for content. The investment in downtime preparation pays dividends in session quality and narrative coherence.
\end{chapter}
