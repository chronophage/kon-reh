\chapter{Campaign Design and Long-Term Play}

In \textbf{Fate’s Edge}, campaigns are not just a string of adventures—they are \textbf{living narratives} shaped by player choices, faction dynamics, and the slow accumulation of influence. As the GM, you are the architect of long-term tension, guiding the story from its first spark to its final reckoning. This chapter introduces the tools that help you build and sustain that tension: the \textbf{Campaign Clocks}, the \textbf{Crown Spread}, and how to scale play for mixed-tier parties.

\section*{Campaign Clocks: Tracking Influence and Pressure}

The \textbf{Campaign Clocks} are two dials that track the ebb and flow of player power and opposition over the course of a campaign. They are not mechanical scoreboards—they are \textbf{narrative thermometers}, showing how the world reacts to the PCs’ actions.

\subsection*{Mandate (0–6)}

\textbf{Mandate} represents the table’s \textbf{public legitimacy and buy-in}. It tracks how much the world accepts the PCs’ authority, influence, or mission.

\begin{itemize}
    \item High Mandate: The PCs are recognized, respected, or feared. Doors open. Allies rally.
    \item Low Mandate: The PCs are ignored, questioned, or hunted. Every step is harder.
\end{itemize}

\subsection*{Crisis (0–6)}

\textbf{Crisis} tracks the \textbf{opposition engine}—rivals, pressure rails, attrition. It shows how much the world pushes back.

\begin{itemize}
    \item High Crisis: Enemies rise. Clocks tick. The world tightens around the PCs.
    \item Low Crisis: The PCs have breathing room. Opportunities bloom.
\end{itemize}

\subsection*{Advancing the Clocks}

At the end of each major scene, you may advance one or both clocks based on:

\begin{itemize}
    \item \textbf{Clean loss}: Rival codifies or escapes with leverage.
    \item \textbf{Public cost paid}: Feast, free day, penance.
    \item \textbf{Asset neglect}: Flagged Major degrades.
    \item \textbf{Evidence shifts}: Immaculate → Scorched.
\end{itemize}

\section*{Calling or Forcing the Crown}

The campaign reaches its crescendo when one of two thresholds is met:

\begin{itemize}
    \item \textbf{Player-Called Finale}: When \textbf{Mandate ≥ 6} and \textbf{Crisis ≤ 3}, the table may schedule the Finale at the next opportune site.
    \item \textbf{Forced Finale}: When \textbf{Crisis ≥ 6} (regardless of Mandate), the Rival forces a decision next arc.
\end{itemize}

A \textbf{Balanced Finale} occurs when both dials sit in the mid-band (4–5). Start both rails at +1; CP budget as normal.

\section*{The Crown Spread: Seeding the Campaign}

At \textbf{Session 0}, draw the \textbf{Crown Spread}—a five-card ritual that seeds the campaign’s themes, rivals, and finale conditions.

\subsection*{Drawing the Spread}

Draw one card each of:

\begin{itemize}
    \item \textbf{Spade}: Crown Site (where the monument is decided).
    \item \textbf{Heart}: Crown Rival (who can still stop it).
    \item \textbf{Club}: Crown Pressure (the rail that will bite if the table turtles).
    \item \textbf{Diamond}: Crown Leverage (the payoff that can be codified).
    \item \textbf{Wild}: Reveal last—Face = hidden patron steps out; Ace = the site becomes a 10-clock.
\end{itemize}

\subsection*{Interpreting the Spread}

\begin{itemize}
    \item \textbf{Spade (Site)}: A fortress? A shrine? A battlefield? The setting of the finale.
    \item \textbf{Heart (Rival)}: A noble? A cult? A spirit? Generate full motives for them (♡, ♣, ♢, ♠).
    \item \textbf{Club (Pressure)}: Crowd, Hazard, Escape Net—pick one and name it now.
    \item \textbf{Diamond (Leverage)}: Seasonal endorsement, city license, doctrinal clause—never rolls, only changes position.
    \item \textbf{Wild (Hidden Force)}: A wildcard element—ally, enemy, or omen.
\end{itemize}

\textbf{Example}: Spade = High-Mist Pass (Aeler); Heart = Margrave of Acasia (Face); Club = Curfew; Diamond = Seasonal Endorsement; Wild = Hidden Patron (Face).

\section*{The Finale Procedure}

When the Crown is called, run the three-beat finale:

\begin{enumerate}
    \item \textbf{Reckoning}: Defend or sanctify the record. Draw the Rival’s motives. Place the Pressure rail.
    \item \textbf{Crossing}: Stage the kinetic rail (Escape/Hunt/Hazard) that threatens to end the scene.
    \item \textbf{Coronation}: Use the Diamond Leverage to sign, seal, or oath the monument.
\end{enumerate}

\subsection*{Twist Collision (Finale Clause)}

Exactly once, when the Rival’s ♠ Twist contradicts their ♣ Belief, the table chooses:

\begin{itemize}
    \item GM +1 CP, or
    \item Players reduce two ticks total across the rails.
\end{itemize}

\section*{Legacy Conversion: Epilogue}

After the Finale, each PC draws 2 cards and answers epilogue prompts by suit. Then convert:

\begin{itemize}
    \item \textbf{Major Asset → Institution} (12 XP): Permanent setting change.
    \item \textbf{Seasonal Endorsement → Doctrine Rider} (4 XP): Fold into the base Accord.
    \item \textbf{Follower (Cap 3+) → Stationed NPC} (0 XP): Promote to Custodian/Deputy Chair.
    \item \textbf{Rival → Fixture}: If they survive, they auto-tick the relevant rail whenever your style shows.
\end{itemize}

\section*{Scaling for Mixed-Tier Parties}

As characters grow, their investments may diverge. One may be a blade-master, another a network architect. Keep scenes tense with these tools:

\begin{itemize}
    \item \textbf{Structural Advantages}: Active buff, venue pennant, Follower Initiative unused, etc.
    \item \textbf{Over-Stack Rule}: If the crew enters with 2+ advantages, start rails at +1 OR GM banks +1 CP.
    \item \textbf{CP Floor}: Set minimum CP based on ΔTier = Obstacle − Highest PC Tier.
\end{itemize}

\textbf{GM Tip}: Let lanes matter. Enforce one assistant max, +3 dice cap. Target consequences fairly—endangering a follower should escalate stakes, not punish creativity.

\section*{Narrative First: Let the World React}

Campaign design in Fate’s Edge is not about railroading—it’s about \textbf{responding to player choices} with escalating consequences. Let the world shift. Let factions rise. Let the dice sing.

And when the Crown is crowned—let the echo be heard across the Amaranthine.

\end{chapter}
