\chapter{Campaigns, Clocks, and Consequences}\index{campaigns}\index{clocks}\index{consequences}

In \textbf{Fate's Edge}, campaigns are not just a string of adventures---they are \textbf{living narratives} shaped by player choices, faction dynamics, and the slow accumulation of influence. As the GM, you are the architect of long-term tension, guiding the story from its first spark to its final reckoning. This chapter introduces the tools that help you build and sustain that tension: the \textbf{Campaign Clocks}, the \textbf{Crown Spread}, and how to manage long-term play.

\section*{Campaign Clocks: Tracking Influence and Pressure}\index{Campaign Clocks}

The \textbf{Campaign Clocks} are two dials that track the ebb and flow of player power and opposition over the course of a campaign. They are not mechanical scoreboards---they are \textbf{narrative thermometers}, showing how the world reacts to the PCs' actions.

\subsection*{Mandate (0--6)}\index{Campaign Clocks!Mandate}

\textbf{Mandate} represents the table's \textbf{public legitimacy and buy-in}. It tracks how much the world accepts the PCs' authority, influence, or mission.

\begin{itemize}
    \item High Mandate: The PCs are recognized, respected, or feared. Doors open. Allies rally.
    \item Low Mandate: The PCs are ignored, questioned, or hunted. Every step is harder.
\end{itemize}

\subsection*{Crisis (0--6)}\index{Campaign Clocks!Crisis}

\textbf{Crisis} tracks the \textbf{opposition engine}---rivals, pressure rails, attrition. It shows how much the world pushes back.

\begin{itemize}
    \item High Crisis: Enemies rise. Clocks tick. The world tightens around the PCs.
    \item Low Crisis: The PCs have breathing room. Opportunities bloom.
\end{itemize}

\subsection*{Advancing the Clocks}\index{Campaign Clocks!advancement}

At the end of each major scene, you may advance one or both clocks based on:

\begin{itemize}
    \item \textbf{Clean loss}: Rival codifies or escapes with leverage.
    \item \textbf{Public cost paid}: Feast, free day, penance.
    \item \textbf{Asset neglect}: Flagged Major degrades.
    \item \textbf{Evidence shifts}: Immaculate → Scorched.
\end{itemize}

\section*{Calling or Forcing the Crown}\index{Crown}

The campaign reaches its crescendo when one of two thresholds is met:

\begin{itemize}
    \item \textbf{Player-Called Finale}: When \textbf{Mandate ≥ 6} and \textbf{Crisis ≤ 3}, the table may schedule the Finale at the next opportune site.
    \item \textbf{Forced Finale}: When \textbf{Crisis ≥ 6} (regardless of Mandate), the Rival forces a decision next arc.
\end{itemize}

A \textbf{Balanced Finale} occurs when both dials sit in the mid-band (4--5). Start both rails at +1; SB budget as normal.

\section*{The Crown Spread: Seeding the Campaign}\index{Crown Spread}

At \textbf{Session 0}, draw the \textbf{Crown Spread}---a five-card ritual that seeds the campaign's themes, rivals, and finale conditions.

\subsection*{Drawing the Spread}\index{Crown Spread!drawing}

Draw one card each of:

\begin{itemize}
    \item \textbf{Spade}: Crown Site (where the monument is decided).
    \item \textbf{Heart}: Crown Rival (who can still stop it).
    \item \textbf{Club}: Crown Pressure (the rail that will bite if the table turtles).
    \item \textbf{Diamond}: Crown Leverage (the payoff that can be codified).
    \item \textbf{Wild}: Reveal last---Face = hidden patron steps out; Ace = the site becomes a 10-clock.
\end{itemize}

\subsection*{Interpreting the Spread}\index{Crown Spread!interpreting}

\begin{itemize}
    \item \textbf{Spade (Site)}: A fortress? A shrine? A battlefield? The setting of the finale.
    \item \textbf{Heart (Rival)}: A noble? A cult? A spirit? Generate full motives for them (♡, ♣, ♢, ♠).
    \item \textbf{Club (Pressure)}: Crowd, Hazard, Escape Net---pick one and name it now.
    \item \textbf{Diamond (Leverage)}: Seasonal endorsement, city license, doctrinal clause---never rolls, only changes position.
    \item \textbf{Wild (Hidden Force)}: A wildcard element---ally, enemy, or omen.
\end{itemize}

\textbf{Example}: Spade = High-Mist Pass (Aeler); Heart = Margrave of Acasia (Face); Club = Curfew; Diamond = Seasonal Endorsement; Wild = Hidden Patron (Face).

\section*{The Finale Procedure}\index{Finale}

When the Crown is called, run the three-beat finale:

\begin{enumerate}
    \item \textbf{Reckoning}: Defend or sanctify the record. Draw the Rival's motives. Place the Pressure rail.
    \item \textbf{Crossing}: Stage the kinetic rail (Escape/Hunt/Hazard) that threatens to end the scene.
    \item \textbf{Coronation}: Use the Diamond Leverage to sign, seal, or oath the monument.
\end{enumerate}

\subsection*{Twist Collision (Finale Clause)}\index{Finale!Twist Collision}

Exactly once, when the Rival's ♠ Twist contradicts their ♣ Belief, the table chooses:

\begin{itemize}
    \item GM +1 SB, or
    \item Players reduce two ticks total across the rails.
\end{itemize}

\section*{Legacy Conversion: Epilogue}\index{Legacy Conversion}\index{epilogue}

After the Finale, each PC draws 2 cards and answers epilogue prompts by suit. Then convert:

\begin{itemize}
    \item \textbf{Major Asset → Institution} (12 XP): Permanent setting change.
    \item \textbf{Seasonal Endorsement → Doctrine Rider} (4 XP): Fold into the base Accord.
    \item \textbf{Follower (Cap 3+) → Stationed NPC} (0 XP): Promote to Custodian/Deputy Chair.
    \item \textbf{Rival → Fixture}: If they survive, they auto-tick the relevant rail whenever your style shows.
\end{itemize}

\section*{Introduction to Clocks in Fate's Edge}\index{clocks!introduction}

Clocks are one of the most important tools in Fate's Edge. They represent ongoing conditions, threats, or progress toward objectives. Think of them as visual progress bars that help everyone track tension and stakes.

\subsection*{Types of Clocks}\index{clocks!types}

\begin{itemize}
\item \textbf{Travel Clocks}\index{clocks!Travel}: Track progress through journey legs (4-10 segments)
\item \textbf{Tactical Clocks}\index{Tactical Clocks}: Represent ongoing combat conditions (Mob Overwhelm, Fatigue Spiral, etc.)
\item \textbf{Campaign Clocks}\index{Campaign Clocks}: Track long-term pressure (Mandate 0-6, Crisis 0-6)
\item \textbf{Scene Clocks}\index{clocks!Scene}: Specific to current situations (Building Collapse, Pursuit, etc.)
\item \textbf{War Clocks}\index{War Clocks}: Large-scale conflict tracking (Supply Lines, Morale, etc.)
\end{itemize}

\subsection*{Common Clock Types and Uses}\index{clocks!common types}

\begin{description}
\item[Environmental Clocks]\index{clocks!Environmental} (4-8 segments): Building collapse, weather, fire, flood
\item[Social Clocks]\index{clocks!Social} (4-6 segments): Escalating tensions, public opinion, scandal  
\item[Pursuit Clocks]\index{clocks!Pursuit} (6 segments): Chase scenes, investigations, hunts
\item[Preparation Clocks]\index{clocks!Preparation} (4-6 segments): Ritual casting, crafting, planning
\item[Corruption Clocks]\index{clocks!Corruption} (4-6 segments): Moral decay, magical taint, addiction
\end{description}

\subsection*{Clock Creation Guidelines}\index{clocks!creation}

\begin{enumerate}
\item \textbf{Announce Clearly}: Always tell players what each clock represents and how it advances.
\item \textbf{Logical Triggers}: Clock advancement should follow from player actions and fictional events.
\item \textbf{Visible Progression}: Use visual tracking so everyone can see tension building.
\item \textbf{Meaningful Consequences}: When clocks fill, the consequences should change the story significantly.
\end{enumerate}

\subsection*{Clock Advancement Rules}\index{clocks!advancement}

\begin{itemize}
\item \textbf{1 SB}: Minor advancement (1 segment)
\item \textbf{2-3 SB}: Moderate advancement (2 segments)  
\item \textbf{4+ SB}: Major advancement (3+ segments) or fill smaller clocks
\item \textbf{Multiple Clocks}: Distribute SB across relevant clocks rather than overfilling one
\end{itemize}

\subsection*{Clock Resolution Strategies}\index{clocks!resolution strategies}

\begin{enumerate}
\item \textbf{Fill for Consequences}: When a clock fills, introduce a significant story turn.
\item \textbf{Reset with Changes}: After resolution, reset clocks but change the situation.
\item \textbf{Cascade Effects}: Filled clocks can trigger advancement in other clocks.
\item \textbf{Player Agency}: Give players meaningful choices in how they deal with filling clocks.
\end{enumerate}

\section*{Scaling for Mixed-Tier Parties}\index{scaling}

As characters grow, their investments may diverge. One may be a blade-master, another a network architect. Keep scenes tense with these tools:

\begin{itemize}
    \item \textbf{Structural Advantages}\index{structural advantages}: Active buff, venue pennant, Follower Initiative unused, etc.
    \item \textbf{Over-Stack Rule}\index{Over-Stack}: If the party enters with 2+ advantages, start rails at +1 OR GM banks +1 SB.
    \item \textbf{SB Floor}: Set minimum SB based on ΔTier = Obstacle − Highest PC Tier.
\end{itemize}

\textbf{GM Tip}: Let lanes matter. Enforce one assistant max, +3 dice cap. Target consequences fairly---endangering a follower should escalate stakes, not punish creativity.

\section*{Campaign Combat Integration}\index{campaign combat}

Extended conflicts and war-level events require special handling to maintain narrative tension while scaling the mechanical scope.

\subsection*{War Clocks}\index{War Clocks}

Large-scale conflicts are tracked through persistent war-level clocks:

\begin{itemize}
    \item \textbf{Supply Lines} (8): Logistics and reinforcement flow
    \item \textbf{Morale} (6): Troop effectiveness and desertion risk
    \item \textbf{Political Support} (6): Civilian and noble backing
    \item \textbf{Strategic Position} (8): Control of key locations and routes
\end{itemize}

\subsection*{Faction Combat}\index{faction combat}

When player factions engage in large-scale conflict:

\begin{itemize}
    \item \textbf{Follower Armies}: Cap 5 followers can represent military units
    \item \textbf{Asset Leverage}: Off-screen assets provide strategic advantages
    \item \textbf{Campaign Clock Impact}: Major battles shift Mandate and Crisis dials
\end{itemize}

\subsection*{Siege Conditions}\index{siege conditions}

Extended combat scenarios create persistent conditions:

\begin{itemize}
    \item \textbf{Resource Depletion}: Supply clock fills rapidly
    \item \textbf{Fatigue Accumulation}: Characters gain Fatigue each day
    \item \textbf{Environmental Hazards}: Weather, disease, or magical effects
\end{itemize}

\section*{Between Sessions: GM Responsibilities}\index{GM responsibilities}

Between game sessions, the Game Master has crucial preparation and administrative tasks to ensure continuity, challenge, and narrative coherence. This downtime is essential for maintaining the campaign's momentum and preparing meaningful challenges for the players.

\subsection*{Mandatory Preparation}\index{GM responsibilities!preparation}

The GM must complete the following essential tasks before the next session:

\textbf{Campaign Clock Updates}: Advance Mandate and Crisis clocks based on session outcomes and player actions. Track significant developments that affect the overall campaign trajectory.

\textbf{Complication Debt Management}\index{Story Beats}: Calculate starting SB for the session based on:
\begin{itemize}
    \item Banked SB from previous sessions (maximum 2 SB carryover)
    \item Active character complications (+1 SB per character with complications)
    \item Asset-generated complications
    \item Epic Hooks spending
\end{itemize}

\textbf{Thread Management}\index{threads}: Review active complication threads and determine which are escalating, resolving, or requiring attention. Ensure no more than (Tier + 1) active threads per scene.

\textbf{Resource Tracking}: Update NPC statuses, faction relationships, and world conditions based on player actions.

\subsection*{Session Planning}\index{GM responsibilities!session planning}

Prepare the following elements for the upcoming session:

\textbf{Scene Preparation}: Design scenes with appropriate SB spending budgets:
\begin{itemize}
    \item Standard Scenes: Maximum 12 SB spending
    \item Climactic Scenes: Maximum 16 SB spending
    \item Session Budget: Maximum 20 SB total
\end{itemize}

\textbf{Complication Hooks}: Develop 3-5 potential complications that connect to player backgrounds, current threads, and campaign themes.

\textbf{Tactical Considerations}: Prepare for expected combat encounters, social challenges, and exploration scenes with appropriate difficulties and positioning.

\textbf{Deck Preparation}: Ensure Consequences Deck is ready with appropriate cards for expected complication types.

\subsection*{Narrative Development}\index{GM responsibilities!narrative development}

Advance the overarching story elements:

\textbf{Character Integration}: Review each player's complications, bonds, and flags to create personalized challenges and opportunities.

\textbf{World Response}: Determine how the game world reacts to player actions. Allies may offer new opportunities; enemies may escalate their responses.

\textbf{Plot Thread Advancement}: Move major campaign arcs forward, introducing new elements or escalating existing tensions.

\textbf{Foreshadowing}: Plant seeds for future complications and story developments that will pay off in later sessions.

\subsection*{Administrative Tasks}\index{GM responsibilities!administrative}

Complete necessary record-keeping:

\textbf{XP Award Calculation}: Tally session awards for each player:
\begin{itemize}
    \item Table Attendance: +2 XP
    \item Major Objectives: +2-4 XP
    \item Discoveries: +1-2 XP
    \item Hard Choices: +1-2 XP
    \item Complication Spotlight: +1-3 XP
    \item Bond/Flag Play: +1-2 XP
    \item GM Curveballs: +0-3 XP
    \item Complication Dividends: Face cards (+1 XP), Aces (+2 XP)
\end{itemize}

\textbf{Asset and Follower Updates}: Track any changes to player assets, followers, and off-screen resources based on session events.

\textbf{Session Zero Integration}: If applicable, incorporate elements from the Crown Spread and campaign seeds into upcoming scenes.

\subsection*{Strategic Considerations}\index{GM responsibilities!strategy}

Plan with the following factors in mind:

\textbf{Pacing Management}: Balance high-tension scenes with recovery opportunities. Avoid spending maximum SB every scene to maintain dramatic impact.

\textbf{Player Agency}: Ensure complications create interesting choices rather than simple obstacles. Every SB spend should offer meaningful player responses.

\textbf{Group Dynamics}: Consider how complications affect different players and maintain equitable challenge distribution.

\textbf{Teaching Moments}: For new groups or experimental rules (like high complication debt), prepare appropriate scaffolding and support.

\section*{Narrative First: Let the World React}\index{narrative first}

Campaign design in Fate's Edge is not about railroading---it's about \textbf{responding to player choices} with escalating consequences. Let the world shift. Let factions rise. Let the dice sing.

And when the Crown is crowned---let the echo be heard across the Amaranthine.

Remember: Your preparation directly impacts player enjoyment and engagement. Thorough session planning allows you to focus on collaborative storytelling during the game rather than scrambling for content. The investment in downtime preparation pays dividends in session quality and narrative coherence.

\end{chapter}
