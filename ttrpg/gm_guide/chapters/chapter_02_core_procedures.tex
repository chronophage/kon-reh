% !TEX root = ../fates_edge_players_guide.tex

\chapter{Running the Game: Core Procedures}

In \textbf{Fate's Edge}, the game flows through a series of \textbf{actions, consequences, and escalating stakes}. As the GM, your role is to guide this flow—not by dictating outcomes, but by \textbf{framing scenes, interpreting rolls, and spending Complication Points} to keep tension alive. This chapter walks you through the core procedures that define play, from the moment a player declares an action to the fallout that follows.

\section*{Scene Framing: Start with Stakes}

Every scene begins with a question: \textbf{What's at risk?} Not just for the characters, but for the world, the mission, or the fragile alliances they've built. As the GM, you frame the scene by establishing:

\begin{itemize}
    \item \textbf{Position}: Is the action \textit{Controlled}, \textit{Risky}, or \textit{Desperate}?
    \item \textbf{Effect}: What happens on a success? What changes?
    \item \textbf{Stakes}: What is gained—or lost—if things go wrong?
\end{itemize}

A scene in the \textbf{Mistlands} might begin with the PCs crossing a flooded causeway at dusk. The bell-line hums with tension. The GM sets the position as Risky—slippery stones, rising mist, and the distant echo of a wraith-call. A failure here could mean separation, exposure, or worse.

\subsection*{Position Descriptions}

\begin{itemize}
    \item \textbf{Controlled}: You act on your terms. Complications are minor, setbacks are rare.
    \item \textbf{Risky}: You act under pressure. Success is possible, but failure brings a cost.
    \item \textbf{Desperate}: The odds are stacked against you. Success is hard-won, and failure is dramatic.
\end{itemize}

Use position to guide the fiction. A controlled entry into a noble salon in \textbf{Vhasia} might allow the PCs to charm or intimidate without resistance. A desperate one—perhaps after triggering an alarm—means blades are drawn before words.

\section*{Adjudicating Rolls: The Outcome Matrix}

When a player rolls, they are not simply trying to "beat" a number. They are engaging with the world. The \textbf{Outcome Matrix} is your guide to interpreting the result in context.

\subsection*{Step-by-Step Roll Resolution}

\begin{enumerate}
    \item \textbf{Player declares action and approach} (Attribute + Skill).
    \item \textbf{GM sets Difficulty Value (DV)} based on stakes and fiction.
    \item \textbf{Player rolls pool of d10s.}
    \item \textbf{Count successes (6+)} and \textbf{Complication Points (1s)}.
    \item \textbf{Compare successes to DV} and apply Outcome Matrix.
    \item \textbf{GM spends CP} or draws from the Deck of Consequences.
\end{enumerate}

\subsection*{Outcome Matrix}

\begin{center}
\begin{tabular}{lll}
\toprule
\textbf{Case} & \textbf{Name} & \textbf{Guidance} \\
\midrule
$S \geq DV$ and $C = 0$ & Clean Success & Deliver the intent crisply. \\
$S \geq DV$ and $C > 0$ & Success \& Cost & Grant the intent; spend/bank CP for complications. \\
$0 < S < DV$ & Partial & Progress with a fork. \\
$S = 0$ & Miss & No progress. Cash/bank CP or offer Devil's Bargain. \\
\bottomrule
\end{tabular}
\end{center}

\subsection*{Difficulty Ladder (Set Before the Roll)}

\begin{center}
\begin{tabular}{cll}
\toprule
\textbf{DV} & \textbf{Name} & \textbf{When to Use} \\
\midrule
1 & Routine & Clear intent, modest stakes, controlled environment. \\
2 & Pressured & Time pressure, mild resistance, partial info. \\
3 & Hard & Hostile conditions, active opposition, precise timing. \\
4+ & Extreme & Multiple constraints, high precision, dramatic failure. \\
\bottomrule
\end{tabular}
\end{center}

A DV should reflect not just mechanical difficulty, but narrative weight. Climbing a wall? That's routine. Climbing it while pursued by Aeler vault-wardens? That's pressured—or worse.

\section*{Complication Points: The Engine of Drama}

Every time a player rolls a \textbf{1}, a Complication Point is generated. These are not mere penalties—they are narrative levers. Spend them to:

\begin{itemize}
    \item Escalate a threat (drawing more enemies, raising the stakes).
    \item Drain resources (time, gear, positioning).
    \item Reveal hidden dangers or betrayals.
    \item Cause collateral damage or unintended consequences.
\end{itemize}

Complication Points should \textbf{push the story forward}, not grind it to a halt. Use them to add pressure, not to punish.

\subsection*{Complication Point (CP) Spend Menu}

\begin{itemize}
    \item \textbf{1 CP}: Minor pressure: noise, trace, +1 Supply segment.
    \item \textbf{2 CP}: Moderate setback: alarm raised, lose position/cover, lesser foe or lock.
    \item \textbf{3 CP}: Serious trouble: reinforcements, key gear breaks, rail tick.
    \item \textbf{4+ CP}: Major turn: trap springs, authority arrives, scene shifts.
\end{itemize}

\subsection*{When to Draw from the Deck of Consequences}

The Deck of Consequences is a powerful tool for \textbf{thematic consistency}. When a player generates CP, you may choose to:

\begin{itemize}
    \item \textbf{Direct Spend}: Translate CP into consequences/rail ticks immediately.
    \item \textbf{Deck Draw}: Draw up to \textbf{min(CP, 3)} cards and \textbf{synthesize a single twist} guided by suit and highest rank.
\end{itemize}

Never do both for the same roll. If the drawn card contradicts established fiction, reinterpret or redraw to fit the suit and tone.

\paragraph{High-Tier CP Sinks.}
For 3–6+ CP spends that move the world (reputation cascades, faction instability, resonance, prophecy), see the stand-alone \emph{High CP Sinks} handout. A good default: at end of leg, \textbf{3 CP → tick 1 Front}.

\section*{Position + Effect in Action}

A player declares a \textbf{Risky} action to \textbf{pick a lock} while guards patrol above. They roll \textbf{Wits + Skullduggery}, get 3 successes and 1 Complication Point.

The GM consults the Outcome Matrix: \textbf{Success \& Cost}. The lock clicks—but a guard's bootstep halts above. The GM spends 1 CP to add tension: the patrol changes direction, heading toward the PCs' position.

\section*{Scene Starters and Hooks}

To keep the game moving, always open a scene with a strong hook:

\begin{itemize}
    \item "The alarm bells ring as you step into the courtyard."
    \item "A courier collapses at your feet, clutching a sealed scroll."
    \item "The tide is turning—the ghost-ferry won't wait."
\end{itemize}

Let the players react. Let the world respond. And always—\textbf{follow the consequences.}

\section*{Setting Stakes Fast (Cheat Prompts)}

\begin{itemize}
    \item If this goes right, what changes?
    \item If this goes wrong, what bites back?
\end{itemize}

\section*{Banking \& Cashing CP}

\begin{itemize}
    \item Banked CP should pay off within the same scene or arc.
    \item Avoid nickel-and-diming. Prefer one memorable complication over many petty penalties.
\end{itemize}

\end{chapter}
