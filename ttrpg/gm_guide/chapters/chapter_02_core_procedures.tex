% !TEX root = ../fates_edge_players_guide.tex

\chapter{Running the Game: Core Procedures}

In \textbf{Fate's Edge}, the game flows through a series of \textbf{actions, consequences, and escalating stakes}. As the GM, your role is to guide this flow—not by dictating outcomes, but by \textbf{framing scenes, interpreting rolls, and spending Complication Points} to keep tension alive. This chapter walks you through the core procedures that define play, from the moment a player declares an action to the fallout that follows.

\section*{Scene Framing: Start with Stakes}

Every scene begins with a question: \textbf{What's at risk?} Not just for the characters, but for the world, the mission, or the fragile alliances they've built. As the GM, you frame the scene by establishing:

\begin{itemize}
    \item \textbf{Position}: Is the action \textit{Controlled}, \textit{Risky}, or \textit{Desperate}?
    \item \textbf{Effect}: What happens on a success? What changes?
    \item \textbf{Stakes}: What is gained—or lost—if things go wrong?
\end{itemize}

A scene in the \textbf{Mistlands} might begin with the PCs crossing a flooded causeway at dusk. The bell-line hums with tension. The GM sets the position as Risky—slippery stones, rising mist, and the distant echo of a wraith-call. A failure here could mean separation, exposure, or worse.

\subsection*{Position Descriptions}

\begin{itemize}
    \item \textbf{Controlled}: You act on your terms. Complications are minor, setbacks are rare.
    \item \textbf{Risky}: You act under pressure. Success is possible, but failure brings a cost.
    \item \textbf{Desperate}: The odds are stacked against you. Success is hard-won, and failure is dramatic.
\end{itemize}

Use position to guide the fiction. A controlled entry into a noble salon in \textbf{Vhasia} might allow the PCs to charm or intimidate without resistance. A desperate one—perhaps after triggering an alarm—means blades are drawn before words.

\section*{Adjudicating Rolls: The Outcome Matrix}

When a player rolls, they are not simply trying to "beat" a number. They are engaging with the world. The \textbf{Outcome Matrix} is your guide to interpreting the result in context.

\subsection*{Step-by-Step Roll Resolution}

\begin{enumerate}
    \item \textbf{Player declares action and approach} (Attribute + Skill).
    \item \textbf{GM sets Difficulty Value (DV)} based on stakes and fiction.
    \item \textbf{Player rolls pool of d10s.}
    \item \textbf{Count successes (6+)} and \textbf{Complication Points (1s)}.
    \item \textbf{Compare successes to DV} and apply Outcome Matrix.
    \item \textbf{GM spends CP} or draws from the Deck of Consequences.
\end{enumerate}

\subsection*{Outcome Matrix}

\begin{center}
\begin{tabular}{lll}
\toprule
\textbf{Case} & \textbf{Name} & \textbf{Guidance} \\
\midrule
$S \geq DV$ and $C = 0$ & Clean Success & Deliver the intent crisply. \\
$S \geq DV$ and $C > 0$ & Success \& Cost & Grant the intent; spend/bank CP for complications. \\
$0 < S < DV$ & Partial & Progress with a fork. \\
$S = 0$ & Miss & No progress. Cash/bank CP or offer Devil's Bargain. \\
\bottomrule
\end{tabular}
\end{center}

\subsection*{Difficulty Ladder (Set Before the Roll)}

\begin{center}
\begin{tabular}{cll}
\toprule
\textbf{DV} & \textbf{Name} & \textbf{When to Use} \\
\midrule
1 & Routine & Clear intent, modest stakes, controlled environment. \\
2 & Pressured & Time pressure, mild resistance, partial info. \\
3 & Hard & Hostile conditions, active opposition, precise timing. \\
4+ & Extreme & Multiple constraints, high precision, dramatic failure. \\
\bottomrule
\end{tabular}
\end{center}

A DV should reflect not just mechanical difficulty, but narrative weight. Climbing a wall? That's routine. Climbing it while pursued by Aeler vault-wardens? That's pressured—or worse.

\section*{Complication Points: The Engine of Drama}

Every time a player rolls a \textbf{1}, a Complication Point is generated. These are not mere penalties—they are narrative levers. Spend them to:

\begin{itemize}
    \item Escalate a threat (drawing more enemies, raising the stakes).
    \item Drain resources (time, gear, positioning).
    \item Reveal hidden dangers or betrayals.
    \item Cause collateral damage or unintended consequences.
\end{itemize}

Complication Points should \textbf{push the story forward}, not grind it to a halt. Use them to add pressure, not to punish.

\subsection*{Complication Point (CP) Spend Menu}

\begin{itemize}
    \item \textbf{1 CP}: Minor pressure: noise, trace, +1 Supply segment.
    \item \textbf{2 CP}: Moderate setback: alarm raised, lose position/cover, lesser foe or lock.
    \item \textbf{3 CP}: Serious trouble: reinforcements, key gear breaks, rail tick.
    \item \textbf{4+ CP}: Major turn: trap springs, authority arrives, scene shifts.
\end{itemize}

\subsection*{When to Draw from the Deck of Consequences}

The Deck of Consequences is a powerful tool for \textbf{thematic consistency}. When a player generates CP, you may choose to:

\begin{itemize}
    \item \textbf{Direct Spend}: Translate CP into consequences/rail ticks immediately.
    \item \textbf{Deck Draw}: Draw up to \textbf{min(CP, 3)} cards and \textbf{synthesize a single twist} guided by suit and highest rank.
\end{itemize}

Never do both for the same roll. If the drawn card contradicts established fiction, reinterpret or redraw to fit the suit and tone.

\paragraph{High-Tier CP Sinks.}
For 3–6+ CP spends that move the world (reputation cascades, faction instability, resonance, prophecy), see the stand-alone \emph{High CP Sinks} handout. A good default: at end of leg, \textbf{3 CP → tick 1 Front}.

\section*{Position + Effect in Action}

A player declares a \textbf{Risky} action to \textbf{pick a lock} while guards patrol above. They roll \textbf{Wits + Skullduggery}, get 3 successes and 1 Complication Point.

The GM consults the Outcome Matrix: \textbf{Success \& Cost}. The lock clicks—but a guard's bootstep halts above. The GM spends 1 CP to add tension: the patrol changes direction, heading toward the PCs' position.

\section*{Scene Starters and Hooks}

To keep the game moving, always open a scene with a strong hook:

\begin{itemize}
    \item "The alarm bells ring as you step into the courtyard."
    \item "A courier collapses at your feet, clutching a sealed scroll."
    \item "The tide is turning—the ghost-ferry won't wait."
\end{itemize}

Let the players react. Let the world respond. And always—\textbf{follow the consequences.}

\section*{Setting Stakes Fast (Cheat Prompts)}

\begin{itemize}
    \item If this goes right, what changes?
    \item If this goes wrong, what bites back?
\end{itemize}

\section*{Banking \& Cashing CP}

\begin{itemize}
    \item Banked CP should pay off within the same scene or arc.
    \item Avoid nickel-and-diming. Prefer one memorable complication over many petty penalties.
\end{itemize}

\section*{Integrated Combat Procedures}

Combat in \textbf{Fate's Edge} follows the same core procedures as all other actions, but with specific applications for violent conflict. Every combat action generates potential for both triumph and complication, with consequences that cascade through the same economy as all other challenges.

\subsection*{Combat Resolution Procedure}

\begin{enumerate}
    \item \textbf{Declare Action}: Player states intent and approach (Attribute + Skill)
    \item \textbf{Set Position}: GM sets Controlled, Risky, or Desperate based on tactical situation
    \item \textbf{Roll Dice}: Roll pool = Attribute + Skill
    \item \textbf{Count Results}: 6+ = Success, 1 = Complication Point (CP)
    \item \textbf{Apply Outcome}: Use standard Outcome Matrix
    \item \textbf{Manage Consequences}: GM spends CP or draws from Consequences Deck
\end{enumerate}

\subsection*{Combat-Specific Position Applications}

\begin{itemize}
    \item \textbf{Controlled}: Advantageous position, minor consequences (flanking, higher ground, surprised foe)
    \item \textbf{Risky}: Even odds, moderate consequences (evenly matched, contested terrain)
    \item \textbf{Desperate}: Disadvantaged, severe consequences (outnumbered, wounded, poor positioning)
\end{itemize}

\subsection*{Combat Consequence Types by Suit}

The Deck of Consequences takes on specific meaning in combat:

\begin{itemize}
    \item \textbf{Hearts}: Morale, fear, command/control breakdown, psychological pressure
    \item \textbf{Spades}: Physical harm, positioning changes, weapon status, tactical wounds
    \item \textbf{Clubs}: Resource depletion, gear damage, fatigue, ammunition/supply issues
    \item \textbf{Diamonds}: Environmental hazards, reinforcements, tactical setbacks, terrain changes
\end{itemize}

\subsection*{Harm Integration with CP Economy}

Harm tracks directly tie to the CP economy, creating cascading consequences:

\begin{itemize}
    \item \textbf{Minor (-)}: Generate 1 CP on next 2 rolls
    \item \textbf{Moderate (=)}: Generate 1 CP on next roll, -1 die to relevant actions
    \item \textbf{Severe (≈)}: Generate 2 CP on next roll, -2 dice to relevant actions  
    \item \textbf{Critical (†)}: Generate 3 CP on next roll, out of action until treated
\end{itemize}

\subsection*{Tactical Clocks}

Persistent combat conditions are tracked through clocks:

\begin{itemize}
    \item \textbf{Mob Overwhelm} (6): Enemy numbers become advantage
    \item \textbf{Fatigue Spiral} (4): Exhaustion affects performance
    \item \textbf{Morale Collapse} (6): Fear undermines effectiveness
    \item \textbf{Environmental Collapse} (8): Terrain/fire/building failure
\end{itemize}

\subsection*{Position Dynamics in Combat}

Position can shift during combat based on CP spending:

\begin{itemize}
    \item \textbf{1 CP}: Shift position one step (GM choice)
    \item \textbf{Player Spending}: 1 CP to improve position one step
    \item \textbf{Narrative Triggers}: Flanking, reinforcement arrival, environmental changes
\end{itemize}

\subsection*{Magic Combat Integration}

Spellcasting in combat feeds the same consequence economy:

\begin{itemize}
    \item Channel/Weave Backlash CP applies to tactical situation
    \item Spells can shift position, create tactical clocks, or generate combat consequences
    \item Magic consequences cascade through existing combat systems
\end{itemize}

\subsection*{Asset/Follower Combat Integration}

\begin{itemize}
    \item \textbf{Follower Risk}: 2+ CP spent in combat can endanger assisting followers
    \item \textbf{Asset Compromise}: Combat in certain locations can damage relevant assets  
    \item \textbf{Offensive Activation}: 1 Boon activates asset for combat advantage
    \item \textbf{Initiative Actions}: Followers can take combat-relevant independent actions
\end{itemize}

\subsection*{Combat Outcome Matrix Application}

Same as standard resolution, but consequences are combat-specific:

\begin{itemize}
    \item \textbf{Clean Success}: Intent achieved with no tactical complications
    \item \textbf{Success \& Cost}: Intent achieved, but GM spends CP for combat consequences
    \item \textbf{Partial}: Progress with tactical fork (accept cost OR concede ground)
    \item \textbf{Miss}: No progress; GM spends CP for combat consequences OR offers tactical bargain
\end{itemize}

\end{chapter}
