% !TEX root = ../fates_edge_players_guide.tex

\chapter{Tier IV and V Play}

As characters reach Tier IV (Paragon) and Tier V (Mythic) levels, the scope of play expands dramatically. What once were local concerns become matters of regional, national, or even world-shaking importance. This chapter provides guidance for managing the unique challenges and opportunities that come with high-tier play.

\section{The Nature of High-Tier Play}

At Tier IV and V, characters are no longer operating on the margins—they are movers and shakers. Their actions have visible, lasting impacts on the world. This shift requires the GM to think bigger, plan longer, and embrace the cascading consequences of player choices.

\subsection*{Key Characteristics}

\begin{itemize}
    \item \textbf{Wider Scope}: Actions affect cities, regions, or nations
    \item \textbf{Longer Timelines}: Consequences unfold over weeks, months, or years
    \item \textbf{Greater Stakes}: Failure means more than personal loss
    \item \textbf{Complex Alliances}: Multiple factions with competing interests
    \item \textbf{Legacy Impact}: Choices create lasting changes to the world
\end{itemize}

\section{Deck-Based Campaign Management}

High-tier play benefits from structured campaign management using the Deck of Consequences and Travel Decks to track large-scale developments.

\subsection*{Campaign Clock Expansion}

Expand beyond Mandate and Crisis to include:

\begin{itemize}
    \item \textbf{Faction Influence} (6): Track major faction relationships
    \item \textbf{Public Opinion} (8): Regional perception of the PCs
    \item \textbf{Resource Network} (6): Economic and logistical reach
    \item \textbf{Legacy Projects} (10): Long-term initiatives with lasting impact
\end{itemize}

\subsection*{Using Cards for World Events}

Draw cards periodically to introduce world events:

\begin{itemize}
    \item \textbf{Spades}: Geographic/political changes
    \item \textbf{Hearts}: Social/cultural shifts
    \item \textbf{Clubs}: Economic/resource developments
    \item \textbf{Diamonds}: Opportunities/leverage points
\end{itemize}

\section{Managing Multiple Assets and Followers}

Tier IV+ characters often command extensive networks. Use these techniques to keep management manageable:

\subsection*{Asset Clustering}

Group related assets into portfolios:
\begin{itemize}
    \item \textbf{Economic}: Trade routes, businesses, investments
    \item \textbf{Political}: Titles, contacts, influence networks
    \item \textbf{Military}: Mercenaries, fortifications, strategic positions
    \item \textbf{Intelligence}: Spies, informants, research facilities
\end{itemize}

\subsection*{Follower Hierarchies}

Create chains of command:
\begin{itemize}
    \item \textbf{Lieutenants} (Cap 4-5): Direct reports who manage others
    \item \textbf{Commanders} (Cap 3): Mid-level managers of specific portfolios
    \item \textbf{Agents} (Cap 2): Field operatives and specialists
\end{itemize}

\section{High-Stakes Consequences}

Complication Points at high tiers should reflect the expanded scope of play:

\subsection*{High-Tier CP Sinks}

\begin{itemize}
    \item \textbf{3-4 CP}: Regional setback, major asset compromised
    \item \textbf{5-6 CP}: Faction relationship damaged, public scandal
    \item \textbf{7-8 CP}: Strategic position lost, major ally turned
    \item \textbf{9+ CP}: Paradigm shift, fundamental world change
\end{itemize}

\subsection*{Deck-Driven Consequences}

Use the Deck of Consequences for major setbacks:
\begin{itemize}
    \item \textbf{Face Cards}: Major faction leaders or institutions affected
    \item \textbf{Aces}: Foundational assumptions challenged
    \item \textbf{Multiple Cards}: Cascade effects across multiple domains
\end{itemize}

\section{Running Epic Campaigns}

High-tier play often involves extended campaigns with multiple acts and lasting consequences.

\subsection*{Act Structure}

\begin{itemize}
    \item \textbf{Act I - Establishment} (Sessions 1-3): Set the stage, establish stakes
    \item \textbf{Act II - Escalation} (Sessions 4-8): Complications multiply, alliances shift
    \item \textbf{Act III - Resolution} (Sessions 9-12): Climactic confrontations, lasting changes
    \item \textbf{Epilogue} (Session 13+): Legacy assessment, new beginnings
\end{itemize}

\subsection*{Campaign Seeds}

Use the full 4-card draw for major campaign hooks:
\begin{itemize}
    \item \textbf{Spade}: Primary location/region of conflict
    \item \textbf{Heart}: Key faction/leader driving events
    \item \textbf{Club}: Major complication/threat
    \item \textbf{Diamond}: Opportunity/resource to exploit
\end{itemize}

\section{Mass Combat and Warfare}

Tier IV+ characters often find themselves commanding armies or influencing wars.

\subsection*{Army Scale Combat}

Simplify large-scale battles:
\begin{itemize}
    \item Treat armies as Cap 5 followers with specialized skills
    \item Use clocks to track morale, supply, and strategic position
    \item Focus rolls on leadership and tactical decisions, not individual combat
\end{itemize}

\subsection*{War Campaigns}

Structure extended conflicts:
\begin{itemize}
    \item \textbf{Strategic Phase}: Resource management, alliance building
    \item \textbf{Tactical Phase}: Key battles, covert operations
    \item \textbf{Political Phase}: Negotiations, aftermath management
\end{itemize}

\section{Mythic Challenges}

At Tier V, characters approach legendary status. Create challenges that match their stature:

\subsection*{Existential Threats}

\begin{itemize}
    \item Cosmic entities beyond normal understanding
    \item Reality-altering phenomena
    \item Threats to entire civilizations or ways of life
\end{itemize}

\subsection*{Legacy Missions}

Missions that will be remembered for generations:
\begin{itemize}
    \item Founding or destroying nations
    \item Ending or beginning ages
    \item Reshaping fundamental aspects of the world
\end{itemize}

\section{Managing Player Agency}

With great power comes the need for great GM flexibility:

\subsection*{Player-Driven Narratives}

\begin{itemize}
    \item Let player choices genuinely reshape the world
    \item Honor long-term commitments and consequences
    \item Provide meaningful opposition that matches their scale
\end{itemize}

\subsection*{World Reactivity}

\begin{itemize}
    \item Factions respond realistically to player actions
    \item Economic and political systems show cause-and-effect
    \item NPCs remember and react to past interactions
\end{itemize}

\section{Rivals and Counterpoints}

High-tier characters attract attention—both positive and negative:

\subsection*{Creating Worthy Opponents}

\begin{itemize}
    \item Mirror player capabilities and resources
    \item Give them their own networks and influence
    \item Create personal connections and history with PCs
\end{itemize}

\subsection*{Dynamic Rivalry}

\begin{itemize}
    \item Rivals evolve based on player actions
    \item Competition across multiple domains (political, economic, social)
    \item Occasional cooperation against greater threats
\end{itemize}

\section{Campaign Legacy}

Help players see the lasting impact of their choices:

\subsection*{Legacy Tracking}

\begin{itemize}
    \item Document major world changes initiated by the PCs
    \item Track faction relationships and their evolution
    \item Record personal legacies and how they're remembered
\end{itemize}

\subsection*{Epilogue Framework}

Use cards to determine long-term outcomes:
\begin{itemize}
    \item Draw 2-3 cards from each suit
    \item Interpret results as 5-10 year outcomes
    \item Let players narrate their characters' final fates
\end{itemize}

\section{GM Preparation Tips}

\subsection*{Think in Campaign Arcs}

\begin{itemize}
    \item Plan 3-5 major story arcs per tier
    \item Each arc should have lasting world impact
    \item Connect arcs through recurring themes or NPCs
\end{itemize}

\subsection*{Prepare Flexible Frameworks}

\begin{itemize}
    \item Create faction relationship matrices
    \item Develop economic and political systems that respond to actions
    \item Build modular locations that can evolve
\end{itemize}

\subsection*{Embrace Player Creativity}

\begin{itemize}
    \item Let player assets genuinely solve problems
    \item Reward creative use of influence and resources
    \item Say "yes" to ambitious player plans, then make them interesting
\end{itemize}

\section{Sample High-Tier Scenario}

\textbf{The Shattered Crown Crisis}

A Tier IV campaign seed:
\begin{itemize}
    \item \textbf{Spade (King's Rest)}: Ancient royal crypts beneath the capital
    \item \textbf{Heart (The Usurper)}: A noble house claiming the vacant throne
    \item \textbf{Club (Fractured Loyalties)}: Regional lords choosing sides
    \item \textbf{Diamond (The Crown's Secret)}: Hidden royal treasures and alliances
\end{itemize}

Clocks: Succession Crisis (8), Noble Conspiracy (6), Public Unrest (6)

This scenario can evolve based on player choices—supporting the usurper, finding the true heir, or establishing a new form of government.

\section{Conclusion}

Tier IV and V play represents the pinnacle of Fate's Edge storytelling. Embrace the epic scope, honor player agency, and let the world truly respond to their legendary actions. Remember: these characters don't just participate in history—they make it.

The dice still matter, consequences still flow, and every choice still carries weight. But now, those choices echo across nations and generations.

Make it legendary.
\end{chapter}

