In \textbf{Fate's Edge}, the Game Master (GM) is not a referee or adversary, but a \textbf{collaborative storyteller} and \textbf{weaver of consequences}\index{weaver of consequences}. You are the guardian of the world's texture, the keeper of tension, and the guide who ensures that every action---no matter how small---ripples outward in meaningful ways.

You are not just running a game. You are stewarding a world where \textbf{the past is never truly gone}, where \textbf{choices echo}, and where \textbf{power always demands a price}. From the fog-shrouded bell-lines of the Mistlands to the sun-scorched marches of Acasia, the world of Fate's Edge is alive with history, ambition, and the quiet weight of forgotten oaths.

\section*{Narrative Primacy}\index{narrative primacy}

At the heart of Fate's Edge lies a single truth: \textbf{mechanics serve the story}\index{mechanics serve the story}. Dice do not merely determine success or failure---they shape the unfolding narrative by introducing new problems, twists, or opportunities. Every roll should change the story, not just resolve an action.

As the GM, your role is to keep the fiction alive. When a player rolls dice, the outcome should never be mechanical alone---it should always feed back into the world, altering the path ahead. Whether it's a moment of triumph or a twist of fate, the story is the ledger. Let the dice guide you, not constrain you.

In the courts of Vhasia, a noble's smile may conceal betrayal. In the under-vaults of Aeler, a misplaced seal can mean the difference between breath and doom. In the Valewood, even the trees remember your name. Your job is to make sure that every roll---every choice---carries that same weight.

\section*{Risk as Drama}\index{risk as drama}

Fate's Edge is built on the idea that \textbf{risk drives drama}\index{risk drives drama}. Every roll carries the potential for both triumph and complication. This is not a game where players accumulate power in a vacuum---every gain comes with a cost, and every victory shifts the balance of the world.

\textbf{Complication Points (CP)}\index{Complication Points (CP)}\index{CP|see{Complication Points}} are your primary tool for introducing tension. They are not penalties---they are narrative fuel. Use them to escalate stakes, introduce new threats, or deepen the emotional weight of a scene. The dice are not your enemy; they are your collaborator in crafting a living, breathing story.

A player may win a duel, but if they rolled a 1, perhaps the crowd begins to murmur that their blade was guided by luck---or something darker. Maybe the duel was witnessed by a rival faction. Maybe the blade itself now hums with a whisper it shouldn't carry. These are the threads you pull.

\section*{Meaningful Growth}\index{meaningful growth}

Characters in Fate's Edge do not level up in the traditional sense. Instead, they grow through \textbf{Experience Points (XP)}\index{Experience Points (XP)}\index{XP|see{Experience Points}}---a currency that represents meaningful choices and narrative investment. Players spend XP to enhance themselves, acquire assets, or unlock cultural talents. As the GM, you are the witness to this growth. You reward choices that shape the world, and you challenge players to live with the consequences of their decisions.

One PC may become a master duelist of Viterra, known for her silver tongue and her blade. Another might build a spy network across the Astroegro Straits, commanding influence from the shadows. A third may become a spirit-shielded dwarf, bearing the weight of ancestral voices. All are valid. All come at a cost. All change the world.

\section*{Your Tools as GM}\index{GM tools}

To guide the story, you have a set of tools designed to keep the narrative alive and evolving:

\begin{itemize}
    \item \textbf{Deck of Consequences}\index{Deck of Consequences}: A 52-card deck that provides thematic complications when players roll 1s. It externalizes risk and ensures that setbacks feel consistent and fair. Draw from it when the dice say the world pushes back.
    \item \textbf{Campaign Clocks}\index{Campaign Clocks}: Mandate and Crisis clocks that track the rise and fall of player influence, culminating in a finale shaped by their choices. These clocks are the heartbeat of long-term play.
    \item \textbf{Complication Points}\index{Complication Points}: Earned from dice rolls, spent to add tension, introduce threats, or reshape the narrative. These are your levers of drama.
    \item \textbf{Crown Spread}\index{Crown Spread}: A Session 0 ritual that seeds the campaign's themes, rivals, and finale conditions. It is here that the fate of nations---or the silence of forgotten gods---may be written.
\end{itemize}

\begin{tcolorbox}[enhanced, sharp corners, boxrule=1pt, colback=gray!5!white, colframe=gray!75!black, title={Flavor is Free}]
\textbf{Players and GMs:} Remember that in Fate's Edge, \textbf{flavor is free}\index{flavor is free}!

This means you can add descriptive details, cultural elements, and atmospheric touches to your actions without spending resources or requiring mechanical justification. Want to perform a parry with the traditional Aelerian bell-guard technique? Go ahead! Want to invoke the seasonal festivals of Theona when making a social roll? Perfect!

Flavor doesn't change the mechanical outcome of your actions, but it makes the world come alive and helps everyone at the table visualize and engage with the rich setting. Describe your character's background, their cultural customs, the local architecture, or the atmospheric details of a scene. These elements enrich the narrative without requiring dice rolls or resource expenditure.

The GM should encourage flavorful descriptions and may even provide additional descriptive details about the world in return. This collaborative approach to world-building through flavor helps create a more immersive experience for everyone involved.

Remember: Mechanics determine success or failure, but flavor determines the story we tell about how that success or failure came to be.
\end{tcolorbox}
