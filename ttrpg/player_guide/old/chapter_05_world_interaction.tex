\chapter{World Interaction}
\label{ch:world-interaction}

In \textbf{Fate's Edge}, the world is not a static backdrop—it reacts to your choices, complicates your plans, and evolves with your story. This chapter explores how you move through the world, what happens when you get lost, and how consequences ripple outward.

\section{Travel Framework}
\index{Travel Framework}

Travel in Fate's Edge is not about counting hexes or tracking rations—it's about narrative pacing and the risks you encounter along the way. The GM uses a \textbf{Travel Framework} to keep movement tense and meaningful.

\subsection*{Two Deck Systems (Compatibility)}
\index{Deck systems}

Fate's Edge uses two distinct card tools:

\paragraph{Travel Decks (regional, 52-card).}
\emph{Spade}=Place, \emph{Heart}=Actor, \emph{Club}=Pressure, \emph{Diamond}=Leverage. These power journeys and gates.

\paragraph{Deck of Consequences (scene drama).}
\emph{Hearts}=social fallout, \emph{Spades}=harm/escalation, \emph{Clubs}=material cost, \emph{Diamonds}=magical/spiritual disturbance.

\textit{Guidance:} Never mix suit meanings across decks. When a rule references ``Spade/Club/Diamond,'' it means \emph{Travel}. When it says ``Hearts/Spades/Clubs/Diamonds,'' it means \emph{Consequences}.

\subsection*{Core Travel Procedure}
\index{Travel Framework!procedure}

For each leg of a journey, draw 3–4 cards using the decks for your destination and controlling authority:

\begin{itemize}
  \item Spade from the destination deck: sets the scene (place).
  \item Heart from the destination deck: introduces the local actor or faction.
  \item Club from the Wilds (general hazards) or destination (if strongly policed): brings pressure.
  \item Diamond from the authority that gates the route: papers, escorts, rights, or exceptions.
\end{itemize}

\subsection*{Clock Size}
\index{Travel Framework!clock size}

The highest card rank determines the \textbf{Clock Size} for the travel leg:

\begin{center}
\begin{tabular}{cl}
\toprule
\textbf{Rank} & \textbf{Clock Size} \\
\midrule
2–5 & 4 segments \\
6–10 & 6 segments \\
J/Q/K & 8 segments \\
Ace & 10 segments \\
\bottomrule
\end{tabular}
\end{center}

\subsection*{Resolving Travel}
\index{Travel Framework!resolution}

Each card drawn introduces a narrative element. The GM describes how it affects the journey. Players may act to resolve complications or advance the clock.

Set a travel clock by the highest rank. On success, advance to the next leg; on failure, mark delay, debt, or diversion and resolve a consequence in the fiction.

\section{Narrative Time}
\index{Narrative Time}

Time in Fate's Edge is flexible and story-driven. Actions are framed in four narrative scales:

\begin{description}
  \item[A Moment] \index{Narrative Time!A Moment} — A heartbeat, a glance, a single strike or word.
  \item[Some Time] \index{Narrative Time!Some Time} — A few minutes: a skirmish, a careful lockpick, a short negotiation.
  \item[Significant Time] \index{Narrative Time!Significant Time} — Hours: travel between locations, working a ritual, recovering from harm.
  \item[Days] \index{Narrative Time!Days} — Large-scale endeavors: marches, training, major recovery.
\end{description}

\section{Movement and Positioning}
\index{Movement}
\index{Positioning}

Movement and positioning in Fate's Edge are handled through narrative bands and the Position dynamic, which affect your ability to act and the consequences of failure.

\subsection*{Range Bands}
\index{Range Bands}

Distance is described in simple, narrative terms:
\begin{description}
  \item[Close] \index{Range!Close} — Arm's length, grapples, knives. You can touch, shove, or clinch.
  \item[Near] \index{Range!Near} — Same room/street segment/skirmish space; a quick step or two away.
  \item[Far] \index{Range!Far} — Same site/area but not in immediate reach; you need time, route, or a long implement.
  \item[Absent] \index{Range!Absent} — Off-screen / away; requires a cut or travel to interact.
\end{description}

\subsection*{Movement Actions}
\index{Movement!actions}
\begin{itemize}
  \item \textbf{1 Move} shifts one band: Close↔Near or Near↔Far.
  \item \textbf{Dash} (your action) shifts two bands: Close→Far or Far→Close in one go.
  \item Moving between bands may require tests if under pressure or in difficult terrain.
\end{itemize}

\subsection*{Position}
\index{Position}

Position sets consequence severity on a Partial/Miss; it does not change Difficulty Value (DV).
\begin{description}
  \item[Controlled] \index{Position!Controlled} — You have time, cover, or advantage.
  \item[Risky] \index{Position!Risky} — You are exposed, rushed, or off-balance.
  \item[Desperate] \index{Position!Desperate} — You are in immediate danger or at a severe disadvantage.
\end{description}

\subsection*{Entering and Leaving Melee}
\index{Melee}
\begin{itemize}
  \item \textbf{Enter:} 1 Move to engage from Near. If under fire, treat as Risky.
  \item \textbf{Leave:} 1 Move to break off; under threat, test to Disengage at Risky. On a Partial/Miss, suffer a soft consequence and remain Engaged.
\end{itemize}

\subsection*{Position Dynamics}
\index{Position!dynamics}
\begin{itemize}
  \item GM Spend (1 CP): Shift Position one step worse for the current action.
  \item Player Spend (1 Boon): Shift Position one step better for your current action.
  \item Narrative Triggers (free): Flanking, reinforcements, or environmental changes can move Position.
\end{itemize}

\section{Social Interactions}
\index{Social Interactions}

Social encounters in Fate's Edge are resolved through the same core mechanics as physical actions, but with different skills and stakes. The key is establishing clear intentions, appropriate skills, and meaningful consequences.

\subsection*{Social Skills}
\index{Social Interactions!skills}

Common skills used for social interactions include:
\begin{description}
  \item[Sway] \index{Skills!Sway} — Persuasion, negotiation, and influencing others through argument or charm.
  \item[Deceive] \index{Skills!Deceive} — Lying, misdirection, and bluffing.
  \item[Rapport] \index{Skills!Rapport} — Building trust, empathy, and emotional connection.
  \item[Command] \index{Skills!Command} — Authoritative direction, intimidation, and issuing orders.
  \item[Perform] \index{Skills!Perform} — Entertainment, oration, and artistic expression to influence.
\end{description}

\subsection*{Social Position}
\index{Social Interactions!position}

Just like physical actions, social interactions are affected by Position:
\begin{description}
  \item[Controlled] \index{Position!Controlled!Social} — You have the upper hand, inside knowledge, or social advantage.
  \item[Risky] \index{Position!Risky!Social} — The situation is uncertain, you're on neutral ground, or there's some risk.
  \item[Desperate] \index{Position!Desperate!Social} — You're at a severe disadvantage, under pressure, or facing an authority.
\end{description}

\subsection*{Social Consequences}
\index{Social Interactions!consequences}

Social actions generate Complication Points just like physical ones. When using the Deck of Consequences, social interactions typically draw from:
\begin{itemize}
  \item \textbf{Hearts:} Relationship damage, loss of trust, emotional fallout.
  \item \textbf{Clubs:} Social resources strained (allies lost, reputation damaged, favors called in).
  \item \textbf{Diamonds:} Supernatural or mystical social entanglements (curses, geas, fae bargains).
\end{itemize}

\subsection*{Establishing Stakes}
\index{Social Interactions!stakes}

Before rolling, clearly establish:
\begin{itemize}
  \item \textbf{Your Intent:} What do you want to achieve?
  \item \textbf{Your Approach:} Which skill and method will you use?
  \item \textbf{The Stakes:} What changes on success? What bites on failure?
  \item \textbf{The DV:} Set by the GM based on the situation's difficulty.
\end{itemize}

\subsection*{Social Combat}
\index{Social Interactions!combat}

Extended social conflicts (debates, court proceedings, negotiations) can use clocks to track progression:
\begin{itemize}
  \item \textbf{Influence Clock:} Track progress in swaying a group or individual.
  \item \textbf{Alliance Clock:} Build or break relationships over time.
  \item \textbf{Reputation Clock:} Measure the spread of information or change in social standing.
\end{itemize}

\subsection*{Bonds and Social Play}
\index{Social Interactions!bonds}

Character Bonds are crucial for social interactions:
\begin{itemize}
  \item Referencing a Bond in an Intricate social description can earn a Boon.
  \item Strong Bonds provide +1 die to relevant social rolls.
  \item Damaging a Bond through poor social choices can generate CP for the GM.
\end{itemize}


\section{Deck of Consequences}
\index{Deck of Consequences}

The \textbf{Deck of Consequences} is a shared storytelling tool. Whenever you roll a 1 and generate a Complication Point, the GM may draw a card instead of improvising a twist.

\subsection*{Using the Deck}
\index{Deck of Consequences!usage}

After a roll that generates CP, the GM chooses one method for that roll:
\begin{enumerate}
  \item \textbf{Direct Spend}: Translate CP into consequences/rail ticks immediately.
  \item \textbf{Deck Draw}: Draw up to \textbf{min(CP, 3)} cards and synthesize a single twist guided by suit and highest rank.
\end{enumerate}

\subsection*{Structure of the Deck}
\index{Deck of Consequences!structure}

\begin{description}
  \item[Suits] Represent domains of complications:
    \begin{itemize}
      \item Hearts — Emotional, social, or relational fallout.
      \item Spades — Harm, danger, or escalation of conflict.
      \item Clubs — Resource strain, economic or material cost.
      \item Diamonds — Magical, spiritual, or cosmic disturbances.
    \end{itemize}
  \item[Ranks] Represent severity:
    \begin{itemize}
      \item Ace–3 — Minor inconvenience or flavor complication.
      \item 4–6 — Moderate setback with some narrative teeth.
      \item 7–9 — Significant consequence altering the course of action.
      \item 10–King — Major fallout, introducing new problems or lasting scars.
    \end{itemize}
\end{description}

\section{Tags and Interactions}
\index{Tags}
\index{Interaction}

Tags are standardized mechanical shorthand for common effects, found on Talents, Abilities, Spell Results, and Outsider templates. They only function when explicitly printed on a source.

\subsection*{Key Interaction Tags}
\index{Tags!interaction}

\begin{description}
  \item[[WARD]] \index{Tags!WARD} Creates a magical edge that challenges Outsiders (or other targets if explicitly stated). Outsiders crossing a [WARD] must make a test (DV = Cap). On a Hit, they cross but their Leash gains +DV segments. On a Partial, they cross and Leash gains +1. On a Miss, they fail to cross this beat.
  \item[[BANISH]] \index{Tags!BANISH} Drives a visible Outsider toward departure. Roll against the target (DV = Cap). On a Hit, add +DV segments to its Leash. On a Partial, add +1 segment. If this fills the Leash, the Outsider departs after a brief "acts to nature" beat.
  \item[[UNWARD]] \index{Tags!UNWARD} Unmakes or suppresses a [WARD] created by a Talent/Ability or Spell result (DV by fiction).
  \item[[MARK]] \index{Tags!MARK} Tag a target for tracking or leverage. Once per scene, you or an ally gain +1 die when acting directly against the Marked target.
  \item[[FORTIFY]] \index{Tags!FORTIFY} Harden a person/place/object against a vector (fire, blades, fear, sway). Raise Position or reduce consequence severity vs. that vector this scene.
  \item[[CONJURE]] \index{Tags!CONJURE} Create a useful object/cover/hazard (DV by fiction). Hit: conjure item/zone with integrity [2/4/6] or a ticking hazard.
\end{description}

\subsection*{Using Tags in the World}
\index{Tags!usage}
\begin{itemize}
  \item Tags on spells, Talents, or Abilities function as written.
  \item Environmental features (like a naturally occurring barrier) do not have Tags unless a spell/Talent/Ability creates them.
  \item Outsiders interact with [WARD] and [BANISH] using the unified DV/Leash system.
\end{itemize}

\section{Understanding Tags}
\index{Tags!understanding}

Tags are standardized mechanical shorthand used throughout Fate's Edge to represent common effects, abilities, and interactions. They appear on Talents, Abilities, Spell Results, and Outsider templates. Tags only function when explicitly printed on a source—you cannot apply Tags through actions alone.

\subsection*{What Tags Do}
\index{Tags!function}

Tags provide clear, consistent mechanical effects:
\begin{itemize}
  \item They define specific interactions with the game systems.
  \item They create predictable outcomes for common magical or special effects.
  \item They allow different sources (spells, Talents, Outsider abilities) to interact with each other using shared language.
\end{itemize}

\subsection*{Common Interaction Tags}
\index{Tags!common}

\begin{description}
  \item[[WARD]] \index{Tags!WARD} Creates a magical boundary that challenges entities attempting to cross it. When an Outsider crosses a [WARD], they must make a test (DV = Outsider's Cap). Success allows crossing with consequences, while failure prevents crossing.
  \item[[BANISH]] \index{Tags!BANISH} Drives a visible Outsider toward departure. A successful [BANISH] test (DV = Outsider's Cap) adds segments to the Outsider's Leash, potentially forcing them to depart.
  \item[[UNWARD]] \index{Tags!UNWARD} Removes or suppresses a [WARD] effect (DV determined by fiction).
  \item[[MARK]] \index{Tags!MARK} Tags a target for tracking or to gain mechanical advantage. Once per scene, you or an ally gain +1 die when acting directly against a Marked target.
  \item[[FORTIFY]] \index{Tags!FORTIFY} Hardens a target against specific threats. Provides improved Position or reduces consequence severity versus the fortified vector for one scene.
  \item[[CONJURE]] \index{Tags!CONJURE} Creates temporary objects, barriers, or hazards (DV determined by fiction). Success creates the conjured item with a specific integrity rating or hazard effect.
  \item[[CLEANSE]] \index{Tags!CLEANSE} Removes or suppresses conditions, poisons, diseases, or [CURSE] effects (DV determined by fiction).
  \item[[BARRIER]] \index{Tags!BARRIER} Creates cover or obstruction (DV determined by fiction). Success places a barrier with specific integrity.
\end{description}

\subsection*{How to Read Tag Entries}
\index{Tags!reading}

When you see a Tag on a Talent, Ability, or Spell Result, it will typically include:
\begin{itemize}
  \item \textbf{Target/Range:} Who or what the Tag affects and from what distance.
  \item \textbf{Test/DV:} What roll is required and its Difficulty Value (often based on fiction or the target's Cap).
  \item \textbf{Effect:} What happens on a Hit, Partial, and Miss.
  \item \textbf{Duration:} How long the effect lasts.
\end{itemize}

\textbf{Example:} \texttt{[WARD] (Near; test Spirit+Lore, DV 3; Hit: +3 Leash segments; Partial: +1; Miss: cannot cross)}

\subsection*{Tags and Outsiders}
\index{Tags!Outsiders}

Tags are particularly important for interacting with Outsiders (non-native entities):
\begin{itemize}
  \item [WARD] and [BANISH] use the unified Outsider system: DV equals the Outsider's Cap.
  \item Successful [WARD] crossings add +DV segments to the Outsider's Leash.
  \item Successful [BANISH] tests add segments to the Leash, potentially forcing departure.
  \item [UNWARD] can remove [WARD] effects created by Talents, Abilities, or spells.
\end{itemize}

\subsection*{Using Tags Creatively}
\index{Tags!creative use}

While Tags only function when printed on sources, understanding them helps you:
\begin{itemize}
  \item Recognize when effects in the world might have Tag-like properties.
  \item Work with the GM to define improvised effects using Tag language.
  \item Understand how different magical or special abilities interact.
\end{itemize}

Remember: Tags provide structure, not limitation. They ensure that when a spell, Talent, or Outsider ability says it does something, everyone understands what that means.

\section{Supply Clock}
\index{Supply Clock}

The \textbf{Supply Clock} tracks the party's access to food, water, and gear.

\begin{center}
\begin{tabular}{cl}
\toprule
\textbf{Segments Filled} & \textbf{Effect} \\
\midrule
0 (Full) & The party is well-equipped. \\
2 (Low) & Minor narrative complications (bland food, damaged arrows, thinning waterskins). \\
3 (Dangerous) & Each character gains Fatigue. \\
4 (Empty) & Severe penalties. \\
\bottomrule
\end{tabular}
\end{center}

\section{Condition Tracks}
\index{Condition Tracks}

Characters and assets have \textbf{Condition Tracks} that reflect wear, neglect, or harm:

\begin{description}
  \item[Assets/Followers] — Maintained → Neglected → Compromised
  \item[Party Resources] — Supply (0-Full → 2-Low → 3-Dangerous → 4-Empty)
  \item[Character State] — Fatigue (1-4 levels, re-roll successes)
\end{description}

\subsection*{Fatigue}
\index{Fatigue}
\begin{itemize}
    \item Effect: On their next roll, a character must reroll one success.
    \item Stacking: Each level adds another forced reroll.
    \item Recovery: A night's rest with adequate supply removes 1 Fatigue.
\end{itemize}

\section{Engaging the World}
\index{World interaction}

The world of Fate's Edge is alive. When you:

\begin{itemize}
  \item Enter a new region, draw cards to seed local flavor.
  \item Negotiate with a faction, consider their suit (Heart = personal, Diamond = leverage).
  \item Face a hazard, let the Deck of Consequences guide the fallout.
  \item Cast a spell with a [WARD] or [BANISH], it interacts with Outsiders using the unified system.
  \item Move through a space, your Position and Range affect your actions and consequences.
\end{itemize}

\section{Summary}

The world of Fate's Edge is not a puzzle to be solved—it is a living, reactive force. Travel is a narrative journey, not a logistical grind. Every step forward risks a twist, and every twist changes the story.

Engage with the world boldly—and let it shape you in return.

\end{chapter}