% !TEX root = ../fates_edge_players_guide.tex

\chapter{Deck Generators}

In \textbf{Fate’s Edge}, the \textbf{Deck of Consequences} and \textbf{Region Generators} are powerful tools for creating dynamic, unpredictable stories. This chapter provides ready-to-use deck generators for major regions, helping you seed adventures, complications, and rewards with nothing more than a standard 52-card deck.

\section{How Deck Generators Work}

Deck generators use a standard playing card deck to randomly determine narrative elements:

\begin{description}
  \item[♠ Spades] — Places (locations, terrain, hidden dangers)
  \item[♡ Hearts] — People and Factions (NPCs, groups, rivals)
  \item[♣ Clubs] — Complications and Threats (obstacles, dangers, setbacks)
  \item[♢ Diamonds] — Rewards and Leverage (treasure, assets, advantages)
\end{description}

\section{Drawing Procedures}

\subsection*{Quick Hook (2 cards)}
Draw one Spade and one Heart. The Spade provides the place, the Heart the faction. Use the higher rank to set the Clock.

\subsection*{Full Seed (4 cards)}
Draw until one card of each suit appears:
\begin{enumerate}
  \item Spade = location
  \item Heart = main actor/faction
  \item Club = complication
  \item Diamond = reward/leverage
\end{enumerate}
The highest rank sets the main Clock size.

\subsection*{Act Builder}
For each act or session, draw three cards: setting, actor, complication. Save Diamonds to foreshadow leverage or as act payoffs.

\section{Rank Severity and Clock Size}

The card rank determines the size of the primary Clock for the scene or mission:

\begin{center}
\begin{tabular}{cl}
\toprule
\textbf{Rank} & \textbf{Clock Size} \\
\midrule
2–5 (Minor) & 4 segments \\
6–10 (Standard) & 6 segments \\
J, Q, K (Major) & 8 segments \\
Ace (Pivotal) & 10 segments \\
\bottomrule
\end{tabular}
\end{center}

\section{Color Influences}

\begin{itemize}
  \item \textbf{Black suits (♠♣)} — Travel hazards, tangible threats, fatigue
  \item \textbf{Red suits (♡♢)} — Social intrigue, reputational pressure
\end{itemize}

\section{Combo Rules}

\begin{description}
  \item[Pair (same rank)] — Recurring motif with a twist
  \item[Run (3+ sequential ranks)] — Momentum—reduce the main Clock by 1 segment
  \item[Flush (3+ same suit)] — Strongly theme the act toward that axis
  \item[Face + Ace] — Reveal a hidden patron or power behind the drawn element
  \item[All one color] — GM gains 1 free Story Beat in that scene
\end{description}

\section{Regional Deck Generators}

Each region has its own themed deck generator. Below are examples for major regions:

\subsection*{Viterra — "The Last Kingdom"}

\textbf{Spades — Places}
\begin{itemize}
  \item 2. Fen causeway stile with a toll-rod and a patient line of eel carts.
  \item 3. Hedgerow muster-green in the Dales; bows strung under apple trees.
  \item 4. Beacon hill above the Highlands; watch-fire grate and slate steps.
  \item 5. Belworth ferry-stairs with wet ledgers and a nervous horn.
  \item 6. Old iron-bloom quarry turned drill yard; hammer echoes carry.
  \item 7. Parish-stone maze where three maps disagree by a field.
  \item 8. Fairport tideworks at the river mouth; Dolmis swell under the planks.
  \item 9. Valora law quarter—archives, oath-rooms, and the Hall of Dawning nearby.
  \item 10. Tarlington counting fields beside the mustering ground; quiet efficiency hums.
  \item J. River dike crown: ring of turf and timber; brotherhood bells for flood watch.
  \item Q. The Queen's Progress encampment—canvas palisade, fresh standards, full schedule.
  \item K. Hall of Dawning tilt-yard at first light; Dawn-knights run clean drills.
  \item A. Queen's Highway mile-stone above the Dolmis road; customs writ posted.
\end{itemize}

\textbf{Hearts — People and Factions}
\begin{itemize}
  \item 2. Fen reeve with a tally-rod; speaks for the dike guilds.
  \item 3. River-carter syndic who moves grain faster than rumor.
  \item 4. Parish surveyor with three maps and one opinion.
  \item 5. Quartermaster of the Dawn (logistics first, lances second).
  \item 6. Dales levy serjeant—longbow calm, cider breath.
  \item 7. Two-altars cleric-pair (Light circuit-preacher vs Everflame canon lawyer).
  \item 8. Fairport shipwright with Dolmis cousins and a quiet skiff.
  \item 9. Fenwood comptroller who can conjure wagons with a signature.
  \item 10. Queen's Justiciar—law on the road, polite as a gallows.
  \item J. Border routier-captain who reads ledgers as well as ambushes.
  \item Q. The Newly Crowned Queen of Viterra—patient sums, sharp promises.
  \item K. The Crown in Council (Fenwood, duchy envoys, guild voices) weighing grain vs. glory.
  \item A. Tarling-blood rumor—a lost sigil surfaces; old loyalties twitch.
\end{itemize}

\textbf{Clubs — Complications and Threats}
\begin{itemize}
  \item 2. Dike breach in a black-rain; carts bog and tempers sink.
  \item 3. Feast-day clash: Light vs Everflame processions collide over tithes.
  \item 4. "Quiet tolls" sprout on the Queen's Highway; escorts sniff a trap.
  \item 5. Counting-house audit freezes your cargo until dawn.
  \item 6. Border-lace snarl: overlapping titles spark arrests mid-parish.
  \item 7. Isle refusal: Theona's moot withholds levy; quay rumors harden.
  \item 8. Delta spat: Fairport vs Marcott customs—barges stack three deep.
  \item 9. Routier arrears: free-company flips unless someone pays.
  \item 10. Salt pinch—import prices spike; bakers barricade.
  \item J. Dawn recall: your escort is pulled to a flood-girded parish.
  \item Q. Aberielist intrigue: a royalist ring stirs against the new crown.
  \item K. Levy call-up: dalesmen muster; your wagons conscripted "for the realm".
  \item A. Dolmis gale train: bora-like winds slam the coast; schedules drown.
\end{itemize}

\textbf{Diamonds — Rewards and Leverage}
\begin{itemize}
  \item 2. Ferry priority at a named Belworth crossing (once).
  \item 3. Dike-work allotment—brotherhood labor on your timetable.
  \item 4. Market day license in Valora's square.
  \item 5. Dawn escort letter (four lances at first light).
  \item 6. River-carter line—guaranteed haul on the grain artery.
  \item 7. Parish-map correction—move a border a hedgerow over.
  \item 8. Fairport customs seal for Dolmis-bound cargo.
  \item 9. County Thing ruling in your favor; local teeth, real bite.
  \item 10. Salt allotment from a guarded depot (winter only).
  \item J. Wardship of a minor fen-keep; men-at-arms "for now".
  \item Q. Private audience with the Queen; one secret exits as policy.
  \item K. Fenwood ducal warrant to enforce Highway customs.
  \item A. Coronation writ—temporary amnesty & tax-remission for those who align now.
\end{itemize}

\section{Other Regional Generators}

The complete set includes generators for:
\begin{itemize}
  \item Acasia — "Broken Marches"
  \item Ecktoria — "Marble & Fire"
  \item Ubral — "The Stone Between Spears"
  \item Thepyrgos — "City of a Thousand Stairs"
  \item Mistlands — "Bells, Salt, and Breath"
  \item Kahfagia — "Pilot's Mirror"
  \item Valewood — "Empire Under Leaves"
  \item The Wilds — "Roads, Ruins, and Weather"
\end{itemize}

Each generator is tailored to its region's themes, conflicts, and opportunities.

\section{Using Deck Generators at the Table}

\begin{enumerate}
  \item \textbf{Seed the Scene} — Draw cards to determine location, actors, and initial tension.
  \item \textbf{Set the Clock} — Use highest rank to determine primary challenge size.
  \item \textbf{Play to Themes} — Let the suits guide the type of complications and rewards.
  \item \textbf{Embrace Surprises} — When the deck gives you unexpected combinations, lean into them.
\end{enumerate}

\section{Summary}

Deck generators are tools for \textbf{collaborative storytelling}:

\begin{itemize}
  \item They introduce unpredictability and tension.
  \item They help GMs prep quickly and creatively.
  \item They make every session feel fresh and dangerous.
\end{itemize}

Let the cards guide your fate—but remember: you always have choices.

