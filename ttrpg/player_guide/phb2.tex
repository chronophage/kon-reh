Here's the revised Player's Guide in LaTeX format, incorporating the intro and fixing the mechanical issues:

```latex
\documentclass[11pt,twoside,openany]{book}
\usepackage[utf8]{inputenc}
\usepackage[T1]{fontenc}
\usepackage{geometry}
\usepackage{titlesec}
\usepackage{titletoc}
\usepackage{fancyhdr}
\usepackage{hyperref}
\usepackage{xcolor}
\usepackage{tcolorbox}
\usepackage{enumitem}
\usepackage{multicol}
\usepackage{tabularx}
\usepackage{makeidx}

\geometry{letterpaper,margin=1in}
\makeindex

\hypersetup{
    colorlinks=true,
    linkcolor=blue,
    filecolor=magenta,      
    urlcolor=cyan,
    pdftitle={Fate's Edge Player's Guide},
    pdfauthor={Nicholas A. Gasper}
}

\title{Fate's Edge: Player's Guide}
\author{Nicholas A. Gasper}
\date{October 12, 2025}

\pagestyle{fancy}
\fancyhf{}
\fancyhead[LE,RO]{\thepage}
\fancyhead[LO]{\leftmark}
\fancyhead[RE]{\rightmark}

\begin{document}

\maketitle

\thispagestyle{empty}
\begin{center}
\textbf{Fate's Edge: Player's Guide}\\
\textcopyright\ Nicholas A. Gasper\\[1em]

This work is licensed under the Creative Commons Attribution 4.0 International License.\\
To view a copy of this license, visit:\\
\url{http://creativecommons.org/licenses/by/4.0/}
\end{center}

\tableofcontents

\chapter{Introduction: The Weight of Choice}
\index{introduction}
\index{GM philosophy}

Welcome, Game Master. You hold a unique role in \textbf{Fate's Edge}. You are not a storyteller in solitude, nor a neutral referee. You are the \textbf{weaver of consequences}\index{weaver of consequences}, the \textbf{architect of a living world}, and the \textbf{guide on a path where every choice echoes}. Your task is to breathe life into a realm of ancient magic, fallen empires, and stubborn, vibrant cultures—and then to let that world truly respond to the players' ambitions.

This is a game where power demands a price, where the past never truly sleeps, and where a single decision can reshape a nation or end an age. From the marble forums of Ecktoria to the mist-drenched fens of the Mistlands, the world is alive with stories waiting to be told. Your job is to provide the stage, set the stakes, and embrace the beautiful, chaotic ripple effects of player agency.

\section*{A World Alive with Consequences}

In \textbf{Fate's Edge}, the fiction is the final authority. The rules in this book are not chains to bind your imagination, but \textbf{tools to give weight to your stories}. They provide a consistent framework for adjudicating risk, tracking progress, and ensuring that success and failure both drive the narrative forward in compelling ways.

Think of yourself as a conductor. The players provide the melody with their characters' actions and ambitions. You provide the harmony and rhythm with the world's response. The rules are your sheet music—a guide to creating a cohesive, dramatic piece, but one that allows for improvisation and adaptation.

\textbf{Your judgment is the cornerstone of the game.} If a rule doesn't serve the moment, change it. If a player's creative idea deserves to succeed, find a way to make it work. The ultimate goal is a collaborative, engaging story that everyone at the table helps to create.

\section*{The Core Philosophy: Narrative First}\index{narrative primacy}

At the heart of \textbf{Fate's Edge} is a simple, powerful idea: \textbf{mechanics serve the story}\index{mechanics serve the story}. A dice roll is never just a binary pass/fail check. It is an event that changes the fictional landscape.

\begin{itemize}
    \item A \textbf{Clean Success} means the plan works as intended—the guard is bribed, the lock clicks open, the argument sways the crowd.
    \item A \textbf{Success with Cost} means you get what you want, but the world pushes back—the guard takes the bribe but becomes a future liability, the lock opens but the mechanism is damaged, the crowd is swayed but a rival noble takes note.
    \item A \textbf{Partial} means you're faced with a difficult choice—you can open the lock but it will take time and risk discovery, or you can sway part of the crowd but alienate another faction.
    \item A \textbf{Complication} means the situation changes dramatically—a new threat appears, a hidden factor is revealed, the stakes are raised.
\end{itemize}

This approach ensures that every roll matters. The story never stalls; it evolves.

\section*{Risk is the Engine of Drama}\index{risk as drama}

\textbf{Fate's Edge} is built on the principle that \textbf{meaningful risk creates compelling drama}\index{risk drives drama}. Safety is boring. It is when characters have something to lose—their reputation, their allies, their ideals, their lives—that their actions become truly heroic or tragically memorable.

Your primary tool for managing this risk is the \textbf{Story Beat (SB)}\index{Story Beats (SB)} economy. When the dice show a 1, it's not merely a failure; it's the world reacting. The GM gains SB to introduce complications, escalate existing threats, or reveal hidden dangers. SB are not punishments; they are the fuel for an unpredictable, responsive narrative.

A successful sword swing might defeat an opponent, but a Story Beat spent could mean the blade is notched and less effective next time, or that the defeat draws the attention of a more powerful foe. The drama continues.

\section*{Characters Who Change the World}

Character growth in \textbf{Fate's Edge} is not about accumulating abstract power. It is about \textbf{meaningful growth}\index{meaningful growth} rooted in the story. Players earn \textbf{Experience Points (XP)}\index{Experience Points (XP)} by engaging with the world's challenges and complexities. They spend XP to improve their capabilities, acquire assets like a ship or a spy network, or unlock unique cultural talents.

This means character advancement is directly tied to the narrative. A character becomes a legendary commander by leading armies, not by killing monsters in a vacuum. They become a master wizard by uncovering forbidden lore and surviving the backlash, not by memorizing spells from a textbook. As the GM, you are the curator of this growth, presenting challenges that allow characters to evolve in ways that feel earned and impactful.

\section*{Your Toolkit}

To help you guide the story, \textbf{Fate's Edge} provides a set of elegant, interconnected tools:

\begin{itemize}
    \item \textbf{The Dice Pool}: The core mechanic. Players roll a number of d10s equal to an Attribute + a Skill. The highest single die determines the degree of success, while any 1s generated provide Story Beats (SB) to the GM.
    \item \textbf{Position & Effect}: Before a roll, you set the character's \textbf{Position} (Controlled, Risky, or Desperate), which defines the stakes of failure, and their \textbf{Effect}, which describes what a clean success will achieve.
    \item \textbf{Clocks}: Visual trackers for ongoing challenges. A 4-segment clock might represent picking a complex lock, while an 8-segment \textbf{Campaign Clock} could track the rise of a villainous faction.
    \item \textbf{The Deck of Consequences}: A standard 52-card deck used to generate inspired, thematic complications when SB are spent. The suit determines the nature of the complication (Social, Physical, etc.), adding a layer of fortune and flavor.
\end{itemize}

These tools are designed to be learned quickly and used intuitively, getting out of the way so you and your players can focus on the story.

\section*{How to Use This Book}

This book is your guide to running the game.
\begin{itemize}
    \item \textbf{Chapters 1-3} cover the core principles and basic procedures.
    \item \textbf{Chapters 4-6} delve into advanced systems for conflict, travel, and long-term play.
    \item \textbf{Chapters 7-9} provide guidance for high-tier campaigns, world-building, and the specific setting of the Amaranthine Sea region.
    \item \textbf{Chapters 10-11} offer practical advice for running scenarios and a comprehensive appendix of tools and tables.
\end{itemize}

You don't need to memorize everything. Use this book as a reference. Return to it when you need clarification or inspiration. The most important chapters to internalize are those on the core philosophy (this chapter) and the basic action resolution (Chapter 2).

\begin{tcolorbox}[enhanced, sharp corners, boxrule=1pt, colback=gray!5!white, colframe=gray!75!black, title={Flavor is Free}]
\textbf{Players and GMs:} Remember that in \textbf{Fate's Edge}, \textbf{flavor is free}\index{flavor is free}!

This means you can add descriptive details, cultural elements, and atmospheric touches to your actions without spending resources or requiring a dice roll. Want your Vhasian duelist to parry with a flourish taught in the royal fencing schools? Go ahead! Want to describe the eerie silence of a Valewood ruin when searching for clues? Perfect!

Flavor enriches the narrative and makes the world feel real and lived-in. It doesn't change the mechanical outcome, but it defines the *how* and the *why*. The GM should encourage this and reciprocate by painting vivid pictures of the world.

Mechanics determine success or failure, but flavor determines the story we tell about it.
\end{tcolorbox}

\begin{tcolorbox}[enhanced, sharp corners, boxrule=1pt, colback=blue!5!white, colframe=blue!75!black, title={A Guide for Veterans: Fate's Edge in a Nutshell}]
\textbf{If you're experienced with other RPGs,} here’s a quick translation guide for how \textbf{Fate's Edge} handles common concepts:

\begin{center}
\begin{tabular}{|p{6cm}|p{6cm}|}
\hline
\textbf{Traditional RPG Concept} & \textbf{Fate's Edge Approach} \\
\hline
Ability Scores \& Skills & \textbf{Attributes} (Body, Wits, etc.) + \textbf{Skills} (Melee, Lore, etc.) form a dice pool. \\
\hline
Skill Checks & Roll Attribute+Skill dice pool. Highest die vs. Difficulty Value (DV). Any 1s give the GM Story Beats (SB). \\
\hline
Hit Points / Health & \textbf{Harm Track} for injuries. \textbf{Fatigue} for exhaustion. Consequences are narrative and mechanical. \\
\hline
Combat Rounds & Fiction-first. Actions are resolved based on narrative timing, not rigid initiative. \\
\hline
Spell Slots / Mana & Magic uses the same core system. Powerful spells may require extra time, resources, or risk generating more SB. \\
\hline
Saving Throws & Roll an appropriate Attribute+Skill combo to resist a effect (e.g., Body+Resolve to resist poison). \\
\hline
Experience \& Leveling Up & Gain XP through play. Spend XP to increase Attributes/Skills, acquire Talents, or buy Assets. Growth is player-directed. \\
\hline
\end{tabular}
\end{center}

The key difference is a consistent, unified mechanic applied across all types of challenges, focused on narrative outcomes.
\end{tcolorbox}

\section*{Begin the Journey}

Your role is a privilege and a creative challenge. You are a facilitator, a fan of the player characters, and the keeper of a world that will challenge and surprise them. Trust the rules to handle the tension, trust your players to drive the story, and trust yourself to weave it all together.

Now, take a deep breath. Shuffle the deck. Let the dice fall where they may.

It's time to guide the edge of fate.




\chapter{Core Mechanics} \label{ch:core}

In this game, every action matters. The dice don't just tell you if you succeed—they shape the story by introducing tension, risk, and consequence. Fate's Edge is designed to keep the story moving forward, even when things go wrong. This chapter covers the core resolution system and how every roll changes the narrative.

\section*{Basic Dice Mechanics} \index{dice mechanics}

When you attempt a significant action, you roll a pool of ten-sided dice (d10s). The size of your pool is determined by two factors:
$$\text{Dice Pool} = \text{Attribute} + \text{Skill}$$

\textbf{Attribute} (1--5) Broad traits like strength, wit, or charm.

\textbf{Skill} (0--5) Training or expertise in a specific area.

\subsection*{Reading the Dice}

Each die that rolls 6 or higher counts as a \textbf{Success}. Each die that rolls a 1 generates a \textbf{Story Beat (SB)}.

\begin{center}
\begin{tabular}{|c|l|}
\hline
\textbf{Die Result} & \textbf{Effect} \\
\hline
6--10 & +1 Success \\
1 & +1 Story Beat (SB) \\
2--5 & No effect \\
\hline
\end{tabular}
\end{center}

\textbf{Example:} Lyra the rogue has Agility 3 and Stealth 2. Her dice pool is 5 dice. She rolls: 6, 4, 3, 1, 6. That gives her 2 Successes and 1 Story Beat. The GM sets the Difficulty Value at 2. Lyra succeeds at sneaking past the guards, but the GM now has 1 SB to spend—perhaps the guards hear something faintly and become suspicious.

\section*{The Description Ladder} \index{description ladder}

Players can enhance their actions through detailed descriptions, which can reduce Story Beats generated by 1s:

\begin{description}
\item[Basic Action] Roll the pool as-is. All 1s remain as Story Beats.
\item[Detailed Action] A clear, descriptive flourish allows the player to re-roll one die showing 1.
\item[Intricate Action] A richly described, multi-sensory action allows the player to re-roll all dice showing 1, and add one positive narrative flourish to the scene if they succeed.
\end{description}

\textbf{Rule:} Re-rolling 1s does not remove the Story Beats already generated by those dice. If any re-rolled dice show 1 again, they generate additional SB as normal.

\section*{Difficulty Value (DV)} \index{Difficulty Value (DV)}

Before rolling, the Game Master sets a \textbf{Difficulty Value (DV)}—the target number of Successes needed.

\begin{center}
\begin{tabular}{|c|l|l|}
\hline
\textbf{DV} & \textbf{Situation} & \textbf{Example} \\
\hline
2 & Routine action, no pressure & Clear intent, modest stakes \\
3 & Pressured, mild opposition & Time pressure, partial information \\
4 & Difficult, active resistance & Hostile conditions, precise timing \\
5+ & Extreme, high stakes & Multiple constraints, dramatic failure potential \\
\hline
\end{tabular}
\end{center}

\textbf{Tip for Players:} A DV of 3 is the most common challenge. Assume that if the GM asks you to roll, there is something at stake—whether it is your safety, your resources, or your reputation.

\section*{Outcome Matrix} \index{outcome matrix}

Compare your Successes against the DV:

\begin{center}
\begin{tabular}{|l|l|}
\hline
\textbf{Outcome} & \textbf{Effect} \\
\hline
Clean Success & Goal achieved cleanly \\
Success \& Cost & Goal achieved with complication \\
Partial & Progress but with difficult choice \\
Miss & No progress; complication occurs \\
\hline
\end{tabular}
\end{center}

Let $S$ be successes and $C$ be SB generated.

\begin{description}
\item[Clean Success] $(S \geq DV \text{ and } C = 0)$ — Deliver the intent crisply.
\item[Success \& Cost] $(S \geq DV \text{ and } C > 0)$ — Grant the intent; the GM spends or banks SB to add friction.
\item[Partial] $(0 < S < DV)$ — Progress with a complication: achieve the goal with added cost, or fail forward to a different advantage.
\item[Miss] $(S = 0)$ — No direct progress. The GM spends or banks SB to introduce immediate consequences.
\end{description}

\textbf{Player-Facing Example:} A fighter swings her sword to disarm a bandit. She rolls 3 Successes against DV 2—a Clean Success. The bandit's blade clatters away. Later, the same fighter tries to kick down a reinforced door with 4 dice against DV 4. She rolls only 2 Successes. This is a Partial. She cracks the door frame, but the noise attracts attention. The story moves forward either way.

\section*{Boons} \index{Boons}

Boons are narrative currency that players can spend to influence the story in their favor. You can hold up to 5 Boons at a time.

\subsection*{Earning Boons}

You gain Boons through:
\begin{itemize}
\item \textbf{Partial Success:} When you achieve a Partial outcome (successes $<$ DV but $> 0$), you gain 1 Boon
\item \textbf{Missed Actions:} When you miss entirely (0 successes), you gain 2 Boons
\item \textbf{Bond-Driven Actions:} When you take an Intricate action that meaningfully engages a character bond, you may gain 1 Boon (once per bond per session)
\item \textbf{GM Award:} The GM may award Boons for creative solutions, spotlighting bonds, or meaningful sacrifices
\end{itemize}

\subsection*{Requirements for Action Awards}

Boons from Partial/miss outcomes are awarded only if:
\begin{enumerate}
\item Procedure was followed correctly (intent declared, DV set, roll resolved)
\item Stakes were clearly stated (what changes on success/failure)
\item Consequence actually occurs (GM spends or banks SB, applies condition, or advances thread)
\end{enumerate}

\textbf{Important Note:} Rehearsal/null-risk probes and repeated identical attempts in the same scene do not award Boons. If it feels like an obvious fishing attempt, don't award a Boon.

\subsection*{Spending Boons}

You can spend Boons to:
\begin{itemize}
\item Re-roll a single die in a pool
\item Activate an on-screen Asset
\item Power a Rite or magical ability
\item Improve Position by 1 step
\item Convert to XP (2 Boons = 1 XP, once per session during downtime, max 2 XP via conversion per session)
\end{itemize}

\subsection*{Carryover Limits}

At the end of each scene, reduce held Boons to a maximum of 2. Excess Boons are lost. This encourages you to spend them rather than hoard.

\begin{tcolorbox}[colback=gray!5!white, colframe=gray!75!black, title=Why This Matters, fonttitle=\bfseries]
The system rewards engagement with risk. Even when you don't fully succeed, you gain resources to help push the story forward. Failures become opportunities, and partial successes still offer chances to turn the tide.
\end{tcolorbox}

\section*{Story Beats (SB)} \index{Story Beats (SB)}

Story Beats are narrative tools the Game Master uses to introduce twists and tension. They keep the story alive with complications and surprises.

\subsection*{What SB Can Do}

The GM may spend SB to:
\begin{itemize}
\item Introduce new threats or complications
\item Drain resources (time, gear, position)
\item Reveal hidden dangers
\item Cause collateral damage
\end{itemize}

\subsection*{SB Spend Examples}

\begin{center}
\begin{tabular}{|c|l|}
\hline
\textbf{SB Cost} & \textbf{Effect} \\
\hline
1 SB & Minor complication, noise, trace \\
2 SB & Moderate setback, alarm raised \\
3 SB & Serious trouble, reinforcements arrive \\
4+ SB & Major turn, scene shifts dramatically \\
\hline
\end{tabular}
\end{center}

\textbf{Player Advice:} Don't fear Story Beats—they're not punishment. They are fuel for drama, ensuring the spotlight never dims.

\section*{Harm and Fatigue} \index{Harm and Fatigue}

Physical injury and exhaustion are tracked through two systems:

\subsection*{Fatigue Track}

Each character has a Fatigue Track equal to their Body attribute. Mark Fatigue for:
\begin{itemize}
\item Physical exertion
\item Magical strain
\item Travel stress
\item Mental pressure
\end{itemize}

\subsection*{When your Fatigue Track fills:}
\begin{enumerate}
\item Increase your Harm level by one step
\item Clear all Fatigue marks
\end{enumerate}

This can happen multiple times in a scene.

\subsection*{Harm Levels}

\begin{center}
\begin{tabular}{|l|l|}
\hline
\textbf{Harm Level} & \textbf{Effects} \\
\hline
Harm 1 & -1 die on related actions \\
Harm 2 & -1 die on most actions until treated \\
Harm 3 & Incapacitated or dying \\
\hline
\end{tabular}
\end{center}

\subsection*{Recovering Fatigue}
\begin{itemize}
\item \textbf{Short Rest} — Remove 2 Fatigue with food/water
\item \textbf{Full Night} — Remove all Fatigue
\end{itemize}

\subsection*{Recovering Harm}
\begin{itemize}
\item \textbf{Minor treatment} — Downgrade Harm with time/rest
\item \textbf{Proper medical care} — Remove Harm levels
\item \textbf{Extended recovery} — Heal severe injuries
\end{itemize}

\textbf{Example:} Jorin the mercenary takes a sword cut (Harm 1). He suffers -1 die to physical actions until treated. After binding the wound and resting, the Harm fades.

\section*{Assistance} \index{Assistance}

Characters can help each other. One helper per action may provide assistance by spending 1 Boon or 1 Stress, adding +1 die to the primary actor's roll. Maximum +3 dice from assists.

\textbf{Example:} Two thieves cooperate to pick a complex lock. The lead thief has Dexterity 3 + Tools 2 = 5 dice. The helper spends 1 Boon to add 1 die, making 6. Cooperation often turns failure into tense success.

\section*{Weapons \& Armor} \index{Weapons}

\subsection*{Weapons by Weight Class}

\begin{itemize}
\item \textbf{Light (4 XP)} — fast, concealable.
\item \textbf{Medium (8 XP)} — balanced, battlefield standard.
\item \textbf{Heavy (12 XP)} — punishing, slow.
\end{itemize}

\subsubsection*{Melee Modifiers}

\begin{center}
\begin{tabular}{|l|c|c|l|}
\hline
\textbf{Weight} & \textbf{Close} & \textbf{Near} & \textbf{Notes} \\
\hline
Light & +2d & +1d & Quick, tight quarters \\
Medium & +1d & +2d & Set 1/scene or –1d first attack \\
Heavy & –1d & +3d & Set 1/scene or –2d first attack \\
\hline
\end{tabular}
\end{center}

\subsubsection*{Ranged \& Tempo}

\begin{center}
\begin{tabular}{|l|l|l|l|l|}
\hline
\textbf{Weight} & \textbf{Tempo} & \textbf{Close} & \textbf{Near} & \textbf{Far} \\
\hline
Light (4 XP) & Fast & Risky & +1d & — \\
Medium (8 XP) & Standard & Desperate & +2d & +1d \\
Heavy (12 XP) & Slow & Desperate & +1d & +3d \\
\hline
\end{tabular}
\end{center}

\textbf{Tempo:} 
\begin{itemize}
\item Fast = Move+Shoot. 
\item Standard = Move or Shoot, Aim +1d/Effect. 
\item Slow = Set/Brace, full reload, cannot Move+Shoot.
\end{itemize}

\subsection*{Weapon Tags (Optional, +4 XP each, max 2)}

Reach, Close, Accurate, Brutal, Hook, Concealable, Quickdraw, Two-Handed, Off-Hand.

\subsection*{Shields (Optional)}

\begin{center}
\begin{tabular}{|l|r|l|l|}
\hline
\textbf{Shield} & \textbf{XP} & \textbf{Benefit} & \textbf{Tradeoff} \\
\hline
Buckler & 4 & +1d Defend vs melee or +1 DV & Off-hand \\
Heater & 8 & +1d Defend; 1 Harm→Fatigue & –1d Ranged \\
Pavise & 12 & Plant: heavy cover cone & Bulky, immobile \\
\hline
\end{tabular}
\end{center}

\subsection*{Armor}

\begin{center}
\begin{tabular}{|l|r|l|l|}
\hline
\textbf{Armor} & \textbf{XP} & \textbf{Conversion} & \textbf{Penalty} \\
\hline
Light & 4 & 1 Harm→1 Fatigue & — \\
Medium & 8 & 2 Harm→1 Fatigue & –1d physical \\
Heavy & 12 & 3 Harm→2 Fatigue & –2d physical, no sprint \\
\hline
\end{tabular}
\end{center}

\textbf{Notes:} Conversion applies per Harm instance before Fatigue is marked. You may still Resist first.

\subsection*{Condition \& Upkeep}

\begin{description}
\item[Neglected] Weapons –1d; Armor: conversion worsens by 1 step.
\item[Compromised] Weapons –1d first attack/round; Armor: no conversion.
\end{description}

\textbf{Fix:} Short Rest/tools remove Neglected. A scene/Smith removes Compromised.

\section*{Ranged Options} \index{Ranged Options}

\begin{itemize}
\item \textbf{Aim:} +1d or +1 Effect.
\item \textbf{Volley:} Extra ammo +1d (max +2).
\item \textbf{Suppress:} Zone fire, foes –1d/Limited Effect.
\item \textbf{Overwatch:} Ready a Risky shot on trigger.
\end{itemize}

\section*{Assets and Allies} \index{Assets}

Your character's resources, contacts, or gear—called Assets—can tilt the odds in your favor.

\begin{description}
\item[On-Screen Assets] — Companions, hirelings, or allies who stand beside you in danger
\item[Off-Screen Assets] — Taverns, estates, titles, or networks of informants
\item[Activation] — Spend 1 Boon to activate an on-screen Asset
\end{description}

\textbf{Narrative Use:} Assets are more than bonuses—they're hooks for roleplay. A friendly tavernkeeper, a noble's signet, or a trusty horse might tip the balance at the perfect moment.

\section*{Game Structure} \index{Game Structure}

\subsection*{Time Scales}

\begin{description}
\item[Moment] A heartbeat, a single action
\item[Some Time] A few minutes, a short activity
\item[Significant Time] Hours, extended effort
\item[Days] Large-scale endeavors
\end{description}

\subsection*{Game Units}

\begin{description}
\item[Scene] Basic narrative unit, covers specific conflict
\item[Player Turn] Individual action within a scene
\item[Round] Simultaneous actions in combat
\item[Session] One game session (3–6 hours)
\item[Campaign] Entire story arc
\end{description}

\textbf{Player Perspective:} Think in scenes, not minutes. Every scene is a chance to shine. Every session builds toward the long arc of your campaign.

\section*{Action Resolution Steps} \index{Action Resolution Steps}

\begin{enumerate}
\item Describe your intent and method
\item Build dice pool: Attribute + Skill (+ gear, assists)
\item Roll d10s, count Successes and Story Beats
\item Compare Successes to DV
\item Apply outcome from matrix
\item Game Master spends SB if applicable
\item Earn Boons for failure.
\end{enumerate}

\begin{tcolorbox}[colback=gray!5!white, colframe=gray!75!black, title=Quick Reference, fonttitle=\bfseries]
\textbf{Dice Pool:} Attribute + Skill d10s\\
\textbf{Success:} 6+ on each die\\
\textbf{Setback:} 1 on any die gives SB to GM\\
\textbf{DV:} 2 (easy) to 5+ (extreme)\\
\textbf{Harm:} 3-level system with penalties\\
\textbf{Boons:} 2 on miss, 1 on partial
\end{tcolorbox}

\section*{Narrative Suggestions} \index{Narrative Suggestions}

\textbf{Collaborative Scene Framing:} Players may suggest scene elements (weather, NPC reactions, environmental details) that fit the established fiction, with GM approval.

\textbf{Intent-Driven Resolution:} For non-combat actions where success is reasonably assured, the GM may ask players to describe \emph{how} they accomplish their goal rather than rolling dice.

\textbf{Flashback Declarations:} Players can declare a flashback scene to establish that something happened in the past (acquiring an item, making a connection, learning information) by spending 1 Boon and describing the scene.

\textbf{Descriptive Assistance:} Players can assist each other by providing vivid, helpful descriptions of the action, granting a +1 die bonus to the primary actor's roll.

\textbf{Proactive Storytelling:} Players can suggest minor favorable details about their character's circumstances by:
\begin{itemize}
\item Introducing a minor NPC who provides useful information or assistance
\item Establishing that they have a useful item on hand (within reason)
\item Creating a favorable environmental detail
\end{itemize}

These suggestions are subject to GM approval and should enhance rather than overshadow the main narrative.

\chapter{Character Advancement} \label{ch:advancement}

In this game, growth isn't just about numbers—it's about defining who your character becomes. Advancement through Experience Points (XP) lets you shape your capabilities, influence, and legacy in the world. Every choice you make with XP is a statement about your character's priorities and the mark they leave behind.

\section*{Earning Experience Points} \index{Experience Points}

XP represents learning through action. You earn it by engaging meaningfully with the world and its challenges, whether that's by triumph, failure, or bold experimentation.

\subsection*{Session Breakdown}

At the end of each session, the Game Master awards XP based on:
\begin{itemize}
\item \textbf{Base Participation:} +2 XP for attending and contributing
\item \textbf{Major Objectives:} +2–4 XP for completing significant story goals
\item \textbf{Discoveries:} +1–2 XP for uncovering important lore, locations, or secrets
\item \textbf{Difficult Choices:} +1–2 XP for making hard moral or strategic decisions
\item \textbf{Story Engagement:} +1–3 XP for embracing complications and narrative twists
\item \textbf{Personal Goals:} +1–2 XP for pursuing your character's individual storylines
\end{itemize}

\textbf{Example:} At the end of a session, the party rogue earns +2 XP for participation, +2 XP for helping the group retrieve an artifact, and +1 XP for pushing a personal rivalry subplot—5 XP total.

\subsection*{Game Pace Options}

The GM can adjust advancement speed to match the campaign tone:

\begin{center}
\begin{tabular}{|l|l|l|}
\hline
\textbf{Mode} & \textbf{XP/Session} & \textbf{Tone} \\
\hline
Gritty & 4–6 XP & Hard choices, slow growth \\
Standard & 6–10 XP & Balanced progression \\
Epic & 10–14 XP & Heroic, rapid development \\
\hline
\end{tabular}
\end{center}

\textbf{Player Tip:} If you want a sweeping, mythic tale, suggest an Epic pace. For a long, hard road where each gain feels hard-earned, lean into Gritty.

\subsection*{Arc Completion Bonus}

When you finish a major story arc (typically 3–6 sessions), everyone receives +8–12 XP. One player may earn an additional +2 XP for a particularly memorable contribution. This celebrates the story's milestones, not just individual rolls.

\section*{Spending Experience Points} \index{Experience Points!spending}

XP is your currency for growth. You can invest it in three broad areas, each representing a different approach to becoming more capable.

\subsection*{1. Personal Improvement}

Invest in your core capabilities—what you can do yourself.

\textbf{Attributes} Cost = New Rating × 3 XP
\begin{itemize}
\item Raising Body from 2 to 3 costs 3 × 3 = 9 XP
\item Raising Spirit from 4 to 5 costs 5 × 3 = 15 XP
\item Requires downtime equal to new rating in days
\end{itemize}

\textbf{Skills} Cost = New Level × 2 XP
\begin{itemize}
\item Improving Lore from 1 to 2 costs 2 × 2 = 4 XP
\item Advancing Melee from 3 to 4 costs 4 × 2 = 8 XP
\item Requires downtime equal to new level in days
\end{itemize}

\textbf{Example:} Kara wants to improve her Swordsmanship from 2 to 3. She saves 6 XP and spends three in-game days training with her mentor. This creates roleplay hooks and a sense of lived growth.

\subsection*{2. Resources and Influence}

Build your worldly presence—what you can command.

\begin{description}
\item[Minor Resource (4 XP, 1 week)] — Small shop, minor contact network, basic workshop. Provides small but reliable benefits. Example: A trusted informant who gathers rumors
\item[Standard Resource (8 XP, 2 weeks)] — Decent-sized business, skilled followers, specialized equipment. Significant benefits with some upkeep. Example: A smuggling operation with two boats
\item[Major Resource (12 XP, 1 month)] — Large enterprise, elite team, rare capabilities. Powerful advantages with substantial upkeep. Example: A trading company with international contacts
\end{description}

\textbf{Player Tip:} Resources expand the story into new directions. A spy network creates intrigue; a workshop sparks invention; a guild hall cements influence.

\subsection*{3. Special Abilities}

Develop unique capabilities that set you apart.

\begin{description}
\item[General Abilities (Cost varies)] — Universal benefits like improved recovery, bonus dice in specific situations, or unique combat techniques. Typically cost 4–8 XP. Example: "Quick Recovery" - heal 1 additional Harm when resting
\item[Cultural Abilities (Cost varies)] — Heritage-based skills tied to your character's background. Often require specific fictional positioning. Example: "Stone Sense" (dwarven) - intuitive understanding of stonework
\item[Advanced Abilities (12+ XP)] — Powerful capstone features available at higher tiers. Often have significant narrative weight and requirements. Example: "Master Diplomat" - can reroll failed social checks once per session
\end{description}

\textbf{Example:} A veteran bard invests in "Silver Tongue" (6 XP), allowing them to sway hostile crowds once per session. This becomes their defining trick in tense negotiations.

\section*{Character Development Paths} \index{Character Development Paths}

Your spending choices define your character's growth direction. Consider these archetypal paths:

\subsection*{The Specialist}

70–90\% personal improvement, 0–10\% resources, 0–20\% abilities
\begin{itemize}
\item \textbf{Strengths:} Exceptional individual capability, reliable in spotlight moments
\item \textbf{Weaknesses:} Limited influence, vulnerable to being isolated
\item \textbf{Best for:} Solo operatives, elite warriors, master artisans
\item \textbf{Example:} A duelist who invests heavily in combat skills and physical attributes
\end{itemize}

\subsection*{The Leader}

50–65\% personal, 15–25\% resources, 15–25\% abilities
\begin{itemize}
\item \textbf{Strengths:} Well-rounded, can handle diverse challenges, good support
\item \textbf{Weaknesses:} Jack-of-all-trades, not exceptional in any area
\item \textbf{Best for:} Party faces, field commanders, investigators
\item \textbf{Example:} A merchant-prince with decent combat skills, good social abilities, and a network of contacts
\end{itemize}

\subsection*{The Mastermind}

25–40\% personal, 35–55\% resources, 20–40\% abilities
\begin{itemize}
\item \textbf{Strengths:} Extensive influence, can solve problems indirectly, strategic power
\item \textbf{Weaknesses:} Personally vulnerable, complex upkeep, domino-effect risks
\item \textbf{Best for:} Spymasters, crime lords, wealthy patrons
\item \textbf{Example:} An information broker with modest personal skills but an extensive spy network
\end{itemize}

\begin{tcolorbox}[colback=gray!5!white, colframe=gray!75!black, title=Player Note, fonttitle=\bfseries]
These are not rigid templates. Mix and match to discover unique growth arcs.
\end{tcolorbox}

\section*{Training and Development Time} \index{Training Time}

Most improvements require downtime to reflect the effort of learning and integration.

\subsection*{Standard Time Requirements}

\begin{itemize}
\item Attribute increase: New rating in days
\item Skill improvement: New level in days
\item Resource acquisition: 1 week to 1 month depending on scope
\item Ability learning: Typically 3–10 days
\end{itemize}

\subsection*{Accelerated Development}

You can attempt to learn things more quickly, but this carries risks:
\begin{itemize}
\item The GM creates a Risk Clock with 4 segments
\item If the clock fills during rushed training, the new capability has flaws:
\begin{itemize}
\item Attribute/Skill: -1 die penalty until you spend proper downtime
\item Resource: Loyalty problems or functional limitations
\item Ability: Unreliable or with unintended side effects
\end{itemize}
\end{itemize}

\textbf{Example:} The wizard crams advanced spellwork into a frantic three days. She gains the ability, but her Risk Clock fills—her spells now sputter unpredictably until she retrains.

\section*{Character Progression Tiers} \index{Character Progression Tiers}

As you accumulate XP and capabilities, you advance through tiers that represent your growing reputation and influence.

\subsection*{Tier I: Novice (0–40 XP)}

\begin{itemize}
\item Learning the ropes, establishing yourself
\item Local reputation, modest capabilities
\item Typical assets: Basic equipment, a few contacts
\end{itemize}

\subsection*{Tier II: Experienced (41–90 XP)}

\begin{itemize}
\item Proven capability, recognized skills
\item Regional reputation, reliable in your specialty
\item Typical assets: Skilled followers, specialized equipment
\end{itemize}

\subsection*{Tier III: Veteran (91–150 XP)}

\begin{itemize}
\item Master of your craft, significant influence
\item National reputation, can handle major challenges
\item Typical assets: Multiple operations, elite teams
\end{itemize}

\subsection*{Tier IV: Elite (151–220 XP)}

\begin{itemize}
\item Exceptional capability, major influence
\item International reputation, shapes events
\item Typical assets: Organizations, unique capabilities
\end{itemize}

\subsection*{Tier V: Master (221+ XP)}

\begin{itemize}
\item Legendary status, world-changing influence
\item Historical reputation, defines eras
\item Typical assets: Nations, legendary artifacts
\end{itemize}

\section*{Managing Allies and Followers} \index{Followers}

Characters who work with you require maintenance and carry risks.

\subsection*{Acquisition Costs}

\begin{itemize}
\item Skilled helper: Capability rating squared in XP
\item Example: A capability 3 scout costs 9 XP
\end{itemize}

\subsection*{Upkeep Requirements}

\begin{itemize}
\item Each downtime period, spend XP equal to their capability rating
\item Alternative: Dedicate a scene to maintaining the relationship
\end{itemize}

\section*{Strategic Advancement Considerations} \index{Advancement Strategy}

\subsection*{Early Game (Tiers I–II)}

Focus on survival and establishing your niche:
\begin{itemize}
\item Invest in core competencies first
\item Build a small but reliable support network
\end{itemize}

\subsection*{Mid Game (Tier III)}

Expand your influence and specialize:
\begin{itemize}
\item Develop your signature capabilities
\item Build substantial resources
\end{itemize}

\subsection*{Late Game (Tiers IV–V)}

Shape the world around you:
\begin{itemize}
\item Pursue advanced abilities
\item Build organizations or movements
\item Leave a legacy
\end{itemize}

\section*{Advancement Philosophy} \index{Advancement Philosophy}

Remember that advancement serves the story. The best choices:
\begin{itemize}
\item Reflect your character's experiences and growth
\item Create interesting new capabilities and complications
\item Enhance the group's collective abilities
\end{itemize}

\begin{tcolorbox}[colback=gray!5!white, colframe=gray!75!black, title=Final Thought, fonttitle=\bfseries]
Every XP spent changes not just your character sheet, but your character's story. Choose investments that make your hero more interesting to play and watch evolve.
\end{tcolorbox}

\begin{tcolorbox}[colback=gray!5!white, colframe=gray!75!black, title=XP Planning Guide, fonttitle=\bfseries]
\textbf{Early Tier Priorities:}
\begin{itemize}
\item Core attribute to 3 (9 XP)
\item Key skills to 2–3 (4–8 XP each)
\item 1–2 minor resources (8 XP total)
\end{itemize}

\textbf{Mid Tier Expansion:}
\begin{itemize}
\item Attributes to 4 (12 XP)
\item Specialization skills to 4 (8 XP)
\item Standard resources (8 XP each)
\item Cultural abilities (6–10 XP)
\end{itemize}

\textbf{Late Tier Mastery:}
\begin{itemize}
\item Capstone abilities (12+ XP)
\item Major resources (12 XP)
\item Legacy projects
\end{itemize}
\end{tcolorbox}

\section*{Narrative-Heavy Advancement Options} \index{Narrative-Heavy Gameplay!Advancement}

For groups that prefer strong narrative focus in advancement, consider these optional approaches:

\textbf{Story-Driven Milestones:} Instead of tracking XP numerically, the GM can award advancement when characters reach significant story milestones. "You've trained with the master for months—you've improved your skill."

\textbf{Experience Through Reflection:} Players can spend downtime scenes reflecting on past experiences to earn XP. A meaningful flashback or character moment can justify growth without tracking specific points.

\textbf{Collaborative Advancement:} The group can discuss and agree on advancement choices, ensuring everyone's growth supports the overall story direction.

\textbf{Narrative Justification Focus:} When spending XP, players should explain how their character gained this capability through in-game experiences, creating richer backstory and continuity.

\chapter{Magic and Special Abilities} \label{ch:magic}

Magic in this game is powerful but dangerous—a negotiation with reality itself that always carries risks. This chapter covers the core magical systems: standard spellcasting, ritual magic, and special pact-based abilities. Throughout, look for examples and player-facing tips to keep the fiction front and center.

\section*{The Nature of Magic} \index{magic!nature}

Magic is not a safe tool but a dangerous force:
\begin{itemize}
\item \textbf{Powerful:} Can reshape battles, stories, or even the world
\item \textbf{Risky:} Every use generates Story Beats (SB) that manifest as backlash
\item \textbf{Thematic:} Effects and consequences align with the type of magic used
\item \textbf{Volatile:} Never fully predictable or controllable
\item \textbf{Narrative:} Casting is always a significant story moment
\end{itemize}

\begin{tcolorbox}[colback=gray!5!white, colframe=gray!75!black, title=Table Vignette, fonttitle=\bfseries]
"I can hold the avalanche," says Mira, fingers trembling. "But something will answer." The party nods—risk accepted, stakes clear.
\end{tcolorbox}

\section*{Basic Spellcasting} \index{magic!spellcasting}

All spellcasting follows the standard action resolution system but with additional considerations for magical effects.

\subsection*{The Casting Process}

\begin{enumerate}
\item Declare Intent: What you want the magic to achieve
\item Choose Approach: Which magical skill and method you'll use
\item Set Position: Controlled, Risky, or Desperate based on circumstances
\item Roll: Attribute + Magical Skill
\item Resolve: Apply outcomes with magical consequences
\end{enumerate}

\subsection*{Magical Skills}

Common magical skills include:
\begin{itemize}
\item \textbf{Arcana:} General magical knowledge and theory
\item \textbf{Elemental Magic:} Fire, water, earth, air manipulation
\item \textbf{Spiritual Magic:} Communing with spirits, divine magic
\item \textbf{Mental Magic:} Telepathy, illusion, mind affecting
\item \textbf{Healing Magic:} Restoration, purification, life magic
\end{itemize}

\textbf{Player Tip:} State a clear intent and a vivid method. The more concrete the fiction, the easier it is to set fair DV and meaningful consequences.

\section*{The Casting Loop} \index{magic!casting loop}

For more significant magical effects, use the structured Casting Loop requiring two actions.

\subsection*{Phase 1: Weave}

Shape the magical effect:
\begin{itemize}
\item Player builds dice pool and rolls
\item On success, they stabilize the spell's form
\item Any 1 rolled may cause narrative backlash related to the Element
\end{itemize}

\subsection*{Phase 2: Cast}

Channel the effect into the world:
\begin{itemize}
\item A second roll channels the effect
\item Backlash: Any 1 rolled may cause narrative backlash related to the Element
\end{itemize}

\textbf{Designer Note:} The Casting Loop requires the Caster's Gift talent (2 XP) and creates spotlight tension: describe effect now, risk Backlash on each roll.

\section*{Backlash Severity} \index{magic!backlash}

\begin{center}
\begin{tabular}{|l|l|}
\hline
\textbf{Roll Result} & \textbf{Backlash Trigger} \\
\hline
Partial/Miss & Minor backlash (choose one) \\
Miss & Major backlash (choose two) \\
Hit with two or more 1s & Minor backlash alongside success \\
\hline
\end{tabular}
\end{center}

\section*{Magical Arts and Traditions} \index{magic!traditions}

Different cultures and traditions approach magic differently.

\subsection*{Elemental Magic}

Manipulation of natural forces:
\begin{description}
\item[Fire Magic:] Heat, light, transformation, destruction
\item[Water Magic:] Flow, healing, divination, adaptation
\item[Earth Magic:] Stability, protection, growth, strength
\item[Air Magic:] Movement, communication, freedom, change
\end{description}

\subsection*{Spiritual Magic}

Interaction with intangible forces:
\begin{description}
\item[Divine Magic:] Power from gods or higher powers
\item[Spirit Magic:] Communing with nature spirits or ancestors
\item[Necromancy:] Interaction with death and the departed
\item[Protection Magic:] Wards, blessings, purification
\end{description}

\subsection*{Mental Magic}

Affecting minds and perceptions:
\begin{description}
\item[Illusion:] Creating false perceptions and images
\item[Telepathy:] Mind reading and communication
\item[Enchantment:] Influencing thoughts and emotions
\item[Divination:] Gaining knowledge through supernatural means
\end{description}

\begin{tcolorbox}[colback=gray!5!white, colframe=gray!75!black, title=Vignette, fonttitle=\bfseries]
The candles lean toward the oracle's breath. "Ask," she whispers, "but truth is sharp."
\end{tcolorbox}

\section*{Ritual Magic} \index{magic!ritual}

Rituals take Significant Time (typically 10-30 minutes) for powerful effects.

\subsection*{Ritual Requirements}

\begin{itemize}
\item Time: Significant Time (typically 10-30 minutes)
\item Preparation: Specific materials, locations, or conditions
\item Focus: Undisturbed concentration and coordination
\end{itemize}

\subsection*{Ritual Procedure}

\begin{enumerate}
\item Preparation: Gather components, prepare space, focus intent
\item Invocation: Perform the Rite as a ritual
\item Completion: Effect manifests, always marks +1 Obligation
\end{enumerate}

\subsection*{Ritual Benefits and Risks}

\begin{description}
\item[Benefits:] Safe casting, no Push It option
\item[Risks:] Time investment, Obligation cost, environmental requirements
\end{description}

\section*{Rites and Pact Magic} \index{magic!rites}

Rites are precise magical effects gained through pacts with powerful entities. There are two main paths to accessing Rites:

\subsection*{The Runekeeper (Rites Path)}

\begin{itemize}
\item Requires Patron + Thiasos (Familiar) + Codex (4 XP)
\item Accesses that Patron's full Rite list
\item Structured, powerful, but accrues Obligation
\item Can Push Rites once per scene for +1 Obligation
\end{itemize}

\subsection*{The Invoker (Symbol Path)}

\begin{itemize}
\item Requires one or more Patron's Symbols (4 XP each)
\item Accesses ritual invocation of Patron's Rites
\item Safe but slow—requires Significant Time
\item Can Crack the Seal for instant cast at steep Obligation cost (+2/+3)
\end{itemize}

\subsection*{Using Rites}

\begin{enumerate}
\item Invocation: Invoke a Rite requires 1 Action
\item Obligation: Each Rite used marks Obligation on its clock
\item Effect: The Rite's specific effect manifests
\end{enumerate}

\subsection*{Rite Invocation via Symbol}

\begin{itemize}
\item Time. Invoking a Rite via Symbol takes DV + 1 rounds.
\item Obligation. On completion, mark +1 Obligation (in addition to any listed Rite costs, if applicable).
\item No Push. Invoker Rites cannot use Push It benefits.
\item Symbol Display. The Symbol must remain visible throughout the invocation.
\item Materials. Symbols replace any Thaisos and Codex requirements.
\end{itemize}

\section*{Obligation Capacity} \index{magic!obligation}

A character's Obligation Capacity equals Spirit + Presence. Track total Obligation segments across all Patrons (or Symbols, for Invokers).

\begin{itemize}
\item Exceeding Capacity: For each segment above Capacity, mark 1 Fatigue. The character cannot Invoke Rites or perform rituals until Obligation is reduced below Capacity.
\item Resolution: Reduce Obligation through Downtime service, Patron tasks, ritual cleansing, or story resolution.
\end{itemize}

\textbf{Example:} Spirit 2 + Presence 3 = Capacity 5. 6 segments → Fatigue 1. 7 segments → Fatigue 2. 10 segments → Harm 1. 11 segments → Harm 2.

\subsection*{Obligation Management}

Your debt to Patrons must be managed:
\begin{itemize}
\item Service: Perform tasks fitting your Patron's nature
\item Offerings: Provide sacrifices or tributes
\item Propagation: Spread your Patron's influence or beliefs
\item Downtime: Clear through fitting service during downtime
\end{itemize}

\subsection*{Obligation Levels}

\begin{center}
\begin{tabular}{|l|l|}
\hline
\textbf{Segments} & \textbf{Consequences} \\
\hline
1–2 & Minor attention, subtle signs \\
3–5 & Noticeable influence, regular demands \\
6–8 & Significant control, major tasks required \\
9+ & Dominant influence, potentially dangerous \\
\hline
\end{tabular}
\end{center}

\begin{tcolorbox}[colback=gray!5!white, colframe=gray!75!black, title=Vignette, fonttitle=\bfseries]
At the crossroads, Ash lays iron nails and salt. The wind shifts. Somewhere, something smiles.
\end{tcolorbox}

\section*{Special Magical Abilities} \index{magic!special abilities}

Some characters develop unique magical capabilities through experience or heritage.

\subsection*{Cultural Magical Traditions}

\begin{description}
\item[Dwarven Stone-Sense:] Intuitive understanding of earth and stone
\item[Elven Memory-Weaving:] Accessing and manipulating ancestral knowledge
\item[Human Versatility:] Adaptable magical approaches from various traditions
\item[Nomadic Spirit-Walking:] Journeying between physical and spiritual realms
\end{description}

\subsection*{Advanced Magical Techniques}

At higher levels, casters can:
\begin{itemize}
\item Specialize: Focus on specific magical traditions
\item Innovate: Create new spells or techniques
\item Teach: Instruct others in magical arts
\item Research: Discover lost or forbidden knowledge
\end{itemize}

\section*{Magical Backlash Examples} \index{magic!backlash examples}

\subsection*{Elemental Backlash}

\begin{description}
\item[Fire:] Burns, flares; vs. Water: slick, sputter, dim
\item[Water:] Slippery tide, slow gear; vs. Fire: smoke, shorted gear
\item[Earth:] Slips, binds, encumbrance; vs. Air: sound carries, exposure
\item[Air:] Scatter, misheard words; vs. Earth: stuck, dust choke
\end{description}

\subsection*{Conceptual Backlash}

\begin{description}
\item[Fate:] Options close, only-one-way; vs. Luck: mischance hits ally
\item[Life:] Growth surge, vines tether; vs. Death/Dreams: numbness, sleep-tug
\item[Luck:] Odds flip; vs. Fate: harsher fixed outcome
\item[Death/Dreams:] Whispers, chill; vs. Life: pain returns, rot
\end{description}

\section*{Magical Item Creation} \index{magic!item creation}

Creating permanent magical items is a complex process.

\subsection*{Creation Requirements}

\begin{itemize}
\item Knowledge: Understanding of the desired effect
\item Materials: Appropriate components with magical properties
\item Time: Significant investment of time and effort
\item Skill: High level of magical and craft skills
\item Facilities: Proper workspace with necessary tools
\end{itemize}

\subsection*{Creation Process}

\begin{enumerate}
\item Design: Plan the item's properties and limitations
\item Gathering: Acquire necessary materials and components
\item Crafting: Physical creation of the item base
\item Enchantment: Magical infusion of the desired properties
\item Finishing: Final adjustments and testing
\end{enumerate}

\subsection*{Item Limitations}

\begin{itemize}
\item Charges: Limited uses before needing recharge
\item Attunement: Required bonding with the user
\item Maintenance: Regular upkeep to preserve functionality
\item Drawbacks: Negative side effects or requirements
\end{itemize}

\section*{Magic in Social Situations} \index{magic!social}

Using magic in social contexts has special considerations.

\subsection*{Social Spellcasting}

\begin{itemize}
\item Discretion: Avoiding detection while casting
\item Consent: Ethical considerations of affecting others' minds
\item Reactions: How different cultures view magical influence
\item Laws: Legal restrictions on magical use in society
\end{itemize}

\subsection*{Social Backlash}

Magical social failures can cause:
\begin{itemize}
\item Distrust: People becoming wary of the caster
\item Resistance: Developing immunity or countermeasures
\item Reputation: Becoming known as a manipulator
\item Legal: Facing consequences from authorities
\end{itemize}

\section*{Learning and Improving Magic} \index{magic!learning}

Magical ability grows through study and practice.

\subsection*{Skill Advancement}

\begin{itemize}
\item Study: Researching magical theory and techniques
\item Practice: Regular casting to improve control
\item Experimentation: Trying new approaches and combinations
\item Instruction: Learning from more experienced casters
\end{itemize}

\subsection*{Advanced Magical Development}

At higher levels, casters can:
\begin{itemize}
\item Specialize: Focus on specific magical traditions
\item Innovate: Create new spells or techniques
\item Teach: Instruct others in magical arts
\item Research: Discover lost or forbidden knowledge
\end{itemize}

\section*{Magical Safety and Ethics} \index{magic!ethics}

Responsible magical practice involves understanding risks and consequences.

\subsection*{Safety Considerations}

\begin{itemize}
\item Containment: Preventing unintended spread of effects
\item Stability: Ensuring magical effects remain controlled
\item Fail-safes: Planning for when magic goes wrong
\item Recovery: Procedures for dealing with backlash
\end{itemize}

\subsection*{Ethical Guidelines}

\begin{itemize}
\item Consent: Respecting others' autonomy regarding magic
\item Transparency: Being honest about magical capabilities
\item Restraint: Using magic judiciously and appropriately
\item Responsibility: Accepting consequences of magical actions
\end{itemize}

\begin{tcolorbox}[colback=gray!5!white, colframe=gray!75!black, title=Magic Quick Reference, fonttitle=\bfseries]
\textbf{Casting (Freeform):}
\begin{itemize}
\item Requires Talent: Caster's Gift (2 XP)
\item Weave \& Cast: Two action effect using the Eight Elements
\item Backlash: Any 1 rolled may cause narrative backlash
\end{itemize}

\textbf{Backlash Severity:}
\begin{itemize}
\item On Partial/Miss: Pick 1-2 consequences flavored by Element
\item Color consequences by Element (fire burns, fate twists, etc.)
\end{itemize}

\textbf{Rites System:}
\begin{itemize}
\item Invoke: 1 action effect
\item Obligation: Mark segments on clock
\item Push It: +1 Obligation for +1 step effect
\end{itemize}

\textbf{Invoker Path:}
\begin{itemize}
\item Symbols (4 XP each) grant ritual access
\item Rituals: Significant Time, always +1 Obligation
\item Crack the Seal: Instant cast (+2/+3 Obligation)
\end{itemize}

\textbf{Safety:} Every roll changes the story. Success without risk is rare.
\end{tcolorbox}

\section*{Practical Magic Examples} \index{magic!examples}

\subsection*{Fire Cast, Partial}

You Weave flame to blind a squad (DV 3). Partial with two 1s. GM spends SB to Position -1 (flare blinds you too) and colors backlash as singed lashes; patrol is alerted (Exposure).

\subsection*{Runekeeper Push and Debt}

You Invoke Circle of Denial [WARD] and Push It to harden the ring. Mark +1 Obligation for the Rite plus +1 for the push. When a demon tests the ring, use [WARD] vs Cap; on its Hit, add +DV to its Leash.

\subsection*{Crack the Seal Under Fire}

You present Ikasha's Symbol and Crack the Seal to lay an instant shadow lane. Symbol → Compromised; mark +2 Obligation. GM immediately spends 1 SB to dim all lights (panic), then the lane forms. During downtime, you restore the Symbol (Arcana DV 3): a shaky hit leaves it Neglected until you perform the full rite of cleaning.

\begin{tcolorbox}[colback=gray!5!white, colframe=gray!75!black, title=Closing Thought, fonttitle=\bfseries]
Magic is a powerful tool but never a safe one. Every casting carries risks, and great power always demands great responsibility. Make bold choices—then let the consequences write the next chapter.
\end{tcolorbox}

\section*{Narrative-Heavy Magic Options} \index{Narrative-Heavy Gameplay!Magic}

For groups that prefer strong narrative focus in magic use, consider these optional approaches:

\textbf{Intent-Driven Magic:} For minor magical effects that don't significantly alter the story, players can simply declare what they want to accomplish and describe how they do it, without rolling dice. The GM determines if the effect is reasonable and what complications might arise.

\textbf{Collaborative Backlash:} Instead of the GM unilaterally determining backlash, players can suggest thematic consequences that fit the fiction, with GM approval. This makes magic feel more collaborative and story-driven.

\textbf{Ritual as Story Beats:} Major magical workings can be treated as scene-defining moments where the group collaboratively describes what happens, with mechanical effects determined by the narrative impact rather than detailed rolls.

\textbf{Patron Relationships:} Focus on the roleplaying aspects of Patron relationships, treating Obligation as a measure of story tension and character development rather than just a mechanical track to be managed.

\textbf{Magic as Character Development:} Use magical experiences as opportunities for character growth and backstory development, allowing players to narrate how their characters learned new abilities through significant story moments.

\chapter{Attributes and Skills} \label{ch:attributes}

Your character's capabilities are built on four core Attributes and specialized Skills. This chapter explains how they work together to define what your character can do and how they interact with the world, with clear examples and player-facing tips.

\section*{Core Attributes} \index{Attributes}

Attributes represent your character's fundamental capabilities. Each is rated from 1 to 5, with higher numbers indicating greater proficiency.

\subsection*{Body}

Physical strength, endurance, coordination, and health.
\begin{itemize}
\item Used for: Melee combat, athletics, endurance tests, physical labor
\item Typical applications: Lifting, running, climbing, fighting, resisting physical harm
\item Associated skills: Athletics, Brawl, Melee, Endurance
\end{itemize}

\textbf{Rating examples:}
\begin{description}
\item[1:] Average person, some physical activity
\item[2:] Fit individual, regular training
\item[3:] Athlete or soldier, excellent condition
\item[4:] Exceptional athlete, near-peak human
\item[5:] Peak human capability, legendary strength
\end{description}

\subsection*{Wits}

Mental acuity, perception, quick thinking, and problem-solving.
\begin{itemize}
\item Used for: Investigation, perception, tactics, quick decisions
\item Typical applications: Spotting details, solving puzzles, planning, reacting quickly
\item Associated skills: Perception, Investigation, Tactics, Lore
\end{itemize}

\textbf{Rating examples:}
\begin{description}
\item[1:] Average awareness, sometimes misses things
\item[2:] Observant, notices important details
\item[3:] Sharp-minded, quick to spot patterns
\item[4:] Exceptionally perceptive, rarely surprised
\item[5:] Near-prescient awareness, sees connections others miss
\end{description}

\subsection*{Spirit}

Willpower, intuition, mental resilience, and connection to intangible forces.
\begin{itemize}
\item Used for: Resisting mental effects, intuition, magical aptitude, determination
\item Typical applications: Resisting fear, sensing danger, magical ability, enduring hardship
\item Associated skills: Resolve, Intuition, Magic, Faith
\end{itemize}

\textbf{Rating examples:}
\begin{description}
\item[1:] Average willpower, somewhat suggestible
\item[2:] Strong-minded, resists ordinary pressure
\item[3:] Very determined, hard to intimidate
\item[4:] Exceptional will, inspires others
\item[5:] Iron will, nearly unshakeable resolve
\end{description}

\subsection*{Presence}

Charisma, social influence, appearance, and force of personality.
\begin{itemize}
\item Used for: Social interactions, leadership, persuasion, intimidation
\item Typical applications: Negotiating, leading, charming, commanding attention
\item Associated skills: Sway, Command, Performance, Deception
\end{itemize}

\textbf{Rating examples:}
\begin{description}
\item[1:] Average presence, doesn't stand out
\item[2:] Noticeable, makes an impression
\item[3:] Charismatic, naturally influential
\item[4:] Commanding presence, people listen
\item[5:] Magnetic personality, can sway crowds
\end{description}

\section*{Skill System} \index{Skills}

Skills represent specialized training and expertise. They combine with Attributes to form your dice pool for actions.

\subsection*{Skill Ratings}

\begin{center}
\begin{tabular}{|c|l|}
\hline
\textbf{Rating} & \textbf{Description} \\
\hline
0 & Untrained — No formal training \\
1 & Novice — Basic understanding \\
2 & Competent — Reliable skill level \\
3 & Professional — Expert capability \\
4 & Master — Renowned expertise \\
5 & Grand Master — Legendary skill \\
\hline
\end{tabular}
\end{center}

\subsection*{Skill Categories}

\subsubsection*{Combat Skills}

\begin{itemize}
\item Melee: Swords, axes, close-quarters weapons
\item Ranged: Bows, crossbows, thrown weapons
\item Brawl: Unarmed combat, grappling
\item Tactics: Battlefield strategy, unit coordination
\end{itemize}

\subsubsection*{Physical Skills}

\begin{itemize}
\item Athletics: Running, climbing, jumping
\item Stealth: Moving unseen, hiding
\item Endurance: Resisting fatigue, harsh conditions
\item Craft: Building, repairing, creating
\end{itemize}

\subsubsection*{Social Skills}

\begin{itemize}
\item Sway: Persuasion, negotiation, charm
\item Command: Leadership, intimidation, authority
\item Deception: Lying, bluffing, misdirection
\item Performance: Entertainment, oration, acting
\end{itemize}

\subsubsection*{Knowledge Skills}

\begin{itemize}
\item Lore: History, culture, general knowledge
\item Investigation: Research, deduction, analysis
\item Medicine: Healing, anatomy, treatment
\item Nature: Wilderness, animals, plants
\end{itemize}

\subsubsection*{Specialized Skills}

\begin{itemize}
\item Arcana: Magic, rituals, mystical knowledge
\item Mechanics: Devices, engineering, construction
\item Diplomacy: Formal negotiation, protocol
\item Streetwise: Urban survival, criminal knowledge
\end{itemize}

\section*{Building Dice Pools} \index{Dice Pool}

Your dice pool for any action is: Attribute + Skill.

\subsection*{Choosing the Right Combination}

The same action can often be approached with different Attribute/Skill combinations:
\begin{itemize}
\item Climbing a wall:
\begin{itemize}
\item Body + Athletics (physical strength)
\item Wits + Athletics (finding the best route)
\item Spirit + Athletics (sheer determination)
\end{itemize}
\item Persuading a guard:
\begin{itemize}
\item Presence + Sway (charm and personality)
\item Wits + Sway (logical arguments)
\item Spirit + Sway (force of conviction)
\end{itemize}
\item Investigating a crime scene:
\begin{itemize}
\item Wits + Investigation (careful observation)
\item Spirit + Investigation (intuitive leaps)
\item Presence + Investigation (getting people to talk)
\end{itemize}
\end{itemize}

\subsection*{Creative Combinations}

With GM approval, you can justify unusual combinations:
\begin{itemize}
\item Body + Lore for recalling physical techniques
\item Presence + Medicine for comforting patients
\item Spirit + Craft for inspired artistic creation
\end{itemize}

\textbf{Example:} A ranger scales an ice wall using Wits + Athletics to route-find, then switches to Body + Athletics to muscle over the lip. The fiction guides the mechanics.

\section*{Skill Advancement} \index{Skills!advancement}

Improving skills requires experience points and training time.

\subsection*{XP Costs}

\begin{center}
\begin{tabular}{|l|r|}
\hline
\textbf{Improvement} & \textbf{XP Cost} \\
\hline
0→1 & 2 XP \\
1→2 & 4 XP \\
2→3 & 6 XP \\
3→4 & 8 XP \\
4→5 & 10 XP \\
\hline
\end{tabular}
\end{center}

\subsection*{Training Time}

\begin{itemize}
\item 0 → 1: 1 day of practice
\item 1 → 2: 3 days of training
\item 2 → 3: 1 week of intensive study
\item 3 → 4: 2 weeks of master training
\item 4 → 5: 1 month of dedicated practice
\end{itemize}

\subsection*{Attribute Limits}

You cannot have a skill rating higher than its primary Attribute. To increase a skill beyond your Attribute, you must first improve the Attribute.

\section*{Synergy Between Skills} \index{Skills!synergy}

Some skills work particularly well together, providing bonuses when used in combination. Synergies are situational and require fictional justification.

\subsection*{Combat Synergies}

\begin{itemize}
\item Tactics + Command: +1 die when leading groups in combat
\item Melee + Athletics: +1 die on movement-based attacks
\item Ranged + Perception: +1 die on aimed shots
\end{itemize}

\subsection*{Social Synergies}

\begin{itemize}
\item Sway + Lore: +1 die when using knowledge in persuasion
\item Deception + Performance: +1 die on sustained deceptions
\item Command + Presence: +1 die on leadership actions
\end{itemize}

\subsection*{Exploration Synergies}

\begin{itemize}
\item Investigation + Perception: +1 die on detailed searches
\item Nature + Survival: +1 die on wilderness navigation
\item Mechanics + Craft: +1 die on complex repairs
\end{itemize}

\section*{Using Skills in Play} \index{Skills!usage}

\subsection*{When to Roll}

Skills are used when:
\begin{itemize}
\item The outcome is uncertain
\item There are meaningful consequences for failure
\item The action is significant to the story
\end{itemize}

\subsection*{Difficulty Values by Skill Level}

\begin{center}
\begin{tabular}{|l|c|c|}
\hline
\textbf{Skill Level} & \textbf{Routine Task} & \textbf{Challenging Task} \\
\hline
0 & DV 2 & DV 4 \\
1 & DV 1 & DV 3 \\
2 & Automatic & DV 2 \\
3 & Automatic & DV 1 \\
4+ & Automatic & Automatic \\
\hline
\end{tabular}
\end{center}

\textbf{Reading the Table:} A Professional (3) auto-succeeds on routine tasks; challenge them with interesting stakes or higher DVs.

\subsection*{Group Skill Use}

When multiple characters use the same skill:
\begin{itemize}
\item Assistance: One character leads, others provide +1 die each (max +3)
\item Cooperation: Multiple characters attempt the same task separately
\item Complementary: Different skills used together for a complex task
\end{itemize}

\section*{Skill Challenges} \index{Skills!challenges}

Complex tasks may require multiple skill uses or extended effort.

\subsection*{Extended Tests}

For tasks taking significant time:
\begin{itemize}
\item Set a clock with 4–8 segments
\item Each successful skill use fills segments
\item Complications may add segments or create setbacks
\end{itemize}

\subsection*{Complex Challenges}

Tasks requiring multiple skills:
\begin{itemize}
\item Different characters use different skills
\item Successes contribute to overall progress
\item Failure in one area may complicate others
\end{itemize}

\textbf{Example (Complex Heist):} Stealth to enter, Mechanics to bypass locks, Investigation to locate the vault, Deception to mislead guards. Each success advances the Heist Clock; SB creates new heat.

\section*{Skill-Based Character Archetypes} \index{Character Archetypes}

\subsection*{The Warrior}

\begin{itemize}
\item Primary: Body + Melee/Ranged
\item Secondary: Spirit + Endurance, Wits + Tactics
\item Key skills: Athletics, Brawl, Command
\item Playstyle: Direct confrontation, physical solutions
\end{itemize}

\subsection*{The Expert}

\begin{itemize}
\item Primary: Wits + Lore/Investigation
\item Secondary: Presence + Sway, Spirit + Resolve
\item Key skills: Mechanics, Medicine, Perception
\item Playstyle: Problem-solving, information gathering
\end{itemize}

\subsection*{The Face}

\begin{itemize}
\item Primary: Presence + Sway/Deception
\item Secondary: Wits + Investigation, Spirit + Performance
\item Key skills: Command, Diplomacy, Streetwise
\item Playstyle: Social manipulation, negotiation
\end{itemize}

\subsection*{The Specialist}

\begin{itemize}
\item Primary: Varies by specialty
\item Secondary: Supporting skills for the specialty
\item Key skills: Craft, Arcana, Nature, etc.
\item Playstyle: Technical expertise, unique capabilities
\end{itemize}

\section*{Improving Your Capabilities} \index{Character Development}

\subsection*{Balanced Development}

\begin{itemize}
\item Improve both Attributes and Skills together
\item Develop complementary skill sets
\item Consider how skills work in combination
\item Plan for both immediate needs and long-term growth
\end{itemize}

\subsection*{Specialized Focus}

\begin{itemize}
\item Maximize one Attribute and related skills
\item Develop deep expertise in one area
\item Become the go-to character for specific challenges
\item Risk being less effective outside your specialty
\end{itemize}

\subsection*{Versatile Approach}

\begin{itemize}
\item Moderate investment in multiple areas
\item Ability to handle diverse situations
\item Less peak capability but more adaptability
\item Good for supporting other characters
\end{itemize}

\begin{tcolorbox}[colback=gray!5!white, colframe=gray!75!black, title=Attributes and Skills Quick Reference, fonttitle=\bfseries]
\textbf{Attributes (1–5):}
\begin{itemize}
\item Body: Physical capability
\item Wits: Mental acuity
\item Spirit: Willpower
\item Presence: Social influence
\end{itemize}

\textbf{Skill Levels:}
\begin{itemize}
\item 0: Untrained | 1: Novice | 2: Competent
\item 3: Professional | 4: Master | 5: Grand Master
\end{itemize}

\textbf{Dice Pool:} Attribute + Skill d10s\\
\textbf{Improvement:} New level × 2 XP (skills)\\
\textbf{Specialization:} +1 die in specific area at level 3+\\
\textbf{Synergy:} Complementary skills give +1 die
\end{tcolorbox}

\section*{Practical Examples} \index{Skills!examples}

\subsection*{Combat Example}

A warrior (Body 4, Melee 3) attacks:
\begin{itemize}
\item Dice pool: 4 + 3 = 7d10
\item Needs 6+ on each die for successes
\item DV set by opponent's defense (typically 2–3)
\end{itemize}

\subsection*{Social Example}

A diplomat (Presence 3, Sway 2) negotiates:
\begin{itemize}
\item Dice pool: 3 + 2 = 5d10
\item Position: Risky (opponent is skeptical)
\item Stakes: Success gets cooperation, failure creates suspicion
\end{itemize}

\subsection*{Exploration Example}

A scout (Wits 3, Perception 2) searches for tracks:
\begin{itemize}
\item Dice pool: 3 + 2 = 5d10
\item DV 2 for fresh tracks, DV 3 for old tracks
\item Success finds trail, partial finds clues, miss misses important signs
\end{itemize}

\begin{tcolorbox}[colback=gray!5!white, colframe=gray!75!black, title=Final Note, fonttitle=\bfseries]
Your Attributes and Skills define not just what you can do, but how you approach challenges. Choose combinations that reflect your character's personality and style, and let the fiction lead your mechanical choices.
\end{tcolorbox}

\section*{Narrative-Heavy Skill Options} \index{Narrative-Heavy Gameplay!Skills}

For groups that prefer strong narrative focus in skill use, consider these optional approaches:

\textbf{Intent-Driven Skills:} For routine tasks that don't significantly impact the story, players can simply declare what they want to accomplish without rolling dice. The GM determines if the action succeeds based on the character's capabilities and the fiction.

\textbf{Descriptive Assistance Bonuses:} Players can provide vivid, helpful descriptions of how they're using their skills to assist allies, granting a +1 die bonus to the primary actor's roll without spending Boons.

\textbf{Skill as Character Development:} Use skill challenges as opportunities for character growth and backstory development, allowing players to narrate how their characters learned new techniques through significant story moments.

\textbf{Collaborative Difficulty Setting:} Instead of the GM unilaterally setting DVs, players can suggest reasonable difficulty levels based on their understanding of the task, with GM approval.

\textbf{Narrative Skill Synergies:} Focus on how skills work together in the story rather than mechanical bonuses. A well-described combination of skills might grant advantage on Position or Effect without requiring specific synergy rules.

\chapter{Experience Paths and Character Building} \label{ch:paths}

How you spend your Experience Points (XP) defines not only your character's capabilities—but also their role in the world. This chapter explores different advancement philosophies and provides practical, legal starting builds that fit the campaign's creation rules.

\section*{Three Advancement Paths} \index{Character Development Paths}

There are three broad approaches to character development, each representing a different philosophy of growth:

\begin{description}
\item[Personal Path] Invest in personal mastery and self-improvement
\item[Balanced Path] Mix personal growth with resources and influence
\item[Influencer Path] Focus on networks, assets, and strategic power
\end{description}

\section*{Path 1: Personal Development} \index{Character Development Paths!Personal}

The Personal Path focuses on individual capability through attributes and skills.

\subsection*{Typical Investment}

\begin{itemize}
\item 70–90\% Personal improvement
\item 0–10\% Resources and assets
\item 0–20\% Special abilities
\end{itemize}

\subsection*{Strengths}

\begin{itemize}
\item Reliable in direct challenges and combat
\item Minimal upkeep or management required
\item Resilient to loss of external resources
\item Consistent performance in spotlight moments
\end{itemize}

\subsection*{Weaknesses}

\begin{itemize}
\item Limited influence in social or strategic scenes
\item May struggle with problems requiring networks
\item Less capable in logistics or large-scale operations
\item Dependent on personal presence for all solutions
\end{itemize}

\subsection*{Build Example: The Duelist (Legal Start)}

Total XP: 30 (34 with +4 from Bonds/Complications; see §6.5)

\begin{itemize}
\item Attributes: Body 3, Wits 2, Spirit 1, Presence 1
\begin{itemize}
\item Costs (Attributes cost new rating ×3 each step): Body 1→2 (6), 2→3 (9) = 15; Wits 1→2 (6) = 6; Spirit/Presence remain 1 = 0. Subtotal: 21 XP
\end{itemize}
\item Skills: Melee 2, Athletics 1
\begin{itemize}
\item Costs (Skills cost new level ×2 each step): Melee 0→1 (2), 1→2 (4) = 6; Athletics 0→1 = 2. Subtotal: 8 XP
\end{itemize}
\item Totals: 21 + 8 = 29 XP. Bank 1
```latex
 XP.
\item With +4 XP (Bonds/Complications): add Perception 0→1 (2) and spend banked 1 XP on Stealth 0→1 (2), or instead take Perception 0→1 (2) and Sway 0→1 (2) for broader utility.
\item Cap: 34 XP.
\end{itemize}

\section*{Path 2: Balanced Approach} \index{Character Development Paths!Balanced}

The Balanced Path mixes personal capability with strategic resources.

\subsection*{Typical Investment}

\begin{itemize}
\item 50–65\% Personal improvement
\item 15–25\% Resources and assets
\item 15–25\% Special abilities
\end{itemize}

\subsection*{Strengths}

\begin{itemize}
\item Adaptable to diverse situations
\item Can handle both direct and indirect challenges
\item Good supporting role for the group
\item Moderate risk profile
\end{itemize}

\subsection*{Weaknesses}

\begin{itemize}
\item Not exceptional in any single area
\item Requires management of resources
\item Moderate upkeep demands
\item Can be outshone by specialists
\end{itemize}

\subsection*{Build Example: The Scout (Legal Start)}

Total XP: 30 (34 with +4 from Bonds/Complications)

\begin{itemize}
\item Attributes: Wits 2, Body 2, Spirit 1, Presence 1
\begin{itemize}
\item Costs: Wits 1→2 (6), Body 1→2 (6) = 12 XP
\end{itemize}
\item Skills: Survival 2, Perception 1, Stealth 1
\begin{itemize}
\item Costs: Survival 0→1 (2), 1→2 (4) = 6; Perception 0→1 2; Stealth 0→1 2. Subtotal: 10 XP
\end{itemize}
\item Resources: Minor equipment cache (camp gear, maps, signal kit) = 4 XP
\item Special Abilities: Wilderness Lore (broad travel benefits) = 4 XP
\item Totals: 12 + 10 + 4 + 4 = 30 XP.
\item With +4 XP: add Perception 1→2 (+4) or take a trained hawk companion (Minor Resource, 4 XP).
\end{itemize}

\section*{Path 3: Influencer Focus} \index{Character Development Paths!Influencer}

The Influencer Path prioritizes networks, assets, and strategic power.

\subsection*{Typical Investment}

\begin{itemize}
\item 25–40\% Personal improvement
\item 35–55\% Resources and assets
\item 20–40\% Special abilities
\end{itemize}

\subsection*{Strengths}

\begin{itemize}
\item Strong strategic and social influence
\item Can solve problems indirectly
\item Excellent at planning and preparation
\item Creates opportunities for the whole group
\end{itemize}

\subsection*{Weaknesses}

\begin{itemize}
\item Personally vulnerable in direct confrontations
\item High maintenance requirements
\item Complications can cascade through networks
\item Dependent on external factors
\end{itemize}

\subsection*{Build Example: The Merchant (Legal Start)}

Total XP: 30 (34 with +4 from Bonds/Complications)

\begin{itemize}
\item Attributes: Presence 2, Wits 2, Spirit 1, Body 1
\begin{itemize}
\item Costs: Presence 1→2 (6), Wits 1→2 (6) = 12 XP
\end{itemize}
\item Skills: Sway 2, Deception 1, Lore 1
\begin{itemize}
\item Costs: Sway 0→1 (2), 1→2 (4) = 6; Deception 0→1 2; Lore 0→1 2. Subtotal: 10 XP
\end{itemize}
\item Resources: Standard trading office (staffed storefront, ledgers, storage) = 8 XP
\item Totals: 12 + 10 + 8 = 30 XP.
\item With +4 XP: add Negotiation Mastery (4 XP general ability) or expand to a second Minor merchant route (4 XP).
\end{itemize}

\section*{Starting Character Guidelines} \index{Character Creation}

\subsection*{Base XP Allocation}

\begin{itemize}
\item Standard Starting XP: 30 points
\item Bonds and Complications: You may take up to two total from any mix of meaningful Bonds (up to 2, +2 XP each) and significant Complications (up to 2, +2 XP each), granting maximum +4 XP.
\item Maximum Starting XP: 34 points
\item Complication Effect: Each unresolved starting Complication adds +1 banked SB to early scenes until cleared.
\end{itemize}

\subsection*{Recommended Starting Ranges}

\begin{center}
\begin{tabular}{|l|r|}
\hline
\textbf{Category} & \textbf{Recommended XP} \\
\hline
Primary Attribute & 9–12 XP (rating 3–4) \\
Secondary Attributes & 0–9 XP each (rating 1–3) \\
Key Skills & 4–6 XP each (rating 2–3) \\
Supporting Skills & 2–4 XP each (rating 1–2) \\
Resources & 0–8 XP total \\
Special Abilities & 0–8 XP total \\
\hline
\end{tabular}
\end{center}

\subsection*{Cost Reminders:}

\begin{itemize}
\item Attributes: Each step costs new rating ×3 XP (e.g., 1→2 costs 6; 2→3 costs 9).
\item Skills: Each step costs new level ×2 XP (e.g., 0→1 costs 2; 1→2 costs 4).
\item Resources: Minor 4 XP; Standard 8 XP; Major 12 XP.
\item Special Abilities: Minor Edge 2 XP; Major Edge 4 XP; Prestige 6+ XP.
\end{itemize}

\section*{Progression Planning} \index{Character Development!Planning}

\subsection*{Early Game (0–40 XP)}

Focus on establishing core capabilities:
\begin{itemize}
\item Reach attribute rating 3 in your primary area
\item Develop 2–3 key skills to rating 2–3
\item Acquire basic resources or one special ability
\item Establish your character's niche in the group
\end{itemize}

\subsection*{Mid Game (41–90 XP)}

Expand and specialize:
\begin{itemize}
\item Increase primary attribute to 4
\item Specialize key skills to rating 3–4
\item Develop supporting capabilities
\item Build strategic resources or networks
\item Acquire signature special abilities
\end{itemize}

\subsection*{Late Game (91–150 XP)}

Master your chosen path:
\begin{itemize}
\item Achieve peak attributes (rating 4–5)
\item Master key skills (rating 4–5)
\item Build substantial influence or unique capabilities
\item Develop advanced special abilities
\item Consider legacy projects or organizations
\end{itemize}

\section*{Path Combination Strategies} \index{Character Development!Combination}

Many players mix elements from different paths:

\subsection*{Combat Specialist with Resources}

\begin{itemize}
\item Strong personal combat capabilities
\item Moderate resource investment for support
\item Good for frontline fighters who need logistical support
\item Example: Warrior with a fortified base and loyal troops
\end{itemize}

\subsection*{Social Character with Personal Skills}

\begin{itemize}
\item Excellent social capabilities
\item Solid personal skills for self-defense
\item Good for diplomats who operate independently
\item Example: Ambassador with combat training and persuasion skills
\end{itemize}

\subsection*{Technical Expert with Networks}

\begin{itemize}
\item Deep technical or magical expertise
\item Network of contacts and resources
\item Good for specialists who need support systems
\item Example: Master crafter with supplier network and apprentices
\end{itemize}

\section*{Resource Management} \index{Resource Management}

Each path requires different management approaches:

\subsection*{Personal Path Management}

\begin{itemize}
\item Minimal upkeep requirements
\item Focus on equipment maintenance
\item Occasional skill practice or training
\item Low complexity, high reliability
\end{itemize}

\subsection*{Balanced Path Management}

\begin{itemize}
\item Moderate upkeep for resources
\item Relationship maintenance with contacts
\item Skill development alongside resource management
\item Balanced time investment
\end{itemize}

\subsection*{Influencer Path Management}

\begin{itemize}
\item Significant upkeep demands
\item Network maintenance and expansion
\item Resource allocation and development
\item Strategic planning and opportunity management
\end{itemize}

\section*{Risk Assessment} \index{Character Development!Risk}

Each path carries different risks:

\subsection*{Personal Path Risks}

\begin{itemize}
\item Over-specialization in one area
\item Vulnerability to problems outside specialty
\item Limited growth options later in game
\item May become predictable in approach
\end{itemize}

\subsection*{Balanced Path Risks}

\begin{itemize}
\item Jack-of-all-trades, master of none
\item Spread too thin across capabilities
\item Moderate risks in multiple areas
\item May lack standout capabilities
\end{itemize}

\subsection*{Influencer Path Risks}

\begin{itemize}
\item Network vulnerability to attacks
\item High maintenance requirements
\item Cascade failure potential
\item Personal safety concerns
\end{itemize}

\section*{Building for Group Synergy} \index{Group Dynamics}

Consider how your path complements other party members:

\subsection*{Complementary Paths}

\begin{itemize}
\item Personal path characters provide reliable combat capability
\item Balanced path characters handle diverse challenges
\item Influencer path characters create opportunities and resources
\item Mixed groups cover all bases effectively
\end{itemize}

\subsection*{Redundant Paths}

\begin{itemize}
\item Multiple personal path characters may overlap in combat
\item Multiple influencer path characters may compete for resources
\item Consider diversifying within similar paths
\item Example: Different combat specialties or resource types
\end{itemize}

\section*{Adapting Your Path} \index{Character Development!Adaptation}

Your chosen path isn't permanent—you can shift focus as the game progresses:

\subsection*{Early Shift (0–40 XP)}

\begin{itemize}
\item Easy to change direction
\item Minimal sunk cost in any approach
\item Good time to experiment with different styles
\item Can respond to group needs or story developments
\end{itemize}

\subsection*{Mid Game Shift (41–90 XP)}

\begin{itemize}
\item Requires more deliberate planning
\item Some capabilities may need to be maintained
\item Can fill emerging gaps in group capability
\item May require temporary performance dip during transition
\end{itemize}

\subsection*{Late Game Shift (91+ XP)}

\begin{itemize}
\item Significant investment in current path
\item Major shift requires substantial XP investment
\item Consider adding complementary capabilities rather than replacing
\item May be better to develop existing strengths further
\end{itemize}

\begin{tcolorbox}[colback=gray!5!white, colframe=gray!75!black, title=XP Path Quick Reference, fonttitle=\bfseries]
\textbf{Personal Path (70–90\% self):}
\begin{itemize}
\item Reliable individual performance
\item Low upkeep, high consistency
\item Best for combat and specialist roles
\end{itemize}

\textbf{Balanced Path (50–65\% self):}
\begin{itemize}
\item Good all-around capability
\item Moderate risk and upkeep
\item Flexible supporting role
\end{itemize}

\textbf{Influencer Path (25–40\% self):}
\begin{itemize}
\item Strategic power and influence
\item High upkeep, high reward
\item Creates opportunities for group
\end{itemize}

\textbf{Starting XP:} 30 base + up to +4 from Bonds/Complications (max start 34).
\end{tcolorbox}

\section*{Practical Building Examples (Narrative Roles, Legal Starts)} \index{Character Creation!Examples}

\subsection*{Example 1: The Guardian}

Path: Personal \hfill Total: 30 XP

\begin{itemize}
\item Attributes: Body 3 (15), Wits 2 (6) = 21 XP
\item Skills: Melee 2 (6), Athletics 1 (2) = 8 XP
\item Bank: 1 XP
\item Role at table: Frontline protection, reliable duel pressure. With +4 XP, add Combat Reflexes (2 XP talent) and Shield Mastery (4 XP talent) using banked 1 + 4 = 5 XP, with 1 XP remaining.
\end{itemize}

\subsection*{Example 2: The Explorer}

Path: Balanced \hfill Total: 30 XP

\begin{itemize}
\item Attributes: Wits 2 (6), Body 2 (6) = 12 XP
\item Skills: Survival 2 (6), Perception 1 (2), Stealth 1 (2) = 10 XP
\item Resources: Minor mapping kit \& route notes = 4 XP
\item Ability: Trail Sense = 4 XP
\item Totals: 30 XP. With +4 XP, raise Perception 1→2 (+4) or add a trained beast (Minor Resource, 4).
\end{itemize}

\subsection*{Example 3: The Schemer}

Path: Influencer \hfill Total: 30 XP

\begin{itemize}
\item Attributes: Presence 2 (6), Wits 2 (6) = 12 XP
\item Skills: Sway 2 (6), Deception 1 (2), Lore 1 (2) = 10 XP
\item Resources: Standard safehouse \& message drops = 8 XP
\item Totals: 30 XP. With +4 XP, take Network Builder (4 XP talent) or add Minor informant ring (4 XP).
\end{itemize}

\begin{tcolorbox}[colback=gray!5!white, colframe=gray!75!black, title=Reminder, fonttitle=\bfseries]
All builds above assume baseline Attributes at 1 and Skills at 0 before spending. Attribute and Skill advances are cumulative by step (see costs in §6.5).

Remember: Your chosen path should reflect both your character concept and your preferred play style. There's no single "correct" path—only what works for you and your group.
\end{tcolorbox}

\section*{Narrative-Heavy Character Building Options} \index{Narrative-Heavy Gameplay!Character Building}

For groups that prefer strong narrative focus in character building, consider these optional approaches:

\textbf{Story-Driven Milestones:} Instead of tracking XP numerically, the GM can award advancement when characters reach significant story milestones. "You've trained with the master for months—you've improved your skill."

\textbf{Experience Through Reflection:} Players can spend downtime scenes reflecting on past experiences to earn XP. A meaningful flashback or character moment can justify growth without tracking specific points.

\textbf{Collaborative Advancement:} The group can discuss and agree on advancement choices, ensuring everyone's growth supports the overall story direction.

\textbf{Narrative Justification Focus:} When spending XP, players should explain how their character gained this capability through in-game experiences, creating richer backstory and continuity.

\textbf{Path as Theme:} Focus on the narrative themes of your chosen path rather than strict XP allocations. A Personal Path character might emphasize their journey of self-mastery, while an Influencer Path character focuses on their growing web of relationships and influence.

\chapter{Talents and Special Abilities} \label{ch:talents}

Talents are the building blocks of character specialization. They represent learned techniques, supernatural gifts, or cultural inheritances. Each Talent costs XP, and their costs are tied to impact.

\section*{Understanding Talents} \index{Talents}

Talents are purchased with Experience Points (XP) and provide special capabilities:
\begin{itemize}
\item They go beyond simple skill bonuses
\item They often have specific activation conditions
\item They may provide narrative permissions (you can try things others cannot)
\item They can define your character's unique identity
\end{itemize}

\subsection*{Talent Costs}

\begin{center}
\begin{tabular}{|l|r|l|}
\hline
\textbf{Type} & \textbf{Cost} & \textbf{Examples} \\
\hline
Minor Edge & 2 XP & Caster's Gift, +1 situational bonus \\
Major Edge & 4 XP & Patron's Symbol, strong summon upgrade \\
Prestige & 6+ XP & Campaign-defining effects \\
\hline
\end{tabular}
\end{center}

\subsection*{Activation Types}

\begin{description}
\item[Passive:] Always on; no action
\item[Active:] Requires an action or scene focus
\item[Reactive:] Triggers on a condition
\end{description}

\subsection*{Limits and Economy}

Unless a talent says otherwise:
\begin{itemize}
\item Per Scene uses refresh at scene end
\item Per Session uses refresh after downtime
\item Some talents allow you to spend Boons to push effects
\end{itemize}

\section*{Talent Categories} \index{Talents!categories}

\subsection*{Minor Edge Talents}

Basic abilities available to any character:
\begin{itemize}
\item Cost: 2 XP
\item Examples: Caster's Gift, Familiar Bond, basic magical abilities
\item Best for: Essential capabilities and access requirements
\end{itemize}

\subsection*{Major Edge Talents}

Significant abilities with moderate requirements:
\begin{itemize}
\item Cost: 4 XP
\item Examples: Patron's Symbol, Codex, significant summon upgrades
\item Best for: Core specialization and magical access
\end{itemize}

\subsection*{Prestige Talents}

Powerful abilities unlocked through mastery or story events:
\begin{itemize}
\item Cost: 6+ XP
\item Examples: Breaking fundamental limits, forbidden summons, rewriting obligations
\item Best for: Campaign-shaping capabilities
\end{itemize}

\section*{Magic Access Talents} \index{Talents!magic access}

\subsection*{Caster's Gift}

Cost: 2 XP

Grants access to Weave \& Cast freeform spellcasting using the Eight Elements. Without this, characters cannot freeform cast.

\subsection*{Familiar}

Cost: 2 XP

Required to access Patron features such as Patron's Gift. Binds a Thiasos.

\subsection*{Codex}

Cost: 4 XP

Required to fully join a Patron's service as a Runekeeper. Grants access to that Patron's Rites and Obligation system.

\subsection*{Patron's Symbol}

Cost: 4 XP

Minor Asset. Allows an Invoker to access a Patron's Rites via ritual precision. Each Patron requires its own Symbol.

\section*{Patron's Gift (Imbuement)} \index{Talents!Patron's Gift}

Cost: Free (requires Thiasos)

Activation: 1 Action once per scene

Duration: Scene

Range: Touch

Effect: Imbue one item with temporary magical power related to your Patron's domain. The item functions as a magical weapon (+1 Melee) and specialized tool (+1 thematic Skill) for the scene.

Push It: The item's power persists for one additional scene but marks +1 Obligation.

\section*{Selecting Talents} \index{Talents!selection}

\subsection*{Consider Your Magical Path}

Choose talents that reinforce your character's magical approach:
\begin{itemize}
\item Caster: Freeform spellcasting talents, elemental control
\item Runekeeper: Rites access, Obligation management, Patron specialization
\item Invoker: Ritual efficiency, Symbol maintenance, invocation speed
\item Specialist: Unique talents matching your specific focus
\end{itemize}

\subsection*{Balance Access and Power}

Consider both access requirements and power talents:
\begin{itemize}
\item Access: Essential prerequisites (Caster's Gift, Familiar)
\item Power: Combat enhancements, magical amplifications
\item Utility: Support abilities, resource management
\end{itemize}

\subsection*{Think About Investment}

Consider how much XP each talent represents:
\begin{itemize}
\item Minor (2 XP): Essential access, small narrative tricks
\item Major (4 XP): Strong upgrades, permanent effects in niche
\item Prestige (6+ XP): Campaign-defining, fundamental limits broken
\end{itemize}

\section*{Talent Building Strategies} \index{Talents!building strategies}

\subsection*{The Specialist}

Focus on talents supporting one primary magical path:
\begin{itemize}
\item Choose talents that synergize with each other
\item Develop a clear specialization identity
\item Become the go-to character for specific magical challenges
\item Risk: May be less effective outside specialty
\end{itemize}

\subsection*{The Generalist}

Spread talents across multiple magical approaches:
\begin{itemize}
\item Cover different types of magical challenges
\item Provide support to other party members
\item Adapt to diverse situations
\item Risk: Less peak capability in any area, increased bookkeeping
\end{itemize}

\subsection*{The Foundation Builder}

Focus on essential access talents first:
\begin{itemize}
\item Prioritize access requirements (Caster's Gift, Familiar)
\item Build toward major capabilities
\item Establish core identity before specialization
\item Risk: May lack immediate power payoff
\end{itemize}

\section*{Talent Examples} \index{Talents!examples}

\subsection*{Magic Access Talents}

\begin{description}
\item[Caster's Gift (2 XP)] — Access to Weave \& Cast freeform spellcasting using the Eight Elements.
\item[Familiar (2 XP)] — Required for Patron's Gift and other Patron features.
\item[Codex (4 XP)] — Full access to a Patron's Rites and Obligation system.
\item[Patron's Symbol (4 XP)] — Ritual access to a Patron's Rites via invocation.
\end{description}

\subsection*{Combat Talents}

\begin{description}
\item[Second Wind (2 XP, Active)] — Once per scene, clear 1 Fatigue when you take a moment to catch your breath.
\item[Combat Reflexes (2 XP, Reactive)] — +1 die on defense rolls when surprised or flanked.
\item[Precise Strike (2 XP, Active)] — Once per scene, ignore armor on one attack if you had Controlled or Risky position.
\item[Weapon Mastery (4 XP, Passive)] — Choose a weapon type; +1 die when using it.
\end{description}

\subsection*{Social Talents}

\begin{description}
\item[Silver Tongue (2 XP, Passive)] — +1 die on persuasion attempts.
\item[Read Emotions (2 XP, Active)] — Once per scene, automatically detect surface emotions in a social exchange.
\item[Command Presence (4 XP, Passive)] — +1 die on leadership and intimidation rolls.
\item[Network Builder (4 XP, Passive)] — Gain a minor contact in each new settlement visited.
\end{description}

\subsection*{Exploration Talents}

\begin{description}
\item[Keen Senses (2 XP, Passive)] — +1 die on perception checks to spot danger or hidden details.
\item[Wilderness Lore (2 XP, Passive)] — Automatically find food and water in hospitable biomes.
\item[Trackless Step (2 XP, Active)] — Leave no trail for the rest of the scene.
\item[Urban Navigation (2 XP, Passive)] — Never get lost in cities.
\end{description}

\section*{Advanced Talent Examples} \index{Talents!advanced examples}

\subsection*{Casting Mastery}

\begin{description}
\item[Spell Shaping (4 XP; Req: Caster's Gift)] — Modify spell factors (range/scale/targeting) by one step when you Weave.
\item[Elemental Mastery (6 XP; Req: Arcana 3)] — Reduce backlash severity by one step when casting spells of your chosen element.
\item[Arcane Dominance (6 XP; Req: Spirit 4, Arcana 4)] — Overpower weaker magical effects automatically when you contest them.
\end{description}

\subsection*{Ritual Expertise}

\begin{description}
\item[Ritual Mastery (4 XP; Req: Familiar)] — Perform rituals with reduced risk: the GM spends 1 fewer SB on ritual backlash.
\item[Efficient Invocation (4 XP; Req: Patron's Symbol)] — Reduce ritual casting time by one step (minimum 1 Player Turn).
\item[Crack Specialist (6 XP; Req: 3 Patron Symbols)] — Reduce Crack the Seal Obligation cost by 1 (minimum +1).
\item[Dual Covenant (6 XP):] Maintain two active summons.
\end{description}

\subsection*{Prestige Abilities}

\begin{description}
\item[Forbidden Knowledge (6 XP; Req: Tier II)] — Access to one forbidden summon or dangerous rite.
\item[Obligation Master (8 XP; Req: Tier III, Codex)] — Reduce all Obligation segment costs by 1 (minimum 1).
\item[Backlash Immunity (10 XP; Req: Tier IV, Spirit 5)] — Ignore minor backlash entirely on casting rolls.
\item[Triad Bond (8 XP):] Maintain three active summons.
\end{description}

\section*{Talent Synergies} \index{Talents!synergies}

Some talents work particularly well together:

\subsection*{Casting Synergies}

\begin{itemize}
\item Caster's Gift + Spell Shaping: Flexible, precise freeform casting
\item Elemental Mastery + Arcane Dominance: Powerful, controlled elemental effects
\item Ritual Mastery + Caster's Gift: Reduced risk on both freeform and ritual casting
\end{itemize}

\subsection*{Social Synergies}

\begin{itemize}
\item Silver Tongue + Command Presence: Charm or command with equal force
\item Read Emotions + Network Builder: Understand and leverage social connections
\item Familiar + Social Talents: Patron-enhanced social abilities
\end{itemize}

\subsection*{Exploration Synergies}

\begin{itemize}
\item Keen Senses + Trackless Step: Find others while leaving no trace
\item Wilderness Lore + Urban Navigation: Comfortable in all environments
\item Familiar + Exploration Talents: Patron-guided exploration
\end{itemize}

\section*{Talent Limitations and Balance} \index{Talents!limitations}

\subsection*{Usage Restrictions}

Most talents have limits to maintain game balance:
\begin{itemize}
\item Per scene: Common for strong actives
\item Per session: Reserved for swingy effects
\item Resource cost: Some require spending Boons or generating Obligation
\item Position requirements: May require specific narrative circumstances
\end{itemize}

\subsection*{Prerequisite Systems}

Advanced talents require meeting certain conditions:
\begin{itemize}
\item Attribute minimums: e.g., Spirit 4, Wits 3
\item Skill requirements: Specific skills at set levels
\item Previous talents: Foundational picks first (Familiar required for Patron features)
\item Tier requirements: Character advancement level
\end{itemize}

\section*{Building Your Talent Set} \index{Talents!building}

\subsection*{Early Game (0–40 XP)}

Focus on essential access and basic capabilities:
\begin{itemize}
\item 1–2 access talents (Caster's Gift, Familiar)
\item 2–3 basic talents for reliability
\item Save XP for major access requirements
\item Choose talents that work with your core concept
\end{itemize}

\subsection*{Mid Game (41–90 XP)}

Develop your specialization:
\begin{itemize}
\item Major access talents (Codex, Patron's Symbol)
\item 2–3 synergistic power talents
\item Balance active and passive picks
\item Plan for prestige abilities
\end{itemize}

\subsection*{Late Game (91+ XP)}

Achieve mastery:
\begin{itemize}
\item 1–2 prestige talents defining your apex
\item Picks that create legacy effects
\item Talents that benefit the whole party
\item Prepare for campaign-defining challenges
\end{itemize}

\section*{Talent Customization} \index{Talents!customization}

Work with your Game Master to create custom talents:
\begin{itemize}
\item Based on story events: Reflect character experiences
\item Balanced costs: Match similar scope to existing talents (2/4/6+ XP)
\item Clear prerequisites: Define requirements clearly
\item Mechanical clarity: Define activation, effects, and limits
\end{itemize}

\section*{Talents and Group Dynamics} \index{Talents!group dynamics}

Consider how your talents complement the party:
\begin{itemize}
\item Fill gaps: Cover party weaknesses in magical capabilities
\item Synergize: Coordinate with other players' magical approaches
\item Avoid overlap: Don't duplicate another character's access path
\item Support role: Talents that help the whole group manage magical risks
\end{itemize}

\section*{Talent Respecification} \index{Talents!respecification}

If your character concept changes, you may respec talents:
\begin{itemize}
\item GM approval required: Discuss proposed changes
\item Downtime cost: Represent retraining (typically 1 downtime period)
\item Story justification: Explain the change in-narrative
\item Limited frequency: Typically once per major story arc
\end{itemize}

\begin{tcolorbox}[colback=gray!5!white, colframe=gray!75!black, title=Talent Selection Guide, fonttitle=\bfseries]
\textbf{Early Game (0–40 XP):}
\begin{itemize}
\item 1–2 access talents (2 XP each)
\item 2–4 basic talents (2 XP each)
\item Focus on essential capabilities
\end{itemize}

\textbf{Mid Game (41–90 XP):}
\begin{itemize}
\item 1–2 major talents (4 XP each)
\item 1–2 advanced talents (4–6 XP each)
\item Plan for prestige prerequisites
\end{itemize}

\textbf{Late Game (91+ XP):}
\begin{itemize}
\item 1–2 prestige talents (6+ XP each)
\item Campaign-defining capabilities
\item Party-supporting abilities
\end{itemize}

\textbf{Remember:} Talents should reflect your character's story and magical growth.
\end{tcolorbox}

\section*{Practical Talent Examples} \index{Talents!practical examples}

\subsection*{Example 1: The Caster}

\begin{itemize}
\item Caster's Gift (2 XP) — Essential access to freeform casting
\item Spell Shaping (4 XP) — Modify spell parameters
\item Elemental Mastery (6 XP) — Reduce casting risks
\item Arcane Dominance (6 XP) — Overpower opposing magic
\item Total: 18 XP invested in casting capabilities
\end{itemize}

\subsection*{Example 2: The Runekeeper}

\begin{itemize}
\item Familiar (2 XP) — Access to Patron features
\item Codex (4 XP) — Full Rites access
\item Ritual Mastery (4 XP) — Reduced ritual risks
\item Obligation Master (8 XP) — Better debt management
\item Total: 18 XP invested in Pact magic
\end{itemize}

\subsection*{Example 3: The Invoker}

\begin{itemize}
\item Patron's Symbol (4 XP) — Ritual access to Patron
\item Efficient Invocation (4 XP) — Faster rituals
\item Crack Specialist (6 XP) — Reduced instant cast costs
\item Ritual Mastery (4 XP) — Reduced backlash
\item Total: 18 XP invested in ritual magic
\end{itemize}

\begin{tcolorbox}[colback=gray!5!white, colframe=gray!75!black, title=Final Note, fonttitle=\bfseries]
The best talents are those that fit your magical concept and table playstyle. Choose abilities you'll enjoy using, that create interesting consequences, and that contribute to your character's unfolding story through the lens of risk and consequence that defines Fate's Edge magic.
\end{tcolorbox}

\section*{Narrative-Heavy Talent Options} \index{Narrative-Heavy Gameplay!Talents}

For groups that prefer strong narrative focus in talent use, consider these optional approaches:

\textbf{Story-Driven Talents:} Instead of mechanical bonuses, some talents can provide narrative permissions or story effects. "Courtly Grace" might allow you to navigate noble society without rolls, while "Wild Empathy" lets you communicate with animals through roleplay rather than dice.

\textbf{Collaborative Talent Activation:} Players can describe how their talents work in the fiction, with GM approval, rather than relying solely on mechanical triggers. A "Master Strategist" might narrate how they reposition allies through clever tactics rather than just declaring the mechanical effect.

\textbf{Talent as Character Development:} Use talent acquisition as opportunities for character growth and backstory development, allowing players to narrate how their characters learned new abilities through significant story moments.

\textbf{Flexible Talent Interpretation:} Focus on the thematic effects of talents rather than strict mechanical applications. A "Weapon Mastery" talent might manifest differently depending on the weapon and situation, with the GM and player collaborating on the specific benefits.

\chapter{Assets and Followers} \label{ch:assets}

Your character's influence extends beyond personal capabilities through Assets and Followers. These represent worldly possessions, connections, and allies that can solve problems, provide assistance, and shape the narrative.

\section*{Understanding Assets and Followers} \index{Assets}\index{Followers}

\subsection*{Key Differences}

\begin{description}
\item[Assets:] Off-screen resources that solve problems between scenes.
\item[Followers:] On-screen allies who assist during gameplay.
\item[Assets change:] The fictional situation before you arrive.
\item[Followers act:] Alongside you in the moment.
\end{description}

\subsection*{Management Requirements}

Both require maintenance and carry risks:
\begin{itemize}
\item Regular upkeep costs (XP or downtime).
\item Vulnerability to complications and attacks.
\item Narrative consequences for misuse or neglect.
\end{itemize}

\section*{Assets System} \index{Assets!system}

Assets are possessions, properties, or resources you control.

\subsection*{Asset Types and Costs}

\begin{center}
\begin{tabular}{|l|r|l|}
\hline
\textbf{Type} & \textbf{XP Cost} & \textbf{Establishment Time} \\
\hline
Minor & 4 XP & 1 day \\
Standard & 8 XP & 1 week \\
Major & 12 XP & 1 month \\
\hline
\end{tabular}
\end{center}

\subsection*{Asset Examples}

\begin{description}
\item[Minor Assets] Small shop, safehouse, minor title, basic workshop.
\item[Standard Assets] Noble title, guild membership, trading post, spy network.
\item[Major Assets] Fortress, city license, major enterprise, regional influence.
\end{description}

\subsection*{Using Assets}

Assets provide benefits in different ways:

\subsubsection*{Free Off-Screen Use}

Each asset has a specific off-screen effect you can use once per session:
\begin{itemize}
\item Safehouse: Provide secure lodging for the party.
\item Spy Network: Gather basic intelligence about a location.
\item Workshop: Repair or create simple items between adventures.
\item Trading Post: Acquire common goods at better prices.
\end{itemize}

\subsubsection*{Boon Activation}

Spend 1 Boon to use an asset dramatically during a scene:
\begin{itemize}
\item Safehouse: Suddenly reveal a hidden escape route.
\item Spy Network: Produce crucial information at a critical moment.
\item Workshop: Create an improvised solution to an immediate problem.
\item Trading Post: Call in a favor from a business contact.
\end{itemize}

\subsubsection*{XP Activation}

Spend 2 XP to use an asset's off-screen effect outside your normal allowance:
\begin{itemize}
\item Emergency use when you've already used your free activation.
\item Additional uses during downtime periods.
\item Special circumstances requiring extra asset support.
\end{itemize}

\section*{Asset Conditions} \index{Assets!conditions}

Assets have condition states affecting their usefulness:

\subsection*{Condition Levels}

\begin{description}
\item[Maintained] Fully functional, no penalties.
\item[Neglected] −1 die when used; requires attention.
\item[Compromised] Unavailable until repaired or recovered.
\end{description}

\subsection*{Maintenance Requirements}

\begin{itemize}
\item Regular Upkeep: Two options per SRD §21.2:
\begin{itemize}
\item Efficient (Higher XP, Less Time): Pay Upkeep XP = max(1, XP Acquisition)/3, minimal effort
\item Intensive (Lower XP, More Time): Pay 1 XP, dedicated downtime action
\end{itemize}
\item Neglect: Assets deteriorate if not maintained.
\item Recovery: Compromised assets require significant effort to restore.
\end{itemize}

\section*{Followers System} \index{Followers!system}

Followers are characters who assist you directly.

\subsection*{Follower Capability Ratings}

Followers are rated by Capability (Cap) from 1 to 5:

\begin{center}
\begin{tabular}{|c|l|}
\hline
\textbf{Cap} & \textbf{Description} \\
\hline
1 & Novice helper, basic assistance \\
2 & Competent assistant, reliable support \\
3 & Skilled specialist, valuable aid \\
4 & Expert ally, significant capability \\
5 & Master companion, exceptional ability \\
\hline
\end{tabular}
\end{center}

\subsection*{Follower Costs}

\begin{itemize}
\item XP Cost: Capability squared (Cap$^2$).
\item Example: Cap 3 follower costs 3$^2$ = 9 XP.
\item Recruitment: 1–3 days downtime to find and brief.
\item Limits: The GM may set maximum followers based on story.
\end{itemize}

\subsection*{Follower Types}

\begin{description}
\item[Combat Allies] Warriors, guards, mercenaries.
\item[Technical Experts] Craftspeople, engineers, specialists.
\item[Social Contacts] Informants, diplomats, agents.
\item[Specialists] Unique capabilities like magic or stealth.
\end{description}

\section*{Using Followers} \index{Followers!usage}

\subsection*{Assistance in Scenes}

Followers can help with your actions:
\begin{itemize}
\item Assist Dice: Add dice equal to min(Cap, relevant skill).
\item Maximum Bonus: +3 dice total from all sources.
\item Cost: Spend 1 Boon or 1 Stress to add +1 die (max +3 from assists).
\item One Helper: Only one follower can assist per action.
\end{itemize}

\subsection*{Independent Actions}

Once per scene (party-wide), a follower can take a small action:
\begin{description}
\item[Scout \& Signal] Change an ally's next action to Controlled position.
\item[Distract \& Draw] Reduce a threat clock by 1 segment.
\item[Fetch \& Carry] Move an object through danger safely.
\end{description}

\subsection*{Cost of Independent Actions}

\begin{itemize}
\item Mark +1 Exposure (attention or stress), or
\item Take Harm 1 (injury or trauma).
\item Cannot be used if the follower is already Compromised.
\end{itemize}

\section*{Follower Conditions} \index{Followers!conditions}

Followers track two condition types:

\subsection*{Exposure}

Represents attention, stress, or narrative pressure:
\begin{itemize}
\item Gains: From independent actions, dangerous situations, complications.
\item Effects: Increased risk, reduced effectiveness, attention from enemies.
\item Recovery: Downtime activities, careful management.
\end{itemize}

\subsection*{Harm}

Represents injury, trauma, or damage:
\begin{itemize}
\item Gains: From combat, accidents, enemy attacks.
\item Effects: Penalties to assistance, possible incapacity.
\item Recovery: Medical care, rest, magical healing.
\end{itemize}

\subsection*{Condition States}

\begin{description}
\item[Maintained] Ready and reliable, full capability.
\item[Neglected] Needs attention, −1 die to assistance.
\item[Compromised] Unavailable: captured, defected, lost, or incapacitated.
\end{description}

\section*{Follower Risks} \index{Followers!risks}

Using followers carries significant risks:

\subsection*{Complication Targeting}

When the GM spends 2+ Story Beats on an action where you have assistance:
\begin{itemize}
\item The follower may face consequences instead of you.
\item Could be injury, capture, betrayal, or other complications.
\item Fictionally appropriate to the situation.
\end{itemize}

\subsection*{Off-Screen Capability}

Once per downtime, a Cap 5 follower can solve a significant problem:
\begin{itemize}
\item But generates 1 Story Beat for the party.
\item The GM describes how this creates story consequences.
\item Useful for emergencies but costly.
\end{itemize}

\section*{Upkeep and Maintenance} \index{Assets!upkeep}\index{Followers!upkeep}

Both assets and followers require regular maintenance.

\subsection*{Asset Upkeep}

Two options per SRD §21.2:
\begin{itemize}
\item Option 1 - Efficient (Higher XP, Less Time):
\begin{itemize}
\item Cost: Pay Upkeep XP = max(1, AcquisitionXP )/3
\item Time: Minimal effort
\end{itemize}
\item Option 2 - Intensive (Lower XP, More Time):
\begin{itemize}
\item Cost: Pay 1 XP
\item Time: Dedicated downtime action with significant personal involvement
\end{itemize}
\item Failure to Pay: Asset becomes Neglected (or Compromised if already Neglected)
\end{itemize}

\subsection*{Follower Upkeep}

Two options per SRD §21.2:
\begin{itemize}
\item Option 1 - Efficient:
\begin{itemize}
\item Cost: Pay Upkeep XP = max(1, Cap$^2$)/3
\item Time: Minimal effort
\end{itemize}
\item Option 2 - Intensive:
\begin{itemize}
\item Cost: Pay 1 XP
\item Time: Dedicated downtime action with significant personal involvement
\end{itemize}
\item Failure to Pay: Follower becomes Wary (or Seized if already Wary)
\end{itemize}

\section*{Strategic Considerations} \index{Assets!strategy}\index{Followers!strategy}

\subsection*{When to Invest in Assets}

\begin{itemize}
\item You need reliable off-screen capabilities.
\item Your character concept involves wealth or influence.
\item The party lacks certain logistical support.
\item You want to build long-term influence.
\end{itemize}

\subsection*{When to Invest in Followers}

\begin{itemize}
\item You need on-screen assistance.
\item Your character works better with support.
\item The party needs specific capabilities you lack.
\item You want character-driven story opportunities.
\end{itemize}

\subsection*{Balance Recommendations}

\begin{center}
\begin{tabular}{|l|l|}
\hline
\textbf{Path} & \textbf{Investment} \\
\hline
Personal Path & 0–10\% assets/followers \\
Balanced Path & 15–25\% assets/followers \\
Influencer Path & 35–55\% assets/followers \\
\hline
\end{tabular}
\end{center}

\section*{Loyalty and Relationships} \index{Followers!loyalty}

\subsection*{Loyalty Levels}

Optional system for tracking follower loyalty:
\begin{description}
\item[Wary] Cautious, may leave if pressured; +1 XP upkeep cost.
\item[Steady] Reliable, standard performance; normal upkeep.
\item[Devoted] Loyal, may sacrifice; can convert one major complication to a minor setback per arc.
\end{description}

\subsection*{Building Loyalty}

\begin{itemize}
\item Fair treatment and respect.
\item Sharing rewards and successes.
\item Protecting followers from harm.
\item Honoring agreements and promises.
\end{itemize}

\subsection*{Losing Loyalty}

\begin{itemize}
\item Mistreatment or disrespect.
\item Unreasonable demands or risks.
\item Broken promises or betrayal.
\item Consistent neglect.
\end{itemize}

\section*{Advanced Follower Management} \index{Followers!advanced}

\subsection*{Follower Groups}

For multiple similar followers, you can manage them as a group:
\begin{itemize}
\item Single Rating: Treat as one entity with combined capability.
\item Condition Tracking: Group shares exposure and harm.
\item Maintenance: Single upkeep cost for the group.
\item Risks: Problems affect the entire group.
\end{itemize}

\subsection*{Follower Advancement}

Followers can improve over time:
\begin{itemize}
\item Experience: Gain capability through successful assistance.
\item Training: Spend XP to improve follower capabilities.
\item Equipment: Better gear can enhance effectiveness.
\item Limits: Followers typically cap at lower levels than PCs.
\end{itemize}

\section*{Risk Management} \index{Assets!risk management}\index{Followers!risk management}

\subsection*{Asset Risks}

\begin{itemize}
\item Financial: Assets can be costly to maintain.
\item Security: Assets can be attacked or stolen.
\item Attention: Valuable assets draw unwanted notice.
\item Dependency: Over-reliance can be problematic.
\end{itemize}

\subsection*{Follower Risks}

\begin{itemize}
\item Safety: Followers can be harmed or captured.
\item Loyalty: Followers may betray or leave.
\item Attention: Followers can draw enemy interest.
\item Morale: Followers have needs and limits.
\end{itemize}

\subsection*{Mitigation Strategies}

\begin{itemize}
\item Diversification: Don't put all resources in one place.
\item Security: Protect valuable assets and followers.
\item Relationships: Maintain good terms with your people.
\item Contingencies: Have backup plans for losses.
\end{itemize}

\begin{tcolorbox}[colback=gray!5!white, colframe=gray!75!black, title=Assets and Followers Quick Reference, fonttitle=\bfseries]
\textbf{Assets:}
\begin{itemize}
\item Minor: 4 XP | Standard: 8 XP | Major: 12 XP
\item Free off-screen use: once per session
\item Boon activation: spend 1 Boon for scene impact
\item Conditions: Maintained → Neglected → Compromised
\end{itemize}

\textbf{Followers:}
\begin{itemize}
\item Cost: Cap$^2$ XP
\item Assistance: + min(Cap, skill) dice (max +3 from all sources)
\item Independent action: once per scene (party-wide)
\item Conditions: Exposure and Harm tracks
\end{itemize}

\textbf{Upkeep Options:}
\begin{itemize}
\item Efficient: max(1, Cost)/3 XP, minimal time
\item Intensive: 1 XP, dedicated downtime action
\end{itemize}
\end{tcolorbox}

\section*{Practical Examples} \index{Assets!examples}\index{Followers!examples}

\subsection*{Asset Example: The Safehouse}

\begin{itemize}
\item Type: Minor Asset (4 XP)
\item Free Use: Secure lodging, basic supplies between adventures.
\item Boon Activation: Reveal a hidden escape route during pursuit.
\item Upkeep: Option 1: 2 XP (4/3 rounded up) or Option 2: 1 XP + downtime action.
\item Risks: Discovery by enemies, maintenance costs.
\end{itemize}

\subsection*{Follower Example: The Scout}

\begin{itemize}
\item Capability: 3 (9 XP cost)
\item Assistance: +3 dice on tracking and survival rolls.
\item Independent Action: Scout ahead to improve party position.
\item Upkeep: Option 1: 3 XP (9/3) or Option 2: 1 XP + downtime action.
\item Risks: Injury in dangerous scouting; disloyalty if mistreated.
\end{itemize}

\subsection*{Combination Example: The Merchant}

\begin{itemize}
\item Assets: Trading post (8 XP), caravan (4 XP) — 12 XP total
\item Followers: Cap 2 guards (4 XP each = 8 XP), Cap 3 factor (9 XP) — 17 XP total
\item Total Investment: 29 XP in assets and followers
\item Upkeep (Efficient Option): Assets 4 XP + Followers 6 XP = 10 XP per downtime period
\item Benefits: Trade income, transport, protection, business contacts
\item Risks: Competition, bandit attacks, employee issues, regulatory attention
\end{itemize}

\begin{tcolorbox}[colback=gray!5!white, colframe=gray!75!black, title=Remember, fonttitle=\bfseries]
Assets and followers can greatly expand your capabilities, but they require careful management and carry significant risks. Invest wisely based on your character concept and the needs of your group. The SRD provides flexible upkeep options to suit different play styles and campaign pacing.
\end{tcolorbox}

\section*{Narrative-Heavy Asset and Follower Options} \index{Narrative-Heavy Gameplay!Assets and Followers}

For groups that prefer strong narrative focus in asset and follower management, consider these optional approaches:

\textbf{Story-Driven Upkeep:} Instead of tracking XP costs for upkeep, the GM can introduce narrative complications that require attention. A neglected asset might attract unwanted attention, while a neglected follower might request a favor or special treatment.

\textbf{Collaborative Management:} Players can describe how they maintain their assets and followers through roleplay rather than mechanical upkeep costs. A well-described scene of tending to a workshop or bonding with followers can fulfill maintenance requirements.

\textbf{Asset and Follower as Character Development:} Use asset and follower management as opportunities for character growth and backstory development, allowing players to narrate how their relationships and holdings evolve through significant story moments.

\textbf{Flexible Condition Tracking:} Focus on the narrative implications of asset and follower conditions rather than strict mechanical penalties. A "Neglected" asset might still function but with interesting complications, while a "Compromised" asset might require creative solutions rather than just XP investment.

\chapter{World Interaction} \label{ch:world}

In Fate's Edge, the world is not a backdrop—it's a partner in the conversation. Dikes groan under black rain in Viterra, clan horns answer across Acasia's ridgelines, Ecktoria's marble halls echo with careful words, and Kahfagia's pilots read storms by taste. Wherever you go, place, culture, and pressure push back.

\section*{Game Structure and Time} \index{Game Structure}

Understanding how time works in Fate's Edge helps you navigate both the mechanical and narrative flow of play.

\subsection*{Basic Units}

\begin{description}
\item[Scene] The basic unit of narrative play, covering a specific situation or conflict (Some Time to Significant Time). Resolves a particular question or challenge.
\item[Player Turn (Beat)] An individual player's action within a scene: Declare action → GM sets position → roll → resolve outcome → manage consequences.
\item[Round] Simultaneous or near-simultaneous actions within a scene (primarily for combat), representing a few seconds of real time.
\item[Session] One complete game session (typically 3–6 hours), containing 2–4 major scenes and resolving significant narrative progress.
\item[Downtime] The narrative time between scenes, used for recovery, advancement, and off-screen activities. Measured in days, weeks, or months depending on fiction.
\item[Campaign] Entire story arc (6–20+ sessions) with major character development and lasting consequences.
\end{description}

\section*{Movement and Positioning} \index{Movement}

Space is tracked with range bands and Position.

\subsection*{Range Bands}

\begin{description}
\item[Close] Touching distance: grapples, knife-work, hand on a relic.
\item[Near] Same room/yard/deck; a rush away.
\item[Far] Same site but distant; requires route or time to reach.
\item[Absent] Off-screen; requires scene change or significant effort to interact.
\end{description}

\subsection*{Movement Actions}

\begin{itemize}
\item Move: Shift one range band as a beat.
\item Dash: Shift two bands as your full action (terrain may require a roll).
\item Melee Flag: Mark when two parties are in Near range and directly engaged in combat.
\end{itemize}

\subsection*{Position States}

\begin{description}
\item[Controlled] You have cover, leverage, or ritual footing. Failure still leaves options.
\item[Risky] Standard case: exposed lanes, rivals near, watchful eyes. Failure has teeth, but not ruin.
\item[Desperate] Bad ground, bad odds, bad timing. Failure is severe; success may bring extra XP.
\end{description}

\subsection*{Position Shifting}

\begin{itemize}
\item GM can spend 1 SB to worsen Position by one step.
\item Player can spend 1 Boon to improve Position by one step (once per action).
\item Narrative triggers (flanking, reinforcements, etc.) can shift Position without cost.
\end{itemize}

\section*{Travel Framework} \index{Travel}

Travel abstracts distance into legs with tension and color rather than miles and meal counts. Each leg has a Travel Clock and draws on a regional deck to seed fiction.

\subsection*{Travel Process}

\begin{enumerate}
\item Set the Leg: Name origin and destination; start a Travel Clock (4-10 segments based on difficulty).
\item Draw Prompts: Draw up to one card from each suit to establish terrain, people, pressures, and leverage.
\item Assign Roles: Players take on travel roles (Guide, Scout, Quartermaster, Watch) to contribute actions.
\item Play the Leg: Players take actions to advance the clock or mitigate complications. GM spends SB from rolls showing 1s to introduce hazards.
\item Resolve: When the clock fills, you arrive—changed by the journey.
\end{enumerate}

\subsection*{Using Assets and Followers During Travel}

\begin{itemize}
\item Assets: Spend 1 Boon to activate an asset for dramatic effect during travel (reveal hidden path, call for emergency aid, etc.).
\item Followers: Assign followers to travel roles for bonuses. A Cap 3 Scout follower adds +3 to navigation rolls, for example.
\item Independent Actions: Once per travel leg, a follower can take an independent action (scout ahead, secure supplies, etc.) at the cost of Exposure or Harm.
\item Off-Screen Solutions: High-Cap followers (4-5) can solve significant travel problems once per downtime, but generate 1 SB for the party.
\end{itemize}

\subsection*{Regional Travel Decks}

Each major region has a themed prompt list or card table:
\begin{description}
\item[Viterra] Fen causeways, dike-brotherhoods, crown law.
\item[Acasia] Border-lace titles, ruined towers, clan tempers.
\item[Ecktoria] Imperial roads, precinct gates, temple schedules.
\item[Ubral] Stone passes, toll-cloisters, ghosted fields.
\item[Kahfagia] Current maps, pilot-mirrors, storm lanes.
\item[Aelinnel] Mist paths, bell-mounds, spirit ways.
\end{description}

\subsection*{Travel Complications}

\begin{itemize}
\item Hazards: Weather, terrain challenges, wildlife encounters.
\item Social: Border checks, local politics, cultural misunderstandings.
\item Supplies: Food shortages, equipment failure, resource management.
\item Pursuit: Being followed, hunted, or racing against time.
\end{itemize}

\section*{Narrative Time} \index{Time}

Time is measured by importance rather than duration.

\begin{description}
\item[A Moment] A glance, a strike, a whisper over a law-stone.
\item[Some Time] A skirmish, a negotiation, a careful climb.
\item[Significant Time] Hours of march, rites, audits, stakeouts.
\item[Days] Drills, recoveries, research, roadwork.
\end{description}

\section*{Social Interactions} \index{Social Interactions}

Social scenes use the same engine with cultural color.

\subsection*{Cultural Skill Emphases}

\begin{description}
\item[Viterra] Rapport with parishes; Sway for markets; Command under writ.
\item[Acasia] Rapport for kin-bridges; Command with banner-rights; Deceive risks honor clocks.
\item[Ecktoria] Sway in salons; Deceive at court; Perform in temple fora.
\item[Kahfagia] Rapport aboard; Sway at piers; Command on a storming deck.
\end{description}

\subsection*{Social Stakes \& Clocks}

\begin{itemize}
\item Alliance Clock (Viterra): Parishes and guilds come to your side.
\item Honor Clock (Acasia): Feasts, oaths, wyrd—trust builds (or frays).
\item Bureau Clock (Ecktoria): Stamps, seals, approvals—delay is pressure.
\item Trust Clock (Kahfagia): Pilots and crews extend favors and routes.
\end{itemize}

\section*{Supply and Resources} \index{Resources}

Track scarcity with a Supply Clock shared by the party's expedition.

\subsection*{Segments}

\begin{center}
\begin{tabular}{|c|l|l|}
\hline
\textbf{Segments} & \textbf{State \& Effects} & \textbf{Notes} \\
\hline
0 (Full) & Well-provisioned; no penalty. & \\
2 (Low) & Minor frictions; -1 to resource checks. & \\
3 (Dangerous) & Each PC gains Fatigue 1. & \\
4 (Empty) & Severe penalties; desperate measures. & \\
\hline
\end{tabular}
\end{center}

\section*{Engaging the World—Player Actions} \index{Player Actions}

\begin{itemize}
\item Scout \& Signal: A follower can make the next travel action Controlled (mark Exposure or Harm 1 on them).
\item Local Color: Briefly state what locals notice about you; GM offers a small fictional edge or a tempting clock—choose.
\item Mark the Map: On arrival, declare one change to the fiction (new ford, patron's shrine, toll-skip). GM may attach a minor clock as cost.
\item Asset Activation: Spend 1 Boon to activate an asset dramatically during a scene.
\item Follower Assistance: Have a follower assist your actions for bonus dice (max +3 from all sources).
\end{itemize}

\section*{Summary} \index{World Interaction!summary}

The world has opinions. Movement is clocks and color, position rises and sinks with weather and words, and every suit you draw speaks in a regional accent. Ask the land for a favor—then pay it back on the road.

\begin{tcolorbox}[colback=gray!5!white, colframe=gray!75!black, title=Remember, fonttitle=\bfseries]
Every interaction with the world is an opportunity. Use your assets, deploy your followers, and engage with the setting actively. The world responds to your choices, and every journey changes both you and the places you pass through.
\end{tcolorbox}

\chapter{Example Character Concepts} \label{ch:examples}

This chapter presents example character concepts to illustrate how the game's systems can create diverse and interesting heroes. These are examples only—not prescriptive templates or exhaustive lists. Use them for inspiration, as pre-generated characters, or as starting points for your own unique creations.

\section*{Important Disclaimer} \index{Character Creation!disclaimer}

These examples are provided for illustrative purposes only. They demonstrate how the game's mechanics can support different character archetypes and play styles. You are encouraged to:
\begin{itemize}
\item Modify these concepts to fit your preferences
\item Create completely original characters
\item Mix and match elements from different examples
\item Work with your Game Master to develop unique concepts
\end{itemize}

The game system is designed to support a wide variety of character types beyond these examples.

\section*{How to Use These Examples} \index{Character Creation!usage}

Each concept includes:
\begin{itemize}
\item Concept Overview: Narrative identity and role
\item Mechanical Foundation: Suggested starting capabilities
\item Play Style: How the character typically engages with challenges
\item Development Path: Potential growth directions
\item Story Hooks: Plot opportunities for the Game Master
\item Build Blocks: A 30 XP starting build, plus an optional 34 XP variant using Bonds/Complications (+4 XP)
\end{itemize}

\section*{1. The Guardian} \index{Character Examples!Guardian}

\textbf{Concept:} A protector who stands between danger and those they've sworn to defend. Steel in hand, vow in heart.

\textbf{Typical Inspiration:} Paladins, knights, bodyguards, sworn shields

\textbf{Mechanical Foundation:}
\begin{itemize}
\item Primary: Body, Spirit
\item Skills: Melee, Athletics, Command
\item Talents: Defensive stance, protective instincts
\end{itemize}

\textbf{Play Style:}
\begin{itemize}
\item Frontline combat and protection
\item Drawing attention away from allies
\item Using presence and authority to control situations
\item Taking risks to protect others
\end{itemize}

\textbf{Development Path:}
\begin{itemize}
\item Increase defensive capabilities
\item Develop leadership skills
\item Acquire better protective gear
\item Learn area control abilities
\end{itemize}

\textbf{Story Hooks:}
\begin{itemize}
\item Who or what are they protecting?
\item What oath or duty drives them?
\item What happens if they fail in their protection?
\item What personal costs do they bear for their role?
\end{itemize}

\textbf{Build Blocks. Starting Build (30 XP).}
\begin{itemize}
\item Attributes (Cost = rating × 3 XP): Body 3 (9), Spirit 2 (6), Wits 1 (3), Presence 1 (3) → 21 XP
\item Skills (Cost = level × 2 XP): Melee 2 (4), Athletics 1 (2), Command 1 (2) → 8 XP
\item Total: 29 XP (bank 1 XP)
\end{itemize}

\textbf{With Bonds/Complications (34 XP).}
\begin{itemize}
\item Add Talent: Combat Reflexes (5 XP) using banked 1 + 4 = 5 XP
\item Revised Total: 34 XP
\end{itemize}

\section*{2. The Scholar} \index{Character Examples!Scholar}

\textbf{Concept:} A seeker of knowledge who uses information as power. Candlesmoke, marginalia, and dangerous truths.

\textbf{Typical Inspiration:} Wizards, sages, researchers, historians

\textbf{Mechanical Foundation:}
\begin{itemize}
\item Primary: Wits, Spirit
\item Skills: Lore, Investigation, Arcana
\item Talents: Quick Study, Research Mastery
\end{itemize}

\textbf{Play Style:}
\begin{itemize}
\item Information gathering and analysis
\item Solving puzzles and mysteries
\item Using knowledge to gain advantages
\item Researching solutions between adventures
\end{itemize}

\textbf{Development Path:}
\begin{itemize}
\item Specialize in specific knowledge areas
\item Develop magical or technical capabilities
\item Build research networks
\item Create unique inventions or discoveries
\end{itemize}

\textbf{Story Hooks:}
\begin{itemize}
\item What knowledge are they seeking?
\item What dangerous information might they uncover?
\item How do they handle forbidden knowledge?
\item Who opposes their research?
\end{itemize}

\textbf{Build Blocks. Starting Build (30 XP).}
\begin{itemize}
\item Attributes: Wits 3 (9), Spirit 2 (6), Body 1 (3), Presence 1 (3) → 21 XP
\item Skills: Lore 2 (4), Investigation 1 (2), Arcana 1 (2) → 8 XP
\item Total: 29 XP (bank 1 XP)
\end{itemize}

\textbf{With Bonds/Complications (34 XP).}
\begin{itemize}
\item Add Talent: Research Mastery (5 XP) using banked 1 + 4 = 5 XP
\item Revised Total: 34 XP
\end{itemize}

\section*{3. The Scout} \index{Character Examples!Scout}

\textbf{Concept:} A wilderness expert who navigates dangerous territories. Quiet footfalls, hawk eyes, and the long road.

\textbf{Typical Inspiration:} Rangers, hunters, trackers, explorers

\textbf{Mechanical Foundation:}
\begin{itemize}
\item Primary: Wits, Body
\item Skills: Survival, Stealth, Perception
\item Talents: Wilderness Lore, Keen Senses
\end{itemize}

\textbf{Play Style:}
\begin{itemize}
\item Scouting ahead and gathering intelligence
\item Wilderness survival and navigation
\item Ambush and skirmish tactics
\item Finding paths and resources
\end{itemize}

\textbf{Development Path:}
\begin{itemize}
\item Improve stealth and tracking abilities
\item Develop animal companions or allies
\item Master specific environments
\item Learn advanced survival techniques
\end{itemize}

\textbf{Story Hooks:}
\begin{itemize}
\item What uncharted territory are they exploring?
\item What secrets have they discovered in the wild?
\item How do they balance civilization and wilderness?
\item What threats have they encountered beyond settled lands?
\end{itemize}

\textbf{Build Blocks. Starting Build (30 XP).}
\begin{itemize}
\item Attributes: Wits 3 (9), Body 2 (6), Spirit 1 (3), Presence 1 (3) → 21 XP
\item Skills: Survival 2 (4), Stealth 2 (4) → 8 XP
\item Total: 29 XP (bank 1 XP)
\end{itemize}

\textbf{With Bonds/Complications (34 XP).}
\begin{itemize}
\item Add Asset: Hidden Cache (Minor Asset, 4 XP) using banked 1 + 4 = 5 XP
\item Revised Total: 34 XP
\end{itemize}

\section*{4. The Diplomat} \index{Character Examples!Diplomat}

\textbf{Concept:} A negotiator who resolves conflicts through words and influence. A smile for the foyer, steel for the parlor.

\textbf{Typical Inspiration:} Bards, ambassadors, merchants, politicians

\textbf{Mechanical Foundation:}
\begin{itemize}
\item Primary: Presence, Wits
\item Skills: Sway, Investigation, Lore
\item Talents: Silver Tongue, Read Emotions
\end{itemize}

\textbf{Play Style:}
\begin{itemize}
\item Social interaction and negotiation
\item Gathering information through contacts
\item Resolving conflicts without violence
\item Building alliances and relationships
\end{itemize}

\textbf{Development Path:}
\begin{itemize}
\item Expand social influence and networks
\item Develop economic or political power
\item Learn cultural specialties
\item Master manipulation or inspiration techniques
\end{itemize}

\textbf{Story Hooks:}
\begin{itemize}
\item What major conflict are they trying to resolve?
\item What alliances have they built or broken?
\item How do they handle betrayal or failed negotiations?
\item What personal relationships affect their diplomacy?
\end{itemize}

\textbf{Build Blocks. Starting Build (30 XP).}
\begin{itemize}
\item Attributes: Presence 3 (9), Wits 2 (6), Spirit 1 (3), Body 1 (3) → 21 XP
\item Skills: Sway 2 (4), Investigation 1 (2), Lore 1 (2) → 8 XP
\item Total: 29 XP (bank 1 XP)
\end{itemize}

\textbf{With Bonds/Complications (34 XP).}
\begin{itemize}
\item Add Talent: Silver Tongue (3 XP) and Skill: Lore +1 (now 2) for 2 XP using banked 1 + 4 = 5 XP
\item Revised Total: 34 XP
\end{itemize}

\section*{5. The Specialist} \index{Character Examples!Specialist}

\textbf{Concept:} An expert with unique capabilities beyond typical roles. The right tool, the right touch, at the right time.

\textbf{Typical Inspiration:} Artisans, healers, engineers, spies

\textbf{Mechanical Foundation:}
\begin{itemize}
\item Primary: Varies by specialty (often Wits or Body)
\item Skills: One specialty at focus, plus two support skills
\item Talents: Unique techniques that unlock niche actions
\end{itemize}

\textbf{Play Style:}
\begin{itemize}
\item Solving problems with unique expertise
\item Creating or repairing specialized items
\item Providing services others cannot
\item Using niche knowledge for advantage
\end{itemize}

\textbf{Development Path:}
\begin{itemize}
\item Master their specialty area
\item Develop related capabilities
\item Build reputation and clientele
\item Create unique inventions or methods
\end{itemize}

\textbf{Story Hooks:}
\begin{itemize}
\item What makes their specialty unique or valuable?
\item How did they acquire their special skills?
\item What problems require their specific expertise?
\item Who seeks to control or exploit their abilities?
\end{itemize}

\textbf{Build Blocks (Artificer example). Starting Build (30 XP).}
\begin{itemize}
\item Attributes: Wits 3 (9), Body 2 (6), Presence 1 (3), Spirit 1 (3) → 21 XP
\item Skills: Craft 2 (4), Mechanics 2 (4) → 8 XP
\item Total: 29 XP (bank 1 XP)
\end{itemize}

\textbf{With Bonds/Complications (34 XP).}
\begin{itemize}
\item Add Talent: Technical Expert (6 XP) - need 6 XP but have 5 XP available (1 banked + 4 from Bonds/Complications)
\item Alternative: Add Talent: Quick Study (3 XP) and bank 2 XP for future use
\item Revised Total: 32 XP (bank 2 XP)
\end{itemize}

\section*{6. The Survivor} \index{Character Examples!Survivor}

\textbf{Concept:} Someone who has endured hardship and developed resilience. Scars are maps; read them well.

\textbf{Typical Inspiration:} Veterans, refugees, outcasts, hardened adventurers

\textbf{Mechanical Foundation:}
\begin{itemize}
\item Primary: Spirit, Body
\item Skills: Endurance, Survival, (optionally) Perception/Insight
\item Talents: Endurance, Adaptable
\end{itemize}

\textbf{Play Style:}
\begin{itemize}
\item Enduring difficult conditions
\item Overcoming physical and mental challenges
\item Using experience to avoid dangers
\item Helping others survive hardships
\end{itemize}

\textbf{Development Path:}
\begin{itemize}
\item Improve physical and mental resilience
\item Develop survival-related skills
\item
```latex
Acquire better equipment and resources
\item Learn to teach survival to others
\end{itemize}

\textbf{Story Hooks:}
\begin{itemize}
\item What trauma or hardship have they survived?
\item How has their past shaped their present?
\item What survival skills have saved them repeatedly?
\item How do they help others facing similar challenges?
\end{itemize}

\textbf{Build Blocks. Starting Build (30 XP).}
\begin{itemize}
\item Attributes: Spirit 3 (9), Body 2 (6), Wits 1 (3), Presence 1 (3) → 21 XP
\item Skills: Endurance 2 (4), Survival 2 (4) → 8 XP
\item Total: 29 XP (bank 1 XP)
\end{itemize}

\textbf{With Bonds/Complications (34 XP).}
\begin{itemize}
\item Add Talent: Endurance (3 XP) using banked 1 + 4 = 5 XP; bank 2 XP
\item Revised Total: 32 XP (bank 2 XP)
\end{itemize}

\section*{7. The Innovator} \index{Character Examples!Innovator}

\textbf{Concept:} A creative problem-solver who finds new solutions. Blueprints on napkins, tomorrow in your pocket.

\textbf{Typical Inspiration:} Inventors, strategists, reformers, visionaries

\textbf{Mechanical Foundation:}
\begin{itemize}
\item Primary: Wits, Presence
\item Skills: Craft, Lore, Investigation
\item Talents: Creative/Innovative thinking, Quick Study
\end{itemize}

\textbf{Play Style:}
\begin{itemize}
\item Finding novel solutions to problems
\item Creating new devices or methods
\item Analyzing systems for improvement
\item Convincing others to try new approaches
\end{itemize}

\textbf{Development Path:}
\begin{itemize}
\item Develop specific technical specialties
\item Create increasingly complex inventions
\item Build support for innovative ideas
\item Overcome resistance to change
\end{itemize}

\textbf{Story Hooks:}
\begin{itemize}
\item What problem are they trying to solve?
\item How do others react to their innovations?
\item What unintended consequences might their creations have?
\item Who benefits or suffers from their changes?
\end{itemize}

\textbf{Build Blocks. Starting Build (30 XP).}
\begin{itemize}
\item Attributes: Wits 3 (9), Presence 2 (6), Body 1 (3), Spirit 1 (3) → 21 XP
\item Skills: Craft 2 (4), Lore 2 (4) → 8 XP
\item Total: 29 XP (bank 1 XP)
\end{itemize}

\textbf{With Bonds/Complications (34 XP).}
\begin{itemize}
\item Add Talent: Quick Study (3 XP) using banked 1 + 4 = 5 XP; bank 2 XP
\item Revised Total: 32 XP (bank 2 XP)
\end{itemize}

\section*{8. The Networker} \index{Character Examples!Networker}

\textbf{Concept:} Someone who builds and leverages social connections. A web of favors, a chorus of names.

\textbf{Typical Inspiration:} Merchants, spies, socialites, community leaders

\textbf{Mechanical Foundation:}
\begin{itemize}
\item Primary: Presence, Wits
\item Skills: Sway, Lore, (optionally) Command/Deception
\item Talents: Network Builder, Command Presence / Silver Tongue
\end{itemize}

\textbf{Play Style:}
\begin{itemize}
\item Building and maintaining relationships
\item Gathering information through contacts
\item Leveraging social influence
\item Navigating complex social situations
\end{itemize}

\textbf{Development Path:}
\begin{itemize}
\item Expand social network and influence
\item Develop specific community ties
\item Acquire political or economic power
\item Master manipulation or leadership techniques
\end{itemize}

\textbf{Story Hooks:}
\begin{itemize}
\item What networks or communities are they part of?
\item How do they balance multiple relationships?
\item What happens when loyalties conflict?
\item How do they handle betrayal or broken trust?
\end{itemize}

\textbf{Build Blocks. Starting Build (30 XP).}
\begin{itemize}
\item Attributes: Presence 3 (9), Wits 2 (6), Body 1 (3), Spirit 1 (3) → 21 XP
\item Skills: Sway 2 (4), Lore 2 (4) → 8 XP
\item Total: 29 XP (bank 1 XP)
\end{itemize}

\textbf{With Bonds/Complications (34 XP).}
\begin{itemize}
\item Add Talent: Silver Tongue (3 XP) using banked 1 + 4 = 5 XP; bank 2 XP
\item Revised Total: 32 XP (bank 2 XP)
\end{itemize}

\section*{Creating Your Own Concept} \index{Character Creation!original}

Start with Narrative
\begin{itemize}
\item What is your character's background and motivation?
\item What role do they play in their community or society?
\item What relationships are important to them?
\item What goals are they pursuing?
\end{itemize}

Add Mechanical Support
\begin{itemize}
\item Choose attributes that support your concept
\item Select skills that reflect their training and experience
\item Consider talents that provide unique capabilities
\item Think about assets that represent their resources
\end{itemize}

Consider Group Role
\begin{itemize}
\item How does your concept complement other party members?
\item What gaps in group capability can you fill?
\item What unique contributions can you make?
\item How will you work with other characters?
\end{itemize}

Plan for Growth
\begin{itemize}
\item What short-term improvements make sense?
\item What long-term development aligns with your concept?
\item How might your character change over time?
\item What legacy do you want to build?
\end{itemize}

\begin{tcolorbox}[colback=gray!5!white, colframe=gray!75!black, title=Character Concept Worksheet, fonttitle=\bfseries]
\textbf{Narrative Elements:}
\begin{itemize}
\item Concept: \_\_\_\_\_\_\_\_\_\_\_\_\_\_\_\_\_\_\_\_\_\_\_\_\_\_\_\_\_\_\_\_\_\_\_\_\_\_\_\_
\item Motivation: \_\_\_\_\_\_\_\_\_\_\_\_\_\_\_\_\_\_\_\_\_\_\_\_\_\_\_\_\_\_\_\_\_\_\_\_\_\_\_\_
\item Background: \_\_\_\_\_\_\_\_\_\_\_\_\_\_\_\_\_\_\_\_\_\_\_\_\_\_\_\_\_\_\_\_\_\_\_\_\_\_\_\_
\item Relationships: \_\_\_\_\_\_\_\_\_\_\_\_\_\_\_\_\_\_\_\_\_\_\_\_\_\_\_\_\_\_\_\_\_\_\_\_\_\_\_\_
\end{itemize}

\textbf{Mechanical Foundation:}
\begin{itemize}
\item Primary Attributes: \_\_\_\_\_\_\_\_\_\_\_\_\_\_\_\_\_\_\_\_\_\_\_\_\_\_\_\_\_\_\_\_\_\_\_\_\_\_\_\_
\item Key Skills: \_\_\_\_\_\_\_\_\_\_\_\_\_\_\_\_\_\_\_\_\_\_\_\_\_\_\_\_\_\_\_\_\_\_\_\_\_\_\_\_
\item Starting Talents: \_\_\_\_\_\_\_\_\_\_\_\_\_\_\_\_\_\_\_\_\_\_\_\_\_\_\_\_\_\_\_\_\_\_\_\_\_\_\_\_
\item Initial Assets: \_\_\_\_\_\_\_\_\_\_\_\_\_\_\_\_\_\_\_\_\_\_\_\_\_\_\_\_\_\_\_\_\_\_\_\_\_\_\_\_
\end{itemize}

\textbf{Development Plan:}
\begin{itemize}
\item Short-term goals: \_\_\_\_\_\_\_\_\_\_\_\_\_\_\_\_\_\_\_\_\_\_\_\_\_\_\_\_\_\_\_\_\_\_\_\_\_\_\_\_
\item Long-term vision: \_\_\_\_\_\_\_\_\_\_\_\_\_\_\_\_\_\_\_\_\_\_\_\_\_\_\_\_\_\_\_\_\_\_\_\_\_\_\_\_
\end{itemize}
\end{tcolorbox}

\section*{Final Notes} \index{Character Creation!conclusion}

Remember that these examples are starting points, not limitations. The most interesting characters often combine elements from multiple concepts or create entirely new approaches. Work with your Game Master to ensure your character concept fits the campaign and provides engaging story opportunities.

The best characters are those that you find interesting to play and that contribute to an enjoyable experience for everyone at the table.

\chapter{World Regions and Cultures} \label{ch:regions}

The world of Fate's Edge is a tapestry of ancient empires, emerging kingdoms, and untamed wilderness. This chapter surveys major regions and cultures that shape the setting—from the marble cities of Ecktoria to the mist-shrouded fields of Aelinnel. These frameworks are yours to adapt, blend, or reimagine.

\section*{The Amaranthine Inland Sea} \index{Regions!Amaranthine Sea}

At the heart of the known world lies the Amaranthine Inland Sea, a wind-gnarled waterway ringed by marble quays, vineyard hills, and smoke-blue mountains. For millennia it has served as the circulatory system of trade, faith, and conquest. Tides are subtle, but seasonal winds and river-feeds set the rhythm of commerce, pilgrimage, and war.

\section*{Northern Shore of the Amaranthine Sea} \index{Regions!Northern Shore}

\subsection*{Ecktoria — The Utaran Imperium Successor}

Once the furnace of empire (Marble \& Fire), Ecktoria remains a palimpsest of power: old stones bearing new banners, old laws written under fresh seals. Though imperial reach waned, its civic habits endure.

\textbf{Marble Cities} Forums, amphitheaters, and aqueducts yet flow. District fountains double as public oaths guaranteed by guild charters.

\textbf{Imperial Roads} Mile-markers of white granite, way-shrines and customary tolls noted for couriers of the Ashen Staves.

\textbf{Legal Legacy} The Utaran Civic Codes govern contracts, inheritance, and war-rights; local custom bends them under licensed variance.

\textbf{Architectural Wonders} Sun-bridges spanning deltas, the Vault of a Thousand Maps, and the Amber Arch petrified by alchemical storm.

\subsection*{Acasia — "The Broken Province"}

Frontiers braided from roads, rivers, and resentments. Here the outer seams of empire frayed first. Fortresses turned manors, manors turned townholds, and banners multiplied like thistles after rain.

\textbf{Petty Kingdoms} Dozens of river-vales ruled by river-kings and banner-queens. Alliances shift with marriages, harvests, and omens.

\textbf{Fortified Towns} Walls for defense, not display. Gate-streets kink for ambush; towers carry horncodes every child knows.

\textbf{Mercenary Culture} Free companies keep a Black Ledger: contracts fulfilled, oaths kept, debts paid.

\textbf{Cultural Mix} Imperial rites meet clan feasts; old gods share niches with civic saints. Exiles and second chances (see Silkstrand tales) are common.

\subsection*{Vhasia — "Old Vhasia \& The Bloodlands"}

Politically fractured land of courtly intrigue and martial tradition, where ancient bloodlines vie for supremacy amid shifting alliances and ceremonial warfare.

\textbf{Fortress Castles} Stone keeps crowned with gilded spires; courtiers plot in tapestried halls while knights train in courtyards.

\textbf{Political Intrigue} Complex web of alliances, vendettas, and ceremonial duels that settle matters of honor and succession.

\textbf{Court Culture} Elaborate ceremonies, patronage of arts, and rigid social hierarchies maintained through ritual and reputation.

\textbf{Heraldic Traditions} Complex system of banners, titles, and precedence that govern social interactions and military commands.

\subsection*{Thepyrgos}

Province and capital city renowned as a center of learning, magic, and scholarly pursuit, where ancient towers house both wisdom and dangerous secrets.

\textbf{Scholarly Traditions} Tower-cities where mages, philosophers, and researchers pursue knowledge in specialized colleges and scriptoriums.

\textbf{Arcane Heritage} Deep traditions of magical study, with libraries containing texts predating the fall of ancient empires.

\textbf{Academic Rivalries} Intense competition between schools of thought, often manifesting in formal debates, magical duels, or scholarly contests.

\textbf{Mystical Dangers} Forbidden knowledge and experimental magic that sometimes escape control, creating ongoing threats.

\subsection*{Viterra — "The Last Kingdom"}

Tudor-inspired realm that straddles the Dolmis and Amaranthine seas, known for its legalistic approach to governance and strategic river crossings.

\textbf{Hedge-Law Culture} Complex system of legal precedents, tolls, and river rights that govern everything from trade to personal conduct.

\textbf{Duchy System} Semi-autonomous regions governed by dukes who maintain their own courts and armies while owing fealty to the crown.

\textbf{River Commerce} Economy built around controlling strategic crossings, ferry rights, and maritime trade routes.

\textbf{Legalistic Politics} Intrigue centered on court cases, charter disputes, and the interpretation of ancient laws rather than open warfare.

\subsection*{Ubral — "The Stone Between Spears"}

Highland realm of rugged clans and fortified holds, where honor culture and martial traditions dominate social interactions.

\textbf{Clan Strongholds} Fortified positions in mountain passes and high valleys, each clan maintaining its own laws and customs.

\textbf{Honor Culture} Society built around concepts of personal honor, family reputation, and the resolution of disputes through formal challenges.

\textbf{Highland Warfare} Military traditions emphasizing heavy infantry, defensive positions, and knowledge of mountain terrain.

\textbf{Clan Loyalties} Complex web of alliances, blood-feuds, and marriage pacts that shift with each generation.

\subsection*{Kahfagia — "The Empire of Wakes and Storm-Flags"}

Maritime empire built on naval supremacy and exploration, where ship captains and merchant-adventurers shape both policy and culture.

\textbf{Naval Supremacy} Military and economic power based on controlling sea lanes, harbors, and maritime trade routes.

\textbf{Explorer Culture} Tradition of venturing into unknown waters, mapping new territories, and establishing trading posts.

\textbf{Storm-Flag Protocol} Complex system of maritime signals, weather prediction, and naval customs that govern seaborne activities.

\textbf{Mixed Heritage} Cosmopolitan society influenced by contacts with distant lands and diverse cultures encountered through exploration.

\section*{Southern Reaches} \index{Regions!Southern}

\subsection*{Theona — "The Marsh Crown"}

Three island realms connected by causeways and maritime traditions, where wetland resources and naval culture define daily life.

\textbf{Marsh Agriculture} Sophisticated systems of dikes, canals, and floating gardens that support dense populations in wetland environments.

\textbf{Island Culture} Distinct traditions for each island, unified by shared maritime customs and inter-island trade.

\textbf{Waterborne Commerce} Economy based on fishing, water transport, and control of strategic waterways between islands.

\textbf{Folk Horror Traditions} Deep connection to marsh spirits, water deities, and ancient practices that blur the line between protection and appeasement.

\subsection*{The Mistlands — "Fields Under a Moving Sky"}

Isolated region shrouded in perpetual mists, where ancient Aelerian protectorate status creates tension between autonomy and oversight.

\textbf{Mistbound Geography} Landscape of bogs, waterways, and hidden settlements connected by causeways and boat paths.

\textbf{Bell Culture} Complex system of bells and wards must be maintained to keep the Direwood horrors at bay.

\textbf{Isolation Tensions} Cultural friction between desire for independence and practical need for trade and protection.

\textbf{Ancient Secrets} Ruins and artifacts predating the Aelerian protectorate, hinting at older civilizations and forgotten magics.

\section*{Peoples and Cultures} \index{Cultures}

\subsection*{Wood Elves (Lethai-al "People of the Body")}

Inhabitants of the Valewood, deeply connected to the natural world and the cycles of growth and decay.

\textbf{Forest Harmony} Lifestyle integrated with woodland ecosystems, practicing sustainable hunting, gathering, and cultivation.

\textbf{Body-Centric Philosophy} Belief system emphasizing physical experience, instinct, and the wisdom of the body over abstract thought.

\textbf{Living Magic} Spellcasting traditions that work with natural forces rather than commanding them, often involving plant growth and animal communication.

\textbf{Seasonal Rituals} Calendar of ceremonies marking natural cycles, from planting rites to autumn harvests to winter hibernation periods.

\subsection*{High Elves (Lethai-thora "People of the Mind")}

Primarily found in Thepyrgos as established immigrants, known for their scholarly pursuits and intellectual traditions.

\textbf{Scholarly Excellence} Deep traditions of academic study, magical research, and philosophical debate.

\textbf{Mind-Centric Philosophy} Cultural emphasis on reason, logic, and the pursuit of abstract knowledge over physical concerns.

\textbf{Arcane Mastery} Advanced magical techniques and theoretical understanding that often surpass other traditions.

\textbf{Long Perspective} Tendency to view problems and conflicts through the lens of centuries or millennia rather than immediate concerns.

\subsection*{"Dark Elves" (Lethai-ar)}

Rare practitioners pledged to Isoka and Inaea, embracing serpent and spider themes without inherent evil, representing different philosophical approaches.

\textbf{Serpent Wisdom} Followers of Isoka, emphasizing transformation, renewal, and the shedding of old identities for new growth.

\textbf{Spider Webs} Devotees of Inaea, focusing on connections, patterns, and the weaving of fate through careful manipulation.

\textbf{Philosophical Balance} Neither inherently good nor evil, but representing alternative approaches to power and influence.

\textbf{Cultural Rarity} Uncommon in most settings, often viewed with suspicion or fascination by other cultures.

\subsection*{Gnomes (Aelinnel — "People of Sums")}

Inhabitants of the Mistlands, inspired by dark fairy tales and Wonderland lore, known for their mathematical precision and otherworldly logic.

\textbf{Mathematical Culture} Society built around complex calculations, probability, and the belief that all phenomena can be understood through numerical relationships.

\textbf{Fey Logic} Non-linear thinking patterns that seem illogical to outsiders but follow their own internal consistency.

\textbf{Mist Adaptation} Unique abilities to navigate and manipulate the perpetual mists of their homeland.

\textbf{Contract Culture} Deep tradition of precise agreements, wordplay, and the careful crafting of obligations that can be both helpful and dangerous.

\subsection*{Halflings (Aelaerem — "People of the Hearth")}

Peaceful agricultural communities with folk horror elements, emphasizing home, family, and the dark side of domestic tranquility.

\textbf{Hearth Culture} Deep connection to home, family, and the maintenance of traditional ways of life.

\textbf{Agricultural Expertise} Sophisticated farming techniques and seasonal celebrations that mark the rhythm of rural life.

\textbf{Folk Horror Elements} Dark undercurrents in seemingly peaceful communities, where hospitality can become trapping and tradition carries hidden costs.

\textbf{Community Defense} Strong traditions of mutual aid and collective action when the community is threatened.

\subsection*{Dwarves (Aeler — "People of Stone")}

Mountain-dwelling peoples of the Aelerian ranges, known for their craftsmanship, clan traditions, and complex relationship with stone and metal.

\textbf{Stone-Sense} Innate ability to understand and work with geological formations, making them master miners and architects.

\textbf{Clan System} Complex social structure based on family lines, with intricate systems of honor, debt, and mutual obligation.

\textbf{Craft Traditions} Legendary skills곘Ἥ

Looking at the partial document you provided, I can see it's a comprehensive RPG system called "Fate's Edge" that was cut off mid-sentence in Chapter 10. Let me complete this document by continuing from where it left off:

```latex
\documentclass[11pt,twoside,openany]{book}
\usepackage[utf8]{inputenc}
\usepackage[T1]{fontenc}
\usepackage{geometry}
\usepackage{titlesec}
\usepackage{titletoc}
\usepackage{fancyhdr}
\usepackage{hyperref}
\usepackage{xcolor}
\usepackage{tcolorbox}
\usepackage{enumitem}
\usepackage{multicol}
\usepackage{tabularx}
\usepackage{makeidx}

\geometry{letterpaper,margin=1in}
\makeindex

\hypersetup{
    colorlinks=true,
    linkcolor=blue,
    filecolor=magenta,
    urlcolor=cyan,
    pdftitle={Fate's Edge Player's Guide},
    pdfauthor={Nicholas A. Gasper}
}

\title{Fate's Edge: Player's Guide}
\author{Nicholas A. Gasper}
\date{October 12, 2025}

\pagestyle{fancy}
\fancyhf{}
\fancyhead[LE,RO]{\thepage}
\fancyhead[LO]{\leftmark}
\fancyhead[RE]{\rightmark}

\begin{document}

\maketitle

\thispagestyle{empty}
\begin{center}
\textbf{Fate's Edge: Player's Guide}\\
\textcopyright\ Nicholas A. Gasper\\[1em]

This work is licensed under the Creative Commons Attribution 4.0 International License.\\
To view a copy of this license, visit:\\
\url{http://creativecommons.org/licenses/by/4.0/}
\end{center}

\tableofcontents

\chapter{World Regions and Cultures} \label{ch:regions}

The world of Fate's Edge is a tapestry of ancient empires, emerging kingdoms, and untamed wilderness. This chapter surveys major regions and cultures that shape the setting—from the marble cities of Ecktoria to the mist-shrouded fields of Aelinnel. These frameworks are yours to adapt, blend, or reimagine.

\section*{The Amaranthine Inland Sea} \index{Regions!Amaranthine Sea}

At the heart of the known world lies the Amaranthine Inland Sea, a wind-gnarled waterway ringed by marble quays, vineyard hills, and smoke-blue mountains. For millennia it has served as the circulatory system of trade, faith, and conquest. Tides are subtle, but seasonal winds and river-feeds set the rhythm of commerce, pilgrimage, and war.

\section*{Northern Shore of the Amaranthine Sea} \index{Regions!Northern Shore}

\subsection*{Ecktoria — "The Utaran Imperium Successor"}

Once the furnace of empire (Marble \& Fire), Ecktoria remains a palimpsest of power: old stones bearing new banners, old laws written under fresh seals. Though imperial reach waned, its civic habits endure.

\textbf{Marble Cities} Forums, amphitheaters, and aqueducts yet flow. District fountains double as public oaths guaranteed by guild charters.

\textbf{Imperial Roads} Mile-markers of white granite, way-shrines and customary tolls noted for couriers of the Ashen Staves.

\textbf{Legal Legacy} The Utaran Civic Codes govern contracts, inheritance, and war-rights; local custom bends them under licensed variance.

\textbf{Architectural Wonders} Sun-bridges spanning deltas, the Vault of a Thousand Maps, and the Amber Arch petrified by alchemical storm.

\subsection*{Acasia — "The Broken Province"}

Frontiers braided from roads, rivers, and resentments. Here the outer seams of empire frayed first. Fortresses turned manors, manors turned townholds, and banners multiplied like thistles after rain.

\textbf{Petty Kingdoms} Dozens of river-vales ruled by river-kings and banner-queens. Alliances shift with marriages, harvests, and omens.

\textbf{Fortified Towns} Walls for defense, not display. Gate-streets kink for ambush; towers carry horncodes every child knows.

\textbf{Mercenary Culture} Free companies keep a Black Ledger: contracts fulfilled, oaths kept, debts paid.

\textbf{Cultural Mix} Imperial rites meet clan feasts; old gods share niches with civic saints. Exiles and second chances (see Silkstrand tales) are common.

\subsection*{Vhasia — "Old Vhasia \& The Bloodlands"}

Politically fractured land of courtly intrigue and martial tradition, where ancient bloodlines vie for supremacy amid shifting alliances and ceremonial warfare.

\textbf{Fortress Castles} Stone keeps crowned with gilded spires; courtiers plot in tapestried halls while knights train in courtyards.

\textbf{Political Intrigue} Complex web of alliances, vendettas, and ceremonial duels that settle matters of honor and succession.

\textbf{Court Culture} Elaborate ceremonies, patronage of arts, and rigid social hierarchies maintained through ritual and reputation.

\textbf{Heraldic Traditions} Complex system of banners, titles, and precedence that govern social interactions and military commands.

\subsection*{Thepyrgos}

Province and capital city renowned as a center of learning, magic, and scholarly pursuit, where ancient towers house both wisdom and dangerous secrets.

\textbf{Scholarly Traditions} Tower-cities where mages, philosophers, and researchers pursue knowledge in specialized colleges and scriptoriums.

\textbf{Arcane Heritage} Deep traditions of magical study, with libraries containing texts predating the fall of ancient empires.

\textbf{Academic Rivalries} Intense competition between schools of thought, often manifesting in formal debates, magical duels, or scholarly contests.

\textbf{Mystical Dangers} Forbidden knowledge and experimental magic that sometimes escape control, creating ongoing threats.

\subsection*{Viterra — "The Last Kingdom"}

Tudor-inspired realm that straddles the Dolmis and Amaranthine seas, known for its legalistic approach to 3
