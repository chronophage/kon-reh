% !TEX root = ../fates_edge_players_guide.tex

\chapter{Core Mechanics}

In \textbf{Fate's Edge}, every action matters. The dice don't just tell you if you succeed—they shape the story by introducing tension, risk, and consequence. This chapter walks you through the core resolution system, Complication Points, and how every roll changes the narrative.

\section{Basic Dice Mechanics}

When you attempt a significant action, you roll a pool of ten-sided dice (d10s). The size of your pool is determined by:

\[
\text{Dice Pool} = \text{Attribute} + \text{Skill}
\]

\begin{description}
  \item[Attribute] (1–5) Represents broad traits like strength, wit, or charm.
  \item[Skill] (0–5) Reflects training or expertise in a specific area.
\end{description}

\subsection*{Counting Successes}

Each die that rolls \textbf{6 or higher} counts as a \textbf{Success}. Each die that rolls a \textbf{1} generates a \textbf{Complication Point (CP)}.

\begin{center}
\begin{tabular}{lc}
\toprule
\textbf{Die Result} & \textbf{Effect} \\
\midrule
6–10 & +1 Success \\
1 & +1 Complication Point (CP) \\
2–5 & No effect \\
\bottomrule
\end{tabular}
\end{center}

\section{Difficulty Value (DV)}

Before rolling, the Game Master (GM) sets a \textbf{Difficulty Value (DV)}—a target number of Successes needed to achieve the intent. DVs typically range from 1 to 4+, depending on the stakes and opposition.

\begin{center}
\begin{tabular}{cl}
\toprule
\textbf{DV} & \textbf{Situation} \\
\midrule
1 & Routine: Clear intent, modest stakes, controlled environment \\
2 & Pressured: Time pressure, mild resistance, partial info \\
3 & Hard: Hostile conditions, active opposition, precise timing \\
4+ & Extreme: Multiple constraints, high precision, dramatic failure \\
\bottomrule
\end{tabular}
\end{center}

\section{Outcome Matrix}

After rolling, compare your total Successes against the DV. The GM then resolves the outcome using the following matrix:

\begin{center}
\begin{tabular}{ll}
\toprule
\textbf{Outcome} & \textbf{Effect} \\
\midrule
\textbf{Clean Success} & Intent achieved crisply (DV met, no CP) \\
\textbf{Success \& Cost} & Intent achieved, but GM spends CP for complications \\
\textbf{Partial} & Progress with a fork (accept cost OR concede ground) \\
\textbf{Miss} & No progress; GM spends CP OR offers Devil's Bargain \\
\bottomrule
\end{tabular}
\end{center}

\section{Complication Points (CP)}

Complication Points are the engine of drama. They are not mere penalties—they are narrative tools the GM uses to introduce twists, tension, and texture into the story.

\subsection*{What CP Can Do}

The GM may spend CP to:
\begin{itemize}
  \item Escalate a threat or introduce a new one.
  \item Drain resources (time, gear, position).
  \item Reveal hidden dangers or betrayals.
  \item Cause collateral damage or unintended consequences.
\end{itemize}

\subsection*{CP Spend Menu (Examples)}

\begin{itemize}
  \item \textbf{1 CP} — Noise, trace, +1 Supply segment.
  \item \textbf{2 CP} — Alarm raised, lose position/cover, lesser foe or lock.
  \item \textbf{3 CP} — Reinforcements, gear breaks, rail tick.
  \item \textbf{4+ CP} — Major turn: trap springs, authority arrives, scene shifts.
\end{itemize}

\section{The Description Ladder}

The way you describe your action affects how the dice fall. Rich, vivid descriptions earn mechanical benefits.

\begin{description}
  \item[Basic Action] Roll as normal. All 1s remain as CP.
  \item[Detailed Action] Re-roll one die showing 1.
  \item[Intricate Action] Re-roll all dice showing 1, and add one narrative flourish on success.
\end{description}

\section{Assistance}

Characters can help each other during actions. One helper per action may provide assistance dice, up to a maximum of +3 dice total from all sources. Exception: The "Exceptional Coordination" Talent allows one follower to provide +4 assist dice.

Assist dice come from the helper, not the leader. When applicable, the follower adds help dice equal to min(C, the helper's relevant Skill), capped at +3 dice.

\section{Boons}

Boons are narrative tokens representing luck, resilience, or growth. You may carry up to \textbf{5 Boons} at once.

\subsection*{What You Can Do With Boons}

\begin{itemize}
  \item Re-roll one die after seeing the pool.
  \item Once per session, in downtime, you may convert 2 Boons → 1 XP (max 2 XP via conversion per session).
\end{itemize}

\section{Deck of Consequences}

Fate's Edge uses two distinct card tools:

\paragraph{Travel Decks (regional, 52-card).}
\emph{Spade}=Place, \emph{Heart}=Actor, \emph{Club}=Pressure, \emph{Diamond}=Leverage. These power journeys and gates.

\paragraph{Deck of Consequences (scene drama).}
\emph{Hearts}=social fallout, \emph{Swords}=harm/escalation, \emph{Pentacles}=material cost, \emph{Wands}=magical/spiritual disturbance.

\textit{Guidance:} Never mix suit meanings across decks. When a rule references ``Spade/Club/Diamond,'' it means \emph{Travel}. When it says ``Hearts/Swords/Pentacles/Wands,'' it means \emph{Consequences}.

\subsection*{Using the Deck}

After a roll that generates CP, the GM chooses one method for that roll:
\begin{enumerate}
  \item \textbf{Direct Spend}: Translate CP into consequences/rail ticks immediately.
  \item \textbf{Deck Draw}: Draw up to \textbf{min(CP, 3)} cards and synthesize a single twist guided by suit and highest rank.
\end{enumerate}

\section{Summary}

The core mechanic of Fate's Edge is simple, but rich in narrative possibility:

\begin{enumerate}
  \item Describe your intent and method.
  \item Build your dice pool: Attribute + Skill.
  \item Roll d10s, count Successes and Complication Points.
  \item Compare Successes to DV, apply Outcome Matrix.
  \item GM spends CP to add twists or tension, or draws from the Deck of Consequences.
  \item You earn Boons for engaging the fiction meaningfully.
\end{enumerate}

Every roll is a story beat. Embrace the risk—and let the consequences unfold.
