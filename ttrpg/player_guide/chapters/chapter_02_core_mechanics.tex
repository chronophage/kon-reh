% !TEX root = ../fates_edge_players_guide.tex

\chapter{Core Mechanics}

In \textbf{Fate’s Edge}, every action matters. The dice don’t just tell you if you succeed—they shape the story by introducing tension, risk, and consequence. This chapter walks you through the core resolution system, Complication Points, and how every roll changes the narrative.

\section{Basic Dice Mechanics}

When you attempt a significant action, you roll a pool of ten-sided dice (d10s). The size of your pool is determined by:

\[
\text{Dice Pool} = \text{Attribute} + \text{Skill}
\]

\begin{description}
  \item[Attribute] (1–5) Represents broad traits like strength, wit, or charm.
  \item[Skill] (0–5) Reflects training or expertise in a specific area.
\end{description}

\subsection*{Counting Successes}

Each die that rolls \textbf{6 or higher} counts as a \textbf{Success}. Each die that rolls a \textbf{1} generates a \textbf{Complication Point (CP)}.

\begin{center}
\begin{tabular}{lc}
\toprule
\textbf{Die Result} & \textbf{Effect} \\
\midrule
6–10 & +1 Success \\
1 & +1 Complication Point (CP) \\
2–5 & No effect \\
\bottomrule
\end{tabular}
\end{center}

\section{Difficulty Value (DV)}

Before rolling, the Game Master (GM) sets a \textbf{Difficulty Value (DV)}—a target number of Successes needed to achieve the intent. DVs typically range from 1 to 4+, depending on the stakes and opposition.

\begin{center}
\begin{tabular}{cl}
\toprule
\textbf{DV} & \textbf{Situation} \\
\midrule
1 & Routine: Clear intent, modest stakes, controlled environment \\
2 & Pressured: Time pressure, mild resistance, partial info \\
3 & Hard: Hostile conditions, active opposition, precise timing \\
4+ & Extreme: Multiple constraints, high precision, dramatic failure \\
\bottomrule
\end{tabular}
\end{center}

\section{Outcome Matrix}

After rolling, compare your total Successes against the DV. The GM then resolves the outcome using the following matrix:

\begin{center}
\begin{tabular}{ll}
\toprule
\textbf{Outcome} & \textbf{Effect} \\
\midrule
\textbf{Clean Success} & Intent achieved crisply (DV met, no CP) \\
\textbf{Success \& Cost} & Intent achieved, but GM spends CP for complications \\
\textbf{Partial} & Progress with a fork (accept cost OR concede ground) \\
\textbf{Miss} & No progress; GM spends CP OR offers Devil’s Bargain \\
\bottomrule
\end{tabular}
\end{center}

\section{Complication Points (CP)}

Complication Points are the engine of drama. They are not mere penalties—they are narrative tools the GM uses to introduce twists, tension, and texture into the story.

\subsection*{What CP Can Do}

The GM may spend CP to:
\begin{itemize}
  \item Escalate a threat or introduce a new one.
  \item Drain resources (time, gear, position).
  \item Reveal hidden dangers or betrayals.
  \item Cause collateral damage or unintended consequences.
\end{itemize}

\subsection*{CP Spend Menu (Examples)}

\begin{itemize}
  \item \textbf{1 CP} — Noise, trace, +1 Supply segment; tool becomes Compromised.
  \item \textbf{2 CP} — Alarm raised, lose position, lesser foe appears.
  \item \textbf{3 CP} — Reinforcements, gear breaks, Fatigue 1.
  \item \textbf{4+ CP} — Major turn: trap springs, rival claims prize, authority arrives.
\end{itemize}

\section{The Description Ladder}

The way you describe your action affects how the dice fall. Rich, vivid descriptions earn mechanical benefits.

\begin{description}
  \item[Basic Action] Roll as normal. All 1s remain as CP.
  \item[Detailed Action] Re-roll one die showing 1.
  \item[Intricate Action] Re-roll all dice showing 1, and add one narrative flourish on success.
\end{description}

\section{Position and Effect}

Actions are resolved with both a \textbf{Position} and an \textbf{Effect}:

\begin{description}
  \item[Position] Describes how risky or safe the action is:
    \begin{itemize}
      \item \textbf{Controlled} — Low risk, high chance of success.
      \item \textbf{Risky} — Moderate risk, balanced stakes.
      \item \textbf{Desperate} — High risk, likely consequences.
    \end{itemize}
  \item[Effect] Describes the narrative impact of a successful action:
    \begin{itemize}
      \item \textbf{Limited} — Minor impact.
      \item \textbf{Standard} — Expected result.
      \item \textbf{Great} — Significant or lasting impact.
    \end{itemize}
\end{description}

\section{Boons and Failure’s Gift}

Even failure in Fate’s Edge moves the story forward. When you fail meaningfully, you earn a \textbf{Boon}—a small narrative token representing resilience, luck, or growth.

\subsection*{What You Can Do With Boons}

\begin{itemize}
  \item Re-roll one die after seeing the result.
  \item Activate an Off-Screen Asset.
  \item Convert 2 Boons into 1 XP (once per session).
\end{itemize}

You may carry up to \textbf{5 Boons} at once. Overflow is converted to XP.

\section{Summary}

The core mechanic of Fate’s Edge is simple, but rich in narrative possibility:

\begin{enumerate}
  \item Describe your intent and method.
  \item Build your dice pool: Attribute + Skill.
  \item Roll d10s, count Successes and Complication Points.
  \item Compare Successes to DV, apply Outcome Matrix.
  \item GM spends CP to add twists or tension.
  \item You earn Boons for engaging the fiction meaningfully.
\end{enumerate}

Every roll is a story beat. Embrace the risk—and let the consequences unfold.


