% !TEX root = ../fates_edge_players_guide.tex

\chapter{Core Mechanics}
\label{ch:core-mechanics}

In \textbf{Fate's Edge}, every action matters. The dice don't just tell you if you succeed—they shape the story by introducing tension, risk, and consequence. This chapter walks you through the core resolution system, Complication Points, and how every roll changes the narrative.

\section{Basic Dice Mechanics}
\index{Dice mechanics}

When you attempt a significant action, you roll a pool of ten-sided dice (d10s). The size of your pool is determined by:

\[
\text{Dice Pool} = \text{Attribute} + \text{Skill}
\]

\begin{description}
  \item[Attribute] \index{Attribute} (1–5) Represents broad traits like strength, wit, or charm.
  \item[Skill] \index{Skill} (0–5) Reflects training or expertise in a specific area.
\end{description}

\subsection*{Counting Successes}

Each die that rolls \textbf{6 or higher} counts as a \textbf{Success}. Each die that rolls a \textbf{1} generates a \textbf{Complication Point (CP)} \index{Complication Point (CP)}.

\begin{center}
\begin{tabular}{lc}
\toprule
\textbf{Die Result} & \textbf{Effect} \\
\midrule
6–10 & +1 Success \\
1 & +1 Complication Point (CP) \\
2–5 & No effect \\
\bottomrule
\end{tabular}
\end{center}

\section{Difficulty Value (DV)}
\index{Difficulty Value (DV)}

Before rolling, the Game Master (GM) sets a \textbf{Difficulty Value (DV)} \index{Difficulty Value (DV)}—a target number of Successes needed to achieve the intent. DVs typically range from 1 to 4+, depending on the stakes and opposition.

\begin{center}
\begin{tabular}{cl}
\toprule
\textbf{DV} & \textbf{Situation} \\
\midrule
1 & Routine: Clear intent, modest stakes, controlled environment \\
2 & Pressured: Time pressure, mild resistance, partial info \\
3 & Hard: Hostile conditions, active opposition, precise timing \\
4+ & Extreme: Multiple constraints, high precision, dramatic failure \\
\bottomrule
\end{tabular}
\end{center}

\section{Outcome Matrix}
\index{Outcome Matrix}

After rolling, compare your total Successes against the DV. The GM then resolves the outcome using the following matrix:

\begin{center}
\begin{tabular}{ll}
\toprule
\textbf{Outcome} & \textbf{Effect} \\
\midrule
\textbf{Clean Success} & Intent achieved crisply (DV met, no CP) \\
\textbf{Success \& Cost} & Intent achieved, but GM spends CP for complications \\
\textbf{Partial} & Progress with a fork (accept cost OR concede ground) \\
\textbf{Miss} & No progress; GM spends CP \\
\bottomrule
\end{tabular}
\end{center}

\section{Fail Forward: Every Roll Matters}\index{Fail Forward}\index{Boons}

When you \textbf{MISS} on a \emph{significant action}, you gain \textbf{1 Boon}. Boons can be spent immediately for re-rolls, Asset activations, Rites, and other abilities. You can hold up to \textbf{5} Boons.

\subsection*{Significant Action (Meaningful Failure)}
A miss only awards a Boon if \textbf{all three} are true:
\begin{enumerate}
  \item \textbf{Procedure followed:} intent and approach declared; DV set; roll resolved.
  \item \textbf{Stakes stated:} what changes on success; what bites on failure.
  \item \textbf{Consequence lands now:} the GM spends or banks CP, applies a condition, or advances a thread.
\end{enumerate}
\noindent Rolling a \textbf{1} always creates CP for the GM. Re-rolling \textbf{1}s never removes CP already generated.

\subsection*{No Boon For}
Rehearsal or null-risk probes, and repeated identical attempts in the same scene without a new approach, position, or stakes.

\subsection*{Other Ways to Gain Boons}
Strong bond-driven play and scene prompts can also award Boons at the GM's discretion. Boons remain capped by the limits below.

\subsection*{Boon Conversion}
Once per session, during a downtime beat, you may convert \textbf{2 Boons} into \textbf{1 Experience Point (XP)}. You may gain a maximum of \textbf{2 XP} via this conversion per session.

\section{Boon Carryover (Scene-Based)}\index{Boon Carryover}
\textbf{Hold Cap:} You can hold up to \textbf{5} Boons.

\textbf{Carryover Limit:} At the \emph{end of each scene}, reduce your held Boons to a maximum of \textbf{2}. Excess Boons are lost.

\textbf{Spend As You Earn:} You may spend Boons at any time during the scene (re-rolls, Asset activations, Rites, abilities, etc.).

\textbf{Multi-Phase Set Pieces:} If the GM declares a multi-phase scene (e.g., chase $\rightarrow$ duel), trim to 2 only when the entire set piece ends.

\subsection*{Rite \& Asset Notes}
High-Power Rites that require \textbf{2 Boons} remain viable—you can start a scene at 2 and must earn more in-scene to chain further Invokes. On-screen Asset activations still cost 1 Boon as normal.

\subsection*{Anti-Fishing Dials}\index{Anti-Fishing}
These optional limits help keep flow healthy:
\begin{itemize}
  \item \textbf{Once/Scene (Failures):} At most \textbf{2 Boons from failures per character per scene}. Further misses still generate CP but no Boon.
  \item \textbf{Repetition Rule:} Same approach \emph{and} same stakes in the same scene cannot award another Boon.
  \item \textbf{Position Gate:} Tests taken primarily for information-gathering or minor positioning, where failure has negligible narrative impact, typically do not award Boons.
\end{itemize}

\subsection*{Optional: Partial-with-Cost Safety Valve}
By default, \textbf{Partial (Success \& Cost)} does \emph{not} grant a Boon. If your table wants more flow, you may award \textbf{1 Boon} on a Partial when the GM spends \textbf{3+ CP} on that outcome (\emph{max once per scene per character}). \emph{Tip:} Use this only when the cost meaningfully changes the situation.

\subsection*{Examples}
\begin{itemize}
  \item \textbf{Boon awarded:} Picking a lock under watch (\emph{Risky}, DV 3). Stakes set: success opens; miss triggers the alarm. The roll \textbf{MISS}es; the GM spends 2 CP to start ``Guards Incoming.'' The player gains \textbf{1 Boon}.
  \item \textbf{No Boon:} Tapping flagstones ``just in case'' (Controlled, no stated stakes). Info only; no CP spent/banked. No Boon.
  \item \textbf{Carryover:} End of scene, a character holds 4 Boons. They trim to \textbf{2} for the next scene. During the next scene, they earn and spend Boons freely, never exceeding the \textbf{5} hold cap in-scene; trim back to 2 when that scene ends.
\end{itemize}

\section{Complication Points (CP)}
\index{Complication Point (CP)}

Complication Points are the engine of drama. They are not mere penalties—they are narrative tools the GM uses to introduce twists, tension, and texture into the story.

\subsection*{What CP Can Do}

The GM may spend CP to:
\begin{itemize}
  \item Escalate a threat or introduce a new one.
  \item Drain resources (time, gear, position).
  \item Reveal hidden dangers or betrayals.
  \item Cause collateral damage or unintended consequences.
\end{itemize}

\subsection*{CP Spend Menu (Examples)}

\begin{itemize}
  \item \textbf{1 CP} — Noise, trace, +1 Supply segment.
  \item \textbf{2 CP} — Alarm raised, lose position/cover, lesser foe or lock.
  \item \textbf{3 CP} — Reinforcements, gear breaks, rail tick.
  \item \textbf{4+ CP} — Major turn: trap springs, authority arrives, scene shifts.
\end{itemize}

\section{The Description Ladder}
\index{Description Ladder}

The way you describe your action affects how the dice fall. Rich, vivid descriptions earn mechanical benefits.

\begin{description}
  \item[Basic Action] \index{Basic Action} Roll as normal. All 1s remain as CP.
  \item[Detailed Action] \index{Detailed Action} Re-roll one die showing 1.
  \item[Intricate Action] \index{Intricate Action} Re-roll all dice showing 1, and add one narrative flourish on success.
\end{description}

\noindent\textbf{CP Note.} Re-rolling 1s does \emph{not} erase their CP; any new 1s on the re-roll add more CP.

\section{Assistance}
\index{Assistance}

Characters can help each other during actions. One helper per action may provide assistance dice, up to a maximum of +3 dice total from all sources. Exception: The "Exceptional Coordination" Talent \index{Exceptional Coordination} allows one follower to provide +4 assist dice.

Assist dice come from the helper, not the leader. When applicable, the follower adds help dice equal to min(C, the helper's relevant Skill), capped at +3 dice.

\section{Bond-Driven Resource Generation}
\index{Bond-Driven Resource Generation}

When you take a significant action to aid an ally with whom you share a bond, and you explicitly reference that bond in an intricate description, you may mark that bond to gain 1 boon after the action resolves.

\textbf{Requirements:}
\begin{itemize}
    \item \textbf{Mutual Bond:} You share a bond with the ally you're aiding
    \item \textbf{Intricate Description:} Describe how your bond motivates your action
    \item \textbf{Significant Aid:} Provide meaningful assistance (not just +1 die)
    \item \textbf{Fiction First:} The bond must genuinely drive your choice to help
\end{itemize}

\textbf{Examples:}
\begin{itemize}
    \item "Remembering how they saved me before, I throw myself in front of the attack meant for them!"
    \item "Thinking of our shared past, I use my last resource to shield them from harm."
    \item "Drawing on our mutual experiences, I rally others to keep fighting alongside them."
\end{itemize}

\textbf{Limitations:}
\begin{itemize}
    \item Once per bond per session
    \item Must be a meaningful sacrifice or risk
    \item GM approval for "significant action"
    \item Cannot be used for basic assistance rolls
\end{itemize}

This mechanic creates a virtuous cycle of cooperation: helping allies strengthens your bond network, which provides resources to help more allies, reinforcing collaborative play.

\section{Asset Activations}
\index{Asset Activations}

Players can activate their Assets through multiple methods:

\begin{itemize}
    \item \textbf{Free Off-Screen Effects}: Use each Asset's listed Off-Screen effect once per session for free.
    \item \textbf{XP Activation}: Spend 2 XP to activate an off-screen Asset effect outside your normal session allowance.
    \item \textbf{Boon Activation}: Spend 1 Boon to dramatically reshape the current scene through Asset intervention.
    \item \textbf{Plausibility Test}: The Asset must have scope and reach for the intended effect.
\end{itemize}

\section{Deck of Consequences}
\index{Deck of Consequences}

Fate's Edge uses two distinct card tools:

\paragraph{Travel Decks (regional, 52-card).}
\emph{Spade}=Place, \emph{Heart}=Actor, \emph{Club}=Pressure, \emph{Diamond}=Leverage. These power journeys and gates.

\paragraph{Deck of Consequences (scene drama).}
\emph{Hearts}=social fallout, \emph{Spades}=harm/escalation, \emph{Clubs}=material cost, \emph{Diamonds}=magical/spiritual disturbance.

\textbf{Important Guidance:} Never mix suit meanings across decks. When a rule references ``Spade/Club/Diamond,'' it means \emph{Travel}. When it says ``Hearts/Spades/Clubs/Diamonds,'' it means \emph{Consequences}.

\subsection*{Using the Deck}

After a roll that generates CP, the GM chooses one method for that roll:
\begin{enumerate}
  \item \textbf{Direct Spend}: Translate CP into consequences/rail ticks immediately.
  \item \textbf{Deck Draw}: Draw up to \textbf{min(CP, 3)} cards and synthesize a single twist guided by suit and highest rank.
\end{enumerate}

\section{Narrative Time}
\index{Narrative Time}

Time in Fate's Edge is measured by story weight, not by clocks:
\begin{description}
  \item[Moment] \index{Moment} A heartbeat, a glance, a single strike or word.
  \item[Some Time] \index{Some Time} A few minutes, enough for a skirmish, a careful lockpick, or a short negotiation.
  \item[Significant Time] \index{Significant Time} Hours, long enough to travel between locations, work a ritual, or endure a siege.
  \item[Days] \index{Days} Large-scale endeavors: marches across a countryside, training a cadre, or recovering from wounds.
\end{description}

\subsection*{Game Structure Definitions}
\begin{description}
  \item[Scene] \index{Scene} The basic unit of narrative gameplay, typically covering ``Some Time'' to ``Significant Time''. Contains multiple player turns, resolves a specific narrative question or conflict.
  \item[Player Turn (Beat)] \index{Player Turn} Individual player agency within the scene flow. Player declares action $\rightarrow$ GM sets position $\rightarrow$ Player rolls $\rightarrow$ Resolve outcome $\rightarrow$ Manage consequences.
  \item[Round] \index{Round} Simultaneous or near-simultaneous actions within a scene (primarily for combat), representing a few seconds of real time.
  \item[Session] \index{Session} One complete game session (typically 3-6 hours of real time), containing 2-4 major scenes and resolving significant narrative progress.
  \item[Campaign] \index{Campaign} Entire story arc (multiple sessions, typically 6-20+ sessions) with major character development and lasting consequences.
\end{description}

\section{Summary}

The core mechanic of Fate's Edge is simple, but rich in narrative possibility:

\begin{enumerate}
  \item Describe your intent and method.
  \item Build your dice pool: Attribute + Skill.
  \item Roll d10s, count Successes and Complication Points.
  \item Compare Successes to DV, apply Outcome Matrix.
  \item GM spends CP to add twists or tension, or draws from the Deck of Consequences.
  \item You earn Boons for engaging the fiction meaningfully.
\end{enumerate}

Every roll is a story beat. Embrace the risk—and let the consequences unfold.

\end{chapter}
