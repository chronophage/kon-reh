% !TEX root = ../fates_edge_players_guide.tex

\chapter{NPC Companions}

In \textbf{Fate’s Edge}, you don’t have to face the world alone. \textbf{NPC Companions}—from faithful familiars to hired muscle—can aid you in scenes, carry story weight, and create meaningful connections. This chapter explores the types of companions available and how they function in play.

\section{Types of Companions}

Companions in Fate’s Edge fall into two broad categories:

\begin{description}
  \item[Familiars and Pets] — Spirit-bound or trained creatures with specific roles.
  \item[Hirelings and Followers] — Professional allies who lend their skills to your cause.
\end{description}

All companions are purchased with XP and tracked through \textbf{Condition Tracks}.

\section{Familiars and Pets}

Familiars are spirits wearing physical forms—birds, beasts, or constructs—bound by vows, names, and roles. They require the \textbf{Familiar Bond} Talent to acquire.

\subsection*{Familiar Rules}

\begin{itemize}
  \item \textbf{Cap:} 1–2 (2 with Spirit Keeper prestige Talent)
  \item \textbf[Upkeep:] None
  \item \textbf[Stat Line:] Role • Cap 1–2 • Bond 1/2 • Exposure 0/2
  \item \textbf[Restrictions:] Cannot hold assets, trigger Diamonds, or act as Stewards
\end{itemize}

\subsection*{Familiar Roles}

\begin{description}
  \item[Scout] — Routes, chases, timing the surge
  \item[Mimic] — Voices, calls, baiting impostors
  \item[Distract] — Crowd manipulation, misdirection
  \item[Fetch] — Lines, keys, notes ferried under pressure
  \item[Sentry] — Watch, warn, early hazard callouts
\end{description}

\subsection*{Familiar Initiative Actions}

Once per scene (total across your Familiars), one may take a small independent action:

\begin{itemize}
  \item \textbf{Scout & Signal} — Change ally's next action position to Controlled or grant +1 effect.
  \item \textbf{Distract & Draw} — Reduce kinetic rail (Hunt/Escape/Hazard) by –1 tick.
  \item \textbf{Fetch & Carry} — Move small object through danger; recipient's next success advances target clock +1.
\end{itemize}

\textbf{Cost:} Mark Exposure +1 or Harm 1 on the Familiar.

\subsection*{Example Familiars}

\begin{description}
  \item[Shadow-Cat (Cap 2, Stealth)] — Can climb, slip through cracks, carry small items. Backlash: draws bad omens.
  \item[Crow Messenger (Cap 2, Perception)] — Delivers notes or warns of danger. Backlash: nosy, sometimes lies.
  \item[Hound of the Fens (Cap 3, Tracking)] — Keen nose, loyal defender. Backlash: loud bark alerts enemies.
\end{description}

\section{Hirelings and Followers}

Professional allies who act in scenes and lend their specialty. Purchased with XP based on Cap (Cap²).

\subsection*{Follower Rules}

\begin{itemize}
  \item \textbf{Cost:} Cap² XP
  \item \textbf[Assist Bonus:] Up to min(Cap, your relevant Skill), max +3 total
  \item \textbf[Slot Limit:] Only one follower may assist a given action
  \item \textbf[Upkeep:] XP equal to Cap or Significant Time scene
\end{itemize}

\subsection*{Follower Assistance Tiers}

\begin{description}
  \item[In-Role Assist] — +Cap bonus when specialty applies (plausible fiction enough).
  \item[Off-Role Assist] — +1 bonus with intricate description (includes Sense, Method, Risk).
  \item[Initiative Action] — Small independent action (see below).
\end{description}

\subsection*{Follower Initiative Actions}

Once per scene (across crew), one on-screen follower may take a small independent action:

\begin{itemize}
  \item \textbf{Scout & Signal} — Change ally's next action position to Controlled or grant +1 effect.
  \item \textbf{Distract & Draw} — Reduce kinetic rail (Hunt/Escape/Hazard) by –1 tick.
  \item \textbf{Fetch & Carry} — Move small object through danger; recipient's next success advances target clock +1.
\end{itemize}

\textbf{Cost:} Mark Exposure +1 or Harm 1 on that follower.

\subsection*{Example Followers}

\begin{description}
  \item[Bodyguard (Cap 4, Melee)] — Grants bonus dice in combat, but draws fire.
  \item[Scribe (Cap 3, Lore)] — Keeps records, interprets contracts, whispers legal loopholes.
  \item[Scout (Cap 3, Survival)] — Knows paths, extends party range, prone to wanderlust.
  \item[Quartermaster (Cap 3, Logistics)] — Manages supplies, repairs gear, keeps trains running.
\end{description}

\section{Condition Tracks for Companions}

All on-screen companions track Bond, Exposure, and Harm:

\begin{description}
  \item[Bond (2–3)] — Narrative connection and loyalty level.
  \item[Exposure (2–4)] — Heat, attention, or narrative stress.
  \item[Harm (1–2)] — Injury or trauma.
\end{description}

\textbf{States:}
\begin{itemize}
  \item \textbf{Maintained} — Engaged and reliable.
  \item \textbf[Neglected] — Strain shows; needs time or gesture.
  \item \textbf[Compromised] — Major story turn—capture, betrayal, departure.
\end{itemize}

\section{Loyalty and Bonds (Optional)}

Track a simple Loyalty tag per companion:
\begin{itemize}
  \item \textbf{Wary} — Distrustful, requires careful handling.
  \item \textbf{Steady} — Reliable, responds to fair treatment.
  \item \textbf{Devoted} — Unwavering loyalty; may convert one CP targeting them into lesser setback.
\end{itemize}

\section{Promotion and Replacement}

\begin{itemize}
  \item \textbf{Promote:} Pay difference in XP to raise Cap; requires brief training or milestone scene.
  \item \textbf{Replace:} Buying similar new follower costs full XP; loyalty starts Wary.
\end{itemize}

\section{Summary}

Companions in Fate’s Edge are \textbf{story agents}, not stat blocks:

\begin{itemize}
  \item Familiars offer specialized aid with no upkeep.
  \item Followers provide flexible assistance but require attention.
  \item Both create narrative opportunities and vulnerabilities.
\end{itemize}

Choose your allies wisely—and remember: they have stories too.

