% !TEX root = ../fates_edge_players_guide.tex

\chapter{Talents}
\label{ch:talents}

In \textbf{Fate's Edge}, \textbf{Talents} \index{Talents} are unique abilities that expand what your character can do. They are purchased with XP and often serve as stepping stones toward Prestige Abilities. Talents can be general, cultural, or narrative capstones that reshape how you engage with the world. Some Talents incorporate \textbf{Tags} (see Chapter \ref{ch:world-interaction}), which are standardized mechanical shorthand for common effects.

\section{Types of Talents}
\index{Talents!types}

Talents are organized into three categories based on cost and impact:

\subsection*{General Talents}
\index{Talents!general}

\begin{itemize}
    \item \textbf{Battle Instincts} (Cost: 6 XP): Once per scene, re-roll a failed defense roll.
    \item \textbf{Silver Tongue} (Cost: 4 XP): Gain +1 die when persuading or deceiving through speech.
    \item \textbf{Iron Stomach} (Cost: 3 XP): Immune to mundane poisons and spoiled food; halve Complications from toxic sources.
    \item \textbf{Exceptional Coordination} (Cost: 8 XP): One follower can provide +4 assist dice.
    \item \textbf{Danger Sense} (Cost: 5 XP): Once per scene, treat the first Complication Point generated against you as if it were a 1 lower (minimum 1). This can turn a 4 CP spend into a 3 CP spend, for example.
    \item \textbf{Steady Hands} (Cost: 4 XP): Ignore the first penalty die from Harm or Fatigue on any action involving fine motor control (e.g., picking locks, crafting, using ranged weapons).
\end{itemize}

\subsection*{Racial or Cultural Talents}
\index{Talents!racial/cultural}

\begin{itemize}
    \item \textbf{Stone-Sense} (Dwarves, Cost: 5 XP): Detect flaws in stone or earth; gain +1 die on Engineering or Craft rolls underground.
    \item \textbf{Backlash Soothing} (Wood Elves, Cost: 6 XP): Once per session, reduce a magical Backlash Complication by 2 points when in natural terrain.
    \item \textbf{Blood Memory} (Ykrul, Cost: 5 XP): After a battle, meditate to gain one temporary Skill die reflecting a foe's tactics for the next scene.
    \item \textbf{Fae-Touched} (Fae Courts, Cost: 7 XP): Gain +1 die on rolls to resist or interact with Fae magic. Once per session, you may treat a roll of 1-2 on a d10 as a 6 for the purposes of generating a Boon from a failure, but you must accept a minor, immediate, Fae-themed complication (GM's discretion).
    \item \textbf{Deep Law Keeper} (Aeler, Cost: 6 XP): Gain +1 die on rolls related to dwarven law, tradition, or resisting enchantments while underground.
\end{itemize}

\subsection*{Prestige Abilities}
\index{Prestige Abilities}

Prestige Abilities are high-impact Talents, often with significant narrative weight and sometimes incorporating Tags. They frequently have multiple prerequisites.

\begin{itemize}
    \item \textbf{Echo-Walker's Step} (High Elf, Cost: 20 XP; Req: Wits 5, Arcana 4): 
1/arc, \emph{observe} a perfect echo of a past event at your location (no retconning). 
GM immediately banks +2 CP; scenes touching that memory carry an omen. Grants DV −1 on one action that uses the revealed truth.
    \item \textbf{Warglord} (Ykrul, Cost: 18 XP; Req: Body 5, Command 3): 
Once per campaign, unify scattered warbands into a single host for a season. Start a \emph{Logistics} clock and a \emph{Grudge} clock; either one filling fractures the host.
    \item \textbf{Spirit-Shield} (Aeler, Cost: 15 XP; Req: Spirit 4, Insight 3): 
1/session, erase up to 3 CP from an ally's \emph{current} roll; you immediately mark Fatigue +1 and the GM banks +1 CP as backlash.
    \item \textbf{Warder's Seal} (Cost: 15 XP; Req: Wits 4, Arcana 3): You may create a permanent magical \TagName{WARD} (see Chapter \ref{ch:world-interaction}) as a Downtime activity. The DV to cross is equal to your Arcana rating. Crafting the \TagName{WARD} costs 2 XP and requires specific materials. Maintaining multiple \TagName{WARD}s requires XP upkeep (1 XP per \TagName{WARD} per month, paid during Downtime).
    \item \textbf{Theurgy} (Cost: 25 XP; Req: Wits 5, Faith 4, relevant Cleric Talent): You may perform minor miracles aligned with your faith. Once per session, you can invoke an effect equivalent to a \TagName{BARRIER}, \TagName{CLEANSE}, or \TagName{FORTIFY} Tag (see Chapter \ref{ch:world-interaction}) without needing to cast a spell or spend Boons. Define the effect and cost with the GM (typically DV 2-3, scene duration). Performing a miracle generates 1 CP for the GM.
    \item \textbf{Runekeeper's Bond} (Cost: 12 XP; Req: Wits 4, Arcana 3): You may bind a Thiasos (a supernatural familiar) to a specific Patron (see Chapter \ref{ch:gods-powers-patrons}). This is a prerequisite for learning and using Rites associated with that Patron. The Thiasos follows standard Familiar rules for Exposure and Harm.
    \item \textbf{Inspire} (Cost: 3 XP; Req: Presence 3, a declared Bonded PC ally): 
1 action, audible/visible cue, Near range. Use: 2/3/4 times between downtime (by Tier I/II/III). Effect: Apply all:
\begin{itemize}
    \item Bonded ally: +1 Boon (PC only) and +1 die on their next roll this scene.
    \item You: +1 die on your next roll this scene.
    \item Each other PC in Near: +1 die on their next roll this scene.
\end{itemize}
\end{itemize}

\section{Tags in Talents}
\index{Tags!in Talents}

Tags are standardized mechanical shorthand that appear on some Talents, Abilities, and Spell Results. When a Talent incorporates a Tag, it means you can use that Tag's effect as described in Chapter \ref{ch:world-interaction}.

\subsection*{How Tags Work with Talents}
\begin{itemize}
    \item \textbf{Activation:} When you use a Talent with a Tag, you activate that Tag's effect according to its rules.
    \item \textbf{Scope:} The Talent defines the Tag's scope, duration, and any limitations.
    \item \textbf{Cost:} Using a Tag through a Talent may have specific costs (like Boons, CP, or Fatigue) defined by the Talent.
    \item \textbf{Example - Warder's Seal:} This Talent allows you to create a \TagName{WARD} Tag effect. The Talent specifies the DV (your Arcana), the duration (permanent with upkeep), and the cost (2 XP to create, 1 XP/month to maintain).
    \item \textbf{Example - Theurgy:} This Talent allows you to invoke \TagName{BARRIER}, \TagName{CLEANSE}, or \TagName{FORTIFY} effects. The Talent specifies this costs 1 CP to the GM and can be done once per session.
\end{itemize}

\subsection*{Common Tag-Using Talents}
\begin{itemize}
    \item Talents that \textbf{Create Tags}: Like Warder's Seal (\TagName{WARD}) or Shadow Broker (\TagName{MARK}).
    \item Talents that \textbf{Invoke Tags}: Like Theurgy (\TagName{BARRIER}/\TagName{CLEANSE}/\TagName{FORTIFY}) or Banewreaker (\TagName{BANISH}).
    \item Talents that \textbf{Modify Tags}: That change how existing Tags affect you or your allies.
\end{itemize}

\section{Purchasing Talents}
\index{Talents!purchasing}

Talents are purchased with XP and may have prerequisites:

\begin{itemize}
  \item \textbf{Attribute/Skill Requirements} — Must meet minimum ratings (permanent, not temporary).
  \item \textbf{Asset/Tag/Talent Requirements} — Some Talents require ownership of specific assets, the ability to use certain Tags, or possession of other Talents.
  \item \textbf{Cultural Requirements} — Certain Talents are restricted to specific cultures or Affinities.
  \item \textbf{Downtime} — Spending XP on a Talent usually requires an equivalent amount of Downtime to represent training or attunement.
\end{itemize}

\textbf{Example:} \textit{Warder's Seal} requires Wits 4, Arcana 3, and represents a significant investment of time and resources.

\section{Prestige Prerequisites}
\index{Prestige Abilities!prerequisites}

\begin{itemize}
    \item Qualifying: Attribute/Skill/Talent prerequisites must be met with permanent ratings.
    \item After purchase: If you later lose a prerequisite (e.g., an Attribute is reduced), you \textbf{keep the Talent} but may not be able to use features that require that prerequisite until it is restored.
    \item Scaling: Many Prestige Abilities can be enhanced with additional XP investment, often to increase uses or potency.
\end{itemize}

\section{Using Talents}
\index{Talents!usage}

Talents are typically used once per scene, session, or arc, as noted in their description. Overuse is restricted to prevent imbalance.

\begin{itemize}
  \item \textbf{Once per Scene} — Can be used multiple times per session, but not on the same action.
  \item \textbf{Once per Session} — Limited to one use between downtimes.
  \item \textbf{Once per Arc/Campaign} — Reserved for narrative milestones.
  \item \textbf{Passive/Constant} — Some Talents provide a continuous benefit (e.g., Iron Stomach).
\end{itemize}

\section{Losing Talent Access}
\index{Talents!losing access}

If you later lose a prerequisite (e.g., an Attribute is reduced or an Asset is lost), you \textbf{keep the Talent} but may not be able to use features that require that prerequisite until it is restored.

\section{Summary}

Talents are the unique expressions of your character's growth:

\begin{itemize}
  \item General Talents offer reliable mechanical benefits for common situations.
  \item Cultural Talents reflect your heritage, specialized training, or unique background.
  \item Prestige Abilities reshape the story at key moments, often incorporating powerful effects or Tags.
  \item Tag-using Talents provide standardized mechanical effects that interact with the broader game systems.
\end{itemize}

Choose Talents that reflect your character's journey—and let them echo through the world.

\section{Skills and Skill Examples}
\index{Skills}

Skills in Fate's Edge represent your character's training, expertise, and specialized knowledge. They are rated from 0 to 5 and are added to an Attribute to determine your dice pool for relevant actions. Unlike Talents, Skills are broad categories of competency rather than specific abilities.

\subsection*{How Skills Work}
\index{Skills!how they work}

\begin{itemize}
  \item Skills are rated from \textbf{0 (untrained)} to \textbf{5 (world-class)}.
  \item Your dice pool for an action is \textbf{Attribute + Skill}.
  \item Skills can be used with different Attributes depending on the approach.
  \item Example: \textbf{Melee 3} might be used with \textbf{Body} for brute force combat (Body + Melee) or with \textbf{Wits} for tactical positioning (Wits + Melee).
\end{itemize}

\subsection*{Common Skill Categories}
\index{Skills!categories}

Fate's Edge uses flexible skill categories that can be interpreted broadly:

\begin{description}
  \item[Academia] \index{Skills!Academia} — Scholarship, research, and theoretical knowledge.
  \item[Arts] \index{Skills!Arts} — Creative expression, crafting beautiful objects.
  \item[Athletics] \index{Skills!Athletics} — Physical prowess, running, jumping, climbing.
  \item[Bonds] \index{Skills!Bonds} — Understanding and relating to specific people or groups.
  \item[Combat] \index{Skills!Combat} — Military training, weapons, and battlefield tactics.
  \item[Craft] \index{Skills!Craft} — Making, repairing, or constructing objects.
  \item[Deceive] \index{Skills!Deceive} — Lying, misdirection, and bluffing.
  \item[Empathy] \index{Skills!Empathy} — Reading emotions, motives, and social cues.
  \item[Lore] \index{Skills!Lore} — Practical knowledge about the world, cultures, and creatures.
  \item[Move] \index{Skills!Move} — Stealth, escape, and physical maneuvering.
  \item[Notice] \index{Skills!Notice} — Perception, investigation, and spotting details.
  \item[Perform] \index{Skills!Perform} — Entertainment, oration, and artistic performance.
  \item[Physique] \index{Skills!Physique} — Endurance, resistance to harm, and physical power.
  \item[Pilot] \index{Skills!Pilot} — Operating vehicles, mounts, or vessels.
  \item[Ranged] \index{Skills!Ranged} — Thrown weapons, bows, and projectile combat.
  \item[Rapport] \index{Skills!Rapport} — Building trust, empathy, and emotional connection.
  \item[Resist] \index{Skills!Resist} — Withstanding poison, disease, magic, or mental intrusion.
  \item[Stealth] \index{Skills!Stealth} — Hiding, sneaking, and avoiding detection.
  \item[Tactics] \index{Skills!Tactics} — Strategic thinking, planning, and battlefield awareness.
  \item[Will] \index{Skills!Will} — Mental fortitude, discipline, and resisting temptation.
\end{description}

\subsection*{Skill Examples in Action}
\index{Skills!examples}

Here are examples of how different Skills might be used with various Attributes:

\textbf{Athletics 3:}
\begin{itemize}
  \item Body + Athletics: Climbing a wall or swimming against a current (DV 2-4)
  \item Wits + Athletics: Navigating an obstacle course or competitive sport (DV 2-3)
  \item Physique + Athletics: Endurance running or holding a breath (DV 3-4)
\end{itemize}

\textbf{Lore 2:}
\begin{itemize}
  \item Wits + Lore: Identifying a creature or recalling historical facts (DV 1-3)
  \item Insight + Lore: Understanding cultural customs or reading social hierarchies (DV 2-3)
  \item Spirit + Lore: Recognizing divine signs or supernatural phenomena (DV 2-4)
\end{itemize}

\textbf{Craft 4:}
\begin{itemize}
  \item Wits + Craft: Designing complex mechanisms or solving engineering problems (DV 3-4)
  \item Body + Craft: Skilled manual labor or creating fine detail work (DV 2-3)
  \item Insight + Craft: Artistic creation or crafting items with symbolic meaning (DV 2-3)
\end{itemize}

\textbf{Combat 3:}
\begin{itemize}
  \item Body + Combat: Brute force melee combat (DV set by opponent)
  \item Wits + Combat: Tactical positioning or fighting multiple opponents (DV 2-4)
  \item Tactics + Combat: Leading troops or planning battlefield maneuvers (DV 3-4)
\end{itemize}

\textbf{Bonds 2:}
\begin{itemize}
  \item Presence + Bonds: Negotiating with a specific faction or group (DV 2-4)
  \item Empathy + Bonds: Understanding the motivations of people you know (DV 1-3)
  \item Sway + Bonds: Persuading someone you have a relationship with (DV 2-3)
\end{itemize}

\subsection*{Skill Specialization}
\index{Skills!specialization}

Characters can specialize in specific applications of broad Skills:
\begin{itemize}
  \item \textbf{Craft (Blacksmithing)} or \textbf{Craft (Alchemy)}
  \item \textbf{Lore (Aeler History)} or \textbf{Lore (Outsider Biology)}
  \item \textbf{Combat (Swordsmanship)} or \textbf{Combat (Formation Fighting)}
\end{itemize}

Specializations provide +1 die when used in relevant situations, but are less versatile than broad Skills.

\subsection*{Improving Skills}
\index{Skills!improving}

Skills are improved by spending XP:
\begin{itemize}
  \item Cost to improve a Skill is \textbf{new level × 2 XP}.
  \item Example: Improving Athletics from 2 to 3 costs 6 XP.
  \item Downtime equal to the new level is required for training.
\end{itemize}

\subsection*{Skills vs. Talents}
\index{Skills!vs. Talents}

\begin{description}
  \item[Skills] Provide broad competency in areas of expertise. They are mechanical bonuses added to Attributes for dice pools.
  \item[Talents] Are specific abilities, tricks, or unique capabilities. They often have special rules, limited uses, or narrative effects beyond simple dice bonuses.
\end{description}

A high Skill rating makes you reliable in your area of expertise, while Talents give you unique options and capabilities that define your character's special qualities.

\end{chapter}
