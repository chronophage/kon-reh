% !TEX root = ../fates_edge_players_guide.tex

\chapter{Character Archetypes}
\label{ch:archetypes}

In \textbf{Fate's Edge}, there are no rigid classes—only \textbf{Archetypes}, flexible narrative frameworks that help guide your character's development and role at the table. These are not mechanical restrictions, but storytelling tools to inspire concept and playstyle.

\section{What Is an Archetype?}
\index{Archetypes}

An Archetype is a broad playstyle or thematic identity. It suggests how you might spend XP, what Talents you might pursue, and how you engage with the world. But it's never a cage—you are free to blend, shift, or subvert any Archetype as your story evolves.

\section{The Three Core Archetypes}
\index{Archetypes!core}

\subsection*{1. The Solo}
\index{Archetypes!Solo}

\textbf{Theme:} The lone wolf whose personal mastery becomes legend.

\textbf{XP Focus:} Attributes and Skills. Minimal followers or holdings.

\textbf{Strengths:}
\begin{itemize}
  \item Always ready for spotlight scenes.
  \item Reliable and self-sufficient.
  \item Few moving parts—less upkeep.
\end{itemize}

\textbf{Risks:}
\begin{itemize}
  \item Narrow toolkit outside their specialty.
  \item May stall in social or logistical scenes.
  \item Vulnerable if core stats are challenged.
\end{itemize}

\textbf{Example Build:} High Body/Wits, Melee 4+, minimal assets.

\textbf{Suggested Talents:} Battle Instincts, Silver Tongue, Duelist's Insight.

\subsection*{2. The Mixed Player}
\index{Archetypes!Mixed Player}

\textbf{Theme:} A hero who balances self-growth with allies, networks, or family.

\textbf{XP Focus:} Balanced investment between self and assets.

\textbf{Strengths:}
\begin{itemize}
  \item Adaptable across many scenes.
  \item Bridges party gaps with followers or assets.
  \item Strong when weaving personal hooks into group play.
\end{itemize}

\textbf{Risks:}
\begin{itemize}
  \item Upkeep pressure on followers/assets.
  \item Helpers can be targeted by GM Complications.
  \item May lag if key assets are compromised.
\end{itemize}

\textbf{Example Build:} Presence 3, Sway 3, Cap 3 Follower, Minor Asset.

\textbf{Suggested Talents:} Familiar Bond, Guild Ties, Hearth-Banner.

\subsection*{3. The Mastermind}
\index{Archetypes!Mastermind}

\textbf{Theme:} The strategist who commands webs of allies, followers, and secrets.

\textbf{XP Focus:} Followers, Assets, and Presence.

\textbf{Strengths:}
\begin{itemize}
  \item Shapes campaigns through schemes and influence.
  \item Strong in planning, logistics, and long-term play.
  \item Can solve problems off-screen.
\end{itemize}

\textbf{Risks:}
\begin{itemize}
  \item Dependency on lanes and helpers.
  \item Complication fallout can cascade.
  \item Vulnerable if assets are seized or followers betrayed.
\end{itemize}

\textbf{Example Build:} Wits 3, Presence 3, Cap 4 Scout, Standard Asset.

\textbf{Suggested Talents:} Coordinated Assault, Shadow Broker, Master of Coin.

\section{Classic Fantasy Archetypes (Reimagined)}
\index{Archetypes!fantasy reimagined}

Fate's Edge reinterprets classic fantasy roles through its narrative lens:

\begin{description}
  \item[The Oath-Bound Blade] — A holy warrior whose power is tied to an unbreakable vow.
  \item[The Death-Speaker] — A necromancer who bargains with the dead, not commands them.
  \item[The Border-Warden] — A ranger sworn to protect a liminal territory.
  \item[The Guild-Approved Shadow] — A licensed rogue operating under factional law.
  \item[The Spirit-Touched Outlander] — A barbarian possessed by ancestral spirits.
  \item[The Scholar of Fractured Truths] — A wizard whose spells are volatile fragments of lore.
  \item[The Caretaker of Cycles] — A druid who maintains the balance of life and death.
  \item[The Chronicler of Consequences] — A bard whose songs define history.
  \item[The Ascetic of the Unbound Body] — A monk who detaches from pain and fatigue.
  \item[The Petitioner of a Silent God] — A cleric who interprets divine silence.
\end{description}

\section{Creating Your Own Archetype}
\index{Archetypes!creating}

To create a custom Archetype:

\begin{enumerate}
  \item \textbf{Name It} — Give it a evocative handle.
  \item \textbf{Define the Theme} — What drives this character?
  \item \textbf{Suggest XP Focus} — Self? Assets? Talents?
  \item \textbf{List Strengths and Risks} — What makes it unique?
  \item \textbf[Optional]{Suggest Talents or Assets} — What tools fit the theme?
\end{enumerate}

\section{Summary}

Archetypes in Fate's Edge are about \textbf{storytelling}, not mechanics:

\begin{itemize}
  \item They guide character development.
  \item They inspire narrative identity.
  \item They help you find your role at the table.
\end{itemize}

\subsection{Starting Build Points}
\index{Character creation!starting points}

Players begin with 30 Experience Points (XP) to allocate during initial character creation. This represents a balanced baseline for competent starting characters.

Choose an Archetype—or blend them. Your story is yours to shape.

\end{chapter}
