% !TEX root = ../fates_edge_players_guide.tex

\chapter{Magic \& The Arts}
\label{ch:magic}

In \textbf{Fate's Edge}, magic is not a tool of convenience—it is a dangerous negotiation with the fabric of reality. Every spell is a story beat, and every casting carries risk. This chapter introduces the \textbf{Casting Loop}, magical Backlash, and the thematic power of the Arts.

\section{Philosophy of Magic}
\index{Magic!philosophy}

Magic in Fate's Edge is:

\begin{itemize}
  \item \textbf{Powerful} — It can reshape the battlefield, the story, or the world.
  \item \textbf{Risky} — Every spell generates Complication Points (CP) \index{Complication Point (CP)}, which manifest as Backlash.
  \item \textbf{Thematic} — Effects and consequences align with the Art invoked (fire, shadow, storm, etc.).
  \item \textbf{Volatile by Design} — Magic is not fully understood.
  \item \textbf{Narrative Weight} — Casting is always a story moment.
\end{itemize}

Magic is not about stacking damage—it's about shifting the narrative.

\section{The Caster's Burden}
\index{Magic!caster's burden}

Magicians are defined not by what they can do, but by what they are willing to risk.

\section{The Casting Loop}
\index{Magic!casting loop}

All spellcasting follows a structured sequence called the \textbf{Casting Loop}. It unfolds in two phases within a single scene:

\subsection*{1. Channel}
\index{Magic!channel phase}

The caster focuses, rolling \textbf{Wits + Arcana} \index{Arcana} to gather \textbf{Potential}.

\begin{itemize}
  \item Each \textbf{Success} becomes a point of Potential.
  \item Each \textbf{1 rolled} generates a Complication Point immediately.
\end{itemize}

\subsection*{2. Weave}
\index{Magic!weave phase}

On the following turn (Player Turn) within the same scene, the caster rolls \textbf{Wits + (Art)} to shape Potential into a defined effect.

\begin{itemize}
  \item The \textbf{Description Ladder} \index{Description Ladder} applies: Intricate actions reroll 1s and may add a flourish.
  \item The spell's effect is determined by successes and the GM's interpretation of position and effect.
\end{itemize}

\subsection*{3. Backlash}
\index{Magic!backlash}

Any Complication Points generated during the Casting Loop are spent by the GM as \textbf{Backlash}.

\begin{itemize}
  \item Backlash is \textbf{thematic}—aligned with the Art used or its opposite (see Elements, p. \pageref{ch:elements}).
  \item Severity scales with the number of CP spent.
  \item Mitigation: Boons \index{Boons} do not reduce CP unless a Talent/Asset explicitly says "Mitigate CP."
\end{itemize}

\section{Backlash Severity Table}

\begin{center}
\begin{tabular}{cl}
\toprule
\textbf{CP Spent} & \textbf{Typical Consequence} \\
\midrule
1–2 & Minor nuisance or tell (noise, fatigue, brief distraction) \\
3–4 & Noticeable setback (hazard clock, condition, new pressure) \\
5+ & Major turn (scene shift, new foe, severe condition) \\
\bottomrule
\end{tabular}
\end{center}

\section{Defining Your Art}
\index{Magic!art definition}

Your Art is the narrative method by which your Weave and Cast work (sigils, sung names, lantern-law, bone charms, contracts, salt-thread, etc.).

When you gain Arcane Initiate, write a one–two line Art describing:
\begin{itemize}
    \item \textbf{Gesture \& medium} (ink, chord, breath, light, bone, law).
    \item \textbf{Typical Elements} you lean on (pick 2 you're often aligned to).
\end{itemize}

\subsection*{Art in Play}
\begin{itemize}
    \item \textbf{Spotlight bump (1/scene):} if your Art is clearly honored in fiction (right tools/time/setting), gain +1 die on your Cast. This counts toward the +3 dice cap.
    \item \textbf{Off-style strain:} if you're forced to work wildly against your Art (no tools, hostile locus), the GM may set a worse Position or require you to accept extra Backlash choices on a Partial.
\end{itemize}

\section{Common Magical Arts}
\index{Magic!arts}

Each Art has its own flavor and risk. Below are examples:

\begin{description}
  \item[Pyromancy] \index{Pyromancy} — Fire and heat. Backlash: Flames leap to unattended surfaces, smoke blinds allies, or the heat weakens structures.
  \item[Umbramancy] \index{Umbramancy} — Shadow and silence. Backlash: Illusions persist too long, unseen things whisper truths best left hidden, morale crumbles.
  \item[Stormcraft] \index{Stormcraft} — Wind and lightning. Backlash: Winds scatter allies' plans, lightning arcs toward unintended targets, storms linger beyond the caster's will.
  \item[Geomancy] \index{Geomancy} — Stone and structure. Backlash: rigidity, slow movement, guardians awaken.
  \item[Hydromancy] \index{Hydromancy} — Water and flow. Backlash: stagnation, flooding, pests drawn.
  \item[Vitalism] \index{Vitalism} — Life and healing. Backlash: overgrowth, exhaustion, sympathetic drain.
  \item[Thaumaturgy] \index{Thaumaturgy} — Divine or holy magic. Backlash: flickering sanctity, beacon effects, spiritual fatigue.
\end{description}

\section{Ritual Casting (Optional Rule)}
\index{Magic!ritual casting}

For greater effects, multiple casters can join in a \textbf{Ritual}.

\begin{itemize}
  \item \textbf{Ritual Helper Cap}: You may draw on ceil(Arcana/2) helpers (max 3).
  \item \textbf{Procedure}:
  \begin{enumerate}
    \item Declare the Ritual.
    \item Channel Together (Scene-long action).
    \item Weave (Scene-long action).
    \item Backlash.
  \end{enumerate}
  \item Helpers may use different relevant skills if their procedure is fictionally distinct.
  \item CP from Channel resolves on that roller. CP from Weave is assigned to the primary caster.
\end{itemize}

Rituals increase the ceiling of magic—but also the risk. Backlash severity increases with each helper beyond the first.

\section{Rites (Pact Magic)}
\index{Magic!rites}

Rites are precise, named effects granted by powerful Patrons. This framework keeps play fast and consistent.

\subsection*{Becoming an Acolyte}
\begin{itemize}
    \item Choose one or two Patrons. (Two is allowed; see Cross-Patron Interference.)
    \item Start with 2 Low and 1 Standard Rite total (split across your Patrons as you wish).
    \item Track an Obligation ledger per Patron (segments of debt accrued through use).
\end{itemize}

\subsection*{Using a Rite}
Unless a Rite says otherwise:
\begin{enumerate}
    \item \textbf{Invoke} (1 action). Speak the name, draw the sign, or employ the proper tool. If a roll is required, the GM sets DV by fiction (typical scene DV 2–4).
    \item \textbf{Mark Obligation}. Invoking a Rite usually marks +1 segment to that Patron's ledger (some Low rites may be free at GM discretion).
    \item \textbf{Push It} (optional). Amplify the effect; mark +1 Obligation. A given Rite may be Pushed at most once per scene.
    \item \textbf{Backlash} (on a 1 or Miss). The GM inflicts a fitting consequence or marks +1 Obligation.
\end{enumerate}

\subsection*{Stacking \& Caps}
\begin{itemize}
    \item Same-source scene buffs do not stack. Take the best version; others are suppressed.
    \item A single Rite can add at most +3 Obligation segments per scene from that Rite (Invoke + Push + possible Backlash).
\end{itemize}

\subsection*{Clearing Obligation}
Between scenes or in downtime, you may clear 1–2 segments per Patron by:
\begin{itemize}
    \item Service fitting the Patron (tend hinges for the Gate, keep sanctioned watch for the Oath, etc.), or
    \item Spending 1–2 Boons to represent offerings, favors, or perfect observance.
\end{itemize}
At +3/+6/+9 total segments, foreshadow manifestations (omens, auditors, bailiffs) as your table prefers.

\section{Spell Design Guidelines}
\index{Magic!spell design}

When creating new spells, follow this format:

\begin{enumerate}
  \item \textbf{Name \& Art} — e.g., Cinder-Fist (Pyromancy)
  \item \textbf{Effect} — One clear narrative change: start Controlled, +1 effect, Hazard –1, etc.
  \item \textbf{Backlash Ladder} — 1–2 CP = minor cost; 3–4 CP = pressure; 5+ CP = scene twist.
\end{enumerate}

\section{Example Spell: Cloak of Shadows}
\index{Spells!Cloak of Shadows}

\begin{description}
  \item[Art] Umbramancy
  \item[DV] 2
  \item[Effect] In dim or darker light, target starts Controlled vs sight-based detection.
  \item[Backlash]
    \begin{itemize}
      \item 1 CP — Slight self-blindness (–1 die on sight checks).
      \item 2 CP — Whispering shadows create a faint tell.
      \item 3 CP — You borrow light from elsewhere, leaving a conspicuously bright patch.
      \item 4+ CP — A shadow-being takes interest.
    \end{itemize}
\end{description}

\section{Prestige Magical Abilities}
\index{Prestige Abilities!magical}

\begin{itemize}
    \item \textbf{Echo-Walker's Step} (High Elf, Cost: 20 XP; Req: Wits 5, Arcana 4): 
1/arc, \emph{observe} a perfect echo of a past event at your location (no retconning). 
GM immediately banks +2 CP; scenes touching that memory carry an omen. Grants DV −1 on one action that uses the revealed truth.
    \item \textbf{Spirit-Shield} (Aeler, Cost: 15 XP; Req: Spirit 4, Insight 3): 
1/session, erase up to 3 CP from an ally's \emph{current} roll; you immediately mark Fatigue +1 and the GM banks +1 CP as backlash.
\end{itemize}

\section{Summary}

Magic in Fate's Edge is never "safe." Every casting:

\begin{itemize}
  \item Advances the story.
  \item Risks Backlash.
  \item Requires a cost—whether in CP, Fatigue, or narrative tension.
\end{itemize}

Embrace the risk—and let the Arts reshape the world.

\end{chapter}
