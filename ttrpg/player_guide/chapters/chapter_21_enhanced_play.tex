\chapter{Enhanced Player Play}

The Crown system becomes more engaging when you actively manage resources and participate in collaborative storytelling. These enhanced mechanics give you more agency and meaningful choices.

\section{Player Resources}

Manage these key resources to enhance your gameplay experience and create more collaborative outcomes.

\subsection{Shared Leverage Pool}

Participate in a collaborative economy where helping each other becomes strategic.

\subsubsection{Pool Participation}

\begin{itemize}
\item Contribute 1 leverage each to shared pool at session start
\item Spend from pool to avoid complications or enhance actions
\item Help other players by contributing to their successes
\item Pool refreshes each session
\end{itemize}

\subsubsection{Spending Options}

\begin{description}
\item[1 Leverage:] Avoid minor complication, gain small advantage
\item[2 Leverage:] Gain significant advantage, rewrite recent outcome
\item[3+ Leverage:] Major plot influence, introduce new element
\end{description}

\subsubsection{Collaborative Spending}

\begin{itemize}
\item Help allies by spending leverage on their behalf
\item Pool leverage for group actions (costs 1 per participant)
\item Trade leverage for information or favors from other players
\end{itemize}

\subsection{Session Investment Tracker}

Rate your engagement to earn collaborative rewards.

\subsubsection{Investment Scale}

Rate your session investment 1-3:
\begin{description}
\item[1 - Low Investment:] Minimal roleplay, following others' lead
\item[2 - Medium Investment:] Active participation, some creative input
\item[3 - High Investment:] Leading scenes, creative problem-solving, strong roleplay
\end{description}

\subsubsection{Investment Rewards}

\begin{itemize}
\item Consistent 3s = Permanent +1 to relationship rolls
\item Mix of investments = Bank 1 leverage per session average
\item GM may offer bonus leverage for exceptional investment
\end{itemize}

\subsection{Cultural Immersion Bonus}

Earn mechanical benefits through immersive roleplay.

\subsubsection{Earning Bonuses}

Gain 1 leverage for:
\begin{itemize}
\item Using appropriate cultural terminology
\item Engaging with generated cultural elements
\item Making culturally consistent character choices
\item Contributing to atmospheric scene-setting
\end{itemize}

\subsubsection{Immersion Rewards}

\begin{itemize}
\item 3 culture points = +1 to related rolls or 1 leverage
\item 5 culture points = Diamond reroll or relationship bonus
\item GM recognition = Additional narrative opportunities
\end{itemize}

\section{Collaborative Play}

Mechanics that make you active participants in narrative creation.

\subsection{Information Trading}

Negotiate for information using leverage and creative problem-solving.

\subsubsection{Requesting Information}

\begin{itemize}
\item Declare information need and offer leverage (1-3)
\item Be specific about what you want to know
\item Accept partial information for reduced cost
\item Negotiate creative information-gathering methods
\end{itemize}

\subsubsection{Information Value Scale}

\begin{description}
\item[1 Leverage:] Basic facts, surface details, common knowledge
\item[2 Leverage:] Strategic insights, tactical advantages, moderate secrets
\item[3+ Leverage:] Major revelations, plot-critical information, deep secrets
\end{description}

\subsubsection{Creative Investigation}

\begin{itemize}
\item Spend leverage for investigation assistance
\item Use relationship dice to gain informant access
\item Trade favors with other players for shared information
\item Accept complications in exchange for clues
\end{itemize}

\subsection{Clock Manipulation}

Influence narrative pacing through strategic resource management.

\subsubsection{Player Options}

\begin{itemize}
\item Spend 1 leverage to slow clock by 1 segment
\item Spend 1 leverage to hasten clock by 1 segment
\item Maximum 1 segment change per player per scene
\item Negotiate with GM for larger changes
\end{itemize}

\subsubsection{Strategic Timing}

\begin{itemize}
\item Slow clock when you need more investigation time
\item Hasten clock when you want to resolve tension
\item Coordinate with allies for maximum effect
\item Accept GM counter-spending as part of negotiation
\end{itemize}

\subsection{Complication Bargaining}

Request specific challenge types to shape your narrative experience.

\subsubsection{Challenge Requests}

Request complications for engagement and reward:
\begin{itemize}
\item Social complications (feuds, negotiations, diplomacy)
\item Physical challenges (combat, exploration, survival)
\item Mystery elements (investigation, puzzles, hidden information)
\item Moral dilemmas (ethical conflicts, difficult choices)
\end{itemize}

\subsubsection{Bargaining Benefits}

\begin{itemize}
\item Requested complications = +1 relationship die with relevant faction
\item Creative complications = 1 temporary leverage
\item Challenging yourself = GM investment in your story
\item Collaborative complications = Group engagement bonus
\end{itemize}

\section{Faction Awareness}

Understand world state and make informed decisions about your actions.

\subsection{Loyalty Recognition}

Learn to read faction relationships for strategic advantage.

\subsubsection{Loyalty Indicators}

Watch for signs of faction attitudes:
\begin{description}
\item[Enemy (-3):] Open hostility, active sabotage, public denunciation
\item[Hostile (-2):] Cold reception, bureaucratic obstacles, passive aggression
\item[Unfriendly (-1):] Minimal cooperation, guarded responses, neutral formality
\item[Neutral (0):] Businesslike interactions, standard procedures, indifference
\item[Friendly (+1):] Helpful suggestions, minor favors, warm reception
\item[Supportive (+2):] Active assistance, shared resources, personal attention
\item[Ally (+3):] Sacrificial support, privileged information, personal loyalty
\end{description}

\subsubsection{Loyalty Shifting}

Your actions change faction attitudes:
\begin{itemize}
\item Consistent help = Gradual loyalty improvement
\item Betrayal = Immediate loyalty drop
\item Neutral actions = Stable relationships
\item Mixed actions = Complex loyalty patterns
\end{itemize}

\subsection{Cross-Cultural Opportunities}

Recognize when elements from different decks create synergy opportunities.

\subsubsection{Synergy Recognition}

Look for connections between:
\begin{itemize}
\item Maritime and criminal elements (Zakov + Kahfagia)
\item Rural and supernatural elements (Aelaerem + Aelinnel)
\item Urban and bureaucratic elements (Ecktoria + Aeler)
\item Military and political elements (Black Banners + Acasia)
\end{itemize}

\subsubsection{Synergy Benefits}

\begin{itemize}
\item Recognize cross-deck connections = +1 to relevant rolls
\item Create perfect matches = Bonus leverage or relationship die
\item Suggest cross-cultural solutions = GM investment bonus
\end{itemize}

\section{Advanced Techniques}

Sophisticated approaches to resource management and collaborative play.

\subsection{Momentum Banking}

Strategic resource management for long-term benefits.

\subsubsection{Earning Momentum}

\begin{itemize}
\item Resolve conflicts under standard time = Bank 1 momentum per segment under
\item Creative problem-solving = Bonus momentum opportunities
\item Helping allies = Shared momentum benefits
\item Strategic retreat = Preserved momentum
\end{itemize}

\subsubsection{Spending Momentum}

\begin{itemize}
\item +1 to any relationship roll
\item 1 free leverage
\item Reroll one diamond draw
\item Minor narrative influence
\end{itemize}

\subsection{Escalation Management}

Strategic approach to conflict and tension.

\subsubsection{Player Escalation Options}

\begin{itemize}
\item Spend 1 leverage to de-escalate conflict
\item Accept minor complication to avoid major threat
\item Redirect conflict toward different target
\item Negotiate temporary truce or ceasefire
\end{itemize}

\subsubsection{Escalation Benefits}

\begin{itemize}
\item Controlled escalation = Narrative investment
\item Strategic retreat = Preserved resources
\item Creative resolution = GM recognition bonus
\item Collaborative conflict = Group advantages
\end{itemize}

\section{Implementation Timeline}

Gradual adoption of enhanced mechanics for smooth learning curve.

\subsection{Getting Started (Sessions 1-3)}

\begin{itemize}
\item Participate in Shared Leverage Pool
\item Rate your Session Investment each session
\item Request Information using leverage
\item Try Complication Bargaining with GM
\end{itemize}

\subsection{Building Skills (Sessions 4-6)}

\begin{itemize}
\item Use Clock Manipulation strategically
\item Earn Cultural Immersion Bonuses
\item Recognize Faction Loyalty patterns
\item Participate in Momentum Banking
\end{itemize}

\subsection{Master Level (Sessions 7+)}

\begin{itemize}
\item Strategic Cross-Deck Synergy creation
\item Advanced Escalation Management
\item Collaborative Resource Trading
\item Leadership in Session Investment
\end{itemize}

These enhanced player mechanics transform you from passive participants to active co-creators of the narrative, with meaningful resources to manage and strategic choices that shape the story.

\subsection{Enhanced Starting Builds}
Players may exceed the standard 30 XP build through narrative engagement:

\begin{itemize}
    \item \textbf{Bonds:} Up to two player-defined mutual bonds may be taken for +2 XP total (Section \ref{sessionawards})
    \item \textbf{Complications:} Up to two initial complications may be accepted for +4 XP total (Section \ref{sessionawards})
\end{itemize}

This allows for a maximum starting build of 34 XP, though players are encouraged to aim for 30 XP and use bonds/complications to mitigate slight overages while maintaining narrative balance.

\subsection{Recommended Approach}
The GM should encourage players to:
\begin{itemize}
    \item Target 30 XP for balanced starting characters
    \item Use bonds and complications to enhance characterization rather than pure mechanical optimization
    \item Consider the narrative implications of any starting advantages.
\end{itemize}
\textbf{NOTE: Scenes start with +1 CP per complication per character until they are cleared}


