% !TEX root = ../fates_edge_players_guide.tex

\chapter{Welcome to Fate’s Edge}

\emph{A world where every choice carries weight, every spell risks backlash, and every legend is written in the shadow of consequence.}

Welcome to \textbf{Fate’s Edge}, a tabletop roleplaying game where narrative drives mechanics, and every decision shapes not only your character’s path—but the world around them. This is not a game of perfect successes or clean victories. It is a game of risk, drama, and legacy.

\section*{What Is Fate’s Edge?}

Fate’s Edge is a narrative-first RPG where:
\begin{itemize}
  \item Every roll introduces the potential for triumph \emph{and} complication.
  \item Magic is powerful—but dangerous.
  \item Choices ripple outward, shaping both character arcs and the setting.
  \item Growth is meaningful, earned through XP spent on skills, assets, and unique talents.
\end{itemize}

This guide is your primer to the world, the people, and the powers that define Fate’s Edge. It is designed to help you build a character, understand the setting, and step into a world where your actions matter.

\section*{Design Philosophy}

Fate’s Edge is built on four core principles:

\begin{description}
  \item[Narrative Primacy] Mechanics serve the story. Rules reward descriptive play and creative problem-solving.
  \item[Risk as Drama] Every roll carries tension. Even success may come at a cost.
  \item[Meaningful Growth] XP is a currency of choice. Invest in yourself, your allies, or your influence on the world.
  \item[Consequence Weight] No action is free. Every choice changes the fiction, for better or worse.
\end{description}

\section*{Tone of Play}

Fate’s Edge encourages cinematic, collaborative storytelling. Expect:
\begin{itemize}
  \item Stories driven by character choices, not predetermined plots.
  \item A world that reacts to your decisions—both big and small.
  \item Themes of legacy, sacrifice, and moral ambiguity.
\end{itemize}

Whether you’re a lone duelist, a scheming mastermind, or a spirit-touched outlander, your path is yours to forge—and the world will remember it.

\section*{What’s in This Guide}

This Player’s Guide is divided into thematic chapters to help you build and play your character with confidence:

\begin{itemize}
  \item \textbf{Core Mechanics} — How to resolve actions, spend XP, and manage consequences.
  \item \textbf{Character Creation} — Attributes, skills, paths, and archetypes.
  \item \textbf{Magic and Talents} — Dangerous arts, cultural abilities, and unique powers.
  \item \textbf{World and Lore} — The lands, peoples, and gods of the Amaranthine Sea.
  \item \textbf{Assets and Followers} — How to build influence beyond the self.
  \item \textbf{Appendices} — Quick reference sheets, compendiums, and deck generators.
\end{itemize}

\section*{How to Use This Guide}

This guide is modular. You can read it cover to cover or jump to sections relevant to your character concept. Each chapter is designed to stand alone while connecting to the broader themes of the game.

Use this guide alongside the \emph{Fate’s Edge System Reference Document (SRD)} for full mechanical support.

\section*{A Final Word}

Fate’s Edge is a game of bold choices and lasting consequences. Your story is not written in dice, but in the decisions you make—and the price you’re willing to pay.

Welcome to the Edge. The world is watching.

\begin{center}
  \emph{What are you willing to risk to reshape the world around you?}
\end{center}

\begin{tcolorbox}[enhanced, sharp corners, boxrule=1pt, colback=gray!5!white, colframe=gray!75!black, title={Flavor is Free}]
\textbf{Players and GMs:} Remember that in Fate's Edge, \textbf{flavor is free}!

This means you can add descriptive details, cultural elements, and atmospheric touches to your actions without spending resources or requiring mechanical justification. Want to perform a parry with the traditional Aelerian bell-guard technique? Go ahead! Want to invoke the seasonal festivals of Theona when making a social roll? Perfect!

Flavor doesn't change the mechanical outcome of your actions, but it makes the world come alive and helps everyone at the table visualize and engage with the rich setting. Describe your character's background, their cultural customs, the local architecture, or the atmospheric details of a scene. These elements enrich the narrative without requiring dice rolls or resource expenditure.

The GM should encourage flavorful descriptions and may even provide additional descriptive details about the world in return. This collaborative approach to world-building through flavor helps create a more immersive experience for everyone involved.

Remember: Mechanics determine success or failure, but flavor determines the story we tell about how that success or failure came to be.
\end{tcolorbox}

\end{chapter}
