% !TEX root = ../fates_edge_players_guide.tex

\chapter{World Interaction}

In \textbf{Fate's Edge}, the world is not a static backdrop—it reacts to your choices, complicates your plans, and evolves with your story. This chapter explores how you move through the world, what happens when you get lost, and how consequences ripple outward.

\section{Travel Framework}

Travel in Fate's Edge is not about counting hexes or tracking rations—it's about narrative pacing and the risks you encounter along the way. The GM uses a \textbf{Travel Framework} to keep movement tense and meaningful.

\subsection*{Two Deck Systems (Compatibility)}

Fate's Edge uses two distinct card tools:

\paragraph{Travel Decks (regional, 52-card).}
\emph{Spade}=Place, \emph{Heart}=Actor, \emph{Club}=Pressure, \emph{Diamond}=Leverage. These power journeys and gates.

\paragraph{Deck of Consequences (scene drama).}
\emph{Hearts}=social fallout, \emph{Swords}=harm/escalation, \emph{Pentacles}=material cost, \emph{Wands}=magical/spiritual disturbance.

\textit{Guidance:} Never mix suit meanings across decks. When a rule references ``Spade/Club/Diamond,'' it means \emph{Travel}. When it says ``Hearts/Swords/Pentacles/Wands,'' it means \emph{Consequences}.

\subsection*{Core Travel Procedure}

For each leg of a journey, draw 3–4 cards using the decks for your destination and controlling authority:

\begin{itemize}
  \item Spade from the destination deck: sets the scene (place).
  \item Heart from the destination deck: introduces the local actor or faction.
  \item Club from the Wilds (general hazards) or destination (if strongly policed): brings pressure.
  \item Diamond from the authority that gates the route: papers, escorts, rights, or exceptions.
\end{itemize}

\subsection*{Clock Size}

The highest card rank determines the \textbf{Clock Size} for the travel leg:

\begin{center}
\begin{tabular}{cl}
\toprule
\textbf{Rank} & \textbf{Clock Size} \\
\midrule
2–5 & 4 segments \\
6–10 & 6 segments \\
J/Q/K & 8 segments \\
Ace & 10 segments \\
\bottomrule
\end{tabular}
\end{center}

\subsection*{Resolving Travel}

Each card drawn introduces a narrative element. The GM describes how it affects the journey. Players may act to resolve complications or advance the clock.

Set a travel clock by the highest rank. On success, advance to the next leg; on failure, mark delay, debt, or diversion and resolve a consequence in the fiction.

\section{Narrative Time}

Time in Fate's Edge is flexible and story-driven. Actions are framed in four narrative scales:

\begin{description}
  \item[A Moment] — A heartbeat, a glance, a single strike or word.
  \item[Some Time] — A few minutes: a skirmish, a careful lockpick, a short negotiation.
  \item[Significant Time] — Hours: travel between locations, working a ritual, recovering from harm.
  \item[Days] — Large-scale endeavors: marches, training, major recovery.
\end{description}

\section{Deck of Consequences}

The \textbf{Deck of Consequences} is a shared storytelling tool. Whenever you roll a 1 and generate a Complication Point, the GM may draw a card instead of improvising a twist.

\subsection*{Using the Deck}

After a roll that generates CP, the GM chooses one method for that roll:
\begin{enumerate}
  \item \textbf{Direct Spend}: Translate CP into consequences/rail ticks immediately.
  \item \textbf{Deck Draw}: Draw up to \textbf{min(CP, 3)} cards and synthesize a single twist guided by suit and highest rank.
\end{enumerate}

\subsection*{Structure of the Deck}

\begin{description}
  \item[Suits] Represent domains of complications:
    \begin{itemize}
      \item Hearts — Emotional, social, or relational fallout.
      \item Swords — Harm, danger, or escalation of conflict.
      \item Pentacles — Resource strain, economic or material cost.
      \item Wands — Magical, spiritual, or cosmic disturbances.
    \end{itemize}
  \item[Ranks] Represent severity:
    \begin{itemize}
      \item Ace–3 — Minor inconvenience or flavor complication.
      \item 4–6 — Moderate setback with some narrative teeth.
      \item 7–9 — Significant consequence altering the course of action.
      \item 10–King — Major fallout, introducing new problems or lasting scars.
    \end{itemize}
\end{description}

\section{Supply Clock}

The \textbf{Supply Clock} tracks the party's access to food, water, and gear.

\begin{center}
\begin{tabular}{cl}
\toprule
\textbf{Segments Filled} & \textbf{Effect} \\
\midrule
0 (Full) & The party is well-equipped. \\
2 (Low) & Minor narrative complications (bland food, damaged arrows, thinning waterskins). \\
3 (Dangerous) & Each character gains Fatigue. \\
4 (Empty) & Severe penalties. \\
\bottomrule
\end{tabular}
\end{center}

\section{Condition Tracks}

Characters and assets have \textbf{Condition Tracks} that reflect wear, neglect, or harm:

\begin{description}
  \item[Assets/Followers] — Maintained → Neglected → Compromised
  \item[Party Resources] — Supply (0-Full → 2-Low → 3-Dangerous → 4-Empty)
  \item[Character State] — Fatigue (1-4 levels, re-roll successes)
\end{description}

\subsection*{Fatigue}
\begin{itemize}
    \item Effect: On their next roll, a character must reroll one success.
    \item Stacking: Each level adds another forced reroll.
    \item Recovery: A night's rest with adequate supply removes 1 Fatigue.
\end{itemize}

\section{Engaging the World}

The world of Fate's Edge is alive. When you:

\begin{itemize}
  \item Enter a new region, draw cards to seed local flavor.
  \item Negotiate with a faction, consider their suit (Heart = personal, Diamond = leverage).
  \item Face a hazard, let the Deck of Consequences guide the fallout.
\end{itemize}

\section{Summary}

The world of Fate's Edge is not a puzzle to be solved—it is a living, reactive force. Travel is a narrative journey, not a logistical grind. Every step forward risks a twist, and every twist changes the story.

Engage with the world boldly—and let it shape you in return.
