% !TEX root = ../fates_edge_players_guide.tex

\chapter{World Interaction}

In \textbf{Fate’s Edge}, the world is not a static backdrop—it reacts to your choices, complicates your plans, and evolves with your story. This chapter explores how you move through the world, what happens when you get lost, and how consequences ripple outward.

\section{Travel Framework}

Travel in Fate’s Edge is not about counting hexes or tracking rations—it’s about narrative pacing and the risks you encounter along the way. The GM uses a \textbf{Travel Framework} to keep movement tense and meaningful.

\subsection*{Drawing Until All Suits Appear}

To resolve a travel leg, draw cards from a standard deck until all four suits appear:

\begin{description}
  \item[♠ Spades] — Places (landmarks, terrain, hidden dangers)
  \item[♡ Hearts] — Actors (NPCs, factions, guides)
  \item[♣ Clubs] — Pressures (weather, hazards, delays)
  \item[♢ Diamonds] — Leverage (shortcuts, resources, safe passage)
\end{description}

\subsection*{Clock Size}

The highest card rank determines the \textbf{Clock Size} for the travel leg:

\begin{center}
\begin{tabular}{cl}
\toprule
\textbf{Rank} & \textbf{Clock Size} \\
\midrule
2–5 & 4 segments \\
6–10 & 6 segments \\
J/Q/K & 8 segments \\
Ace & 10 segments \\
\bottomrule
\end{tabular}
\end{center}

\subsection*{Resolving Travel}

Each card drawn introduces a narrative element. The GM describes how it affects the journey. Players may act to resolve complications or advance the clock.

\section{Narrative Time}

Time in Fate’s Edge is flexible and story-driven. Actions are framed in four narrative scales:

\begin{description}
  \item[A Moment] — A heartbeat, a glance, a single strike or word.
  \item[Some Time] — A few minutes: a skirmish, a careful lockpick, a short negotiation.
  \item[Significant Time] — Hours: travel between locations, working a ritual, recovering from harm.
  \item[Days] — Large-scale endeavors: marches, training, major recovery.
\end{description}

\section{Deck of Consequences}

The \textbf{Deck of Consequences} is a shared storytelling tool. Whenever you roll a 1 and generate a Complication Point, the GM may draw a card instead of improvising a twist.

\subsection*{Structure of the Deck}

\begin{description}
  \item[Suits] Represent domains of complications:
    \begin{itemize}
      \item ♡ Hearts — Emotional or social fallout
      \item ♢ Diamonds — Resource or wealth loss
      \item ♣ Clubs — Physical harm or obstacles
      \item ♠ Spades — Mystical or narrative twists
    \end{itemize}
  \item[Ranks] Represent severity:
    \begin{itemize}
      \item 2–5 — Minor inconvenience
      \item 6–9 — Moderate setback
      \item 10–King — Severe twist
      \item Ace — Catastrophic turn
    \end{itemize}
\end{description}

\section{Supply Clock}

The \textbf{Supply Clock} tracks the party’s access to food, water, and gear.

\begin{center}
\begin{tabular}{cl}
\toprule
\textbf{Segments Filled} & \textbf{Effect} \\
\midrule
0 (Full) & No penalties \\
2 (Low) & Minor complications \\
3 (Dangerous) & Each PC gains Fatigue \\
4 (Empty) & Severe penalties; starvation or gear failure \\
\bottomrule
\end{tabular}
\end{center}

\section{Condition Tracks}

Characters and assets have \textbf{Condition Tracks} that reflect wear, neglect, or harm:

\begin{description}
  \item[Assets/Followers] — Maintained → Neglected → Compromised
  \item[Party Resources] — Supply Clock (0–4 segments)
  \item[Character State] — Fatigue (1–4 levels), Harm (1–3 levels)
\end{description}

\section{Engaging the World}

The world of Fate’s Edge is alive. When you:

\begin{itemize}
  \item Enter a new region, draw a card to seed local flavor.
  \item Negotiate with a faction, consider their suit (Hearts = personal, Spades = structural).
  \item Face a hazard, let the Deck of Consequences guide the fallout.
\end{itemize}

\section{Summary}

The world of Fate’s Edge is not a puzzle to be solved—it is a living, reactive force. Travel is a narrative journey, not a logistical grind. Every step forward risks a twist, and every twist changes the story.

Engage with the world boldly—and let it shape you in return.


