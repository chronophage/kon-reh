\chapter{Assets and Followers}
\label{ch:assets-followers}

Your character's influence extends beyond personal capabilities through \textbf{Assets} and \textbf{Followers}. These represent worldly possessions, connections, and allies that can solve problems, provide assistance, and shape the narrative.

\section{Understanding Assets and Followers}
\index{assets}\index{followers}

\subsection*{Key Differences}
\begin{itemize}
\item \textbf{Assets}: Off-screen resources that solve problems between scenes.
\item \textbf{Followers}: On-screen allies who assist during gameplay.
\item \textbf{Assets} change the fictional situation before you arrive.
\item \textbf{Followers} act alongside you in the moment.
\end{itemize}

\subsection*{Management Requirements}
Both require maintenance and carry risks:
\begin{itemize}
\item Regular upkeep costs (XP or downtime).
\item Vulnerability to complications and attacks.
\item Narrative consequences for misuse or neglect.
\end{itemize}

\section{Assets System}
\index{assets!system}

Assets are possessions, properties, or resources you control.

\subsection*{Asset Types and Costs}
\begin{center}
\small
\begin{tabular}{lll}
\toprule
\textbf{Type} & \textbf{XP Cost} & \textbf{Establishment Time} \\
\midrule
Minor   & 4 XP  & 1 day \\
Standard& 8 XP  & 1 week \\
Major   & 12 XP & 1 month \\
\bottomrule
\end{tabular}
\end{center}

\subsection*{Asset Examples}
\begin{description}
\item[Minor Assets] Small shop, safehouse, minor title, basic workshop.
\item[Standard Assets] Noble title, guild membership, trading post, spy network.
\item[Major Assets] Fortress, city license, major enterprise, regional influence.
\end{description}

\subsection*{Using Assets}
Assets provide benefits in different ways:

\paragraph{Free Off-Screen Use}
Each asset has a specific off-screen effect you can use once per session:
\begin{itemize}
\item \textbf{Safehouse}: Provide secure lodging for the party.
\item \textbf{Spy Network}: Gather basic intelligence about a location.
\item \textbf{Workshop}: Repair or create simple items between adventures.
\item \textbf{Trading Post}: Acquire common goods at better prices.
\end{itemize}

\paragraph{Boon Activation}
Spend \textbf{1 Boon} to use an asset dramatically during a scene:
\begin{itemize}
\item \textbf{Safehouse}: Suddenly reveal a hidden escape route.
\item \textbf{Spy Network}: Produce crucial information at a critical moment.
\item \textbf{Workshop}: Create an improvised solution to an immediate problem.
\item \textbf{Trading Post}: Call in a favor from a business contact.
\end{itemize}

\paragraph{XP Activation}
Spend \textbf{2 XP} to use an asset's off-screen effect outside your normal allowance:
\begin{itemize}
\item Emergency use when you've already used your free activation.
\item Additional uses during downtime periods.
\item Special circumstances requiring extra asset support.
\end{itemize}

\section{Asset Conditions}
\index{assets!conditions}

Assets have condition states affecting their usefulness:

\subsection*{Condition Levels}
\begin{description}
\item[Maintained] Fully functional, no penalties.
\item[Neglected] $-1$ die when used; requires attention.
\item[Compromised] Unavailable until repaired or recovered.
\end{description}

\subsection*{Maintenance Requirements}
\begin{itemize}
\item \textbf{Regular Upkeep}: Two options per SRD §21.2:
      \begin{itemize}
      \item \textbf{Efficient} (Higher XP, Less Time): Pay Upkeep XP = $\max(1, \text{XP Acquisition})/3$, minimal effort
      \item \textbf{Intensive} (Lower XP, More Time): Pay 1 XP, dedicated downtime action
      \end{itemize}
\item \textbf{Neglect}: Assets deteriorate if not maintained.
\item \textbf{Recovery}: Compromised assets require significant effort to restore.
\end{itemize}

\section{Followers System}
\index{followers}

Followers are characters who assist you directly.

\subsection*{Follower Capability Ratings}
Followers are rated by Capability (\textbf{Cap}) from 1 to 5:
\begin{center}
\small
\begin{tabular}{ll}
\toprule
\textbf{Cap} & \textbf{Description} \\
\midrule
1 & Novice helper, basic assistance \\
2 & Competent assistant, reliable support \\
3 & Skilled specialist, valuable aid \\
4 & Expert ally, significant capability \\
5 & Master companion, exceptional ability \\
\bottomrule
\end{tabular}
\end{center}

\subsection*{Follower Costs}
\begin{itemize}
\item \textbf{XP Cost}: Capability squared (\emph{Cap}$^2$).
\item \textbf{Example}: Cap 3 follower costs $3^2=9$ XP.
\item \textbf{Recruitment}: 1--3 days downtime to find and brief.
\item \textbf{Limits}: The GM may set maximum followers based on story.
\end{itemize}

\subsection*{Follower Types}
\begin{description}
\item[Combat Allies] Warriors, guards, mercenaries.
\item[Technical Experts] Craftspeople, engineers, specialists.
\item[Social Contacts] Informants, diplomats, agents.
\item[Specialists] Unique capabilities like magic or stealth.
\end{description}

\section{Using Followers}
\index{followers!usage}

\subsection*{Assistance in Scenes}
Followers can help with your actions:
\begin{itemize}
\item \textbf{Assist Dice}: Add dice equal to $\min(\text{Cap}, \text{relevant skill})$.
\item \textbf{Maximum Bonus}: +3 dice total from all sources.
\item \textbf{Cost}: Spend 1 Boon or 1 Stress to add +1 die (max +3 from assists).
\item \textbf{One Helper}: Only one follower can assist per action.
\end{itemize}

\subsection*{Independent Actions}
Once per scene (party-wide), a follower can take a small action:
\begin{description}
\item[\emph{Scout \& Signal}] Change an ally's next action to \textbf{Dominant} position.
\item[\emph{Distract \& Draw}] Reduce a threat clock by 1 segment.
\item[\emph{Fetch \& Carry}] Move an object through danger safely.
\end{description}

\subsection*{Cost of Independent Actions}
\begin{itemize}
\item Mark +1 \textbf{Exposure} (attention or stress), \emph{or}
\item Take \textbf{Harm 1} (injury or trauma).
\item Cannot be used if the follower is already \textbf{Compromised}.
\end{itemize}

\section{Follower Conditions}
\index{followers!conditions}

Followers track two condition types:

\subsection*{Exposure}
Represents attention, stress, or narrative pressure:
\begin{itemize}
\item \textbf{Gains}: From independent actions, dangerous situations, complications.
\item \textbf{Effects}: Increased risk, reduced effectiveness, attention from enemies.
\item \textbf{Recovery}: Downtime activities, careful management.
\end{itemize}

\subsection*{Harm}
Represents injury, trauma, or damage:
\begin{itemize}
\item \textbf{Gains}: From combat, accidents, enemy attacks.
\item \textbf{Effects}: Penalties to assistance, possible incapacity.
\item \textbf{Recovery}: Medical care, rest, magical healing.
\end{itemize}

\subsection*{Condition States}
\begin{description}
\item[Maintained] Ready and reliable, full capability.
\item[Neglected] Needs attention, $-1$ die to assistance.
\item[Compromised] Unavailable: captured, defected, lost, or incapacitated.
\end{description}

\section{Follower Risks}
\index{followers!risks}

Using followers carries significant risks:

\subsection*{Complication Targeting}
When the GM spends \textbf{2+ Story Beats} on an action where you have assistance:
\begin{itemize}
\item The follower may face consequences instead of you.
\item Could be injury, capture, betrayal, or other complications.
\item Fictionally appropriate to the situation.
\end{itemize}

\subsection*{Off-Screen Capability}
Once per downtime, a \textbf{Cap 5} follower can solve a significant problem:
\begin{itemize}
\item But generates \textbf{1 Story Beat} for the party.
\item The GM describes how this creates story consequences.
\item Useful for emergencies but costly.
\end{itemize}

\section{Upkeep and Maintenance}
\index{upkeep}

Both assets and followers require regular maintenance.

\subsection*{Asset Upkeep}
Two options per SRD §21.2:
\begin{itemize}
\item \textbf{Option 1 - Efficient} (Higher XP, Less Time): 
      \begin{itemize}
      \item Cost: Pay Upkeep XP = $\max(1, \text{Acquisition XP})/3$
      \item Time: Minimal effort
      \end{itemize}
\item \textbf{Option 2 - Intensive} (Lower XP, More Time):
      \begin{itemize}
      \item Cost: Pay 1 XP
      \item Time: Dedicated downtime action with significant personal involvement
      \end{itemize}
\item \textbf{Failure to Pay}: Asset becomes \emph{Neglected} (or \emph{Compromised} if already Neglected)
\end{itemize}

\subsection*{Follower Upkeep}
Two options per SRD §21.2:
\begin{itemize}
\item \textbf{Option 1 - Efficient}: 
      \begin{itemize}
      \item Cost: Pay Upkeep XP = $\max(1, \text{Cap}^2)/3$ 
      \item Time: Minimal effort
      \end{itemize}
\item \textbf{Option 2 - Intensive}:
      \begin{itemize}
      \item Cost: Pay 1 XP
      \item Time: Dedicated downtime action with significant personal involvement
      \end{itemize}
\item \textbf{Failure to Pay}: Follower becomes \emph{Wary} (or \emph{Seized} if already Wary)
\end{itemize}

\section{Strategic Considerations}
\index{strategic considerations}

\subsection*{When to Invest in Assets}
\begin{itemize}
\item You need reliable off-screen capabilities.
\item Your character concept involves wealth or influence.
\item The party lacks certain logistical support.
\item You want to build long-term influence.
\end{itemize}

\subsection*{When to Invest in Followers}
\begin{itemize}
\item You need on-screen assistance.
\item Your character works better with support.
\item The party needs specific capabilities you lack.
\item You want character-driven story opportunities.
\end{itemize}

\subsection*{Balance Recommendations}
\begin{itemize}
\item \textbf{Personal Path}: 0--10\% assets/followers.
\item \textbf{Balanced Path}: 15--25\% assets/followers.
\item \textbf{Influencer Path}: 35--55\% assets/followers.
\end{itemize}

\section{Loyalty and Relationships}
\index{loyalty}\index{relationships}

\subsection*{Loyalty Levels}
Optional system for tracking follower loyalty:
\begin{description}
\item[Wary] Cautious, may leave if pressured; +1 XP upkeep cost.
\item[Steady] Reliable, standard performance; normal upkeep.
\item[Devoted] Loyal, may sacrifice; can convert one major complication to a minor setback per arc.
\end{description}

\subsection*{Building Loyalty}
\begin{itemize}
\item Fair treatment and respect.
\item Sharing rewards and successes.
\item Protecting followers from harm.
\item Honoring agreements and promises.
\end{itemize}

\subsection*{Losing Loyalty}
\begin{itemize}
\item Mistreatment or disrespect.
\item Unreasonable demands or risks.
\item Broken promises or betrayal.
\item Consistent neglect.
\end{itemize}

\section{Advanced Follower Management}
\index{followers!advanced}

\subsection*{Follower Groups}
For multiple similar followers, you can manage them as a group:
\begin{itemize}
\item \textbf{Single Rating}: Treat as one entity with combined capability.
\item \textbf{Condition Tracking}: Group shares exposure and harm.
\item \textbf{Maintenance}: Single upkeep cost for the group.
\item \textbf{Risks}: Problems affect the entire group.
\end{itemize}

\subsection*{Follower Advancement}
Followers can improve over time:
\begin{itemize}
\item \textbf{Experience}: Gain capability through successful assistance.
\item \textbf{Training}: Spend XP to improve follower capabilities.
\item \textbf{Equipment}: Better gear can enhance effectiveness.
\item \textbf{Limits}: Followers typically cap at lower levels than PCs.
\end{itemize}

\section{Risk Management}
\index{risk management}

\subsection*{Asset Risks}
\begin{itemize}
\item \textbf{Financial}: Assets can be costly to maintain.
\item \textbf{Security}: Assets can be attacked or stolen.
\item \textbf{Attention}: Valuable assets draw unwanted notice.
\item \textbf{Dependency}: Over-reliance can be problematic.
\end{itemize}

\subsection*{Follower Risks}
\begin{itemize}
\item \textbf{Safety}: Followers can be harmed or captured.
\item \textbf{Loyalty}: Followers may betray or leave.
\item \textbf{Attention}: Followers can draw enemy interest.
\item \textbf{Morale}: Followers have needs and limits.
\end{itemize}

\subsection*{Mitigation Strategies}
\begin{itemize}
\item \textbf{Diversification}: Don't put all resources in one place.
\item \textbf{Security}: Protect valuable assets and followers.
\item \textbf{Relationships}: Maintain good terms with your people.
\item \textbf{Contingencies}: Have backup plans for losses.
\end{itemize}

\begin{tcolorbox}[colback=blue!5!white,colframe=blue!75!black,title=Assets and Followers Quick Reference,fonttitle=\bfseries]
\textbf{Assets:}
\begin{itemize}
\item Minor: 4 XP \;|\; Standard: 8 XP \;|\; Major: 12 XP
\item Free off-screen use: once per session
\item Boon activation: spend 1 Boon for scene impact
\item Conditions: \emph{Maintained} $\rightarrow$ \emph{Neglected} $\rightarrow$ \emph{Compromised}
\end{itemize}

\textbf{Followers:}
\begin{itemize}
\item Cost: \emph{Cap}$^2$ XP
\item Assistance: $+\min(\text{Cap},\text{skill})$ dice (max +3 from all sources)
\item Independent action: once per scene (party-wide)
\item Conditions: \emph{Exposure} and \emph{Harm} tracks
\end{itemize}

\textbf{Upkeep Options:}
\begin{itemize}
\item Efficient: $\max(1,\text{Cost})/3$ XP, minimal time
\item Intensive: 1 XP, dedicated downtime action
\end{itemize}
\end{tcolorbox}

\section{Practical Examples}

\subsection*{Asset Example: The Safehouse}
\begin{itemize}
\item \textbf{Type}: Minor Asset (4 XP)
\item \textbf{Free Use}: Secure lodging, basic supplies between adventures.
\item \textbf{Boon Activation}: Reveal a hidden escape route during pursuit.
\item \textbf{Upkeep}: Option 1: 2 XP (4/3 rounded up) or Option 2: 1 XP + downtime action.
\item \textbf{Risks}: Discovery by enemies, maintenance costs.
\end{itemize}

\subsection*{Follower Example: The Scout}
\begin{itemize}
\item \textbf{Capability}: 3 (9 XP cost)
\item \textbf{Assistance}: +3 dice on tracking and survival rolls.
\item \textbf{Independent Action}: Scout ahead to improve party position.
\item \textbf{Upkeep}: Option 1: 3 XP (9/3) or Option 2: 1 XP + downtime action.
\item \textbf{Risks}: Injury in dangerous scouting; disloyalty if mistreated.
\end{itemize}

\subsection*{Combination Example: The Merchant}
\begin{itemize}
\item \textbf{Assets}: Trading post (8 XP), caravan (4 XP) --- \emph{12 XP total}
\item \textbf{Followers}: Cap 2 guards (4 XP each = 8 XP), Cap 3 factor (9 XP) --- \emph{17 XP total}
\item \textbf{Total Investment}: \textbf{29 XP} in assets and followers
\item \textbf{Upkeep (Efficient Option)}: Assets 4 XP + Followers 6 XP = \textbf{10 XP} per downtime period
\item \textbf{Benefits}: Trade income, transport, protection, business contacts
\item \textbf{Risks}: Competition, bandit attacks, employee issues, regulatory attention
\end{itemize}

Remember: Assets and followers can greatly expand your capabilities, but they require careful management and carry significant risks. Invest wisely based on your character concept and the needs of your group. The SRD provides flexible upkeep options to suit different play styles and campaign pacing.

\section{Narrative-Heavy Asset and Follower Options}

For groups that prefer strong narrative focus in asset and follower management, consider these optional approaches:

\textbf{Story-Driven Upkeep:} Instead of tracking XP costs for upkeep, the GM can introduce narrative complications that require attention. A neglected asset might attract unwanted attention, while a neglected follower might request a favor or special treatment.

\textbf{Collaborative Management:} Players can describe how they maintain their assets and followers through roleplay rather than mechanical upkeep costs. A well-described scene of tending to a workshop or bonding with followers can fulfill maintenance requirements.

\textbf{Asset and Follower as Character Development:} Use asset and follower management as opportunities for character growth and backstory development, allowing players to narrate how their relationships and holdings evolve through significant story moments.

\textbf{Flexible Condition Tracking:} Focus on the narrative implications of asset and follower conditions rather than strict mechanical penalties. A "Neglected" asset might still function but with interesting complications, while a "Compromised" asset might require creative solutions rather than just XP investment.

\section{Terrestrial Patrons}

Not all power comes from cosmic entities or divine forces. Sometimes your greatest allies—and most demanding creditors—are mortal factions who can offer protection, resources, and influence in the world of men.

\subsection{What Are Terrestrial Patrons?}

Terrestrial Patrons represent your ongoing relationships with powerful mortal organizations:
\begin{itemize}
    \item Noble houses and courtly factions
    \item Merchant guilds and trading consortiums
    \item Military units and mercenary companies
    \item Criminal organizations and smuggling rings
    \item Religious orders and temple hierarchies
    \item Scholarly colleges and arcane societies
\end{itemize}

Unlike supernatural Patrons, Terrestrial Patrons don't grant magic, but they offer leverage: protection, resources, sanctuary, information, and political influence.

\subsection{Gaining a Terrestrial Patron}

To gain a Terrestrial Patron, complete one of these significant actions:
\begin{itemize}
    \item Complete a major job or service for them
    \item Swear a formal Oath of service
    \item Enter into legal or financial binding
    \item Share criminal secrets or be compromised by them
    \item Perform a notable service that earns their favor
\end{itemize}

Mark the Patron on your character sheet and write one sentence: "They want me because \_\_\_\_\_\_\_\_\_\_\_\_\_\_\_\_\_\_"

\subsection{Obligation System}

Terrestrial Patrons use the same Obligation track as supernatural Patrons, but consequences are social, legal, or economic rather than mystical.

\begin{itemize}
    \item \textbf{Calling in Favors:} When you request Patron influence, add +1 Obligation
    \item \textbf{Capacity:} Spirit + Presence (same as supernatural Patrons)
    \item \textbf{Overflow:} Each segment above capacity inflicts 1 Fatigue
    \item \textbf{Resolution:} Reduce through service, downtime actions, or fulfilling demands
\end{itemize}

\subsection{Patron Perks}

Each Terrestrial Patron offers 2--3 repeatable benefits that require no rolls to access:

\textbf{Common Perks Include:}
\begin{itemize}
    \item Sanctuary in their territory or establishments
    \item Legal relief or protection from certain authorities
    \item Access to black market goods or specialized services
    \item Elite followers or hirelings at reduced cost
    \item Forged documents or identity protection
    \item Military backing or private security
    \item Information networks and rumor access
    \item Trade advantages or exclusive contracts
\end{itemize}

Using a perk never requires a roll—Fate has already been paid. Each use comes with the ongoing cost of Obligation.

\subsection{Patron Demands}

Terrestrial Patrons always want something in return:
\begin{itemize}
    \item Silence about sensitive information
    \item Loyalty over other conflicting interests
    \item Completion of specific jobs or tasks
    \item Delivery of valuable names or secrets
    \item Political or social support for their causes
    \item Elimination of rivals or threats to their interests
\end{itemize}

\textbf{Refusing a Demand:} Raises Obligation by 1 segment and may trigger immediate complications.

\subsection{When Obligation Fills}

At 6 Obligation segments, the Patron acts—this is not optional. Choose one:

\begin{enumerate}
    \item \textbf{Unavoidable Job:} You must complete a task you cannot refuse
    \item \textbf{Severe Price:} Pay significant cost (legal, social, or material)
    \item \textbf{First Strike:} They act against you—reputation damage, warrants, bounty, blackmail
\end{enumerate}

After the consequence lands, reduce Obligation to 3 segments.

\subsection{Cutting Ties}

You may sever a Terrestrial Patron relationship, but doing so has serious fallout:
\begin{itemize}
    \item Lose all current perks immediately
    \item Gain a new Rival faction that opposes your interests
    \item Take a lasting Complication (Curse, Bounty, or Scandal) that follows you
    \item May trigger immediate retaliation depending on the Patron's nature
\end{itemize}

Some Patrons never forgive betrayal, while others can be bought off with significant compensation.

\subsection{Redemption and Favor}

If you perform a monumental service for a Patron—something beyond what was asked—reduce Obligation by 2 and gain a permanent Favor:

\textbf{Favor Benefits:}
\begin{itemize}
    \item Noble title or official position
    \item Land grant or property rights
    \item Permanent access to exclusive resources
    \item Unique asset or specialized equipment
    \item Political influence or legal immunity
    \item Protection from their rivals or enemies
\end{itemize}

\subsection{Example: The Black Ledger}

The Black Ledger is a smuggling syndicate that operates in the coastal city of Kahfagia.

\textbf{Gaining Their Favor:} Successfully complete three smuggling runs without detection.

\textbf{Perks:}
\begin{itemize}
    \item Sanctuary in hidden safehouses throughout the city
    \item Access to rare imported goods and contraband
    \item Information about port authorities and customs schedules
\end{itemize}

\textbf{Demands:}
\begin{itemize}
    \item Maintain silence about syndicate operations
    \item Transport occasional "special packages" (may be illegal)
    \item Eliminate rival smugglers who threaten their routes
\end{itemize}

\textbf{Sample Scenario:} Rellan calls on the Ledger for a smuggled border crossing. The GM rules it succeeds automatically, but adds +1 Obligation. Later, the Ledger demands he silence a witness who saw too much. If he refuses, Obligation rises again. If Obligation ever reaches 6, the Ledger collects: accounts frozen, bounty posted, or a rival informant sent after him.

\subsection{Creating Your Own Terrestrial Patrons}

When designing a Terrestrial Patron, consider:

\begin{enumerate}
    \item \textbf{Nature:} What type of organization are they?
    \item \textbf{Motivation:} What drives their goals and actions?
    \item \textbf{Resources:} What can they realistically offer?
    \item \textbf{Methods:} How do they typically operate?
    \item \textbf{Enemies:} Who opposes their interests?
    \item \textbf{Perks:} 2--3 concrete benefits they provide
    \item \textbf{Demands:} 3--5 typical requests they make
\end{enumerate}

Remember: Terrestrial Patrons make the mortal world feel alive and reactive. They represent the complex web of loyalty, debt, and influence that shapes everyday life in Fate's Edge.