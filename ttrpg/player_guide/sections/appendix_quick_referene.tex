\section{Safety and Inclusivity}

\subsection{Content Warnings System}

\subsubsection{X-Card Implementation}

\textbf{Basic Functionality}:
\begin{itemize}
\item \textbf{X-Card}: "I'm not comfortable with this content. Let's change direction."
\item \textbf{O-Card}: "I'm excited about this content. Let's lean into it."
\item \textbf{N-Card}: "I need a break from this content. Let's pause."
\end{itemize}

\textbf{Session Integration}:
\begin{enumerate}
\item Introduce X-Card during Session Zero with practice scenarios
\item Keep X-Cards visible and accessible during all sessions
\item Model appropriate X-Card usage as the GM
\item Address any misuse or abuse of the system immediately
\end{enumerate}

\subsubsection{Script Change Integration}

\textbf{Core Questions}:
\begin{itemize}
\item \textbf{Pause}: "Are you okay with the content right now?"
\item \textbf{Rewind}: "Would you like to go back and change something?"
\item \textbf{Fast Forward}: "Would you like to skip ahead to something else?"
\end{itemize}

\textbf{Application Timing}:
\begin{itemize}
\item Before potentially sensitive content is introduced
\item During scenes that might trigger discomfort
\item After sessions to gather feedback on content experience
\end{itemize}

\subsection{Inclusive Worldbuilding}

\subsubsection{Representation Guidelines}

\textbf{Cultural Sensitivity}:
\begin{itemize}
\item Research real-world cultures thoroughly before incorporating elements
\item Avoid stereotypes and oversimplifications of cultural practices
\item Consult with individuals from relevant backgrounds when possible
\item Create fictional cultures inspired by but distinct from real-world sources
\end{itemize}

\textbf{Diverse Character Creation}:
\begin{itemize}
\item Provide character options that don't rely on real-world cultural markers
\item Include mechanical benefits that reflect diverse backgrounds and experiences
\item Create backstory elements that celebrate different life experiences
\item Avoid mechanics that reinforce harmful stereotypes
\end{itemize}

\subsubsection{Accessibility Considerations}

\textbf{Physical Accessibility}:
\begin{itemize}
\item Provide digital versions of all materials for screen readers
\item Use high-contrast colors and clear fonts in visual materials
\item Offer alternative formats for dice-based mechanics
\item Minimize requirements for fine motor skills in gameplay
\end{itemize}

\textbf{Cognitive Accessibility (continued)}:
\begin{itemize}
\item Offer simplified versions of complex mechanics
\item Break down rules into step-by-step processes
\item Provide examples for abstract concepts
\item Create quick-reference charts for common actions
\end{itemize}

\subsection{Conflict Resolution}

\subsubsection{Preventive Measures}

\textbf{Clear Communication Protocols}:
\begin{itemize}
\item Establish regular check-ins for player satisfaction
\item Create anonymous feedback channels for concerns
\item Define GM authority boundaries and player agency limits
\item Set expectations for group decision-making processes
\end{itemize}

\textbf{Session Structure for Harmony}:
\begin{itemize}
\item Begin sessions with brief mood check-ins
\item Include structured time for character interaction
\item Balance individual spotlight time with group activities
\item End sessions with reflection on positive moments
\end{itemize}

\subsubsection{Active Mediation Techniques}

\textbf{When Conflicts Arise}:
\begin{enumerate}
\item \textbf{Immediate Pause}: Stop gameplay to address the issue
\item \textbf{Private Discussion}: Separate conflicting parties if necessary
\item \textbf{Neutral Facilitation}: Focus on understanding rather than determining fault
\item \textbf{Collaborative Solution}: Work together to find mutually acceptable resolutions
\item \textbf{Documentation}: Record agreements to prevent future misunderstandings
\end{enumerate}

\textbf{Common Conflict Types and Solutions}:
\begin{itemize}
\item \textbf{Rules Disputes}: Establish GM ruling as final for session, research and discuss later
\item \textbf{Character Interference}: Create clear boundaries and consequences for unwanted intrusions
\item \textbf{Play Style Mismatches}: Find compromise activities that satisfy different preferences
\item \textbf{Narrative Control Conflicts}: Establish clear division between player agency and GM narrative authority
\end{itemize}


\chapter{Introduction: The Weight of Choice}
\index{introduction}
\index{GM philosophy}

Welcome, Game Master. You hold a unique role in \textbf{Fate's Edge}. You are not a storyteller in solitude, nor a neutral referee. You are the \textbf{weaver of consequences}\index{weaver of consequences}, the \textbf{architect of a living world}, and the \textbf{guide on a path where every choice echoes}. Your task is to breathe life into a realm of ancient magic, fallen empires, and stubborn, vibrant cultures—and then to let that world truly respond to the players' ambitions.

This is a game where power demands a price, where the past never truly sleeps, and where a single decision can reshape a nation or end an age. From the marble forums of Ecktoria to the mist-drenched fens of the Mistlands, the world is alive with stories waiting to be told. Your job is to provide the stage, set the stakes, and embrace the beautiful, chaotic ripple effects of player agency.

\section*{A World Alive with Consequences}

In \textbf{Fate's Edge}, the fiction is the final authority. The rules in this book are not chains to bind your imagination, but \textbf{tools to give weight to your stories}. They provide a consistent framework for adjudicating risk, tracking progress, and ensuring that success and failure both drive the narrative forward in compelling ways.

Think of yourself as a conductor. The players provide the melody with their characters' actions and ambitions. You provide the harmony and rhythm with the world's response. The rules are your sheet music—a guide to creating a cohesive, dramatic piece, but one that allows for improvisation and adaptation.

\textbf{Your judgment is the cornerstone of the game.} If a rule doesn't serve the moment, change it. If a player's creative idea deserves to succeed, find a way to make it work. The ultimate goal is a collaborative, engaging story that everyone at the table helps to create.

\section*{The Core Philosophy: Narrative First}\index{narrative primacy}

At the heart of \textbf{Fate's Edge} is a simple, powerful idea: \textbf{mechanics serve the story}\index{mechanics serve the story}. A dice roll is never just a binary pass/fail check. It is an event that changes the fictional landscape.

\begin{itemize}
    \item A \textbf{Clean Success} means the plan works as intended—the guard is bribed, the lock clicks open, the argument sways the crowd.
    \item A \textbf{Success with Cost} means you get what you want, but the world pushes back—the guard takes the bribe but becomes a future liability, the lock opens but the mechanism is damaged, the crowd is swayed but a rival noble takes note.
    \item A \textbf{Partial} means you're faced with a difficult choice—you can open the lock but it will take time and risk discovery, or you can sway part of the crowd but alienate another faction.
    \item A \textbf{Complication} means the situation changes dramatically—a new threat appears, a hidden factor is revealed, the stakes are raised.
\end{itemize}

This approach ensures that every roll matters. The story never stalls; it evolves.

\section*{Risk is the Engine of Drama}\index{risk as drama}

\textbf{Fate's Edge} is built on the principle that \textbf{meaningful risk creates compelling drama}\index{risk drives drama}. Safety is boring. It is when characters have something to lose—their reputation, their allies, their ideals, their lives—that their actions become truly heroic or tragically memorable.

Your primary tool for managing this risk is the \textbf{Story Beat (SB)}\index{Story Beats (SB)} economy. When the dice show a 1, it's not merely a failure; it's the world reacting. The GM gains SB to introduce complications, escalate existing threats, or reveal hidden dangers. SB are not punishments; they are the fuel for an unpredictable, responsive narrative.

A successful sword swing might defeat an opponent, but a Story Beat spent could mean the blade is notched and less effective next time, or that the defeat draws the attention of a more powerful foe. The drama continues.

\section*{Characters Who Change the World}

Character growth in \textbf{Fate's Edge} is not about accumulating abstract power. It is about \textbf{meaningful growth}\index{meaningful growth} rooted in the story. Players earn \textbf{Experience Points (XP)}\index{Experience Points (XP)} by engaging with the world's challenges and complexities. They spend XP to improve their capabilities, acquire assets like a ship or a spy network, or unlock unique cultural talents.

This means character advancement is directly tied to the narrative. A character becomes a legendary commander by leading armies, not by killing monsters in a vacuum. They become a master wizard by uncovering forbidden lore and surviving the backlash, not by memorizing spells from a textbook. As the GM, you are the curator of this growth, presenting challenges that allow characters to evolve in ways that feel earned and impactful.

\section*{Your Toolkit}

To help you guide the story, \textbf{Fate's Edge} provides a set of elegant, interconnected tools:

\begin{itemize}
    \item \textbf{The Dice Pool}: The core mechanic. Players roll a number of d10s equal to an Attribute + a Skill. The highest single die determines the degree of success, while any 1s generated provide Story Beats (SB) to the GM.
    \item \textbf{Position & Effect}: Before a roll, you set the character's \textbf{Position} (Dominant, Controlled, or Desperate), which defines the stakes of failure, and their \textbf{Effect}, which describes what a clean success will achieve.
    \item \textbf{Clocks}: Visual trackers for ongoing challenges. A 4-segment clock might represent picking a complex lock, while an 8-segment \textbf{Campaign Clock} could track the rise of a villainous faction.
    \item \textbf{The Deck of Consequences}: A standard 52-card deck used to generate inspired, thematic complications when SB are spent. The suit determines the nature of the complication (Social, Physical, etc.), adding a layer of fortune and flavor.
\end{itemize}

These tools are designed to be learned quickly and used intuitively, getting out of the way so you and your players can focus on the story.

\section*{How to Use This Book}

This book is your guide to running the game.
\begin{itemize}
    \item \textbf{Chapters 1-3} cover the core principles and basic procedures.
    \item \textbf{Chapters 4-6} delve into advanced systems for conflict, travel, and long-term play.
    \item \textbf{Chapters 7-9} provide guidance for high-tier campaigns, world-building, and the specific setting of the Amaranthine Sea region.
    \item \textbf{Chapters 10-11} offer practical advice for running scenarios and a comprehensive appendix of tools and tables.
\end{itemize}

You don't need to memorize everything. Use this book as a reference. Return to it when you need clarification or inspiration. The most important chapters to internalize are those on the core philosophy (this chapter) and the basic action resolution (Chapter 2).

\begin{tcolorbox}[enhanced, sharp corners, boxrule=1pt, colback=gray!5!white, colframe=gray!75!black, title={Flavor is Free}]
\textbf{Players and GMs:} Remember that in \textbf{Fate's Edge}, \textbf{flavor is free}\index{flavor is free}!

This means you can add descriptive details, cultural elements, and atmospheric touches to your actions without spending resources or requiring a dice roll. Want your Vhasian duelist to parry with a flourish taught in the royal fencing schools? Go ahead! Want to describe the eerie silence of a Valewood ruin when searching for clues? Perfect!

Flavor enriches the narrative and makes the world feel real and lived-in. It doesn't change the mechanical outcome, but it defines the \emph{how} and the \emph{why}. The GM should encourage this and reciprocate by painting vivid pictures of the world.

Mechanics determine success or failure, but flavor determines the story we tell about it.
\end{tcolorbox}

\begin{tcolorbox}[enhanced, sharp corners, boxrule=1pt, colback=blue!5!white, colframe=blue!75!black, title={A Guide for Veterans: Fate's Edge in a Nutshell}]
\textbf{If you're experienced with other RPGs,} here's a quick translation guide for how \textbf{Fate's Edge} handles common concepts:

\begin{center}
\begin{tabular}{|p{6cm}|p{6cm}|}
\hline
\textbf{Traditional RPG Concept} & \textbf{Fate's Edge Approach} \\
\hline
Ability Scores \& Skills & \textbf{Attributes} (Body, Wits, etc.) + \textbf{Skills} (Melee, Lore, etc.) form a dice pool. \\
\hline
Skill Checks & Roll Attribute+Skill dice pool. Highest die vs. Difficulty Value (DV). Any 1s give the GM Story Beats (SB). \\
\hline
Hit Points / Health & \textbf{Harm Track} for injuries. \textbf{Fatigue} for exhaustion. Consequences are narrative and mechanical. \\
\hline
Combat Rounds & Fiction-first. Actions are resolved based on narrative timing, not rigid initiative. \\
\hline
Spell Slots / Mana & Magic uses the same core system. Powerful spells may require extra time, resources, or risk generating more SB. \\
\hline
Saving Throws & Roll an appropriate Attribute+Skill combo to resist an effect (e.g., Body+Resolve to resist poison). \\
\hline
Experience \& Leveling Up & Gain XP through play. Spend XP to increase Attributes/Skills, acquire Talents, or buy Assets. Growth is player-directed. \\
\hline
\end{tabular}
\end{center}

The key difference is a consistent, unified mechanic applied across all types of challenges, focused on narrative outcomes.
\end{tcolorbox}

\section*{Begin the Journey}

Your role is a privilege and a creative challenge. You are a facilitator, a fan of the player characters, and the keeper of a world that will challenge and surprise them. Trust the rules to handle the tension, trust your players to drive the story, and trust yourself to weave it all together.

Now, take a deep breath. Shuffle the deck. Let the dice fall where they may.

It's time to guide the edge of fate.

