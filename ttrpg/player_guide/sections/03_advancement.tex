% !TEX root = ../fates_edge_players_guide.tex

\chapter{Character Advancement}
\label{ch:advancement}

\begin{multicols}{2}

In this game, growth isn't just about numbers—it's about defining who your character becomes. Advancement through \textbf{Experience Points (XP)}\index{Experience Points} lets you shape your capabilities, influence, and legacy in the world. Every choice you make with XP is a statement about your character’s priorities and the mark they leave behind.

\section{Earning Experience Points}
\index{experience points!earning}

XP represents learning through action. You earn it by engaging meaningfully with the world and its challenges, whether that’s by triumph, failure, or bold experimentation.

\subsection*{Session Breakdown}
At the end of each session, the Game Master awards XP based on:

\begin{itemize}
\item \textbf{Base Participation}\index{Experience Points!participation}: +2 XP for attending and contributing
\item \textbf{Major Objectives}\index{Experience Points!objectives}: +2–4 XP for completing significant story goals
\item \textbf{Discoveries}\index{Experience Points!discoveries}: +1–2 XP for uncovering important lore, locations, or secrets
\item \textbf{Difficult Choices}\index{Experience Points!choices}: +1–2 XP for making hard moral or strategic decisions
\item \textbf{Story Engagement}\index{Experience Points!engagement}: +1–3 XP for embracing complications and narrative twists
\item \textbf{Personal Goals}\index{Experience Points!personal goals}: +1–2 XP for pursuing your character's individual storylines
\end{itemize}

\paragraph{Example:}  
At the end of a session, the party rogue earns +2 XP for participation, +2 XP for helping the group retrieve an artifact, and +1 XP for pushing a personal rivalry subplot—5 XP total.

\subsection*{Game Pace Options}
The GM can adjust advancement speed to match the campaign tone:

\begin{center}
\small
\begin{tabular}{lcc}
\toprule
\textbf{Mode} & \textbf{XP/Session} & \textbf{Tone} \\
\midrule
Gritty & 4–6 XP & Hard choices, slow growth \\
Standard & 6–10 XP & Balanced progression \\
Epic & 10–14 XP & Heroic, rapid development \\
\bottomrule
\end{tabular}
\end{center}

\paragraph{Player Tip:} If you want a sweeping, mythic tale, suggest an Epic pace. For a long, hard road where each gain feels hard-earned, lean into Gritty.

\subsection*{Arc Completion Bonus}
When you finish a major story arc (typically 3–6 sessions), everyone receives +8–12 XP. One player may earn an additional +2 XP for a particularly memorable contribution. This celebrates the story’s milestones, not just individual rolls.

\section{Spending Experience Points}
\index{experience points!spending}

XP is your currency for growth. You can invest it in three broad areas, each representing a different approach to becoming more capable.

\subsection*{1. Personal Improvement}
\index{advancement!personal improvement}

Invest in your core capabilities—what you can do yourself.

\paragraph{Attributes} Cost = New Rating × 3 XP \index{Attributes!advancement}
\begin{itemize}
\item Raising Body from 2 to 3 costs 3 × 3 = 9 XP
\item Raising Spirit from 4 to 5 costs 5 × 3 = 15 XP
\item Requires downtime equal to new rating in days
\end{itemize}

\paragraph{Skills} Cost = New Level × 2 XP \index{Skills!advancement}
\begin{itemize}
\item Improving Lore from 1 to 2 costs 2 × 2 = 4 XP
\item Advancing Melee from 3 to 4 costs 4 × 2 = 8 XP
\item Requires downtime equal to new level in days
\end{itemize}

\paragraph{Example:}  
Kara wants to improve her Swordsmanship from 2 to 3. She saves 6 XP and spends three in-game days training with her mentor. This creates roleplay hooks and a sense of lived growth.

\subsection*{2. Resources and Influence}
\index{advancement!resources}

Build your worldly presence—what you can command.

\paragraph{Minor Resource} (4 XP, 1 week)
\begin{itemize}
\item Small shop, minor contact network, basic workshop
\item Provides small but reliable benefits
\item Example: A trusted informant who gathers rumors
\end{itemize}

\paragraph{Standard Resource} (8 XP, 2 weeks)
\begin{itemize}
\item Decent-sized business, skilled followers, specialized equipment
\item Significant benefits with some upkeep
\item Example: A smuggling operation with two boats
\end{itemize}

\paragraph{Major Resource} (12 XP, 1 month)
\begin{itemize}
\item Large enterprise, elite team, rare capabilities
\item Powerful advantages with substantial upkeep
\item Example: A trading company with international contacts
\end{itemize}

\paragraph{Player Tip:} Resources expand the story into new directions. A spy network creates intrigue; a workshop sparks invention; a guild hall cements influence.

\subsection*{3. Special Abilities}
\index{advancement!abilities}

Develop unique capabilities that set you apart.

\paragraph{General Abilities} (Cost varies) \index{Abilities!general}
\begin{itemize}
\item Universal benefits like improved recovery, bonus dice in specific situations, or unique combat techniques
\item Typically cost 4–8 XP
\item Example: "Quick Recovery" - heal 1 additional Harm when resting
\end{itemize}

\paragraph{Cultural Abilities} (Cost varies) \index{Abilities!cultural}
\begin{itemize}
\item Heritage-based skills tied to your character's background
\item Often require specific fictional positioning
\item Example: "Stone Sense" (dwarven) - intuitive understanding of stonework
\end{itemize}

\paragraph{Advanced Abilities} (12+ XP) \index{Abilities!advanced}
\begin{itemize}
\item Powerful capstone features available at higher tiers
\item Often have significant narrative weight and requirements
\item Example: "Master Diplomat" - can reroll failed social checks once per session
\end{itemize}

\paragraph{Example:}  
A veteran bard invests in “Silver Tongue” (6 XP), allowing them to sway hostile crowds once per session. This becomes their defining trick in tense negotiations.

\section{Character Development Paths}
\index{development paths}

Your spending choices define your character's growth direction. Consider these archetypal paths:

\subsection*{The Specialist}
70–90\% personal improvement, 0–10\% resources, 0–20\% abilities
\begin{itemize}
\item \textbf{Strengths}: Exceptional individual capability, reliable in spotlight moments
\item \textbf{Weaknesses}: Limited influence, vulnerable to being isolated
\item \textbf{Best for}: Solo operatives, elite warriors, master artisans
\item \textbf{Example}: A duelist who invests heavily in combat skills and physical attributes
\end{itemize}

\subsection*{The Leader}
50–65\% personal, 15–25\% resources, 15–25\% abilities
\begin{itemize}
\item \textbf{Strengths}: Well-rounded, can handle diverse challenges, good support
\item \textbf{Weaknesses}: Jack-of-all-trades, not exceptional in any area
\item \textbf{Best for}: Party faces, field commanders, investigators
\item \textbf{Example}: A merchant-prince with decent combat skills, good social abilities, and a network of contacts
\end{itemize}

\subsection*{The Mastermind}
25–40\% personal, 35–55\% resources, 20–40\% abilities
\begin{itemize}
\item \textbf{Strengths}: Extensive influence, can solve problems indirectly, strategic power
\item \textbf{Weaknesses}: Personally vulnerable, complex upkeep, domino-effect risks
\item \textbf{Best for}: Spymasters, crime lords, wealthy patrons
\item \textbf{Example}: An information broker with modest personal skills but an extensive spy network
\end{itemize}

\paragraph{Player Note:} These are not rigid templates. Mix and match to discover unique growth arcs.

\section{Training and Development Time}
\index{training time}

Most improvements require downtime to reflect the effort of learning and integration.

\subsection*{Standard Time Requirements}
\begin{itemize}
\item \textbf{Attribute increase}: New rating in days
\item \textbf{Skill improvement}: New level in days
\item \textbf{Resource acquisition}: 1 week to 1 month depending on scope
\item \textbf{Ability learning}: Typically 3–10 days
\end{itemize}

\subsection*{Accelerated Development}
You can attempt to learn things more quickly, but this carries risks:

\begin{itemize}
\item The GM creates a \textbf{Risk Clock}\index{Risk Clock} with 4 segments
\item If the clock fills during rushed training, the new capability has flaws:
\begin{itemize}
\item Attribute/Skill: -1 die penalty until you spend proper downtime
\item Resource: Loyalty problems or functional limitations
\item Ability: Unreliable or with unintended side effects
\end{itemize}
\end{itemize}

\paragraph{Example:}  
The wizard crams advanced spellwork into a frantic three days. She gains the ability, but her Risk Clock fills—her spells now sputter unpredictably until she retrains.

\section{Character Progression Tiers}
\index{progression tiers}

As you accumulate XP and capabilities, you advance through tiers that represent your growing reputation and influence.

\subsection*{Tier I: Novice (0–40 XP)}
\begin{itemize}
\item Learning the ropes, establishing yourself
\item Local reputation, modest capabilities
\item \textbf{Typical assets}: Basic equipment, a few contacts
\end{itemize}

\subsection*{Tier II: Experienced (41–90 XP)}
\begin{itemize}
\item Proven capability, recognized skills
\item Regional reputation, reliable in your specialty
\item \textbf{Typical assets}: Skilled followers, specialized equipment
\end{itemize}

\subsection*{Tier III: Veteran (91–150 XP)}
\begin{itemize}
\item Master of your craft, significant influence
\item National reputation, can handle major challenges
\item \textbf{Typical assets}: Multiple operations, elite teams
\end{itemize}

\subsection*{Tier IV: Elite (151–220 XP)}
\begin{itemize}
\item Exceptional capability, major influence
\item International reputation, shapes events
\item \textbf{Typical assets}: Organizations, unique capabilities
\end{itemize}

\subsection*{Tier V: Master (221+ XP)}
\begin{itemize}
\item Legendary status, world-changing influence
\item Historical reputation, defines eras
\item \textbf{Typical assets}: Nations, legendary artifacts
\end{itemize}

\section{Managing Allies and Followers}
\index{allies}
\index{followers}

Characters who work with you require maintenance and carry risks.

\subsection*{Acquisition Costs}
\begin{itemize}
\item \textbf{Skilled helper}: Capability rating squared in XP
\item \textbf{Example}: A capability 3 scout costs 9 XP
\end{itemize}

\subsection*{Upkeep Requirements}
\begin{itemize}
\item Each downtime period, spend XP equal to their capability rating
\item Alternative: Dedicate a scene to maintaining the relationship
\end{itemize}

\subsection*{Risk Management}
\begin{itemize}
\item When the GM spends 2+ Complication Points, allies may face consequences instead of you
\item Allies can solve problems off-screen once per downtime, but this generates complications
\end{itemize}

\section{Strategic Advancement Considerations}
\index{advancement!strategy}

\subsection*{Early Game (Tiers I–II)}
Focus on survival and establishing your niche:
\begin{itemize}
\item Invest in core competencies first
\item Build a small but reliable support network
\end{itemize}

\subsection*{Mid Game (Tier III)}
Expand your influence and specialize:
\begin{itemize}
\item Develop your signature capabilities
\item Build substantial resources
\end{itemize}

\subsection*{Late Game (Tiers IV–V)}
Shape the world around you:
\begin{itemize}
\item Pursue advanced abilities
\item Build organizations or movements
\item Leave a legacy
\end{itemize}

\section{Advancement Philosophy}
\index{advancement!philosophy}

Remember that advancement serves the story. The best choices:
\begin{itemize}
\item Reflect your character's experiences and growth
\item Create interesting new capabilities and complications
\item Enhance the group's collective abilities
\end{itemize}

\paragraph{Final Thought:}  
Every XP spent changes not just your character sheet, but your character's story. Choose investments that make your hero more interesting to play and watch evolve.

\begin{tcolorbox}[colback=green!5!white,colframe=green!75!black,title=XP Planning Guide,fonttitle=\bfseries]
\textbf{Early Tier Priorities:}
\begin{itemize}
\item Core attribute to 3 (9 XP)
\item Key skills to 2–3 (4–8 XP each)
\item 1–2 minor resources (8 XP total)
\end{itemize}

\textbf{Mid Tier Expansion:}
\begin{itemize}
\item Attributes to 4 (12 XP)
\item Specialization skills to 4 (8 XP)
\item Standard resources (8 XP each)
\item Cultural abilities (6–10 XP)
\end{itemize}

\textbf{Late Tier Mastery:}
\begin{itemize}
\item Capstone abilities (12+ XP)
\item Major resources (12 XP)
\item Legacy projects
\end{itemize}
\end{tcolorbox}

\end{multicols}
\end{chapter}
