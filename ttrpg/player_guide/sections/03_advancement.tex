\chapter{Character Advancement and Capabilities}
\label{chap:advancement}

In this game, growth is not just about numbers---it is about deciding who your
character becomes. Advancement through \textbf{Experience Points (XP)} lets you
shape your capabilities, influence, and legacy in the world. Every XP choice is
a statement about your character's priorities and the mark they leave behind.

Your capabilities are built on four core \textbf{Attributes} and a set of
specialized \textbf{Skills}. This chapter explains how they work together,
what advancement looks like in play, and offers player-facing tips for making
satisfying choices.

% ============================================================
% CORE ATTRIBUTES
% ============================================================

\section{Core Attributes}
\label{sec:core-attributes}

Attributes represent your character's fundamental capabilities. Each is rated
from 1 to 5; higher numbers mean greater potential in that area.

\subsection{Body}

Physical strength, endurance, coordination, and health.

\begin{itemize}
  \item \textbf{Used for:} melee combat, athletics, endurance tests, physical labor.
  \item \textbf{Typical applications:} lifting, running, climbing, fighting, resisting physical harm.
  \item \textbf{Associated skills:} Athletics, Brawl, Melee, Endurance.
\end{itemize}

\noindent\textbf{Rating guide:}
\begin{itemize}
  \item 1: Average person; some physical activity.
  \item 2: Fit; trains or works physically on a regular basis.
  \item 3: Athlete or soldier; excellent condition.
  \item 4: Exceptional athlete; near peak human.
  \item 5: Peak human capability; legendary strength and stamina.
\end{itemize}

\subsection{Wits}

Mental acuity, perception, quick thinking, and problem-solving.

\begin{itemize}
  \item \textbf{Used for:} investigation, perception, tactics, snap decisions.
  \item \textbf{Typical applications:} spotting details, solving puzzles, planning, reacting quickly.
  \item \textbf{Associated skills:} Perception, Investigation, Tactics, Lore.
\end{itemize}

\noindent\textbf{Rating guide:}
\begin{itemize}
  \item 1: Average awareness; sometimes misses important cues.
  \item 2: Observant; notices most relevant details.
  \item 3: Sharp-minded; quick to spot patterns.
  \item 4: Exceptionally perceptive; rarely surprised.
  \item 5: Almost uncanny awareness; sees connections others miss.
\end{itemize}

\subsection{Spirit}

Willpower, intuition, mental resilience, and connection to intangible forces.

\begin{itemize}
  \item \textbf{Used for:} resisting mental effects, intuition, magical aptitude, persistence.
  \item \textbf{Typical applications:} resisting fear, sensing danger, channeling magic, enduring hardship.
  \item \textbf{Associated skills:} Resolve, Intuition, Magic, Faith.
\end{itemize}

\noindent\textbf{Rating guide:}
\begin{itemize}
  \item 1: Average willpower; somewhat suggestible.
  \item 2: Strong-minded; resists ordinary pressure.
  \item 3: Very determined; hard to intimidate.
  \item 4: Exceptional will; can inspire others.
  \item 5: Iron will; nearly unshakeable resolve.
\end{itemize}

\subsection{Presence}

Charisma, social influence, appearance, and force of personality.

\begin{itemize}
  \item \textbf{Used for:} social interactions, leadership, persuasion, intimidation.
  \item \textbf{Typical applications:} negotiating, leading, charming, commanding attention.
  \item \textbf{Associated skills:} Sway, Command, Performance, Deception.
\end{itemize}

\noindent\textbf{Rating guide:}
\begin{itemize}
  \item 1: Average presence; does not particularly stand out.
  \item 2: Noticeable; leaves a mild impression.
  \item 3: Charismatic; naturally influential.
  \item 4: Commanding; people listen when you speak.
  \item 5: Magnetic; can sway crowds and redefine the mood of a room.
\end{itemize}

% ============================================================
% SKILL SYSTEM
% ============================================================

\section{Skill System}
\label{sec:skills}

Skills represent specialized training and expertise. In play, you combine an
Attribute with a relevant Skill to form your dice pool.

\subsection{Skill Ratings}

\begin{center}
\begin{tabular}{cl}
\toprule
\textbf{Rating} & \textbf{Description} \\
\midrule
0 & Untrained --- no formal experience \\
1 & Novice --- basic understanding \\
2 & Competent --- reliable and capable \\
3 & Professional --- expert in the field \\
4 & Master --- renowned specialist \\
5 & Grand Master --- legendary skill \\
\bottomrule
\end{tabular}
\end{center}

\subsection{Skill Categories}

Below are common skills grouped by theme. Your table may rename or reskin
these, but the roles remain similar.

\paragraph{Combat Skills}
\begin{itemize}
  \item \textbf{Melee}: swords, axes, close-quarters weapons.
  \item \textbf{Ranged}: bows, crossbows, thrown weapons, firearms (if present).
  \item \textbf{Brawl}: unarmed combat, grappling, improvised fighting.
  \item \textbf{Tactics}: battlefield strategy, unit coordination, ambush planning.
\end{itemize}

\paragraph{Physical Skills}
\begin{itemize}
  \item \textbf{Athletics}: running, climbing, jumping, swimming.
  \item \textbf{Stealth}: moving unseen, hiding, blending into shadows.
  \item \textbf{Endurance}: resisting fatigue, harsh weather, poison, disease.
  \item \textbf{Craft}: building, repairing, creating tools, art, or structures.
\end{itemize}

\paragraph{Social Skills}
\begin{itemize}
  \item \textbf{Sway}: persuasion, negotiation, charm.
  \item \textbf{Command}: leadership, intimidation, issuing orders.
  \item \textbf{Deception}: lying, bluffing, misdirection.
  \item \textbf{Performance}: entertainment, oration, acting, musical display.
\end{itemize}

\paragraph{Knowledge Skills}
\begin{itemize}
  \item \textbf{Lore}: history, culture, religions, general knowledge.
  \item \textbf{Investigation}: research, deduction, analysis of clues.
  \item \textbf{Medicine}: healing, anatomy, first aid, treatment.
  \item \textbf{Nature}: wilderness, animals, plants, weather patterns.
\end{itemize}

\paragraph{Specialized Skills}
\begin{itemize}
  \item \textbf{Arcana}: magic theory, rituals, mystical phenomena.
  \item \textbf{Mechanics}: devices, engineering, construction, traps.
  \item \textbf{Diplomacy}: formal negotiation, etiquette, protocol.
  \item \textbf{Streetwise}: urban survival, criminal underworld, gossip networks.
\end{itemize}

% ============================================================
% BUILDING DICE POOLS
% ============================================================

\section{Building Dice Pools}
\label{sec:dice-pools}

Your dice pool for any action is:
\[
\text{Dice Pool} = \text{Attribute} + \text{Skill}.
\]

You roll that many d10s and interpret the results according to the core rules.

\subsection{Choosing the Right Combination}

The same fictional action can often be approached with different Attribute + Skill
pairs, depending on how your character does it.

\paragraph{Climbing a wall}
\begin{itemize}
  \item \textbf{Body + Athletics}: sheer strength and stamina.
  \item \textbf{Wits + Athletics}: clever route-finding and leverage.
  \item \textbf{Spirit + Athletics}: grim determination against fear or pain.
\end{itemize}

\paragraph{Persuading a guard}
\begin{itemize}
  \item \textbf{Presence + Sway}: charm and personality.
  \item \textbf{Wits + Sway}: logical, well-constructed arguments.
  \item \textbf{Spirit + Sway}: conviction that burns through doubt.
\end{itemize}

\paragraph{Investigating a crime scene}
\begin{itemize}
  \item \textbf{Wits + Investigation}: methodical observation and deduction.
  \item \textbf{Spirit + Investigation}: intuitive leaps and gut feelings.
  \item \textbf{Presence + Investigation}: getting witnesses to open up.
\end{itemize}

\subsection{Creative Combinations}

With your GM's agreement, you can justify unusual combinations that fit the
fiction:

\begin{itemize}
  \item \textbf{Body + Lore}: recalling combat drills or physical techniques.
  \item \textbf{Presence + Medicine}: soothing and reassuring a patient.
  \item \textbf{Spirit + Craft}: inspired, visionary craftsmanship.
\end{itemize}

\noindent\textit{Example:} A ranger scales an ice wall using \textbf{Wits + Athletics}
to find the safest holds, then switches to \textbf{Body + Athletics} to muscle over the
lip. The fiction guides which combination fits each moment.

% ============================================================
% EARNING XP
% ============================================================

\section{Earning Experience Points}
\label{sec:earning-xp}

XP represents learning through action. You gain it by engaging meaningfully
with the world: overcoming challenges, taking risks, making hard choices, and
leaning into your character's story.

\subsection{Session Breakdown}

At the end of each session, XP is awarded based on what happened at the table.
A typical breakdown might look like:

\begin{itemize}
  \item \textbf{Base Participation}: +2~XP for attending and contributing.
  \item \textbf{Major Objectives}: +2--4~XP for completing significant story goals.
  \item \textbf{Discoveries}: +1--2~XP for uncovering important lore, locations, or secrets.
  \item \textbf{Difficult Choices}: +1--2~XP for making hard moral or strategic decisions.
  \item \textbf{Story Engagement}: +1--3~XP for embracing complications and twists.
  \item \textbf{Personal Goals}: +1--2~XP for pursuing your character's individual storylines.
\end{itemize}

\noindent\textit{Example:} At the end of a session, the party rogue gains:

\begin{itemize}
  \item +2~XP for participation,
  \item +2~XP for helping retrieve a cursed artifact,
  \item +1~XP for pushing a personal rivalry subplot.
\end{itemize}

\noindent Total: 5~XP.

\subsection{Game Pace Options}

You and your GM can decide how quickly characters advance, matching the tone
you want:

\begin{center}
\begin{tabular}{cll}
\toprule
\textbf{Mode} & \textbf{XP / Session} & \textbf{Tone} \\
\midrule
Gritty   & 4--6  & Hard choices, slow growth \\
Standard & 6--10 & Balanced progression \\
Epic     & 10--14 & Heroic, rapid development \\
\bottomrule
\end{tabular}
\end{center}

\noindent\textbf{Player Tip:} If you want a sweeping, mythic tale, ask for an
\emph{Epic} pace. If you want advancement to feel hard-won, lean toward
\emph{Gritty}.

\subsection{Arc Completion Bonus}

When a major story arc finishes (typically 3--6 sessions), everyone gains an
additional +8--12~XP. One player may receive +2~XP more for a particularly
memorable contribution. This rewards the story as a whole, not just individual
rolls.

% ============================================================
% SPENDING XP
% ============================================================

\section{Spending Experience Points}
\label{sec:spending-xp}

XP is your currency for growth. You can invest it in three broad areas:

\begin{enumerate}
  \item \textbf{Personal Improvement} (Attributes and Skills)
  \item \textbf{Resources and Influence}
  \item \textbf{Special Abilities}
\end{enumerate}

Think of these as three pillars: \textit{what you can do}, \textit{what you own or
command}, and \textit{what makes you uniquely special}.

\subsection{Personal Improvement}

\subsubsection*{Attributes}

Attributes are expensive but powerful. The cost to raise an Attribute is:
\[
\text{XP Cost} = 3 \times \text{new rating}.
\]

\noindent Examples:
\begin{itemize}
  \item Raising Body from 2 to 3 costs $3 \times 3 = 9$ XP.
  \item Raising Spirit from 4 to 5 costs $3 \times 5 = 15$ XP.
\end{itemize}

\noindent Attribute increases usually require downtime equal to the new rating in
days, spent training, reflecting, or transforming yourself.

\subsubsection*{Skills}

Skills are cheaper and represent focused practice. The cost is:
\[
\text{XP Cost} = 2 \times \text{new level}.
\]

\noindent Examples:
\begin{itemize}
  \item Lore 1 $\rightarrow$ 2 costs $2 \times 2 = 4$ XP.
  \item Melee 3 $\rightarrow$ 4 costs $2 \times 4 = 8$ XP.
\end{itemize}

\noindent Skill increases typically require downtime equal to the new level in days.

\medskip

\noindent\textit{Example:} Kara wants to improve her swordsmanship (Melee) from 2 to 3.  
She saves 6~XP and spends three in-game days training with her mentor. This
training becomes part of the story, not just a line on a sheet.

\subsection{Resources and Influence}

Resources represent what you can call on beyond yourself: businesses, safe
houses, crews, labs, or organizations.

\begin{description}
  \item[Minor Resource (4~XP, $\sim$1 week)]  
    A small shop, basic workshop, discreet safe house, or minor contact
    network. Offers modest but reliable benefits.

  \item[Standard Resource (8~XP, $\sim$2 weeks)]  
    A decent-sized business, crew, or operation with real impact. Comes with
    some upkeep and obligations.

  \item[Major Resource (12~XP, $\sim$1 month)]  
    A large enterprise, elite team, or rare capability. Very potent but also
    demanding; often becomes a major part of your character's story.
\end{description}

\noindent\textbf{Player Tip:} Resources pull the camera out from your character
to your \emph{sphere of influence}. A spy network means intrigue. A workshop
means invention. A guild hall means politics.

\subsection{Special Abilities}

Special abilities are unique moves, talents, or tricks that define your style.

\begin{description}
  \item[General Abilities (4--8~XP)]  
    Universal benefits, like improved recovery, bonus dice in a common
    situation, or a signature combat technique.  
    \emph{Example:} ``Quick Recovery'' --- heal 1 extra Harm when you rest.

  \item[Cultural Abilities (variable)]  
    Abilities tied to heritage or background. They may require specific
    fictional positioning.  
    \emph{Example:} ``Stone Sense'' --- an instinctive feel for stonework and tunnels.

  \item[Advanced Abilities (12+~XP)]  
    High-impact features available at higher tiers, often with strong narrative
    hooks or prerequisites.  
    \emph{Example:} ``Master Diplomat'' --- once per session, you can reroll or
    soften the outcome of a failed major negotiation.
\end{description}

\medskip

\noindent\textit{Example:} A veteran bard purchases \emph{Silver Tongue} for 6~XP,
gaining the ability to sway a hostile crowd once per session. This quickly
becomes their trademark in tense public scenes.

% ============================================================
% DEVELOPMENT PATHS
% ============================================================

\section{Character Development Paths}
\label{sec:development-paths}

How you spend XP shapes your character's growth. You do not need to follow a
strict template, but it can help to think in rough patterns.

\subsection{The Specialist}

\begin{itemize}
  \item \textbf{Typical XP split:} 70--90\% personal improvement, 0--10\% resources, 0--20\% abilities.
  \item \textbf{Strengths:} Extremely effective in a narrow field; shines in spotlight moments.
  \item \textbf{Weaknesses:} Limited influence, vulnerable when pulled outside their niche.
  \item \textbf{Best for:} Solo operatives, elite warriors, master artisans.
\end{itemize}

\noindent\textit{Example:} A duelist who pours most XP into Body, Melee, and related
abilities.

\subsection{The Leader}

\begin{itemize}
  \item \textbf{Typical XP split:} 50--65\% personal, 15--25\% resources, 15--25\% abilities.
  \item \textbf{Strengths:} Well-rounded, boosts the whole group, strong support presence.
  \item \textbf{Weaknesses:} Less likely to be the \emph{best} at any single thing.
  \item \textbf{Best for:} Party faces, commanders, investigators, organizers.
\end{itemize}

\noindent\textit{Example:} A merchant-prince with solid skills, strong social abilities,
and a network of trade contacts.

\subsection{The Mastermind}

\begin{itemize}
  \item \textbf{Typical XP split:} 25--40\% personal, 35--55\% resources, 20--40\% abilities.
  \item \textbf{Strengths:} Solves problems indirectly through assets and preparations.
  \item \textbf{Weaknesses:} Personally vulnerable; more moving parts, more risk.
  \item \textbf{Best for:} Spymasters, crime lords, wealthy patrons, power brokers.
\end{itemize}

\noindent\textbf{Player Note:} These patterns are \emph{guides}, not rules. You can
shift between them as your story evolves.

% ============================================================
% TRAINING & TIME
% ============================================================

\section{Training and Development Time}
\label{sec:training-time}

Most improvements take time in the fiction. Training montages, study scenes,
and building projects are all opportunities for character moments.

\subsection{Standard Time Requirements}

\begin{itemize}
  \item \textbf{Attribute increase:} new rating in days.
  \item \textbf{Skill improvement:} new level in days.
  \item \textbf{Resource acquisition:} about 1~week to 1~month depending on scope.
  \item \textbf{Ability learning:} typically 3--10 days.
\end{itemize}

\subsection{Accelerated Development}

You \emph{can} attempt to learn faster, but rushing has a price.

\begin{itemize}
  \item The group tracks a short \textbf{Risk Clock} (4 segments) for your crash training.
  \item If it fills, the new capability has flaws:
  \begin{itemize}
    \item Attribute/Skill: temporary $-1$ die penalty until you retrain properly.
    \item Resource: loyalty issues, partial effectiveness, or hidden problems.
    \item Ability: unreliable effects or quirky side consequences.
  \end{itemize}
\end{itemize}

\noindent\textit{Example:} A wizard crams advanced spellwork into three frantic days.
She gains the ability, but the Risk Clock fills: her spells spark and flicker
until she takes time to train safely.

% ============================================================
% TIERS
% ============================================================

\section{Character Progression Tiers}
\label{sec:tiers}

Tiers are broad bands of advancement that reflect your reputation and impact
on the world. They are descriptive, not hard mechanical gates.

\begin{description}
  \item[Tier I: Novice (0--40~XP)]  
    Learning the ropes and finding your place. Local reputation at best.  
    \textbf{Typical assets:} basic gear, a few contacts.

  \item[Tier II: Experienced (41--90~XP)]  
    Proven and reliable. Regional reputation in your specialty.  
    \textbf{Typical assets:} skilled helpers, specialized equipment.

  \item[Tier III: Veteran (91--150~XP)]  
    A master of your craft with real influence. National or cross-regional reputation.  
    \textbf{Typical assets:} multiple operations, elite teams.

  \item[Tier IV: Elite (151--220~XP)]  
    Exceptional capability and broad influence. You shape events.  
    \textbf{Typical assets:} organizations, unique capabilities.

  \item[Tier V: Master (221+~XP)]  
    Legendary status. Your name can define an era.  
    \textbf{Typical assets:} nations, legendary artifacts, movements.
\end{description}

% ============================================================
% ALLIES & FOLLOWERS
% ============================================================

\section{Managing Allies and Followers}
\label{sec:allies-followers}

Allies can extend your reach but also create obligations and risks.

\subsection{Acquisition Costs}

A skilled helper's cost in XP is roughly their \emph{capability rating squared}.

\noindent\textit{Example:} A capability 3 scout costs $3^2 = 9$~XP.

\subsection{Upkeep Requirements}

Each downtime period, you either:
\begin{itemize}
  \item spend XP equal to their capability rating, \textbf{or}
  \item dedicate a meaningful scene to maintaining the relationship (time, favors, protection).
\end{itemize}

\subsection{Risk Management}

\begin{itemize}
  \item When big complications arise, your allies can be targeted instead of you.
  \item Allies can handle one significant task off-screen per downtime, but this often creates extra story fallout.
\end{itemize}

% ============================================================
% SKILL ADVANCEMENT
% ============================================================

\section{Skill Advancement}
\label{sec:skill-advancement}

\subsection{XP Costs}

\begin{center}
\begin{tabular}{cl}
\toprule
\textbf{Improvement} & \textbf{XP Cost} \\
\midrule
0 $\rightarrow$ 1 & 2 XP \\
1 $\rightarrow$ 2 & 4 XP \\
2 $\rightarrow$ 3 & 6 XP \\
3 $\rightarrow$ 4 & 8 XP \\
4 $\rightarrow$ 5 & 10 XP \\
\bottomrule
\end{tabular}
\end{center}

\subsection{Training Time}

\begin{itemize}
  \item 0 $\rightarrow$ 1: 1 day of practice.
  \item 1 $\rightarrow$ 2: 3 days of training.
  \item 2 $\rightarrow$ 3: 1 week of intensive study.
  \item 3 $\rightarrow$ 4: 2 weeks with a master or rigorous practice.
  \item 4 $\rightarrow$ 5: 1 month of dedicated focus.
\end{itemize}

\subsection{Attribute Limits}

You cannot have a Skill rating higher than the relevant Attribute. To advance
a Skill further, you must first raise its primary Attribute.

% ============================================================
% SKILL SYNERGY
% ============================================================

\section{Synergy Between Skills}
\label{sec:skill-synergy}

Some skill combinations make special sense together. When the fiction supports
it, you may gain a small bonus by deliberately leaning into those synergies.

\subsection{Combat Synergies}

\begin{itemize}
  \item \textbf{Tactics + Command}: coordinating a squad or battle plan.
  \item \textbf{Melee + Athletics}: acrobatic strikes, charges, and movement-based attacks.
  \item \textbf{Ranged + Perception}: careful aim, sniping, and spotting targets.
\end{itemize}

\subsection{Social Synergies}

\begin{itemize}
  \item \textbf{Sway + Lore}: persuasion backed by specific knowledge.
  \item \textbf{Deception + Performance}: long cons, false identities, staged scenes.
  \item \textbf{Command + Presence}: inspiring or intimidating leadership.
\end{itemize}

\subsection{Exploration Synergies}

\begin{itemize}
  \item \textbf{Investigation + Perception}: thorough searches and clue analysis.
  \item \textbf{Nature + Survival}: navigation, tracking, and wilderness foraging.
  \item \textbf{Mechanics + Craft}: complex repairs, modifications, and inventions.
\end{itemize}

% ============================================================
% USING SKILLS IN PLAY
% ============================================================

\section{Using Skills in Play}
\label{sec:using-skills}

\subsection{When to Roll}

You roll when:
\begin{itemize}
  \item the outcome is uncertain,
  \item failure would matter to the story, and
  \item the action is significant enough to deserve the spotlight.
\end{itemize}

For routine actions with no real risk, the group can often simply agree that
you succeed.

\subsection{Difficulty and Skill Level}

As your skill rises, routine tasks become trivial. A rough guideline:

\begin{center}
\begin{tabular}{cll}
\toprule
\textbf{Skill Level} & \textbf{Routine Task} & \textbf{Challenging Task} \\
\midrule
0 & Needs a roll; modest DV & Needs a roll; higher DV \\
1 & Often needs a roll      & Challenging but fair \\
2 & Usually automatic       & Needs a roll; moderate DV \\
3 & Automatic                & Needs a roll; interesting stakes \\
4+ & Automatic               & Automatic for simple cases \\
\bottomrule
\end{tabular}
\end{center}

\noindent\textbf{Reading this as a player:} by Skill 3, basic tasks in your area of
expertise should rarely require rolls. You are there to be impressive, not to
trip over routine jobs.

\subsection{Group Skill Use}

When multiple characters contribute:

\begin{itemize}
  \item \textbf{Assistance:} One character is primary; up to three others add +1 die each by describing how they help.
  \item \textbf{Cooperation:} Several characters each roll; the group succeeds or fails based on combined results.
  \item \textbf{Complementary:} Different skills cover different parts of a larger task (e.g.\ one distracts, one picks a lock).
\end{itemize}

% ============================================================
% SKILL CHALLENGES
% ============================================================

\section{Skill Challenges}
\label{sec:skill-challenges}

Some goals are too big for a single roll.

\subsection{Extended Tests}

For long, involved tasks:

\begin{itemize}
  \item The group sets a progress clock (4--8 segments).
  \item Each significant success fills one or more segments.
  \item Complications can add segments or create new problems.
\end{itemize}

\subsection{Complex Challenges}

For multi-step or multi-skill situations:

\begin{itemize}
  \item Different skills tackle different phases of the challenge.
  \item Success in one area creates opportunities; failure creates obstacles.
\end{itemize}

\noindent\textit{Example (Heist):} \textbf{Stealth} to infiltrate, \textbf{Mechanics} to bypass
locks, \textbf{Investigation} to find the vault, \textbf{Deception} to mislead guards.
Each roll advances a \emph{Heist Clock}; complications add heat, suspicion, or
security changes.

% ============================================================
% STRATEGIC ADVANCEMENT
% ============================================================

\section{Strategic Advancement Considerations}
\label{sec:advancement-strategy}

\subsection{Early Game (Tiers I--II)}

\begin{itemize}
  \item Focus on survival and a clear niche.
  \item Raise a key Attribute to 3 early.
  \item Bring core Skills to 2--3.
  \item Establish one or two modest resources or safe havens.
\end{itemize}

\subsection{Mid Game (Tier III)}

\begin{itemize}
  \item Develop signature moves or combinations.
  \item Expand your resource base.
  \item Consider your role in the group and lean into it.
\end{itemize}

\subsection{Late Game (Tiers IV--V)}

\begin{itemize}
  \item Choose advanced abilities that reshape the story around you.
  \item Build organizations, movements, or legacies.
  \item Decide how your character will leave their mark on the world.
\end{itemize}

% ============================================================
% ARCHETYPES
% ============================================================

\section{Skill-Based Character Archetypes}
\label{sec:archetypes}

These archetypes are starting points---feel free to remix or invent your own.

\subsection{The Warrior}

\begin{itemize}
  \item \textbf{Primary:} Body + Melee or Body + Ranged.
  \item \textbf{Secondary:} Spirit + Endurance, Wits + Tactics.
  \item \textbf{Key skills:} Athletics, Brawl, Command.
  \item \textbf{Playstyle:} Direct confrontation, physical solutions, tactical presence.
\end{itemize}

\subsection{The Expert}

\begin{itemize}
  \item \textbf{Primary:} Wits + Lore/Investigation.
  \item \textbf{Secondary:} Presence + Sway, Spirit + Resolve.
  \item \textbf{Key skills:} Mechanics, Medicine, Perception.
  \item \textbf{Playstyle:} Problem-solving, information gathering, analysis.
\end{itemize}

\subsection{The Face}

\begin{itemize}
  \item \textbf{Primary:} Presence + Sway/Deception.
  \item \textbf{Secondary:} Wits + Investigation, Spirit + Performance.
  \item \textbf{Key skills:} Command, Diplomacy, Streetwise.
  \item \textbf{Playstyle:} Social maneuvering, negotiation, public scenes.
\end{itemize}

\subsection{The Specialist}

\begin{itemize}
  \item \textbf{Primary:} Varies with concept (Arcana, Craft, Nature, etc.).
  \item \textbf{Secondary:} Supporting skills that keep your specialty relevant.
  \item \textbf{Playstyle:} Technical solutions, unique tricks, niche expertise.
\end{itemize}

% ============================================================
% IMPROVING CAPABILITIES
% ============================================================

\section{Improving Your Capabilities}
\label{sec:improving-capabilities}

\subsection{Balanced Development}

\begin{itemize}
  \item Raise both Attributes and Skills over time.
  \item Develop complementary skill sets (e.g.\ combat + tactics + logistics).
  \item Look for pairs of skills that feed the way you like to play.
\end{itemize}

\subsection{Specialized Focus}

\begin{itemize}
  \item Push one Attribute high and build several key Skills under it.
  \item Aim to be the person the group turns to for certain problems.
  \item Accept that outside that niche, you may lean on others.
\end{itemize}

\subsection{Versatile Approach}

\begin{itemize}
  \item Spread your investments for flexibility.
  \item Be ready to cover gaps when others are absent or busy.
  \item Trade raw power for adaptability and resilience.
\end{itemize}

% ============================================================
% ADVANCEMENT PHILOSOPHY
% ============================================================

\section{Advancement Philosophy}
\label{sec:advancement-philosophy}

Advancement should serve the story, not overshadow it. The best XP choices:

\begin{itemize}
  \item reflect what your character has experienced,
  \item open new kinds of scenes and problems,
  \item and support the group's overall story arc.
\end{itemize}

\noindent\textit{Final Thought:} Every XP spent changes not just your numbers, but your
character's journey. Choose advancements that will be fun to play, fun to
watch, and true to who your character is becoming.

% ============================================================
% PRACTICAL EXAMPLES
% ============================================================

\section{Practical Examples}
\label{sec:practical-examples}

\subsection{Combat Example}

A warrior with Body~4 and Melee~3 attacks:

\begin{itemize}
  \item Dice pool: $4 + 3 = 7$d10.
  \item Opponent's defenses and the fiction help set the difficulty.
  \item Strong successes translate into higher Effect or better Position.
\end{itemize}

\subsection{Social Example}

A diplomat with Presence~3 and Sway~2 negotiates:

\begin{itemize}
  \item Dice pool: $3 + 2 = 5$d10.
  \item Position might be \emph{Controlled} if circumstances favor them, or worse if under suspicion.
  \item Consequences for failure could be mistrust, delays, or new complications.
\end{itemize}

\subsection{Exploration Example}

A scout with Wits~3 and Perception~2 searches for tracks:

\begin{itemize}
  \item Dice pool: $3 + 2 = 5$d10.
  \item Easier tasks (fresh tracks) are more forgiving; older or obscured tracks are harder.
  \item Partial successes might reveal some information but not all, or come with side effects.
\end{itemize}

\noindent Your Attributes and Skills determine not just \emph{if} you succeed, but
\emph{how} you tackle problems. Lean into combinations that express your
character's style.

% ============================================================
% NARRATIVE-HEAVY OPTIONS
% ============================================================

\section{Narrative-Heavy Options}
\label{sec:narrative-options}

If your group prefers a strong narrative focus, you can tilt advancement and
skill use toward story-first play.

\subsection{Advancement Options}

\begin{description}
  \item[Story-Driven Milestones:]  
    Instead of tracking precise XP, your GM may grant advances when you reach
    clear story milestones. ``You have trained with the swordmaster for
    months; raise Melee by 1.''

  \item[Experience Through Reflection:]  
    Use downtime scenes, flashbacks, or heartfelt conversations as ways to
    justify new capabilities. A powerful character moment can be worth as
    much as a battle.

  \item[Collaborative Advancement:]  
    Talk as a group about where each character is headed. Agree on advancement
    that supports ensemble stories as well as individual arcs.

  \item[Narrative Justification Focus:]  
    Whenever you spend XP, explain \emph{how} your character learned this. Tie
    progress to scenes you played, not just to numbers.
\end{description}

\subsection{Skill Options}

\begin{description}
  \item[Intent-Driven Skills:]  
    For straightforward, low-stakes actions, you can skip rolling entirely and
    simply describe what you accomplish.

  \item[Descriptive Assistance Bonuses:]  
    When helping someone, a vivid description of \emph{how} you assist can be
    enough to grant them a bonus die.

  \item[Skill as Character Development:]  
    Treat difficult skill uses as turning points: failures can inspire training
    montages or new goals.

  \item[Collaborative Difficulty Sense:]  
    Players can suggest whether something feels routine, tough, or desperate.
    The GM still sets the final difficulty, but everyone shares the feel.

  \item[Narrative Skill Synergies:]  
    Focus more on how skills interact in the fiction than on stacking bonuses.
    A clever plan that combines multiple skills might earn better Position or
    Effect even without extra dice.
\end{description}

% ============================================================
% CHARACTER CREATION WORKSHOP
% ============================================================

\section{Character Creation Workshop}
\label{sec:character-workshop}

This framework helps you and your group build characters who fit both the
rules and the story.

\subsection{Phase 1: Concept Development (about 30 minutes)}

\begin{enumerate}
  \item \textbf{Background Selection:} Choose culture, upbringing, and social class.
  \item \textbf{Motivation:} Decide what your character wants and why.
  \item \textbf{Relationships:} Map connections to other PCs or key NPCs.
  \item \textbf{Arc Planning:} Sketch how you would like your character to grow.
\end{enumerate}

\subsection{Phase 2: Mechanical Foundation (about 45 minutes)}

\begin{enumerate}
  \item \textbf{Attribute Allocation:} Spend your points based on concept priorities.
  \item \textbf{Skill Selection:} Choose starting skills that match your background.
  \item \textbf{Talents / Abilities:} Pick a few starting tricks or edges.
  \item \textbf{Resource Planning:} Decide on any initial assets or followers.
\end{enumerate}

\subsection{Phase 3: Narrative Integration (about 30 minutes)}

\begin{enumerate}
  \item \textbf{Backstory Refinement:} Connect your history to the campaign's themes.
  \item \textbf{Bond Establishment:} Build meaningful ties with other characters.
  \item \textbf{Complications:} Add flaws, debts, or enemies that will drive scenes.
  \item \textbf{Campaign Alignment:} Make sure your concept fits the group's tone.
\end{enumerate}

% ============================================================
% QUICK REFERENCE BOXES
% ============================================================

\begin{tcolorbox}[colback=green!5!white,colframe=green!75!black,title=XP Planning Guide,fonttitle=\bfseries]
\textbf{Early Tier Priorities:}
\begin{itemize}
  \item Raise a core Attribute to 3 (9~XP).
  \item Bring key Skills to 2--3 (4--6~XP each).
  \item Claim 1--2 minor resources (4~XP each).
\end{itemize}

\textbf{Mid Tier Expansion:}
\begin{itemize}
  \item Push core Attributes toward 4 (12~XP).
  \item Specialize key Skills to 4 (8~XP).
  \item Add standard resources (8~XP each).
  \item Explore cultural or thematic abilities (6--10~XP).
\end{itemize}

\textbf{Late Tier Mastery:}
\begin{itemize}
  \item Purchase capstone abilities (12+~XP).
  \item Establish major resources (12~XP).
  \item Launch legacy projects, organizations, or movements.
\end{itemize}
\end{tcolorbox}

\begin{tcolorbox}[colback=blue!5!white,colframe=blue!75!black,title=Attributes and Skills Quick Reference,fonttitle=\bfseries]
\textbf{Attributes (1--5):}
\begin{itemize}
  \item \textbf{Body}: physical capability
  \item \textbf{Wits}: mental acuity
  \item \textbf{Spirit}: willpower and inner strength
  \item \textbf{Presence}: social influence
\end{itemize}

\textbf{Skill Levels:}
\begin{itemize}
  \item 0: Untrained \quad 1: Novice \quad 2: Competent
  \item 3: Professional \quad 4: Master \quad 5: Grand Master
\end{itemize}

\textbf{Dice Pool:} Attribute $+$ Skill d10s.

\textbf{Skill Improvement:} New level $\times$ 2~XP (plus training time).

\textbf{Specialization:} At Skill~3+, you are a known expert in that field.

\textbf{Synergy:} Well-justified combinations of skills can grant +1 die or
better Position/Effect when the fiction supports it.
\end{tcolorbox}