\chapter{The Legacy Engine}
\label{ch:legacy-engine}

\section*{Succession \& Faction Building}
\addcontentsline{toc}{section}{Succession \& Faction Building}

\begin{quote}
\emph{“Your legacy is not what you build — it is who you build.”}
\end{quote}

The Legacy Engine is a narrative-first framework that transforms followers into future player characters, allowing campaigns to evolve across generations without losing focus or emotional continuity.

Rather than encouraging character replacement or optimization, this system treats succession as a \textbf{story event} — earned through action, consequence, and choice.

This expansion supports:
\begin{itemize}
  \item Natural character retirement, death, or transition
  \item Factions that grow through people, not territory alone
  \item Long-form campaigns spanning multiple generations
  \item Meaningful consequences that persist beyond a single PC
\end{itemize}

\section{Core Philosophy}

The Legacy Engine follows three principles:

\begin{enumerate}
  \item \textbf{Narrative Primacy} — Succession occurs because the story demands it, not because a clock fills.
  \item \textbf{People Before Power} — Followers become PCs because of who they are, not what they can do.
  \item \textbf{Consequences Persist} — Every legacy creates debts, rivals, and unfinished work.
\end{enumerate}

Legacy is not inherited cleanly. It carries weight.

\section{The Succession Path}

Followers do not become player characters automatically. Succession occurs through a three-stage narrative arc.

\subsection{The Call to Action}

A follower faces a moment where they must act beyond their role.

\textbf{Common Triggers:}
\begin{itemize}
  \item The PC is incapacitated, absent, or compromised
  \item A decision must be made that no one else can make
  \item A crisis demands leadership rather than obedience
\end{itemize}

\textbf{Mechanical Signal:}
When a follower takes a decisive action that alters the situation, the GM may mark \textbf{+1 segment} on that follower’s Succession Clock.

\subsection{The Trial}

The follower is tested.

This may be:
\begin{itemize}
  \item A single high-stakes scene
  \item A short sequence of escalating challenges
  \item A moral or political decision with lasting impact
\end{itemize}

\textbf{Outcome:}
\begin{itemize}
  \item \textbf{Success}: The follower is eligible for ascension.
  \item \textbf{Failure}: The story changes — but the follower remains significant.
\end{itemize}

Failure does not erase progress. It creates scars.

\subsection{The Ascension}

At a dramatically appropriate moment — often during downtime or immediately after a crisis — the follower becomes a player character.

This is not a reset. It is a continuation.

\section{Follower to PC Mechanics}

\subsection{Succession Clocks}

Key followers track a \textbf{Succession Clock} (4–6 segments).

\textbf{Fills Through:}
\begin{itemize}
  \item Solving a problem the PC could not
  \item Leading others under pressure
  \item Making a costly choice on behalf of the group
  \item Demonstrating values aligned with the faction
\end{itemize}

Routine assistance does not advance succession.

When the clock fills, the follower may ascend at the next narrative opening.

\subsection{Legacy Transfer}

When a follower becomes a PC, they receive \textbf{1d4 Legacy Points}.

These may be spent on:
\begin{itemize}
  \item Attributes (max +2 to any one)
  \item Skills (max +3 to any one)
  \item A legacy-linked Asset (banner, codex, title, relic)
\end{itemize}

Legacy points cannot be spent on Talents.

\subsection{Legacy Debt}

Every successor must choose one unresolved burden:

\begin{description}
  \item[Oath of Succession (6-clock)] Honor the predecessor’s ideals or commands.
  \item[Unfinished Task (4-clock)] Complete a mission left unresolved.
  \item[Rival Claim (4-clock)] Another party disputes the succession.
\end{description}

These clocks advance through Story Beats and narrative pressure.

\subsection{The Legacy Bond}

The new PC begins play with a bond to their predecessor.

\textbf{Once per session}, spend 1 Boon to:
\begin{itemize}
  \item Recall a memory
  \item Receive instinctive guidance
  \item Gain insight into a familiar situation
\end{itemize}

Each use marks an \textbf{Echo Clock} (6).  
When full, the bond must be resolved — through acceptance, rejection, or transformation.

\section{Building a Living Faction}

\subsection{The Legacy Web}

Each active PC within the lineage adds \textbf{+1} to the faction’s \textbf{Influence Pool}.

Influence may be spent to:
\begin{itemize}
  \item Gain access to restricted locations
  \item Secure political leverage
  \item Resolve internal faction tensions
\end{itemize}

Unused influence attracts attention.

\subsection{Faction Clocks}

\paragraph{Growth Clock (6–8)}
Tracks expansion, reputation, and reach.

\paragraph{Health Clock (6)}
Tracks internal strain, enemies, and instability.

When the Health Clock fills, a major crisis erupts.  
Ignoring it risks schism, collapse, or corruption.

\subsection{Faction Abilities}

As Growth advances, factions unlock narrative permissions:
\begin{itemize}
  \item Sanctuary
  \item Intelligence Network
  \item Legacy Resonance (ancestral aid, symbolism)
\end{itemize}

Abilities define tone, not raw power.

\section{Succession Events}

\subsection{The Heir’s Choice}

When a PC retires or dies:
\begin{itemize}
  \item The player selects a successor
  \item The successor inherits faction progress
  \item The former PC leaves behind one unresolved hook
\end{itemize}

\subsection{The Succession Scene}

A pivotal narrative moment:
\begin{itemize}
  \item The departing PC may spend 1 Boon to aid
  \item The heir rolls under narrative pressure
  \item Failure creates complications, not cancellation
\end{itemize}

\section{Campaign Phases}

\begin{description}
  \item[Foundation] Establish followers and values.
  \item[Expansion] Manage growth and competing claims.
  \item[Maturity] Influence reshapes the world.
  \item[Reckoning] The legacy is tested.
\end{description}

\section{End of Campaign: The Final Legacy}

At campaign end, remaining Legacy Points may be spent to:
\begin{itemize}
  \item Define the faction’s future
  \item Shape the political landscape
  \item Establish institutions or myths
\end{itemize}

The story ends — but the legacy remains.

\begin{quote}
\emph{A character dies. A faction endures. A story echoes forward.}
\end{quote}
