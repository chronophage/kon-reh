\chapter{Talents and Special Abilities}
\label{ch:talents}

Talents are the building blocks of character specialization. They represent learned techniques, supernatural gifts, or cultural inheritances. Each Talent costs XP, and their costs are tied to impact. Only one talent can be active at a time unless otherwise specified

\section{Understanding Talents}
\index{talents}

Talents are purchased with \textbf{Experience Points (XP)} and provide special capabilities:
\begin{itemize}
\item They go beyond simple skill bonuses
\item They often have specific \textbf{activation conditions}
\item They may provide \textbf{narrative permissions} (you can try things others cannot)
\item They can define your character's \textbf{unique identity}
\end{itemize}

\subsection*{Talent Costs}
\begin{center}
\small
\begin{longtable}{lll}
\toprule
\textbf{Type} & \textbf{Cost} & \textbf{Examples} \\
\midrule
Minor Edge & 2 XP & Quick Hands, +1 situational bonus \\
Major Edge & 4 XP & Patron's Symbol, strong summon upgrade \\
Prestige & 6+ XP & Campaign-defining effects \\
\bottomrule
\end{longtable}
\end{center}

\paragraph{Activation Types.}
\begin{itemize}
\item \textbf{Passive}: Always on; no action
\item \textbf{Active}: Requires an action or scene focus
\item \textbf{Reactive}: Triggers on a condition
\end{itemize}

\paragraph{Limits and Economy.}
Unless a talent says otherwise:
\begin{itemize}
\item \textbf{Per Scene} uses refresh at scene end
\item \textbf{Per Session} uses refresh after downtime
\item Some talents allow you to spend \textbf{Boons} to push effects
\end{itemize}

\section{Talent Categories}
\index{talents!categories}

\subsection*{Minor Edge Talents}
\index{talents!minor}
Basic abilities available to any character:
\begin{itemize}
\item \textbf{Cost}: 6 XP
\item \textbf{Examples}: Spellcraft, Familiar Bond, basic magical abilities
\item \textbf{Best for}: Essential capabilities and access requirements
\end{itemize}

\subsection*{Major Edge Talents}
\index{talents!major}
Significant abilities with moderate requirements:
\begin{itemize}
\item \textbf{Cost}: 4 XP
\item \textbf{Examples}: Patron's Symbol, Codex, significant summon upgrades
\item \textbf{Best for}: Core specialization and magical access
\end{itemize}

\subsection*{Prestige Talents}
\index{talents!prestige}
Powerful abilities unlocked through mastery or story events:
\begin{itemize}
\item \textbf{Cost}: 6+ XP
\item \textbf{Examples}: Breaking fundamental limits, forbidden summons, rewriting obligations
\item \textbf{Best for}: Campaign-shaping capabilities
\end{itemize}

\section{Magic Access Talents}
\index{talents!magic access}

\subsection*{Spellcraft \textnormal{[FIRE] [WATER]}}
\textbf{Cost}: 6 XP \\
Grants access to Weave \& Cast freeform spellcasting using the Eight Elements. Without this, characters cannot freeform cast.

\subsection*{Familiar \textnormal{[SPIRIT] [BOND]}}
\textbf{Cost}: 2 XP \\
Required to access Patron features such as Patron's Gift. Binds a Thiasos.

\subsection*{Codex \textnormal{[FATE] [RITUAL]}}
\textbf{Cost}: 4 XP \\
Required to fully join a Patron's service as a Runekeeper. Grants access to that Patron's Rites and Obligation system.

\subsection*{Patron's Symbol \textnormal{[WARD] [SIGIL]}}
\textbf{Cost}: 4 XP \\
Minor Asset. Allows an Invoker to access a Patron's Rites via ritual precision. Each Patron requires its own Symbol.

\section{Patron's Gift (Imbuement)}
\index{talents!imbuement}

\textbf{Cost}: Free (requires Thiasos) \\
\textbf{Activation}: 1 Action once per scene \\
\textbf{Duration}: Scene \\
\textbf{Range}: Touch \\
\textbf{Effect}: Imbue one item with temporary magical power related to your Patron's domain. The item functions as a magical weapon (+1 Melee) and specialized tool (+1 thematic Skill) for the scene.

\textbf{Push It}: The item's power persists for one additional scene but marks +1 Obligation.

\section{Monk Talents}
\label{sec:monk-talents}

\subsection*{Core Concept}
Monks channel inner discipline into supernatural martial prowess, combining unarmed combat mastery with spiritual focus.

\subsection*{Starting Talent}
\paragraph{Disciplined Body (3 XP --- Minor Talent) \textnormal{[STRENGTH] [RECOVERY]}} 
\textbf{Requirements:} Melee 1+, Body 2+. \\
\textbf{Benefits:}
\begin{itemize}
  \item +1 die to unarmed combat attacks.
  \item Convert 1 Harm to Fatigue once per scene.
  \item Once per scene, improve Position by one step.
\end{itemize}

\subsection*{Advanced Talents}
\paragraph{Iron Fist Way (6 XP --- Minor Talent) \textnormal{[STRIKE] [BURNING]}} 
\textbf{Benefits:} +1 die to unarmed attacks; strikes count as enchanted.

\paragraph{Flowing Spirit Way (8 XP --- Major Talent) \textnormal{[WIND] [STRENGTH]}} 
\textbf{Benefits:} Convert up to 1 Harm into Fatigue per attack; +1 die against fear or charm.

\paragraph{Perfect Timing Way (7 XP --- Major Talent) \textnormal{[FATE] [MOVEMENT]}} 
\textbf{Benefits:} Twice per scene, improve Position by +1 step; +1 die to reactions.

\paragraph{Untouchable Way (12 XP --- Major Talent) \textnormal{[EVADE] [STRENGTH]}} 
\textbf{Prerequisites:} Iron Fist + Flowing Spirit. \\
\textbf{Benefits:} +1 die to unarmed attacks; convert 2 Harm into Fatigue; cannot be grappled.

\paragraph{Inevitable Way (15 XP --- Major Talent) \textnormal{[FATE] [FORCE]}} 
\textbf{Prerequisites:} Iron Fist + Perfect Timing. \\
\textbf{Benefits:} +2 dice to unarmed attacks; ignore 1 Armor; may counterattack when an enemy misses.

\paragraph{Transcendent Harmony (18 XP --- Epic Talent) \textnormal{[LIFE] [WIND]}} 
\textbf{Prerequisites:} Flowing Spirit + Perfect Timing, Spirit 4+. \\
\textbf{Benefits:} Convert 2 Harm into Fatigue; once per session become immune to Harm; allies gain +1 defense.

\subsection*{Progression Path}
Monks specialize early (6--8 XP), combine paths mid-tier (12--15 XP), and achieve transcendence late (18 XP). Each path represents a distinct combat philosophy and playstyle.

\section{Selecting Talents}
\index{talents!selection}

\subsection*{Consider Your Magical Path}
Choose talents that reinforce your character's magical approach:
\begin{itemize}
\item \textbf{Caster}: Freeform spellcasting talents, elemental control
\item \textbf{Runekeeper}: Rites access, Obligation management, Patron specialization
\item \textbf{Invoker}: Ritual efficiency, Symbol maintenance, invocation speed
\item \textbf{Specialist}: Unique talents matching your specific focus
\end{itemize}

\subsection*{Balance Access and Power}
Consider both access requirements and power talents:
\begin{itemize}
\item \textbf{Access}: Essential prerequisites (Spellcraft, Familiar)
\item \textbf{Power}: Combat enhancements, magical amplifications
\item \textbf{Utility}: Support abilities, resource management
\end{itemize}

\subsection*{Think About Investment}
Consider how much XP each talent represents:
\begin{itemize}
\item \textbf{Minor (2 XP)}: Essential access, small narrative tricks
\item \textbf{Major (4 XP)}: Strong upgrades, permanent effects in niche
\item \textbf{Prestige (6+ XP)}: Campaign-defining, fundamental limits broken
\end{itemize}

\section{Talent Building Strategies}
\index{talents!building strategies}

\subsection*{The Specialist}
Focus on talents supporting one primary magical path:
\begin{itemize}
\item Choose talents that \textbf{synergize} with each other
\item Develop a clear specialization identity
\item Become the go-to character for specific magical challenges
\item \textbf{Risk}: May be less effective outside specialty
\end{itemize}

\subsection*{The Generalist}
Spread talents across multiple magical approaches:
\begin{itemize}
\item Cover different types of magical challenges
\item Provide support to other party members
\item Adapt to diverse situations
\item \textbf{Risk}: Less peak capability in any area, increased bookkeeping
\end{itemize}

\subsection*{The Foundation Builder}
Focus on essential access talents first:
\begin{itemize}
\item Prioritize access requirements (Spellcraft, Familiar)
\item Build toward major capabilities
\item Establish core identity before specialization
\item \textbf{Risk}: May lack immediate power payoff
\end{itemize}

\section{Talent Examples}
\index{talents!examples}

\subsection*{Magic Access Talents}
\begin{description}
\item[Spellcraft \textnormal{[FIRE] [WATER] \textit{(Minor)}}] (6 XP) --- Access to Weave \& Cast freeform spellcasting using the Eight Elements.
\item[Familiar \textnormal{[SPIRIT] [BOND] \textit{(Minor)}}] (2 XP) --- Required for Patron's Gift and other Patron features.
\item[Codex \textnormal{[FATE] [RITUAL] \textit{(Major)}}] (4 XP) --- Full access to a Patron's Rites and Obligation system.
\item[Patron's Symbol \textnormal{[WARD] [SIGIL] \textit{(Major)}}] (4 XP) --- Ritual access to a Patron's Rites via invocation.
\end{description}

\subsection*{Combat Talents}
\begin{description}
\item[Second Wind \textnormal{[RECOVERY] \textit{(Minor)}}] (2 XP, \emph{Active}) --- Once per scene, clear 1 Fatigue when you take a moment to catch your breath.
\item[Combat Reflexes \textnormal{[REACTIVE] \textit{(Minor)}}] (2 XP, \emph{Reactive}) --- +1 die on defense rolls when surprised or flanked.
\item[Precise Strike \textnormal{[STRIKE] \textit{(Minor)}}] (2 XP, \emph{Active}) --- Once per scene, ignore armor on one attack if you had \textbf{Dominant} or \textbf{Controlled} position.
\item[Weapon Mastery \textnormal{[WEAPON] [STRIKE] \textit{(Major)}}] (4 XP, \emph{Passive}) --- Choose a weapon type; +1 die when using it.
\end{description}

\subsection*{Social Talents}
\begin{description}
\item[Silver Tongue \textnormal{[PERSUASION] \textit{(Minor)}}] (2 XP, \emph{Passive}) --- +1 die on persuasion attempts.
\item[Read Emotions \textnormal{[INSIGHT] \textit{(Minor)}}] (2 XP, \emph{Active}) --- Once per scene, automatically detect surface emotions in a social exchange.
\item[Command Presence \textnormal{[LEADERSHIP] \textit{(Major)}}] (4 XP, \emph{Passive}) --- +1 die on leadership and intimidation rolls.
\item[Network Builder \textnormal{[BOND] [RELATIONSHIP] \textit{(Major)}}] (4 XP, \emph{Passive}) --- Gain a minor contact in each new settlement visited.
\end{description}

\subsection*{Exploration Talents}
\begin{description}
\item[Keen Senses \textnormal{[PERCEPTION] \textit{(Minor)}}] (2 XP, \emph{Passive}) --- +1 die on perception checks to spot danger or hidden details.
\item[Wilderness Lore \textnormal{[NATURE] \textit{(Minor)}}] (2 XP, \emph{Passive}) --- Automatically find food and water in hospitable biomes.
\item[Trackless Step \textnormal{[EVADE] \textit{(Minor)}}] (2 XP, \emph{Active}) --- Leave no trail for the rest of the scene.
\item[Urban Navigation \textnormal{[DIRECTION] \textit{(Minor)}}] (2 XP, \emph{Passive}) --- Never get lost in cities.
\end{description}
\section{Advanced Talent Examples}
\index{talents!advanced}

\subsection*{Casting Mastery}
\begin{description}
\item[Spell Shaping \textnormal{[CONTROL] [FORM] \textit{(Major)}}] (4 XP; Req: Spellcraft) --- Modify spell factors (range/scale/targeting) by one step when you Weave.
\item[Elemental Mastery \textnormal{[ELEMENT] [HARMONY] \textit{(Prestige)}}] (6 XP; Req: Arcana 3) --- Reduce backlash severity by one step when casting spells of your chosen element.
\item[Arcane Dominance \textnormal{[POWER] [OVERRIDE] \textit{(Prestige)}}] (6 XP; Req: Spirit 4, Arcana 4) --- Overpower weaker magical effects automatically when you contest them.
\end{description}

\subsection*{Ritual Expertise}
\begin{description}
\item[Ritual Mastery \textnormal{[RITUAL] [STABILITY] \textit{(Major)}}] (4 XP; Req: Familiar) --- Perform rituals with reduced risk: the GM spends 1 fewer SB on ritual backlash.
\item[Efficient Invocation \textnormal{[RITUAL] [SPEED] \textit{(Major)}}] (4 XP; Req: Patron's Symbol) --- Reduce ritual casting time by one step (minimum 1 Player Turn).
\item[Crack Specialist \textnormal{[RITUAL] [INSTANT] \textit{(Prestige)}}] (6 XP; Req: 3 Patron Symbols) --- Reduce Crack the Seal Obligation cost by 1 (minimum +1).
\item{Dual Covenant \textnormal{[SYMBOL] [BOND] \textit{(Prestige)}} (6 XP):} Maintain two active summons.
\end{description}

\subsection*{Prestige Abilities}
\begin{description}
\item[Forbidden Knowledge \textnormal{[SECRETS] [DANGEROUS] \textit{(Prestige)}}] (6 XP; Req: Tier II) --- Access to one forbidden summon or dangerous rite.
\item[Obligation Master \textnormal{[DEBT] [CONTROL] \textit{(Prestige)}}] (8 XP; Req: Tier III, Codex) --- Reduce all Obligation segment costs by 1 (minimum 1).
\item[Backlash Immunity \textnormal{[PROTECT] [RESISTANCE] \textit{(Prestige)}}] (10 XP; Req: Tier IV, Spirit 5) --- Ignore minor backlash entirely on casting rolls.
\item{Triad Bond \textnormal{[BOND] [SYNCHRONIZATION] \textit{(Prestige)}} (8 XP):} Maintain three active summons.
\end{description}

\section{Talent Synergies}
\index{talents!synergies}

Some talents work particularly well together:

\subsection*{Casting Synergies}
\begin{itemize}
\item \textbf{Spellcraft + Spell Shaping}: Flexible, precise freeform casting
\item \textbf{Elemental Mastery + Arcane Dominance}: Powerful, controlled elemental effects
\item \textbf{Ritual Mastery + Spellcraft}: Reduced risk on both freeform and ritual casting
\end{itemize}

\subsection*{Social Synergies}
\begin{itemize}
\item \textbf{Silver Tongue + Command Presence}: Charm or command with equal force
\item \textbf{Read Emotions + Network Builder}: Understand and leverage social connections
\item \textbf{Familiar + Social Talents}: Patron-enhanced social abilities
\end{itemize}

\subsection*{Exploration Synergies}
\begin{itemize}
\item \textbf{Keen Senses + Trackless Step}: Find others while leaving no trace
\item \textbf{Wilderness Lore + Urban Navigation}: Comfortable in all environments
\item \textbf{Familiar + Exploration Talents}: Patron-guided exploration
\end{itemize}

\section{Talent Limitations and Balance}
\index{talents!limitations}

\subsection*{Usage Restrictions}
Most talents have limits to maintain game balance:
\begin{itemize}
\item \textbf{Per scene}: Common for strong actives
\item \textbf{Per session}: Reserved for swingy effects
\item \textbf{Resource cost}: Some require spending Boons or generating Obligation
\item \textbf{Position requirements}: May require specific narrative circumstances
\end{itemize}

\subsection*{Prerequisite Systems}
\index{talents!prerequisites}
Advanced talents require meeting certain conditions:
\begin{itemize}
\item \textbf{Attribute minimums}: e.g., Spirit 4, Wits 3
\item \textbf{Skill requirements}: Specific skills at set levels
\item \textbf{Previous talents}: Foundational picks first (Familiar required for Patron features)
\item \textbf{Tier requirements}: Character advancement level
\end{itemize}

\section{Building Your Talent Set}
\index{talents!building}

\subsection*{Early Game (0--40 XP)}
Focus on essential access and basic capabilities:
\begin{itemize}
\item 1--2 access talents (Spellcraft, Familiar)
\item 2--3 basic talents for reliability
\item Save XP for major access requirements
\item Choose talents that work with your core concept
\end{itemize}

\subsection*{Mid Game (41--90 XP)}
Develop your specialization:
\begin{itemize}
\item Major access talents (Codex, Patron's Symbol)
\item 2--3 synergistic power talents
\item Balance active and passive picks
\item Plan for prestige abilities
\end{itemize}

\subsection*{Late Game (91+ XP)}
Achieve mastery:
\begin{itemize}
\item 1--2 prestige talents defining your apex
\item Picks that create legacy effects
\item Talents that benefit the whole party
\item Prepare for campaign-defining challenges
\end{itemize}

\section{Talent Customization}
\index{talents!customization}

Work with your Game Master to create custom talents:
\begin{itemize}
\item \textbf{Based on story events}: Reflect character experiences
\item \textbf{Balanced costs}: Match similar scope to existing talents (2/4/6+ XP)
\item \textbf{Clear prerequisites}: Define requirements clearly
\item \textbf{Mechanical clarity}: Define activation, effects, and limits
\end{itemize}

\section{Talents and Group Dynamics}
\index{talents!group dynamics}

Consider how your talents complement the party:
\begin{itemize}
\item \textbf{Fill gaps}: Cover party weaknesses in magical capabilities
\item \textbf{Synergize}: Coordinate with other players' magical approaches
\item \textbf{Avoid overlap}: Don't duplicate another character's access path
\item \textbf{Support role}: Talents that help the whole group manage magical risks
\end{itemize}

\section{Talent Respecification}
\index{talents!respecification}

If your character concept changes, you may respec talents:
\begin{itemize}
\item \textbf{GM approval required}: Discuss proposed changes
\item \textbf{Downtime cost}: Represent retraining (typically 1 downtime period)
\item \textbf{Story justification}: Explain the change in-narrative
\item \textbf{Limited frequency}: Typically once per major story arc
\end{itemize}

\begin{tcolorbox}[colback=purple!5!white,colframe=purple!75!black,title=Talent Selection Guide,fonttitle=\bfseries]
\textbf{Early Game (0--40 XP):}
\begin{itemize}
\item 1--2 access talents (2 XP each)
\item 2--4 basic talents (2 XP each)
\item Focus on essential capabilities
\end{itemize}

\textbf{Mid Game (41--90 XP):}
\begin{itemize}
\item 1--2 major talents (4 XP each)
\item 1--2 advanced talents (4--6 XP each)
\item Plan for prestige prerequisites
\end{itemize}

\textbf{Late Game (91+ XP):}
\begin{itemize}
\item 1--2 prestige talents (6+ XP each)
\item Campaign-defining capabilities
\item Party-supporting abilities
\end{itemize}

%===========================================================
\subsection*{Talents that Key Off the Defend Action}
\index{Talents!Defensive Synergies}
%===========================================================

\paragraph{Counterpoise (2 XP) \textnormal{[EVADE] [MOVEMENT]}} 
\textbf{Requirement:} Athletics 1+ or Melee 1+ \\
\textbf{Effect:} When you take a \textbf{Defend} action and score a \emph{Partial}, you may immediately move 1 range band or reposition within Near without provoking retaliation.  
This represents instinctive recoil and evasive reflex.

\medskip

\paragraph{Riposte Instinct (4 XP) \textnormal{[COUNTER] [STRIKE]}} 
\textbf{Requirement:} Melee 2+ \\
\textbf{Effect:} When you \textbf{Defend} and score a \emph{Success}, you may make a \textit{free Melee attack} against the attacker at \textbf{Position -1}.  
Once per scene.

\medskip

\paragraph{Iron Patience (2 XP) \textnormal{[RESISTANCE] [STRENGTH]}} 
\textbf{Requirement:} Resolve 1+ \\
\textbf{Effect:} When you \textbf{Defend}, gain \textbf{+1d} to resist Conditions (fear, charm, fatigue) for the rest of the exchange, even outside the triggering roll.

\medskip

\paragraph{Guarded Step (4 XP) \textnormal{[STANCE] [MOVEMENT]}} 
\textbf{Requirement:} Athletics 2+ or Shield Proficiency \\
\textbf{Effect:} When Defending, you may \textbf{declare your defense pool after the GM calls Position}.  
Once per scene, you may instead choose the Attribute + Skill pair after the roll.

\medskip

\paragraph{Stand Firm (2 XP) \textnormal{[PROTECT] [STABILITY]}} 
\textbf{Requirement:} Body 2+ \\
\textbf{Effect:} When you \textbf{Defend} and roll a \emph{Miss}, downgrade incoming \textbf{Harm by 1 level} (minimum 0).

\medskip

\paragraph{Measured Guard (4 XP) \textnormal{[TACTICS] [FLEXIBILITY]}} 
\textbf{Requirement:} Perception 2+ \\
\textbf{Effect:} When Defending, you may \textbf{declare your defense pool after the GM calls Position}.  
Once per scene, you may instead choose the Attribute + Skill pair after the roll.

\medskip

\paragraph{High Stance Duelist (6 XP) \textnormal{[DOMINANCE] [TECHNIQUE]}} 
\textbf{Requirement:} Melee 3+, Presence 1+ \\
\textbf{Effect:} While Defending:
\begin{itemize}
  \item Enemies suffer \(-1d\) to attacks against you.
  \item On Success, your riposte becomes \textbf{Dominant} instead of Controlled.
\end{itemize}

\medskip

\paragraph{Wardbreaker Reflex (4 XP) \textnormal{[WARD] [DEFENSE]}} 
\textbf{Requirement:} Spirit 2+ or any magical Path \\
\textbf{Effect:} When Defending against \emph{spells or supernatural effects}, you may reroll 1 failed die.  
If your reroll succeeds, mark +1 Fatigue (arcane recoil).

\medskip

\paragraph{Shrike Parry (Prestige, 6 XP) \textnormal{[COUNTER] [PARRY]}} 
\textbf{Requirement:} Melee 3+, Athletics 2+, notable duel history \\
\textbf{Effect:} When Defending against melee attacks:
\begin{itemize}
  \item On Success — negate the hit and gain a free attack at \textbf{equal Position}.
  \item On Partial — reduce Harm by two levels instead of one.
  \item On Miss — you automatically shift Position upward next turn without needing to Defend again.
\end{itemize}

\medskip

\paragraph{Mystic Bulwark (Prestige, 6 XP) \textnormal{[PROTECT] [BOND]}} 
\textbf{Requirement:} Resolve 3+, Spirit 2+, access to Rites or Cantos \\
\textbf{Effect:} When you Defend, you may \textbf{Shield an adjacent ally}:
\begin{itemize}
  \item They gain your Position bonus.
  \item If you succeed, both of you avoid the effect.
  \item Once per scene, you may spend 2 Fatigue to extend this to all allies in Near.
\end{itemize}

\paragraph{Defensive Style Tags}
Some characters express defense through distinct \emph{styles}.  

Choose one (optional):
\begin{itemize}
  \item \textbf{Evasive \textnormal{[EVADE] [AGILITY]}}: +1d to avoid, but cannot Riposte this scene.
  \item \textbf{Interposing \textnormal{[PROTECT] [BOND]}}: You may Defend for an adjacent ally, but gain Fatigue on Miss.
  \item \textbf{Punishing \textnormal{[STRIKE] [COUNTER]}}: Gain +1d to Riposte effects, but Defense rolls count 1s twice for SB.
\end{itemize}

\section{Melee Combat Talents}

\subsection{Minor Talents}

\subsubsection{Defensive Survival (3 XP) \textnormal{[DEFENSE] [RECOVERY]}}
\textbf{Requirements:} Melee 2+ \\
\textbf{Effect:} +1 die to defense rolls while engaged in melee. Once per scene, convert first Harm 1 from melee to Fatigue. \\
\textbf{Narrative:} Years of combat teaching you to read attacks and flow with them.

\subsubsection{Tactical Movement (4 XP) \textnormal{[MOVEMENT] [POSITION]}}
\textbf{Requirements:} Athletics 2+ \\
\textbf{Effect:} Move within engagement zone as Move action (instead of full action). Once per scene, disengage from Close as Move action. \\
\textbf{Narrative:} Footwork and positioning that keeps you alive in the press.

\subsubsection{Conditioning (4 XP) \textnormal{[RESISTANCE] [ENDURANCE]}}
\textbf{Requirements:} Body 3+ \\
\textbf{Effect:} Body attribute counts as +1 for Fatigue track calculations. +1 die to resist Fatigue overflow effects. \\
\textbf{Narrative:} Physical conditioning that lets you endure punishment.

\subsubsection{Weapon Master (5 XP) \textnormal{[MASTERY] [STRIKE]}}
\textbf{Requirements:} Melee 2+ \\
\textbf{Effect:} +2 dice (instead of +1) with chosen weapon category. Once per scene, +1 Effect with signature weapon. \\
\textbf{Narrative:} Mastery of specific weapons that makes them extensions of yourself.

\subsection{Major Talents}

\subsubsection{Flurry Strike (7 XP) \textnormal{[STRIKE] [SPEED]}}
\textbf{Requirements:} Melee 3+, Body 3+ \\
\textbf{Effect:} When engaged with multiple opponents, make 2 attacks as one action. Each attack at -1 die. \\
\textbf{Narrative:} Training that lets you fight multiple enemies simultaneously.

\subsubsection{Duelist's Edge (8 XP) \textnormal{[DOMINANCE] [TECHNIQUE]}}
\textbf{Requirements:} Melee 3+, Wits 3+ \\
\textbf{Effect:} When engaged with single opponent: +1 die to all melee rolls. Once per scene, ignore first Harm 1 or 2 from that opponent. \\
\textbf{Narrative:} Psychological and tactical dominance in one-on-one combat.

\subsubsection{Battlefield Mastery (8 XP) \textnormal{[STRATEGY] [LEADERSHIP]}}
\textbf{Requirements:} Melee 4+, Wits 4+, Command 2+ \\
\textbf{Effect:} Once per scene, when engaged with 3+ opponents, declare "Battlefield Mastery." For next 3 exchanges:
\begin{itemize}
    \item All melee attacks gain +1 Effect
    \item Enemies act at -1 die due to disorientation
    \item Your Position improves by one step
    \item Convert one Harm 1→Fatigue per exchange
\end{itemize}
\textbf{Narrative:} When surrounded, you enter a state of perfect combat flow where enemies become obstacles rather than threats.

\paragraph*{Subtle Casting (Major Talent -- 8 XP) \textnormal{[HIDDEN] [CONTROL]}}%
\textit{Prerequisite: Lore 3+, Performance 2+ \textbf{or} Runekeeper with Codex}\\[3pt]

\textbf{Effect:}
Make a \textbf{Performance + Lore} roll to quietly cast a spell, invoke a Rite, or sing a Cantos against DV (Tier). If successful, the casting does \emph{not} generate Story Beats on the Channel or initial roll. Any SB generated are \textbf{banked by the GM} and applied at dramatically appropriate moments.

\medskip

\textbf{Limitations:}
Cannot be used for \textbf{Great} or \textbf{Extreme} Tier effects. The Weave phase (if applicable) still generates normal SB.

\begin{quote}
\emph{"True subtlety is not silence, but harmony --- when even the wind believes it sang the song."}
\end{quote}

\paragraph*{Berserker Rage (Major Talent -- 8 XP) \textnormal{[FURY] [POWER]}}%
\textit{Prerequisite: Melee 3+, Spirit 3+, Body 3+}\\[3pt]

\textbf{Effect:}
\begin{itemize}
  \item Gain +3 dice to all melee attacks.
  \item Ignore first Harm~1 each round while raging.
  \item Ignot any Harm or Fatigue penalties while raging.
  \item Lasts 3 rounds.
  \item When Rage ends, mark +Spirit Fatigue.
  \item Cannot be ended early.
  \item Position becomes Desperate while raging (re-roll successes).
  \item \textbf{While raging, you cannot benefit from Armor conversion effects.}
  \item \textbf{While raging, you may activate one additional Major, Prestige, or Epic talent by accepting one of these costs:}
  \begin{itemize}
    \item \textbf{Reduce rage bonus to +2 dice,} or
    \item \textbf{Mark +1 additional Fatigue when rage ends,} or
    \item \textbf{Cannot ignore first Harm~1 this round,} or
    \item \textbf{Extend rage duration by 1 round (instead of reducing it).}
  \end{itemize}
  \item \textbf{Once per scene.}
  \item \textbf{After rage ends, you cannot take actions requiring combat or mental focus until your next turn (Recovery Period).}
\end{itemize}

\noindent
\textbf{Narrative Integration:}\\
\textit{``The battle-fury strips away all pretense of defense. You become a weapon of pure destruction, but your flesh bears the cost of such power. Armor becomes a cage that cannot contain the storm within.''}

\paragraph{Backstab (Major Talent, 8 XP) \textnormal{[STEALTH] [STRIKE]}} 
\textbf{Req:} Stealth~2+, Melee~2+, Light weapon.  

\textbf{Effect:} When you attack an \textbf{Unaware} or \textbf{Engaged} foe from \textbf{Stealth}, deal \emph{+1 Harm} and ignore \emph{1 point of their Armor}.  

\textbf{Definitions:}  
\begin{itemize}
  \item \textbf{Unaware:} The target is not aware of your presence or hostile intent. This typically requires being \emph{Hidden} or having succeeded on a \emph{Stealth} test.  
  \item \textbf{Engaged:} The target is currently taking an \emph{Attack} action against another character, or casting a spell/ritual that specifically targets another character.  
  \begin{itemize}
    \item In miniatures/tactical play: the target is in melee range (\emph{Close}) with another PC/NPC and actively fighting them.  
  \end{itemize}
\end{itemize}

\textbf{Limit:} Once per scene. To use again, you must first \emph{re-enter Stealth} (DV by narrative) and mark 1 \emph{Fatigue} (e.g., via \emph{Shadow Dance}).  

\textbf{On a Miss:} You are \emph{Exposed} --- drop to \emph{Desperate Position} or mark 1 \emph{Harm}.    

\paragraph{Shadow Dance (Synergy Talent, 10 XP) \textnormal{[EVADE] [STEALTH]}}  
\textbf{Req:} Backstab, Stealth~3+, Mobility~2+.  

\textbf{Effect:} After a successful \textbf{Backstab}, you may immediately test \textbf{Stealth} vs. DV (Tier).  
\begin{itemize}
  \item On success: You \emph{re-enter Stealth} and may either \textbf{clear 1 Fatigue} or \textbf{improve Position +1}.  
  \item On failure: You remain \emph{Exposed} and must mark 1 \emph{Fatigue}.  
\end{itemize}  

\textbf{Limit:} May only chain once per scene.   

\paragraph{Deathblow (Capstone Talent, 12 XP) \textnormal{[DEATH] [STRIKE]}}  
\textbf{Req:} Shadow Dance, Stealth~4+, Melee/Ranged~3+.  

\textbf{Effect:} When you strike from \textbf{Dominant Position} or after re-entering \textbf{Stealth} via \emph{Shadow Dance}, you may declare a \textbf{Deathblow}.  
\begin{itemize}
  \item On a hit: Deal \emph{triple Harm}. If the attack incapacitates the target, you may immediately attempt a free \textbf{Stealth} test (DV by narrative) to vanish.  
  \item On a miss: You are \emph{Exposed} — drop to Desperate Position and mark 1 \emph{Harm}.  
\end{itemize}  

\textbf{Limit:} Once per scene. You may mark 1 \emph{Fatigue} to attempt a second time.  

\subsubsection*{Elusive Dodge (Major Talent, 8 XP) \textnormal{[DEFENSE] [EVADE]}}
\index{Talents!Elusive Dodge}

\textbf{Requirements:} Stealth 2+, Wits 3+, Tier II+

\paragraph{Effect:}
Once per exchange when targeted by a physical attack,  
you may immediately \textbf{enter a Defend stance} as a free reaction.

\paragraph{Cost:}
After use, mark \textbf{1 Fatigue}.

\paragraph{Limit:}
Cannot be used if Helpless, Surprised, or Exhausted.

\subsubsection*{Miststep Reflex (Capstone Talent, 12 XP) \textnormal{[DEFENSE] [MOVEMENT]}} 
\index{Talents!Miststep Reflex}

\textbf{Requirements:} Elusive Dodge, Tier III+, Stealth 4+, Athletics 3+

\paragraph{Effect:}
You are always treated as \textbf{Defending} against physical attacks  
unless you are Exposed, Grappled, or Surprised.

If you take the Defend action normally, improve Position by \textbf{+2 steps} instead.

\paragraph{Limit:}
Once per scene when you roll 1s while defending, you must mark \textbf{2 Fatigue}
or lose the benefit for the rest of the scene.
\subsection{Prestige Talents}

\subsubsection{Battlefield Terror (12 XP) \textnormal{[FEAR] [DOMINANCE]}}
\textbf{Requirements:} Melee 4+, Body 4+, Harm 2+ experience \\
\textbf{Effect:} Enemies in Close range act at -1 die due to intimidation. Once per scene, convert enemy's success to partial with cost. \\
\textbf{Narrative:} Reputation and presence that makes opponents hesitate.

\subsection{Epic Talents}

\subsubsection{Blade Dance (18 XP) \textnormal{[STRIKE] [MASTERY]}}
\textbf{Requirements:} Melee 5+, Duelist's Edge, Flurry Strike \\
\textbf{Effect:} Engage and attack up to 3 targets in one action. Each attack at -1 die, but Position improves by one step. \\
\textbf{Narrative:} Legendary skill that makes you a whirlwind of death.

\subsection{Combat Balance Notes}

These talents are designed to enhance melee viability while maintaining Fate's Edge's core tension between risk and reward. Melee combat should remain \textbf{manageably deadly} - dangerous enough to require tactical skill, but with meaningful options for skilled fighters to excel.

\textbf{Key Principles:}
\begin{itemize}
    \item Talents enhance existing mechanics rather than replace them
    \item Specialization provides clear advantages for focused builds
    \item High-cap opponents remain genuinely threatening
    \item Positioning and tactical decision-making remain crucial
    \item Story Beat escalation continues to compound challenges
\end{itemize}

\textbf{Role Balance:} Enhanced melee fighters complement rather than overshadow other roles. Ranged characters maintain mobility advantages, magic users provide battlefield control, and support characters enable team effectiveness.

\subsection{Embrace the Void (Major Talent, 8 XP) \textnormal{[CORRUPTION] [POWER]}}
\paragraph{For those who walk the knife-edge between power and damnation.}

\textbf{Prerequisites:} Any character with 2+ levels in a skill tied to their Patron's domain, and at least one segment of \indexterm{Obligation} to that Patron.

\textbf{Effect:} Once per session, you may choose to fully embrace your Patron's corrupting influence to gain significant temporary power.

\textbf{Activation:}
\begin{itemize}
    \item Immediately mark 2 segments of \indexterm{Obligation} to your chosen Patron.
    \item Mark 1 segment on that Patron's specific \indexterm{Corruption Table}.
    \item Gain one of the following benefits for the remainder of the scene:
    \begin{itemize}
        \item \textbf{Power Surge:} +1 die and +1 effect on all rolls related to that Patron's domain.
        \item \textbf{Defiance:} Immunity to one specific consequence type (fear, charm, physical harm, etc.) for the scene.
        \item \textbf{Forbidden Rite:} Use one Rite of that Patron without marking additional Obligation (Backlash still applies).
        \item \textbf{Tempting Tongue:} +1 effect on all social manipulations for the scene.
    \end{itemize}
\end{itemize}

\textbf{Cost:}
\begin{itemize}
    \item A permanent mark on your character sheet indicating embraced corruption.
    \item Your Patron's influence deepens: the GM gains +1 \indexterm{Story Beat} to spend against you whenever that Patron is relevant.
    \item You must roleplay the corruption's manifestations in future scenes.
    \item This Talent cannot be activated again until you clear at least 2 segments of Obligation through proper service to your Patron.
\end{itemize}

\textbf{Narrative Integration:} This Talent represents the Faustian bargain at the heart of Patron magic—power for a price. Players gain agency over their corruption, while ensuring that it always carries meaningful consequences.

\begin{tcolorbox}[title=Example Corruptions by Patron]
\textbf{Ikasha (Shadows):} You cannot lie about secrets you have learned; you compulsively seek hidden truths. \\
\textbf{Malachai (Chains \& Curses):} You bear a visible corruption mark; you crave increasingly dangerous curses to feel alive. \\
\textbf{Raéyn (Sea):} You draw the attention of sea creatures; you suffer --1 die on land-based actions. \\
\textbf{The Sealed Gate:} You attract entities seeking to cross thresholds; you compulsively seal or lock doors, gates, and bindings.
\end{tcolorbox}

\paragraph{Final Note.}
The best talents are those that fit your magical concept and table playstyle. Choose abilities you'll enjoy using, that create interesting consequences, and that contribute to your character's unfolding story through the lens of risk and consequence that defines Fate's Edge magic.

\section{Narrative-Heavy Talent Options}

For groups that prefer strong narrative focus in talent use, consider these optional approaches:

\textbf{Story-Driven Talents:} Instead of mechanical bonuses, some talents can provide narrative permissions or story effects. "Courtly Grace" might allow you to navigate noble society without rolls, while "Wild Empathy" lets you communicate with animals through roleplay rather than dice.

\textbf{Collaborative Talent Activation:} Players can describe how their talents work in the fiction, with GM approval, rather than relying solely on mechanical triggers. A "Master Strategist" might narrate how they reposition allies through clever tactics rather than just declaring the mechanical effect.

\textbf{Talent as Character Development:} Use talent acquisition as opportunities for character growth and backstory development, allowing players to narrate how their characters learned new abilities through significant story moments.

\textbf{Flexible Talent Interpretation:} Focus on the thematic effects of talents rather than strict mechanical applications. A "Weapon Mastery" talent might manifest differently depending on the weapon and situation, with the GM and player collaborating on the specific benefits.

\paragraph*{Elemental Mastery (18 XP Epic Talent) \textnormal{[ELEMENT] [HARMONY]}}%
\textit{Prerequisite: Tier III+, Spellcraft, Arcana 5, 2+ Focus Elements}\\[3pt]
You have achieved perfect harmony with your chosen elements, bending their chaotic nature to your will.

\begin{itemize}
  \item \textbf{Backlash Reduction:} Reduce all Backlash severity by one step when casting spells aligned with your focus elements
  \item \textbf{Elemental Synergy:} When casting with two focus elements, gain +2 dice and +1 Effect
  \item \textbf{Pure Channel:} Once per scene, cast a spell without generating SB on the Weave roll
  \item \textbf{Elemental Resonance:} Allies within Near gain +1 die to rolls when you successfully cast an aligned element spell
  \item \textbf{Requires casting a spell using your focus elements to activate}
\end{itemize}

\noindent
\textbf{Narrative Integration:}\\
\textit{``The elements no longer rage randomly - they dance to your conductor's baton. Fire burns where you will, water flows to your purpose, and lightning strikes with your intent.''}

\paragraph*{Crack the Seal Mastery (18 XP Epic Talent) \textnormal{[RITUAL] [PRECISION]}}%
\textit{Prerequisite: Tier III+, Invoker's Grimoire, 3+ Patron Symbols}\\[3pt]
Your mastery over ritual precision has reached such heights that you can bend the fundamental laws of invocation.

\begin{itemize}
  \item \textbf{Seal Control:} When Cracking the Seal, choose one benefit:
    \begin{itemize}
      \item Reduce Obligation cost by 1 (minimum 0)
      \item Prevent Symbol from becoming Compromised
      \item Gain +1 Effect on the ritual
    \end{itemize}
  \item \textbf{Symbol Resonance:} With 4+ Symbols, reduce all ritual casting times by 1 round
  \item \textbf{Emergency Cascade:} Once per scene, Crack the Seal on up to 3 different Symbols as a single action
  \item \textbf{Ritual Mastery:} Ignore the No Symbol penalties when performing rituals without the proper Symbol
  \item \textbf{Requires at least one Maintained Symbol}
\end{itemize}

\noindent
\textbf{Narrative Integration:}\\
\textit{``The seals speak to you now, not as barriers but as willing gates. Your will has become law in the spaces between worlds.''}
