\chapter{World Interaction}
\label{ch:world-interaction}

In \textbf{Fate's Edge}, the world is not a backdrop—it's a partner in the conversation. Dikes groan under black rain in Viterra, clan horns answer across Acasia's ridgelines, Ecktoria's marble halls echo with careful words, and Kahfagia's pilots read storms by taste. Wherever you go, place, culture, and pressure push back.

\section{Game Structure and Time}
\index{Time}\index{Game Structure}

Understanding how time works in Fate's Edge helps you navigate both the mechanical and narrative flow of play.

\subsection*{Basic Units}
\begin{description}
\item[\textbf{Scene}] \index{Scene} The basic unit of narrative play, covering a specific situation or conflict (Some Time to Significant Time). Resolves a particular question or challenge.
\item[\textbf{Player Turn (Beat)}] \index{Turn} An individual player's action within a scene: Declare action $\rightarrow$ GM sets position $\rightarrow$ roll $\rightarrow$ resolve outcome $\rightarrow$ manage consequences.
\item[\textbf{Round}] \index{Round} Simultaneous or near-simultaneous actions within a scene (primarily for combat), representing a few seconds of real time.
\item[\textbf{Session}] \index{Session} One complete game session (typically 3–6 hours), containing 2–4 major scenes and resolving significant narrative progress.
\item[\textbf{Downtime}] \index{Downtime} The narrative time between scenes, used for recovery, advancement, and off-screen activities. Measured in days, weeks, or months depending on fiction.
\item[\textbf{Campaign}] \index{Campaign} Entire story arc (6–20+ sessions) with major character development and lasting consequences.
\end{description}

\section{Movement and Positioning}
\index{Movement}\index{Position}

Space is tracked with \textbf{range bands} and \textbf{Position}.

\subsection*{Range Bands}
\index{Range}
\begin{description}
\item[\textbf{Close}] \index{Range!Close} Touching distance: grapples, knife-work, hand on a relic.
\item[\textbf{Near}] \index{Range!Near} Same room/yard/deck; a rush away.
\item[\textbf{Far}] \index{Range!Far} Same site but distant; requires route or time to reach.
\item[\textbf{Absent}] \index{Range!Absent} Off-screen; requires scene change or significant effort to interact.
\end{description}

\subsection*{Movement Actions}
\begin{itemize}
\item \textbf{Move}: Shift one range band as a \emph{beat}.
\item \textbf{Dash}: Shift two bands as your full action (terrain may require a roll).
\item \textbf{Melee Flag}: Mark when two parties are in Near range and directly engaged in combat.
\end{itemize}

\subsection*{Position States}
\index{Position!Dominant}\index{Position!Controlled}\index{Position!Desperate}
\begin{description}
\item[\textbf{Dominant}] \index{Position!Dominant} You have cover, leverage, or ritual footing. Failure still leaves options.
\item[\textbf{Controlled}] \index{Position!Controlled} Standard case: exposed lanes, rivals near, watchful eyes. Failure has teeth, but not ruin.
\item[\textbf{Desperate}] \index{Position!Desperate} Bad ground, bad odds, bad timing. Failure is severe; success may bring extra XP.
\end{description}

\paragraph{Position Shifting:}
\begin{itemize}
\item GM can spend \textbf{1 SB} to worsen Position by one step.
\item Player can spend \textbf{1 Boon} to improve Position by one step (once per action).
\item Narrative triggers (flanking, reinforcements, etc.) can shift Position without cost.
\end{itemize}

\section{Travel Framework}
\index{Travel}\index{Travel Framework}

Travel abstracts distance into \emph{legs} with tension and color rather than miles and meal counts. Each leg has a \textbf{Travel Clock} and draws on a \textbf{regional deck} to seed fiction.

\subsection*{Travel Process}
\begin{enumerate}
\item \textbf{Set the Leg:} Name origin and destination; start a Travel Clock (4-10 segments based on difficulty).
\item \textbf{Draw Prompts:} Draw up to one card from each suit to establish terrain, people, pressures, and leverage.
\item \textbf{Assign Roles:} Players take on travel roles (Guide, Scout, Quartermaster, Watch) to contribute actions.
\item \textbf{Play the Leg:} Players take actions to advance the clock or mitigate complications. GM spends SB from rolls showing \textbf{1}s to introduce hazards.
\item \textbf{Resolve:} When the clock fills, you arrive—changed by the journey.
\end{enumerate}

\subsection*{Using Assets and Followers During Travel}
\begin{itemize}
\item \textbf{Assets}: Spend 1 Boon to activate an asset for dramatic effect during travel (reveal hidden path, call for emergency aid, etc.).
\item \textbf{Followers}: Assign followers to travel roles for bonuses. A Cap 3 Scout follower adds +3 to navigation rolls, for example.
\item \textbf{Independent Actions}: Once per travel leg, a follower can take an independent action (scout ahead, secure supplies, etc.) at the cost of Exposure or Harm.
\item \textbf{Off-Screen Solutions}: High-Cap followers (4-5) can solve significant travel problems once per downtime, but generate 1 SB for the party.
\end{itemize}

\subsection*{Regional Travel Decks}
\index{Travel!regional decks}
Each major region has a themed prompt list or card table (see \S\ref{ch:deck-generators}):
\begin{description}
\item[Viterra] \index{Viterra} Queendoom, Fen causeways, dike-brotherhoods, crown law.
\item[Acasia] \index{Acasia} Fallen Province, Border-lace titles, ruined towers, clan tempers.
\item[Ecktoria] \index{Ecktoria} City & Province, Imperial roads, precinct gates, temple schedules.
\item[Ubral] \index{Ubral} Kingdom, Stone passes, toll-cloisters, ghosted fields.
\item[Kahfagia] \index{Kahfagia} Thalassocracy, Current maps, pilot-mirrors, storm lanes.
\item[Aelinnel] \index{Aelinnel} Realm of Aevrossa, Mist paths, bell-mounds, spirit ways.
\end{description}

\subsection*{Travel Complications}
\begin{itemize}
\item \textbf{Hazards}: Weather, terrain challenges, wildlife encounters.
\item \textbf{Social}: Border checks, local politics, cultural misunderstandings.
\item \textbf{Supplies}: Food shortages, equipment failure, resource management.
\item \textbf{Pursuit}: Being followed, hunted, or racing against time.
\end{itemize}

\section{Narrative Time}
\index{Time!narrative}
Time is measured by \emph{importance} rather than duration.
\begin{description}
\item[\textbf{A Moment}] \index{Time!Moment} A glance, a strike, a whisper over a law-stone.
\item[\textbf{Some Time}] \index{Time!Some Time} A skirmish, a negotiation, a careful climb.
\item[\textbf{Significant Time}] \index{Time!Significant Time} Hours of march, rites, audits, stakeouts.
\item[\textbf{Days}] \index{Time!Days} Drills, recoveries, research, roadwork.
\end{description}

\section{Social Interactions}
\index{Social}\index{Culture}
Social scenes use the same engine with \textbf{cultural color}.

\subsection*{Cultural Skill Emphases}
\index{Social!culture}
\begin{description}
\item[\textbf{Viterra}] \index{Viterra!social} Rapport with parishes; Sway for markets; Command under writ.
\item[\textbf{Acasia}] \index{Acasia!social} Rapport for kin-bridges; Command with banner-rights; Deceive risks honor clocks.
\item[\textbf{Ecktoria}] \index{Ecktoria!social} Sway in salons; Deceive at court; Perform in temple fora.
\item[\textbf{Kahfagia}] \index{Kahfagia!social} Rapport aboard; Sway at piers; Command on a storming deck.
\end{description}

\subsection*{Social Stakes \& Clocks}
\index{Social!stakes}\index{Clocks!social}
\begin{itemize}
\item \textbf{Alliance Clock (Viterra):} Parishes and guilds come to your side.
\item \textbf{Honor Clock (Acasia):} Feasts, oaths, wyrd—trust builds (or frays).
\item \textbf{Bureau Clock (Ecktoria):} Stamps, seals, approvals—delay is pressure.
\item \textbf{Trust Clock (Kahfagia):} Pilots and crews extend favors and routes.
\end{itemize}

\section{Supply and Resources}
\label{world:supply}
\index{Supplies}\index{Supply Clock}
Track scarcity with a \textbf{Supply Clock} shared by the party's expedition.
\begin{center}
\begin{tabular}{cl}
\toprule
\textbf{Segments} & \textbf{State \& Effects} \\
\midrule
0 (Full) & Well-provisioned; no penalty. \\
2 (Low) & Minor frictions; -1 to resource checks. \\
3 (Dangerous) & Each PC gains \emph{Fatigue 1}. \\
4 (Empty) & Severe penalties; desperate measures. \\
\bottomrule
\end{tabular}
\end{center}

subsection{Using Tags}
Tags only function when \emph{printed on a Talent, an Ability, or as the result of a Spell/Rite}. 
They do nothing on their own. 
Unless specified otherwise, \textbf{DV is set by fiction}, and duration defaults to \textbf{Scene}. 
When a Tag affects an \textbf{Outsider}, use the unified rules in \S\ref{sec:tags-outsiders}.

\paragraph{Example: Disabling a Magical Trap (\WARD).}
A magical trap is represented by the \WARD tag. Its Difficulty Value (DV) to disable is usually the same DV used to cast or sustain the ward.

\textbf{Approaches (examples).}
\begin{itemize}
  \item \textbf{Wits + Arcana}: analyze and unravel the binding.
  \item \textbf{Wits + Tinker}: mechanically bypass the trigger/anchor.
  \item \textbf{Body + Agility}: carefully avoid or physically disarm the trigger.
\end{itemize}

\textbf{Position sets DV (Ladder).}
\begin{itemize}
  \item \textbf{Dominant} (ample time, proper tools, safe access): \textbf{DV 2}.
  \item \textbf{Controlled} (under pressure, limited time, partial access): \textbf{DV 3}.
  \item \textbf{Desperate} (activating/compromised access): \textbf{DV 4–5+} (GM sets by threat).
\end{itemize}

\textbf{Talents \& Tools.} A relevant Talent or Tool may unlock an alternate approach or grant +1d / +1 Effect; proper tools may improve Position at the GM’s discretion.

\textbf{Outcomes.}
\begin{itemize}
  \item \textbf{Success}: the \WARD is suppressed, bypassed, or its trigger safely disarmed.
  \item \textbf{Partial}: the \WARD is affected but \emph{unstable} or a new complication appears (GM may start/advance a related clock or spend SB for an intrusion).
  \item \textbf{Miss}: the \WARD remains and may trigger; generate SB as complications (backlash, mechanism damage, alarm to the creator, etc.).
\end{itemize}

\section{Engaging the World—Player Actions}
\index{Procedures!world}
\begin{itemize}
\item \textbf{Scout \& Signal:} A follower can make the next travel action \emph{Dominant} (mark Exposure or Harm 1 on them).
\item \textbf{Local Color:} Briefly state what locals notice about you; GM offers a small fictional edge \emph{or} a tempting clock—choose.
\item \textbf{Mark the Map:} On arrival, declare one change to the fiction (new ford, patron's shrine, toll-skip). GM may attach a minor clock as cost.
\item \textbf{Asset Activation:} Spend 1 Boon to activate an asset dramatically during a scene.
\item \textbf{Follower Assistance:} Have a follower assist your actions for bonus dice (max +3 from all sources).
\end{itemize}

\section{Summary}
\index{World interaction!summary}
The world has opinions. Movement is clocks and color, position rises and sinks with weather and words, and every suit you draw speaks in a regional accent. Ask the land for a favor—then pay it back on the road.

\paragraph{Remember:} Every interaction with the world is an opportunity. Use your assets, deploy your followers, and engage with the setting actively. The world responds to your choices, and every journey changes both you and the places you pass through.

\clearpage

\chapter{Lore in Brief}

\section*{1. Utar (The Utaran Imperium \& Successor Northlands)}

\subsection*{Overview}
Once a vast empire, Utar fractured into a patchwork of successor states. Now, the north is a land of law, memory, and fading glory.

\subsection*{Key Regions}
\begin{itemize}[leftmargin=*]
    \item \textbf{Ecktoria} – The marble heart of the old empire, now a relic-state clinging to forms.
    \item \textbf{Acasia} – A broken realm of warlords and the cosmopolitan city of Silkstrand.
    \item \textbf{Vhasia} – The fractured sun; a kingdom in name only, ruled by ducal houses.
    \item \textbf{Viterra} – The last true kingdom, orderly and martial.
    \item \textbf{Thepyrgos} – A city of stairs, learning, and high-elf enclaves.
\end{itemize}

\section*{2. Kahfagia — The Empire of Wakes and Storm-Flags}

\subsection*{Overview}
A human maritime oligarchy that straddles the Titan’s Throat. Kahfagia is a realm of trade, kraken-priests, and privateers.

\subsection*{Key Features}
\begin{itemize}[leftmargin=*]
    \item Controls key sea lanes.
    \item Cities like \textbf{Kassamira} and \textbf{Stormspire} are centers of naval power.
    \item Lantern-law jurisdiction shifts with the tide.
\end{itemize}

\section*{3. Theona — The Marsh Crown}

\subsection*{Overview}
A trio of green isles pledged to Viterra, but only when it suits them. Theona is a place of quiet customs, feuds, and the ``No Ninth'' taboo.

\subsection*{Key Features}
\begin{itemize}[leftmargin=*]
    \item Peat bogs, hedged fields, and small marble outcrops.
    \item Taboo of the Ninth: omissions, unspoken names, missing steps.
    \item Known for coracle fleets and the Moot Hill.
\end{itemize}

\section*{4. The Mistlands — Bells, Salt, and Breath}

\subsection*{Overview}
A fog-drenched breadbasket guarded by dwarven law and ancient bell-lines. A frontier between the Direwood and the civilized north.

\subsection*{Key Features}
\begin{itemize}[leftmargin=*]
    \item Bell-line levees and reed-maze causeways.
    \item Warded by the Weeping Gate and Pall Watchtowers.
    \item Home to the Protectorate and the Legate of the Mists.
\end{itemize}

\section*{5. Valewood — The Forest That Remembers You Wrong}

\subsection*{Overview}
A living forest that shifts and forgets. The Valewood is an empire that never truly fell, full of fey remnants and relic-logic.

\subsection*{Key Features}
\begin{itemize}[leftmargin=*]
    \item Phasing ruins, star-roads, and sentient trees.
    \item Inhabited by Lethai-ar (wood elves) and the fae-like Green Neighbors.
    \item Empire echoes: ancient laws and ruins still function.
\end{itemize}

\section*{6. Ykrul — Storm on the Steppe}

\subsection*{Overview}
Nomadic clans of horse-riders and raiders. The Ykrul live by omen, pasture, and blood-tanistry.

\subsection*{Key Features}
\begin{itemize}[leftmargin=*]
    \item Seasonal migrations and clan confederations.
    \item Shamans and the Faith of the Open Sky.
    \item Known for dragon boats and the Kurultai councils.
\end{itemize}

\section*{7. Zakov — Salt \& Serpent}

\subsection*{Overview}
A pirate haven and crime nexus on a Dolmis island. Controlled by the Seven Guilds, Zakov is lawless but ritualized.

\subsection*{Key Features}
\begin{itemize}[leftmargin=*]
    \item Cities built from wrecks and stolen goods.
    \item The Pirate Syndicate and the Salt Prince rule in shadow.
    \item Known for the Serpent’s Spine reef and the Crimson Docks.
\end{itemize}

\section*{8. Ubral — The Stone Between Spears}

\subsection*{Overview}
Highland clans, cairns, and iron oaths. Ubral is a land of shepherds, reivers, and hill-forts caught between Viterra and Vhasia.

\subsection*{Key Features}
\begin{itemize}[leftmargin=*]
    \item Hill dwarves in Khaz-Vurim.
    \item Guest-right tokens and feud-brokering.
    \item Known for the Pass of Ashes and Dun Caerloch.
\end{itemize}

\section*{9. Linn — Skerries \& Storm-Oaths}

\subsection*{Overview}
Southernmost Linnic tribes. A maritime culture of fjords, skerries, and dragon boats.

\subsection*{Key Features}
\begin{itemize}[leftmargin=*]
    \item Raiders, fur-hunters, and riverfolk.
    \item Thing-holm and the Sea-Queen’s court.
    \item Known for the Whale-road and the Volva of the Mist.
\end{itemize}

\section*{10. Aelinnel — Stone, Bough, and Bright Things (Averossa)}

\subsection*{Overview}
Gnomish homeland of Aevrossa, stone spires and moonlit groves. Averossa is a realm of charms, geasa, and tide-gates.

\subsection*{Key Features}
\begin{itemize}[leftmargin=*]
    \item Tide-rift steps and dolmen stairs.
    \item Hedge-witches and oath-carvers.
    \item Known for the Green Gate and the Thorn Court.
\end{itemize}

\section*{11. Aelaerem — Hearth \& Hollow (Amedell)}

\subsection*{Overview}
Halfling downs and orchards. Amedell is a quiet land of hearth-fires, orchards, and folklore.

\subsection*{Key Features}
\begin{itemize}[leftmargin=*]
    \item Moot Oaks, hedge-witches, and mummers.
    \item Red thread motifs and quiet bells.
    \item Known for the Apple-Matron and the Pale Shepherd.
\end{itemize}

\section*{12. Aeler — Crowns \& Under-Vaults}

\subsection*{Overview}
Dwarven mountain holds beneath the Aelerian peaks. Aeler is a realm of stone, breath, and ancient law.

\subsection*{Key Features}
\begin{itemize}[leftmargin=*]
    \item Vaultmouth gates and under-roads.
    \item Spirit Shield Warriors and the High King Beneath the Peaks.
    \item Known for Khaz-Vurim and the Vault-Queen.
\end{itemize}

\section*{13. Black Banners — Condotta \& Crowns}

\subsection*{Overview}
Mercenary lands and war camps. Black Banners is a frontier of shifting loyalties, condotta contracts, and frozen battlefields.

\subsection*{Key Features}
\begin{itemize}[leftmargin=*]
    \item Condotta companies and the Bannerless One.
    \item War-camps and siege ruins.
    \item Known for the Singing Wastes and the Bone Fields.
\end{itemize}

\section*{14. Vilikari — Laurels \& Longhouses}

\subsection*{Overview}
A federated frontier mixing Utaran law with barbarian custom. Vilikari is a land of two laws: wolf and eagle.

\subsection*{Key Features}
\begin{itemize}[leftmargin=*]
    \item Foedus Stone and mixed courts.
    \item March towns and villa-forts.
    \item Known for the Queen of the Marches and the Dux’s Palace.
\end{itemize}

\section*{15. The Wilds — Roads, Ruins, and Weather}

\subsection*{Overview}
Untamed lands that shift by biome. The Wilds are a reskin palette for any terrain—forest, desert, tundra, or coast.

\subsection*{Key Features}
\begin{itemize}[leftmargin=*]
    \item Crossing points, shelter hollows, and old road traces.
    \item Forager children and roving war-bands.
    \item Known for lingering omens and elemental threats.
\end{itemize}

\section*{16. Tulkani — Road-Kin of the Ember Line}

\subsection*{Overview}
Nomadic clans of painted wagons and braided oaths. Once rooted, now scattered, the Tulkani call the road their homeland.

\subsection*{Key Features}
\begin{itemize}[leftmargin=*]
    \item Wagon rings and fire-cults.
    \item Kuva of the Hearth-Road and the Family of the Raven Road.
    \item Known for songs for the living and bargains with dusk.
\end{itemize}

\section*{17. Ikari — First Plough, First Oath}

\subsection*{Overview}
Native tillers and smiths of the northern continent. The Ikari are a people of hearth and edge.

\subsection*{Key Features}
\begin{itemize}[leftmargin=*]
    \item Tribes like Kreki (fishermen) and Smeinnoii (smiths).
    \item Ancestral fires and law-keepers.
    \item Known for seasonal raids and the Ondriti code.
\end{itemize}

\section*{18. Midh Adkaz — Where War Became a Market}

\subsection*{Overview}
A frontier camp turned crossroads city. Midh Adkaz is a place where oaths are traded like grain.

\subsection*{Key Features}
\begin{itemize}[leftmargin=*]
    \item The Red Ditch and Stakefield fairground.
    \item The Six Hands council.
    \item Known for the Boar Gate and River Gate.
\end{itemize}

\section*{19. Haayr Peninsula — Anvils Between Two Seas}

\subsection*{Overview}
Mountain tongues and broken coasts. The Haayr Peninsula is a strategic chokepoint between seas.

\subsection*{Key Features}
\begin{itemize}[leftmargin=*]
    \item Spine of Haayr mountains and limestone passes.
    \item Cities like Khar-Myra and Theressos.
    \item Known for the Hook Road and Pass of Ten Towers.
\end{itemize}

\section*{20. Dhahara — Monsoon of Empires}

\subsection*{Overview}
A land of oases, monsoons, and marching armies. Dhahara is a frontier where caravans meet fleets.

\subsection*{Key Features}
\begin{itemize}[leftmargin=*]
    \item The Himdal Marches and Jade Oases.
    \item Cities like Sarvash and Thalara.
    \item Known for the Monsoon Bells and Incense Belt.
\end{itemize}

\section*{21. Oshiira — The Ledger Empire}

\subsection*{Overview}
A confederation of canals and ledgers. Oshiira is a realm of precision, irrigation, and the Spirit of the Long Sorrow.

\subsection*{Key Features}
\begin{itemize}[leftmargin=*]
    \item Canal webs and numbered lines.
    \item The Mbari-style Senate and Prefects.
    \item Known for the Crimson Basin and Sekogo.
\end{itemize}

\section*{22. Sekogo — Where the Roads Meet the Tide}

\subsection*{Overview}
A crossroads of river and sea. Sekogo is a land of grove-masters, lagoon wardens, and the Tide Ledger.

\subsection*{Key Features}
\begin{itemize}[leftmargin=*]
    \item Mbaro Quays and the Brasswater Row.
    \item Quay Syndicates and River Pilots.
    \item Known for the Circle of Unkwa and the Spice Shade.
\end{itemize}

\section*{23. Taharka — Monsoon Crown, Terrace Throne}

\subsection*{Overview}
A highland kingdom of canals, convoys, and coins. Taharka is a realm of water and stone.

\subsection*{Key Features}
\begin{itemize}[leftmargin=*]
    \item Mkusaro Highlands and the Canal Collegium.
    \item Siatwe, the capital city of spiraling bazaars.
    \item Known for the Mint of Siatwe and meltwater courts.
\end{itemize}

\section*{24. Ameria — Between Bay and Throat}

\subsection*{Overview}
A divided realm between Kahfagia and the Titan’s Throat. Ameria is a buffer of royal forms and neutral regencies.

\subsection*{Key Features}
\begin{itemize}[leftmargin=*]
    \item Shoreless Bay and Throatward Ports.
    \item Khol-Amar and Cape Verdan.
    \item Known for Consular Row and the Neutral Regency.
\end{itemize}

\section*{25. Ngomebe — Stone Men of the Moving Cities}

\subsection*{Overview}
Ironworkers and wall-builders of Akilan. Ngomebe is a realm of walking cities and iron mothers.

\subsection*{Key Features}
\begin{itemize}[leftmargin=*]
    \item Mkusaro ridges and Ekale Spurs.
    \item Khazembo and Duma-Sete.
    \item Known for the Mason-Kin Houses and Gate-Voices.
\end{itemize}

\section*{26. Ashaan — Gem of the Sea, Shadow on the River}

\subsection*{Overview}
A fallen slaver-empire. Ashaan is a land of sorcery, assassins, and the Three Sisters.

\subsection*{Key Features}
\begin{itemize}[leftmargin=*]
    \item The Veiled, Helmed, and Masked Sisters.
    \item The Esoti and the Black Hand.
    \item Known for Galanina and the Shadows of Ashaan.
\end{itemize}

\section*{27. Sihai — The Central Kingdom, The Ordered Land}

\subsection*{Overview}
An immense, ancient empire of rigid hierarchy. Sihai is a land of the Son of Heaven and the Mandate of Heaven.

\subsection*{Key Features}
\begin{itemize}[leftmargin=*]
    \item The Sihon River Basin and the Himadri Mountains.
    \item The Bureaucracy and the Warrior Monks.
    \item Known for the Imperial Army and the Hintara Ocean Coast.
\end{itemize}

\section*{28. Nihori — The Isles of the Dawn Spirit}

\subsection*{Overview}
A storm-wracked archipelago of clans, spirits, and samurai. Nihori is a realm of divine emperors and the code of Bushidō.

\subsection*{Key Features}
\begin{itemize}[leftmargin=*]
    \item The Shōgun and the Daimyō.
    \item The Samurai and the Shinobi.
    \item Known for the Inland Sea and the Fire-Mountains.
\end{itemize}

\section*{29. Ayokha — The Monsoon Throne, The River of Heaven}

\subsection*{Overview}
A sprawling jungle kingdom of temples and monsoons. Ayokha is a land of the Devaraja and celestial bureaucracy.

\subsection*{Key Features}
\begin{itemize}[leftmargin=*]
    \item The Sona River and the Jade Coast.
    \item War Elephants and the Royal Guard.
    \item Known for the Inner Jungle and monsoon-riding junks.
\end{itemize}

\section*{30. Alberriden Sea — Cold Mirror of the North}

\subsection*{Overview}
A brackish, iron-skied sea. The Alberriden is a mirror of the north, full of mist, mountain winds, and root-cellars.

\subsection*{Key Features}
\begin{itemize}[leftmargin=*]
    \item The Yrolka Mouth and Brack Marsh Rim.
    \item Valewood Edge and Mistland Coast.
    \item Known for the Haravoa Shadow and Dwarf-cut harbors.
\end{itemize}

\section*{31. The Crimson Basin}

\subsection*{Overview}
A vast rainforest contested between wood elves and Oshiiran settlers. The Crimson Basin is a realm of rivers, treaties, and farm-forests.

\subsection*{Key Features}
\begin{itemize}[leftmargin=*]
    \item The Enjwe Trunk and the Heartwood.
    \item Wood Elves and Oshiiran Basinfolk.
    \item Known for the Treaty of Three Waters and canoe-trains.
\end{itemize}

\section*{32. Linnstad}

\subsection*{Overview}
A northern city-state and fur-trade hub. Linnstad is the southernmost outpost of the Linnic tribes.

\subsection*{Key Features}
\begin{itemize}[leftmargin=*]
    \item Claim to inventing dragon boats.
    \item Fierce, pragmatic culture.
    \item Known for the fur-trade and the Yrolka Mouth.
\end{itemize}

\section*{33. Rabelle}

\subsection*{Overview}
A mountain city of ore and gem wealth. Rabelle is a realm of red-haired Rabellans and giant ancestry.

\subsection*{Key Features}
\begin{itemize}[leftmargin=*]
    \item Fine metalwork and dwarven trade.
    \item Treacherous pass toward the Mistlands.
    \item Known for blunt pride and mountain identity.
\end{itemize}

\section*{34. Northpass}

\subsection*{Overview}
A frontier town and gateway to the Mistlands. Northpass is a bare-bones outpost of vigilance.

\subsection*{Key Features}
\begin{itemize}[leftmargin=*]
    \item The Ermine Inn and caravan stops.
    \item Carefully monitored by dwarves.
    \item Known for rough edges and the Mistlands beyond.
\end{itemize}

\section*{35. The Northern \& Eastern City-States}

\subsection*{Overview}
Independent city-states, mountain strongholds, and coastal havens. These are bound by trade, survival, and necessity.

\subsection*{Key Features}
\begin{itemize}[leftmargin=*]
    \item Independence and cultural blend.
    \item Dangerous, decadent, or dishonorable—but everyone trades.
    \item Known for the dangers and rewards of the frontier.
\end{itemize}

\subsection*{Amaranthine Coastway}
Kahfagia $\rightarrow$ Ecktoria $\rightarrow$ Acasia $\rightarrow$ Marcott (Vhasia) $\rightarrow$ Fairport (Viterra).
 