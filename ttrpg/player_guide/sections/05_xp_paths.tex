\chapter{Experience Paths and Character Building}
\label{ch:xp-paths}

How you spend your \textbf{Experience Points (XP)}\index{XP} shapes more than your
numbers on a sheet—it defines your character's \emph{career}, their growth
over time, and the kind of stories they invite into play.

This chapter focuses on:
\begin{itemize}
  \item turning a loose idea into a clear \textbf{character concept},
  \item mapping that concept into \textbf{starting choices} that fit the campaign rules,
  \item choosing a \textbf{career path} (how you plan to grow),
  \item and understanding how your character can evolve across \textbf{early, mid, and late} game.
\end{itemize}

All examples in this chapter are \emph{legal starting builds} under the standard
creation rules, and are written to be directly usable by players.

% ============================================================
% FROM CONCEPT TO CAREER
% ============================================================

\section{From Concept to Career}
\index{concept mapping}\index{career planning}

Before you spend a single XP, start with the story you want to tell.

\subsection*{Step 1: Concept Seed}

Begin with a simple statement of who your character is:
\begin{itemize}
  \item ``Stoic bodyguard who will take the hit for anyone else.''
  \item ``Nervy courier who knows every back alley in the city.''
  \item ``Soft-spoken merchant who pulls strings behind the scenes.''
\end{itemize}

This is your \textbf{concept seed}. Everything else grows from here.

\subsection*{Step 2: Role in the Party}

Decide what you want to \emph{do at the table}:
\begin{itemize}
  \item Take hits and lock down threats?
  \item Spot danger and solve practical problems?
  \item Talk through conflicts and manage social fallout?
  \item Create tools, magic, or plans that enable everyone else?
\end{itemize}

Your role helps you choose a \textbf{Primary Attribute} and 2--3 \textbf{Key Skills}
(see Chapter~\ref{chap:advancement}).

\subsection*{Step 3: Career Theme}

Ask yourself:
\begin{itemize}
  \item Is this a story of \emph{self-mastery}? (training, discipline, pushing limits)
  \item A story of \emph{connections}? (alliances, obligations, webs of influence)
  \item A story of \emph{balance}? (learning to juggle many demands at once)
\end{itemize}

Your answer naturally points you toward one of the three \textbf{XP paths} in
this chapter.

\subsection*{Step 4: Long Arc in One Sentence}

Write a single sentence for where you imagine your character at high XP:
\begin{itemize}
  \item ``I want to become the greatest duelist in the realm.''
  \item ``I want to map the unknown and bring others safely home.''
  \item ``I want to run the network everyone relies on but no one sees.''
\end{itemize}

You do not have to stick to this forever, but it gives you a \emph{direction}
for your early choices.

\begin{tcolorbox}[colback=blue!5!white,colframe=blue!75!black,title=Character Career Checklist,fonttitle=\bfseries]
\textbf{Before spending XP, decide:}
\begin{enumerate}
  \item Your \textbf{concept seed} (one sentence).
  \item Your \textbf{role in the party} (what you do most).
  \item Your preferred \textbf{career theme}: self, balance, or influence.
  \item A rough \textbf{long arc} (where you want to end up).
\end{enumerate}
Start simple. You can always refine these as the campaign unfolds.
\end{tcolorbox}

% ============================================================
% MAPPING CONCEPT TO MECHANICS
% ============================================================

\section{Mapping Concept to Mechanics}
\index{concept mapping!to mechanics}

Once you have a concept, you turn it into numbers. Use this as a guideline:

\begin{enumerate}
  \item Pick a \textbf{Primary Attribute} that supports your role
    (Body, Wits, Spirit, or Presence).
  \item Choose 2--3 \textbf{Key Skills} that you want to be noticeably good at.
  \item Choose a \textbf{Career Path} (Personal, Balanced, Influencer) that matches
        your theme.
  \item Use the starting XP budget in Section~\ref{ch:xp-paths:starting} to buy:
    \begin{itemize}
      \item Attribute steps for your primary and possibly secondary stats.
      \item Skill steps for your key and supporting skills.
      \item Optional resources or special abilities, if your path calls for them.
    \end{itemize}
\end{enumerate}

Throughout this chapter, each example shows how a concept becomes a legal build.

% ============================================================
% THREE CAREER PATHS
% ============================================================

\section{Three Career Paths}
\index{advancement paths}\index{career paths}

There are three broad approaches to character development, each representing a
different \textbf{career philosophy}:

\begin{description}
\item[Personal Path] Focus on personal mastery and self-improvement.
\item[Balanced Path] Mix personal growth with resources and influence.
\item[Influencer Path] Build networks, assets, and strategic power.
\end{description}

These paths are not rigid classes. They are \emph{patterns} that help you
decide where most of your XP will go, especially in the early game.

% ============================================================
% PATH 1: PERSONAL DEVELOPMENT
% ============================================================

\section{Path 1: Personal Development}
\index{personal path}

The \textbf{Personal Path} centers your character's career on what \emph{they}
can do: their body, mind, and honed skills. This is the path of the duelist,
the ascetic, the elite scout, the master healer.

\subsection*{Typical Investment}
\begin{itemize}
\item 70--90\% Personal improvement (Attributes and Skills)
\item 0--10\% Resources and assets
\item 0--20\% Special abilities
\end{itemize}

\subsection*{Strengths}
\begin{itemize}
\item Reliable in direct challenges and combat
\item Minimal upkeep or management required
\item Resilient to loss of external resources
\item Consistent performance in spotlight moments
\end{itemize}

\subsection*{Weaknesses}
\begin{itemize}
\item Limited influence in social or strategic scenes
\item May struggle with problems requiring networks
\item Less capable in logistics or large-scale operations
\item Dependent on personal presence for all solutions
\end{itemize}

\subsection*{Career View}

On this path, your ``career ladder'' is mostly about pushing numbers related to
\textbf{you}: raising Attributes, deepening Skills, and picking a few defining
abilities. Your late-game questions are:
\begin{itemize}
  \item ``What does a legendary version of this look like?''
  \item ``What scenes prove that I have truly mastered my craft?''
\end{itemize}

\subsection*{Build Example: The Duelist (Legal Start)}

\textbf{Concept Seed:} ``A proud blade whose entire life is training and challenge.''

\medskip

\noindent\textbf{Total XP: 30} \quad (\textbf{34} with +4 from Bonds/Complications; see \S\ref{ch:xp-paths:starting})

\begin{itemize}
\item \textbf{Attributes}\index{attributes}: Body 3, Wits 2, Spirit 1, Presence 1
  \begin{itemize}
  \item Costs (Attributes cost \emph{new rating} $\times 3$):  
        Body 1$\to$2 (6), 2$\to$3 (9) = \textbf{15};  
        Wits 1$\to$2 (6) = \textbf{6};  
        Spirit/Presence remain 1 = \textbf{0}.  
        \emph{Subtotal: 21 XP}
  \end{itemize}
\item \textbf{Skills}\index{skills}: Melee 2, Athletics 1
  \begin{itemize}
  \item Costs (Skills cost \emph{new level} $\times 2$):  
        Melee 0$\to$1 (2), 1$\to$2 (4) = \textbf{6};  
        Athletics 0$\to$1 = \textbf{2}.  
        \emph{Subtotal: 8 XP}
  \end{itemize}
\item \textbf{Totals}: 21 + 8 = \textbf{29 XP}. Bank \textbf{1 XP}.
\item \textbf{With +4 XP (Bonds/Complications)}:  
  Add \emph{Perception 0$\to$1} (2) and spend banked 1 XP on \emph{Stealth 0$\to$1} (2),  
  or instead take \emph{Perception 0$\to$1} (2) and \emph{Sway 0$\to$1} (2) for broader utility.  
  \emph{Cap: 34 XP}.
\end{itemize}

\noindent\textbf{Career Hook:} Early scenes show duels and drills. Mid game focuses on
notorious opponents. Late game tells the story of a blade whose name carries
across borders.

% ============================================================
% PATH 2: BALANCED APPROACH
% ============================================================

\section{Path 2: Balanced Approach}
\index{balanced path}

The \textbf{Balanced Path} mixes personal capability with supportive resources.
This is the path of the scout-captain, the field medic with a small clinic, the
problem-solver who always seems to have \emph{something} ready.

\subsection*{Typical Investment}
\begin{itemize}
\item 50--65\% Personal improvement
\item 15--25\% Resources and assets
\item 15--25\% Special abilities
\end{itemize}

\subsection*{Strengths}
\begin{itemize}
\item Adaptable to diverse situations
\item Handles both direct and indirect challenges
\item Excellent supporting role for the group
\item Moderate risk profile
\end{itemize}

\subsection*{Weaknesses}
\begin{itemize}
\item Not exceptional in any single area
\item Requires management of resources
\item Moderate upkeep demands
\item Can be outshone by focused specialists
\end{itemize}

\subsection*{Career View}

Balanced characters often have careers that feel like juggling:
\begin{itemize}
  \item they add contacts, gear, or small bases \emph{and}
  \item keep raising the core skills that let them use those tools well.
\end{itemize}

Their late-game questions are:
\begin{itemize}
  \item ``What does my little network or operation look like at full size?''
  \item ``How do my personal skills and resources fit together as a whole?'' 
\end{itemize}

\subsection*{Build Example: The Scout (Legal Start)}

\textbf{Concept Seed:} ``A pathfinder who knows how to get in, get out, and get others home.''

\medskip

\noindent\textbf{Total XP: 30} \quad (\textbf{34} with +4 from Bonds/Complications)

\begin{itemize}
\item \textbf{Attributes}: Wits 2, Body 2, Spirit 1, Presence 1
  \begin{itemize}
  \item Costs: Wits 1$\to$2 (6), Body 1$\to$2 (6) = \textbf{12 XP}
  \end{itemize}
\item \textbf{Skills}: Survival 2, Perception 1, Stealth 1
  \begin{itemize}
  \item Costs: Survival 0$\to$1 (2), 1$\to$2 (4) = \textbf{6};  
        Perception 0$\to$1 \textbf{2};  
        Stealth 0$\to$1 \textbf{2}.  
        \emph{Subtotal: 10 XP}
  \end{itemize}
\item \textbf{Resources}\index{resources}: \emph{Minor equipment cache} (camp gear, maps, signal kit) = \textbf{4 XP}
\item \textbf{Special Abilities}\index{special abilities}: \emph{Wilderness Lore} (broad travel benefits) = \textbf{4 XP}
\item \textbf{Totals}: 12 + 10 + 4 + 4 = \textbf{30 XP}.
\item \textbf{With +4 XP}: add \emph{Perception 1$\to$2} (+4) \emph{or} take a \emph{trained hawk companion} (Minor Resource, 4 XP).
\end{itemize}

\noindent\textbf{Career Hook:} Early stories show small journeys. Mid game explores
dangerous routes and rescue missions. Late game might see the scout running
expeditions, caravans, or an entire mapping guild.

% ============================================================
% PATH 3: INFLUENCER FOCUS
% ============================================================

\section{Path 3: Influencer Focus}
\index{influencer path}

The \textbf{Influencer Path} emphasizes networks, assets, and strategic power.
This is the path of the merchant-prince, the fixer, the spymaster, the noble
with more favors owed than coins in their purse.

\subsection*{Typical Investment}
\begin{itemize}
\item 25--40\% Personal improvement
\item 35--55\% Resources and assets
\item 20--40\% Special abilities
\end{itemize}

\subsection*{Strengths}
\begin{itemize}
\item Strong strategic and social influence
\item Can solve problems indirectly
\item Excellent at planning and preparation
\item Creates opportunities for the whole group
\end{itemize}

\subsection*{Weaknesses}
\begin{itemize}
\item Personally vulnerable in direct confrontations
\item High maintenance requirements
\item Complications can cascade through networks
\item Dependent on external factors
\end{itemize}

\subsection*{Career View}

Influencer careers are about building and surviving \textbf{webs}:
\begin{itemize}
  \item webs of contacts, debts, trade routes, cells, or clients.
\end{itemize}
Late game questions become:
\begin{itemize}
  \item ``How big can this web reasonably grow within the fiction?''
  \item ``What happens when a strand of my web is cut, or pulled tight?''
\end{itemize}

\subsection*{Build Example: The Merchant (Legal Start)}

\textbf{Concept Seed:} ``A friendly shopkeeper whose real power lies in who owes them favors.''

\medskip

\noindent\textbf{Total XP: 30} \quad (\textbf{34} with +4 from Bonds/Complications)

\begin{itemize}
\item \textbf{Attributes}: Presence 2, Wits 2, Spirit 1, Body 1
  \begin{itemize}
  \item Costs: Presence 1$\to$2 (6), Wits 1$\to$2 (6) = \textbf{12 XP}
  \end{itemize}
\item \textbf{Skills}: Sway 2, Deception 1, Lore 1
  \begin{itemize}
  \item Costs: Sway 0$\to$1 (2), 1$\to$2 (4) = \textbf{6};  
        Deception 0$\to$1 \textbf{2};  
        Lore 0$\to$1 \textbf{2}.  
        \emph{Subtotal: 10 XP}
  \end{itemize}
\item \textbf{Resources}: \emph{Standard trading office} (staffed storefront, ledgers, storage) = \textbf{8 XP}
\item \textbf{Totals}: 12 + 10 + 8 = \textbf{30 XP}.
\item \textbf{With +4 XP}: add \emph{Negotiation Mastery} (4 XP general ability) \emph{or} expand to a second \emph{Minor merchant route} (4 XP).
\end{itemize}

\noindent\textbf{Career Hook:} Early scenes are about one shop and a few key clients.
Mid game introduces caravans, deals, and rivals. Late game might place the
merchant at the center of regional politics and supply chains.

% ============================================================
% STARTING GUIDELINES
% ============================================================

\section{Starting Character Guidelines}
\label{ch:xp-paths:starting}
\index{character creation}

\subsection*{Base XP Allocation}
\begin{itemize}
\item \textbf{Standard Starting XP}: \textbf{30} points
\item \textbf{Bonds and Complications}: You may take up to \textbf{two total} from any mix of meaningful \emph{Bonds} (up to 2, +2 XP each) and significant \emph{Complications} (up to 2, +2 XP each), granting maximum \textbf{+4 XP}.
\item \textbf{Maximum Starting XP}: \textbf{34} points
\item \textbf{Complication Effect}: Each unresolved starting Complication adds +1 banked SB to early scenes until cleared.
\end{itemize}

\subsection*{Recommended Starting Ranges}

\begin{center}
\small
\begin{tabular}{ll}
\toprule
\textbf{Category} & \textbf{Recommended XP} \\
\midrule
Primary Attribute & 9--12 XP (rating 3--4) \\
Secondary Attributes & 0--9 XP each (rating 1--3) \\
Key Skills & 4--6 XP each (rating 2--3) \\
Supporting Skills & 2--4 XP each (rating 1--2) \\
Resources & 0--8 XP total \\
Special Abilities & 0--8 XP total \\
\bottomrule
\end{tabular}
\end{center}

\paragraph{Cost Reminders:}
\begin{itemize}
\item \textbf{Attributes}: Each step costs \emph{new rating} $\times 3$ XP (e.g., 1$\to$2 costs 6; 2$\to$3 costs 9).
\item \textbf{Skills}: Each step costs \emph{new level} $\times 2$ XP (e.g., 0$\to$1 costs 2; 1$\to$2 costs 4).
\item \textbf{Resources}: Minor 4 XP; Standard 8 XP; Major 12 XP.
\item \textbf{Special Abilities}: Minor Edge 2 XP; Major Edge 4 XP; Prestige 6+ XP.
\end{itemize}

% ============================================================
% PROGRESSION PLANNING
% ============================================================

\section{Progression Planning}
\index{progression planning}

Think of your character's career in phases. These XP bands are approximate and
assume you began at 30--34 XP.

\subsection*{Early Game (0--40 XP)}

Focus on establishing core capabilities:
\begin{itemize}
\item Reach attribute rating 3 in your primary area.
\item Develop 2--3 key skills to rating 2--3.
\item Acquire basic resources or one special ability (if your path uses them).
\item Clarify your character's niche in the group.
\end{itemize}

\subsection*{Mid Game (41--90 XP)}

Expand and specialize:
\begin{itemize}
\item Increase your primary attribute to 4.
\item Specialize key skills to rating 3--4.
\item Develop supporting capabilities that fit your concept.
\item Build strategic resources or networks if on the Balanced or Influencer path.
\item Acquire signature special abilities that define your style.
\end{itemize}

\subsection*{Late Game (91--150 XP)}

Master your chosen path:
\begin{itemize}
\item Achieve peak attributes (rating 4--5).
\item Master key skills (rating 4--5).
\item Build substantial influence or unique capabilities.
\item Develop advanced special abilities.
\item Consider legacy projects, organizations, or long-term changes to the setting.
\end{itemize}

% ============================================================
% PATH COMBINATIONS
% ============================================================

\section{Path Combination Strategies}
\index{path combinations}

Many players mix elements from different paths as their character's life
unfolds.

\subsection*{Combat Specialist with Resources}
\begin{itemize}
\item Strong personal combat capabilities.
\item Moderate resource investment for support.
\item Good for frontline fighters who need logistical backup.
\item Example: Warrior with a fortified base and loyal troops.
\end{itemize}

\subsection*{Social Character with Personal Skills}
\begin{itemize}
\item Excellent social capabilities.
\item Solid personal skills for self-defense or utility.
\item Good for diplomats who operate independently.
\item Example: Ambassador with combat training and persuasion skills.
\end{itemize}

\subsection*{Technical Expert with Networks}
\begin{itemize}
\item Deep technical or magical expertise.
\item Network of contacts and resources to enable projects.
\item Good for specialists who need supply chains and helpers.
\item Example: Master crafter with supplier network and apprentices.
\end{itemize}

% ============================================================
% RESOURCE MANAGEMENT
% ============================================================

\section{Resource Management}
\index{resource management}

Each path asks for different management styles across your character's career.

\subsection*{Personal Path Management}
\begin{itemize}
\item Minimal upkeep requirements.
\item Focus on equipment, training scenes, and personal growth.
\item Low complexity, high reliability.
\end{itemize}

\subsection*{Balanced Path Management}
\begin{itemize}
\item Moderate upkeep for resources.
\item Relationship maintenance with contacts.
\item Skill development alongside resource management.
\item Balanced time investment between self and assets.
\end{itemize}

\subsection*{Influencer Path Management}
\begin{itemize}
\item Significant upkeep demands.
\item Network maintenance and expansion.
\item Resource allocation and development.
\item Strategic planning and opportunity management.
\end{itemize}

% ============================================================
% RISK ASSESSMENT
% ============================================================

\section{Risk Assessment}
\index{risk assessment}

Each path carries different career risks.

\subsection*{Personal Path Risks}
\begin{itemize}
\item Over-specialization in one area.
\item Vulnerability to problems outside your specialty.
\item Limited diversification later in the game.
\item May become predictable in approach.
\end{itemize}

\subsection*{Balanced Path Risks}
\begin{itemize}
\item Jack-of-all-trades, master of none.
\item Spread too thin across capabilities.
\item Moderate risks in multiple areas.
\item May lack standout, defining capabilities if unfocused.
\end{itemize}

\subsection*{Influencer Path Risks}
\begin{itemize}
\item Networks are vulnerable to attack and betrayal.
\item High maintenance requirements.
\item Cascade failure potential if a key asset falls.
\item Personal safety concerns when the web is targeted.
\end{itemize}

% ============================================================
% GROUP SYNERGY
% ============================================================

\section{Building for Group Synergy}
\index{group synergy}

Your character's career path matters most in context with the rest of the
party.

\subsection*{Complementary Paths}
\begin{itemize}
\item Personal path characters provide reliable combat and specialist solutions.
\item Balanced path characters handle diverse, connective challenges.
\item Influencer path characters create opportunities and resources.
\item Together, a mixed party can cover nearly every type of problem.
\end{itemize}

\subsection*{Redundant Paths}
\begin{itemize}
\item Multiple personal path characters may overlap in combat or role.
\item Multiple influencer path characters may compete for spotlight and assets.
\item Consider diversifying within similar paths: different weapons, domains, or networks.
\item Example: Two combatants, one a duelist, one a shield-focused guardian.
\end{itemize}

% ============================================================
% ADAPTING YOUR PATH
% ============================================================

\section{Adapting Your Path}
\index{path adaptation}

Your chosen path is a starting point, not a contract. You can shift focus as
your character's story changes.

\subsection*{Early Shift (0--40 XP)}
\begin{itemize}
\item Easy to change direction.
\item Minimal sunk cost in any single approach.
\item Good time to experiment with different styles.
\item You can respond quickly to group needs or unexpected story turns.
\end{itemize}

\subsection*{Mid Game Shift (41--90 XP)}
\begin{itemize}
\item Requires more deliberate planning.
\item Some capabilities need to be maintained while others change.
\item You can fill emerging gaps in group capability.
\item Temporary performance dips may occur during the transition.
\end{itemize}

\subsection*{Late Game Shift (91+ XP)}
\begin{itemize}
\item Significant investment in your current path.
\item Major shifts require substantial XP and story justification.
\item Often better to add complementary capabilities than completely pivot.
\item Consider how changes reflect pivotal events in your character's life.
\end{itemize}

\begin{tcolorbox}[colback=green!5!white,colframe=green!75!black,title=XP Path Quick Reference,fonttitle=\bfseries]
\textbf{Personal Path (70--90\% self):}
\begin{itemize}
\item Reliable individual performance.
\item Low upkeep, high consistency.
\item Best for combatants and focused specialists.
\end{itemize}

\textbf{Balanced Path (50--65\% self):}
\begin{itemize}
\item Good all-around capability.
\item Moderate risk and upkeep.
\item Flexible support and problem-solving role.
\end{itemize}

\textbf{Influencer Path (25--40\% self):}
\begin{itemize}
\item Strategic power and influence.
\item High upkeep, high reward.
\item Creates opportunities for the entire group.
\end{itemize}

\textbf{Starting XP:} \textbf{30} base \;+\; up to \textbf{+4} from Bonds/Complications (max start \textbf{34}).
\end{tcolorbox}

% ============================================================
% PRACTICAL BUILDING EXAMPLES
% ============================================================

\section{Practical Building Examples (Narrative Roles, Legal Starts)}

Each of the following examples combines:
\begin{itemize}
  \item a short \textbf{concept},
  \item a \textbf{path choice},
  \item a legal \textbf{starting build},
  \item and a sketch of their \textbf{career arc}.
\end{itemize}

\subsection*{Example 1: The Guardian}

\textbf{Path}: Personal \quad \textbf{Total: 30 XP}

\medskip

\noindent\textbf{Concept Seed:} ``Quiet wall of muscle who would rather bleed than lose anyone.''

\begin{itemize}
\item \textbf{Attributes}: Body 3 (1$\to$2: 6, 2$\to$3: 9), Wits 2 (1$\to$2: 6) = \textbf{21 XP}
\item \textbf{Skills}: Melee 2 (0$\to$1: 2, 1$\to$2: 4 = 6), Athletics 1 (0$\to$1: 2) = \textbf{8 XP}
\item \textbf{Bank}: \textbf{1 XP}
\item \textbf{Role at table}: Frontline protection and reliable pressure in fights.
\item \textbf{With +4 XP}: add \emph{Combat Reflexes} (2 XP talent) and \emph{Shield Mastery} (4 XP talent) using banked 1 + 4 = 5 XP (GM may allow rounding or an extra minor Complication), or instead buy more supporting skills.
\end{itemize}

\noindent\textbf{Career Arc:} Early on, the Guardian is ``the one who stands in
front.'' Mid game adds tactical awareness and maybe leadership. Late game might
see them training others or becoming the symbol of a cause.

\subsection*{Example 2: The Explorer}

\textbf{Path}: Balanced \quad \textbf{Total: 30 XP}

\medskip

\noindent\textbf{Concept Seed:} ``Curious wanderer whose maps are more valuable than gold.''

\begin{itemize}
\item \textbf{Attributes}: Wits 2 (6), Body 2 (6) = \textbf{12 XP}
\item \textbf{Skills}: Survival 2 (0$\to$1: 2, 1$\to$2: 4 = 6), Perception 1 (2), Stealth 1 (2) = \textbf{10 XP}
\item \textbf{Resources}: Minor mapping kit \& route notes = \textbf{4 XP}
\item \textbf{Ability}: Trail Sense = \textbf{4 XP}
\item \textbf{Totals}: \textbf{30 XP}. With +4 XP, raise \emph{Perception 1$\to$2} (+4) or add a trained beast (Minor Resource, 4 XP).
\end{itemize}

\noindent\textbf{Career Arc:} Early stories focus on survival and discovery. Mid
game might see them guiding trade, armies, or refugees. Late game, they may
literally redraw the map of the setting.

\subsection*{Example 3: The Schemer}

\textbf{Path}: Influencer \quad \textbf{Total: 30 XP}

\medskip

\noindent\textbf{Concept Seed:} ``The person who always knows someone who knows someone.''

\begin{itemize}
\item \textbf{Attributes}: Presence 2 (6), Wits 2 (6) = \textbf{12 XP}
\item \textbf{Skills}: Sway 2 (0$\to$1: 2, 1$\to$2: 4 = 6), Deception 1 (2), Lore 1 (2) = \textbf{10 XP}
\item \textbf{Resources}: Standard safehouse \& message drops = \textbf{8 XP}
\item \textbf{Totals}: \textbf{30 XP}. With +4 XP, take \emph{Network Builder} (4 XP talent) or add \emph{Minor informant ring} (4 XP).
\end{itemize}

\noindent\textbf{Career Arc:} Early on, the Schemer is just ``the one with
connections.'' Mid game grows their network and enemies. Late game, they might
operate entire webs of spies, informants, merchants, or political factions.

\paragraph{Reminder:}
All builds above assume baseline \emph{Attributes at 1} and \emph{Skills at 0}
before spending. Attribute and Skill advances are cumulative by step (see costs
in \S\ref{ch:xp-paths:starting}).

Remember: Your chosen path should reflect both your character concept and your
preferred play style. There is no single ``correct'' path—only what works for
you, your group, and the story you want to tell.

% ============================================================
% NARRATIVE-HEAVY BUILDING
% ============================================================

\section{Narrative-Heavy Character Building Options}
\index{narrative advancement}

For groups that prefer a strong narrative focus in character building, you can
treat XP as a tool for telling the story of a career, not just a currency.

\paragraph{Story-Driven Milestones}
Instead of tracking XP numerically, the GM can award advancement when
characters reach significant story milestones.  
``You have trained with the master for months—you've improved your skill.''

\paragraph{Experience Through Reflection}
Players can spend downtime scenes reflecting on past experiences to earn XP.
A meaningful flashback, confession, or revelation can justify growth without
counting every point.

\paragraph{Collaborative Advancement}
The group can discuss and agree on advancement choices, ensuring everyone's
growth supports the overall story direction and shared arcs.

\paragraph{Narrative Justification Focus}
When spending XP, players should explain how their character gained this
capability through in-game experiences, creating richer backstory and
continuity. ``I raise Presence because I've been forced to handle the group's
negotiations for weeks.''

\paragraph{Path as Theme}
Use your chosen path as a \textbf{theme} rather than a strict budget:
\begin{itemize}
  \item A Personal Path character emphasizes physical, mental, or spiritual self-mastery.
  \item A Balanced Path character emphasizes learning how to juggle tools and relationships.
  \item An Influencer Path character emphasizes their growing web of relationships and obligations.
\end{itemize}

In all cases, let your character's \textbf{career} on the page match the career
they are living in the fiction.