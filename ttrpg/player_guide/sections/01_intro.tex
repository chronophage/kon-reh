\chapter{Welcome to Fate's Edge} \label{ch:intro}

\emph{A world where every choice carries weight, every spell risks backlash, and every legend is written in the shadow of consequence.}

Welcome to \textbf{Fate's Edge}, a tabletop roleplaying game where narrative drives mechanics, and every decision shapes not only your character's path—but the world around them. This is not a game of perfect successes or clean victories. It is a game of risk, drama, and legacy.

\section*{What Is This Game?} \index{game concept}

Fate's Edge is a narrative-first RPG where: 
\begin{itemize} 
\item Every roll introduces potential for triumph \emph{and} complication 
\item Magic is powerful—but dangerous 
\item Choices ripple outward, shaping character arcs and the setting 
\item Growth is meaningful, earned through experience spent on skills and assets 
\end{itemize}

This guide helps you build a character, understand the setting, and step into a world where your actions matter.

\section*{Core Principles} \index{design philosophy}

The game is built on four key ideas:

\begin{description} 
\item[Narrative First] Mechanics serve the story. Rules reward descriptive play and creative problem-solving. 
\item[Risk Creates Drama] Every roll carries tension. Even success may come at a cost. 
\item[Meaningful Growth] Experience is a currency of choice. Invest in yourself or your influence on the world. 
\item[Consequences Matter] No action is free. Every choice changes the fiction. 
\end{description}

\section*{Style of Play} \index{tone of play}

Expect cinematic, collaborative storytelling: 
\begin{itemize} 
\item Stories driven by character choices 
\item A world that reacts to your decisions 
\item Themes of legacy, sacrifice, and moral choices 
\end{itemize}

Whether you're a lone duelist, a scheming mastermind, or a spirit-touched outlander, your path is yours to forge.

\section*{Guide Structure}

This Player's Guide contains:

\begin{itemize} 
  \item \textbf{Core Mechanics} — Action resolution, experience spending, consequences 
  \item \textbf{Character Creation} — Attributes, skills, paths, and archetypes 
  \item \textbf{Magic and Talents} — Dangerous arts and unique abilities 
  \item \textbf{World and Lore} — Lands, peoples, and cultures 
  \item \textbf{Assets and Allies} — Building influence beyond yourself 
  \item \textbf{Appendices} — Quick references and generators 
\end{itemize}

\section*{How to Use This Book}

Read cover to cover or jump to relevant sections. Each chapter stands alone while connecting to broader themes.

Use with the \emph{System Reference Document} for full mechanical support.

\section*{Getting Started}

This is a game of bold choices and lasting consequences. Your story is written in decisions—not dice rolls.

Welcome to the Edge. The world is watching.

\begin{center} 
\emph{What will you risk to reshape the world?} 
\end{center}

\begin{tcolorbox}[colback=gray!5!white, colframe=gray!75!black, title=Flavor is Free, fonttitle=\bfseries] \textbf{Players:} Remember that \textbf{flavor is free}!

Add descriptive details, cultural elements, and atmospheric touches without spending resources or requiring rolls. Want to parry with a traditional technique? Go ahead! Want to describe seasonal festivals during a social roll? Perfect!

Flavor enriches the narrative without changing mechanical outcomes. Describe your character's background, customs, or scene details. The Game Master should encourage this and reciprocate.

Mechanics determine success or failure, but flavor determines the story we tell.\end{tcolorbox}

\section*{Narrative-Heavy Gameplay Options}

For groups that prefer strong narrative focus, consider these optional approaches:

\textbf{Collaborative Scene Framing:} Players may suggest scene elements (weather, NPC reactions, environmental details) that fit the established fiction, with GM approval.

\textbf{Intent-Driven Resolution:} For non-combat actions where success is reasonably assured, the GM may ask players to describe \emph{how} they accomplish their goal rather than rolling dice.

\textbf{Flashback Declarations:} Players can declare a flashback scene to establish that something happened in the past (acquiring an item, making a connection, learning information) by spending 1 Boon and describing the scene.

\textbf{Descriptive Assistance:} Players can assist each other by providing vivid, helpful descriptions of the action, granting a +1 die bonus to the primary actor's roll.

\textbf{Narrative Control Points:} Each player starts each session with 1 Narrative Control Point. They can spend it to:
\begin{itemize}
\item Introduce a minor NPC who provides useful information or assistance
\item Establish that they have a useful item on hand (within reason)
\item Create a favorable environmental detail
\end{itemize}

These points refresh each session and encourage proactive storytelling.