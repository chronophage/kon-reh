\chapter{Welcome to Fate's Edge} \label{ch:intro}

\emph{A world where every choice carries weight, every spell risks backlash, and every legend is written in the shadow of consequence.}

Welcome to \textbf{Fate's Edge}, a tabletop roleplaying game where narrative drives mechanics, and every decision shapes not only your character's path—but the world around them. This is not a game of perfect successes or clean victories. It is a game of risk, drama, and legacy.

\section*{What Is This Game?} \index{game concept}

Fate's Edge is a narrative-first RPG where: 
\begin{itemize} 
\item Every roll introduces potential for triumph \emph{and} complication 
\item Magic is powerful—but dangerous 
\item Choices ripple outward, shaping character arcs and the setting 
\item Growth is meaningful, earned through experience spent on skills and assets 
\end{itemize}

This guide helps you build a character, understand the setting, and step into a world where your actions matter.

\section*{Core Principles} \index{design philosophy}

The game is built on four key ideas:

\begin{description} 
\item[Narrative First] Mechanics serve the story. Rules reward descriptive play and creative problem-solving. 
\item[Risk Creates Drama] Every roll carries tension. Even success may come at a cost. 
\item[Meaningful Growth] Experience is a currency of choice. Invest in yourself or your influence on the world. 
\item[Consequences Matter] No action is free. Every choice changes the fiction. 
\end{description}

\section*{Style of Play} \index{tone of play}

Expect cinematic, collaborative storytelling: 
\begin{itemize} 
\item Stories driven by character choices 
\item A world that reacts to your decisions 
\item Themes of legacy, sacrifice, and moral choices 
\end{itemize}

Whether you're a lone duelist, a scheming mastermind, or a spirit-touched outlander, your path is yours to forge.

\section*{Guide Structure}

This Player's Guide contains:

\begin{itemize} 
  \item \textbf{Core Mechanics} — Action resolution, experience spending, consequences 
  \item \textbf{Character Creation} — Attributes, skills, paths, and archetypes 
  \item \textbf{Magic and Talents} — Dangerous arts and unique abilities 
  \item \textbf{World and Lore} — Lands, peoples, and cultures 
  \item \textbf{Assets and Allies} — Building influence beyond yourself 
  \item \textbf{Appendices} — Quick references and generators 
\end{itemize}

\section*{How to Use This Book}

Read cover to cover or jump to relevant sections. Each chapter stands alone while connecting to broader themes.

Use with the \emph{System Reference Document} for full mechanical support.

\section*{Getting Started}

This is a game of bold choices and lasting consequences. Your story is written in decisions—not dice rolls.

Welcome to the Edge. The world is watching.

\begin{center} 
\emph{What will you risk to reshape the world?} 
\end{center}

\section{The Philosophy of Narrative First}
\label{sec:narrative-first}

In Fate's Edge, the story is not a byproduct of the rules—it is the reason the rules exist. When you sit down to play, you're not rolling dice to see what happens; you're telling a story together, with dice as your co-authors. This is the \textit{narrative first} philosophy, and it changes everything about how you approach the game.

\subsection{Why Narrative First?}
Imagine this scene:

\begin{quote}
\textit{Your character, a wounded duelist, faces the baron's champion in the royal arena. Your legs ache from the earlier fight, and the crowd's roar has become a physical pressure. You know one strike will end it—but you also know you're one misstep from death.}
\end{quote}

Now consider two ways this might play out:

\begin{itemize}
    \item \textbf{Mechanics-First Approach:} "I roll Body 3 + Melee 2 to attack. I get 3 successes. The baron's champion has 2 Harm. I win."
    
    \item \textbf{Narrative-First Approach:} "I lunge forward, my blade a silver arc against the red of the arena sand. But my injured leg buckles—\textit{I spend 1 Boon to re-roll the die that came up 1}—and I twist to drive my shoulder into his chest instead. The crowd roars as we both fall to the sand, my blade at his throat. \textit{GM: ``You have the advantage, but your wound has reopened. What do you do?''}''
\end{itemize}

The first version tells you who won. The second tells you \textit{how} you won and what it cost you. This is the heart of narrative first: your choices matter not because of numbers on a sheet, but because of the story they create.

\subsection{Your Role in the Story}
In Fate's Edge, you are:
\begin{itemize}
    \item \textbf{A storyteller}—Your descriptions shape the world around you
    \item \textbf{An active participant}—Your choices drive the narrative forward
    \item \textbf{A co-creator}—You and the GM build the story together
\end{itemize}

This means:
\begin{itemize}
    \item \textbf{Flavor is free:} Describe how you fight, what your character looks like, or what the world feels like—no mechanical cost, just story enrichment.
    
    \textit{Example:} "I parry with the grace of a dancer, my movements a memory of my mother's training, even as blood drips from my brow."
    
    \item \textbf{Describe your action before rolling:} Don't just say "I attack"—show us how you do it. This informs the GM's response and often determines the Position (Dominant, Controlled, Desperate).
    
    \textit{Example:} "I use the crowd's noise as cover to slip behind him and strike his exposed back." (This would likely be \textit{Dominant Position} for a stealthy approach)
    
    \item \textbf{Embrace failure as opportunity:} When you miss a roll, it's not "you fail to hit." It's "the blade glances off his armor, and you see the baron's hand reach for the hilt of his dagger." Failures should always move the story forward.
\end{itemize}

\subsection{The Golden Rule}
\textbf{When in doubt, make the choice that serves the story.} If you find yourself asking "What's the optimal move?" try reframing it as "What would be most interesting for the story?"

\begin{center}
    \fbox{\parbox{0.9\textwidth}{
        \textbf{Remember:} In Fate's Edge, the dice don't determine your success or failure. \textit{You} determine the story. The dice just help decide which path you take to get there.
    }}
\end{center}

This philosophy isn't just about storytelling—it's about creating a shared experience where every player feels their choices matter, every moment has weight, and the story grows richer with every roll. Welcome to a game where the world responds to your choices, not the other way around.

\begin{center}
    \textit{Your story begins now. What will you risk to reshape the world?}
\end{center}

\begin{tcolorbox}[colback=gray!5!white, colframe=gray!75!black, title=Flavor is Free, fonttitle=\bfseries] \textbf{Players:} Remember that \textbf{flavor is free}!

Add descriptive details, cultural elements, and atmospheric touches without spending resources or requiring rolls. Want to parry with a traditional technique? Go ahead! Want to describe seasonal festivals during a social roll? Perfect!

Flavor enriches the narrative without changing mechanical outcomes. Describe your character's background, customs, or scene details. The Game Master should encourage this and reciprocate.

Mechanics determine success or failure, but flavor determines the story we tell.\end{tcolorbox}

\section*{Narrative-Heavy Gameplay Options}

For groups that prefer strong narrative focus, consider these optional approaches:

\textbf{Collaborative Scene Framing:} Players may suggest scene elements (weather, NPC reactions, environmental details) that fit the established fiction, with GM approval.

\textbf{Intent-Driven Resolution:} For non-combat actions where success is reasonably assured, the GM may ask players to describe \emph{how} they accomplish their goal rather than rolling dice.

\textbf{Flashback Declarations:} Players can declare a flashback scene to establish that something happened in the past (acquiring an item, making a connection, learning information) by spending 1 Boon and describing the scene.

\textbf{Descriptive Assistance:} Players can assist each other by providing vivid, helpful descriptions of the action, granting a +1 die bonus to the primary actor's roll.

\textbf{Narrative Control Points:} Each player starts each session with 1 Narrative Control Point. They can spend it to:
\begin{itemize}
\item Introduce a minor NPC who provides useful information or assistance
\item Establish that they have a useful item on hand (within reason)
\item Create a favorable environmental detail
\end{itemize}

These points refresh each session and encourage proactive storytelling.