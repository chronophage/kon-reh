\chapter{Core Mechanics} \label{ch:core-mechanics}

\begin{playermanaged}
  \textbf{You Track Your Own Story} 
  
  In Fate's Edge, you are the primary manager of your character's mechanical state. This isn't just about convenience—it's a core design principle that makes the game more engaging and reduces GM overhead.
  
  \textbf{What You Manage:}
  \begin{itemize}
      \item \textbf{Obligation Tracks} - One per Patron or Symbol
      \item \textbf{Corruption Clocks} - Especially for Cantors and Pact-Whisperers  
      \item \textbf{Leash Clocks} - For summoned spirits and bindings
      \item \textbf{Asset States} - Maintained, Neglected, or Compromised
      \item \textbf{Follower Conditions} - Exposure and Harm tracks
      \item \textbf{Progression Clocks} - Repertoire, Spirit Bond, etc.
      \item \textbf{Resource Status} - Boons, Favor, Leverage
  \end{itemize}
  
  \textbf{Your Responsibilities:}
  \begin{enumerate}
      \item \textbf{Mark Immediately} - When a rule says "mark +X," do it now
      \item \textbf{Declare Thresholds} - "My Obligation is filling!" or "Corruption full!"
      \item \textbf{Keep It Visible} - Use trackers everyone can see
      \item \textbf{Maintain Ownership} - You know your states better than the GM
  \end{enumerate}
  
  \textbf{Why This Matters:}
  This approach keeps you invested in your character's mechanical journey, reduces table downtime, and lets the GM focus on creating compelling complications rather than tracking your resources. The world responds to what you declare about your character's state.
  
  \textbf{GM Support:} The GM spot-checks and provides consequences, but you drive the mechanical narrative of your character's growth and struggles.
  \end{playermanaged}

In this game, every action matters. The dice don't just tell you if you succeed—they shape the story by introducing tension, risk, and consequence. Fate's Edge is designed to keep the story moving forward, even when things go wrong. This chapter covers the core resolution system and how every roll changes the narrative.

\section{Basic Dice Mechanics} \index{dice mechanics}

When you attempt a significant action, you roll a pool of ten-sided dice (d10s). The size of your pool is determined by two factors:

\[
\text{Dice Pool} = \text{Attribute} + \text{Skill}
\]

\begin{description}
  \item[Attribute] (1--5) Broad traits like strength, wit, or charm. \index{Attribute}
  \item[Skill] (0--5) Training or expertise in a specific area. \index{Skill}
\end{description}

\subsection*{Reading the Dice}

Each die that rolls \textbf{6 or higher} counts as a \textbf{Success}\index{Success}.  
Each die that rolls a \textbf{1} generates a \textbf{Story Beat (SB)}\index{Story Beats}.

\begin{center}
\small
\begin{longtable}{lc}
\toprule
\textbf{Die Result} & \textbf{Effect} \\
\midrule
6--6 & +1 Success \\
1 & +1 Story Beat (SB) \\
2--5 & No effect \\
\bottomrule
\end{longtable}
\end{center}

\paragraph{Example:}  
Lyra the rogue has Agility 3 and Stealth 2. Her dice pool is 5 dice. She rolls: 6, 4, 3, 1, 6. That gives her 2 Successes and 1 Story Beat. The GM sets the Difficulty Value at 2. Lyra succeeds at sneaking past the guards, but the GM now has 1 SB to spend—perhaps the guards hear something faintly and become suspicious.

\section{The Description Ladder} \index{description ladder}

Players can enhance their actions through detailed descriptions, which can reduce Story Beats generated by 1s:

\begin{description}
  \item[Basic Action] Roll the pool as-is. All 1s remain as Story Beats.
  \item[Detailed Action] A clear, descriptive flourish allows the player to re-roll one die showing 1.
  \item[Intricate Action] A richly described, multi-sensory action allows the player to re-roll all dice showing 1, and add one positive narrative flourish to the scene if they succeed.
\end{description}

\textbf{Rule:} Re-rolling 1s does not remove the Story Beats already generated by those dice. If any re-rolled dice show 1 again, they generate additional SB as normal.

\section{Difficulty Value (DV)} \index{difficulty value}

Before rolling, the Game Master sets a \textbf{Difficulty Value (DV)}\index{Difficulty Value (DV)}—the target number of Successes needed.

\begin{center}
\small
\begin{longtable}{cl}
\toprule
\textbf{DV} & \textbf{Situation} \\
\midrule
2 & Routine action, no pressure \\
3 & Pressured, mild opposition \\
4 & Difficult, active resistance \\
5+ & Extreme, high stakes \\
\bottomrule
\end{longtable}
\end{center}

\paragraph{Tip for Players:} A DV of 3 is the most common challenge. Assume that if the GM asks you to roll, there is something at stake—whether it is your safety, your resources, or your reputation.

\section{Outcome Matrix}
\index{outcome matrix}

When you take a risky action, roll your dice and compare the number of \textbf{Successes} to the Difficulty Value (DV).
The result determines how the story moves forward.

\begin{center}
\small
\begin{longtable}{ll}
\toprule
\textbf{Outcome} & \textbf{What It Means} \\
\midrule
\textbf{Clean Success}\index{Success!Clean} 
& You achieve your goal cleanly and as intended. \\

\textbf{Success with Cost}\index{Success!with Cost} 
& You achieve your goal, but something complicates the situation. \\

\textbf{Partial Success}\index{Partial} 
& You make meaningful progress, but the situation remains unresolved or constrained. \\

\textbf{Miss}\index{Miss} 
& You fail to achieve your goal and the situation escalates or changes. \\
\bottomrule
\end{longtable}
\end{center}

\paragraph{What a Partial Means.}
A Partial means your action \emph{matters}. You don’t fail—but you don’t fully solve the problem yet.

Depending on the situation, a Partial might:
\begin{itemize}
  \item Improve your \textbf{Position} for a follow-up attempt
  \item Reduce the effective \textbf{DV} of the next roll
  \item Achieve the goal at reduced scale or limited duration
  \item Succeed fully, but introduce a complication
  \item Reveal information or open a new path forward
\end{itemize}

The exact outcome is decided by the GM based on the fiction and stakes.

\paragraph{Example.}
A fighter tries to disarm a bandit, rolling \textbf{3 Successes} against DV~2 — a \textbf{Clean Success}. The weapon flies free.

Later, she tries to kick down a reinforced door (DV~3) and rolls \textbf{2 Successes}. This is a \textbf{Partial}.  
The door splinters and weakens, granting better Position on the next attempt — but the crash draws attention.  
Either way, the story moves forward.

\subsection{Critical Successes}

A roll of \textbf{10} counts as \textbf{two Successes}.

\paragraph{Narrative Weight.}
When the total number of Successes meets or exceeds the \textbf{DV} (a Success), any 10s rolled represent exceptional execution and add narrative weight to the outcome.

\begin{itemize}
  \item \textbf{One 10:} A strong success — gain a flourish such as improved Position, a minor advantage, or a stylish outcome.
  \item \textbf{Two 10s:} Exceptional success — choose two benefits or one especially potent effect.
  \item \textbf{Three 10s:} Legendary success — resolve the conflict decisively and progress or clear 1 segment on a relevant secondary Clock.
  \item \textbf{Four or more 10s:} Mythic success — reshape the scene, clear 1--2 segments on a secondary Clock, or introduce a major story development.
\end{itemize}

\noindent If no 10s are rolled, resolve the outcome normally based on Successes versus DV.

\medskip
\noindent\textbf{Important:}
\begin{itemize}
  \item 10s are \emph{never re-rolled} by Position effects or other mechanics.
  \item Critical effects apply only if the roll succeeds, even if complications occur.
  \item At the GM’s discretion, a critical success may reduce Backlash, Obligation, or Corruption severity by one tier.
\end{itemize}

\section{Boons}
\index{boons}

Boons represent momentum, narrative leverage, and the spotlight turning in your favor.
They are a shared language between players and the fiction.

You may hold up to \textbf{5 Boons} at a time.

\subsection*{Earning Boons}

You gain Boons when things don’t go perfectly—or when you lean into the story.

\begin{itemize}
  \item \textbf{Partial Success}: Gain \textbf{1 Boon}
  \item \textbf{Miss}: Gain \textbf{2 Boons}
  \item \textbf{Bond-Driven Action}: Once per bond per session, gain \textbf{1 Boon} when an action meaningfully engages that bond
  \item \textbf{GM Award}: The GM may award Boons for creativity, sacrifice, or strong narrative play
\end{itemize}

\subsection*{When Boons Are Awarded}

Boons from rolls are only granted when:
\begin{enumerate}
  \item The action followed the normal procedure (intent, DV, roll)
  \item The stakes were clear before the roll
  \item A real consequence occurred in the fiction
\end{enumerate}

\paragraph{Important Note.}
Safe rehearsal actions, null-risk probing, or repeated identical attempts in the same scene do not generate Boons.
If an action doesn’t meaningfully risk change, it doesn’t generate momentum.

\subsection*{Spending Boons}
You can spend Boons to:
\begin{itemize}
\item Re-roll a single die in a pool
\item Activate an on-screen Asset
\item Power a Rite or magical ability
\item Improve Position by 1 step before a roll
\item Convert to XP (2 Boons = 1 XP, once per session during downtime, max 2 XP via conversion per session)
\end{itemize}

\subsection*{Carryover Limits}
At the end of each scene, reduce held Boons to a maximum of \textbf{2}. Excess Boons are lost. This encourages you to spend them rather than hoard.

\paragraph{Why This Matters:}
The system rewards engagement with risk. Even when you don't fully succeed, you gain resources to help push the story forward. Failures become opportunities, and partial successes still offer chances to turn the tide.

\subsection{Position}
\label{subsec:position}
\index{Position}

Every action in \indexterm{Fate's Edge} takes place from a \textbf{Position} that reflects the character’s advantage or disadvantage in the scene. Position sets the tone for the roll, narratively and mechanically. It comes in three states:

\begin{itemize}
  \item \textbf{Dominant:} You act from a place of control, leverage, or overwhelming advantage.
  \item \textbf{Controlled:} The standard state of play. Outcomes are uncertain but balanced.
  \item \textbf{Desperate:} You act from dire straits, cornered or overmatched, with everything at stake.
\end{itemize}

\paragraph{Re-roll Mechanic.}  
Position modifies the dice pool through simple re-rolls:
\begin{center}
\begin{longtable}{@{}lll@{}}
\toprule
\textbf{Position} & \textbf{Narrative Frame} & \textbf{Mechanical Effect} \\
\midrule
Dominant & You press your advantage & Re-roll one \emph{failure} \\
Controlled    & The balanced norm & No re-rolls \\
Desperate & You act under duress & Re-roll one \emph{success} \\
\bottomrule
\end{longtable}
\end{center}

\paragraph{Boon Interaction with Fatigue}
When Fatigue would force one or more success re-rolls:
\begin{itemize}
  \item Spending \textbf{1 Boon} negates \textbf{one} Fatigue-imposed re-roll.
  \item Additional Fatigue re-rolls must be resolved normally.
\end{itemize}

\noindent
\emph{Note:} Boons may be spent \textbf{after seeing} the result of a Fatigue re-roll.

\section{Story Beats (SB)} \index{story beats}

Story Beats are narrative tools the Game Master uses to introduce twists and tension. They keep the story alive with complications and surprises.

\subsection*{What SB Can Do}
The GM may spend SB to:
\begin{itemize}
  \item Introduce new threats or complications
  \item Drain resources (time, gear, position)
  \item Reveal hidden dangers
  \item Cause collateral damage
\end{itemize}

\subsection*{SB Spend Examples}
\begin{itemize}
  \item \textbf{1 SB} — Minor complication, noise, trace
  \item \textbf{2 SB} — Moderate setback, alarm raised
  \item \textbf{3 SB} — Serious trouble, reinforcements arrive
  \item \textbf{4+ SB} — Major turn, scene shifts dramatically
\end{itemize}

\paragraph{Player Advice:}  
Don't fear Story Beats—they're not punishment. They are fuel for drama, ensuring the spotlight never dims.

\section{Harm and Fatigue} \index{harm and fatigue}

Physical injury and exhaustion are tracked through two systems:

\subsection{Initiative and Turn Order}

Fate's Edge does not use fixed initiative. 
Turn order follows the fiction and the GM's facilitation:
\begin{itemize}
    \item \textbf{Narrative Fiat:} The GM frames spotlight order based on circumstances, tension, and narrative flow.
    \item \textbf{Player Input:} Players may suggest acting when it makes sense in the fiction. 
    \item \textbf{Surprise:} Ambushers act first; targets respond after the opening exchange.
    \item \textbf{Flexibility:} Spotlight may shift mid-scene if fictionally appropriate (e.g., reacting to a falling ceiling, seizing a moment).
\end{itemize}

This ensures pacing and drama guide the sequence of actions, not rigid turn structures.

\subsection{Turn Economy (Quick Rules)}
\label{subsec:turn-economy-quick}

\paragraph{Two Actions.}
Each character takes \emph{1 Action and 1 Move} on their turn. Actions and Moves may be taken in any order; repeating the same Action is not allowed unless noted. A character may use a Boon to re-roll their action at the expense of their move if they still have it available. Some weapon tempos effect whether you can take an attack and a move.

\paragraph{Move.}
Traverse up to your normal movement. \emph{Disengage:} move without provoking; your next offensive action is \textbf{Controlled}. \emph{Dash:} move again this turn; your next defense is \textbf{Desperate}. \textbf{Stand:} Use your move action to stand from prone.

\paragraph{Attack.}
Make a melee or ranged attack versus DV set by the GM and fiction. Teamwork/Assist costs 1 Boon.

\paragraph{Observe / Change Position (+1).}
Take a beat to read the field or set angles; gain \textbf{+1 Position} for one action this turn (e.g., Controlled$\to$Dominant). Limit: once/turn; cannot exceed \textbf{Dominant}.

\paragraph{Activate an Asset.}
Use gear, symbol, tool, or feature per its text/tags (e.g., torch, grapnel, smoke vial, rune focus). Items with \texttt{[Action]} consume one Action; \texttt{[Free]} do not.

\paragraph{Setup (Teamwork).}
Create advantage for an ally; on success, grant their next action \textbf{+1 Position} or step up Effect (GM’s call).

\paragraph{Assist (Teamwork).}
Spend \emph{1 Boon} to give an ally \emph{+1 die} on their current roll; you share appropriate risk/consequence.

\paragraph{Defend / Protect.}
Adopt a guarding stance or body-block. Choose a nearby ally; until your next turn you may intercept one hit on them and roll to resist it. On success, reduce/negate Harm; you take any fallout the GM assigns.

\paragraph{Channel / Weave.}
Runekeeper/ritual flow: \emph{Channel} (prime power) then \emph{Weave} (shape/release). Disruption or engagement may worsen Position; if \emph{Interrupted}, the casting fails.

\paragraph{Cast Rite / Song (Cantor).}
Perform a Rite/Song per its write-up. You may \emph{Push} to accelerate or empower at the cost of Fatigue/Corruption per class rules.

\paragraph{Interact.}
Lift, pull, flip a lever, shove a foe, break an object, apply a poultice, reload, draw/stow, etc. GM sets DV/Effect.

\paragraph{Free Items.}
Short shouts, dropping an item, quick glance. Longer or tactical assessments require \emph{Observe / Change Position} or \emph{Interact}.

\paragraph{Reactions (Out of Turn).}
\emph{Protection} may trigger when an ally is hit and you are in position. Class/Asset reactions fire as written (e.g., counter-runes, ripostes).

\paragraph{Position Caps.}
Bonuses cannot raise Position above \textbf{Dominant}; penalties cannot drop below \textbf{Desperate}. Beyond these caps, adjust DV or Effect instead.



\subsection*{Fatigue Track}
Each character has a Fatigue Track equal to their Body attribute. Mark Fatigue for:
\begin{itemize}
  \item Physical exertion
  \item Magical strain
  \item Travel stress
  \item Mental pressure
\end{itemize}
\subsection{Fatigue}
\label{subsec:fatigue}
\index{Fatigue}

\textbf{Track:} Each character has a Fatigue track equal to \textbf{Body}. Mark Fatigue for exertion, strain, or backlash.

\textbf{In Play:} Each Fatigue step worsens your \textbf{Position} by one level 
(Dominant $\rightarrow$ Controlled $\rightarrow$ Desperate). 
If you are already \textbf{Desperate}, instead apply a \textbf{--1 die} penalty per Fatigue to that roll.

\textbf{Overflow:} When your Fatigue track fills, immediately increase \textbf{Harm by 1 step} and clear all Fatigue to 0. 
If this raises Harm to a level that incapacitates you, you fall out of the scene as normal for Harm.

\textbf{Recovery:} Short rest clears 1--2 Fatigue; a full night's rest clears all Fatigue.

\subsection*{Harm Levels}
\begin{center}
\small
\begin{longtable}{ll}
\toprule
\textbf{Harm Level} & \textbf{Effects} \\
\midrule
\textbf{Harm 1} & -1 die on related actions \\
\textbf{Harm 2} & -1 die on most actions until treated \\
\textbf{Harm 3} & Incapacitated or dying \\
\bottomrule
\end{longtable}
\end{center}

\subsection*{Recovering Fatigue}
\begin{itemize}
  \item \textbf{Short Rest} — Remove 2 Fatigue with food/water
  \item \textbf{Full Night} — Remove all Fatigue
\end{itemize}

\subsection*{Recovering Harm}
\begin{itemize}
  \item \textbf{Minor treatment} — Downgrade Harm with time/rest
  \item \textbf{Proper medical care} — Remove Harm levels
  \item \textbf{Extended recovery} — Heal severe injuries
\end{itemize}

\paragraph{Example:}  
Jorin the mercenary takes a sword cut (Harm 1). He suffers -1 die to physical actions until treated. After binding the wound and resting, the Harm fades.

\section{Assistance} \index{assistance}

Characters can help each other. One helper per action may provide assistance by spending 1 Boon or 1 Stress, adding +1 die to the primary actor's roll. Maximum +3 dice from assists.

\paragraph{Example:}  
Two thieves cooperate to pick a complex lock. The lead thief has Dexterity 3 + Tools 2 = 5 dice. The helper spends 1 Boon to add 1 die, making 6. Cooperation often turns failure into tense success.

\subsection*{Defend Action (Optional Core Rule)}
\index{Actions!Defend}
\index{Defense}

Characters may explicitly \textbf{Defend} against incoming attacks or effects.

\paragraph{Declare Defense}
On your turn, you may spend either a \emph{Move} or \emph{Standard} action to enter a defensive stance.

\begin{itemize}
  \item You do not attack or perform another major action.
  \item Until the start of your next turn, you count as \textbf{Defending}.
\end{itemize}

\paragraph{Mechanical Effect}
When you are targeted by a harmful action, roll an appropriate \textbf{Attribute + Skill} pool to defend:

\begin{itemize}
  \item \emph{Body + Athletics} to dodge or endure.
  \item \emph{Body + Melee} to parry or intercept with a weapon.
  \item \emph{Wits + Perception} to anticipate or pre-empt a strike.
  \item \emph{Spirit + Resolve} to resist fear, domination, or mystical influence.
\end{itemize}

Your \textbf{Position improves by one step} for this roll.

If already \textbf{Dominant}, instead gain \textbf{+1d} to the defense roll.

\paragraph{Resolution}
Resolve the roll using the standard Outcome Matrix:

\begin{description}
  \item[Success:] You avoid or fully negate the harm or effect.
  \item[Partial:] Reduce its severity by one tier (e.g., Harm 2→1, Condition becomes lesser).
  \item[Miss:] The full effect occurs, but your \textbf{next Position against that attacker improves} by one step.
\end{description}

\paragraph{Why Use This Action?}
\begin{itemize}
  \item Provides a tactical alternative to attacking.
  \item Gives control to players facing overwhelming enemies.
  \item Creates meaningful choices under pressure—survive now, strike later.
\end{itemize}

\section{Weapons \& Armor}
\label{app:weapons-armor}
\index{Weapons}\index{Armor}

\subsection{Weapons by Weight Class}
\begin{itemize}
  \item \textbf{Light (4 XP)} — fast, concealable.
  \item \textbf{Medium (8 XP)} — balanced, battlefield standard.
  \item \textbf{Heavy (12 XP)} — punishing, slow.
\end{itemize}

\subsection*{Melee}
\begin{longtable}{llll}
\toprule
\textbf{Weight} & \textbf{Close} & \textbf{Near} & \textbf{Notes} \\
\midrule
Light & +2d & +1d & Quick, tight quarters \\
Medium & +1d & +2d & \emph{Set} 1/scene or –1d first attack \\
Heavy & –1d & +3d & \emph{Set} 1/scene or –2d first attack \\
\bottomrule
\end{longtable}

\subsection*{Ranged \& Tempo}
\begin{longtable}{lllll}
\toprule
\textbf{Weight} & \textbf{Tempo} & \textbf{Close} & \textbf{Near} & \textbf{Far} \\
\midrule
Light (4 XP) & Fast & Controlled & +1d & — \\
Medium (8 XP) & Standard & Desperate & +2d & +1d \\
Heavy (12 XP) & Slow & Desperate & +1d & +3d \\
\bottomrule
\end{longtable}

\paragraph{Tempo:} \textbf{Fast} = Move+Shoot. \textbf{Standard} = Move or Shoot, Aim +1d/Effect. \textbf{Slow} = Set/Brace, full reload, cannot Move+Shoot.

\subsection{Weapon Tags (Optional, +4 XP each, max 2)}
\index{Weapons!Tags}
\textbf{Reach, Close, Accurate, Brutal, Hook, Concealable, Quickdraw, Two-Handed, Off-Hand.}

\subsection{Shields (Optional)}
\begin{longtable}{llll}
\toprule
\textbf{Shield} & \textbf{XP} & \textbf{Benefit} & \textbf{Tradeoff} \\
\midrule
Buckler & 4 & +1d Defend vs melee or +1 DV & Off-hand \\
Heater  & 8 & +1d Defend; 1 Harm→Fatigue & –1d Ranged \\
Pavise  & 12 & \emph{Plant}: heavy cover cone & Bulky, immobile \\
\bottomrule
\end{longtable}

\subsection{Armor}
\begin{longtable}{llll}
\toprule
\textbf{Armor} & \textbf{XP} & \textbf{Conversion} & \textbf{Penalty} \\
\midrule
Light  & 4  & 1 Harm→1 Fatigue & — \\
Medium & 8  & 2 Harm→1 Fatigue & –1d physical \\
Heavy  & 12 & 3 Harm→2 Fatigue & –2d physical, no sprint \\
\bottomrule
\end{longtable}

\paragraph{Notes:} Conversion applies per Harm instance before Fatigue is marked. You may still Resist first.

\subsection{Condition \& Upkeep}
\begin{description}
  \item[\textbf{Neglected}] Weapons –1d; Armor: conversion worsens by 1 step.
  \item[\textbf{Compromised}] Weapons –1d first attack/round; Armor: no conversion.
\end{description}
\emph{Fix:} Short Rest/tools remove Neglected. A scene/Smith removes Compromised.

\section{Ranged Options}
\begin{itemize}
  \item \textbf{Aim:} +1d or +1 Effect.  
  \item \textbf{Volley:} Extra ammo +1d (max +2).  
  \item \textbf{Suppress:} Zone fire, foes –1d/Limited Effect.  
  \item \textbf{Overwatch:} Ready a Controlled shot on trigger.  
\end{itemize}

\section{Followers and Assets}

Followers and Assets represent the lasting impact of your character’s actions on the world.  
They are not abstractions or passive bonuses—they are \emph{people, resources, and structures} that exist in the fiction and can act independently of your character.

Together, Followers and Assets allow characters to project influence beyond a single roll or scene, shaping the world through relationships, preparation, and legacy.

\subsection{Followers}

\textbf{Followers} are named NPCs who have a personal relationship with your character.  
They may be students, allies, retainers, agents, family members, or dependents. Followers act in the fiction and can grow, change, or be lost as the story progresses.

\paragraph{What Followers Do}
Followers:
\begin{itemize}
  \item Assist you during scenes through \textbf{Bonds} and \textbf{Assists}
  \item Act offscreen to advance goals, gather information, or manage responsibilities
  \item Accumulate narrative weight and may eventually become player characters
\end{itemize}

Followers are not disposable tools. They have motives, limitations, and vulnerabilities, and the GM may place them at risk when the fiction demands it.

\paragraph{Using Followers}
Followers typically influence play in three ways:
\begin{itemize}
  \item \textbf{Assists}: Granting +1 die or narrative positioning when they are present and able to help
  \item \textbf{Bond Activation}: Spending a Boon to draw on trust, loyalty, or shared history for insight or support
  \item \textbf{Offscreen Action}: Handling tasks that would otherwise demand the PC’s time or attention
\end{itemize}

\paragraph{Limits}
Followers:
\begin{itemize}
  \item Do not replace player actions
  \item Do not make rolls unless the GM explicitly calls for it
  \item Cannot solve major conflicts without consequences
\end{itemize}

When Followers succeed or fail, those outcomes become part of the story.

\newpage
\section{Assets}

\textbf{Assets} are durable resources, positions, or structures under your character’s control.  
They represent \emph{leverage}—things that allow you to influence the world at scale.

Examples include:
\begin{itemize}
  \item A safehouse, laboratory, or fortress
  \item A mercantile network or smuggling route
  \item A political office or religious authority
  \item A powerful relic, symbol, or institutional role
\end{itemize}

\paragraph{What Assets Do}
Assets:
\begin{itemize}
  \item Create advantages that persist beyond a single scene
  \item Resolve problems that cannot be solved with a single roll
  \item Shape the direction and scope of the campaign
\end{itemize}

Assets are not passive bonuses. They matter only when invoked in the fiction.

\paragraph{Activating Assets}
Most Assets require the expenditure of \textbf{Boons} to activate.

When you spend a Boon on an Asset, you may:
\begin{itemize}
  \item Advance or halt a Clock
  \item Introduce or remove a complication
  \item Secure resources, protection, or access
  \item Influence events happening offscreen
\end{itemize}

The GM determines the exact impact based on the scale of the Asset and the current situation.

\paragraph{Assets and Risk}
Assets can be threatened, degraded, or lost.
\begin{itemize}
  \item Overuse attracts attention
  \item Failure creates new problems
  \item Enemies may target your Assets directly
\end{itemize}

An Asset that never faces risk is not part of the story.

\newpage

\section{Followers vs.\ Assets}

\begin{center}
\small
\begin{tabular}{p{3cm}p{5cm}p{5cm}}
\toprule
 & \textbf{Followers} & \textbf{Assets} \\
\midrule
Nature & People & Resources / Structures \\
Primary Role & Personal support & Strategic leverage \\
Activation & Assists, Bonds & Boon expenditure \\
Risk & Emotional and narrative & Structural and political \\
Growth & Can become PCs & Can expand or collapse \\
\bottomrule
\end{tabular}
\end{center}

\paragraph{Design Principle}
Followers create \emph{depth}.  
Assets create \emph{reach}.

High-tier play depends less on raw die bonuses and more on how well you manage both.

\subsection{High-Tier Play}

As characters advance, Followers and Assets become increasingly important.  
Higher-tier Talents and Assets consume Boons, forcing players to choose between immediate success and long-term influence.

Power at high tier is not about having more Boons—it is about deciding where to spend them.

\textit{Victory is no longer just surviving the roll.  
It is choosing which parts of the world move when you do.}

\textit{\textbf{Assets} should only solve problems that dice cannot.}
\newpage

\section{Game Structure} \index{game structure}

\subsection*{Time Scales}
\begin{description}
  \item[Moment] A heartbeat, a single action
  \item[Some Time] A few minutes, a short activity
  \item[Significant Time] Hours, extended effort
  \item[Days] Large-scale endeavors
\end{description}

\subsection*{Game Units}
\begin{description}
  \item[Scene] Basic narrative unit, covers specific conflict
  \item[Player Turn] Individual action within a scene
  \item[Round] Simultaneous actions in combat
  \item[Session] One game session (3--6 hours)
  \item[Campaign] Entire story arc
\end{description}

\paragraph{Player Perspective:}  
Think in scenes, not minutes. Every scene is a chance to shine. Every session builds toward the long arc of your campaign.

\section{Action Resolution Steps}

\begin{enumerate}
  \item Describe your intent and method
  \item Build dice pool: Attribute + Skill (+ gear, assists)
  \item Roll d10s, count \textbf{Successes}\index{Success} and \textbf{Story Beats}\index{Story Beats}
  \item Compare Successes to \textbf{DV}\index{Difficulty Value (DV)}
  \item Apply outcome from \textbf{matrix}\index{Outcome Matrix}
  \item Game Master spends \textbf{SB}\index{Story Beats!spend} if applicable
  \item Earn \textbf{Boons}\index{Boons} for failure.
\end{enumerate}

\begin{tcolorbox}[colback=blue!5!white,colframe=blue!75!black,title=Quick Reference,fonttitle=\bfseries]
\textbf{Dice Pool:} Attribute + Skill d10s \index{Dice Pool}\\
\textbf{Success:} 6 on each die \index{Success}\\
\textbf{Setback:} 1 on any die gives SB to GM \index{Story Beats}\\
\textbf{DV:} 2 (easy) to 5+ (extreme) \index{Difficulty Value (DV)}\\
\textbf{Harm:} 3-level system with penalties \index{Harm Levels}\\
\textbf{Boons:} 2 on miss, 1 on partial \index{Boons}
\end{tcolorbox}

\section{Narrative Suggestions}

\textbf{Collaborative Scene Framing:} Players may suggest scene elements (weather, NPC reactions, environmental details) that fit the established fiction, with GM approval.

\textbf{Intent-Driven Resolution:} For non-combat actions where success is reasonably assured, the GM may ask players to describe \emph{how} they accomplish their goal rather than rolling dice.

\textbf{Flashback Declarations:} Players can declare a flashback scene to establish that something happened in the past (acquiring an item, making a connection, learning information) by spending 1 Boon and describing the scene.

\textbf{Descriptive Assistance:} Players can assist each other by providing vivid, helpful descriptions of the action, granting a +1 die bonus to the primary actor's roll.

\textbf{Proactive Storytelling:} Players can suggest minor favorable details about their character's circumstances by:
\begin{itemize}
\item Introducing a minor NPC who provides useful information or assistance
\item Establishing that they have a useful item on hand (within reason)
\item Creating a favorable environmental detail
\end{itemize}

These suggestions are subject to GM approval and should enhance rather than overshadow the main narrative. 

\subsection{Small Folk of the Threshold (Aelaerem \& Aelinnel)}
\label{subsec:small-folk-threshold}

\index{Aelaerem}\index{Aelinnel}\index{Small Folk}

The Aelaerem and Aelinnel are diminutive peoples attuned to liminal spaces and hidden ways. Their stature grants them agility and subtlety, though at the cost of bearing heavy arms or armor.

\begin{itemize}
  \item \textbf{Restriction:} Cannot use \emph{Heavy Armor} or \emph{Heavy Weapons}.
  \item \textbf{Bonus:} Gain +1 \emph{Position} when Dodging or Resisting Knockback, and +1 die on \emph{Hide} or \emph{Evasion} rolls made while in cover.
\end{itemize}

Their presence in the world is often underestimated, but their knack for slipping unseen through thresholds and enduring where others falter has earned them a quiet reverence.
