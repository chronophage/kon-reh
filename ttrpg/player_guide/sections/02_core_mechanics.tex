\chapter{Core Mechanics} \label{ch:core-mechanics}

In this game, every action matters. The dice don't just tell you if you succeed—they shape the story by introducing tension, risk, and consequence. Fate's Edge is designed to keep the story moving forward, even when things go wrong. This chapter covers the core resolution system and how every roll changes the narrative.

\section{Basic Dice Mechanics} \index{dice mechanics}

When you attempt a significant action, you roll a pool of ten-sided dice (d10s). The size of your pool is determined by two factors:

\[
\text{Dice Pool} = \text{Attribute} + \text{Skill}
\]

\begin{description}
  \item[Attribute] (1--5) Broad traits like strength, wit, or charm. \index{Attribute}
  \item[Skill] (0--5) Training or expertise in a specific area. \index{Skill}
\end{description}

\subsection*{Reading the Dice}

Each die that rolls \textbf{6 or higher} counts as a \textbf{Success}\index{Success}.  
Each die that rolls a \textbf{1} generates a \textbf{Story Beat (SB)}\index{Story Beats}.

\begin{center}
\small
\begin{tabular}{lc}
\toprule
\textbf{Die Result} & \textbf{Effect} \\
\midrule
6--6 & +1 Success \\
1 & +1 Story Beat (SB) \\
2--5 & No effect \\
\bottomrule
\end{tabular}
\end{center}

\paragraph{Example:}  
Lyra the rogue has Agility 3 and Stealth 2. Her dice pool is 5 dice. She rolls: 6, 4, 3, 1, 6. That gives her 2 Successes and 1 Story Beat. The GM sets the Difficulty Value at 2. Lyra succeeds at sneaking past the guards, but the GM now has 1 SB to spend—perhaps the guards hear something faintly and become suspicious.

\section{The Description Ladder} \index{description ladder}

Players can enhance their actions through detailed descriptions, which can reduce Story Beats generated by 1s:

\begin{description}
  \item[Basic Action] Roll the pool as-is. All 1s remain as Story Beats.
  \item[Detailed Action] A clear, descriptive flourish allows the player to re-roll one die showing 1.
  \item[Intricate Action] A richly described, multi-sensory action allows the player to re-roll all dice showing 1, and add one positive narrative flourish to the scene if they succeed.
\end{description}

\textbf{Rule:} Re-rolling 1s does not remove the Story Beats already generated by those dice. If any re-rolled dice show 1 again, they generate additional SB as normal.

\section{Difficulty Value (DV)} \index{difficulty value}

Before rolling, the Game Master sets a \textbf{Difficulty Value (DV)}\index{Difficulty Value (DV)}—the target number of Successes needed.

\begin{center}
\small
\begin{tabular}{cl}
\toprule
\textbf{DV} & \textbf{Situation} \\
\midrule
2 & Routine action, no pressure \\
3 & Pressured, mild opposition \\
4 & Difficult, active resistance \\
5+ & Extreme, high stakes \\
\bottomrule
\end{tabular}
\end{center}

\paragraph{Tip for Players:} A DV of 3 is the most common challenge. Assume that if the GM asks you to roll, there is something at stake—whether it is your safety, your resources, or your reputation.

\section{Outcome Matrix} \index{outcome matrix}

Compare your Successes against the DV:

\begin{center}
\small
\begin{tabular}{ll}
\toprule
\textbf{Outcome} & \textbf{Effect} \\
\midrule
\textbf{Clean Success}\index{Success!Clean} & Goal achieved cleanly \\
\textbf{Success \& Cost}\index{Success!with Cost} & Goal achieved with complication \\
\textbf{Partial}\index{Partial} & Progress but with difficult choice \\
\textbf{Miss}\index{Miss} & No progress; complication occurs \\
\bottomrule
\end{tabular}
\end{center}

\paragraph{Player-Facing Example:}  
A fighter swings her sword to disarm a bandit. She rolls 3 Successes against DV 2—a Clean Success. The bandit's blade clatters away.  
Later, the same fighter tries to kick down a reinforced door with 4 dice against DV 4. She rolls only 2 Successes. This is a Partial. She cracks the door frame, but the noise attracts attention. The story moves forward either way.

\subsection{Critical Success}
Rolling a \textbf{10} on any die indicates a critical tier of success. Each 10 adds weight to the outcome:

\begin{itemize}
  \item \textbf{One 10:} Strong success with a free boon, improved Position, or other narrative flourish.
  \item \textbf{Two 10s:} Exceptional success; choose two benefits or a single powerful effect.
  \item \textbf{Three 10s:} Legendary success; the action transcends mortal limits and resolves the conflict dramatically.
  \item \textbf{Four+ 10s:} Mythic success; the GM and table agree the result reshapes the scene or story outright.
\end{itemize}

\noindent
If no 10s are rolled, resolve the action normally by the highest die result.

\section{Boons} \index{boons}

Boons are narrative currency that players can spend to influence the story in their favor. You can hold up to 5 Boons at a time.

\subsection*{Earning Boons}
You gain Boons through:
\begin{itemize}
\item \textbf{Partial Success}: When you achieve a Partial outcome (successes < DV but > 0), you gain \textbf{1 Boon}
\item \textbf{Missed Actions}: When you miss entirely (0 successes), you gain \textbf{2 Boons}
\item \textbf{Bond-Driven Actions}: When you take an Intricate action that meaningfully engages a character bond, you may gain 1 Boon (once per bond per session)
\item \textbf{GM Award}: The GM may award Boons for creative solutions, spotlighting bonds, or meaningful sacrifices
\end{itemize}

\subsection*{Requirements for Action Awards}
Boons from Partial/miss outcomes are awarded only if:
\begin{enumerate}
\item Procedure was followed correctly (intent declared, DV set, roll resolved)
\item Stakes were clearly stated (what changes on success/failure)
\item Consequence actually occurs (GM spends or banks SB, applies condition, or advances thread)
\end{enumerate}

\paragraph{Important Note:}
Rehearsal/null-risk probes and repeated identical attempts in the same scene do not award Boons. If it feels like an obvious fishing attempt, don't award a Boon.

\subsection*{Spending Boons}
You can spend Boons to:
\begin{itemize}
\item Re-roll a single die in a pool
\item Activate an on-screen Asset
\item Power a Rite or magical ability
\item Improve Position by 1 step
\item Convert to XP (2 Boons = 1 XP, once per session during downtime, max 2 XP via conversion per session)
\end{itemize}

\subsection*{Carryover Limits}
At the end of each scene, reduce held Boons to a maximum of \textbf{2}. Excess Boons are lost. This encourages you to spend them rather than hoard.

\paragraph{Why This Matters:}
The system rewards engagement with risk. Even when you don't fully succeed, you gain resources to help push the story forward. Failures become opportunities, and partial successes still offer chances to turn the tide.

\section{Story Beats (SB)} \index{story beats}

Story Beats are narrative tools the Game Master uses to introduce twists and tension. They keep the story alive with complications and surprises.

\subsection*{What SB Can Do}
The GM may spend SB to:
\begin{itemize}
  \item Introduce new threats or complications
  \item Drain resources (time, gear, position)
  \item Reveal hidden dangers
  \item Cause collateral damage
\end{itemize}

\subsection*{SB Spend Examples}
\begin{itemize}
  \item \textbf{1 SB} — Minor complication, noise, trace
  \item \textbf{2 SB} — Moderate setback, alarm raised
  \item \textbf{3 SB} — Serious trouble, reinforcements arrive
  \item \textbf{4+ SB} — Major turn, scene shifts dramatically
\end{itemize}

\paragraph{Player Advice:}  
Don't fear Story Beats—they're not punishment. They are fuel for drama, ensuring the spotlight never dims.

\section{Harm and Fatigue} \index{harm and fatigue}

Physical injury and exhaustion are tracked through two systems:

\subsection*{Fatigue Track}
Each character has a Fatigue Track equal to their Body attribute. Mark Fatigue for:
\begin{itemize}
  \item Physical exertion
  \item Magical strain
  \item Travel stress
  \item Mental pressure
\end{itemize}

When your Fatigue Track fills:
\begin{enumerate}
  \item Increase your Harm level by one step
  \item Clear all Fatigue marks
\end{enumerate}

This can happen multiple times in a scene.

\subsection*{Harm Levels}
\begin{center}
\small
\begin{tabular}{ll}
\toprule
\textbf{Harm Level} & \textbf{Effects} \\
\midrule
\textbf{Harm 1} & -1 die on related actions \\
\textbf{Harm 2} & -1 die on most actions until treated \\
\textbf{Harm 3} & Incapacitated or dying \\
\bottomrule
\end{tabular}
\end{center}

\subsection*{Recovering Fatigue}
\begin{itemize}
  \item \textbf{Short Rest} — Remove 2 Fatigue with food/water
  \item \textbf{Full Night} — Remove all Fatigue
\end{itemize}

\subsection*{Recovering Harm}
\begin{itemize}
  \item \textbf{Minor treatment} — Downgrade Harm with time/rest
  \item \textbf{Proper medical care} — Remove Harm levels
  \item \textbf{Extended recovery} — Heal severe injuries
\end{itemize}

\paragraph{Example:}  
Jorin the mercenary takes a sword cut (Harm 1). He suffers -1 die to physical actions until treated. After binding the wound and resting, the Harm fades.

\section{Assistance} \index{assistance}

Characters can help each other. One helper per action may provide assistance by spending 1 Boon or 1 Stress, adding +1 die to the primary actor's roll. Maximum +3 dice from assists.

\paragraph{Example:}  
Two thieves cooperate to pick a complex lock. The lead thief has Dexterity 3 + Tools 2 = 5 dice. The helper spends 1 Boon to add 1 die, making 6. Cooperation often turns failure into tense success.

\section{Weapons \& Armor}
\label{app:weapons-armor}
\index{Weapons}\index{Armor}

\subsection{Weapons by Weight Class}
\begin{itemize}
  \item \textbf{Light (4 XP)} — fast, concealable.
  \item \textbf{Medium (8 XP)} — balanced, battlefield standard.
  \item \textbf{Heavy (12 XP)} — punishing, slow.
\end{itemize}

\subsection*{Melee}
\begin{tabular}{llll}
\toprule
\textbf{Weight} & \textbf{Close} & \textbf{Near} & \textbf{Notes} \\
\midrule
Light & +2d & +1d & Quick, tight quarters \\
Medium & +1d & +2d & \emph{Set} 1/scene or –1d first attack \\
Heavy & –1d & +3d & \emph{Set} 1/scene or –2d first attack \\
\bottomrule
\end{tabular}

\subsection*{Ranged \& Tempo}
\begin{tabular}{lllll}
\toprule
\textbf{Weight} & \textbf{Tempo} & \textbf{Close} & \textbf{Near} & \textbf{Far} \\
\midrule
Light (4 XP) & Fast & Risky & +1d & — \\
Medium (8 XP) & Standard & Desperate & +2d & +1d \\
Heavy (12 XP) & Slow & Desperate & +1d & +3d \\
\bottomrule
\end{tabular}

\paragraph{Tempo:} \textbf{Fast} = Move+Shoot. \textbf{Standard} = Move or Shoot, Aim +1d/Effect. \textbf{Slow} = Set/Brace, full reload, cannot Move+Shoot.

\subsection{Weapon Tags (Optional, +4 XP each, max 2)}
\index{Weapons!Tags}
\textbf{Reach, Close, Accurate, Brutal, Hook, Concealable, Quickdraw, Two-Handed, Off-Hand.}

\subsection{Shields (Optional)}
\begin{tabular}{llll}
\toprule
\textbf{Shield} & \textbf{XP} & \textbf{Benefit} & \textbf{Tradeoff} \\
\midrule
Buckler & 4 & +1d Defend vs melee or +1 DV & Off-hand \\
Heater  & 8 & +1d Defend; 1 Harm→Fatigue & –1d Ranged \\
Pavise  & 12 & \emph{Plant}: heavy cover cone & Bulky, immobile \\
\bottomrule
\end{tabular}

\subsection{Armor}
\begin{tabular}{llll}
\toprule
\textbf{Armor} & \textbf{XP} & \textbf{Conversion} & \textbf{Penalty} \\
\midrule
Light  & 4  & 1 Harm→1 Fatigue & — \\
Medium & 8  & 2 Harm→1 Fatigue & –1d physical \\
Heavy  & 12 & 3 Harm→2 Fatigue & –2d physical, no sprint \\
\bottomrule
\end{tabular}

\paragraph{Notes:} Conversion applies per Harm instance before Fatigue is marked. You may still Resist first.

\subsection{Condition \& Upkeep}
\begin{description}
  \item[\textbf{Neglected}] Weapons –1d; Armor: conversion worsens by 1 step.
  \item[\textbf{Compromised}] Weapons –1d first attack/round; Armor: no conversion.
\end{description}
\emph{Fix:} Short Rest/tools remove Neglected. A scene/Smith removes Compromised.

\section{Ranged Options}
\begin{itemize}
  \item \textbf{Aim:} +1d or +1 Effect.  
  \item \textbf{Volley:} Extra ammo +1d (max +2).  
  \item \textbf{Suppress:} Zone fire, foes –1d/Limited Effect.  
  \item \textbf{Overwatch:} Ready a Risky shot on trigger.  
\end{itemize}

\section{Assets and Allies} \index{assets}

Your character's resources, contacts, or gear—called \textbf{Assets}\index{Assets}—can tilt the odds in your favor.

\begin{itemize}
  \item \textbf{On-Screen Assets} — Companions, hirelings, or allies who stand beside you in danger
  \item \textbf{Off-Screen Assets} — Taverns, estates, titles, or networks of informants
  \item \textbf{Activation} — Spend 1 Boon to activate an on-screen Asset
\end{itemize}

\paragraph{Narrative Use:}  
Assets are more than bonuses—they're hooks for roleplay. A friendly tavernkeeper, a noble's signet, or a trusty horse might tip the balance at the perfect moment.

\section{Game Structure} \index{game structure}

\subsection*{Time Scales}
\begin{description}
  \item[Moment] A heartbeat, a single action
  \item[Some Time] A few minutes, a short activity
  \item[Significant Time] Hours, extended effort
  \item[Days] Large-scale endeavors
\end{description}

\subsection*{Game Units}
\begin{description}
  \item[Scene] Basic narrative unit, covers specific conflict
  \item[Player Turn] Individual action within a scene
  \item[Round] Simultaneous actions in combat
  \item[Session] One game session (3--6 hours)
  \item[Campaign] Entire story arc
\end{description}

\paragraph{Player Perspective:}  
Think in scenes, not minutes. Every scene is a chance to shine. Every session builds toward the long arc of your campaign.

\section{Action Resolution Steps}

\begin{enumerate}
  \item Describe your intent and method
  \item Build dice pool: Attribute + Skill (+ gear, assists)
  \item Roll d10s, count \textbf{Successes}\index{Success} and \textbf{Story Beats}\index{Story Beats}
  \item Compare Successes to \textbf{DV}\index{Difficulty Value (DV)}
  \item Apply outcome from \textbf{matrix}\index{Outcome Matrix}
  \item Game Master spends \textbf{SB}\index{Story Beats!spend} if applicable
  \item Earn \textbf{Boons}\index{Boons} for failure.
\end{enumerate}

\begin{tcolorbox}[colback=blue!5!white,colframe=blue!75!black,title=Quick Reference,fonttitle=\bfseries]
\textbf{Dice Pool:} Attribute + Skill d10s \index{Dice Pool}\\
\textbf{Success:} 6 on each die \index{Success}\\
\textbf{Setback:} 1 on any die gives SB to GM \index{Story Beats}\\
\textbf{DV:} 2 (easy) to 5+ (extreme) \index{Difficulty Value (DV)}\\
\textbf{Harm:} 3-level system with penalties \index{Harm Levels}\\
\textbf{Boons:} 2 on miss, 1 on partial \index{Boons}
\end{tcolorbox}

\section{Narrative Suggestions}

\textbf{Collaborative Scene Framing:} Players may suggest scene elements (weather, NPC reactions, environmental details) that fit the established fiction, with GM approval.

\textbf{Intent-Driven Resolution:} For non-combat actions where success is reasonably assured, the GM may ask players to describe \emph{how} they accomplish their goal rather than rolling dice.

\textbf{Flashback Declarations:} Players can declare a flashback scene to establish that something happened in the past (acquiring an item, making a connection, learning information) by spending 1 Boon and describing the scene.

\textbf{Descriptive Assistance:} Players can assist each other by providing vivid, helpful descriptions of the action, granting a +1 die bonus to the primary actor's roll.

\textbf{Proactive Storytelling:} Players can suggest minor favorable details about their character's circumstances by:
\begin{itemize}
\item Introducing a minor NPC who provides useful information or assistance
\item Establishing that they have a useful item on hand (within reason)
\item Creating a favorable environmental detail
\end{itemize}

These suggestions are subject to GM approval and should enhance rather than overshadow the main narrative. 
