% !TEX root = ../fates_edge_players_guide.tex

\chapter{Core Mechanics} \label{ch:core-mechanics}

\begin{multicols}{2}

In this game, every action matters. The dice don't just tell you if you succeed—they shape the story by introducing tension, risk, and consequence. Fate’s Edge is designed to keep the story moving forward, even when things go wrong. This chapter covers the core resolution system and how every roll changes the narrative.

\section{Basic Dice Mechanics} \index{dice mechanics}

When you attempt a significant action, you roll a pool of six-sided dice (d6s). The size of your pool is determined by two factors:

\[
\text{Dice Pool} = \text{Attribute} + \text{Skill}
\]

\begin{description}
  \item[Attribute] (1–5) Broad traits like strength, wit, or charm. \index{Attribute}
  \item[Skill] (0–5) Training or expertise in a specific area. \index{Skill}
\end{description}

\subsection*{Reading the Dice}

Each die that rolls \textbf{4 or higher} counts as a \textbf{Success}\index{Success}.  
Each die that rolls a \textbf{1} generates a \textbf{Complication Point (CP)}\index{Complication Points}.

\begin{center}
\small
\begin{tabular}{lc}
\toprule
\textbf{Die Result} & \textbf{Effect} \\
\midrule
4--6 & +1 Success \\
1 & +1 Complication Point (CP) \\
2--3 & No effect \\
\bottomrule
\end{tabular}
\end{center}

\paragraph{Example:}  
Lyra the rogue has Agility 3 and Stealth 2. Her dice pool is 5 dice. She rolls: 6, 4, 2, 1, 5. That gives her 3 Successes and 1 Complication Point. The GM sets the Difficulty Value at 2. Lyra succeeds at sneaking past the guards, but the GM now has 1 CP to spend—perhaps the guards hear something faintly and become suspicious.

\section{Difficulty Value (DV)} \index{difficulty value}

Before rolling, the Game Master sets a \textbf{Difficulty Value (DV)}\index{Difficulty Value (DV)}—the target number of Successes needed.

\begin{center}
\small
\begin{tabular}{cl}
\toprule
\textbf{DV} & \textbf{Situation} \\
\midrule
1 & Routine action, no pressure \\
2 & Pressured, mild opposition \\
3 & Difficult, active resistance \\
4+ & Extreme, high stakes \\
\bottomrule
\end{tabular}
\end{center}

\paragraph{Tip for Players:} A DV of 2 is the most common challenge. Assume that if the GM asks you to roll, there is something at stake—whether it is your safety, your resources, or your reputation.

\section{Outcome Matrix} \index{outcome matrix}

Compare your Successes against the DV:

\begin{center}
\small
\begin{tabular}{ll}
\toprule
\textbf{Outcome} & \textbf{Effect} \\
\midrule
\textbf{Clean Success}\index{Success!Clean} & Goal achieved cleanly \\
\textbf{Success \& Cost}\index{Success!with Cost} & Goal achieved with complication \\
\textbf{Partial}\index{Partial} & Progress but with difficult choice \\
\textbf{Miss}\index{Miss} & No progress; complication occurs \\
\bottomrule
\end{tabular}
\end{center}

\paragraph{Player-Facing Example:}  
A fighter swings her sword to disarm a bandit. She rolls 2 Successes against DV 2—a Clean Success. The bandit’s blade clatters away.  
Later, the same fighter tries to kick down a reinforced door with 3 dice against DV 3. She rolls only 2 Successes. This is a Partial. She cracks the door frame, but the noise attracts attention. The story moves forward either way.

\section{Advancement Points} \index{advancement points}

When you \textbf{miss}\index{Miss} on a significant action, you gain \textbf{1 Advancement Point}\index{Advancement Points}. These points are a reward for engaging with the system: you grow when you stumble.

\subsection*{Significant Action Requirements}
A miss awards an Advancement Point only if:
\begin{enumerate}
  \item Procedure was followed correctly
  \item Stakes were clearly stated
  \item Consequence actually occurs
\end{enumerate}

\paragraph{Why This Matters:}  
You are never punished for playing the game boldly. Even failure teaches your character something—mechanically through Advancement Points, narratively through scars, experience, or wisdom.

\subsection*{Carryover Limits}
At the end of each scene, reduce held Advancement Points to a maximum of \textbf{2}. Excess points are lost. This encourages you to spend them rather than hoard.

\subsection*{Conversion to Experience}
Once per session, during downtime, you may convert \textbf{2 Advancement Points} into \textbf{1 Experience Point}\index{Experience Points}. Maximum 2 XP per session this way.

\section{Complication Points (CP)} \index{setback points}

Complication Points are narrative tools the Game Master uses to introduce twists and tension. They keep the story alive with complications and surprises.

\subsection*{What CP Can Do}
The GM may spend CP to:
\begin{itemize}
  \item Introduce new threats or complications
  \item Drain resources (time, gear, position)
  \item Reveal hidden dangers
  \item Cause collateral damage
\end{itemize}

\subsection*{CP Spend Examples}
\begin{itemize}
  \item \textbf{1 CP} — Minor complication, noise, trace
  \item \textbf{2 CP} — Moderate setback, alarm raised
  \item \textbf{3 CP} — Serious trouble, reinforcements arrive
  \item \textbf{4+ CP} — Major turn, scene shifts dramatically
\end{itemize}

\paragraph{Player Advice:}  
Don’t fear Complication Points—they’re not punishment. They are fuel for drama, ensuring the spotlight never dims.

\section{The Harm Clock} \index{harm clock}

Physical injury is tracked on a \textbf{Harm Clock}\index{Harm Clock} with 4 segments:

\begin{center}
\small
\begin{tabular}{ll}
\toprule
\textbf{Harm Level} & \textbf{Effects} \\
\midrule
\textbf{Harm 1} & -1 die on physical actions \\
\textbf{Harm 2} & -2 dice, movement halved \\
\textbf{Harm 3} & -3 dice, incapacitated \\
\textbf{Harm 4} & Critical condition \\
\bottomrule
\end{tabular}
\end{center}

\subsection*{Recovering Harm}
\begin{itemize}
  \item \textbf{Minor treatment} — Remove 1 Harm after rest
  \item \textbf{Proper medical care} — Remove 2 Harm after significant time
  \item \textbf{Extended recovery} — Remove all Harm after days/weeks
\end{itemize}

\paragraph{Example:}  
Jorin the mercenary takes a sword cut (Harm 1). He suffers -1 die to physical actions until treated. After binding the wound and resting, the Harm fades.

\section{Fatigue} \index{fatigue}

Fatigue represents exhaustion, strain, or mental stress. It mirrors Harm in penalties but recovers faster with proper rest, food, or encouragement. Roleplay this—it’s a chance to show your character’s humanity.

\section{Assistance} \index{assistance}

Characters can help each other. One helper per action may provide assistance dice, up to +3 dice total from all sources.

\paragraph{Example:}  
Two thieves cooperate to pick a complex lock. The lead thief has Dexterity 3 + Tools 2 = 5 dice. The helper adds 1 die, making 6. Cooperation often turns failure into tense success.

\section{Asset Use} \index{assets}

Your character’s resources, contacts, or gear—called \textbf{Assets}\index{Assets}—can tilt the odds in your favor.

\begin{itemize}
  \item \textbf{Free use} — Each Asset has a free effect per session
  \item \textbf{Experience activation} — Spend 2 XP for additional uses
  \item \textbf{Advancement Point activation} — Spend 1 Point for dramatic effect
\end{itemize}

\paragraph{Narrative Use:}  
Assets are more than bonuses—they’re hooks for roleplay. A friendly tavernkeeper, a noble’s signet, or a trusty horse might tip the balance at the perfect moment.

\section{Game Structure} \index{game structure}

\subsection*{Time Scales}
\begin{description}
  \item[Moment] A heartbeat, a single action
  \item[Some Time] A few minutes, a short activity
  \item[Significant Time] Hours, extended effort
  \item[Days] Large-scale endeavors
\end{description}

\subsection*{Game Units}
\begin{description}
  \item[Scene] Basic narrative unit, covers specific conflict
  \item[Player Turn] Individual action within a scene
  \item[Session] One game session (3--6 hours)
  \item[Campaign] Entire story arc
\end{description}

\paragraph{Player Perspective:}  
Think in scenes, not minutes. Every scene is a chance to shine. Every session builds toward the long arc of your campaign.

\section{Action Resolution Steps}

\begin{enumerate}
  \item Describe your intent and method
  \item Build dice pool: Attribute + Skill
  \item Roll d6s, count \textbf{Successes}\index{Success} and \textbf{Complication Points}\index{Complication Points}
  \item Compare Successes to \textbf{DV}\index{Difficulty Value (DV)}
  \item Apply outcome from \textbf{matrix}\index{Outcome Matrix}
  \item Game Master spends \textbf{CP}\index{Complication Points!spend} if applicable
  \item Earn \textbf{Advancement Points}\index{Advancement Points} for meaningful engagement
\end{enumerate}

\begin{tcolorbox}[colback=blue!5!white,colframe=blue!75!black,title=Quick Reference,fonttitle=\bfseries]
\textbf{Dice Pool:} Attribute + Skill d6s \index{Dice Pool}\\
\textbf{Success:} 4--6 on each die \index{Success}\\
\textbf{Setback:} 1 on any die gives CP to GM \index{Complication Points}\\
\textbf{DV:} 1 (easy) to 4+ (extreme) \index{Difficulty Value (DV)}\\
\textbf{Harm:} 4-level clock with penalties \index{Harm Clock}\\
\textbf{Advancement:} Gain on significant misses \index{Advancement Points}
\end{tcolorbox}

\end{multicols}
