% !TEX root = ../fates_edge_players_guide.tex

\chapter{Talents and Special Abilities}
\label{ch:talents}

\begin{multicols}{2}

Talents are unique abilities that expand your character's capabilities beyond basic attributes and skills. They represent specialized training, innate gifts, or hard-won expertise that sets your character apart.

\section{Understanding Talents}
\index{talents}

Talents are purchased with \textbf{Experience Points (XP)} and provide special capabilities:
\begin{itemize}
\item They go beyond simple skill bonuses
\item They often have specific \textbf{activation conditions}
\item They may provide \textbf{narrative permissions} (you can try things others cannot)
\item They can define your character's \textbf{unique identity}
\end{itemize}

\subsection*{Talent Costs at a Glance}
\begin{center}
\small
\begin{tabular}{lll}
\toprule
\textbf{Tier} & \textbf{Typical Cost} & \textbf{Scope} \\
\midrule
General & 3--8 XP & Broad, low prerequisites \\
Cultural & 4--10 XP & Background/heritage-tied \\
Advanced & 12--25 XP & High reqs, campaign-defining \\
\bottomrule
\end{tabular}
\end{center}

\paragraph{Activation Types.}
\begin{itemize}
\item \textbf{Passive}: Always on; no action
\item \textbf{Active}: Requires an action or scene focus
\item \textbf{Reactive}: Triggers on a condition (e.g., when surprised)
\end{itemize}

\paragraph{Limits and Economy.}
Unless a talent says otherwise:
\begin{itemize}
\item \textbf{Per Scene} uses refresh at scene end
\item \textbf{Per Session} uses refresh after downtime
\item Some talents allow you to spend \textbf{Advancement Points} to push effects
\end{itemize}

\section{Talent Categories}
\index{talents!categories}

\subsection*{General Talents}
\index{talents!general}
Broad abilities available to any character:
\begin{itemize}
\item \textbf{Cost}: 3--8 XP typically
\item \textbf{Prerequisites}: Minimal or none
\item \textbf{Examples}: Combat reflexes, quick thinking, resistance to elements
\item \textbf{Best for}: Rounding out character capabilities
\end{itemize}

\subsection*{Cultural Talents}
\index{talents!cultural}
Abilities tied to specific backgrounds or heritages:
\begin{itemize}
\item \textbf{Cost}: 4--10 XP typically
\item \textbf{Prerequisites}: Cultural background or specific training
\item \textbf{Examples}: Stone-sense (dwarven), wildcraft (nomadic), courtly grace (noble)
\item \textbf{Best for}: Emphasizing character origins and heritage
\end{itemize}

\subsection*{Advanced Talents}
\index{talents!advanced}
Powerful abilities with significant requirements:
\begin{itemize}
\item \textbf{Cost}: 12--25 XP typically
\item \textbf{Prerequisites}: High attributes, specific skills, or other talents
\item \textbf{Examples}: Master spellcasting, leadership auras, legendary crafts
\item \textbf{Best for}: Defining character pinnacle capabilities
\end{itemize}

\section{Selecting Talents}
\index{talents!selection}

\subsection*{Consider Your Character Concept}
Choose talents that reinforce your character's identity:
\begin{itemize}
\item \textbf{Warrior}: Combat talents, physical enhancements, tactical abilities
\item \textbf{Expert}: Knowledge talents, craft specialties, investigation abilities
\item \textbf{Socialite}: Persuasion talents, network building, influence abilities
\item \textbf{Specialist}: Unique talents matching your specific focus
\end{itemize}

\subsection*{Balance Offense and Defense}
Consider both active and passive talents:
\begin{itemize}
\item \textbf{Active}: Abilities you choose to use (attacks, creations, influences)
\item \textbf{Passive}: Constant benefits (resistance, bonuses, immunities)
\item \textbf{Reactive}: Abilities triggered by events (counterattacks, escapes)
\end{itemize}

\subsection*{Think About Frequency}
Consider how often you'll use each talent:
\begin{itemize}
\item \textbf{Constant use}: Passive benefits always active
\item \textbf{Frequent use}: Several times per session
\item \textbf{Occasional use}: Once per session or scene
\item \textbf{Rare use}: Campaign-defining moments
\end{itemize}

\section{Talent Building Strategies}
\index{talents!building strategies}

\subsection*{The Specialist}
Focus on talents supporting one primary role:
\begin{itemize}
\item Choose talents that \textbf{synergize} with each other
\item Develop a clear specialty identity
\item Become the go-to character for specific challenges
\item \textbf{Risk}: May be less effective outside specialty
\end{itemize}

\subsection*{The Generalist}
Spread talents across multiple areas:
\begin{itemize}
\item Cover different types of challenges
\item Provide support to other party members
\item Adapt to diverse situations
\item \textbf{Risk}: Less peak capability in any area
\end{itemize}

\subsection*{The Combo Builder}
Choose talents that work together in combinations:
\begin{itemize}
\item Look for talent synergies
\item Plan activation sequences
\item Create powerful combined effects
\item \textbf{Risk}: May require specific conditions to be effective
\end{itemize}

\section{Talent Examples}
\index{talents!examples}

\subsection*{Combat Talents}
\begin{description}
\item[Quick Draw] (3 XP, \emph{Active}) --- Draw and ready a weapon as a free action once per scene.
\item[Precise Strike] (4 XP, \emph{Active}) --- Once per scene, ignore armor on one attack if you had \textbf{Controlled} or \textbf{Risky} position.
\item[Combat Reflexes] (5 XP, \emph{Reactive}) --- +1 die on defense rolls when surprised or flanked.
\item[Weapon Mastery] (6 XP, \emph{Passive}) --- Choose a weapon type; +1 die when using it.
\end{description}

\subsection*{Social Talents}
\begin{description}
\item[Silver Tongue] (3 XP, \emph{Passive}) --- +1 die on persuasion attempts when stakes are clearly stated.
\item[Read Emotions] (4 XP, \emph{Active}) --- Once per scene, automatically detect surface emotions in a social exchange.
\item[Command Presence] (5 XP, \emph{Passive}) --- +1 die on leadership and intimidation rolls when you hold \textbf{Risky} or better position.
\item[Network Builder] (6 XP, \emph{Passive}) --- Gain a minor contact in each new settlement visited (GM defines details).
\end{description}

\subsection*{Exploration Talents}
\begin{description}
\item[Keen Senses] (3 XP, \emph{Passive}) --- +1 die on perception checks to spot danger or hidden details.
\item[Wilderness Lore] (4 XP, \emph{Passive}) --- Automatically find food and water in hospitable biomes.
\item[Trackless Step] (5 XP, \emph{Active}) --- Leave no trail for the rest of the scene (or day during travel).
\item[Urban Navigation] (6 XP, \emph{Passive}) --- Never get lost in cities; the GM will offer a shortcut or side path each session.
\end{description}

\subsection*{Knowledge Talents}
\begin{description}
\item[Quick Study] (3 XP, \emph{Passive}) --- Learn new information twice as fast during downtime.
\item[Linguist] (4 XP, \emph{Passive}) --- Learn new languages in half the usual time; +1 die to decipher.
\item[Research Mastery] (5 XP, \emph{Active}) --- +2 dice on a single research or investigation roll, once per scene.
\item[Technical Expert] (6 XP, \emph{Passive}) --- Understand and operate most unfamiliar mechanisms after brief inspection.
\end{description}

\section{Cultural Talent Examples}
\index{talents!cultural}

\subsection*{Dwarven Talents}
\begin{description}
\item[Stone Sense] (5 XP) --- Detect structural weaknesses in stone; +1 die on engineering rolls underground.
\item[Deep Memory] (6 XP) --- Perfect recall of underground layouts once visited.
\item[Ancestral Craft] (7 XP) --- Create items with dwarf-made quality bonuses (GM sets exact tags).
\end{description}

\subsection*{Elven Talents}
\begin{description}
\item[Wild Empathy] (5 XP) --- Communicate simple intent with animals; +1 die on Nature rolls.
\item[Graceful Movement] (6 XP) --- Move silently in natural environments automatically.
\item[Ancient Lore] (7 XP) --- +2 dice on rolls involving ancient history or magic once per scene.
\end{description}

\subsection*{Human Talents}
\begin{description}
\item[Adaptable] (4 XP) --- +1 die when attempting unfamiliar tasks or mixed-method approaches.
\item[Ambitious Drive] (5 XP) --- Reroll one failed roll per session when pursuing declared goals.
\item[Innovative Thinking] (6 XP) --- Propose a plausible tool or method to reframe a challenge; the GM adjusts DV down by 1 once per scene.
\end{description}

\section{Advanced Talent Examples}
\index{talents!advanced}

\subsection*{Combat Mastery}
\begin{description}
\item[Weapon Grand Mastery] (15 XP; Req: Weapon Mastery) --- +2 dice with the chosen weapon type; add a distinctive flourish.
\item[Battlefield Dominance] (18 XP; Req: Body 4, Tactics 2) --- Affect multiple nearby foes with a single attack once per scene.
\item[Untouchable Defense] (20 XP; Req: Combat Reflexes) --- Automatically avoid the first successful attack against you each combat.
\end{description}

\subsection*{Social Influence}
\begin{description}
\item[Master Diplomat] (15 XP; Req: Sway 3) --- In a \textbf{Controlled} social scene, resolve a dispute with one decisive conversation once per session.
\item[Kingmaker] (18 XP; Req: Network Builder) --- Install an ally into a meaningful position of local power over an arc; unlocks faction clocks in your favor.
\item[Legendary Reputation] (20 XP; Req: any 2 Social talents) --- Your name opens doors; begin important social scenes at \textbf{Controlled} unless fiction forbids.
\end{description}

\subsection*{Magical Arts}
\begin{description}
\item[Spell Shaping] (15 XP; Req: Arcana 3) --- Modify non-ritual spell factors (range/scale/targeting) by one step when you Weave.
\item[Ritual Mastery] (18 XP; Req: Ritual practice) --- Perform powerful rituals with reduced risk: the GM spends 1 fewer SP on ritual backlash (min 0).
\item[Arcane Dominance] (20 XP; Req: Spirit 4, Arcana 4) --- Overpower weaker magical effects automatically when you contest them.
\end{description}

\section{Talent Synergies}
\index{talents!synergies}

Some talents work particularly well together:

\subsection*{Combat Synergies}
\begin{itemize}
\item \textbf{Quick Draw + Weapon Mastery}: Ready and strike with a bonus in one beat.
\item \textbf{Precise Strike + Battlefield Dominance}: Pierce armor on multiple targets during your surge.
\item \textbf{Combat Reflexes + Untouchable Defense}: Nearly impossible to surprise or land the first hit on.
\end{itemize}

\subsection*{Social Synergies}
\begin{itemize}
\item \textbf{Silver Tongue + Command Presence}: Charm or command with equal force.
\item \textbf{Read Emotions + Master Diplomat}: Diagnose the room, then end the conflict.
\item \textbf{Network Builder + Kingmaker}: Grow contacts and place them where they matter.
\end{itemize}

\subsection*{Exploration Synergies}
\begin{itemize}
\item \textbf{Keen Senses + Trackless Step}: Find others while leaving no trace.
\item \textbf{Wilderness Lore + Urban Navigation}: Comfortable in wilds and streets alike.
\item \textbf{Quick Study + Research Mastery}: Learn fast, dig deep.
\end{itemize}

\section{Talent Limitations and Balance}
\index{talents!limitations}

\subsection*{Usage Restrictions}
Most talents have limits to maintain game balance:
\begin{itemize}
\item \textbf{Per scene}: Common for strong actives and reactives
\item \textbf{Per session}: Reserved for swingy effects
\item \textbf{Per story arc}: Campaign-defining uses
\item \textbf{Resource cost}: Some require spending Advancement Points or consuming items
\end{itemize}

\subsection*{Prerequisite Systems}
\index{talents!prerequisites}
Advanced talents require meeting certain conditions:
\begin{itemize}
\item \textbf{Attribute minimums}: e.g., Body 4, Wits 3
\item \textbf{Skill requirements}: Specific skills at set levels
\item \textbf{Previous talents}: Foundational picks first
\item \textbf{Story achievements}: Complete relevant quests or milestones
\end{itemize}

\section{Building Your Talent Set}
\index{talents!building}

\subsection*{Early Game (0--40 XP)}
Focus on foundational talents:
\begin{itemize}
\item 1--2 general talents for reliability
\item 1 cultural talent for identity
\item Save XP for attribute and skill improvements
\item Choose talents that work with your core capabilities
\end{itemize}

\subsection*{Mid Game (41--90 XP)}
Develop your specialty:
\begin{itemize}
\item 2--3 synergistic talents
\item Aim toward advanced-prereq milestones
\item Balance active and passive picks
\item Plan for your character's peak moments
\end{itemize}

\subsection*{Late Game (91+ XP)}
Achieve mastery:
\begin{itemize}
\item 1--2 advanced talents defining your apex
\item Picks that create legacy effects
\item Talents that benefit the whole party
\item Prepare for end-game challenges
\end{itemize}

\section{Talent Customization}
\index{talents!customization}

Work with your Game Master to create custom talents:
\begin{itemize}
\item \textbf{Based on story events}: Reflect character experiences
\item \textbf{Unique concepts}: Fit your specific character niche
\item \textbf{Balanced costs}: Match similar scope to existing talents
\item \textbf{Clear rules}: Define activation, effects, and limits
\end{itemize}

\section{Talents and Group Dynamics}
\index{talents!group dynamics}

Consider how your talents complement the party:
\begin{itemize}
\item \textbf{Fill gaps}: Cover party weaknesses
\item \textbf{Synergize}: Coordinate with other players
\item \textbf{Avoid overlap}: Don’t duplicate another character’s specialty
\item \textbf{Support role}: Talents that help the whole group
\end{itemize}

\section{Talent Respecification}
\index{talents!respecification}

If your character concept changes, you may respec talents:
\begin{itemize}
\item \textbf{GM approval required}: Discuss proposed changes
\item \textbf{Downtime cost}: Represent retraining or relearning
\item \textbf{Story justification}: Explain the change in-narrative
\item \textbf{Limited frequency}: Avoid frequent reshuffles
\end{itemize}

\begin{tcolorbox}[colback=purple!5!white,colframe=purple!75!black,title=Talent Selection Guide,fonttitle=\bfseries]
\textbf{Early Game (0--40 XP):}
\begin{itemize}
\item 1--2 general talents (3--6 XP each)
\item 1 cultural talent (4--7 XP)
\item Focus on core-concept support
\end{itemize}

\textbf{Mid Game (41--90 XP):}
\begin{itemize}
\item 2--3 synergistic talents (5--8 XP each)
\item Plan advanced prerequisites
\item Balance active/passive abilities
\end{itemize}

\textbf{Late Game (91+ XP):}
\begin{itemize}
\item 1--2 advanced talents (12--20 XP)
\item Legacy-defining capabilities
\item Party-supporting abilities
\end{itemize}

\textbf{Remember}: Talents should reflect your character's story and growth.
\end{tcolorbox}

\section{Practical Talent Examples}

\subsection*{Example 1: The Guardian}
\begin{itemize}
\item \textbf{Combat Reflexes} (5 XP) --- Better defense when surprised
\item \textbf{Shield Mastery} (6 XP) --- +1 die with shield attacks and defense
\item \textbf{Bodyguard} (4 XP) --- Intercept attacks aimed at allies
\item \textbf{Endurance} (3 XP) --- Resist fatigue and environmental effects
\item \textbf{Total}: \textbf{18 XP} invested in protective talents
\end{itemize}

\subsection*{Example 2: The Scholar}
\begin{itemize}
\item \textbf{Quick Study} (3 XP) --- Learn information quickly
\item \textbf{Research Mastery} (5 XP) --- +2 dice on research rolls
\item \textbf{Linguist} (4 XP) --- Learn languages rapidly
\item \textbf{Technical Expert} (6 XP) --- Understand unfamiliar technology
\item \textbf{Total}: \textbf{18 XP} invested in knowledge talents
\end{itemize}

\subsection*{Example 3: The Face}
\begin{itemize}
\item \textbf{Silver Tongue} (3 XP) --- +1 die on persuasion
\item \textbf{Read Emotions} (4 XP) --- Detect surface emotions
\item \textbf{Network Builder} (6 XP) --- Gain contacts everywhere
\item \textbf{Command Presence} (5 XP) --- +1 die on leadership
\item \textbf{Total}: \textbf{18 XP} invested in social talents
\end{itemize}

\paragraph{Final Note.}
The best talents are those that fit your concept and table playstyle. Choose abilities you’ll enjoy using, that create interesting consequences, and that contribute to your character’s unfolding story.

\end{multicols}
\end{chapter}
