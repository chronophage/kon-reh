% !TEX root = ../fates_edge_players_guide.tex

\chapter{Example Character Concepts}
\label{ch:example-concepts}

\begin{multicols}{2}

This chapter presents example character concepts to illustrate how the game's systems can create diverse and interesting heroes. These are \textbf{examples only}—not prescriptive templates or exhaustive lists. Use them for inspiration, as pre-generated characters, or as starting points for your own unique creations.

\section{Important Disclaimer}
\index{character concepts!disclaimer}

\textbf{These examples are provided for illustrative purposes only.} They demonstrate how the game's mechanics can support different character archetypes and play styles. You are encouraged to:
\begin{itemize}
\item Modify these concepts to fit your preferences
\item Create completely original characters
\item Mix and match elements from different examples
\item Work with your Game Master to develop unique concepts
\end{itemize}

The game system is designed to support a wide variety of character types beyond these examples.

\section{How to Use These Examples}
\index{character concepts!usage}

Each concept includes:
\begin{itemize}
\item \textbf{Concept Overview}: Narrative identity and role
\item \textbf{Mechanical Foundation}: Suggested starting capabilities
\item \textbf{Play Style}: How the character typically engages with challenges
\item \textbf{Development Path}: Potential growth directions
\item \textbf{Story Hooks}: Plot opportunities for the Game Master
\item \textbf{Build Blocks}: A \emph{legal 30 XP} starting build, plus an optional \emph{34 XP} variant using Bonds/Complications (+4 XP)
\end{itemize}

\section{1. The Guardian}
\index{character concepts!guardian}

\textbf{Concept}: A protector who stands between danger and those they've sworn to defend. \emph{Steel in hand, vow in heart.}

\textbf{Typical Inspiration}: Paladins, knights, bodyguards, sworn shields

\textbf{Mechanical Foundation}:
\begin{itemize}
\item \textbf{Primary}: Body, Spirit
\item \textbf{Skills}: Melee, Athletics, Command
\item \textbf{Talents}: Defensive stance, protective instincts
\end{itemize}

\textbf{Play Style}:
\begin{itemize}
\item Frontline combat and protection
\item Drawing attention away from allies
\item Using presence and authority to control situations
\item Taking risks to protect others
\end{itemize}

\textbf{Development Path}:
\begin{itemize}
\item Increase defensive capabilities
\item Develop leadership skills
\item Acquire better protective gear
\item Learn area control abilities
\end{itemize}

\textbf{Story Hooks}:
\begin{itemize}
\item Who or what are they protecting?
\item What oath or duty drives them?
\item What happens if they fail in their protection?
\item What personal costs do they bear for their role?
\end{itemize}

\paragraph{Build Blocks.}
\textbf{Legal Starting Build (30 XP).}
\begin{itemize}
\item \textbf{Attributes} (Cost = rating $\times$ 3 XP): Body 3 (9), Spirit 2 (6), Wits 1 (3), Presence 1 (3) $\rightarrow$ \textbf{21 XP}
\item \textbf{Skills} (Cost = level $\times$ 2 XP): Melee 2 (4), Athletics 1 (2), Command 1 (2) $\rightarrow$ \textbf{8 XP}
\item \textbf{Total}: 29 XP (bank 1 XP)
\end{itemize}
\textbf{With Bonds/Complications (34 XP).}
\begin{itemize}
\item Add \textbf{Talent}: Combat Reflexes (5 XP) using banked 1 + 4 = 5
\item \textbf{Revised Total}: 34 XP
\end{itemize}

\section{2. The Scholar}
\index{character concepts!scholar}

\textbf{Concept}: A seeker of knowledge who uses information as power. \emph{Candlesmoke, marginalia, and dangerous truths.}

\textbf{Typical Inspiration}: Wizards, sages, researchers, historians

\textbf{Mechanical Foundation}:
\begin{itemize}
\item \textbf{Primary}: Wits, Spirit
\item \textbf{Skills}: Lore, Investigation, Arcana
\item \textbf{Talents}: Quick Study, Research Mastery
\end{itemize}

\textbf{Play Style}:
\begin{itemize}
\item Information gathering and analysis
\item Solving puzzles and mysteries
\item Using knowledge to gain advantages
\item Researching solutions between adventures
\end{itemize}

\textbf{Development Path}:
\begin{itemize}
\item Specialize in specific knowledge areas
\item Develop magical or technical capabilities
\item Build research networks
\item Create unique inventions or discoveries
\end{itemize}

\textbf{Story Hooks}:
\begin{itemize}
\item What knowledge are they seeking?
\item What dangerous information might they uncover?
\item How do they handle forbidden knowledge?
\item Who opposes their research?
\end{itemize}

\paragraph{Build Blocks.}
\textbf{Legal Starting Build (30 XP).}
\begin{itemize}
\item \textbf{Attributes}: Wits 3 (9), Spirit 2 (6), Body 1 (3), Presence 1 (3) $\rightarrow$ \textbf{21 XP}
\item \textbf{Skills}: Lore 2 (4), Investigation 1 (2), Arcana 1 (2) $\rightarrow$ \textbf{8 XP}
\item \textbf{Total}: 29 XP (bank 1 XP)
\end{itemize}
\textbf{With Bonds/Complications (34 XP).}
\begin{itemize}
\item Add \textbf{Talent}: Research Mastery (5 XP) using banked 1 + 4
\item \textbf{Revised Total}: 34 XP
\end{itemize}

\section{3. The Scout}
\index{character concepts!scout}

\textbf{Concept}: A wilderness expert who navigates dangerous territories. \emph{Quiet footfalls, hawk eyes, and the long road.}

\textbf{Typical Inspiration}: Rangers, hunters, trackers, explorers

\textbf{Mechanical Foundation}:
\begin{itemize}
\item \textbf{Primary}: Wits, Body
\item \textbf{Skills}: Survival, Stealth, Perception
\item \textbf{Talents}: Wilderness Lore, Keen Senses
\end{itemize}

\textbf{Play Style}:
\begin{itemize}
\item Scouting ahead and gathering intelligence
\item Wilderness survival and navigation
\item Ambush and skirmish tactics
\item Finding paths and resources
\end{itemize}

\textbf{Development Path}:
\begin{itemize}
\item Improve stealth and tracking abilities
\item Develop animal companions or allies
\item Master specific environments
\item Learn advanced survival techniques
\end{itemize}

\textbf{Story Hooks}:
\begin{itemize}
\item What uncharted territory are they exploring?
\item What secrets have they discovered in the wild?
\item How do they balance civilization and wilderness?
\item What threats have they encountered beyond settled lands?
\end{itemize}

\paragraph{Build Blocks.}
\textbf{Legal Starting Build (30 XP).}
\begin{itemize}
\item \textbf{Attributes}: Wits 3 (9), Body 2 (6), Spirit 1 (3), Presence 1 (3) $\rightarrow$ \textbf{21 XP}
\item \textbf{Skills}: Survival 2 (4), Stealth 2 (4) $\rightarrow$ \textbf{8 XP}
\item \textbf{Total}: 29 XP (bank 1 XP)
\end{itemize}
\textbf{With Bonds/Complications (34 XP).}
\begin{itemize}
\item Add \textbf{Asset}: Hidden Cache (Minor Asset, 4 XP) using banked 1 + 4
\item \textbf{Revised Total}: 33 XP (bank 1 XP remains)
\end{itemize}

\section{4. The Diplomat}
\index{character concepts!diplomat}

\textbf{Concept}: A negotiator who resolves conflicts through words and influence. \emph{A smile for the foyer, steel for the parlor.}

\textbf{Typical Inspiration}: Bards, ambassadors, merchants, politicians

\textbf{Mechanical Foundation}:
\begin{itemize}
\item \textbf{Primary}: Presence, Wits
\item \textbf{Skills}: Sway, Investigation, Lore
\item \textbf{Talents}: Silver Tongue, Read Emotions
\end{itemize}

\textbf{Play Style}:
\begin{itemize}
\item Social interaction and negotiation
\item Gathering information through contacts
\item Resolving conflicts without violence
\item Building alliances and relationships
\end{itemize}

\textbf{Development Path}:
\begin{itemize}
\item Expand social influence and networks
\item Develop economic or political power
\item Learn cultural specialties
\item Master manipulation or inspiration techniques
\end{itemize}

\textbf{Story Hooks}:
\begin{itemize}
\item What major conflict are they trying to resolve?
\item What alliances have they built or broken?
\item How do they handle betrayal or failed negotiations?
\item What personal relationships affect their diplomacy?
\end{itemize}

\paragraph{Build Blocks.}
\textbf{Legal Starting Build (30 XP).}
\begin{itemize}
\item \textbf{Attributes}: Presence 3 (9), Wits 2 (6), Spirit 1 (3), Body 1 (3) $\rightarrow$ \textbf{21 XP}
\item \textbf{Skills}: Sway 2 (4), Investigation 1 (2), Lore 1 (2) $\rightarrow$ \textbf{8 XP}
\item \textbf{Total}: 29 XP (bank 1 XP)
\end{itemize}
\textbf{With Bonds/Complications (34 XP).}
\begin{itemize}
\item Add \textbf{Talent}: Silver Tongue (3 XP); add \textbf{Skill} Lore +1 (now 2) for 2 XP using remaining 1 + 4 = 5
\item \textbf{Revised Totals}: Skills Sway 2, Investigation 1, Lore 2; \textbf{34 XP exact}
\end{itemize}

\section{5. The Specialist}
\index{character concepts!specialist}

\textbf{Concept}: An expert with unique capabilities beyond typical roles. \emph{The right tool, the right touch, at the right time.}

\textbf{Typical Inspiration}: Artisans, healers, engineers, spies

\textbf{Mechanical Foundation}:
\begin{itemize}
\item \textbf{Primary}: Varies by specialty (often Wits or Body)
\item \textbf{Skills}: One specialty at focus, plus two support skills
\item \textbf{Talents}: Unique techniques that unlock niche actions
\end{itemize}

\textbf{Play Style}:
\begin{itemize}
\item Solving problems with unique expertise
\item Creating or repairing specialized items
\item Providing services others cannot
\item Using niche knowledge for advantage
\end{itemize}

\textbf{Development Path}:
\begin{itemize}
\item Master their specialty area
\item Develop related capabilities
\item Build reputation and clientele
\item Create unique inventions or methods
\end{itemize}

\textbf{Story Hooks}:
\begin{itemize}
\item What makes their specialty unique or valuable?
\item How did they acquire their special skills?
\item What problems require their specific expertise?
\item Who seeks to control or exploit their abilities?
\end{itemize}

\paragraph{Build Blocks (Artificer example).}
\textbf{Legal Starting Build (30 XP).}
\begin{itemize}
\item \textbf{Attributes}: Wits 3 (9), Body 2 (6), Presence 1 (3), Spirit 1 (3) $\rightarrow$ \textbf{21 XP}
\item \textbf{Skills}: Craft 2 (4), Mechanics 2 (4) $\rightarrow$ \textbf{8 XP}
\item \textbf{Total}: 29 XP (bank 1 XP)
\end{itemize}
\textbf{With Bonds/Complications (34 XP).}
\begin{itemize}
\item Add \textbf{Talent}: Technical Expert (6 XP) using banked 1 + 4 = 5 (insufficient) \emph{or} choose \textbf{Quick Study} (3 XP) and bank 2 XP
\item \textbf{Revised Total}: 32 XP (bank 2 XP for an early upgrade)
\end{itemize}

\section{6. The Survivor}
\index{character concepts!survivor}

\textbf{Concept}: Someone who has endured hardship and developed resilience. \emph{Scars are maps; read them well.}

\textbf{Typical Inspiration}: Veterans, refugees, outcasts, hardened adventurers

\textbf{Mechanical Foundation}:
\begin{itemize}
\item \textbf{Primary}: Spirit, Body
\item \textbf{Skills}: Endurance, Survival, (optionally) Perception/Insight
\item \textbf{Talents}: Endurance, Adaptable
\end{itemize}

\textbf{Play Style}:
\begin{itemize}
\item Enduring difficult conditions
\item Overcoming physical and mental challenges
\item Using experience to avoid dangers
\item Helping others survive hardships
\end{itemize}

\textbf{Development Path}:
\begin{itemize}
\item Improve physical and mental resilience
\item Develop survival-related skills
\item Acquire better equipment and resources
\item Learn to teach survival to others
\end{itemize}

\textbf{Story Hooks}:
\begin{itemize}
\item What trauma or hardship have they survived?
\item How has their past shaped their present?
\item What survival skills have saved them repeatedly?
\item How do they help others facing similar challenges?
\end{itemize}

\paragraph{Build Blocks.}
\textbf{Legal Starting Build (30 XP).}
\begin{itemize}
\item \textbf{Attributes}: Spirit 3 (9), Body 2 (6), Wits 1 (3), Presence 1 (3) $\rightarrow$ \textbf{21 XP}
\item \textbf{Skills}: Endurance 2 (4), Survival 2 (4) $\rightarrow$ \textbf{8 XP}
\item \textbf{Total}: 29 XP (bank 1 XP)
\end{itemize}
\textbf{With Bonds/Complications (34 XP).}
\begin{itemize}
\item Add \textbf{Talent}: Endurance (3 XP) using banked 1 + 4; bank 2 XP
\item \textbf{Revised Total}: 32 XP (bank 2 XP)
\end{itemize}

\section{7. The Innovator}
\index{character concepts!innovator}

\textbf{Concept}: A creative problem-solver who finds new solutions. \emph{Blueprints on napkins, tomorrow in your pocket.}

\textbf{Typical Inspiration}: Inventors, strategists, reformers, visionaries

\textbf{Mechanical Foundation}:
\begin{itemize}
\item \textbf{Primary}: Wits, Presence
\item \textbf{Skills}: Craft, Lore, Investigation
\item \textbf{Talents}: Creative/Innovative thinking, Quick Study
\end{itemize}

\textbf{Play Style}:
\begin{itemize}
\item Finding novel solutions to problems
\item Creating new devices or methods
\item Analyzing systems for improvement
\item Convincing others to try new approaches
\end{itemize}

\textbf{Development Path}:
\begin{itemize}
\item Develop specific technical specialties
\item Create increasingly complex inventions
\item Build support for innovative ideas
\item Overcome resistance to change
\end{itemize}

\textbf{Story Hooks}:
\begin{itemize}
\item What problem are they trying to solve?
\item How do others react to their innovations?
\item What unintended consequences might their creations have?
\item Who benefits or suffers from their changes?
\end{itemize}

\paragraph{Build Blocks.}
\textbf{Legal Starting Build (30 XP).}
\begin{itemize}
\item \textbf{Attributes}: Wits 3 (9), Presence 2 (6), Body 1 (3), Spirit 1 (3) $\rightarrow$ \textbf{21 XP}
\item \textbf{Skills}: Craft 2 (4), Lore 2 (4) $\rightarrow$ \textbf{8 XP}
\item \textbf{Total}: 29 XP (bank 1 XP)
\end{itemize}
\textbf{With Bonds/Complications (34 XP).}
\begin{itemize}
\item Add \textbf{Talent}: Quick Study (3 XP); bank 2 XP
\item \textbf{Revised Total}: 32 XP (bank 2 XP)
\end{itemize}

\section{8. The Networker}
\index{character concepts!networker}

\textbf{Concept}: Someone who builds and leverages social connections. \emph{A web of favors, a chorus of names.}

\textbf{Typical Inspiration}: Merchants, spies, socialites, community leaders

\textbf{Mechanical Foundation}:
\begin{itemize}
\item \textbf{Primary}: Presence, Wits
\item \textbf{Skills}: Sway, Lore, (optionally) Command/Deception
\item \textbf{Talents}: Network Builder, Command Presence / Silver Tongue
\end{itemize}

\textbf{Play Style}:
\begin{itemize}
\item Building and maintaining relationships
\item Gathering information through contacts
\item Leveraging social influence
\item Navigating complex social situations
\end{itemize}

\textbf{Development Path}:
\begin{itemize}
\item Expand social network and influence
\item Develop specific community ties
\item Acquire political or economic power
\item Master manipulation or leadership techniques
\end{itemize}

\textbf{Story Hooks}:
\begin{itemize}
\item What networks or communities are they part of?
\item How do they balance multiple relationships?
\item What happens when loyalties conflict?
\item How do they handle betrayal or broken trust?
\end{itemize}

\paragraph{Build Blocks.}
\textbf{Legal Starting Build (30 XP).}
\begin{itemize}
\item \textbf{Attributes}: Presence 3 (9), Wits 2 (6), Body 1 (3), Spirit 1 (3) $\rightarrow$ \textbf{21 XP}
\item \textbf{Skills}: Sway 2 (4), Lore 2 (4) $\rightarrow$ \textbf{8 XP}
\item \textbf{Total}: 29 XP (bank 1 XP)
\end{itemize}
\textbf{With Bonds/Complications (34 XP).}
\begin{itemize}
\item Add \textbf{Talent}: Silver Tongue (3 XP); bank 2 XP
\item \textbf{Revised Total}: 32 XP (bank 2 XP)
\end{itemize}

\section{Creating Your Own Concept}
\index{character concepts!creation}

\subsection*{Start with Narrative}
\begin{itemize}
\item What is your character's background and motivation?
\item What role do they play in their community or society?
\item What relationships are important to them?
\item What goals are they pursuing?
\end{itemize}

\subsection*{Add Mechanical Support}
\begin{itemize}
\item Choose attributes that support your concept
\item Select skills that reflect their training and experience
\item Consider talents that provide unique capabilities
\item Think about assets that represent their resources
\end{itemize}

\subsection*{Consider Group Role}
\begin{itemize}
\item How does your concept complement other party members?
\item What gaps in group capability can you fill?
\item What unique contributions can you make?
\item How will you work with other characters?
\end{itemize}

\subsection*{Plan for Growth}
\begin{itemize}
\item What short-term improvements make sense?
\item What long-term development aligns with your concept?
\item How might your character change over time?
\item What legacy do you want to build?
\end{itemize}

\begin{tcolorbox}[colback=green!5!white,colframe=green!75!black,title=Character Concept Worksheet,fonttitle=\bfseries]
\textbf{Narrative Elements:}
\begin{itemize}
\item Concept: \_\_\_\_\_\_\_\_\_\_\_\_\_\_\_\_\_\_\_\_\_\_
\item Motivation: \_\_\_\_\_\_\_\_\_\_\_\_\_\_\_\_\_\_\_\_\_\_
\item Background: \_\_\_\_\_\_\_\_\_\_\_\_\_\_\_\_\_\_\_\_\_\_
\item Relationships: \_\_\_\_\_\_\_\_\_\_\_\_\_\_\_\_\_\_\_\_\_\_
\end{itemize}

\textbf{Mechanical Foundation:}
\begin{itemize}
\item Primary Attributes: \_\_\_\_\_\_\_\_\_\_\_\_\_\_\_\_\_\_\_\_\_\_
\item Key Skills: \_\_\_\_\_\_\_\_\_\_\_\_\_\_\_\_\_\_\_\_\_\_
\item Starting Talents: \_\_\_\_\_\_\_\_\_\_\_\_\_\_\_\_\_\_\_\_\_\_
\item Initial Assets: \_\_\_\_\_\_\_\_\_\_\_\_\_\_\_\_\_\_\_\_\_\_
\end{itemize}

\textbf{Development Plan:}
\begin{itemize}
\item Short-term goals: \_\_\_\_\_\_\_\_\_\_\_\_\_\_\_\_\_\_\_\_\_\_
\item Long-term vision: \_\_\_\_\_\_\_\_\_\_\_\_\_\_\_\_\_\_\_\_\_\_
\end{itemize}
\end{tcolorbox}

\section{Final Notes}

Remember that these examples are starting points, not limitations. The most interesting characters often combine elements from multiple concepts or create entirely new approaches. Work with your Game Master to ensure your character concept fits the campaign and provides engaging story opportunities.

The best characters are those that you find interesting to play and that contribute to an enjoyable experience for everyone at the table.

\end{multicols}

