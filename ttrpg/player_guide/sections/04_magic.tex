\chapter{Magic and Special Abilities} \label{ch:magic}

Magic in this game is powerful but dangerous—a negotiation with reality itself that always carries risks. This chapter covers the core magical systems: standard \textbf{spellcasting}\index{magic!spellcasting}, \textbf{ritual magic}\index{magic!rituals}, and special \textbf{pact-based abilities}\index{magic!pacts}. Throughout, look for examples and player-facing tips to keep the fiction front and center.

\section{The Nature of Magic} \index{magic!nature}

Magic is not a safe tool but a dangerous force:
\begin{itemize}
\item \textbf{Powerful}\index{magic!power}: Can reshape battles, stories, or even the world
\item \textbf{Controlled}\index{magic!risk}: Every use generates \textbf{Story Beats (SB)}\index{Story Beats} that manifest as backlash
\item \textbf{Thematic}\index{magic!themes}: Effects and consequences align with the type of magic used
\item \textbf{Volatile}\index{magic!volatility}: Never fully predictable or controllable
\item \textbf{Narrative}\index{magic!narrative}: Casting is always a significant story moment
\end{itemize}

\paragraph{Table Vignette:}
\emph{``I can hold the avalanche,'' says Mira, fingers trembling. ``But something will answer.''} The party nods—risk accepted, stakes clear.

\section{Basic Spellcasting} \index{magic!spellcasting}

All spellcasting follows the standard action resolution system but with additional considerations for magical effects.

\subsection*{The Casting Process}
\begin{enumerate}
\item \textbf{Declare Intent}\index{magic!intent}: What you want the magic to achieve
\item \textbf{Choose Approach}\index{magic!approach}: Which magical skill and method you'll use
\item \textbf{Set Position}\index{position}: \textbf{Dominant}\index{position!Dominant}, \textbf{Controlled}\index{position!Controlled}, or \textbf{Desperate}\index{position!Desperate} based on circumstances
\item \textbf{Roll}\index{dice pools}: Attribute + Magical Skill
\item \textbf{Resolve}\index{outcomes}: Apply outcomes with magical consequences
\end{enumerate}

\subsection*{Magical Skills}
Common magical skills include:
\begin{itemize}
\item \textbf{Arcana}\index{Arcana}: General magical knowledge and theory
\item \textbf{Elemental Magic}\index{Elemental Magic}: Fire, water, earth, air manipulation
\item \textbf{Spiritual Magic}\index{Spiritual Magic}: Communing with spirits, divine magic
\item \textbf{Mental Magic}\index{Mental Magic}: Telepathy, illusion, mind affecting
\item \textbf{Healing Magic}\index{Healing Magic}: Restoration, purification, life magic
\end{itemize}

\paragraph{Player Tip:}
State a clear \textbf{intent}\index{magic!intent} and a vivid \textbf{method}\index{magic!approach}. The more concrete the fiction, the easier it is to set fair \textbf{DV}\index{Difficulty Value (DV)} and meaningful consequences.

\section{The Casting Loop} \index{magic!casting loop}

For more significant magical effects, use the structured Casting Loop requiring two actions.

\subsection*{Phase 1: Weave} \index{magic!weave}
Shape the magical effect:
\begin{itemize}
\item Player builds dice pool and rolls
\item On success, they stabilize the spell's form
\item Any 1 rolled may cause narrative backlash related to the Element
\end{itemize}

\subsection*{Phase 2: Cast} \index{magic!cast}
Channel the effect into the world:
\begin{itemize}
\item A second roll channels the effect
\item Backlash: Any 1 rolled may cause narrative backlash related to the Element
\end{itemize}

\paragraph{Designer Note:}
The \textbf{Casting Loop}\index{magic!casting loop} requires the \textbf{Caster's Gift}\index{Caster's Gift} talent (2 XP) and creates spotlight tension: describe effect now, risk \textbf{Backlash} on each roll.

\section{Backlash Severity} \index{magic!backlash}

\begin{center}
\small
\begin{tabular}{ll}
\toprule
\textbf{Roll Result} & \textbf{Backlash Trigger} \\
\midrule
Partial/Miss & Minor backlash (choose one) \\
Miss & Major backlash (choose two) \\
Hit with two or more 1s & Minor backlash alongside success \\
\bottomrule
\end{tabular}
\end{center}

\section{Magical Arts and Traditions} \index{magic!arts}

Different cultures and traditions approach magic differently.

\subsection*{Elemental Magic} \index{Elemental Magic}
Manipulation of natural forces:
\begin{itemize}
\item \textbf{Fire Magic}\index{Elemental Magic!Fire}: Heat, light, transformation, destruction
\item \textbf{Water Magic}\index{Elemental Magic!Water}: Flow, healing, divination, adaptation
\item \textbf{Earth Magic}\index{Elemental Magic!Earth}: Stability, protection, growth, strength
\item \textbf{Air Magic}\index{Elemental Magic!Air}: Movement, communication, freedom, change
\end{itemize}

\subsection*{Spiritual Magic} \index{Spiritual Magic}
Interaction with intangible forces:
\begin{itemize}
\item \textbf{Divine Magic}\index{Spiritual Magic!Divine}: Power from gods or higher powers
\item \textbf{Spirit Magic}\index{Spiritual Magic!Spirit}: Communing with nature spirits or ancestors
\item \textbf{Necromancy}\index{Spiritual Magic!Necromancy}: Interaction with death and the departed
\item \textbf{Protection Magic}\index{Spiritual Magic!Protection}: Wards, blessings, purification
\end{itemize}

\subsection*{Mental Magic} \index{Mental Magic}
Affecting minds and perceptions:
\begin{itemize}
\item \textbf{Illusion}\index{Mental Magic!Illusion}: Creating false perceptions and images
\item \textbf{Telepathy}\index{Mental Magic!Telepathy}: Mind reading and communication
\item \textbf{Enchantment}\index{Mental Magic!Enchantment}: Influencing thoughts and emotions
\item \textbf{Divination}\index{Mental Magic!Divination}: Gaining knowledge through supernatural means
\end{itemize}

\paragraph{Vignette:}
\emph{The candles lean toward the oracle's breath. ``Ask,'' she whispers, ``but truth is sharp.''}

\section{Ritual Magic} \index{magic!rituals}

Rituals take Significant Time (typically 10-30 minutes) for powerful effects.

\subsection*{Ritual Requirements}
\begin{itemize}
\item \textbf{Time}\index{rituals!time}: Significant Time (typically 10-30 minutes)
\item \textbf{Preparation}\index{rituals!preparation}: Specific materials, locations, or conditions
\item \textbf{Focus}\index{rituals!focus}: Undisturbed concentration and coordination
\end{itemize}

\subsection*{Ritual Procedure}
\begin{enumerate}
\item \textbf{Preparation}\index{rituals!preparation}: Gather components, prepare space, focus intent
\item \textbf{Invocation}\index{rituals!invocation}: Perform the Rite as a ritual
\item \textbf{Completion}\index{rituals!completion}: Effect manifests, always marks +1 Obligation
\end{enumerate}

\subsection*{Ritual Benefits and Risks}
\begin{itemize}
\item \textbf{Benefits}\index{rituals!benefits}: Safe casting, no Push It option
\item \textbf{Risks}\index{rituals!risks}: Time investment, Obligation cost, environmental requirements
\end{itemize}

\section{Rites and Pact Magic} \index{magic!rites} \index{magic!pacts}

Rites are precise magical effects gained through \textbf{pacts}\index{pacts} with powerful entities. There are two main paths to accessing Rites:

\subsection*{The Runekeeper (Rites Path)}
\begin{itemize}
\item Requires Patron + Thiasos (Familiar) + Codex (4 XP)
\item Accesses that Patron's full Rite list
\item Structured, powerful, but accrues \textbf{Obligation}
\item Can Push Rites once per scene for +1 Obligation
\end{itemize}

\begin{fatebox}[Invoker Path Features]
  \begin{tabularx}{\textwidth}{lX}
  \toprule
  \textbf{Feature} & \textbf{Description and Limitations} \\
  \midrule
  Invoker's Grimoire & Major Talent, 6 XP. Grants knowledge of Ritual Magic theory and access to perform a limited number of Rites. \\
  Ritual Repertoire & Start with knowledge of \textbf{2} Low or Standard Rites from any Patrons you research. Learn new Rites through Downtime study (see below). \\
  Ritual Invocation & Takes \(\text{DV}\) rounds (default 2--3 rounds). Requires specific components/materials. \\
  Base Cost & Mark \textbf{+1 Obligation} when you successfully resolve any known Rite (Low or Standard). \textit{(High-Power/High Rites are normally unavailable; if the Keeper permits, treat their \emph{base} Obligation as +2.)}\\
  Symbol Enhancement & Possessing the correct Patron's Symbol for a Rite you are casting reduces its \textbf{DV by 1} and its \textbf{Obligation cost by 1} (minimum 0). Only one Symbol may apply to a given Rite. \\
  \textbf{No Symbol (Explicit Penalties)} & You may attempt the Rite without the Patron's Symbol, but suffer: \textbf{+1 DV}, \textbf{+1 Obligation} (in addition to Base), and \textbf{+1 round} casting time. On \emph{Partial/Failure}, generate \textbf{+1 extra SB}. \\
  Symbol Display & The Symbol must be visible/active throughout the ritual. If it is concealed, disrupted, or removed mid-cast: immediately \textbf{+1 DV}; on Failure, apply \emph{Backlash} (see below). \\
  Crack the Seal & Desperate technique. Instantly cast any known Rite by setting the relevant Symbol to \textsc{Compromised}. Mark \textbf{+2 Obligation} (\textbf{+3} for High-Power Rites). Does not reduce Base Obligation below 0. \\
  Optional Push & Invokers may \emph{Push} a Rite: choose one (\(+2\) dice \emph{or} +1 Effect \emph{or} resolve one round faster). Always mark \textbf{+1 Obligation} \emph{and} generate \textbf{1 SB}, in addition to other costs. \\
  Cross-Resonance & If you cast Rites from \emph{different Patrons} in the same scene, each Patron after the first adds \textbf{+1 DV} to that Rite. \\
  \bottomrule
  \end{tabularx}
  \end{fatebox}
  
  \paragraph{Symbol States \& Repair}
  \begin{itemize}
    \item \textsc{Compromised:} A Symbol set to \textsc{Compromised} (e.g., via \emph{Crack the Seal}) provides \emph{no} DV/Obligation reduction until repaired. Casting with a \textsc{Compromised} Symbol imposes \(-1\) die on the Casting Test.
    \item \textsc{Shattered:} If you \emph{Crack the Seal} again while the Symbol is \textsc{Compromised}, it becomes \textsc{Shattered} and cannot be used until replaced (Asset lost).
    \item \textbf{Repair (Downtime):} 1 day of focused work and a \emph{Craft or Lore + Tinker} test vs.\ DV~3. Success: clear \textsc{Compromised}. Failure: no progress. Alternatively, spend \textbf{1 XP} to auto-repair.
  \end{itemize}
  
  \paragraph{Backlash \& Failure (Explicit)}
  \begin{itemize}
    \item \textbf{Success:} Rite resolves; apply Base/added Obligation and any SB from Push or No-Symbol clauses.
    \item \textbf{Partial:} Effect \(-1\) step \emph{or} shortened duration; mark \textbf{Fatigue 1}. If cast \emph{without} a Symbol, Keeper gains \textbf{+1 SB} (in addition to normal SB generation).
    \item \textbf{Failure:} No effect; mark \textbf{Fatigue 1}; Keeper gains \textbf{+1 SB}. Then test \emph{Spirit + Resolve} vs.\ DV~3:
      \begin{itemize}
        \item On Fail: suffer \textbf{Harm 1 (Shock)} or start \textbf{Backlash Static [4]} (Keeper's choice).
        \item If the Symbol was disrupted/hidden mid-cast \emph{or} you \emph{Cracked the Seal}: upgrade to \textbf{Harm 2 (Shock)}.
      \end{itemize}
    \item \textbf{Interrupted:} Harm, Silence, or disruption before resolution counts as \emph{Failure}.
  \end{itemize}
  
  \textbf{Example:} Magus Vex, bearing the \textbf{Invoker's Grimoire}, has studied the rites of Raéyn and the Sealed Gate. He knows Raéyn's \emph{Whispering Currents} (Low) and the Sealed Gate's \emph{Circle of Denial} (Standard). Faced with a collapsing tunnel, he attempts the Sealed Gate's ritual. It's a Standard Rite, so \textbf{DV 3}, taking \textbf{3 rounds}, and costs \textbf{+1 Obligation}. He has the Sealed Gate's Symbol, reducing the DV to \textbf{2} and the Obligation cost to \textbf{0}. When ambushed, he needs quick protection. He \textbf{Cracks the Seal} on the \emph{Circle of Denial}. The Symbol becomes \textsc{Compromised}, the Rite is instant, and he marks \textbf{+2 Obligation}. Later, needing to bind a particularly strong foe, he \textbf{Pushes} the Rite, marking an additional \textbf{+1 Obligation} and generating \textbf{1 SB}; the barrier strengthens. If he tried a Raéyn Rite afterwards in the same scene, \emph{Cross-Resonance} would add \textbf{+1 DV} to that casting.
  
  \subsubsection*{Learning New Rites}
  An Invoker can expand their \textbf{Ritual Repertoire} through dedicated study during \textbf{Downtime}.
  \begin{itemize}
      \item \textbf{Cost:} 1 week of Downtime + 2 XP.
      \item \textbf{Requirement:} Access to texts, a teacher, or direct observation of the Rite being performed by another adept.
      \item \textbf{Test:} \emph{Lore + Investigation} (or a relevant skill) vs.\ DV~3--5 (based on Rite rarity/complexity).
      \item \textbf{Success:} Add the Rite to your Ritual Repertoire.
      \item \textbf{Failure:} Cannot learn this specific Rite for a significant time (GM discretion). The Keeper may set a relevant Complication (e.g., \emph{Forbidden Knowledge Pursued}).
  \end{itemize}
  
  \subsubsection*{Symbols as Assets}
  \begin{itemize}
      \item A Patron's Symbol is a \textbf{Minor Asset (4 XP)} whose primary value is as a \textbf{ritual focus/component}.
      \item You \emph{can} attempt any ritual \textbf{without} the Symbol, but you incur these \textbf{No Symbol} penalties: \textbf{+1 DV} \emph{(and therefore +1 round to cast, since casting time = DV rounds)}, \textbf{+1 Obligation} \emph{(in addition to Base)}, and on \emph{Partial/Failure} the Keeper gains \textbf{+1 extra SB}.    \item Symbols can be \textbf{maintained/upgraded} like other Assets. Example upgrades: \emph{Hardened} (ignore the first application of \textsc{Compromised} per session), \emph{Bright} (treat as \emph{visible} for Symbol Display while concealed on your person).
  \end{itemize}

  \subsection*{Invoker (Ritualist's Path)}\index{Invoker}\index{Ritualist}\index{Grimoire}

Requires the \textbf{Invoker's Grimoire} talent (6 XP) and study of specific rites. Grants deep knowledge of ritual magic and the ability to perform Rites from multiple Patrons. Symbols are potent tools that enhance this knowledge.

\begin{fatebox}[Invoker Path Features]
\begin{tabularx}{\textwidth}{lX}
\toprule
\textbf{Feature} & \textbf{Description and Limitations} \\
\midrule
Invoker's Grimoire & Major Talent, 6 XP. Grants knowledge of Ritual Magic theory and access to perform a limited number of Rites. \\
Ritual Repertoire & Start with knowledge of \textbf{2} Low or Standard Rites from any Patrons you research. Learn new Rites through Downtime study (see below). \\
Ritual Invocation & Takes \(\text{DV}\) rounds (default 2--3 rounds). Requires specific components/materials. \\
Base Cost & Mark \textbf{+1 Obligation} when you successfully resolve any known Rite (Low or Standard). \textit{(High-Power/High Rites are normally unavailable; if the Keeper permits, treat their \emph{base} Obligation as +2.)}\\
Symbol Enhancement & Possessing the correct Patron's Symbol for a Rite you are casting reduces its \textbf{DV by 1} and its \textbf{Obligation cost by 1} (minimum 0). Only one Symbol may apply to a given Rite. \\
\textbf{No Symbol (Explicit Penalties)} & You may attempt the Rite without the Patron's Symbol, but suffer: \textbf{+1 DV}, \textbf{+1 Obligation} (in addition to Base), and \textbf{+1 round} casting time. On \emph{Partial/Failure}, generate \textbf{+1 extra SB}. \\
Symbol Display & The Symbol must be visible/active throughout the ritual. If it is concealed, disrupted, or removed mid-cast: immediately \textbf{+1 DV}; on Failure, apply \emph{Backlash} (see below). \\
Crack the Seal & Desperate technique. Instantly cast any known Rite by setting the relevant Symbol to \textsc{Compromised}. Mark \textbf{+2 Obligation} (\textbf{+3} for High-Power Rites). Does not reduce Base Obligation below 0. \\
Optional Push & Invokers may \emph{Push} a Rite: choose one (\(+2\) dice \emph{or} +1 Effect \emph{or} resolve one round faster). Always mark \textbf{+1 Obligation} \emph{and} generate \textbf{1 SB}, in addition to other costs. \\
Cross-Resonance & If you cast Rites from \emph{different Patrons} in the same scene, each Patron after the first adds \textbf{+1 DV} to that Rite. \\
\bottomrule
\end{tabularx}
\end{fatebox}

\paragraph{Symbol States \& Repair}
\begin{itemize}
  \item \textsc{Compromised:} A Symbol set to \textsc{Compromised} (e.g., via \emph{Crack the Seal}) provides \emph{no} DV/Obligation reduction until repaired. Casting with a \textsc{Compromised} Symbol imposes \(-1\) die on the Casting Test.
  \item \textsc{Shattered:} If you \emph{Crack the Seal} again while the Symbol is \textsc{Compromised}, it becomes \textsc{Shattered} and cannot be used until replaced (Asset lost).
  \item \textbf{Repair (Downtime):} 1 day of focused work and a \emph{Craft or Lore + Tinker} test vs.\ DV~3. Success: clear \textsc{Compromised}. Failure: no progress. Alternatively, spend \textbf{1 XP} to auto-repair.
\end{itemize}

\paragraph{Backlash \& Failure (Explicit)}
\begin{itemize}
  \item \textbf{Success:} Rite resolves; apply Base/added Obligation and any SB from Push or No-Symbol clauses.
  \item \textbf{Partial:} Effect \(-1\) step \emph{or} shortened duration; mark \textbf{Fatigue 1}. If cast \emph{without} a Symbol, Keeper gains \textbf{+1 SB} (in addition to normal SB generation).
  \item \textbf{Failure:} No effect; mark \textbf{Fatigue 1}; Keeper gains \textbf{+1 SB}. Then test \emph{Spirit + Resolve} vs.\ DV~3:
    \begin{itemize}
      \item On Fail: suffer \textbf{Harm 1 (Shock)} or start \textbf{Backlash Static [4]} (Keeper's choice).
      \item If the Symbol was disrupted/hidden mid-cast \emph{or} you \emph{Cracked the Seal}: upgrade to \textbf{Harm 2 (Shock)}.
    \end{itemize}
  \item \textbf{Interrupted:} Harm, Silence, or disruption before resolution counts as \emph{Failure}.
\end{itemize}

\textbf{Example:} Magus Vex, bearing the \textbf{Invoker's Grimoire}, has studied the rites of Raéyn and the Sealed Gate. He knows Raéyn's \emph{Whispering Currents} (Low) and the Sealed Gate's \emph{Circle of Denial} (Standard). Faced with a collapsing tunnel, he attempts the Sealed Gate's ritual. It's a Standard Rite, so \textbf{DV 3}, taking \textbf{3 rounds}, and costs \textbf{+1 Obligation}. He has the Sealed Gate's Symbol, reducing the DV to \textbf{2} and the Obligation cost to \textbf{0}. When ambushed, he needs quick protection. He \textbf{Cracks the Seal} on the \emph{Circle of Denial}. The Symbol becomes \textsc{Compromised}, the Rite is instant, and he marks \textbf{+2 Obligation}. Later, needing to bind a particularly strong foe, he \textbf{Pushes} the Rite, marking an additional \textbf{+1 Obligation} and generating \textbf{1 SB}; the barrier strengthens. If he tried a Raéyn Rite afterwards in the same scene, \emph{Cross-Resonance} would add \textbf{+1 DV} to that casting.

\subsubsection*{Learning New Rites}
An Invoker can expand their \textbf{Ritual Repertoire} through dedicated study during \textbf{Downtime}.
\begin{itemize}
    \item \textbf{Cost:} 1 week of Downtime + 2 XP.
    \item \textbf{Requirement:} Access to texts, a teacher, or direct observation of the Rite being performed by another adept.
    \item \textbf{Test:} \emph{Lore + Investigation} (or a relevant skill) vs.\ DV~3--5 (based on Rite rarity/complexity).
    \item \textbf{Success:} Add the Rite to your Ritual Repertoire.
    \item \textbf{Failure:} Cannot learn this specific Rite for a significant time (GM discretion). The Keeper may set a relevant Complication (e.g., \emph{Forbidden Knowledge Pursued}).
\end{itemize}

\subsubsection*{Symbols as Assets}
\begin{itemize}
    \item A Patron's Symbol is a \textbf{Minor Asset (4 XP)} whose primary value is as a \textbf{ritual focus/component}.
    \item You \emph{can} attempt any ritual \textbf{without} the Symbol, but you incur these \textbf{No Symbol} penalties: \textbf{+1 DV} \emph{(and therefore +1 round to cast, since casting time = DV rounds)}, \textbf{+1 Obligation} \emph{(in addition to Base)}, and on \emph{Partial/Failure} the Keeper gains \textbf{+1 extra SB}.    \item Symbols can be \textbf{maintained/upgraded} like other Assets. Example upgrades: \emph{Hardened} (ignore the first application of \textsc{Compromised} per session), \emph{Bright} (treat as \emph{visible} for Symbol Display while concealed on your person).
\end{itemize}

\subsection*{Borrowed Grace}
\label{talent:borrowed-grace}
\index{Talents!Invoker}\index{Imbuement!Lesser}

\textbf{Type:} Invoker Talent — \textit{Lesser Imbuement}

\subsubsection*{Use}
\begin{itemize}
  \item \textbf{Cost:} \textbf{1 Boon}, \textbf{1 action}.
  \item \textbf{Effect (pick one on use):} \textbf{+1 Melee} \emph{or} \textbf{+1 Thematic} (your table's signature/thematic Skill).
  \item \textbf{Duration:} \textit{Single action/attack} (instantaneous boost only).
  \item \textbf{Requirement:} Wield/display the relevant Patron's \textbf{Symbol}.
  \item \textbf{Obligation:} Immediately mark \textbf{+1 Obligation} to that Patron (see \S\ref{sec:obligation}).
  \item \textbf{Limits:} Cannot be extended, stacked, or \emph{Pushed} for duration. Using \emph{Borrowed Grace} while the Symbol is \textsc{Compromised} lowers your \textbf{Position} by one step \emph{(or imposes \(-1\) die if already \textbf{Desperate})}.)
\end{itemize}

\section{Obligation Capacity}

A character’s \textbf{Obligation Capacity} equals Spirit + Presence.  
Track total Obligation segments across all Patrons (or Symbols, for Invokers).

\begin{itemize}
  \item \textbf{Exceeding Capacity:} For each segment above Capacity, mark 1 Fatigue. The character cannot Invoke Rites or perform rituals until Obligation is reduced below Capacity.
  \item \textbf{Resolution:} Reduce Obligation through Downtime service, Patron tasks, ritual cleansing, or story resolution.
\end{itemize}

\textbf{Example:} Spirit~2 + Presence~3 = Capacity 5.  
6 segments → Fatigue~1.  
7 segments → Fatigue~2.  
10 segments → Harm~1.  
11 segments → Harm~2.

\subsection*{Obligation Management}
Your debt to Patrons must be managed:
\begin{itemize}
\item \textbf{Service}\index{Obligation!service}: Perform tasks fitting your Patron's nature
\item \textbf{Offerings}\index{Obligation!offerings}: Provide sacrifices or tributes
\item \textbf{Propagation}\index{Obligation!propagation}: Spread your Patron's influence or beliefs
\item \textbf{Downtime}\index{Obligation!downtime}: Clear through fitting service during downtime
\end{itemize}

\subsection*{Obligation Levels}
\begin{center}
\small
\begin{tabular}{ll}
\toprule
\textbf{Segments} & \textbf{Consequences} \\
\midrule
1--2 & Minor attention, subtle signs \\
3--5 & Noticeable influence, regular demands \\
6--8 & Significant control, major tasks required \\
9+   & Dominant influence, potentially dangerous \\
\bottomrule
\end{tabular}
\end{center}

\paragraph{Vignette:}
\emph{At the crossroads, Ash lays iron nails and salt. The wind shifts. Somewhere, something smiles.}

\section{Special Magical Abilities} \index{magic!special abilities}

Some characters develop unique magical capabilities through experience or heritage.

\subsection*{Cultural Magical Traditions}
\begin{itemize}
\item \textbf{Dwarven Stone-Sense}\index{Stone-Sense}: Intuitive understanding of earth and stone
\item \textbf{Elven Memory-Weaving}\index{Memory-Weaving}: Accessing and manipulating ancestral knowledge
\item \textbf{Human Versatility}\index{Versatility (human)}: Adaptable magical approaches from various traditions
\item \textbf{Nomadic Spirit-Walking}\index{Spirit-Walking}: Journeying between physical and spiritual realms
\end{itemize}

\subsection*{Advanced Magical Techniques}
\begin{itemize}
\item \textbf{Spell Shaping}\index{Spell Shaping}: Modifying non-ritual spell factors (range/scale/targeting)
\item \textbf{Ritual Mastery}\index{Ritual Mastery}: Perform powerful rituals with reduced risk
\item \textbf{Arcane Dominance}\index{Arcane Dominance}: Overpower weaker magical effects automatically
\end{itemize}

\section{Magical Backlash Examples} \index{magic!backlash examples}

\subsection*{Elemental Backlash}
\begin{itemize}
\item \textbf{Fire}\index{backlash!Fire}: Burns, flares; vs. Water: slick, sputter, dim
\item \textbf{Water}\index{backlash!Water}: Slippery tide, slow gear; vs. Fire: smoke, shorted gear
\item \textbf{Earth}\index{backlash!Earth}: Slips, binds, encumbrance; vs. Air: sound carries, exposure
\item \textbf{Air}\index{backlash!Air}: Scatter, misheard words; vs. Earth: stuck, dust choke
\end{itemize}

\subsection*{Conceptual Backlash}
\begin{itemize}
\item \textbf{Fate}\index{backlash!Fate}: Options close, only-one-way; vs. Luck: mischance hits ally
\item \textbf{Life}\index{backlash!Life}: Growth surge, vines tether; vs. Death/Dreams: numbness, sleep-tug
\item \textbf{Luck}\index{backlash!Luck}: Odds flip; vs. Fate: harsher fixed outcome
\item \textbf{Death/Dreams}\index{backlash!Death}: Whispers, chill; vs. Life: pain returns, rot
\end{itemize}

\section{Magical Item Creation} \index{magic!item creation}

Creating permanent magical items is a complex process.

\subsection*{Creation Requirements}
\begin{itemize}
\item \textbf{Knowledge}\index{item creation!knowledge}: Understanding of the desired effect
\item \textbf{Materials}\index{item creation!materials}: Appropriate components with magical properties
\item \textbf{Time}\index{item creation!time}: Significant investment of time and effort
\item \textbf{Skill}\index{item creation!skill}: High level of magical and craft skills
\item \textbf{Facilities}\index{item creation!facilities}: Proper workspace with necessary tools
\end{itemize}

\subsection*{Creation Process}
\begin{enumerate}
\item \textbf{Design}\index{item creation!design}: Plan the item's properties and limitations
\item \textbf{Gathering}\index{item creation!gathering}: Acquire necessary materials and components
\item \textbf{Crafting}\index{item creation!crafting}: Physical creation of the item base
\item \textbf{Enchantment}\index{item creation!enchantment}: Magical infusion of the desired properties
\item \textbf{Finishing}\index{item creation!finishing}: Final adjustments and testing
\end{enumerate}

\subsection*{Item Limitations}
\begin{itemize}
\item \textbf{Charges}\index{items!charges}: Limited uses before needing recharge
\item \textbf{Attunement}\index{items!attunement}: Required bonding with the user
\item \textbf{Maintenance}\index{items!maintenance}: Regular upkeep to preserve functionality
\item \textbf{Drawbacks}\index{items!drawbacks}: Negative side effects or requirements
\end{itemize}

\section{Magic in Social Situations} \index{magic!social use}

Using magic in social contexts has special considerations.

\subsection*{Social Spellcasting}
\begin{itemize}
\item \textbf{Discretion}\index{social magic!discretion}: Avoiding detection while casting
\item \textbf{Consent}\index{ethics!consent}: Ethical considerations of affecting others' minds
\item \textbf{Reactions}\index{social magic!reactions}: How different cultures view magical influence
\item \textbf{Laws}\index{social magic!laws}: Legal restrictions on magical use in society
\end{itemize}

\subsection*{Social Backlash}
Magical social failures can cause:
\begin{itemize}
\item \textbf{Distrust}\index{social backlash!distrust}: People becoming wary of the caster
\item \textbf{Resistance}\index{social backlash!resistance}: Developing immunity or countermeasures
\item \textbf{Reputation}\index{reputation}: Becoming known as a manipulator
\item \textbf{Legal}\index{social backlash!legal}: Facing consequences from authorities
\end{itemize}

\section{Learning and Improving Magic} \index{magic!improvement}

Magical ability grows through study and practice.

\subsection*{Skill Advancement}
\begin{itemize}
\item \textbf{Study}\index{magic!study}: Researching magical theory and techniques
\item \textbf{Practice}\index{magic!practice}: Regular casting to improve control
\item \textbf{Experimentation}\index{magic!experimentation}: Trying new approaches and combinations
\item \textbf{Instruction}\index{magic!instruction}: Learning from more experienced casters
\end{itemize}

\subsection*{Advanced Magical Development}
At higher levels, casters can:
\begin{itemize}
\item \textbf{Specialize}\index{magic!specialize}: Focus on specific magical traditions
\item \textbf{Innovate}\index{magic!innovate}: Create new spells or techniques
\item \textbf{Teach}\index{magic!teach}: Instruct others in magical arts
\item \textbf{Research}\index{magic!research}: Discover lost or forbidden knowledge
\end{itemize}

\section{Magical Safety and Ethics} \index{magic!safety} \index{magic!ethics}

Responsible magical practice involves understanding risks and consequences.

\subsection*{Safety Considerations}
\begin{itemize}
\item \textbf{Containment}\index{safety!containment}: Preventing unintended spread of effects
\item \textbf{Stability}\index{safety!stability}: Ensuring magical effects remain controlled
\item \textbf{Fail-safes}\index{safety!failsafes}: Planning for when magic goes wrong
\item \textbf{Recovery}\index{safety!recovery}: Procedures for dealing with backlash
\end{itemize}

\subsection*{Ethical Guidelines}
\begin{itemize}
\item \textbf{Consent}\index{ethics!consent}: Respecting others' autonomy regarding magic
\item \textbf{Transparency}\index{ethics!transparency}: Being honest about magical capabilities
\item \textbf{Restraint}\index{ethics!restraint}: Using magic judiciously and appropriately
\item \textbf{Responsibility}\index{ethics!responsibility}: Accepting consequences of magical actions
\end{itemize}

\begin{tcolorbox}[colback=purple!5!white,colframe=purple!75!black,title=Magic Quick Reference,fonttitle=\bfseries]
\textbf{Casting (Freeform)}\index{magic!casting}:
\begin{itemize}
\item Requires Talent: \textbf{Caster's Gift} (2 XP)
\item \textbf{Weave \& Cast}: Two action effect using the Eight Elements
\item \textbf{Backlash}\index{magic!backlash}: Any 1 rolled may cause narrative backlash
\end{itemize}

\textbf{Backlash Severity}\index{magic!backlash}:
\begin{itemize}
\item On Partial/Miss: Pick 1-2 consequences flavored by Element
\item Color consequences by Element (fire burns, fate twists, etc.)
\end{itemize}

\textbf{Rites System}\index{Rites}:
\begin{itemize}
\item \textbf{Invoke}: 1 action effect
\item \textbf{Obligation}: Mark segments on clock
\item \textbf{Push It}: +1 Obligation for +1 step effect
\end{itemize}

\begin{table}[h]
\centering
\caption{Universal Push It Costs}
\begin{tabular}{|l|l|}
\hline
\textbf{Cost Component} & \textbf{Effect} \\ 
\hline
+1 SB & Escalate effect immediately \\ 
+1 Fatigue & Immediate physical/mental strain \\ 
+1 Corruption Clock Segment & Long-term Patron influence (unless otherwise specified) \\ 
GM spends 1 SB & Thematic complication (unless otherwise specified) \\ 
\hline
\end{tabular}
\end{table}

Note: Some talents, Rites, or magical paths may specify alternative corruption costs or additional consequences for Push It actions. When explicitly stated, those specific rules override the universal costs.

\paragraph{Clearing Corruption}
Corruption may be reduced through \textit{purging rituals}, such as exorcisms, sacred songs, or rites of contrition. 
These require a test (typically \textbf{Lore + Spirit}) against a DV equal to the character’s current corruption level.  
On success, reduce corruption by 1. On failure, the corruption manifests violently, imposing a temporary Condition or advancing its narrative expression.  

Optional: A \textbf{Story Beat} may also be spent to attempt such a ritual, representing the personal cost of atonement. Patrons may demand specific acts of service, sacrifice, or obligation as part of the purging process.

\textbf{Invoker Path}\index{Invoker}:
\begin{itemize}
\item \textbf{Symbols} (4 XP each) grant ritual access
\item \textbf{Rituals}: Significant Time, always +1 Obligation
\item \textbf{Crack the Seal}: Instant cast (+2/+3 Obligation)
\end{itemize}

\textbf{Safety}\index{magic!safety}: Every roll changes the story. Success without risk is rare.
\end{tcolorbox}

\section{Practical Magic Examples} \index{examples}

\subsection*{Fire Cast, Partial}
You Weave flame to blind a squad (DV 3). Partial with two 1s. GM spends SB to Position -1 (flare blinds you too) and colors backlash as singed lashes; patrol is alerted (Exposure).

\subsection*{Runekeeper Push and Debt}
You Invoke Circle of Denial [WARD] and Push It to harden the ring. Mark +1 Obligation for the Rite plus +1 for the push. When a demon tests the ring, use [WARD] vs Cap; on its Hit, add +DV to its Leash.

\subsection*{Crack the Seal Under Fire}
You present Ikasha's Symbol and Crack the Seal to lay an instant shadow lane. Symbol $\rightarrow$ Compromised; mark +2 Obligation. GM immediately spends 1 SB to dim all lights (panic), then the lane forms. During downtime, you restore the Symbol (Arcana DV 3): a shaky hit leaves it Neglected until you perform the full rite of cleaning.


\section{Talent: Cantor's Path --- ``Songs of the Low Rites''}
\label{talent:cantors-path}

\begin{tcolorbox}[colback=black!3,colframe=black!40!white,title={Cantor's Path}]
You echo the liturgies of Patrons through breath and string. Not a sworn celebrant but a perilous mimic, you weave Low Rites into song. It is slower, riskier, and beautiful---but never free.
\end{tcolorbox}

\paragraph*{Type} Major Talent (8 XP) \quad % Reduced from 15
\paragraph*{Prerequisites} \textbf{Lore 1+}, \textbf{Performance 2+}, \textbf{Presence 2+} \quad
\paragraph*{Access} Any character (does not require Thiasos membership).

\subsection*{Effect}
You may learn and perform \textbf{Low Rites as Songs}. Each Song counts as knowing the associated Low Rite for performance purposes only.

\begin{itemize}
  \item \textbf{Casting Test:} \emph{Lore + Performance vs.\ DV} (default DV = 2--3).
  \item \textbf{Action Economy:} \emph{1 action to begin;} Song \emph{resolves at the start of your next turn} unless accelerated.
  \item \textbf{Scope:} \emph{Low Rites only.} Standard/High Rites remain exclusive to Patrons and Thiasos initiates.
  \item \textbf{Costs:} Pay any \emph{materials} listed. On success you do \emph{not} mark Obligation.
\end{itemize}

\subsection*{Corruption Clock}
\begin{itemize}
  \item You gain a personal \textbf{Corruption Clock} equal in segments to your \textbf{Body} rating.
  \item \textbf{Mark Corruption when:}
    \begin{itemize}
        \item You \textbf{Push It} (Song resolves immediately). % Moved from Push It section
        \item You perform a \textbf{Resonant Rite} (see below).
        \item The Keeper spends a Story Beat involving your psionic/occult activities. % Clarified
        % Removed: "Each time you cast a Song" - This was the main source of harshness
    \end{itemize}
  \item When the Clock fills:
    \begin{itemize}
      \item You immediately gain a trait of corruption from the \textbf{last Patron} whose Rite you performed.
      \item All of your followers, retainers, or familiars also gain a trait of the same corruption (NPCs manifest visibly unsettling traits).
      \item Reset the Clock to empty.
    \end{itemize}
  \item Corruption traits gained in this way fade at the next \indexterm{Downtime}, unless reinforced by further Patron influence or \textbf{Embraced}.
\end{itemize}

% NEW SUBSECTION
\subsection*{Resonant Rites}
Some powerful or thematically significant Low Rites carry the weight of the Patron's direct influence. Performing these Rites is a conscious act of drawing deep power.

\begin{itemize}
    \item When learning a Song that mimics such a Rite, the GM or the rules text will designate it as \textbf{Resonant}.
    \item Performing a \textbf{Resonant Rite Song} successfully allows you to mark +1 segment on your Corruption Clock. This represents the lingering echo of power.
    \item \textbf{Choosing to Resonate} is optional. You can perform the Rite normally without marking Corruption.
    \item This choice adds a layer of strategy: is the Rite's power worth the potential long-term cost?
\end{itemize}

\subsection*{Outcomes}
\begin{description}
\item[Success:] The Low Rite takes effect as written.
\item[Partial:] The Rite manifests with reduced effect (–1 step) or shortened duration. Mark \textbf{Fatigue 1}.
\item[Failure:] No effect; mark \textbf{Fatigue 1} and the Keeper gains \textbf{+1 SB (Hearts)}. % Kept SB generation for failure
\item[Interrupted:] Harm, Silence, or disruption before resolution = treat as Failure.
\end{description}

\subsection*{Push It}
When you Push:
\begin{itemize}
  \item Song resolves immediately instead of next round.
  \item Mark \textbf{Fatigue 1}.
  \item \textbf{Mark +1 segment on your Corruption Clock.} % Moved here from Corruption Clock section
  \item Keeper immediately triggers a \textbf{Story Beat}, representing fallout from a Patron, the Road, or social attention.
\end{itemize}

\subsection*{Limits \& Interactions}
\begin{itemize}
  \item \textbf{Stacking:} Cannot benefit from the same Rite twice.
  \item \textbf{Visibility:} Songs are inherently noticeable. On Failure or Push, assume observers take note.
  \item \textbf{Silence/Disruption:} Impose –1 to –3 dice at Keeper’s discretion.
  \item \textbf{Obligation Transference:} % Kept, but made more specific
  Whenever a Rite would normally increase Obligation (e.g., a Rite that explicitly says "Mark +1 Obligation"), it instead increases Corruption by that amount.
\end{itemize}

% Simplified Downtime Transition - Made it a choice with a cost/benefit
\subsection*{Downtime Transition}
A character with \textbf{Cantor’s Path} may, during a significant Downtime, choose to seek formal recognition from a \indexterm{Thiasos} whose Patron's rites they have sung.
\begin{itemize}
    \item This requires a scene of roleplay or a test (e.g., Wits + Lore vs. DV 4) to represent the audition/negotiation.
    \item \textbf{Success:} You gain a \textbf{Familiar} and a \textbf{Codex} for that Patron. You lose the Cantor's Path talent and your Corruption Clock. You now use the standard \textbf{Obligation} mechanics for that Patron. Any existing Corruption traits may persist as initial quirks or marks of your unique path to the Thiasos.
    \item \textbf{Failure:} You remain a Cantor, but the attempt may have drawn attention (GM Discretion: +1 SB related to Patron notice, or a new Complication).
\end{itemize}

% Simplified Corruption Fading section
\subsection*{Corruption Fading}
\label{subsec:corruption-fading}
\index{Corruption!Fading}

Corruption does not fade easily. It requires deliberate action and often, a price.

\begin{description}
  \item[\indexterm{Natural Fading}]  
  At the beginning of each Downtime, reduce a character's current \textbf{Corruption} by~1~segment, \emph{if no new segments were added during the last session/arc}. Lingering effects persist until actively addressed.

  \medskip
  \item[\indexterm{Act of Contrition}]  
  Perform a genuine act that contradicts the Patron’s influence or repairs its harm (GM/Player agreement on suitability). \textbf{Effect:} Remove~1~Corruption segment and clear one persistent effect. Costs the character something significant.

  \medskip
  \item[\indexterm{Ritual Purification}]  
  Undertake a significant act of cleansing (pilgrimage, service, seeking rival absolution). \textbf{Effect:} Remove~2~Corruption segments and clear all persistent effects. Likely requires marking Fatigue or temporary Obligation.

  \medskip
  \item[\indexterm{Embrace Corruption}] \label{talent:embrace-corruption} % Make this a separate, clear talent
  \textbf{Type:} Major Talent (6 XP) \quad
  \textbf{Prerequisite:} 2+ levels of Corruption. \\
  You accept the creeping decay, transforming it into a permanent Talent. \textbf{Embracing never reduces Corruption — it reshapes it.} The deeper the corruption, the greater the power and the cost.
    \begin{itemize}
        \item Gain a \textbf{Minor} permanent thematic boon/condition related to the Patron (e.g., +1 die to Stealth in shadows for Ikasha, but -1 die in bright light).
        \item Your Corruption cannot naturally fade below the level at which you Embraced it.
        \item The Keeper gains +1 SB to spend against you related to that Patron's themes.
    \end{itemize}
    \textbf{Narrative Integration:} This Talent represents the Faustian bargain. Players gain agency over their corruption, ensuring that it always carries meaningful consequences.

  \medskip
  \item[\indexterm{Patron Bargain}]  
  Negotiate directly with the Patron. \textbf{Effect:} Remove~1--3~Corruption segments based on the exchange's gravity. Always comes with a narrative cost or condition set by the Keeper.

  \medskip
  \item[\indexterm{Persistence}]  
  Corruption effects do not clear through rest. They require deliberate narrative resolution or specific actions listed above. Every method is an opportunity for character development.
\end{description}

\paragraph*{High Cantor (18 XP Prestige Talent)} % Reduced from 24
\textit{Prerequisite: Tier II+, Cantor's Path, Performance 3+}\\[3pt]
You have learned to weave the sacred tongue through breath and pulse rather than word or gesture. You may now learn and cast \textbf{Standard Rites}, as a \textbf{High Cant}.  

\begin{itemize}
  \item The Rite resolves instantly.
  \item Gain +1 die to its primary effect.  
  \item \textbf{Mark +1 segment on your Corruption Clock.} % Simplified cost
\end{itemize}

\noindent
\textbf{Special:}  
Each Patron’s resonance colors the manifestation differently—flame halos for the Oath, rippling silence for the Choir, tolling harmonics for the Confessor. High Canting is recognizable to other adepts; it draws attention. Repeated use within a single scene risks moral fatigue: add +1 DV to all subsequent \textit{Resolve} rolls against fear, charm, or social pressure in that scene.

\begin{quote}
“The louder the hymn, the nearer the flame.”  
\end{quote}


\noindent Embraced Talents always reflect the Patron’s themes (e.g., shadow, indulgence, empathy, paranoia). They grant great power, but mark the character irrevocably as claimed.


%----------------------------------------
%----------------------------------------
\subsection{Summoning (Pact-Whisperer)}
\label{subsec:summoning}

Summoning is the disciplined art of calling and binding Outsiders for temporary aid.  
This path requires the \textbf{Pact-Whisperer} Talent (2 XP).  
Each summoned being is restrained by a metaphysical tether called a \textit{Leash}, representing the summoner’s control and the strain of sustaining the bond.

\paragraph{Talents \& Access.}
\begin{itemize}
  \item \textbf{Lesser Pactwright:} You may \emph{Call} spirits of \textbf{Cap~1}.
  \item \textbf{Greater Pactwright:} You may also \emph{Call} spirits of \textbf{Cap~3}.
  \item \textbf{Dual Pactwright:} With both Lesser and Greater Pactwright, you may maintain one spirit of each Cap simultaneously.
\end{itemize}

\begin{fatebox}[Summoning Core Mechanics]
\begin{tabularx}{\textwidth}{lX}
\toprule
\textbf{Mechanic} & \textbf{Description and Requirements} \\
\midrule
\textbf{Call} & 1 Action to manifest the spirit at \textit{Near} range; choose a Spirit Template aligned to fiction or Patron domain. \\
\textbf{Bind} & Spend 1 Boon \emph{or} mark 1 Fatigue to establish initial control. \\
\textbf{Leash} & Set Leash = \textbf{Cap + Command} segments.  
(\textit{Cap} is the Outsider’s tier: Cap~1 for Lesser, Cap~3 for Greater.) \\
\textbf{Tick Leash} & Whenever the spirit takes Harm, you command it against its nature, you split focus, a rival contests it, it moves \textit{Close → Far} rapidly, or crosses a \texttt{[WARD]} (\textit{DV = Cap}). \\
\textbf{Departure} & When the Leash fills, the spirit acts to its nature once, then departs (or turns hostile at GM discretion). \\
\bottomrule
\end{tabularx}
\end{fatebox}

\paragraph{Procedure.}
\begin{enumerate}
  \item \textbf{Call (1 Action):} A spirit manifests at \textit{Near}. Choose a Spirit Template appropriate to the scene or Patron.
  \item \textbf{Bind:} Spend 1 Boon \emph{or} mark 1 Fatigue to anchor the connection.
  \item \textbf{Leash:} Record Leash = \textbf{Cap + Command} segments. Draw a clock to track strain.
  \item \textbf{Command:} Each round, issuing a meaningful order uses your Action. Commands contrary to the spirit’s nature tick the Leash.
  \item \textbf{Maintain:} If you split focus or perform other significant actions while it acts on your order, tick the Leash.
  \item \textbf{Departure:} When the Leash fills, the spirit acts to its nature once, then departs. Use this to escalate or reveal consequences.
\end{enumerate}

\paragraph{Economy \& Limits.}
\begin{itemize}
  \item \textbf{Boon Finesse:} Once per round, spend 1 Boon to clear 1 Leash tick (before it fills). Represents appeasement or renewed focus.
  \item \textbf{Action Economy:} Issuing commands uses your Action; most spirits act immediately after their summoner.
  \item \textbf{Concurrency:} Only one active summoned spirit at a time unless a Talent states otherwise. Exceeding this limit inflicts 1 Fatigue per extra Cap point.
  \item \textbf{Downtime:} All summons end at Downtime unless explicitly sustained by a Rite or Asset.
\end{itemize}

\paragraph{Example.}
\textit{Kestra calls a Cap~3 fire elemental to aid in battle. She spends 1 Boon to Bind it.  
The elemental’s Leash is 7 segments (3~+~Command~4). When it takes Harm, the GM ticks the Leash. Later, Kestra splits focus to issue orders while attacking, ticking again.  
Careful management and Boon Finesse keep the bond stable—until the elemental’s fury tests her will.}

\subsection{Paths of Magic: Complete Comparison}
\label{subsec:magic-comparison}
\index{Magic!Comparison}

Five distinct paths define supernatural power in \textsc{Fate's Edge}. Each carries a unique risk, cadence, and narrative flavor. These paths are intentionally \textit{asymmetric}—balanced through story consequences and tactical tradeoffs, not identical mechanics.

\begin{center}
\renewcommand{\arraystretch}{1.15}
\begin{tabularx}{\textwidth}{@{}l Y Y Y Y Y@{}}
\toprule
\textbf{Feature} &
\textbf{Summoner (Pact-Whisperer)} &
\textbf{Cantor's Path} &
\textbf{Caster (Freeform)} &
\textbf{Runekeeper (Rites)} &
\textbf{Invoker (Symbols)} \\
\midrule

\textbf{Core Identity} &
The \textit{Conjurer}: calls and commands spirits as allies &
The \textit{Bootlegger}: steals magic through song &
The \textit{Artist}: improvises magic via elemental will &
The \textit{Devotee}: channels a Patron's power &
The \textit{Ritualist}: works slow, precise magic via Symbols \\

\textbf{Access} &
\textit{Pact-Whisperer (2 XP)}, then Pactwright Talents &
\textit{Cantor's Path (15 XP)} &
\textit{Caster's Gift (2 XP)} &
\textit{Codex (4 XP)} + Familiar (2 XP) &
\textit{Patron's Symbol (4 XP each)} \\

\textbf{How It Works} &
Call (1 action) $\rightarrow$ Bind (Boon/Fatigue) $\rightarrow$ Command.  
Spirit acts each round, tied to a \textbf{Leash} clock &
Perform Song (1 action) $\rightarrow$ effect next beat.  
Mimics Low Rites &
Weave + Cast (2 actions). Highly flexible element magic &
Invoke Rite (1 action). Immediate supernatural effect &
Ritual Invocation (multiple rounds).
\textbf{Crack the Seal} for instant power \\

\textbf{Primary Risk} &
\textbf{Loss of Control}: fill the Leash, spirit acts independently &
\textbf{Corruption}: personal decay and aura effects &
\textbf{Backlash}: volatile elemental consequences &
\textbf{Obligation}: narrative debt owed to Patron &
\textbf{Ritual Cost}: Symbol damage or Obligation \\

\textbf{Power Source} &
Bound spirits and Outsiders &
Stolen resonance, no pact &
Personal discipline + elements &
Formal pact with a Patron &
Consecrated Symbol + precise lore \\

\textbf{Flexibility} &
\textbf{Extreme (via proxy)}: flight, phasing, stealth, combat, etc. &
Structured: mimic known Low Rites &
\textbf{Very high}: any describable effect &
Moderate: Patron Rite list &
Moderate: Symbols owned \\

\textbf{Speed} &
Fast: Spirit acts each round, but commands cost actions &
Moderate: 1 action to begin, effect next beat &
Moderate: 2 actions per spell &
\textbf{Very fast}: 1 action &
\textbf{Very slow}: multi-round rituals \\

\textbf{Key Mechanic} &
\textbf{The Leash} + Boon Finesse (clear ticks with Boons) &
Corruption Clock \& Push It &
GM-set DV \& Element choice &
Push It (gain Obligation) &
Crack the Seal (instant cast at high cost) \\

\textbf{Player Fantasy} &
\textit{The Tactician}: minion control, economy, versatility &
\textit{The Gambler}: risk-for-power, stolen magic &
\textit{The Improviser}: creative problem-solving &
\textit{The Dramatist}: pact, faith, narrative consequences &
\textit{The Planner}: preparation and precision \\
\bottomrule
\end{tabularx}
\end{center}

\paragraph{Balance by Asymmetry.}
\index{Balance}
These paths do not share identical mechanics. They are balanced narratively:
\begin{itemize}
  \item \textbf{Summoners} gain sustained power and versatility, but risk catastrophic loss of control.
  \item \textbf{Cantors} enjoy quick access to magic without a Patron, but corruption erodes them over time.
  \item \textbf{Casters} can attempt nearly anything, but risk explosive elemental backlash.
  \item \textbf{Runekeepers} unleash powerful effects instantly, but every use deepens Patron obligations.
  \item \textbf{Invokers} can safely reshape the world through ritual, but rarely in the heat of battle.
\end{itemize}

Collectively, they form a complete \textbf{pentarchy of power}—distinct, dramatic, and tactically meaningful. No path is universally superior; each shines in different challenges and story arcs.

\section*{Free Casting (TAGS System)}
Some casters do not prepare rote rites. They shape raw forces through shared arcane grammar known as \textbf{TAGS}. A spell is constructed at the table using a short phrase of TAGS. You only need the fiction, the TAG selection, and a casting roll.

\subsection*{Spell Structure}
\textbf{Intent} + \textbf{Target} + \textbf{Tags} = effect.

Example formula:
\begin{quote}
``I unleash Burning • Area • Force against the marauders.''
\end{quote}

The GM sets a Difficulty Value (DV) based on TAG complexity and danger.

\subsection*{Base Difficulty Value (DV)}
Start at DV 1 and add +1 for each TAG used.

\begin{center}
\textbf{DV = 1 + number of TAGS}
\end{center}

Adding powerful or perilous TAGS (Teleportation, Transformation, Dominate) adds +2 instead.

Mastery, focus, or appropriate tools may lower DV by 1.

\subsection*{Casting Roll}
Roll \textbf{Wits + Arcana} (or Ritual, Channeling, etc.).  
Success = spell goes off.  
Failure or 1 = Backlash (see below).

\subsection*{Backlash}
Whenever a Free Caster fails—or pushes power beyond safety—the magic pushes back. Choose one:
\begin{itemize}
\item Harm 2 (Arcane)
\item +2 Fatigue
\item Corruption +1
\item Catastrophic side effect (GM describes)
\end{itemize}

If the spell included a ``Dangerous'' TAG, Backlash triggers on \emph{mixed} results as well.

\newpage

\section*{TAG Library}
Pick 1–3 for minor spells.  
Pick 4–6 for heavy magic (very dangerous).  
More than 6 is suicidal.

\subsection*{Elemental TAGS}
\begin{itemize}[leftmargin=*]
\item \textbf{Burning}: flame, heat, combustion.
\item \textbf{Freezing}: ice, slowing, brittle shatter.
\item \textbf{Storm}: lightning, crackling arcs, thunder shock.
\item \textbf{Stone}: walls, spikes, tremors, armor.
\item \textbf{Wave}: crushing water, currents, pressure.
\item \textbf{Wind}: levitate, gusts, deflection.
\end{itemize}

\subsection*{Force TAGS}
\begin{itemize}[leftmargin=*]
\item \textbf{Force}: pure kinetic power, shields, blasts.
\item \textbf{Area}: cone, circle, corridor, zone.
\item \textbf{Strike}: single target precision.
\item \textbf{Wall}: barrier or blockade.
\item \textbf{Bind}: restrain, hold, suspend.
\item \textbf{Dispel}: suppress magic, unravel effects.
\end{itemize}

\subsection*{Mind \& Veil TAGS}
\begin{itemize}[leftmargin=*]
\item \textbf{Veil}: conceal, blur, illusion, silence.
\item \textbf{Scry}: reveal hidden, see distance, read traces.
\item \textbf{Memory}: erase, alter, restore.
\item \textbf{Command}: compel short action.
\item \textbf{Fear}: panic, flee, break morale.
\end{itemize}

\subsection*{Life \& Body TAGS}
\begin{itemize}[leftmargin=*]
\item \textbf{Mend}: close wounds, restore flesh, reduce Harm 1.
\item \textbf{Purify}: remove poison, corruption, disease.
\item \textbf{Strengthen}: enhance body, armor, senses.
\item \textbf{Waken}: counter sleep, paralysis, stun.
\item \textbf{Beast}: speak with or influence animals.
\end{itemize}

\subsection*{Space \& Motion TAGS (Always +2 DV Each)}
\begin{itemize}[leftmargin=*]
\item \textbf{Leap}: jump far, blink across short space.
\item \textbf{Fold}: short-range teleport, vanish–reappear.
\item \textbf{Gate}: long distance passage, open/close path.
\item \textbf{Gravity}: crush, lift, suspend, walk skyward.
\end{itemize}

\subsection*{Creation \& Transformation TAGS (Always +2 DV Each)}
\begin{itemize}[leftmargin=*]
\item \textbf{Create}: manifest matter briefly.
\item \textbf{Summon}: call a being or construct.
\item \textbf{Transmute}: turn one thing into another.
\item \textbf{Animate}: make objects act with intent.
\end{itemize}