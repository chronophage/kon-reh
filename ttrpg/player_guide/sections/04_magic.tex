\chapter{Magic and Special Abilities} \label{ch:magic}

Magic in this game is powerful but dangerous—a negotiation with reality itself that always carries risks. This chapter covers the core magical systems: standard \textbf{spellcasting}\index{magic!spellcasting}, \textbf{ritual magic}\index{magic!rituals}, and special \textbf{pact-based abilities}\index{magic!pacts}. Throughout, look for examples and player-facing tips to keep the fiction front and center.

\section{The Nature of Magic} \index{magic!nature}

Magic is not a safe tool but a dangerous force:
\begin{itemize}
\item \textbf{Powerful}\index{magic!power}: Can reshape battles, stories, or even the world
\item \textbf{Risky}\index{magic!risk}: Every use generates \textbf{Story Beats (SB)}\index{Story Beats} that manifest as backlash
\item \textbf{Thematic}\index{magic!themes}: Effects and consequences align with the type of magic used
\item \textbf{Volatile}\index{magic!volatility}: Never fully predictable or controllable
\item \textbf{Narrative}\index{magic!narrative}: Casting is always a significant story moment
\end{itemize}

\paragraph{Table Vignette:}
\emph{``I can hold the avalanche,'' says Mira, fingers trembling. ``But something will answer.''} The party nods—risk accepted, stakes clear.

\section{Basic Spellcasting} \index{magic!spellcasting}

All spellcasting follows the standard action resolution system but with additional considerations for magical effects.

\subsection*{The Casting Process}
\begin{enumerate}
\item \textbf{Declare Intent}\index{magic!intent}: What you want the magic to achieve
\item \textbf{Choose Approach}\index{magic!approach}: Which magical skill and method you'll use
\item \textbf{Set Position}\index{position}: \textbf{Controlled}\index{position!Controlled}, \textbf{Risky}\index{position!Risky}, or \textbf{Desperate}\index{position!Desperate} based on circumstances
\item \textbf{Roll}\index{dice pools}: Attribute + Magical Skill
\item \textbf{Resolve}\index{outcomes}: Apply outcomes with magical consequences
\end{enumerate}

\subsection*{Magical Skills}
Common magical skills include:
\begin{itemize}
\item \textbf{Arcana}\index{Arcana}: General magical knowledge and theory
\item \textbf{Elemental Magic}\index{Elemental Magic}: Fire, water, earth, air manipulation
\item \textbf{Spiritual Magic}\index{Spiritual Magic}: Communing with spirits, divine magic
\item \textbf{Mental Magic}\index{Mental Magic}: Telepathy, illusion, mind affecting
\item \textbf{Healing Magic}\index{Healing Magic}: Restoration, purification, life magic
\end{itemize}

\paragraph{Player Tip:}
State a clear \textbf{intent}\index{magic!intent} and a vivid \textbf{method}\index{magic!approach}. The more concrete the fiction, the easier it is to set fair \textbf{DV}\index{Difficulty Value (DV)} and meaningful consequences.

\section{The Casting Loop} \index{magic!casting loop}

For more significant magical effects, use the structured Casting Loop requiring two actions.

\subsection*{Phase 1: Weave} \index{magic!weave}
Shape the magical effect:
\begin{itemize}
\item Player builds dice pool and rolls
\item On success, they stabilize the spell's form
\item Any 1 rolled may cause narrative backlash related to the Element
\end{itemize}

\subsection*{Phase 2: Cast} \index{magic!cast}
Channel the effect into the world:
\begin{itemize}
\item A second roll channels the effect
\item Backlash: Any 1 rolled may cause narrative backlash related to the Element
\end{itemize}

\paragraph{Designer Note:}
The \textbf{Casting Loop}\index{magic!casting loop} requires the \textbf{Caster's Gift}\index{Caster's Gift} talent (2 XP) and creates spotlight tension: describe effect now, risk \textbf{Backlash} on each roll.

\section{Backlash Severity} \index{magic!backlash}

\begin{center}
\small
\begin{tabular}{ll}
\toprule
\textbf{Roll Result} & \textbf{Backlash Trigger} \\
\midrule
Partial/Miss & Minor backlash (choose one) \\
Miss & Major backlash (choose two) \\
Hit with two or more 1s & Minor backlash alongside success \\
\bottomrule
\end{tabular}
\end{center}

\section{Magical Arts and Traditions} \index{magic!arts}

Different cultures and traditions approach magic differently.

\subsection*{Elemental Magic} \index{Elemental Magic}
Manipulation of natural forces:
\begin{itemize}
\item \textbf{Fire Magic}\index{Elemental Magic!Fire}: Heat, light, transformation, destruction
\item \textbf{Water Magic}\index{Elemental Magic!Water}: Flow, healing, divination, adaptation
\item \textbf{Earth Magic}\index{Elemental Magic!Earth}: Stability, protection, growth, strength
\item \textbf{Air Magic}\index{Elemental Magic!Air}: Movement, communication, freedom, change
\end{itemize}

\subsection*{Spiritual Magic} \index{Spiritual Magic}
Interaction with intangible forces:
\begin{itemize}
\item \textbf{Divine Magic}\index{Spiritual Magic!Divine}: Power from gods or higher powers
\item \textbf{Spirit Magic}\index{Spiritual Magic!Spirit}: Communing with nature spirits or ancestors
\item \textbf{Necromancy}\index{Spiritual Magic!Necromancy}: Interaction with death and the departed
\item \textbf{Protection Magic}\index{Spiritual Magic!Protection}: Wards, blessings, purification
\end{itemize}

\subsection*{Mental Magic} \index{Mental Magic}
Affecting minds and perceptions:
\begin{itemize}
\item \textbf{Illusion}\index{Mental Magic!Illusion}: Creating false perceptions and images
\item \textbf{Telepathy}\index{Mental Magic!Telepathy}: Mind reading and communication
\item \textbf{Enchantment}\index{Mental Magic!Enchantment}: Influencing thoughts and emotions
\item \textbf{Divination}\index{Mental Magic!Divination}: Gaining knowledge through supernatural means
\end{itemize}

\paragraph{Vignette:}
\emph{The candles lean toward the oracle's breath. ``Ask,'' she whispers, ``but truth is sharp.''}

\section{Ritual Magic} \index{magic!rituals}

Rituals take Significant Time (typically 10-30 minutes) for powerful effects.

\subsection*{Ritual Requirements}
\begin{itemize}
\item \textbf{Time}\index{rituals!time}: Significant Time (typically 10-30 minutes)
\item \textbf{Preparation}\index{rituals!preparation}: Specific materials, locations, or conditions
\item \textbf{Focus}\index{rituals!focus}: Undisturbed concentration and coordination
\end{itemize}

\subsection*{Ritual Procedure}
\begin{enumerate}
\item \textbf{Preparation}\index{rituals!preparation}: Gather components, prepare space, focus intent
\item \textbf{Invocation}\index{rituals!invocation}: Perform the Rite as a ritual
\item \textbf{Completion}\index{rituals!completion}: Effect manifests, always marks +1 Obligation
\end{enumerate}

\subsection*{Ritual Benefits and Risks}
\begin{itemize}
\item \textbf{Benefits}\index{rituals!benefits}: Safe casting, no Push It option
\item \textbf{Risks}\index{rituals!risks}: Time investment, Obligation cost, environmental requirements
\end{itemize}

\section{Rites and Pact Magic} \index{magic!rites} \index{magic!pacts}

Rites are precise magical effects gained through \textbf{pacts}\index{pacts} with powerful entities. There are two main paths to accessing Rites:

\subsection*{The Runekeeper (Rites Path)}
\begin{itemize}
\item Requires Patron + Thiasos (Familiar) + Codex (4 XP)
\item Accesses that Patron's full Rite list
\item Structured, powerful, but accrues \textbf{Obligation}
\item Can Push Rites once per scene for +1 Obligation
\end{itemize}

\subsection*{The Invoker (Symbol Path)}
\begin{itemize}
\item Requires one or more \textbf{Patron's Symbols} (4 XP each)
\item Accesses ritual invocation of Patron's Rites
\item Safe but slow—requires Significant Time
\item Can Crack the Seal for instant cast at steep Obligation cost (+2/+3)
\end{itemize}

\subsection*{Using Rites}
\begin{enumerate}
\item \textbf{Invocation}\index{Rites!invocation}: Invoke a Rite requires 1 Action
\item \textbf{Obligation}\index{Obligation}: Each Rite used marks Obligation on its clock
\item \textbf{Effect}\index{Rites!effect}: The Rite's specific effect manifests
\end{enumerate}

\paragraph{Rite Invocation via Symbol}
\begin{itemize}
  \item \textbf{Time.} Invoking a Rite via Symbol takes \(\text{DV} + 1\) rounds.
  \item \textbf{Obligation.} On completion, mark +1 Obligation (in addition to any listed Rite costs, if applicable).
  \item \textbf{No Push.} Invoker Rites cannot use \emph{Push It} benefits.
  \item \textbf{Symbol Display.} The Symbol must remain visible throughout the invocation.
  \item \textbf{Materials.} Symbols replace any Thaisos and Codex requirements.
\end{itemize}
\section{Obligation Capacity}

A character’s \textbf{Obligation Capacity} equals Spirit + Presence.  
Track total Obligation segments across all Patrons (or Symbols, for Invokers).

\begin{itemize}
  \item \textbf{Exceeding Capacity:} For each segment above Capacity, mark 1 Fatigue. The character cannot Invoke Rites or perform rituals until Obligation is reduced below Capacity.
  \item \textbf{Resolution:} Reduce Obligation through Downtime service, Patron tasks, ritual cleansing, or story resolution.
\end{itemize}

\textbf{Example:} Spirit~2 + Presence~3 = Capacity 5.  
6 segments → Fatigue~1.  
7 segments → Fatigue~2.  
10 segments → Harm~1.  
11 segments → Harm~2.

\subsection*{Obligation Management}
Your debt to Patrons must be managed:
\begin{itemize}
\item \textbf{Service}\index{Obligation!service}: Perform tasks fitting your Patron's nature
\item \textbf{Offerings}\index{Obligation!offerings}: Provide sacrifices or tributes
\item \textbf{Propagation}\index{Obligation!propagation}: Spread your Patron's influence or beliefs
\item \textbf{Downtime}\index{Obligation!downtime}: Clear through fitting service during downtime
\end{itemize}

\subsection*{Obligation Levels}
\begin{center}
\small
\begin{tabular}{ll}
\toprule
\textbf{Segments} & \textbf{Consequences} \\
\midrule
1--2 & Minor attention, subtle signs \\
3--5 & Noticeable influence, regular demands \\
6--8 & Significant control, major tasks required \\
9+   & Dominant influence, potentially dangerous \\
\bottomrule
\end{tabular}
\end{center}

\paragraph{Vignette:}
\emph{At the crossroads, Ash lays iron nails and salt. The wind shifts. Somewhere, something smiles.}

\section{Special Magical Abilities} \index{magic!special abilities}

Some characters develop unique magical capabilities through experience or heritage.

\subsection*{Cultural Magical Traditions}
\begin{itemize}
\item \textbf{Dwarven Stone-Sense}\index{Stone-Sense}: Intuitive understanding of earth and stone
\item \textbf{Elven Memory-Weaving}\index{Memory-Weaving}: Accessing and manipulating ancestral knowledge
\item \textbf{Human Versatility}\index{Versatility (human)}: Adaptable magical approaches from various traditions
\item \textbf{Nomadic Spirit-Walking}\index{Spirit-Walking}: Journeying between physical and spiritual realms
\end{itemize}

\subsection*{Advanced Magical Techniques}
\begin{itemize}
\item \textbf{Spell Shaping}\index{Spell Shaping}: Modifying non-ritual spell factors (range/scale/targeting)
\item \textbf{Ritual Mastery}\index{Ritual Mastery}: Perform powerful rituals with reduced risk
\item \textbf{Arcane Dominance}\index{Arcane Dominance}: Overpower weaker magical effects automatically
\end{itemize}

\section{Magical Backlash Examples} \index{magic!backlash examples}

\subsection*{Elemental Backlash}
\begin{itemize}
\item \textbf{Fire}\index{backlash!Fire}: Burns, flares; vs. Water: slick, sputter, dim
\item \textbf{Water}\index{backlash!Water}: Slippery tide, slow gear; vs. Fire: smoke, shorted gear
\item \textbf{Earth}\index{backlash!Earth}: Slips, binds, encumbrance; vs. Air: sound carries, exposure
\item \textbf{Air}\index{backlash!Air}: Scatter, misheard words; vs. Earth: stuck, dust choke
\end{itemize}

\subsection*{Conceptual Backlash}
\begin{itemize}
\item \textbf{Fate}\index{backlash!Fate}: Options close, only-one-way; vs. Luck: mischance hits ally
\item \textbf{Life}\index{backlash!Life}: Growth surge, vines tether; vs. Death/Dreams: numbness, sleep-tug
\item \textbf{Luck}\index{backlash!Luck}: Odds flip; vs. Fate: harsher fixed outcome
\item \textbf{Death/Dreams}\index{backlash!Death}: Whispers, chill; vs. Life: pain returns, rot
\end{itemize}

\section{Magical Item Creation} \index{magic!item creation}

Creating permanent magical items is a complex process.

\subsection*{Creation Requirements}
\begin{itemize}
\item \textbf{Knowledge}\index{item creation!knowledge}: Understanding of the desired effect
\item \textbf{Materials}\index{item creation!materials}: Appropriate components with magical properties
\item \textbf{Time}\index{item creation!time}: Significant investment of time and effort
\item \textbf{Skill}\index{item creation!skill}: High level of magical and craft skills
\item \textbf{Facilities}\index{item creation!facilities}: Proper workspace with necessary tools
\end{itemize}

\subsection*{Creation Process}
\begin{enumerate}
\item \textbf{Design}\index{item creation!design}: Plan the item's properties and limitations
\item \textbf{Gathering}\index{item creation!gathering}: Acquire necessary materials and components
\item \textbf{Crafting}\index{item creation!crafting}: Physical creation of the item base
\item \textbf{Enchantment}\index{item creation!enchantment}: Magical infusion of the desired properties
\item \textbf{Finishing}\index{item creation!finishing}: Final adjustments and testing
\end{enumerate}

\subsection*{Item Limitations}
\begin{itemize}
\item \textbf{Charges}\index{items!charges}: Limited uses before needing recharge
\item \textbf{Attunement}\index{items!attunement}: Required bonding with the user
\item \textbf{Maintenance}\index{items!maintenance}: Regular upkeep to preserve functionality
\item \textbf{Drawbacks}\index{items!drawbacks}: Negative side effects or requirements
\end{itemize}

\section{Magic in Social Situations} \index{magic!social use}

Using magic in social contexts has special considerations.

\subsection*{Social Spellcasting}
\begin{itemize}
\item \textbf{Discretion}\index{social magic!discretion}: Avoiding detection while casting
\item \textbf{Consent}\index{ethics!consent}: Ethical considerations of affecting others' minds
\item \textbf{Reactions}\index{social magic!reactions}: How different cultures view magical influence
\item \textbf{Laws}\index{social magic!laws}: Legal restrictions on magical use in society
\end{itemize}

\subsection*{Social Backlash}
Magical social failures can cause:
\begin{itemize}
\item \textbf{Distrust}\index{social backlash!distrust}: People becoming wary of the caster
\item \textbf{Resistance}\index{social backlash!resistance}: Developing immunity or countermeasures
\item \textbf{Reputation}\index{reputation}: Becoming known as a manipulator
\item \textbf{Legal}\index{social backlash!legal}: Facing consequences from authorities
\end{itemize}

\section{Learning and Improving Magic} \index{magic!improvement}

Magical ability grows through study and practice.

\subsection*{Skill Advancement}
\begin{itemize}
\item \textbf{Study}\index{magic!study}: Researching magical theory and techniques
\item \textbf{Practice}\index{magic!practice}: Regular casting to improve control
\item \textbf{Experimentation}\index{magic!experimentation}: Trying new approaches and combinations
\item \textbf{Instruction}\index{magic!instruction}: Learning from more experienced casters
\end{itemize}

\subsection*{Advanced Magical Development}
At higher levels, casters can:
\begin{itemize}
\item \textbf{Specialize}\index{magic!specialize}: Focus on specific magical traditions
\item \textbf{Innovate}\index{magic!innovate}: Create new spells or techniques
\item \textbf{Teach}\index{magic!teach}: Instruct others in magical arts
\item \textbf{Research}\index{magic!research}: Discover lost or forbidden knowledge
\end{itemize}

\section{Magical Safety and Ethics} \index{magic!safety} \index{magic!ethics}

Responsible magical practice involves understanding risks and consequences.

\subsection*{Safety Considerations}
\begin{itemize}
\item \textbf{Containment}\index{safety!containment}: Preventing unintended spread of effects
\item \textbf{Stability}\index{safety!stability}: Ensuring magical effects remain controlled
\item \textbf{Fail-safes}\index{safety!failsafes}: Planning for when magic goes wrong
\item \textbf{Recovery}\index{safety!recovery}: Procedures for dealing with backlash
\end{itemize}

\subsection*{Ethical Guidelines}
\begin{itemize}
\item \textbf{Consent}\index{ethics!consent}: Respecting others' autonomy regarding magic
\item \textbf{Transparency}\index{ethics!transparency}: Being honest about magical capabilities
\item \textbf{Restraint}\index{ethics!restraint}: Using magic judiciously and appropriately
\item \textbf{Responsibility}\index{ethics!responsibility}: Accepting consequences of magical actions
\end{itemize}

\begin{tcolorbox}[colback=purple!5!white,colframe=purple!75!black,title=Magic Quick Reference,fonttitle=\bfseries]
\textbf{Casting (Freeform)}\index{magic!casting}:
\begin{itemize}
\item Requires Talent: \textbf{Caster's Gift} (2 XP)
\item \textbf{Weave \& Cast}: Two action effect using the Eight Elements
\item \textbf{Backlash}\index{magic!backlash}: Any 1 rolled may cause narrative backlash
\end{itemize}

\textbf{Backlash Severity}\index{magic!backlash}:
\begin{itemize}
\item On Partial/Miss: Pick 1-2 consequences flavored by Element
\item Color consequences by Element (fire burns, fate twists, etc.)
\end{itemize}

\textbf{Rites System}\index{Rites}:
\begin{itemize}
\item \textbf{Invoke}: 1 action effect
\item \textbf{Obligation}: Mark segments on clock
\item \textbf{Push It}: +1 Obligation for +1 step effect
\end{itemize}

\textbf{Invoker Path}\index{Invoker}:
\begin{itemize}
\item \textbf{Symbols} (4 XP each) grant ritual access
\item \textbf{Rituals}: Significant Time, always +1 Obligation
\item \textbf{Crack the Seal}: Instant cast (+2/+3 Obligation)
\end{itemize}

\textbf{Safety}\index{magic!safety}: Every roll changes the story. Success without risk is rare.
\end{tcolorbox}

\section{Practical Magic Examples} \index{examples}

\subsection*{Fire Cast, Partial}
You Weave flame to blind a squad (DV 3). Partial with two 1s. GM spends SB to Position -1 (flare blinds you too) and colors backlash as singed lashes; patrol is alerted (Exposure).

\subsection*{Runekeeper Push and Debt}
You Invoke Circle of Denial [WARD] and Push It to harden the ring. Mark +1 Obligation for the Rite plus +1 for the push. When a demon tests the ring, use [WARD] vs Cap; on its Hit, add +DV to its Leash.

\subsection*{Crack the Seal Under Fire}
You present Ikasha's Symbol and Crack the Seal to lay an instant shadow lane. Symbol $\rightarrow$ Compromised; mark +2 Obligation. GM immediately spends 1 SB to dim all lights (panic), then the lane forms. During downtime, you restore the Symbol (Arcana DV 3): a shaky hit leaves it Neglected until you perform the full rite of cleaning.

\section{Talent: Cantor's Path --- ``Songs of the Low Rites''}
\label{talent:cantors-path}

\begin{tcolorbox}[colback=black!3,colframe=black!40!white,title={Cantor's Path}]
You echo the liturgies of Patrons through breath and string. Not a sworn celebrant but a perilous mimic, you weave Low Rites into song. It is slower, riskier, and beautiful---but never free.
\end{tcolorbox}

\paragraph*{Type} Minor Talent (10 XP) \quad
\paragraph*{Prerequisites} \textbf{Lore 1+}, \textbf{Performance 2+}, \textbf{Presence 2+} \quad
\paragraph*{Access} Any character (does not require Thiasos membership).

\subsection*{Effect}
You may learn and perform \textbf{Low Rites as Songs}. Each Song counts as knowing the associated Low Rite for performance purposes only.

\begin{itemize}
\item \textbf{Casting Test:} \emph{Lore + Performance vs.\ DV} (default DV = 2--3; see \S\ref{talent:cantors-path-dv}).
\item \textbf{Action Economy:} \emph{1 action to begin;} Song \emph{resolves at the start of your next turn} unless accelerated.
\item \textbf{Scope:} \emph{Low Rites only.} Standard/High Rites remain exclusive to Patrons and Thiasos initiates.
\item \textbf{Costs:} Pay any \emph{materials} listed. On success you do \emph{not} mark Obligation.
\end{itemize}

\subsection*{Outcomes}
\begin{description}
\item[Success:] The Low Rite takes effect as written.
\item[Partial:] The Rite manifests with \emph{reduced Effect} (–1 step) or \emph{shortened duration}. Mark \textbf{Fatigue 1}.
\item[Failure:] No effect; mark \textbf{Fatigue 1} and the Keeper gains \textbf{+1 SB (Hearts)}. You \emph{do not} mark Obligation.
\item[Interrupted:] Harm, Silence, or disruption before resolution = treat as Failure.
\end{description}

\subsection*{Push It}
You may Push \emph{when you begin casting}:
\begin{itemize}
\item Song resolves immediately instead of next round.
\item Mark \textbf{Fatigue 1}.
\item Keeper immediately triggers a \textbf{Story Beat}, representing fallout from a Patron, the Road, or social attention.
\end{itemize}

\subsection*{Limits \& Interactions}
\begin{itemize}
\item \textbf{Stacking:} Cannot benefit from the same Rite twice.
\item \textbf{Social Visibility:} Songs are inherently noticeable. On Failure or Push, assume observers take note.
\item \textbf{Silence/Disruption:} Impose \emph{–1 to –3 dice} at Keeper’s discretion.
\end{itemize}

\subsection*{DV Guidance}
\label{talent:cantors-path-dv}
\begin{description}
\item[DV 2:] Personal augments, simple glamours.
\item[DV 3:] Zone effects, multi-target edges, time-bending moods.
\item[+1 DV:] Hostile crowds, loud environments, warded spaces.
\end{description}

\subsection*{Examples}
\begin{itemize}
\item \textbf{Perfect Note (Low):} Song resolves next round (DV 3). Push it to resolve immediately, mark +1 Fatigue, and trigger a Story Beat.
\item \textbf{Endless Revel (Low):} Creates revel zone on next turn. Push it to start now with the same costs.
\end{itemize}

\paragraph{Closing Thought:}
\textbf{Magic}\index{magic} is a powerful tool but never a safe one. Every casting carries risks, and great power always demands great responsibility. Make bold choices—then let the consequences write the next chapter.

\section{Narrative-Heavy Magic Options}

For groups that prefer strong narrative focus in magic use, consider these optional approaches:

\textbf{Intent-Driven Magic:} For minor magical effects that don't significantly alter the story, players can simply declare what they want to accomplish and describe how they do it, without rolling dice. The GM determines if the effect is reasonable and what complications might arise.

\textbf{Collaborative Backlash:} Instead of the GM unilaterally determining backlash, players can suggest thematic consequences that fit the fiction, with GM approval. This makes magic feel more collaborative and story-driven.

\textbf{Ritual as Story Beats:} Major magical workings can be treated as scene-defining moments where the group collaboratively describes what happens, with mechanical effects determined by the narrative impact rather than detailed rolls.

\textbf{Patron Relationships:} Focus on the roleplaying aspects of Patron relationships, treating Obligation as a measure of story tension and character development rather than just a mechanical track to be managed.

\textbf{Magic as Character Development:} Use magical experiences as opportunities for character growth and backstory development, allowing players to narrate how their characters learned new abilities through significant story moments.
