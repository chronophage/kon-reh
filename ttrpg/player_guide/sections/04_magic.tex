\chapter{Magic and Special Abilities} \label{ch:magic}

Magic in this game is powerful but dangerous—a negotiation with reality itself that always carries risks. This chapter covers the core magical systems: standard \textbf{spellcasting}\index{magic!spellcasting}, \textbf{ritual magic}\index{magic!rituals}, and special \textbf{pact-based abilities}\index{magic!pacts}. Throughout, look for examples and player-facing tips to keep the fiction front and center.

\begin{tcolorbox}[colback=black!3,colframe=black!40!white,title={Sidebar: \texttt{[TAGS]} \& Casting},enhanced]
  \label{sidebar:tags-casting}
  \textbf{What are \texttt{[TAGS]}?} Effects in \textit{Fate's Edge} are communicated via \texttt{[TAGS]}. Each \texttt{[TAG]} is a discrete effect gated behind a Talent, Rite, spell, or asset—\emph{it cannot be invoked spontaneously} unless a rule grants access.
  
  \medskip
  \textbf{How they’re used.} \texttt{[TAGS]} provide a common language for describing effects, especially when players invent spells via \emph{Free Casting}. Many prewritten spells and abilities also list their \texttt{[TAGS]} for clarity.
  
  \medskip
  \textbf{Cross-reference.} For the canonical glossary and full list of available \texttt{[TAGS]}, see \S\ref{magic:tags}.
  \end{tcolorbox}

\section{The Nature of Magic} \index{magic!nature}

Magic is not a safe tool but a dangerous force:
\begin{itemize}
\item \textbf{Powerful}\index{magic!power}: Can reshape battles, stories, or even the world
\item \textbf{Controlled}\index{magic!risk}: Every use generates \textbf{Story Beats (SB)}\index{Story Beats} that manifest as backlash
\item \textbf{Thematic}\index{magic!themes}: Effects and consequences align with the type of magic used
\item \textbf{Volatile}\index{magic!volatility}: Never fully predictable or controllable
\item \textbf{Narrative}\index{magic!narrative}: Casting is always a significant story moment
\end{itemize}

\paragraph{Table Vignette:}
\emph{``I can hold the avalanche,'' says Mira, fingers trembling. ``But something will answer.''} The party nods—risk accepted, stakes clear.

\section{Basic Spellcasting} \index{magic!spellcasting}

All spellcasting follows the standard action resolution system but with additional considerations for magical effects.

\subsection*{The Casting Process}
\begin{enumerate}
\item \textbf{Declare Intent}\index{magic!intent}: What you want the magic to achieve
\item \textbf{Choose Approach}\index{magic!approach}: Which magical skill and method you'll use
\item \textbf{Set Position}\index{position}: \textbf{Dominant}\index{position!Dominant}, \textbf{Controlled}\index{position!Controlled}, or \textbf{Desperate}\index{position!Desperate} based on circumstances
\item \textbf{Roll}\index{dice pools}: Attribute + Magical Skill
\item \textbf{Resolve}\index{outcomes}: Apply outcomes with magical consequences
\end{enumerate}

\subsection*{Magical Skills}
Common magical skills include:
\begin{itemize}
\item \textbf{Arcana}\index{Arcana}: General magical knowledge and theory
\item \textbf{Elemental Magic}\index{Elemental Magic}: Fire, water, earth, air manipulation
\item \textbf{Spiritual Magic}\index{Spiritual Magic}: Communing with spirits, divine magic
\item \textbf{Mental Magic}\index{Mental Magic}: Telepathy, illusion, mind affecting
\item \textbf{Healing Magic}\index{Healing Magic}: Restoration, purification, life magic
\end{itemize}

\paragraph{Player Tip:}
State a clear \textbf{intent}\index{magic!intent} and a vivid \textbf{method}\index{magic!approach}. The more concrete the fiction, the easier it is to set fair \textbf{DV}\index{Difficulty Value (DV)} and meaningful consequences.

\section{The Casting Loop} \index{magic!casting loop}

For more significant magical effects, use the structured Casting Loop requiring two actions.

\subsection*{Phase 1: Weave} \index{magic!weave}
Shape the magical effect:
\begin{itemize}
\item Player builds dice pool and rolls
\item On success, they stabilize the spell's form
\item Any 1 rolled may cause narrative backlash related to the Element
\end{itemize}

\subsection*{Phase 2: Cast} \index{magic!cast}
Channel the effect into the world:
\begin{itemize}
\item A second roll channels the effect
\item Backlash: Any 1 rolled may cause narrative backlash related to the Element
\end{itemize}

\paragraph{Designer Note:}
The \textbf{Casting Loop}\index{magic!casting loop} requires the \textbf{Spellcraft}\index{Spellcraft} talent (6 XP) and creates spotlight tension: describe effect now, risk \textbf{Backlash} on each roll.

\section{Backlash Severity} \index{magic!backlash}

\begin{center}
\small
\begin{longtable}{ll}
\toprule
\textbf{Roll Result} & \textbf{Backlash Trigger} \\
\midrule
Partial/Miss & Minor backlash (choose one) \\
Miss & Major backlash (choose two) \\
Hit with two or more 1s & Minor backlash alongside success \\
\bottomrule
\end{longtable}
\end{center}

\subsection*{Free Casting Cheat Sheet (Player-Facing)}

\paragraph*{What You Can Do}
\begin{itemize}
  \item You can attempt any effect you can describe using Elements and \texttt{[TAGS]}.
  \item You reshape or redirect what already exists (fire, stone, air, life) rather than creating something from nothing.
  \item You can heal, protect, hinder, and reshape the scene within your Tier.
\end{itemize}

\paragraph*{What You \emph{Cannot} Do}
\begin{itemize}
  \item No true resurrection, instant death, or ``I win'' buttons.
  \item No matter-from-nothing or free creation of life.
  \item No safe, casual teleport chains or permanent world edits via one roll.
\end{itemize}

\paragraph*{Safe Limits}
\begin{itemize}
  \item \textbf{TAGS per casting:} Up to 6 total; more is legendary/suicidal. Suggested limit is to Tier +1
  \item \textbf{Dangerous \texttt{[TAGS]}:} \texttt{[TELEPORT]}, \texttt{[TRANSFORM]}, \texttt{[DOMINATE]} and similar always carry extra risk.
  \item Combining several Dangerous \texttt{[TAGS]} or opposing elements (Fire/Water, Earth/Air) makes Backlash harsher.
\end{itemize}

\paragraph*{Costs You Should Expect}
\begin{itemize}
  \item Each casting gives the GM Story Beats (SB) to spend on twists and backlash.
  \item Big, flashy, or repeated spells can cause Fatigue or worsen your Position.
  \item Ongoing effects usually cost Mental Fatigue each scene to sustain.
\end{itemize}

\paragraph*{How to Keep It Working for You}
\begin{itemize}
  \item \textbf{Be Clear:} Say exactly what you want the spell to do in the fiction.
  \item \textbf{Stay On-Theme:} Use elements and \texttt{[TAGS]} that fit your style/Patron for more reliable results.
  \item \textbf{Practice:} Repeating the same basic trick makes it easier and safer over time (your ``signature spells'').
  \item \textbf{Scale Back:} If the GM says, ``That’s a lot,'' offer a smaller version (less area, fewer targets, shorter time).
\end{itemize}

\medskip
\noindent\textit{Free casting is flexible: be creative, stay in-theme, and expect bigger magic to come with bigger consequences.}

\section{Magical Arts and Traditions} \index{magic!arts}

Different cultures and traditions approach magic differently.

\subsection*{Elemental Magic} \index{Elemental Magic}
Manipulation of natural forces:
\begin{itemize}
\item \textbf{Fire Magic}\index{Elemental Magic!Fire}: Heat, light, transformation, destruction
\item \textbf{Water Magic}\index{Elemental Magic!Water}: Flow, healing, divination, adaptation
\item \textbf{Earth Magic}\index{Elemental Magic!Earth}: Stability, protection, growth, strength
\item \textbf{Air Magic}\index{Elemental Magic!Air}: Movement, communication, freedom, change
\end{itemize}

\subsection*{Spiritual Magic} \index{Spiritual Magic}
Interaction with intangible forces:
\begin{itemize}
\item \textbf{Divine Magic}\index{Spiritual Magic!Divine}: Power from gods or higher powers
\item \textbf{Spirit Magic}\index{Spiritual Magic!Spirit}: Communing with nature spirits or ancestors
\item \textbf{Necromancy}\index{Spiritual Magic!Necromancy}: Interaction with death and the departed
\item \textbf{Protection Magic}\index{Spiritual Magic!Protection}: Wards, blessings, purification
\end{itemize}

\subsection*{Mental Magic} \index{Mental Magic}
Affecting minds and perceptions:
\begin{itemize}
\item \textbf{Illusion}\index{Mental Magic!Illusion}: Creating false perceptions and images
\item \textbf{Telepathy}\index{Mental Magic!Telepathy}: Mind reading and communication
\item \textbf{Enchantment}\index{Mental Magic!Enchantment}: Influencing thoughts and emotions
\item \textbf{Divination}\index{Mental Magic!Divination}: Gaining knowledge through supernatural means
\end{itemize}

\paragraph{Vignette:}
\emph{The candles lean toward the oracle's breath. ``Ask,'' she whispers, ``but truth is sharp.''}

\section{Ritual Magic} \index{magic!rituals}

Rituals take Significant Time (typically 10-30 minutes) for powerful effects.

\subsection*{Ritual Requirements}
\begin{itemize}
\item \textbf{Time}\index{rituals!time}: Significant Time (typically 10-30 minutes)
\item \textbf{Preparation}\index{rituals!preparation}: Specific materials, locations, or conditions
\item \textbf{Focus}\index{rituals!focus}: Undisturbed concentration and coordination
\end{itemize}

\subsection*{Ritual Procedure}
\begin{enumerate}
\item \textbf{Preparation}\index{rituals!preparation}: Gather components, prepare space, focus intent
\item \textbf{Invocation}\index{rituals!invocation}: Perform the Rite as a ritual
\item \textbf{Completion}\index{rituals!completion}: Effect manifests, always marks +1 Obligation
\end{enumerate}

\subsection*{Ritual Benefits and Risks}
\begin{itemize}
\item \textbf{Benefits}\index{rituals!benefits}: Safe casting, no Push It option
\item \textbf{Risks}\index{rituals!risks}: Time investment, Obligation cost, environmental requirements
\end{itemize}

\newpage
\section{Rites and Pact Magic} \index{magic!rites} \index{magic!pacts}

Rites are precise magical effects gained through \textbf{pacts}\index{pacts} with powerful entities. There are two main paths to accessing Rites:

\subsection*{The Runekeeper (Rites Path)}
\begin{itemize}
\item Requires Patron + Thiasos (Familiar) + Codex (4 XP)
\item Accesses that Patron's full Rite list
\item Structured, powerful, but accrues \textbf{Obligation}
\item Can Push Rites once per scene for +1 Obligation
\end{itemize}
 
\label{sec:advanced_rituals}

For players seeking deeper ritual optimization, the following house rules expand the Invoker's capabilities while maintaining game balance:

\subsubsection{Collaborative Ritual Casting}
Multiple casters can combine efforts to reduce casting time and DV for complex rites:
\begin{itemize}
    \item \textbf{Assistant Requirement}: Each assistant must possess at least Familiar Bond and relevant magical training
    \item \textbf{Time Reduction}: -1 round per assistant (minimum 1 round)
    \item \textbf{DV Reduction}: -1 per assistant (minimum DV 1)
    \item \textbf{Obligation Sharing}: Split total Obligation cost among participants
    \item \textbf{Risk Distribution}: Story Beats from failures affect all participants
\end{itemize}

\subsubsection{Ritual Optimization Talents}
\begin{description}
    \item[Ritual Synchronization (6 XP)] Prereq: Efficient Invocation, Crack Specialist. 
        When assisting another Invoker, reduce their DV by an additional 1 and share any SB costs.
        
    \item[Concurrent Casting (8 XP)] Prereq: Tier III+, Invoker's Grimoire. 
        Maintain one ritual while beginning another, but mark +1 Fatigue each round both are active.
        
    \item[Sympathetic Resonance (4 XP)] Prereq: Two Patron Symbols. 
        When casting rites from allied Patrons, reduce Cross-Resonance penalty by 1.
        
    \item[Ritual Efficiency Matrix (12 XP)] Prereq: Tier III+, 4+ Patron Symbols. 
        Create a permanent ritual circle that reduces DV by 2 for all rites cast within, but requires 
        monthly maintenance (1 XP) and attracts +1 SB attention per session.
\end{description}

\subsubsection{Ritual Casting Modifiers}

\caption{Advanced Ritual Optimization Modifiers}
\begin{longtable}{|l|l|l|}
\hline
\textbf{Modifier} & \textbf{Effect} & \textbf{Cost/Requirement} \\
\hline
Prepared Components & -1 DV & Spend 1 day gathering rare materials \\
Focused Environment & -1 DV & Dedicated ritual space (Minor Asset) \\
Patron Alignment & -1 DV & Rite matches Patron's domain exactly \\
Ritual Assistant & -1 DV, -1 round & Qualified assistant present \\
Symbol Resonance & -1 DV & Using Symbol from ritual's origin Patron \\
\hline
\end{longtable}


\subsubsection{Mass Ritual Framework}
For large-scale collaborative workings involving 3+ casters:
\begin{itemize}
    \item \textbf{Lead Caster}: Controls primary ritual parameters
    \item \textbf{Support Casters}: Provide +1 die per assistant to Channel/Weave rolls
    \item \textbf{Synergy Bonus}: If all casters share the same Patron, gain +1 Effect
    \item \textbf{Catastrophic Backlash}: Failed mass rituals generate 2× normal SB
\end{itemize}

\section{Obligation Capacity}

A character’s \textbf{Obligation Capacity} equals Spirit + Presence.  
Track total Obligation segments across all Patrons (or Symbols, for Invokers).

\begin{itemize}
  \item \textbf{Exceeding Capacity:} For each segment above Capacity, mark 1 Fatigue. The character cannot Invoke Rites or perform rituals until Obligation is reduced below Capacity.
  \item \textbf{Resolution:} Reduce Obligation through Downtime service, Patron tasks, ritual cleansing, or story resolution.
\end{itemize}

\textbf{Example:} Spirit~2 + Presence~3 = Capacity 5.  
6 segments → Fatigue~1.  
7 segments → Fatigue~2.  
10 segments → Harm~1.  
11 segments → Harm~2.

\subsection*{Obligation Management}
Your debt to Patrons must be managed:
\begin{itemize}
\item \textbf{Service}\index{Obligation!service}: Perform tasks fitting your Patron's nature
\item \textbf{Offerings}\index{Obligation!offerings}: Provide sacrifices or tributes
\item \textbf{Propagation}\index{Obligation!propagation}: Spread your Patron's influence or beliefs
\item \textbf{Downtime}\index{Obligation!downtime}: Clear through fitting service during downtime
\end{itemize}

\subsection*{Obligation Levels}
\begin{center}
\small
\begin{longtable}{ll}
\toprule
\textbf{Segments} & \textbf{Consequences} \\
\midrule
1--2 & Minor attention, subtle signs \\
3--5 & Noticeable influence, regular demands \\
6--8 & Significant control, major tasks required \\
9+   & Dominant influence, potentially dangerous \\
\bottomrule
\end{longtable}
\end{center}

\paragraph{Vignette:}
\emph{At the crossroads, Ash lays iron nails and salt. The wind shifts. Somewhere, something smiles.}

\section{Special Magical Abilities} \index{magic!special abilities}

Some characters develop unique magical capabilities through experience or heritage.

\subsection*{Cultural Magical Traditions}
\begin{itemize}
\item \textbf{Dwarven Stone-Sense}\index{Stone-Sense}: Intuitive understanding of earth and stone
\item \textbf{Elven Memory-Weaving}\index{Memory-Weaving}: Accessing and manipulating ancestral knowledge
\item \textbf{Human Versatility}\index{Versatility (human)}: Adaptable magical approaches from various traditions
\item \textbf{Nomadic Spirit-Walking}\index{Spirit-Walking}: Journeying between physical and spiritual realms
\end{itemize}

\subsection*{Advanced Magical Techniques}
\begin{itemize}
\item \textbf{Spell Shaping}\index{Spell Shaping}: Modifying non-ritual spell factors (range/scale/targeting)
\item \textbf{Ritual Mastery}\index{Ritual Mastery}: Perform powerful rituals with reduced risk
\item \textbf{Arcane Dominance}\index{Arcane Dominance}: Overpower weaker magical effects automatically
\end{itemize}

\section{Magical Backlash Examples} \index{magic!backlash examples}

\subsection*{Elemental Backlash}
\begin{itemize}
\item \textbf{Fire}\index{backlash!Fire}: Burns, flares; vs. Water: slick, sputter, dim
\item \textbf{Water}\index{backlash!Water}: Slippery tide, slow gear; vs. Fire: smoke, shorted gear
\item \textbf{Earth}\index{backlash!Earth}: Slips, binds, encumbrance; vs. Air: sound carries, exposure
\item \textbf{Air}\index{backlash!Air}: Scatter, misheard words; vs. Earth: stuck, dust choke
\end{itemize}

\subsection*{Conceptual Backlash}
\begin{itemize}
\item \textbf{Fate}\index{backlash!Fate}: Options close, only-one-way; vs. Luck: mischance hits ally
\item \textbf{Life}\index{backlash!Life}: Growth surge, vines tether; vs. Death/Dreams: numbness, sleep-tug
\item \textbf{Luck}\index{backlash!Luck}: Odds flip; vs. Fate: harsher fixed outcome
\item \textbf{Death/Dreams}\index{backlash!Death}: Whispers, chill; vs. Life: pain returns, rot
\end{itemize}

\section{Magical Item Creation} \index{magic!item creation}

Creating permanent magical items is a complex process that demands mastery of both arcane theory and material craft. Unlike temporary enchantments or Patron's Gift (Imbunements), true magical items possess a spark of enduring power that must be carefully woven into their very essence. The creation of such artifacts is often a campaign-defining endeavor, requiring significant resources, expertise, and often, great personal sacrifice.

\subsection*{Creation Requirements}
\begin{itemize}
\item \textbf{Knowledge}\index{item creation!knowledge}: Understanding of the desired effect
    \begin{itemize}
        \item Requires theoretical mastery equivalent to \textit{Arcana + Lore} skill rating equal to half the item's Tier (rounded up).
        \item Must successfully complete a \textit{Lore + Investigation} test (DV 3-5, based on item rarity and complexity) to decipher or develop the underlying magical formula.
        \item For items replicating known Rites or Talents, creator must have personally used or witnessed the effect.
    \end{itemize}
    
\item \textbf{Materials}\index{item creation!materials}: Appropriate components with magical properties
    \begin{itemize}
        \item \textbf{Base Material}: Rare substance appropriate to the item's nature (e.g., cold iron for weapons, star-silver for jewelry, wyrm-scale for armor). Cost: 2 XP per item Tier.
        \item \textbf{Essence Catalyst}: A material embodying the desired effect's element or concept (e.g., phoenix feather for fire effects, deep-sea pearl for water magic, philosopher's stone for transmutation). Cost: 3 XP.
        \item \textbf{Binding Agent}: Rare reagent that fuses magic to matter (e.g., dragon's blood, crystallized mana, soul-shard). Cost: 2 XP.
        \item \textbf{Total Material Cost}: Minimum 7 XP for a Tier I item, scaling by 5 XP per Tier increase.
    \end{itemize}
    
\item \textbf{Time}\index{item creation!time}: Significant investment of time and effort
    \begin{itemize}
        \item \textbf{Design Phase}: 1 week per item Tier for research and planning.
        \item \textbf{Gathering Phase}: Variable, but typically 2-4 weeks to acquire rare materials.
        \item \textbf{Creation Phase}: 1 month per item Tier for actual crafting and enchantment.
        \item \textbf{Total Time Investment}: Minimum 2 months for a Tier I item, potentially years for artifacts.
        \item \textbf{Rushed Creation}: Halving time requires a successful \textit{Wits + Resolve} test (DV 4) or the item gains the \textit{Unstable} drawback.
    \end{itemize}
    
\item \textbf{Skill}\index{item creation!skill}: High level of magical and craft skills
    \begin{itemize}
        \item Requires \textit{Arcana 3+} and relevant \textit{Craft} skill (Weaponsmith, Armorsmith, Jeweler, etc.) at 2+.
        \item For items above Tier II, requires either \textit{Arcana 4+} or a relevant \textit{Prestige Talent} (e.g., \textit{Elemental Mastery}, \textit{Ritual Mastery}).
        \item May substitute \textit{Lore 4+} for \textit{Arcana} when creating knowledge-based items (tomes, artifacts of understanding).
    \end{itemize}
    
\item \textbf{Facilities}\index{item creation!facilities}: Proper workspace with necessary tools
    \begin{itemize}
        \item Requires a dedicated workshop appropriate to the item type (smithy, laboratory, enchanting chamber).
        \item Must be consecrated or prepared for magical work: \textit{Lore + Craft} test (DV 3) to properly sanctify.
        \item For items above Tier III, requires a \textit{Major Magical Workshop Asset} (12 XP) or equivalent Patron blessing.
        \item Failure to maintain proper facilities risks \textit{Catastrophic Backlash} during creation.
    \end{itemize}
\end{itemize}

\subsection*{Creation Process}
\begin{enumerate}
\item \textbf{Design}\index{item creation!design}: Plan the item's properties and limitations
    \begin{itemize}
        \item Define the item's core effect using [TAG] notation (see §4.17).
        \item Determine item Tier (I-V) based on power level and scope of effect.
        \item Establish base limitations: charges, attunement requirements, activation method.
        \item Calculate total XP cost: Base Tier Cost + Material Cost + Facility Requirements.
        \item \textbf{Base Tier Costs}: Tier I (5 XP), Tier II (8 XP), Tier III (12 XP), Tier IV (18 XP), Tier V (25 XP).
        \item Must present complete design to GM for approval and potential modification based on campaign balance.
    \end{itemize}
    
\item \textbf{Gathering}\index{item creation!gathering}: Acquire necessary materials and components
    \begin{itemize}
        \item Material acquisition often involves quests, negotiations, or dangerous expeditions.
        \item Each rare component should have its own minor adventure or significant cost.
        \item GM may require \textit{Presence + Sway} or \textit{Wits + Investigation} tests to locate suppliers.
        \item Failure to acquire proper materials may force substitutions that create item drawbacks.
    \end{itemize}
    
\item \textbf{Crafting}\index{item creation!crafting}: Physical creation of the item base
    \begin{itemize}
        \item Requires extended use of appropriate Craft skill.
        \item Each week of crafting requires a successful skill test (DV based on item complexity).
        \item Failure results in material waste (lose 1 XP worth of materials) and setback (add 1 week to creation time).
        \item Critical failure may result in a \textit{Catastrophic Backlash} or creation of a \textit{Cursed} item.
        \item May take Downtime Actions to accelerate progress, but risks introducing flaws.
    \end{itemize}
    
\item \textbf{Enchantment}\index{item creation!enchantment}: Magical infusion of the desired properties
    \begin{itemize}
        \item The most dangerous phase, requiring precise channeling of magical energy.
        \item Perform an extended ritual: \textit{Wits + Arcana} test with DV equal to item's Tier + 2.
        \item Ritual duration: 1 hour per item Tier.
        \item Generate Story Beats equal to item's Tier - treat each SB as a potential \textit{Magical Backlash}.
        \item GM may spend SB to introduce item-specific quirks or limitations.
        \item Success imbues the item with its magical properties but may mark the creator with \textit{Magical Scarring} (permanent +1 Obligation to relevant Patron or -1 die to one magical skill).
    \end{itemize}
    
\item \textbf{Finishing}\index{item creation!finishing}: Final adjustments and testing
    \begin{itemize}
        \item Test the item's functionality with a \textit{Wits + Arcana} test (DV = Item Tier).
        \item Failure may result in unstable enchantment or incomplete activation.
        \item Successful completion allows the item to be used according to its design parameters.
        \item Creator gains 1 XP for successfully completing the creation process, representing hard-won experience.
        \item Item is now a permanent Asset, subject to standard upkeep rules (see §8.3).
    \end{itemize}
\end{enumerate}

\subsection*{Item Limitations}
\begin{itemize}
\item \textbf{Charges}\index{items!charges}: Limited uses before needing recharge
    \begin{itemize}
        \item Items without charges cost 50\% more XP to create.
        \item Base charges: 1d6 + Item Tier per day of use.
        \item Recharging typically requires 1 day per charge expended, plus appropriate materials.
        \item Expending the last charge may risk \textit{Item Burnout}: 1 SB spent by GM to determine consequence.
    \end{itemize}
    
\item \textbf{Attunement}\index{items!attunement}: Required bonding with the user
    \begin{itemize}
        \item Most magical items above Tier I require attunement.
        \item Attunement process: 1 hour of uninterrupted focus + \textit{Spirit + Resolve} test (DV 3).
        \item Failure means item functions at -1 Effect or with significant drawbacks.
        \item Characters may be attuned to a number of items equal to their Tier.
        \item Dying while attuned to an item may trap part of the character's soul within it.
    \end{itemize}
    
\item \textbf{Maintenance}\index{items!maintenance}: Regular upkeep to preserve functionality
    \begin{itemize}
        \item Items require maintenance equal to their Tier in XP per major Downtime period.
        \item Neglect results in item becoming \textit{Neglected} (-1 Effect) or \textit{Dormant} (loses magical properties).
        \item Restoration from Dormant state requires half the original creation time and materials.
        \item Some items may require specific upkeep actions (feeding, ritual cleansing, etc.).
    \end{itemize}
    
\item \textbf{Drawbacks}\index{items!drawbacks}: Negative side effects or requirements
    \begin{itemize}
        \item \textbf{Corruption}: Item generates 1 SB per day worn/carried.
        \item \textbf{Hunger}: Item drains 1 Fatigue per hour from attuned user.
        \item \textbf{Demanding}: Item requires specific action or sacrifice once per session.
        \item \textbf{Sentient}: Item possesses its own personality and agenda.
        \item \textbf{Cursed}: Item cannot be removed without powerful magic and has harmful effects.
        \item Drawbacks can reduce the XP cost of item creation by 10-30\%.
    \end{itemize}
\end{itemize}

\subsection*{Artifact Creation (Tier IV-V)}
\index{magic!artifact creation}
\index{items!artifacts}

True artifacts transcend normal magical items, often possessing intelligence, vast power, and the ability to reshape reality within their sphere of influence.

\begin{itemize}
    \item \textbf{Requirements}: Tier V minimum, creator must be Tier IV+, requires a \textit{Mythic Quest} as part of the gathering phase.
    \item \textbf{Cost}: Minimum 50 XP base cost, plus materials (typically 15+ XP).
    \item \textbf{Creation Time}: Minimum 6 months, often requiring multiple creators or Patron intervention.
    \item \textbf{Unique Properties}: May possess multiple [TAG] effects, generate their own Story Beats, or alter fundamental rules within their domain.
    \item \textbf{Risks}: Creation often involves \textit{Permanent Sacrifice} (loss of Talent, Attribute point, or significant life experience). Failure may result in \textit{Cataclysmic Backlash} affecting entire regions.
\end{itemize}

\textbf{Note}: The creation of artifacts should fundamentally change the campaign setting and is subject to GM approval and significant narrative justification.

\section{Magic in Social Situations} \index{magic!social use}

Using magic in social contexts has special considerations.

\subsection*{Social Spellcasting}
\begin{itemize}
\item \textbf{Discretion}\index{social magic!discretion}: Avoiding detection while casting
\item \textbf{Consent}\index{ethics!consent}: Ethical considerations of affecting others' minds
\item \textbf{Reactions}\index{social magic!reactions}: How different cultures view magical influence
\item \textbf{Laws}\index{social magic!laws}: Legal restrictions on magical use in society
\end{itemize}

\subsection*{Social Backlash}
Magical social failures can cause:
\begin{itemize}
\item \textbf{Distrust}\index{social backlash!distrust}: People becoming wary of the caster
\item \textbf{Resistance}\index{social backlash!resistance}: Developing immunity or countermeasures
\item \textbf{Reputation}\index{reputation}: Becoming known as a manipulator
\item \textbf{Legal}\index{social backlash!legal}: Facing consequences from authorities
\end{itemize}

\section{Learning and Improving Magic} \index{magic!improvement}

Magical ability grows through study and practice.

\subsection*{Skill Advancement}
\begin{itemize}
\item \textbf{Study}\index{magic!study}: Researching magical theory and techniques
\item \textbf{Practice}\index{magic!practice}: Regular casting to improve control
\item \textbf{Experimentation}\index{magic!experimentation}: Trying new approaches and combinations
\item \textbf{Instruction}\index{magic!instruction}: Learning from more experienced casters
\end{itemize}

\subsection*{Advanced Magical Development}
At higher levels, casters can:
\begin{itemize}
\item \textbf{Specialize}\index{magic!specialize}: Focus on specific magical traditions
\item \textbf{Innovate}\index{magic!innovate}: Create new spells or techniques
\item \textbf{Teach}\index{magic!teach}: Instruct others in magical arts
\item \textbf{Research}\index{magic!research}: Discover lost or forbidden knowledge
\end{itemize}

\section{Magical Safety and Ethics} \index{magic!safety} \index{magic!ethics}

Responsible magical practice involves understanding risks and consequences.

\subsection*{Safety Considerations}
\begin{itemize}
\item \textbf{Containment}\index{safety!containment}: Preventing unintended spread of effects
\item \textbf{Stability}\index{safety!stability}: Ensuring magical effects remain controlled
\item \textbf{Fail-safes}\index{safety!failsafes}: Planning for when magic goes wrong
\item \textbf{Recovery}\index{safety!recovery}: Procedures for dealing with backlash
\end{itemize}

\subsection*{Ethical Guidelines}
\begin{itemize}
\item \textbf{Consent}\index{ethics!consent}: Respecting others' autonomy regarding magic
\item \textbf{Transparency}\index{ethics!transparency}: Being honest about magical capabilities
\item \textbf{Restraint}\index{ethics!restraint}: Using magic judiciously and appropriately
\item \textbf{Responsibility}\index{ethics!responsibility}: Accepting consequences of magical actions
\end{itemize}

\begin{tcolorbox}[colback=purple!5!white,colframe=purple!75!black,title=Magic Quick Reference,fonttitle=\bfseries]
\textbf{Casting (Freeform)}\index{magic!casting}:
\begin{itemize}
\item Requires Talent: \textbf{Spellcraft} (6 XP)
\item \textbf{Weave \& Cast}: Two action effect using the Eight Elements
\item \textbf{Backlash}\index{magic!backlash}: Any 1 rolled may cause narrative backlash
\end{itemize}

\textbf{Backlash Severity}\index{magic!backlash}:
\begin{itemize}
\item On Partial/Miss: Pick 1-2 consequences flavored by Element
\item Color consequences by Element (fire burns, fate twists, etc.)
\end{itemize}

\textbf{Rites System}\index{Rites}:
\begin{itemize}
\item \textbf{Invoke}: 1 action effect
\item \textbf{Obligation}: Mark segments on clock
\item \textbf{Push It}: +1 Obligation for +1 step effect
\end{itemize}


\centering
\caption{Universal Push It Costs}
\begin{longtable}{|l|l|}
\hline
\textbf{Cost Component} & \textbf{Effect} \\ 
\hline
+1 SB & Escalate effect immediately \\ 
+1 Fatigue & Immediate physical/mental strain \\ 
+1 Corruption Clock Segment & Long-term Patron influence (unless otherwise specified) \\ 
GM spends 1 SB & Thematic complication (unless otherwise specified) \\ 
\hline
\end{longtable}


Note: Some talents, Rites, or magical paths may specify alternative corruption costs or additional consequences for Push It actions. When explicitly stated, those specific rules override the universal costs.

\paragraph{Clearing Corruption}
Corruption may be reduced through \textit{purging rituals}, such as exorcisms, sacred songs, or rites of contrition. 
These require a test (typically \textbf{Lore + Spirit}) against a DV equal to the character’s current corruption level.  
On success, reduce corruption by 1. On failure, the corruption manifests violently, imposing a temporary Condition or advancing its narrative expression.  

Optional: A \textbf{Story Beat} may also be spent to attempt such a ritual, representing the personal cost of atonement. Patrons may demand specific acts of service, sacrifice, or obligation as part of the purging process.

\textbf{Invoker Path}\index{Invoker}:
\begin{itemize}
\item \textbf{Symbols} (4 XP each) grant ritual access
\item \textbf{Rituals}: Significant Time, always +1 Obligation
\item \textbf{Crack the Seal}: Instant cast (+2/+3 Obligation)
\end{itemize}

\textbf{Safety}\index{magic!safety}: Every roll changes the story. Success without risk is rare.
\end{tcolorbox}

\section{Practical Magic Examples} \index{examples}

\subsection*{Fire Cast, Partial}
You Weave flame to blind a squad (DV 3). Partial with two 1s. GM spends SB to Position -1 (flare blinds you too) and colors backlash as singed lashes; patrol is alerted (Exposure).

\subsection*{Runekeeper Push and Debt}
You Invoke Circle of Denial [WARD] and Push It to harden the ring. Mark +1 Obligation for the Rite plus +1 for the push. When a demon tests the ring, use [WARD] vs Cap; on its Hit, add +DV to its Leash.

\subsection*{Crack the Seal Under Fire}
You present Ikasha's Symbol and Crack the Seal to lay an instant shadow lane. Symbol $\rightarrow$ Compromised; mark +2 Obligation. GM immediately spends 1 SB to dim all lights (panic), then the lane forms. During downtime, you restore the Symbol (Arcana DV 3): a shaky hit leaves it Neglected until you perform the full rite of cleaning.

\section{Talent: Cantor's Path --- ``Songs of the Low Rites''}
\label{talent:cantors-path}

\begin{tcolorbox}[colback=black!3,colframe=black!40!white,title={Cantor's Path}]
You echo the liturgies of Patrons through breath and string. Not a sworn celebrant but a perilous mimic, you weave Low Rites into song. It is slower, riskier, and beautiful---but never free.
\end{tcolorbox}

\paragraph*{Type} Major Talent (8 XP) \quad
\paragraph*{Prerequisites} \textbf{Lore 1+}, \textbf{Performance 2+}, \textbf{Presence 2+} \quad
\paragraph*{Access} Any character (does not require Thiasos membership).

\subsection*{Effect}
You may learn and perform \textbf{Low Rites as Songs}. Each Song counts as knowing the associated Low Rite for performance purposes only.

\begin{itemize}
  \item \textbf{Casting Test:} \emph{Lore + Performance vs.\ DV} (default DV = 2--3).
  \item \textbf{Action Economy:} \emph{1 action to begin;} the Song \emph{resolves at the start of your next turn} unless accelerated.
  \item \textbf{Scope:} \emph{Low Rites only.} Standard/High Rites remain exclusive to Patrons and Thiasos initiates.
  \item \textbf{Costs:} Pay any \emph{materials} listed. On success you do \emph{not} mark Obligation.
\end{itemize}

\subsection*{Performance Integration}
Songs are most effective when performed as part of social performances:
\begin{itemize}
  \item \textbf{Audience Awareness:} Perform in front of 5+ observers for +1 die but +1 Corruption risk.
  \item \textbf{Cultural Context:} Appropriate venues/occasions grant +1 Effect.
  \item \textbf{Social Momentum:} Successful performances create opportunities for additional Songs in the same scene.
\end{itemize}

\subsection*{Song Repertoire Progression}
Develop a \textbf{Repertoire Clock [6]} to track learned Songs:
\begin{itemize}
  \item Mark a segment for each \emph{unique} Song learned through practice or exposure.
  \item At 2 segments: Reduce base DV of Songs by 1 (minimum 2).
  \item At 4 segments: Gain +1 die to Song performances.
  \item At 6 segments: Learn one \emph{Standard Rite as a Song} (temporary, requires ongoing practice).
\end{itemize}

\subsection*{Corruption Clock}
\begin{itemize}
  \item You gain a personal \textbf{Corruption Clock} with segments equal to your \textbf{Body} rating.
  \item \textbf{Mark Corruption when:}
    \begin{itemize}
        \item You \textbf{Push It} (Song resolves immediately).
        \item You perform a \textbf{Resonant Rite}.
        \item The Keeper spends a Story Beat involving your psionic/occult activities.
    \end{itemize}
  \item \textbf{Corruption Accumulation:} Multiple triggers may be required to mark a segment:
    \begin{itemize}
        \item \textbf{2 Push It uses} = +1 Corruption segment
        \item \textbf{1 Push It + 1 Resonant Rite} = +1 Corruption segment
        \item \textbf{3 GM SB spends} on occult activities = +1 Corruption segment
        \item \textbf{1 High Cantor Standard Rite} = +1 Corruption segment
    \end{itemize}
  \item When the Clock fills:
    \begin{itemize}
      \item You immediately gain a \textbf{thematic benefit} and \textbf{drawback} from the last Patron whose Rite you performed.
      \item All of your followers, retainers, or familiars also gain a trait of the same corruption.
      \item Reset the Clock, but it cannot go below your character's \textbf{Tier} (minimum corruption).
    \end{itemize}
  \item Corruption traits can be \textbf{Embraced} for permanent thematic advantages.
\end{itemize}

\subsection*{Thematic Corruption Benefits}
Instead of purely punitive effects, Corruption creates character-defining traits:
\begin{description}
  \item[Ikasha (Shadow):] +1 die to Stealth in shadows, but $-1$ die in bright light; always noticed by shadow-dwellers.
  \item[Inaea (Mercy):] +1 die to social manipulation, but $-1$ die when alone; compelled to offer aid to the helpless.
  \item[Isoka (Change):] +1 die to escape/transform actions, but $-1$ die to maintain consistency; physical changes become visible.
  \item[Raéyn (Sea):] +1 die to water/navigational tasks, but $-1$ die on land; attracts sea creatures.
  \item[Aveh (Freedom):] +1 die to escape/avoidance, but $-1$ die to commitments; leaves traces of passage.
\end{description}

\subsection*{Resonant Rites}
Some powerful or thematically significant Low Rites carry the weight of the Patron's direct influence. Performing these Rites is a conscious act of drawing deep power.
\begin{itemize}
    \item When learning a Song that mimics such a Rite, the GM or the rules text will designate it as \textbf{Resonant}.
    \item Performing a \textbf{Resonant Rite Song} successfully allows you to mark +1 segment on your Corruption Clock. This represents the lingering echo of power.
    \item \textbf{Choosing to Resonate} is optional. You can perform the Rite normally without marking Corruption.
    \item This choice adds a layer of strategy: is the Rite's power worth the potential long-term cost?
\end{itemize}

\subsection*{Song Synergy System}
Create combinations and interactions between Songs:
\begin{itemize}
  \item \textbf{Harmony:} Performing two compatible Songs grants +1 Effect to both.
  \item \textbf{Counterpoint:} Using opposing Songs can cancel negative effects.
  \item \textbf{Chorus:} With allies, combine Songs for amplified effects (+1 Effect per participant).
\end{itemize}

\subsection*{Outcomes}
\begin{description}
\item[Success:] The Low Rite takes effect as written.
\item[Partial:] The Rite manifests with reduced effect (one step) or shortened duration. Mark \textbf{Fatigue 1}.
\item[Failure:] No effect; mark \textbf{Fatigue 1} and the Keeper gains \textbf{+1 SB (Hearts)}.
\item[Interrupted:] Harm, Silence, or disruption before resolution = treat as Failure.
\end{description}

\subsection*{Push It}
When you Push:
\begin{itemize}
  \item The Song resolves immediately instead of next round.
  \item Mark \textbf{Fatigue 1}.
  \item \textbf{Mark toward Corruption accumulation} (see Corruption Clock).
  \item The Keeper immediately triggers a \textbf{Story Beat}, representing fallout from a Patron, the Road, or social attention.
\end{itemize}

\subsection*{Enhanced Departure Options}
\begin{itemize}
  \item \textbf{Graceful Coda:} End a Song early to gain +1 Boon and reduce Corruption accumulation progress by 1 (if any progress exists).
  \item \textbf{Lingering Verse:} Song effect continues for one round after ending, but mark +1 Fatigue.
  \item \textbf{Audience Impact:} A successful Song performance improves social Position +1 for the next interaction.
\end{itemize}

\subsection*{Limits \& Interactions}
\begin{itemize}
  \item \textbf{Stacking:} Cannot benefit from the same Rite twice.
  \item \textbf{Visibility:} Songs are inherently noticeable. On Failure or Push, assume observers take note.
  \item \textbf{Silence/Disruption:} Impose $-1$ to $-3$ dice at the Keeper's discretion.
  \item \textbf{Obligation Transference:} Whenever a Rite would normally increase Obligation, it instead increases Corruption accumulation progress.
\end{itemize}

\subsection*{Downtime Activities}
\begin{itemize}
  \item \textbf{Song Composition:} Practice and refine Songs, potentially reducing their DV or Corruption risk.
  \item \textbf{Performance Practice:} Improve Performance skill and social reputation.
  \item \textbf{Patron Study:} Research new Rites to add to your Repertoire.
  \item \textbf{Audience Building:} Cultivate followers who provide +1 die to future performances.
\end{itemize}

\subsection*{Talents}

\subsection*{Talent: Resonant Performance (3 XP)}
\textbf{Requirements:} Cantor's Path, Performance 2+ \\
\textbf{Effect:} When performing a Song in front of an audience of 5+ people, reduce Corruption generation requirements by 1 (minimum 1 trigger) and gain +1 die to the performance.

\subsection*{Talent: Song Weaver (4 XP)}
\textbf{Requirements:} Cantor's Path, Repertoire Clock at 4+ segments \\
\textbf{Effect:} Combine two compatible Songs for +1 Effect to both. Once per scene, create Harmony between Songs for all participants.

\subsection*{Talent: Siren's Call (Major Talent - 8 XP)}
\textbf{Requirements:} Cantor's Path, Performance 3+, Repertoire 4+ \\
\textbf{Effect:} Your Songs can compel supernatural beings.
\begin{itemize}
  \item \texttt{[COMMAND]} effects work on Outsiders (Cap 3 or less)  
  \item Resistance is Spirit + Resolve vs. your Performance + Lore
  \item On success: outsider acts as commanded for one exchange
  \item On failure: generate 2 SB, outsider becomes hostile
\end{itemize}

\subsection*{Song Specialization Paths}
\begin{description}
  \item[Battle Cantor:] War Songs grant allies +1 Position in combat; Hymn of Fury converts 1 Harm to Fatigue for allies Near you; Anthem of the Fallen allows departed allies to return as spectral echoes (1/session).
  
  \item[Shadow Cantor:] Songs of Veiling create \texttt{[VEIL]} effects without ritual components; Melody of Misdirection imposes -1d to Notice rolls on enemies; Dirge of Passing enables communication with dead and scrying through recent deaths.
  
  \item[Healing Cantor:] Songs of Restoration heal +1 Harm; Chant of Purification removes poison/disease; Hymn of Vitality grants temporary +1 Body.
  
  \item[Knowledge Cantor:] Lore Songs reveal hidden knowledge; Chant of Understanding grants +2d to Investigation/Lore; Ode to Memory allows perfect recall of witnessed events.
\end{description}

\subsection*{Corruption Fading}
\label{subsec:corruption-fading}
\index{Corruption!Fading}

Corruption does not fade easily. It requires deliberate action and often, a price.
\begin{description}
  \item[\indexterm{Natural Fading}]  
  At the beginning of each Downtime, reduce a character's current \textbf{Corruption accumulation progress} by 1 step, and reduce the total \textbf{Corruption segments} by 1 (to a minimum of the character's Tier). Lingering effects persist until actively addressed.

  \medskip
  \item[\indexterm{Act of Contrition}]  
  Perform a genuine act that contradicts the Patron's influence or repairs its harm (GM/Player agreement on suitability). \textbf{Effect:} Remove 1 Corruption segment and clear one persistent effect. Costs the character something significant.

  \medskip
  \item[\indexterm{Ritual Purification}]  
  Undertake a significant act of cleansing (pilgrimage, service, seeking rival absolution). \textbf{Effect:} Remove 2 Corruption segments and clear all persistent effects. Likely requires marking Fatigue or temporary Obligation.

  \medskip
  \item[\indexterm{Embrace Corruption}] \label{talent:embrace-corruption}
  \textbf{Type:} Major Talent (6 XP) \quad
  \textbf{Prerequisite:} 2+ levels of Corruption. \\
  You accept the creeping decay, transforming it into a permanent Talent. \textbf{Embracing locks your Corruption at its current level---it reshapes it.} The deeper the corruption, the greater the power and the cost.
    \begin{itemize}
        \item Gain a \textbf{Minor} permanent thematic boon/condition related to the Patron (e.g., +1 die to Stealth in shadows for Ikasha, but $-1$ die in bright light).
        \item Your Corruption cannot naturally fade below the level at which you Embraced it.
        \item The Keeper gains +1 SB to spend against you related to that Patron's themes.
    \end{itemize}
    \textbf{Narrative Integration:} This Talent represents the Faustian bargain. Players gain agency over their corruption, ensuring that it always carries meaningful consequences.

  \medskip
  \item[\indexterm{Patron Bargain}]  
  Negotiate directly with the Patron. \textbf{Effect:} Remove 1--3 Corruption segments based on the exchange's gravity. Always comes with a narrative cost or condition set by the Keeper.

  \medskip
  \item[\indexterm{Persistence}]  
  Corruption effects do not clear through rest. They require deliberate narrative resolution or specific actions listed above. Every method is an opportunity for character development.
\end{description}

\paragraph*{High Cantor (18 XP Prestige Talent)}%
\textit{Prerequisite: Tier II+, Cantor's Path, Performance 3+}\\[3pt]
You have learned to weave the sacred tongue through breath and pulse rather than word or gesture. You may now learn and cast \textbf{Standard Rites}, as a \textbf{High Cant}.
\begin{itemize}
  \item The Rite resolves instantly.
  \item Gain +1 die to its primary effect.
  \item \textbf{Mark toward Corruption accumulation} (1 High Cantor Standard Rite = 1 Corruption trigger).
\end{itemize}

\noindent
\textbf{Special:}  
Each Patron's resonance colors the manifestation differently---flame halos for the Oath, rippling silence for the Choir, tolling harmonics for the Confessor. High Canting is recognizable to other adepts; it draws attention. Repeated use within a single scene risks moral fatigue: add +1 DV to all subsequent \emph{Resolve} rolls against fear, charm, or social pressure in that scene.

\subsection*{Divine Resonance (Major Talent - 15 XP)}
\textit{Prerequisite: High Cantor, Performance 4+, Tier III+}\\[3pt]
Your voice carries divine authority. Once per scene, spend 2 Boons:
\begin{itemize}
  \item \textbf{Command Effect:} Issue a \texttt{[COMMAND]} that affects up to (Presence) targets simultaneously
  \item \textbf{Miracle Effect:} Replicate any Low Rite without marking Corruption (but generate 1 SB)
  \item \textbf{Omen Effect:} Gain insight into a major threat - ask 3 questions about one enemy/faction
\end{itemize}
\textbf{Cost:} Mark +2 Corruption segments, immediately trigger Patron attention.

\begin{quote}
``The louder the hymn, the nearer the flame.''
\end{quote}


%----------------------------------------
%----------------------------------------
\section{Summoning (Pact-Whisperer) }
\label{subsec:summoning}

Summoning is the disciplined art of calling and binding Outsiders for temporary aid.  
This path requires the \textbf{Pact-Whisperer} Talent (2 XP).  
Each summoned being is restrained by a metaphysical tether called a \textit{Leash}, representing the summoner's control and the strain of sustaining the bond.

\paragraph*{Talents \& Access.}
\begin{itemize}
  \item \textbf{Lesser Pactwright:} You may \emph{Call} spirits of \textbf{Cap~1}.
  \item \textbf{Greater Pactwright:} You may also \emph{Call} spirits of \textbf{Cap~3}.
  \item \textbf{Dual Pactwright:} With both Lesser and Greater Pactwright, you may maintain one spirit of each Cap simultaneously.
\end{itemize}

\begin{fatebox}[Summoning Core Mechanics]
\begin{tabularx}{\textwidth}{lX}
\toprule
\textbf{Mechanic} & \textbf{Description and Requirements} \\
\midrule
\textbf{Call} & \emph{1 Action} to manifest the spirit at \textit{Near} range; choose a Spirit Template aligned to fiction or Patron domain. \\
\textbf{Bind} & Spend 1 Boon \emph{or} mark 1 Fatigue to establish initial control. \\
\textbf{Leash Capacity} & Set Leash Capacity $=$ \textbf{Cap $+$ Spirit} \emph{segments}. \\
& (\textit{Cap} is the Outsider's tier: Cap~1 for Lesser, Cap~3 for Greater.) \\
\textbf{Tick Leash} & Whenever the spirit takes Harm; you command it against its nature; you perform a separate concentration-requiring action while commanding it; a rival contests its actions; or it crosses a \texttt{[WARD]} successfully \big($\mathrm{DV}=\mathrm{Cap}$\big). \\
\textbf{Departure} & When the Leash fills, the spirit acts to its nature once, then departs (or turns hostile at GM discretion). \\
\bottomrule
\end{tabularx}
\end{fatebox}

\paragraph*{Spirit Bond Progression.}
Each spirit you summon regularly can develop a \textbf{Spirit Bond Clock [4]}:
\begin{itemize}
  \item Mark segments for successful commands, shared victories, or acts of mutual aid.
  \item At 2 segments: +1 die to communicate with this spirit type.
  \item At 4 segments: Spirit grants +1 Boon when departing naturally and becomes \textbf{Favored} (Leash reduced by 1).
  \item Reset: Spirit departs as ally and may return in future scenes with +1 Effect.
\end{itemize}

\noindent\textbf{Near-Miss Progress.}
If a \emph{Call/Bind} fails or a spirit departs immediately after manifesting, mark \textbf{+1} on that spirit type's \textbf{Spirit Bond Clock} once per session (per spirit type), provided a meaningful attempt was made in-scene.

\paragraph*{Spirit Specialization Paths.}
Spirits can develop specialized capabilities through repeated summoning:
\begin{itemize}
  \item \textbf{Combat Specialist:} +1 Harm in melee; ignore first Harm when attacking.
  \item \textbf{Scout Form:} Extended range, stealth bonuses, can carry small items. \textit{Carry limits:} Cap~1 up to \textbf{2\,kg} (5\,lb); Cap~3 up to \textbf{10\,kg} (22\,lb). Dragging (not lifting) allows up to \textbf{3$\times$} these amounts across smooth ground. Overburdening immediately ticks the Leash.
  \item \textbf{Utility Spirit:} Perform simple tasks (lockpicking, carrying, environmental interaction).
  \item \textbf{Shield Guardian:} Interpose to protect allies; convert Harm to Fatigue.
  \item \textbf{Scholar Spirit:} Gather information, \texttt{[REVEAL]} hidden knowledge, store/cast one Rite/Lore spell through spirit bond.
  \item \textbf{Battle Spirit:} Enhanced combat abilities, Spirit Shield Wall (+1d Defense for allies in Near), tactical coordination.
\end{itemize}

\paragraph*{Procedure.}
\begin{enumerate}
  \item \textbf{Call (1 Action):} A spirit manifests at \textit{Near}. Choose a Spirit Template appropriate to the scene or Patron.
  \item \textbf{Bind:} Spend 1 Boon \emph{or} mark 1 Fatigue to anchor the connection.
  \item \textbf{Leash Capacity:} Record Leash Capacity $=$ \textbf{Cap $+$ Spirit} \emph{segments}. Draw a clock to track strain (the Leash).
  \item \textbf{Command:} Each round, issuing a meaningful order uses your Action. Commands contrary to the spirit's nature tick the Leash.
  \item \textbf{Maintain:} If you perform a separate action requiring concentration (e.g., casting a spell, picking a lock) while actively directing the spirit's complex actions, tick the Leash.
  \item \textbf{Departure:} When the Leash fills, the spirit acts to its nature once, then departs. Use this to escalate or reveal consequences.
\end{enumerate}

\paragraph*{Enhanced Action Economy.}
\begin{itemize}
  \item \textbf{Spirit Assist:} Once per scene, the spirit can grant +2 dice to an ally's roll instead of acting.
  \item \textbf{Quick Command:} Simple commands (attack, move, defend) do not require a full Action for the summoner.
  \item \textbf{Spirit Resonance:} When commanding multiple spirits of the same type, +1 Effect.
  \item \textbf{Honorable Departure:} Voluntarily end a summon early to gain +1 Boon and reduce Leash by 2.
  \item \textbf{Spirit Link (Major Talent - 10 XP):} Your spirits act on your turn, not their own initiative. Issue commands as free actions (not full Actions). Spirits move/act immediately when commanded. Reduce Leash ticking for natural behaviors by 1.
\end{itemize}

\paragraph*{Quick Command Examples.}
The following orders qualify as \emph{Quick Command} and do not consume the summoner's full Action:
\begin{itemize}
  \item \textbf{Strike Nearest:} Attack the closest hostile.
  \item \textbf{Hold the Line:} Defend a doorway/ally; intercept the next entrant.
  \item \textbf{Relocate:} Move to that ledge/cover/marker within \textit{Near}.
  \item \textbf{Retrieve:} Fetch a dropped item within \textit{Near} and return.
  \item \textbf{Screen:} Impose disadvantage on the next hostile advance (tick Leash if against nature).
  \item \textbf{Scout Peek:} Look into the next room/corridor and report (no lingering).
\end{itemize}

\paragraph*{Economy \& Limits.}
\begin{itemize}
  \item \textbf{Boon Finesse:} Once per round, spend 1 Boon to clear 1 Leash tick (before it fills). Represents appeasement or renewed focus.
  \item \textbf{Action Economy:} Issuing commands uses your Action; most spirits act immediately after the command is given. Quick Commands do not use your Action.
  \item \textbf{Concurrency:} Only one active summoned spirit at a time unless a Talent states otherwise. Exceeding this limit inflicts 1 Fatigue per extra Cap point.
  \item \textbf{Downtime:} All summons end at Downtime unless explicitly sustained by a Rite or Asset.
\end{itemize}

\paragraph*{Talents.}

\paragraph*{Spirit Synergy (4 XP).}\\
\textbf{Requirements:} Pact-Whisperer, Lesser Pactwright.\\
\textbf{Effect:} When commanding two or more spirits simultaneously, reduce each Leash by 1 segment and gain +1 die to Command rolls.

\paragraph*{Bonded Summoner (3 XP).}\\
\textbf{Requirements:} Pact-Whisperer, Spirit Bond Clock at 2+ segments with any spirit type.\\
\textbf{Effect:} Favored spirits reduce their Leash cost by 2 (minimum 3). Once per session, recall a departed Favored spirit by spending 2 Boons.

\paragraph*{True Name Keeper (Prestige Talent - 15 XP).}\\
\textbf{Requirements:} Tier III+, Bonded Summoner, 6+ different spirit types.\\
\textbf{Effect:} You know the true names of Outsiders.
\begin{itemize}
  \item Call any previously encountered spirit by true name
  \item Reduce Leash Capacity by 2 for known spirits
  \item Banish Effect: When a spirit's Leash fills, you may instead:
  \begin{itemize}
    \item Permanently bind it as a Familiar (lose other familiar slot)
    \item Negotiate terms for continued service (+1 Obligation but no departure)
    \item Sacrifice the binding to gain major boon from Patron
  \end{itemize}
\end{itemize}

\paragraph*{Legion Master (Prestige Talent - 18 XP).}\\
\textbf{Requirements:} Tier III+, Spirit Synergy, 4+ different spirit types bonded.\\
\textbf{Effect:} You become a true commander of otherworldly forces.
\begin{itemize}
  \item Maintain up to (Presence) spirits simultaneously  
  \item Issue tactical commands as free actions to all spirits
  \item Legion's Will: Spirits gain +1 Effect when acting in coordinated groups
  \item Ultimate Ability: Call to Arms - summon one spirit of each bonded type (once/session)
\end{itemize}

\paragraph*{Example.}
\textit{Kestra calls a Cap~3 fire elemental to aid in battle. She spends 1 Boon to Bind it.  
The elemental's Leash Capacity is 7 segments (\textit{Cap}~3 $+$ \textit{Spirit}~4). When it takes Harm, the GM ticks the Leash. Later, Kestra casts a spell while directing the elemental, ticking the Leash again for splitting focus.  
Careful management and Boon Finesse keep the bond stable---until the elemental's fury tests her will. After the battle, she marks her Spirit Bond Clock +1 for the shared victory.}

\paragraph{Balance by Asymmetry.}
\index{Balance}
These paths do not share identical mechanics. They are balanced narratively:
\begin{itemize}
  \item \textbf{Summoners} gain sustained power and versatility, but risk catastrophic loss of control.
  \item \textbf{Cantors} enjoy quick access to magic without a Patron, but corruption erodes them over time.
  \item \textbf{Casters} can attempt nearly anything, but risk explosive elemental backlash.
  \item \textbf{Runekeepers} unleash powerful effects instantly, but every use deepens Patron obligations.
  \item \textbf{Invokers} can safely reshape the world through ritual, but rarely in the heat of battle.
\end{itemize}

Collectively, they form a complete \textbf{pentarchy of power}—distinct, dramatic, and tactically meaningful. No path is universally superior; each shines in different challenges and story arcs.

\section*{\[TAGS\] System)}
Some casters do not prepare rote rites. They shape raw forces through shared arcane grammar known as \textbf{TAGS}. A spell is constructed at the table using a short phrase of TAGS. You only need the fiction, the TAG selection, and a casting roll.
\label{magic:tags}

\subsection*{Spell Structure}
\textbf{Intent} + \textbf{Target} + \textbf{Tags} = effect.

Example formula:
\begin{quote}
``I unleash \[BURNING\] • Area • Force against the marauders.''
\end{quote}

The GM sets a Difficulty Value (DV) based on TAG complexity and danger.

\subsection*{Base Difficulty Value (DV)}
Start at DV 1 and add +1 for each TAG used.

\begin{center}
\textbf{DV = 1 + number of TAGS}
\end{center}

Adding powerful or perilous TAGS (Teleportation, Transformation, Dominate) adds +2 instead.

Mastery, focus, or appropriate tools may lower DV by 1.

\subsection*{Casting Roll}
Roll \textbf{Wits + Arcana} (or Ritual, Channeling, etc.).  
Success = spell goes off.  
Failure or 1 = Backlash (see below).

\subsection*{Backlash}
Whenever a Free Caster fails—or pushes power beyond safety—the magic pushes back. Choose one:
\begin{itemize}
\item Harm 2 (Arcane)
\item +2 Fatigue
\item Corruption +1
\item Catastrophic side effect (GM describes)
\end{itemize}

If the spell included a ``Dangerous'' TAG, Backlash triggers on \emph{mixed} results as well.

\newpage

\section*{TAG Library}
Pick 1–3 for minor spells.  
Pick 4–6 for heavy magic (very dangerous).  
More than 6 is suicidal.

\subsection*{Elemental TAGS}
\begin{itemize}[leftmargin=*]
\item \textbf{Burning}: flame, heat, combustion.
\item \textbf{Freezing}: ice, slowing, brittle shatter.
\item \textbf{Storm}: lightning, crackling arcs, thunder shock.
\item \textbf{Stone}: walls, spikes, tremors, armor.
\item \textbf{Wave}: crushing water, currents, pressure.
\item \textbf{Wind}: levitate, gusts, deflection.
\end{itemize}

\subsection*{Force TAGS}
\begin{itemize}[leftmargin=*]
\item \textbf{Force}: pure kinetic power, shields, blasts.
\item \textbf{Area}: cone, circle, corridor, zone.
\item \textbf{Strike}: single target precision.
\item \textbf{Wall}: barrier or blockade.
\item \textbf{Bind}: restrain, hold, suspend.
\item \textbf{Dispel}: suppress magic, unravel effects.
\end{itemize}

\subsection*{Mind \& Veil TAGS}
\begin{itemize}[leftmargin=*]
\item \textbf{Veil}: conceal, blur, illusion, silence.
\item \textbf{Scry}: reveal hidden, see distance, read traces.
\item \textbf{Memory}: erase, alter, restore.
\item \textbf{Command}: compel short action.
\item \textbf{Fear}: panic, flee, break morale.
\end{itemize}

\subsection*{Life \& Body TAGS}
\begin{itemize}[leftmargin=*]
\item \textbf{Mend}: close wounds, restore flesh, reduce Harm 1.
\item \textbf{Purify}: remove poison, corruption, disease.
\item \textbf{Strengthen}: enhance body, armor, senses.
\item \textbf{Waken}: counter sleep, paralysis, stun.
\item \textbf{Beast}: speak with or influence animals.
\end{itemize}

\subsection*{Space \& Motion TAGS (Always +2 DV Each)}
\begin{itemize}[leftmargin=*]
\item \textbf{Leap}: jump far, blink across short space.
\item \textbf{Fold}: short-range teleport, vanish–reappear.
\item \textbf{Gate}: long distance passage, open/close path.
\item \textbf{Gravity}: crush, lift, suspend, walk skyward.
\end{itemize}

\subsection*{Creation \& Transformation TAGS (Always +2 DV Each)}
\begin{itemize}[leftmargin=*]
\item \textbf{Create}: manifest matter briefly.
\item \textbf{Summon}: call a being or construct.
\item \textbf{Transmute}: turn one thing into another.
\item \textbf{Animate}: make objects act with intent.
\end{itemize}
