\chapter{Mechanics Reference}

\section{Core Mechanic: The Art of Consequence}

All significant actions follow a three-step process:

\begin{enumerate}
\item \textbf{Approach} --- The player describes both what their character wants and how they attempt it. This defines the primary Skill and clarifies the fiction.
\item \textbf{Execution} --- Build a dice pool equal to Attribute + Skill and roll that many \dice{10}s. Each die of 6 or higher counts as a success. Each 1 rolled generates a Complication Point.
\item \textbf{Outcome} --- The GM interprets total successes against the difficulty of the task. Complication Points are then spent to weave narrative setbacks, collateral costs, or escalating danger.
\end{enumerate}

\section{The Description Ladder}

The quality of the player's description affects the resilience of their roll against complication:

\begin{description}
\item[Basic Action] Roll the pool as-is. All 1s remain as Complication Points.
\item[Detailed Action] A clear, descriptive flourish allows the player to re-roll one die showing 1.
\item[Intricate Action] A richly described, multi-sensory action allows the player to re-roll all dice showing 1, and add one positive narrative flourish to the scene if they succeed.
\end{description}

\noindent\textbf{CP Note.} Re-rolling 1s does \emph{not} erase their CP; any new 1s on the re-roll add more CP.

\section{Difficulty Ladder}

\begin{table}[htbp]
\centering
\begin{tabular}{cll}
\toprule
\textbf{DV} & \textbf{Name} & \textbf{When to Use} \\
\midrule
1 & Routine & Clear intent, modest stakes, controlled environment \\
2 & Pressured & Time pressure, mild resistance, partial information \\
3 & Hard & Hostile conditions, active opposition, precise timing \\
4+ & Extreme & Multiple constraints, high precision or secrecy \\
\bottomrule
\end{tabular}
\caption{Difficulty Ladder}
\end{table}

\section{Outcome Matrix}

Let $S$ be successes ($\geq 6$) and $C$ be Complication Points (number of 1s rolled).

\begin{description}
\item[Clean Success] ($S \geq DV$ and $C = 0$) --- Deliver the intent crisply. Offer a small positional or information edge if description was Intricate.
\item[Success \& Cost] ($S \geq DV$ and $C > 0$) --- Grant the intent; spend/bank CP to add friction (noise, time lost, resource wear, new eyes on the scene).
\item[Partial] ($0 < S < DV$) --- Progress with a fork: Get it but... (time/position/gear cost) or Leave it and... (take safety, new intel).
\item[Miss] ($S = 0$) --- No progress. Cash some CP now or bank for a coming beat. Consider offering a Devil's Bargain: succeed at $DV-1$ if you accept a named complication.
\end{description}

\section{Complication Point Spend Menu}

\subsection{Universal CP Options}

\begin{description}
\item[1 CP] Minor pressure: noise, trace, +1 Supply segment.
\item[2 CP] Moderate setback: alarm raised, lose position/cover, lesser foe or lock.
\item[3 CP] Serious trouble: reinforcements, key gear breaks, rail tick.
\item[4+ CP] Major turn: trap springs, authority arrives, scene shifts.
\end{description}

\subsection{Combat}

\begin{description}
\item[1 CP] lose footing (next defense --1d).
\item[2 CP] weapon or battlefield shifts (fireline, cave-in, cavalry arrives).
\item[3 CP] pinned, disarmed, or separated.
\end{description}

\subsection{Stealth \& Intrusion}

\begin{description}
\item[1 CP] footstep/squeak; shadow seen.
\item[2 CP] patrol changes; lock resists (extra test).
\item[3 CP] partial alarm (initiated).
\item[4 CP] full alarm and lockdown protocol.
\end{description}

\subsection{Social}

\begin{description}
\item[1 CP] rumor cost or faux pas (future --1d with this contact).
\item[2 CP] a concession is now required (gift, favor).
\item[3 CP] rival interjects with leverage.
\item[4 CP] patron turns, audience hostile; by Pillar (Examples) or oath invoked.
\end{description}

\subsection{Travel \& Survival}

\begin{description}
\item[1 CP] lose time; minor injury; weather turns.
\item[2 CP] Supply +1 segment; mount lamed.
\item[3 CP] wrong valley or blocked pass; Fatigue 1 to all.
\item[4 CP] storm, rockslide, flood---scene rewritten.
\end{description}

\subsection{Arcana \& Ritual}

\begin{description}
\item[1 CP] prickle of backlash; sensory bleed.
\item[2 CP] unintended side-effect (cold from fire, echoes draw attention).
\item[3 CP] residue anchors a foe/hex.
\item[4 CP] backlash condition or manifestation; ritual mark persists.
\end{description}

\paragraph{High-Tier CP Sinks.}
For 3–6+ CP spends that move the world (reputation cascades, faction instability, resonance, prophecy), see the stand-alone \emph{High CP Sinks} handout. A good default: at end of leg, \textbf{3 CP → tick 1 Front}.

\section{Boon Economy}

\begin{itemize}
    \item Holding cap: You can hold at most 5 Boons.
    \item Conversion: Once per session, in downtime, you may convert 2 Boons → 1 XP (max 2 XP via conversion per session).
\end{itemize}

\section{XP Economy}

\subsection{Awards (Session)}

\begin{itemize}
\item Table Attendance: +2 XP
\item Major Objective Reached: +2--4 XP
\item Discovery or Lore Unlocked: +1--2 XP
\item Hard Choice Embraced: +1--2 XP
\item Complication Spotlight (leaning into drawn Complications): +1--3 XP
\item Bond/Flag Driven Play: +1--2 XP
\item GM Curveball Award: +0--3 XP for standout creativity
\end{itemize}

\subsection{Milestones}

At the conclusion of a major story arc:
\begin{itemize}
\item +8--12 XP to all players
\item +2 XP bonus to one player for a signature moment of the arc
\end{itemize}

\subsection{Complication Dividend}
\begin{itemize}
    \item Face Card: +1 XP
    \item Ace: +2 XP
\end{itemize}

\subsection{Spending}

\begin{description}
\item[Attributes] Cost = new rating $\times$ 3. Downtime = new rating in days.
\item[Skills] Cost = new level $\times$ 2. Downtime = new level in days.
\item[On-Screen Followers] Cost = Cap$^2$. Downtime = 1--3 days to recruit and brief.
\item[Off-Screen Assets] Minor (4 XP, 1 day), Standard (8 XP, 1 week), Major (12 XP, 1 month).
\end{description}

\section{Rush Rule}
A player may skip downtime, but the GM creates a Haste clock of four segments. If the clock fills, the new ability or asset carries flaws or narrative complications.

\section{Tiers of Reputation}

\begin{description}
\item[Tier I -- Rookie] (0--40 XP): Local reputation; prestige locked.
\item[Tier II -- Seasoned] (41--90): Regional notice; prestige abilities may be unlocked.
\item[Tier III -- Veteran] (91--150): National influence; second follower slot suggested.
\item[Tier IV -- Paragon] (151--220): Movers and shakers; rivals emerge to challenge.
\item[Tier V -- Mythic] (221+): Legendary status; kingdoms and cults respond.
\end{description}
\end{chapter}
