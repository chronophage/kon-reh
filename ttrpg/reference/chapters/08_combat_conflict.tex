\chapter{Combat and Conflict}

\section{Position and Effect}

\subsection{Position States}

\begin{description}
\item[Controlled] Advantageous position
\item[Risky] Even odds
\item[Desperate] Disadvantageous position
\end{description}

\subsection{Effect Scale}

\begin{description}
\item[Limited] Minor impact
\item[Standard] Clear impact
\item[Strong] Major impact
\end{description}

\section{Combat Procedures}

\subsection{Initiative and Actions}

Players declare actions simultaneously. The GM resolves actions in order of narrative priority, typically:
\begin{enumerate}
\item Reactions/defenses
\item Movement/positioning
\item Attacks/effects
\item Consequences and follow-up
\end{enumerate}

\subsection{Defense}

When attacked, a character may defend by rolling an appropriate Attribute + Skill:
\begin{itemize}
\item Melee defense: Body + Athletics or Wits + Melee
\item Ranged defense: Wits + Athletics or Body + Survival
\item Spell defense: Spirit + Resolve or Wits + Arcana
\end{itemize}

Each 1 rolled in defense generates a Complication Point for the defender.

\subsection{Attacks}

Attacks follow the standard resolution procedure:
\begin{enumerate}
\item Declare attack type and target
\item Set Difficulty (typically opponent's defense pool or fixed DV)
\item Roll attack pool (Attribute + Skill)
\item Count successes and Complication Points (1s)
\item Apply outcome and spend Complication Points
\end{enumerate}

\section{Harm and Injury}

\subsection{Harm Tracks}

Characters track harm through narrative conditions rather than hit points:

\begin{description}
\item[Minor] (-) Superficial wounds, bruises, temporary impairments
\item[Moderate] (=) Noticeable injuries, lasting impairments, reduced effectiveness
\item[Severe] ($\approx$) Major trauma, significant debilitation, long-term consequences
\item[Critical] ($\dagger$) Life-threatening conditions, permanent damage, near-death
\end{description}

\subsection{Injury Effects}

\begin{description}
\item[Minor] No mechanical penalty; narrative limitation
\item[Moderate] --1 die to relevant actions; may generate additional CP on related rolls
\item[Severe] --2 dice to relevant actions; clear only with significant rest/treatment
\item[Critical] Character is incapacitated; requires immediate intervention to survive
\end{description}

\subsection{Recovery}

\begin{itemize}
\item Minor: Clear with rest or basic treatment
\item Moderate: Requires significant rest (days) or medical attention
\item Severe: Requires expert care and extended recovery (weeks)
\item Critical: Requires immediate expert intervention; may leave permanent consequences
\end{itemize}

\section{Stress, Harm, and Loss (GM Tools)}

\subsection{Follower Consequences}

When the GM spends 2+ Complication Points on an action involving a follower:

\begin{description}
\item[Pin] The follower is separated/boxed out; no assist next roll/scene
\item[Wound] The follower is Injured: until treated off-screen, their Cap counts as 1 lower
\item[Burn] Mark Neglected immediately (blown cover, angry creditors)
\item[Seize] Escalate to Compromised (capture, flight, betrayal) if dramatically earned
\end{description}

\subsection{PC Choice Lever}

The GM should offer the player a save: protect the follower (accept a harsher on-screen complication for the PC) or let the follower take the hit.

\section{Social Conflict}

\subsection{Persuasion and Influence}

Social conflicts use the same core mechanics but with different skills:
\begin{itemize}
\item Diplomacy for negotiation and formal discourse
\item Deception for lies, misdirection, and manipulation
\item Performance for oratory, entertainment, and emotional appeal
\item Insight for reading opponents and detecting deception
\end{itemize}

\subsection{Social Position}

\begin{description}
\item[Controlled] You have leverage, information, or social advantage
\item[Risky] Even social ground; standard interaction
\item[Desperate] You're at a disadvantage; opponent has leverage
\end{description}

\subsection{Social Consequences}

Complication Points in social conflicts might manifest as:
\begin{itemize}
\item Rumors spread that harm reputation
\item Allies become suspicious or hostile
\item Obligations or concessions must be made
\item Social standing or access is compromised
\end{itemize}

\section{Mass Combat and Warfare}

\subsection{Command and Leadership}

Characters with appropriate skills (Command, Tactics, Leadership) can direct groups in mass conflicts:

\begin{itemize}
\item Assist allies in combat rolls
\item Coordinate tactical maneuvers
\item Rally broken units
\item Influence battle momentum
\end{itemize}

\subsection{Initiative Action}
\begin{itemize}
    \item Cost choice: An Initiative Action by a follower costs either Exposure +1 or Harm 1.
    \item Cadence: By default, the crew has 1 Follower Initiative window per scene.
\end{itemize}

\subsection{Warfare Clocks}

Large-scale conflicts often use clocks to track:
\begin{itemize}
\item Army morale and cohesion
\item Supply and logistics
\item Strategic positioning
\item Political support and reinforcement
\end{itemize}

\subsection{Command Complications}

When directing mass forces, Complication Points might represent:
\begin{itemize}
\item Units becoming disorganized or scattered
\item Communication breakdowns
\item Supply line disruptions
\item Political interference or betrayal
\end{itemize}

\section{Environmental Hazards}

\subsection{Natural Hazards}

Environmental dangers follow the standard resolution system:
\begin{itemize}
\item Set DV based on hazard severity
\item Players roll appropriate skills to avoid or mitigate
\item Complications represent exposure or partial success
\end{itemize}

\subsection{Hazard Clocks}

Persistent environmental threats can be tracked with clocks:
\begin{description}
\item[Fire] Spreading flames, smoke inhalation, structural damage
\item[Flood] Rising water, current strength, debris hazards
\item[Storm] Wind force, precipitation, visibility, structural stress
\item[Earthquake] Ground shaking, structural collapse, aftershocks
\end{description}

\subsection{Environmental Complications}

Complication Points from environmental hazards might cause:
\begin{itemize}
\item Gear damage or loss
\item Terrain changes that complicate movement
\item Exposure leading to Fatigue or Injury
\item Separation of party members
\end{itemize}
\end{chapter}
