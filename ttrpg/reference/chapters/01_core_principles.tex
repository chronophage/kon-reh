\chapter{Core Principles}

\section{The Central Question}

At its heart, Fate's Edge asks:
\begin{quote}
\textbf{What are you willing to risk, and what are you willing to pay, to reshape the world around you?}
\end{quote}

This question is both philosophical and mechanical. Players gamble with fate every time they act, and the consequences---good or ill---become the foundation of their legend.

\section{Key Concepts}

\subsection{Narrative Time}

Time in Fate's Edge is measured by story weight, not by clocks. Actions are framed in four narrative scales:

\begin{description}
\item[A Moment] A heartbeat, a glance, a single strike or word.
\item[Some Time] A few minutes, enough for a skirmish, a careful lockpick, or a short negotiation.
\item[Significant Time] Hours, long enough to travel between locations, work a ritual, or endure a siege.
\item[Days] Large-scale endeavors: marches across a countryside, training a cadre, or recovering from wounds.
\end{description}

\subsection{Complication Points}

Whenever a player rolls dice, each result of 1 generates a Complication Point (CP). These are not mere penalties---they are narrative fuel. The GM spends them to introduce twists:

\begin{itemize}
\item Escalation --- drawing more enemies, raising the stakes.
\item Exhaustion --- draining time, resources, or positioning.
\item Exposure --- revealing hidden actions, alerting foes.
\item Collateral --- harm or danger spilling over onto allies, innocents, or surroundings.
\end{itemize}

\subsection{Affinity}

Races and cultures in Fate's Edge do not define characters through numbers alone. Instead, each provides an Affinity: a narrative edge or metaphysical bond. Affinities make certain Arts, skills, or actions more reliable, weaving identity into mechanics.

\subsection{Prestige Abilities}

Prestige Abilities are high-level talents unlocked by mastering cultural arts or philosophies. They are narrative milestones as much as mechanical ones.

\subsection{On-Screen vs. Off-Screen}

Fate's Edge distinguishes between resources you see at the table and those that shape the world in the background:

\begin{description}
\item[On-Screen Resources] are companions, hirelings, or allies who stand beside you in danger. They add dice pools and flavor, but they can falter, be taken, or die.
\item[Off-Screen Resources] are taverns, estates, titles, or networks of informants. They never swing a blade in combat, but they shape the story between sessions, turning XP into narrative leverage.
\end{description}

\section{Design Philosophy}

\subsection{Core Principles}
\begin{enumerate}
    \item \textbf{Narrative Primacy}: Mechanics serve story, not replace it.
    \item \textbf{Risk as Drama}: Every roll carries potential for triumph + complication.
    \item \textbf{Meaningful Growth}: XP investment creates lasting character/world change.
    \item \textbf{Consequence Weight}: Choices ripple outward, nothing is free.
\end{enumerate}

\subsection{Mechanical Constraints}
\begin{itemize}
    \item \textbf{ASSIST MAX}: +3 dice total per roll, regardless of helpers. Exception: The "Exceptional Coordination" Talent allows one follower to provide +4 assist dice.
    \item \textbf{BOON MAX}: 5 total, 2→1 XP conversion once/session (max 2 XP via conversion per session).
    \item \textbf{INITIATIVE}: 1 Follower Action per scene crew-wide.
    \item \textbf{OVER-STACK}: 2+ structural advantages = start rails +1 OR GM banks +1 CP.
    \item \textbf{POSITION}: Controlled | Risky | Desperate (affects success/failure texture).
\end{itemize}
\end{chapter}
