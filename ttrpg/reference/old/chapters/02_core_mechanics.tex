\chapter{Mechanics Reference}

\section{Core Mechanic: The Art of Consequence}\index{Core Mechanic}

All significant actions follow a three-step process:

\begin{enumerate}
\item \textbf{Approach} --- The player describes both what their character wants and how they attempt it. This defines the primary Skill and clarifies the fiction.\index{Core Mechanic!Approach}
\item \textbf{Execution} --- Build a dice pool equal to Attribute + Skill and roll that many \dice{10}s. Each die of 6 or higher counts as a success. Each 1 rolled generates a Story Beat.\index{Core Mechanic!Execution}
\item \textbf{Outcome} --- The GM interprets total successes against the difficulty of the task. Story Beats are then spent to weave narrative setbacks, collateral costs, or escalating danger.\index{Core Mechanic!Outcome}
\end{enumerate}

\section{The Description Ladder}\index{Description Ladder}

The quality of the player's description affects the resilience of their roll against complication:

\begin{description}
\item[Basic Action] Roll the pool as-is. All 1s remain as Story Beats.\index{Description Ladder!Basic Action}
\item[Detailed Action] A clear, descriptive flourish allows the player to re-roll one die showing 1.\index{Description Ladder!Detailed Action}
\item[Intricate Action] A richly described, multi-sensory action allows the player to re-roll all dice showing 1, and add one positive narrative flourish to the scene if they succeed.\index{Description Ladder!Intricate Action}
\end{description}

\noindent\textbf{SB Note.} Re-rolling 1s does \emph{not} erase their SB; any new 1s on the re-roll add more SB.\index{Story Beats!Re-rolling}

\section{Difficulty Ladder}\index{Difficulty}

\begin{table}[htbp]
\centering
\begin{tabular}{cll}
\toprule
\textbf{DV} & \textbf{Name} & \textbf{When to Use} \\
\midrule
2 & Routine & Clear intent, modest stakes, controlled environment \\
3 & Pressured & Time pressure, mild resistance, partial information \\
4 & Hard & Hostile conditions, active opposition, precise timing \\
5+ & Extreme & Multiple constraints, high precision or secrecy \\
\bottomrule
\end{tabular}
\caption{Difficulty Ladder}
\end{table}

\section{Outcome Matrix}\index{Outcome Matrix}

Let $S$ be successes ($\geq 6$) and $C$ be Story Beats (number of 1s rolled).

\begin{description}
\item[Clean Success] ($S \geq DV$ and $C = 0$) --- Deliver the intent crisply. Offer a small positional or information edge if description was Intricate.\index{Outcome Matrix!Clean Success}
\item[Success \& Cost] ($S \geq DV$ and $C > 0$) --- Grant the intent; spend/bank SB to add friction (noise, time lost, resource wear, new eyes on the scene).\index{Outcome Matrix!Success \& Cost}
\item[Partial] ($0 < S < DV$) --- Progress with a fork: Get it but... (time/position/gear cost) or Leave it and... (take safety, new intel).\index{Outcome Matrix!Partial}
\item[Miss] ($S = 0$) --- No progress. Cash some SB now or bank for a coming beat.\index{Outcome Matrix!Miss}
\end{description}

\section{Story Beat Spend Menu}\index{Story Beats!Spend Menu}

\subsection{Universal SB Options}\index{Story Beats!Universal Options}

\begin{description}
\item[1 SB] Minor pressure: noise, trace, +1 Supply segment.\index{Story Beats!1 SB}
\item[2 SB] Moderate setback: alarm raised, lose position/cover, lesser foe or lock.\index{Story Beats!2 SB}
\item[3 SB] Serious trouble: reinforcements, key gear breaks, rail tick.\index{Story Beats!3 SB}
\item[4+ SB] Major turn: trap springs, authority arrives, scene shifts.\index{Story Beats!4+ SB}
\end{description}

\subsection{Combat}\index{Story Beats!Combat Options}

\begin{description}
\item[1 SB] lose footing (next defense --1d).\index{Story Beats!Combat!1 SB}
\item[2 SB] weapon or battlefield shifts (fireline, cave-in, cavalry arrives).\index{Story Beats!Combat!2 SB}
\item[3 SB] pinned, disarmed, or separated.\index{Story Beats!Combat!3 SB}
\end{description}

\subsection{Stealth \& Intrusion}\index{Story Beats!Stealth Options}

\begin{description}
\item[1 SB] footstep/squeak; shadow seen.\index{Story Beats!Stealth!1 SB}
\item[2 SB] patrol changes; lock resists (extra test).\index{Story Beats!Stealth!2 SB}
\item[3 SB] partial alarm (initiated).\index{Story Beats!Stealth!3 SB}
\item[4 SB] full alarm and lockdown protocol.\index{Story Beats!Stealth!4 SB}
\end{description}

\subsection{Social}\index{Story Beats!Social Options}

\begin{description}
\item[1 SB] rumor cost or faux pas (future --1d with this contact).\index{Story Beats!Social!1 SB}
\item[2 SB] a concession is now required (gift, favor).\index{Story Beats!Social!2 SB}
\item[3 SB] rival interjects with leverage.\index{Story Beats!Social!3 SB}
\item[4 SB] patron turns, audience hostile; by Pillar (Examples) or oath invoked.\index{Story Beats!Social!4 SB}
\end{description}

\subsection{Travel \& Survival}\index{Story Beats!Travel Options}

\begin{description}
\item[1 SB] lose time; minor injury; weather turns.\index{Story Beats!Travel!1 SB}
\item[2 SB] Supply +1 segment; mount lamed.\index{Story Beats!Travel!2 SB}
\item[3 SB] wrong valley or blocked pass; Fatigue 1 to all.\index{Story Beats!Travel!3 SB}
\item[4 SB] storm, rockslide, flood---scene rewritten.\index{Story Beats!Travel!4 SB}
\end{description}

\subsection{Arcana \& Ritual}\index{Story Beats!Arcana Options}

\begin{description}
\item[1 SB] prickle of backlash; sensory bleed.\index{Story Beats!Arcana!1 SB}
\item[2 SB] unintended side-effect (cold from fire, echoes draw attention).\index{Story Beats!Arcana!2 SB}
\item[3 SB] residue anchors a foe/hex.\index{Story Beats!Arcana!3 SB}
\item[4 SB] backlash condition or manifestation; ritual mark persists.\index{Story Beats!Arcana!4 SB}
\end{description}

\paragraph{High-Tier SB Sinks.}\index{Story Beats!High-Tier Sinks}
For 3–6+ SB spends that move the world (reputation cascades, faction instability, resonance, prophecy), see the stand-alone \emph{High SB Sinks} handout. A good default: at end of leg, \textbf{3 SB → tick 1 Front}.

\section{Fail Forward: Every Roll Matters}\index{Fail Forward}\index{Boons}

When you \textbf{MISS} on a \emph{significant action}, you gain \textbf{1 Boon}. Boons can be spent immediately for re-rolls, Asset activations, Rites, and other abilities. You can hold up to \textbf{5} Boons.

\subsection{Significant Action (Meaningful Failure)}
A miss only awards a Boon if \textbf{all three} are true:
\begin{enumerate}
  \item \textbf{Procedure followed:} intent and approach declared; DV set; roll resolved.
  \item \textbf{Stakes stated:} what changes on success; what bites on failure.
  \item \textbf{Consequence lands now:} the GM spends or banks SB, applies a condition, or advances a thread.
\end{enumerate}
\noindent Rolling a \textbf{1} always creates SB for the GM. Re-rolling \textbf{1}s never removes SB already generated.

\subsection{No Boon For}
Rehearsal or null-risk probes, and repeated identical attempts in the same scene without a new approach, position, or stakes.

\subsection{Other Ways to Gain Boons}
Strong bond-driven play and scene prompts can also award Boons at the GM's discretion. Boons remain capped by the limits below.

\subsection{Boon Carryover (Scene-Based)}\index{Boon Carryover}
\textbf{Hold Cap:} You can hold up to \textbf{5} Boons.

\textbf{Carryover Limit:} At the \emph{end of each scene}, reduce your held Boons to a maximum of \textbf{2}. Excess Boons are lost.

\textbf{Spend As You Earn:} You may spend Boons at any time during the scene (re-rolls, Asset activations, Rites, abilities, etc.).

\textbf{Multi-Phase Set Pieces:} If the GM declares a multi-phase scene (e.g., chase $\rightarrow$ duel), trim to 2 only when the entire set piece ends.

\subsection{Rite \& Asset Notes}
High-Power Rites that require \textbf{2 Boons} remain viable—you can start a scene at 2 and must earn more in-scene to chain further Invokes. On-screen Asset activations still cost 1 Boon as normal.

\subsection{Anti-Fishing Dials}\index{Anti-Fishing}
These optional limits help keep flow healthy:
\begin{itemize}
  \item \textbf{Once/Scene (Failures):} At most \textbf{2 Boons from failures per character per scene}. Further misses still generate SB but no Boon.
  \item \textbf{Repetition Rule:} Same approach \emph{and} same stakes in the same scene cannot award another Boon.
  \item \textbf{Position Gate:} \textbf{Controlled} tests with trivial fallout do not award Boons; they're for information or positioning, not currency.
\end{itemize}

\subsection{Examples}
\begin{itemize}
  \item \textbf{Boon awarded:} Picking a lock under watch (\emph{Risky}, DV 3). Stakes set: success opens; miss triggers the alarm. The roll \textbf{MISS}es; the GM spends 2 SB to start ``Guards Incoming.'' The player gains \textbf{1 Boon}.
  \item \textbf{No Boon:} Tapping flagstones ``just in case'' (Controlled, no stated stakes). Info only; no SB spent/banked. No Boon.
  \item \textbf{Carryover:} End of scene, a character holds 4 Boons. They trim to \textbf{2} for the next scene. During the next scene, they earn and spend Boons freely, never exceeding the \textbf{5} hold cap in-scene; trim back to 2 when that scene ends.
\end{itemize}

\section{Boon Economy}\index{Boons}

\begin{itemize}
    \item Holding cap: You can hold at most 5 Boons.\index{Boons!Holding Cap}
    \item Conversion: Once per session, in downtime, you may convert 2 Boons → 1 XP (max 2 XP via conversion per session).\index{Boons!Conversion}
    \item Scene limit: Maximum 2 Boons from failures per character per scene.\index{Boons!Scene Limit}
    \item End of scene: Reduce held Boons to maximum 2.\index{Boons!Carryover Limit}
\end{itemize}

\section{Asset Activations}\index{Assets!Activations}

Players can activate their Assets through multiple methods:

\begin{itemize}
    \item \textbf{Free Off-Screen Effects}: Use each Asset's listed Off-Screen effect once per session for free.\index{Assets!Free Activation}
    \item \textbf{XP Activation}: Spend 2 XP to activate an off-screen Asset effect outside your normal session allowance.\index{Assets!XP Activation}
    \item \textbf{Boon Activation}: Spend 1 Boon to dramatically reshape the current scene through Asset intervention.\index{Assets!Boon Activation}
    \item \textbf{Plausibility Test}: The Asset must have scope and reach for the intended effect.\index{Assets!Plausibility Test}
\end{itemize}

\section{XP Economy}\index{Experience Points}

\subsection{Awards (Session)}\index{Experience Points!Session Awards}

\begin{itemize}
\item Table Attendance: +2 XP\index{Experience Points!Attendance}
\item Major Objective Reached: +2--4 XP\index{Experience Points!Objectives}
\item Discovery or Lore Unlocked: +1--2 XP\index{Experience Points!Discovery}
\item Hard Choice Embraced: +1--2 XP\index{Experience Points!Hard Choices}
\item Complication Spotlight (leaning into drawn Complications): +1--3 XP\index{Experience Points!Complications}
\item Bond/Flag Driven Play: +1--2 XP\index{Experience Points!Bonds}
\item GM Curveball Award: +0--3 XP for standout creativity\index{Experience Points!Creativity}
\end{itemize}

\subsection{Milestones}\index{Experience Points!Milestones}

At the conclusion of a major story arc:
\begin{itemize}
\item +8--12 XP to all players\index{Experience Points!Arc Completion}
\item +2 XP bonus to one player for a signature moment of the arc\index{Experience Points!Signature Moments}
\end{itemize}

\subsection{Complication Dividend}\index{Experience Points!Complication Dividend}
\begin{itemize}
    \item Face Card: +1 XP\index{Experience Points!Face Cards}
    \item Ace: +2 XP\index{Experience Points!Aces}
\end{itemize}

\subsection{Spending}\index{Experience Points!Spending}

\begin{description}
\item[Attributes] Cost = new rating $\times$ 3. Downtime = new rating in days.\index{Experience Points!Attributes}
\item[Skills] Cost = new level $\times$ 2. Downtime = new level in days.\index{Experience Points!Skills}
\item[On-Screen Followers] Cost = Cap$^2$. Downtime = 1--3 days to recruit and brief.\index{Experience Points!Followers}
\item[Off-Screen Assets] Minor (4 XP, 1 day), Standard (8 XP, 1 week), Major (12 XP, 1 month).\index{Experience Points!Assets}
\end{description}

\section{Rush Rule}\index{Advancement!Rush Rule}
A player may skip downtime, but the GM creates a Haste clock of four segments. If the clock fills, the new ability or asset carries flaws or narrative complications.

\section{Tiers of Reputation}\index{Reputation Tiers}

\begin{description}
\item[Tier I -- Rookie] (0--40 XP): Local reputation; prestige locked.\index{Reputation Tiers!Rookie}
\item[Tier II -- Seasoned] (41--90): Regional notice; prestige abilities may be unlocked.\index{Reputation Tiers!Seasoned}
\item[Tier III -- Veteran] (91--150): National influence; second follower slot suggested.\index{Reputation Tiers!Veteran}
\item[Tier IV -- Paragon] (151--220): Movers and shakers; rivals emerge to challenge.\index{Reputation Tiers!Paragon}
\item[Tier V -- Mythic] (221+): Legendary status; kingdoms and cults respond.\index{Reputation Tiers!Mythic}
\end{description}
\end{chapter}

