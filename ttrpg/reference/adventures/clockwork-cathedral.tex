\documentclass{article}
\usepackage[utf8]{inputenc}
\usepackage{geometry}
\geometry{a4paper, margin=1in}
\usepackage{graphicx}
\usepackage{fancyhdr}
\usepackage{titling}
\usepackage{lipsum}
\usepackage{xcolor}
\usepackage{tcolorbox}
\usepackage{array}
\usepackage{longtable}
\usepackage{enumitem}
\usepackage{framed}
\usepackage{multicol}

\definecolor{adventureblue}{RGB}{40, 80, 120}
\definecolor{mechanicgreen}{RGB}{40, 120, 80}
\definecolor{storyred}{RGB}{120, 40, 40}

\newtcolorbox{adventurebox}{
    colback=adventureblue!10,
    colframe=adventureblue,
    boxrule=0.5pt,
    arc=3mm,
    title=\textbf{Adventure Element}
}

\newtcolorbox{mechanicbox}{
    colback=mechanicgreen!10,
    colframe=mechanicgreen,
    boxrule=0.5pt,
    arc=3mm,
    title=\textbf{Mechanics Integration}
}

\newtcolorbox{storybox}{
    colback=storyred!10,
    colframe=storyred,
    boxrule=0.5pt,
    arc=3mm,
    title=\textbf{Story Beat}
}

\pagestyle{fancy}
\fancyhf{}
\rhead{The Clockwork Cathedral}
\lhead{Fate's Edge Adventure}
\rfoot{\thepage}

\title{The Clockwork Cathedral}
\author{Fate's Edge Adventure for D\&D 5e Fans}
\date{}

\begin{document}

\maketitle

\begin{center}
\textit{"The gears of fate turn in both directions. Sometimes they build, sometimes they grind to dust."}
\end{center}

\begin{abstract}
The Clockwork Cathedral of Aeler, abandoned for centuries, has begun to activate its clockwork heart. But something is wrong—the gears turn backward, and the sacred cogs are grinding to dust. The Aeler people are desperate for help before the temple collapses and takes the entire mountain with it. This adventure is designed for Fate's Edge and specifically targets fans of D\&D 5e, preserving tactical combat while emphasizing narrative consequences. The cathedral is alive, its will responding to the players' actions through a "Clock" that tracks its mood from Calm to Fracturing. Players must navigate living architecture, repair sacred clockwork using Aeler ritual magic, and ultimately confront the corrupted heart of the cathedral before it shatters the mountain.
\end{abstract}

\section{Adventure Overview}

\begin{adventurebox}
\textbf{Setting:} The Clockwork Cathedral of Aeler, deep in the Iron Marches \\
\textbf{Tier:} I--II (can be adjusted for higher tiers) \\
\textbf{Themes:} Divine magic, tactical combat, high-stakes temple adventure \\
\textbf{Player Count:} 3--5 \\
\textbf{Estimated Play Time:} 3--4 sessions
\end{adventurebox}

The Clockwork Cathedral is a temple built by the Aeler, a people who believe time itself is sacred. Centuries ago, the Aeler abandoned the cathedral when its clockwork heart began to fail. Now, after centuries of silence, the heart has begun to wind back up—but it's turning backward, grinding the sacred cogs to dust. If the cathedral collapses, it will take the entire mountain with it, destroying the region.

The adventure is structured around three major beats, each with its own clocks and challenges. The cathedral itself is a living entity that responds to the players' actions, with a "Cathedral Mood Clock" tracking its state from Calm to Fracturing. As the mood worsens, the cathedral's architecture becomes more hostile and unpredictable.

\begin{storybox}
\textbf{The Hook:} The Aeler have sent for help, but few have answered. The PCs arrive to find the cathedral's outer doors grinding open and closing erratically, as if breathing. The Aeler elders beg for assistance, explaining that the cathedral's heart is failing and will collapse the mountain if not repaired. They cannot enter themselves—the cathedral rejects all Aeler, recognizing only those who can repair its sacred clockwork.
\end{storybox}

\section{Cathedral Mood Clock}

The Cathedral Mood Clock tracks the cathedral's state, affecting Position and Difficulty Values throughout the adventure.

\begin{center}
\begin{longtable}{|c|c|c|c|}
\hline
\textbf{Calm} & \textbf{Tense} & \textbf{Stressed} & \textbf{Fracturing} \\
\hline
(0--2) & (3--5) & (6--8) & (9--10) \\
\hline
\end{longtable}
\end{center}

\begin{itemize}
    \item \textbf{Calm (0--2 segments):} The cathedral's architecture is stable. Position is typically Dominant or Controlled for repairs. DV for clockwork-related checks is -1.
    \item \textbf{Tense (3--5 segments):} The cathedral shows signs of stress. Corridors shift slowly. Position is typically Controlled. DV for clockwork-related checks is normal.
    \item \textbf{Stressed (6--8 segments):} The cathedral is actively shifting. Corridors change position each scene. Position is typically Controlled or Desperate. DV for clockwork-related checks is +1.
    \item \textbf{Fracturing (9--10 segments):} The cathedral is falling apart. Corridors shift during scenes. Position is typically Desperate. DV for clockwork-related checks is +2.
\end{itemize}

\textbf{How the Mood Changes:}
\begin{itemize}
    \item \textbf{Worsens by 1:} On a miss when repairing clockwork, when the players cause damage to the cathedral, when the players ignore a "living gear" request.
    \item \textbf{Improves by 1:} On a clean success when repairing clockwork, when the players successfully interact with a "living gear," when the players perform a ritual to stabilize the cathedral.
\end{itemize}

\section{Adventure Beats}

\subsection{Beat 1: The Broken Gear}

\begin{mechanicbox}
\textbf{Cathedral Mood Clock:} Begins at Tense (3 segments) \\
\textbf{Primary Clock:} Broken Gear [4] \\
\textbf{Key Mechanics:} Position, Story Beats, Living Gears
\end{mechanicbox}

The first challenge is a massive gear in the cathedral's entrance hall that has broken free from its axle and is grinding against the stone floor, creating sparks and a deafening noise. This is preventing the cathedral from stabilizing.

\subsubsection{The Scene}

The entrance hall is a vast, domed chamber. In the center, a gear the size of a wagon wheel spins erratically, grinding against the floor. Stone dust fills the air, and the walls are etched with intricate clockwork patterns that seem to shift when not directly observed.

The broken gear is not just a mechanical problem—it's alive, and it's in pain. The cathedral is trying to expel it, but the gear is too large to pass through the corridors. The Aeler elders explain that only someone who can "speak" to the gear can return it to its proper place.

\subsubsection{The Living Gear}

The gear is a \textbf{Living Gear}—a sentient clockwork component that has become disconnected from the cathedral. It communicates through vibrations, clicks, and the shifting of its teeth.

\begin{itemize}
    \item \textbf{Approach:} The players must communicate with the gear using the \textbf{Cathedral Tongue} (a dialect of the Aeler language that describes mechanical principles as divine concepts).
    \item \textbf{Position:} Controlled (the gear is hostile, but the players have time to work with it)
    \item \textbf{DV:} 3 (pressured by the noise and danger)
    \item \textbf{Skill:} Lore (to understand the Cathedral Tongue) or Craft (Tinker) (to understand the mechanical aspect)
    \item \textbf{Effect:} Standard (if successful, the gear calms and reveals how to repair it)
\end{itemize}

If the players succeed, the gear communicates its pain: a shard of broken crystal is embedded in its axle. The players must remove the shard before they can reposition the gear.

\textbf{Critical Success (10+ on a die):} The gear not only calms but also reveals a hidden compartment containing a \textbf{Cathedral Key} (a minor asset that can unlock any clockwork mechanism in the cathedral).

\subsubsection{Removing the Crystal Shard}

\begin{itemize}
    \item \textbf{Position:} Desperate (working near a grinding, spinning gear)
    \item \textbf{DV:} 4 (hard; the shard is fragile and the gear is unstable)
    \item \textbf{Skill:} Craft (Tinker) or Body (for strength)
    \item \textbf{Effect:} Limited (removing the shard without damaging the gear)
\end{itemize}

\begin{storybox}
\textbf{Story Beat:} If the players damage the gear while removing the shard (a miss), the Cathedral Mood worsens by 2. The gear's distress causes a section of the ceiling to collapse, requiring a quick Athletics check to avoid being hit (DV 3, Desperate/Standard).
\end{storybox}

\subsubsection{Repairing the Gear}

\begin{itemize}
    \item \textbf{Position:} Controlled (with the shard removed, the gear is calmer)
    \item \textbf{DV:} 3 (pressured; the gear must be precisely aligned)
    \item \textbf{Skill:} Craft (Tinker) or Lore
    \item \textbf{Effect:} Standard (returning the gear to its proper position)
\end{itemize}

\begin{storybox}
\textbf{Story Beat:} On a clean success, the gear settles into place with a deep chime, and the Cathedral Mood improves by 1. The walls of the entrance hall shift, revealing a previously hidden corridor.
\end{storybox}

\subsection{Beat 2: The Living Walls}

\begin{mechanicbox}
\textbf{Cathedral Mood Clock:} Depends on previous actions (starts at Tense or Stressed) \\
\textbf{Primary Clock:} Shifting Corridors [6] \\
\textbf{Key Mechanics:} Position Shifts, Clocks, Tactical Movement
\end{mechanicbox}

After repairing the broken gear, the cathedral's architecture begins to shift as the players move deeper inside. Corridors change position, stairs appear and disappear, and rooms rotate on hidden axes. The cathedral is testing the players to see if they are worthy of approaching the heart.

\subsubsection{The Shifting Corridors}

The cathedral's interior is a maze of shifting corridors. The players must navigate through a series of interconnected rooms to reach the inner sanctum. Each time they enter a new area, the GM rolls secretly to see if the corridors shift.

\begin{itemize}
    \item \textbf{DV:} 3 (pressured; the corridors shift unpredictably)
    \item \textbf{Skill:} Lore (to understand the cathedral's patterns) or Survival (to navigate by the shifting walls)
    \item \textbf{Effect:} Standard (finding the correct path)
\end{itemize}

\begin{storybox}
\textbf{Story Beat:} If the players fail, the corridors shift violently. The Cathedral Mood worsens by 1, and the GM spends 1 SB to create a complication (e.g., a dead end that requires backtracking, a corridor that narrows dangerously, or a section of the floor that drops away).
\end{storybox}

\subsubsection{The Guardian Gears}

In one of the shifting corridors, the players encounter a set of \textbf{Guardian Gears}—smaller, mobile clockwork constructs that serve as the cathedral's protectors. They are not hostile unless the players threaten the cathedral.

\begin{itemize}
    \item \textbf{Position:} Controlled (the Guardians are neutral)
    \item \textbf{DV:} 3 (pressured; the Guardians are wary)
    \item \textbf{Skill:} Presence (to communicate) or Lore (to understand their purpose)
    \item \textbf{Effect:} Standard (gaining the Guardians' trust)
\end{itemize}

If the players successfully communicate with the Guardians, they will reveal a secret path through the shifting corridors, bypassing the need for navigation checks. If the players attack the Guardians, they will defend themselves and the Cathedral Mood worsens by 2.

\subsubsection{The Shifting Corridors Clock}

The Shifting Corridors Clock [6] tracks the players' progress through the maze:

\begin{itemize}
    \item \textbf{+1 per successful navigation check}
    \item \textbf{+2 per failed navigation check (GM spends 1 SB)}
    \item \textbf{+0 if using the Guardian Gears' path}
\end{itemize}

When the clock fills, the players have successfully navigated the shifting corridors and reached the threshold of the inner sanctum.

\begin{storybox}
\textbf{Story Beat:} When the Shifting Corridors Clock fills, the walls settle into place, and the Cathedral Mood improves by 1 as the cathedral recognizes the players' worthiness. However, the inner sanctum door is sealed by a massive lock that requires a \textbf{Cathedral Key} to open (the one found in Beat 1, or another that must be discovered).
\end{storybox}

\subsection{Beat 3: The Heart of Time}

\begin{mechanicbox}
\textbf{Cathedral Mood Clock:} Depends on previous actions (starts at Tense or Stressed) \\
\textbf{Primary Clock:} Heart Corruption [8] \\
\textbf{Key Mechanics:} Runekeeper Mechanics, Position, Tactical Clocks
\end{mechanicbox}

The inner sanctum contains the cathedral's heart—a massive, intricate clockwork mechanism suspended in the center of a vast chamber. But something is wrong: the heart is turning backward, and black, oily tendrils are spreading across its gears, corrupting them.

\subsubsection{The Corrupted Heart}

The Heart of Time is the cathedral's core. It has been corrupted by a \textbf{Time Devourer}, a parasitic entity that feeds on temporal energy. The Time Devourer has taken root in the heart and is accelerating the cathedral's collapse.

\begin{itemize}
    \item \textbf{Position:} Desperate (the heart is failing and the Time Devourer is actively hostile)
    \item \textbf{DV:} 4 (hard; the heart is unstable)
    \item \textbf{Skill:} Lore (to understand the corruption) or Arcana (if a Caster) or Runekeeper (if a Runekeeper)
    \item \textbf{Effect:} Standard (assessing the corruption)
\end{itemize}

\begin{storybox}
\textbf{Story Beat:} On a success, the players identify the Time Devourer and learn that it must be removed before the heart can be repaired. On a partial success, they understand the corruption but trigger a defensive mechanism—the Cathedral Mood worsens by 1 and the GM spends 1 SB to create a complication (e.g., a section of the heart breaks off, or the Time Devourer launches an attack).
\end{storybox}

\subsubsection{The Time Devourer}

The Time Devourer is a spectral entity that manifests as a swirling vortex of black oil and broken clockwork. It attacks by manipulating time around the players.

\begin{center}
\textbf{Time Devourer}
\end{center}

\begin{itemize}
    \item \textbf{Position:} Desperate (the Devourer is actively attacking)
    \item \textbf{DV:} 4 (hard; the Devourer is powerful)
    \item \textbf{Skills:} Spirit (4), Lore (3), Body (2)
    \item \textbf{Harm:} 2 (corrosive time effects)
    \item \textbf{Special:} \textbf{Time Dilation}—each time it is attacked, the Devourer can force the attacker to re-roll one die (desperate effect)
\end{itemize}

The Time Devourer has a \textbf{Time Devourer Health Clock} [4]:

\begin{itemize}
    \item \textbf{-1 per successful attack}
    \item \textbf{-2 if using a Cathedral Key to disrupt its temporal anchor}
    \item \textbf{+1 per failed attack (the Devourer feeds on the players' energy)}
\end{itemize}

\subsubsection{Repairing the Heart}

While fighting the Time Devourer, the players can attempt to repair the heart. This requires:

\begin{itemize}
    \item \textbf{Position:} Controlled or Desperate (depending on the fight)
    \item \textbf{DV:} 4 (hard; the heart is unstable)
    \item \textbf{Skill:} Craft (Tinker) or Lore or Runekeeper (if a Runekeeper)
    \item \textbf{Effect:} Great (if successful, the heart begins to turn forward again)
\end{itemize}

\begin{storybox}
\textbf{Story Beat:} Each time the players successfully repair the heart, the Cathedral Mood improves by 1. On a miss, the Cathedral Mood worsens by 1 and the GM spends 2 SB to create a major complication (e.g., a gear shatters, requiring a replacement, or the Time Devourer gains strength).
\end{storybox}

\subsubsection{The Heart Corruption Clock}

The Heart Corruption Clock [8] tracks the state of the heart:

\begin{itemize}
    \item \textbf{-1 per successful repair}
    \item \textbf{+1 per failed repair or Time Devourer attack}
    \item \textbf{-2 if the Time Devourer is defeated}
\end{itemize}

When the clock reaches 0, the heart is fully repaired, and the cathedral stabilizes. When it reaches 8, the cathedral collapses, taking the mountain with it.

\section{NPCs}

\subsection{The Aeler Elders}

\begin{itemize}
    \item \textbf{Description:} The Aeler are a small, elderly people with silver hair and clockwork amulets. They are desperate to save the cathedral but cannot enter it themselves.
    \item \textbf{Role:} They provide the hook and guide the players to the cathedral's entrance.
    \item \textbf{Secret:} They know the cathedral rejected them because they failed to properly maintain its sacred clockwork, but they fear the truth will make the players abandon the mission.
\end{itemize}

\subsection{The Time Devourer}

\begin{itemize}
    \item \textbf{Description:} A swirling vortex of black oil and broken clockwork, with no fixed shape. It emits a low, grinding hum.
    \item \textbf{Role:} The primary antagonist of the final beat. It feeds on the cathedral's temporal energy and seeks to consume it entirely.
    \item \textbf{Secret:} It is not inherently evil—it is a natural predator of temporal energy, drawn to the cathedral's failing heart. It can be reasoned with or banished, but not destroyed.
\end{itemize}

\section{Adventure Conclusion}

The adventure has three possible conclusions, depending on the players' actions:

\subsection{Success: The Cathedral Stabilized}

If the players repair the heart and defeat the Time Devourer, the cathedral stabilizes. The Aeler return to their holy site, and the cathedral's clockwork heart begins to turn forward again. The Aeler are eternally grateful and offer the players:

\begin{itemize}
    \item A \textbf{Cathedral Blessing} (a one-time +2 to any roll)
    \item A \textbf{Cathedral Key} (a minor asset that can unlock any clockwork mechanism in the cathedral)
    \item A \textbf{Rune of Time} (a minor talent that allows the player to re-roll one die per session)
\end{itemize}

The Cathedral Mood Clock resets to Calm, and the cathedral becomes a safe haven for the players in future adventures.

\subsection{Partial Success: The Heart Preserved, But Corrupted}

If the players defeat the Time Devourer but fail to fully repair the heart, the cathedral stabilizes but with a lingering corruption. The Aeler can return, but the cathedral's clockwork is now imperfect. The players receive:

\begin{itemize}
    \item A \textbf{Cathedral Blessing} (a one-time +2 to any roll)
    \item A \textbf{Corrupted Gear} (a minor asset that can be used to manipulate time, but with a risk of worsening the Cathedral Mood)
\end{itemize}

The Cathedral Mood Clock resets to Tense, and the cathedral remains a potential source of danger.

\subsection{Failure: The Cathedral Collapses}

If the Cathedral Mood Clock reaches Fracturing and the Heart Corruption Clock fills, the cathedral collapses. The players must make a final Survival check (DV 5) to escape the collapsing mountain. If they succeed, they escape with their lives but the Aeler's holy site is lost forever. If they fail, they are trapped in the ruins, facing certain death unless they can find a way out.

\section{Fate's Edge Mechanics Integration}

\begin{mechanicbox}
\textbf{Runekeeper Integration:} The adventure is designed to showcase the Runekeeper path. Only a Runekeeper of the Clockwork Monad can interface with the central mechanism of the cathedral, but they must first earn the cathedral's trust through ritual.
\end{mechanicbox}

\subsection{Runekeeper of the Clockwork Monad}

A Runekeeper of the Clockwork Monad has special abilities related to the cathedral:

\begin{itemize}
    \item \textbf{Cathedral Tongue:} Can understand and speak to the cathedral's living gears.
    \item \textbf{Clockwork Ritual:} Can perform a ritual to stabilize the cathedral (DV 3, Controlled/Standard), improving the Cathedral Mood by 1.
    \item \textbf{Heart Interface:} Can directly interact with the heart of time (DV 3, Controlled/Standard), repairing it with greater effect.
\end{itemize}

\textbf{Obligation:} Each time a Runekeeper uses their abilities, they mark 1 Obligation to the Clockwork Monad. If they exceed their Obligation Capacity (Spirit + Presence), they immediately take 1 Fatigue.

\subsection{Position and Effect}

The shifting architecture of the cathedral creates dynamic Position changes:

\begin{itemize}
    \item \textbf{Dominant:} When the players have a moment to prepare or find a stable position (e.g., when the corridors settle).
    \item \textbf{Controlled:} The default state in most areas of the cathedral.
    \item \textbf{Desperate:} When the cathedral is shifting violently or the players are in danger (e.g., near the grinding gear or fighting the Time Devourer).
\end{itemize}

\subsection{Story Beats and Boons}

\begin{itemize}
    \item \textbf{Story Beats:} The cathedral generates SB through its shifting architecture. Each time the Cathedral Mood worsens, the GM gains 1 SB. The GM should spend SB on complications related to the cathedral's architecture.
    \item \textbf{Boons:} Players earn Boons for meaningful failures (e.g., failing to repair a gear but learning how to communicate with it). They can spend Boons to improve their Position or to stabilize the cathedral.
\end{itemize}

\begin{tcolorbox}[colback=blue!5,colframe=blue!50!black,title=Design Note]
This adventure was specifically designed to preserve the tactical depth of D\&D 5e while emphasizing Fate's Edge's narrative flexibility. The shifting architecture creates dynamic battlefields that require tactical thinking, while the Cathedral Mood Clock and Living Gears add narrative consequences that go beyond mere combat. The adventure is designed so that players who focus on repairing the cathedral (rather than fighting) will have the best chance of success, reflecting Fate's Edge philosophy of "narrative first, mechanics serving the story."
\end{tcolorbox}

\section{Adaptation Notes for GMs}

\subsection{For D\&D 5e Fans}

This adventure preserves the tactical combat D\&D fans love while adding narrative depth through the Position and Effect system. The shifting architecture creates dynamic battlefields that require tactical thinking, and the adventure includes clear class roles:

\begin{itemize}
    \item \textbf{Guardian (Front-line):} Fights the Time Devourer and protects the repairers.
    \item \textbf{Scout (Puzzle-solver):} Navigates the shifting corridors and finds the broken gear.
    \item \textbf{Diplomat (Negotiating with living gears):} Communicates with the living gears and the Guardian Gears.
    \item \textbf{Specialist (Mechanic):} Repairs the clockwork and interfaces with the heart.
\end{itemize}

\subsection{Scaling for Higher Tiers}

For Tier II+ characters:

\begin{itemize}
    \item Increase the DVs by 1 for all checks.
    \item Increase the size of the clocks by 2 (e.g., Broken Gear [6] instead of [4]).
    \item Add a second Time Devourer or a more powerful variant.
    \item The Time Devourer gains additional abilities (e.g., can shift time for a single player, forcing them to re-roll all dice).
\end{itemize}

\section{Conclusion}

The Clockwork Cathedral is an adventure that bridges the gap between tactical combat and narrative consequences. It preserves what D\&D fans love about tactical depth while introducing them to the narrative flexibility of Fate's Edge. The living architecture creates dynamic challenges that require both strategic thinking and narrative creativity, and the Cathedral Mood Clock ensures that every action has consequences that ripple through the adventure.

The adventure ends with a moment of triumph—not through defeating a monster, but through repairing a sacred mechanism and restoring balance to a broken world. This reflects Fate's Edge's core philosophy: that narrative comes first, and mechanics exist to serve the story.

\begin{center}
\textit{"Time does not heal all wounds. But sometimes, with careful hands and a willing heart, it can be made to flow in the right direction again."}
\end{center}

\end{document}