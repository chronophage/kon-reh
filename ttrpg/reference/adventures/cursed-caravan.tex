\documentclass[12pt,letterpaper]{article}
\usepackage[utf8]{inputenc}
\usepackage[margin=1in]{geometry}
\usepackage{titlesec}
\usepackage{enumitem}
\usepackage{hyperref}
\usepackage{xcolor}
\usepackage{multicol}

% Define colors
\definecolor{acasiadust}{RGB}{184, 134, 11}
\definecolor{acasiablood}{RGB}{139, 0, 0}
\definecolor{acasiagrey}{RGB}{105, 105, 105}

% Section formatting
\titleformat{\section}
{\color{acasiadust}\normalfont\Large\bfseries}
{\color{acasiadust}\thesection}{1em}{}

\titleformat{\subsection}
{\color{acasiablood}\normalfont\large\bfseries}
{\color{acasiablood}\thesubsection}{1em}{}

% Item spacing
\setlist{itemsep=2pt}

\title{\color{acasiadust} \textbf{The Cursed Caravan}\\ \large An Adventure for Fate's Edge - Acasia}
\author{}
\date{}

\begin{document}

\maketitle

\section*{Adventure Overview}

\textbf{Title:} The Cursed Caravan\\
\textbf{Region:} Acasia - Broken Marches\\
\textbf{Theme:} Supernatural horror, cursed objects, moral ambiguity\\
\textbf{Level:} Seasoned (Tier II)\\
\textbf{Length:} 3-4 Sessions

\section{Premise}

A merchant caravan has been cursed while crossing the Pale Causeway, causing supplies to rot, animals to die, and strange omens to appear. The players must escort the caravan to safety while dealing with the curse's escalating effects. However, they soon discover that the curse is tied to a larger conspiracy involving a powerful entity that feeds on suffering and despair, and that the caravan's troubles may be just the beginning of something far worse.

\section{Hook}

The adventure begins when the players encounter one of the following scenarios:

\begin{itemize}
  \item They find the caravan stalled on the road, with the merchant desperately seeking help for his dying animals and rotting supplies. He offers them a substantial reward to help get the caravan to the nearest settlement.
  \item A Margravine's bounty poster shows the players' faces with the accusation that they're responsible for the curse, forcing them to investigate to clear their names.
  \item A hedge-witch approaches them in a tavern, warning that something dark is moving along the trade routes and that they may be the only ones who can stop it.
\end{itemize}

\section{Key Factions}

\subsection{The Merchant's Caravan}
Led by Merchant Aldric Thorne, a desperate trader trying to save his livelihood and his family's reputation. His caravan includes:
\begin{itemize}
  \item \textbf{Serah Thorne:} Aldric's daughter, a practical young woman who's beginning to suspect the curse has a supernatural origin.
  \item \textbf{Garrick the Guard:} A veteran caravan guard who's seen his share of trouble and is growing increasingly superstitious.
  \item \textbf{Mason the Cook:} A nervous man who's been having terrible dreams since the curse began.
  \item \textbf{The Cargo:} A mixed load of trade goods including textiles, spices, and a mysterious sealed chest that may be connected to the curse.
\end{itemize}

\subsection{The Pale Shepherd}
A mysterious figure who appears at crossroads and offers guidance to those who are lost. In Acasia, the Pale Shepherd is associated with transitions and endings, and may be trying to help the caravan complete its journey - or prevent it from reaching its destination.

\subsection{The Hungering Dark}
An ancient entity that feeds on suffering, despair, and the life force of living things. It was accidentally awakened when the cursed object was removed from its resting place, and now it's growing stronger as it feeds on the caravan's troubles.

\section{Key NPCs}

\subsection{Merchant Aldric Thorne}
A middle-aged trader whose family has been in the caravan business for generations. He's practical and skeptical, but the curse is pushing him to the edge of desperation. He's willing to listen to supernatural solutions but needs concrete results.

\subsection{Hedge-Witch Mira Blackwood}
An elderly woman who lives in a charcoal burner's hollow near the Pale Causeway. She knows more about the curse than she initially reveals and has her own reasons for wanting to see it resolved. She can provide information about local customs and supernatural protections.

\subsection{Margravine Elara Rothari}
The iron-willed ruler of the Broken Marches who maintains order through a combination of force and cunning. She has her own interests in seeing the curse resolved, as it's disrupting trade and potentially threatening her authority.

\subsection{The Cursed Child}
A legendary figure from Acasian folklore who is said to appear when curses reach their most terrible point. The child's laughter can end sieges, but at a terrible cost. Some say the child is a manifestation of the land's suffering, while others believe it's something far more ancient and malevolent.

\section{Key Locations}

\subsection{The Pale Causeway}
The last high road that survives spring thaws, a crucial trade route that crosses dangerous terrain. The causeway is lined with warnings and protective charms, but the curse is overwhelming these defenses. The road itself seems to shift and change, making navigation difficult.

\subsection{Blackwood Hollow}
A charcoal burner's clearing deep in the woods where Hedge-Witch Mira lives. The hollow is protected by various charms and wards, but even here the curse's influence can be felt. The area is rich with local folklore and supernatural knowledge.

\subsection{The Margravine's Court}
A fortress-like structure that serves as the seat of regional authority. The court is a place of iron law and harsh justice, where disputes are settled quickly and decisively. The Margravine maintains order through a combination of legitimate authority and the fear inspired by her reputation.

\subsection{The Salt Road Ford}
A crucial crossing point where the caravan route intersects with a river. The ford is normally safe, but the curse has caused the water to become unnaturally cold and the crossing to become treacherous. Bones sometimes appear in the chalk banks, adding to the ominous atmosphere.

\subsection{The Cursed Crossroads}
A place where multiple roads meet, marked by a stone that always shows the wrong time. This is where the curse's influence is strongest, and where the entity feeding on the caravan's suffering is most active. The crossroads is a place of choice and consequence, where decisions have lasting effects.

\section{Plot Structure}

\subsection{Session 1: The Cursed Caravan}
The players encounter the cursed caravan and begin to investigate the source of its troubles. They should:
\begin{itemize}
  \item Experience the immediate effects of the curse firsthand
  \item Meet the key members of the caravan
  \item Begin to suspect supernatural involvement
  \item Make initial decisions about how to proceed
\end{itemitemize}

\subsection{Session 2: Into the Blackwood}
The players seek help from Hedge-Witch Mira and delve deeper into the curse's origins. They should:
\begin{itemize}
  \item Learn about local folklore and supernatural protections
  \item Discover the curse's connection to the sealed chest
  \item Face the growing influence of the Hungering Dark
  \item Make moral choices about how to deal with the curse
\end{itemize}

\subsection{Session 3: The Margravine's Justice}
The players must deal with the political and legal complications of the curse while it continues to escalate. They should:
\begin{itemize}
  \item Navigate the Margravine's court and its harsh justice
  \item Deal with the curse's effects on the local community
  \item Confront the entity behind the curse directly
  \item Make crucial decisions about the curse's ultimate fate
\end{itemize}

\subsection{Session 4: The Pale Child}
The curse reaches its climax as the Cursed Child appears and offers a final resolution. The players should:
\begin{itemize}
  \item Face the ultimate consequences of their choices
  \item Deal with the Cursed Child's terrible bargain
  \item Resolve the immediate threat to the caravan
  \item Determine the long-term effects of their actions
\end{itemize}

\section{Key Mechanics}

\subsection{Curse Escalation Clock [8]}
Tracks the curse's growing power and influence. Advances when:
\begin{itemize}
  \item Animals die or supplies rot
  \item NPCs become possessed or influenced
  \item Players fail to properly contain the curse
  \item The Hungering Dark successfully feeds
\end{itemize}

\subsection{Caravan Supplies [6]}
Represents the caravan's remaining resources. Depletes when:
\begin{itemize}
  \item Food and water spoil due to the curse
  \item Equipment breaks or becomes unusable
  \item Animals die or become unable to work
  \item Players make poor resource management decisions
\end{itemize}

\subsection{Hungering Dark's Strength [6]}
Measures the entity's growing power as it feeds on suffering. Increases when:
\begin{itemize}
  \item NPCs experience fear, despair, or physical suffering
  \item The curse's effects become more severe
  \item Players use violent solutions that create more suffering
  \item Proper supernatural protections are not maintained
\end{itemize}

\subsection{Folk Protection Tokens}
Local charms and wards that can provide temporary protection:
\begin{itemize}
  \item \textbf{Butter-Left Charm:} Place butter in niches to keep the curse at bay (1 use)
  \item \textbf{Red Thread Binding:} Tie red thread around wrists to maintain connection to the living world (+1 die to resist possession)
  \item \textbf{Salt-Warding:} Draw salt circles to contain supernatural entities (lasts until broken)
  \item \textbf{Hedge-Witch's Blessing:} Gain +1 die to supernatural knowledge rolls (1 scene)
\end{itemize}

\section{Possible Resolutions}

\subsection{Proper Containment}
The players discover how to properly contain the cursed object and prevent the Hungering Dark from feeding further:
\begin{itemize}
  \item Perform the correct rituals to bind the curse
  \item Return the object to its resting place with proper ceremony
  \item Gain the blessing of local supernatural forces
\end{itemize}
\textbf{Consequences:} Curse contained, grateful merchant, potential for future supernatural allies.

\subsection{Sacrificial Bargain}
The players make a terrible bargain with the Hungering Dark or the Cursed Child to end the immediate threat:
\begin{itemize}
  \item Offer a willing sacrifice to sate the entity
  \item Trade away something of great personal value
  \item Accept a geas that will have future consequences
\end{itemize}
\textbf{Consequences:} Immediate safety but long-term supernatural debt, moral compromise, ongoing connection to dark forces.

\subsection{Destruction Solution}
The players find a way to destroy the cursed object and the entity feeding on it:
\begin{itemize}
  \item Locate and use the proper tools or rituals for destruction
  \item Accept the risks involved in such a dangerous undertaking
  \item Deal with the backlash from destroying a powerful supernatural force
\end{itemize}
\textbf{Consequences:} Complete resolution but potential for new supernatural threats, recognition as powerful agents, possible loss of valuable items.

\subsection{Escape and Abandonment}
The players choose to save themselves and abandon the caravan to its fate:
\begin{itemize}
  \item Flee the cursed area and leave others to deal with the consequences
  \item Accept the moral cost of their abandonment
  \item Face the long-term effects of their choice
\end{itemize}
\textbf{Consequences:} Personal survival but guilt and reputation damage, potential future encounters with the curse, loss of potential rewards.

\section{Rewards and Consequences}

\subsection{Immediate Rewards}
\begin{itemize}
  \item Toll-exemption plaque for one bridge
  \item Monastery letter for bed-and-bread on a named road
  \item Wine-right on an abandoned terrace
  \item Condotta contract for one battle
  \item Tithe-remission writ for a village
  \item Border-stone adjustment
  \item Pass-key charm recognized by Pale Causeway watchmen
\end{itemize}

\subsection{Long-term Consequences}
\begin{itemize}
  \item Reputation as either heroes or cowards in the Broken Marches
  \item Supernatural connections that may prove beneficial or dangerous
  \item Ongoing relationships with local authorities and folk practitioners
  \item Potential for future encounters with the Hungering Dark or similar entities
  \item Moral and psychological effects of dealing with supernatural horror
\end{itemize}

\section{GM Notes}

\subsection{Atmosphere and Tone}
Emphasize the bleak, desperate atmosphere of Acasia. The Broken Marches are a place where hope is scarce and survival is hard-won. The supernatural elements should feel genuinely threatening and morally complex rather than simply monstrous.

\subsection{Folk Horror Elements}
Use local customs, superstitions, and folk practices to create an authentic atmosphere. The players should feel like they're dealing with a living culture rather than generic fantasy elements. Local knowledge should be valuable and specific.

\subsection{Moral Ambiguity}
The curse and its resolution involve difficult moral choices. There may not be a perfect solution, and the players' choices should have meaningful consequences. The right thing to do may not be the easy thing, and helping some may harm others.

\subsection{Escalating Tension}
Gradually increase the curse's effects and the supernatural threats. Start with minor disturbances and build to genuinely terrifying encounters. The players should feel like they're racing against time to prevent a catastrophe.

\subsection{Player Agency}
Provide multiple paths to resolution that respect different player approaches. Some players may prefer direct supernatural confrontation, others might focus on investigation and ritual, and still others might try to work within the existing power structure.

\subsection{Story Beats}
Use the curse's influence to generate Story Beats that complicate the players' efforts while advancing the plot. The Hungering Dark should feel like an active antagonist rather than a passive threat, manipulating events to feed on suffering and despair.

\end{document}
