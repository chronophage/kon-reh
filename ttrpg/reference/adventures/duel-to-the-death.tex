\documentclass[11pt,oneside]{article}

% ----------- PACKAGES -----------
\usepackage[margin=1in]{geometry}
\usepackage{tcolorbox}
\usepackage{enumitem}
\usepackage{amsmath, amssymb}
\usepackage{titlesec}
\usepackage{hyperref}

% optional: nicer fonts
\usepackage{lmodern}

% tcolorbox setup (clean black frame, no shadows)
\tcbset{
  enhanced,
  boxsep=4pt,
  arc=2pt,
  colback=white,
  colframe=black,
}

% tighter item spacing globally
\setlist[itemize]{nosep}

% section formatting (optional, SRD-style)
\titleformat{\section}{\normalfont\bfseries\Large}{}{0pt}{}
\titleformat{\subsection}{\normalfont\bfseries\large}{}{0pt}{}

\begin{document}

% -------------------------------------------------
% Seven Bell Court Preamble (drop-in)
% -------------------------------------------------

\begin{tcolorbox}[
  colback=white,
  colframe=black,
  fonttitle=\bfseries,
  title=The Judgment of the Seven Bell Court]

In Ayokha, a duel is not merely the meeting of steel and skill.
To the Celestial Order, combat without meaning is savagery.
To Sihai philosophers, technique without restraint is chaos.
To Nihon poets, victory without honor is an empty shadow.

Thus, every formal bout is overseen by the \textbf{Seven Bell Court}:
a neutral panel of masters, scholars, and priest-judges who weigh each clash
by three sacred pillars of excellence:

\begin{itemize}
  \item \textbf{Form} – discipline, precision, and mastery of technique.
  \item \textbf{Spirit} – composure, respect, and dignity in conflict.
  \item \textbf{Intent} – purpose beyond ego: protection, duty, justice, or peace.
\end{itemize}

A fighter may fall, yet be honored as victorious.  
A fighter may stand triumphant, yet be judged disgraced.

The Scorecard system turns every match into a story:
\begin{itemize}
  \item A reckless strike might land—yet shame the audience.
  \item A humble bow might calm a volatile crowd.
  \item A moment of mercy might prevent a war.
\end{itemize}

Use the Judges’ Scorecards in tournaments, duels of honor,
trial-by-combat, diplomatic exhibitions, or ritual tests of skill.
When the world watches, a single point of Spirit may be worth more than blood.

\end{tcolorbox}

% -------------------------------------------------
% (Your rules, scorecard tables, or generators go here)
% -------------------------------------------------

% =========================================================
% DUEL TO THE DEATH: WU-XING vs. NINJA (LaTeX Conversion)
% Requires: tcolorbox, tabularx, longtable, xcolor, enumitem
% =========================================================
\begin{tcolorbox}[
  colback=white,
  colframe=black,
  fonttitle=\bfseries,
  title=The Judgment of the Seven Bell Court]

In Ayokha, a duel is not merely the meeting of steel and skill.
To the Celestial Order, combat without meaning is savagery.
To Sihai philosophers, technique without restraint is chaos.
To Nihon poets, victory without honor is an empty shadow.

Thus, every formal bout is overseen by the \textbf{Seven Bell Court}:
a neutral panel of masters, scholars, and priest-judges who weigh each clash
by three sacred pillars of excellence:

\begin{itemize}[nosep]
  \item \textbf{Form} – discipline, precision, and mastery of technique.
  \item \textbf{Spirit} – composure, respect, and dignity in conflict.
  \item \textbf{Intent} – purpose beyond ego: protection, duty, justice, or peace.
\end{itemize}

A fighter may fall, yet be honored as victorious.  
A fighter may stand triumphant, yet be judged disgraced.

The Scorecard system turns every match into a story:
\begin{itemize}[nosep]
  \item A reckless strike might land—yet shame the audience.
  \item A humble bow might calm a volatile crowd.
  \item A moment of mercy might prevent a war.
\end{itemize}

Use the Judges’ Scorecards in tournaments, duels of honor,
trial-by-combat, diplomatic exhibitions, or ritual tests of skill.
When the world watches, a single point of Spirit may be worth more than blood.

\end{tcolorbox}

\section{Duel to the Death: Wu\,-Xing vs.\ Ninja}
\label{sec:duel-wuxing-ninja}
\subsection*{Conspiracy of Shadows and Elements}

% ---------- OVERVIEW ----------
\subsection{Adventure Overview}
\textbf{Premise.} An ``honorable'' martial arts tournament between Sihai and Nihon masks a joint conspiracy by corrupt merchants from both nations. The real prize is control of the Ayokhan spice routes, not the public trading post. Two legendary champions are being used as pawns, and their duel could trigger war.

\medskip
\textbf{Stakes.} PCs must uncover the scheme before the finale, prevent an international incident, and decide whether to expose the truth (risking war) or defuse it without igniting hostilities.

% ---------- THE CONSPIRACY ----------
\subsection{The Conspiracy Unveiled}

\paragraph{The True Players}
\textbf{The Crimson Merchants' Guild}
\begin{itemize}[leftmargin=1.25em]
  \item \textbf{Membership:} Corrupt traders from Sihai and Nihon.
  \item \textbf{Leadership:} \textbf{Merchant Lord Zhao} (Sihai) and \textbf{Daimyo Kuroda} (Nihon; secretly a merchant in disguise).
  \item \textbf{Goal:} Monopolize Ayokha's spice trade by eliminating legitimate competition.
\end{itemize}

\paragraph{The Plan}
\begin{enumerate}[leftmargin=1.25em]
  \item \textbf{Manufacture Conflict:} Staged incidents inflame Sihai--Nihon tensions.
  \item \textbf{Heroic Distraction:} Famous champions become unwitting figureheads.
  \item \textbf{Economic Warfare:} Corner markets while eyes are on the tournament.
  \item \textbf{Political Manipulation:} Orchestrate ``outrage'' and sell themselves as the cure.
\end{enumerate}

% ---------- THE CHAMPIONS ----------
\subsection{The Champions: Pawns in a Larger Game}

\paragraph{Master Li Wei (Sihai) --- \textit{The Honorable Warrior, Blind to the Truth}}
\textbf{Background.} A disciplined warrior-monk convinced he defends Sihai's honor and safeguards trade for common folk.\\
\textbf{Blind Spot.} Unaware his patrons are manipulating him; holds a personal vendetta against Kage after a humiliating past encounter.\\
\textbf{Personality.} Noble, doctrinaire, believes in Sihai philosophical rigor; wavers when Kage's unconventional methods blunt his forms.

\paragraph{Shinobi Kage (Nihon) --- \textit{The Pragmatic Killer with Hidden Honor}}
\textbf{Background.} A veteran shinobi who believes he serves national interest, though cynically aware of political rot.\\
\textbf{Blind Spot.} Respects Li Wei despite animosity; underestimates the conspiracy's scope.\\
\textbf{Personality.} Dry, professionally proud, grudging respect for Li Wei's skill.

% ---------- TOURNAMENT FACADE ----------
\subsection{The Tournament Structure (Facade)}

\paragraph{Phase 1: The Gathering Storm}
Delegations arrive; tensions and odd incidents mount.\\
\textbf{PC Roles.} Participants; staff (healers/smiths/cooks); journalists; diplomatic observers; mercenary security.

\paragraph{Phase 2: Trials of Elements}
Showcases of styles and traditions; sabotage attempts increase.

\paragraph{Phase 3: The Final Duel}
Li Wei vs.\ Kage --- engineered as a flashpoint for staged ``outrage.''

% ---------- WEB OF DECEPTION ----------
\subsection{The Web of Deception}

\paragraph{The Poisoned Chalice}
\begin{itemize}[leftmargin=1.25em]
  \item Banquet fare laced with truth-serum and loyalty binders.
  \item Used to extract intel and nudge behavior.
\end{itemize}

\paragraph{The Shadow Network}
\begin{itemize}[leftmargin=1.25em]
  \item Bribed Ayokhan officials look away.
  \item Merchant ships rerouted to dodge legitimate tariffs.
\end{itemize}

\paragraph{The False Evidence}
\begin{itemize}[leftmargin=1.25em]
  \item Forged documents ``prove'' the losing nation's aggression.
  \item Timed discovery just after the finals to justify sanctions.
\end{itemize}

% ---------- MECHANICAL FRAMEWORK ----------
\subsection{Mechanical Framework}

\paragraph{Investigation Clocks}
\begin{itemize}[leftmargin=1.25em]
  \item \textbf{Uncovering the Conspiracy [8]}
  \item \textbf{Political Tensions [6]}
  \item \textbf{Tournament Integrity [4]}
  \item \textbf{Champions' Relationship [6]}
  \item \textbf{Conspiracy Clock [6]} (advances when PCs fail with cost or the champions' relationship worsens)
\end{itemize}

\paragraph{Key NPCs \& Secrets}
\textbf{Merchant Lord Zhao (Sihai)} --- Embezzling from the Imperial Treasury; needs spice profits to plug theft; arrogant.\\
\textbf{Daimyo Kuroda (Nihon; actually a merchant)} --- Illegitimate lineage; assumes false identity; paranoid about exposure.\\
\textbf{Governor Priya of Ayokha} --- Aware of the scheme, blackmailed; values citizens' safety; family held hostage.

% ---------- CHAMPION SHEETS ----------
\subsection{Champion Character Sheets}

\begin{tcolorbox}[colback=white,colframe=black!70,title={\textbf{Master Li Wei (Sihai Wu\,-Xing Master)}}]
\begin{tabularx}{\linewidth}{@{}lX@{}}
\textbf{Attributes} & Body 3, Wits 2, Spirit 4, Presence 2 \\
\textbf{Skills} & Melee 3, Arcana 3, Athletics 2, Insight 2 \\
\textbf{Talents} & Caster's Gift; Elemental Harmony; Perfect Timing Way; Transcendent Harmony \\
\textbf{Approach} & Controlled/Standard for elemental arts; Desperate/Great when emotionally compromised \\
\end{tabularx}
\end{tcolorbox}

\begin{tcolorbox}[colback=white,colframe=black!70,title={\textbf{Shinobi Kage (Nihon Shadow Warrior)}}]
\begin{tabularx}{\linewidth}{@{}lX@{}}
\textbf{Attributes} & Body 2, Wits 4, Spirit 3, Presence 2 \\
\textbf{Skills} & Stealth 4, Melee 3, Deception 3, Survival 2 \\
\textbf{Talents} & Shadow Dance; Backstab; Deathblow; Conditioning \\
\textbf{Approach} & Dominant/Standard in shadows; Controlled/Limited when exposed \\
\end{tabularx}
\end{tcolorbox}

% ---------- ADVENTURE HOOKS BY ROLE ----------
\subsection{Adventure Hooks by PC Type}

\paragraph{Participants}
Opponents seem pre-briefed; prize money trails to shell accounts; bracket manipulation.\\
\textit{Hook Roll (Insight + Notice vs.\ DV 3):} Success reveals patterns; Partial sees ``something off''; Miss leads into a trap.

\paragraph{Support Staff}
Overhear collusion; restricted shipments; preferential treatment.\\
\textit{Hook Roll (Stealth + Survival vs.\ DV 4):} Success gains intel; Partial raises suspicion; Miss compromises position.

\paragraph{Journalists}
Assigned puff piece; conflicting tips; cross-border sources disagree.\\
\textit{Hook Roll (Sway + Lore vs.\ DV 3):} Success uncovers contradictions; Partial reveals spin; Miss gets discredited.

\paragraph{Diplomats}
Protocol manipulation; coded dispatches; access to intel networks.\\
\textit{Hook Roll (Command + Insight vs.\ DV 4):} Success IDs conspirators; Partial shows pressure; Miss triggers an incident.

\paragraph{Mercenaries}
Conflicting orders; ignored breaches; pressure to silence witnesses.\\
\textit{Hook Roll (Athletics + Survival vs.\ DV 3):} Success exposes sabotage; Partial notes payments; Miss makes you accessories.

% ---------- KEY INVESTIGATION SCENES ----------
\subsection{Key Investigation Scenes}

\paragraph{Scene 1: The Poisoned Banquet}
\textit{Reconciliation dinner; detect serums, observe odd champion behavior, avoid being dosed.}\\
\textbf{Investigation (Lore + Insight vs.\ DV 4):}
\begin{itemize}[leftmargin=1.25em]
  \item \textit{Success:} Identify serum ingredients; note odd behavior.
  \item \textit{Partial:} Detect tampering; not specifics; champions act strangely.
  \item \textit{Miss:} Dosed; reveal sensitive info; GM spends 2 SB (compromised leverage).
\end{itemize}
\textbf{Alternative (Stealth + Survival vs.\ DV 3):}
\begin{itemize}[leftmargin=1.25em]
  \item \textit{Success:} Avoid food; observe reactions.
  \item \textit{Partial:} Small dose; notice taste, muted effects.
  \item \textit{Miss:} Fully dosed; act against interests next scene.
\end{itemize}

\paragraph{Scene 2: The Sabotaged Dojo}
\textit{``Accidental'' facility damage; trace conspirators; prevent repeats.}\\
\textbf{Investigation (Craft + Investigation vs.\ DV 4):}
\begin{itemize}[leftmargin=1.25em]
  \item \textit{Success:} Trace tools/materials; identify fingerprints.
  \item \textit{Partial:} Know it’s sabotage; source unclear; prevent immediate repeats.
  \item \textit{Miss:} Blamed; Tournament Integrity +1; GM spends 1 SB (suspicion).
\end{itemize}
\textbf{Repair (Tinker + Athletics vs.\ DV 3):}
\begin{itemize}[leftmargin=1.25em]
  \item \textit{Success:} Restore facilities; +1 Position for training.
  \item \textit{Partial:} Temporary fix; risk later collapse.
  \item \textit{Miss:} Worsen damage; training blocked; GM spends 1 SB (schedule hit).
\end{itemize}

\paragraph{Scene 3: The Midnight Meeting}
\textit{Eavesdrop leaders’ huddle; or survive a honeytrap.}\\
\textbf{Infiltration (Stealth + Deception vs.\ DV 5):}
\begin{itemize}[leftmargin=1.25em]
  \item \textit{Success:} Full conversation; key players and plans.
  \item \textit{Partial:} Fragmentary info; suspicion rises.
  \item \textit{Miss:} Discovered; combat/rout; GM spends 2 SB (capture risk).
\end{itemize}
\textbf{Social (Presence + Sway vs.\ DV 4):}
\begin{itemize}[leftmargin=1.25em]
  \item \textit{Success:} Invitation secured; inner circle access.
  \item \textit{Partial:} Limited access; curated info.
  \item \textit{Miss:} Outed as impostor; become target; GM spends 2 SB.
\end{itemize}

\paragraph{Scene 4: The False Evidence Plant}
\textit{Catch the forgery room; weigh truth vs.\ protecting innocents.}\\
\textbf{Detection (Insight + Notice vs.\ DV 4):}
\begin{itemize}[leftmargin=1.25em]
  \item \textit{Success:} Identify operation; gather evidence; find originals.
  \item \textit{Partial:} See forgeries; originals missing; risk being framed.
  \item \textit{Miss:} Framed; Political Tensions +2; GM spends 3 SB (arrest now).
\end{itemize}
\textbf{Analysis (Lore + Craft vs.\ DV 3):}
\begin{itemize}[leftmargin=1.25em]
  \item \textit{Success:} Prove forgery; ID methods/forger.
  \item \textit{Partial:} Inconsistencies noted; proof lacking.
  \item \textit{Miss:} Accept as genuine; investigation misled; GM spends 2 SB.
\end{itemize}

% ---------- CLIMAX PATHS ----------
\subsection{Climax: Multiple Paths}

\paragraph{Path 1: Prevent the Final Duel}
\textit{Sway both champions; handle fallout of a canceled bout.}\\
\textbf{(Sway + Insight vs.\ DV 5):}
\begin{itemize}[leftmargin=1.25em]
  \item \textit{Success:} Both unite against conspirators.
  \item \textit{Partial:} One convinced; internal rift.
  \item \textit{Miss:} Both distrust PCs; GM spends 3 SB (hostility).
\end{itemize}

\paragraph{Path 2: Let the Duel Proceed, Expose the Truth}
\textit{Keep honor of the ring; unmask plot at peak.}\\
\textbf{(Wits + Athletics vs.\ DV 4):}
\begin{itemize}[leftmargin=1.25em]
  \item \textit{Success:} Perfect timing; max impact; minimal collateral.
  \item \textit{Partial:} Good timing; some bystander risk.
  \item \textit{Miss:} Poor timing; confusion/casualties; GM spends 2 SB.
\end{itemize}

\paragraph{Path 3: Turn the Tables}
\textit{Run a sting during the finals; high risk to bystanders if timing slips.}\\
\textbf{(Command + Tinker vs.\ DV 5):}
\begin{itemize}[leftmargin=1.25em]
  \item \textit{Success:} Flawless execution; conspirators caught.
  \item \textit{Partial:} Some escape; evidence compromised.
  \item \textit{Miss:} Plan fails; innocents endangered; GM spends 4 SB.
\end{itemize}

% ---------- CONSEQUENCES ----------
\subsection{Consequences \& Endings}

\paragraph{Exposed Early}
War averted; trade normalizes; reputational damage and personal reckonings.\\
\textit{Reward:} +4 XP each; +1 Tier if applicable; Patron favor (Truth/Justice domains).

\paragraph{Partial Success}
Some arrests; tensions linger; roots remain.\\
\textit{Reward:} +2 XP each; mixed consequences; ongoing subplot hooks.

\paragraph{Conspiracy Succeeds}
War, economic collapse, and civilian suffering.\\
\textit{Consequence:} --2 XP each; Patron debt (Failure); new enemies.

\paragraph{PCs Fail to Act}
Default catastrophe: inter-nation war; Ayokhan trade implosion.\\
\textit{Consequence:} --4 XP each; major Patron debt; campaign-altering fallout.

% ---------- CHARACTER DEVELOPMENT ----------
\subsection{Character Development}

\paragraph{Li Wei}
Choose between rigid honor and adaptive pragmatism.\\
\textit{Growth (Spirit + Insight vs.\ DV 4):}
\begin{itemize}[leftmargin=1.25em]
  \item \textit{Success:} Integrates pragmatism with honor; becomes mentor.
  \item \textit{Partial:} Temporary compromise; internal conflict lingers.
  \item \textit{Miss:} Loses faith; redemption arc seed.
\end{itemize}

\paragraph{Kage}
Decide whether cynicism equals complicity.\\
\textit{Growth (Presence + Survival vs.\ DV 3):}
\begin{itemize}[leftmargin=1.25em]
  \item \textit{Success:} Rediscovers purpose; gains honor.
  \item \textit{Partial:} Selective morality; measured choices.
  \item \textit{Miss:} Drifts toward antagonist; future redemption hook.
\end{itemize}

\paragraph{For PCs}
Moral calculus; cultural literacy; growth in investigation, diplomacy, or combat.\\
\textit{Cultural Check (Lore + Insight vs.\ DV 3):}
\begin{itemize}[leftmargin=1.25em]
  \item \textit{Success:} +1d to Sihai/Nihon/Ayokhan interactions.
  \item \textit{Partial:} Basic understanding; modest social edge.
  \item \textit{Miss:} Faux pas; --1 Position on related interactions.
\end{itemize}

% ---------- CULTURAL NOTES ----------
\subsection{Cultural Notes}

\paragraph{Historical Parallels (Fictionalized)}
Sihai: ideal vs.\ governance friction; bureaucracy exploited.\\
Nihon: clan politics; duty weaponized pragmatically.\\
Ayokha: trade hub mediating cultures under pressure.

\paragraph{Champion Conflict}
Embodies philosophical differences; shows manipulation of noble intent; personal grudges as levers.

\paragraph{Ayokhan Cultural Elements}
\textbf{Monsoon Timing:} Investigation effectiveness varies with seasonal winds.\\
\textbf{Spirit World:} Local \emph{nat} may aid (Arcana + Spirit vs.\ DV 4).\\
\textbf{Mandala Politics:} Multi-faction negotiation with vassal kings and sea-lords.

% ---------- ADVENTURE PACING ----------
\subsection{Adventure Pacing}

\paragraph{Sessions 1--2}
Introductions, tensions, first clues.\\
\textit{Focus:} Establish tournament, champions, first conspiracy hints.

\paragraph{Sessions 3--4}
Escalation, sabotage, revelations.\\
\textit{Focus:} Deepen investigation; confront conspirators; champions' relationship deteriorates.

\paragraph{Session 5}
Finale: sting, exposure, or duel; fallout.\\
\textit{Focus:} Climactic confrontation; resolve threads; apply consequences.

% ---------- GM NOTES ----------
\subsection{GM Notes}

\paragraph{Balance}
Conspiracy is discoverable but not obvious; champions have legitimate motives; multiple PC victory routes.\\
\textit{Guideline:} Each major clue requires a meaningful test, not passive observation.

\paragraph{Sensitivity}
Individuals are culpable, not cultures; showcase virtues across traditions; Ayokha as bridge culture.\\
\textit{Reminder:} Champions embody ideals, not stereotypes.

\paragraph{Agency}
Multiple investigative vectors; flexible alliances; consequences drive story.\\
\textit{Mechanic:} Allow temporary, goal-focused faction alliances.

\paragraph{Fate's Edge Integration}
Use Position/Effect for major challenges; spend SB for conspiracy complications (1--4 severity); award Boons for clever cultural leverage and moral choices; during downtime, 2 Boons $\rightarrow$ 1 XP for significant discoveries.

\paragraph{Monsoon Clock [8]}
Tracks seasonal changes affecting the investigation:
\begin{itemize}[leftmargin=1.25em]
  \item Seg.\ 1--2: Pre-monsoon --- dry conditions; easier surveillance.
  \item Seg.\ 3--4: Early monsoon --- activity spikes; shipping disruptions.
  \item Seg.\ 5--6: Peak monsoon --- investigation hindered; indoor focus.
  \item Seg.\ 7--8: Late monsoon --- clearing weather; finale windows open.
\end{itemize}

% ---------- SAMPLE INVESTIGATION ENCOUNTERS ----------
\subsection{Sample Investigation Encounters}

\paragraph{Document Forgery Detection}
\textit{Lore + Insight vs.\ DV 4}\\
\textit{Success:} Identify false documents; trace to forger.\\
\textit{Partial:} Inconsistencies found; more evidence needed.\\
\textit{Miss:} Forgeries accepted; investigation misdirected.

\paragraph{Bribe Detection}
\textit{Insight + Deception vs.\ DV 3}\\
\textit{Success:} Recognize corrupt behavior; gather proof.\\
\textit{Partial:} Sense corruption; proof lacking; suspicion rises.\\
\textit{Miss:} Become a target; compromised posture.

\paragraph{Shadow Network Tracking}
\textit{Investigation + Survival vs.\ DV 4}\\
\textit{Success:} Trace routes; identify actors.\\
\textit{Partial:} Evidence found; conspirators alerted; time pressure.\\
\textit{Miss:} False trail; wasted resources; GM spends 2 SB.

% ---------- TOURNAMENT COMBAT GUIDELINES ----------
\subsection{Tournament Combat Guidelines}

\paragraph{Phase 1 \& 2 Matches}
Position: Controlled/Standard (honorable competition). Effect: Limited/Great (style demonstration). Cultural considerations may tilt judging.

\paragraph{Final Duel}
Position: Risky/Desperate (stakes and manipulation). Effect: Standard/Great (champions at full capability). Audience reactions create environmental factors.

\paragraph{Conspiracy Intervention}
Position shifts with timing/approach; Effect modified by evidence presented; cultural honor stakes influence resolution options.
% =========================================================
% GAZETTEER GENERATORS — NIHON • AYOKHA • SIHAI
% Balanced presentation and quick-roll tables
% =========================================================

\section{The Isles of the Dawn Spirit (Nihon)}
\label{sec:nihon}

\begin{tcolorbox}[title={\textbf{Nihon --- The Isles of the Dawn Spirit}},colback=white,colframe=black!70]
\textbf{Tagline.} A storm-wracked archipelago of fierce clans, living spirits, and master artisans. Nihon refines many eastern ideas into forms uniquely its own, standing in creative tension with the continental power of Sihai.
\end{tcolorbox}

\subsection*{Overview (Balanced Framing)}
Nihon is contrast made culture: serene moss temples beneath volcanic rims; disciplined warriors amid fractious lordships. The islands absorb, test, and reforge influences (including Sihai’s), claiming neither imitation nor isolation but \emph{reinterpretation}.

% ---------- Geography ----------
\subsection{Geography Generator (d12)}
\begin{longtable}{@{}p{0.8cm}p{12.2cm}@{}}
\toprule
\textbf{d12} & \textbf{Feature} \\
\midrule
1 & Four great islands ring an \emph{Inland Sea}; trade junks tack between shrine-harbors. \\
2 & Fire-mountains smolder; ash fertilizes terraced paddies. \\
3 & Knife-edged coasts with storm-carved arches; hidden coves host clandestine docks. \\
4 & Cedar-clad ridges with mist stairways and bell-lines for avalanches. \\
5 & Typhoon corridors marked by stone beacons and wind shrines. \\
6 & Pearl banks guarded by reef kami; divers trade with temple fleets. \\
7 & Bamboo valleys echoing with practice blades at dawn. \\
8 & Black-sand beaches speckled with meteoric iron. \\
9 & Sky-bridges of rope and lacquer spanning ravines to fortress towns. \\
10 & Volcanic hot springs claimed by monasteries and swordsmiths alike. \\
11 & Cliff-temples where drums speak weather omens. \\
12 & Lantern-lit fishing villages; storm bells double as invasion alarms. \\
\bottomrule
\end{longtable}

% ---------- Polity ----------
\subsection{Polity \& Power (d10)}
\begin{tabularx}{\linewidth}{@{}p{0.8cm}X@{}}
\toprule
\textbf{d10} & \textbf{Detail} \\
\midrule
1 & \textbf{Heavenly Sovereign} holds ritual primacy; temples legitimize rule. \\
2 & \textbf{Shōgunate claim} contested; generals court temple backing. \\
3 & \textbf{Daimyō league} forms non-aggression pact against pirates. \\
4 & \textbf{Sword Monasteries} arbitrate disputes with trial bouts. \\
5 & \textbf{Harbor Councils} tax trade; share typhoon shelters. \\
6 & \textbf{Clanship Oaths} bind ashigaru to rice pledges. \\
7 & \textbf{Smith Guild Compact}: one masterpiece per year per forge. \\
8 & \textbf{Spirit-Mediator Shrine} licenses exorcists for pay. \\
9 & \textbf{Pirate Amnesty} offered for service against foreign raiders. \\
10 & \textbf{Isle Diet} convenes; decisions are advisory but symbolically powerful. \\
\bottomrule
\end{tabularx}

% ---------- Society ----------
\subsection{Society \& Arts (d8)}
\begin{tabularx}{\linewidth}{@{}p{0.8cm}X@{}}
\toprule
\textbf{d8} & \textbf{Custom / Art} \\
\midrule
1 & Swordsmithing houses pass secrets by \emph{failure journals}, not manuals. \\
2 & Rustic tea pavilions host ceasefires and spies in equal measure. \\
3 & Noh-style masked dramas teach clan history in allegory. \\
4 & Pottery kilns revere asymmetry and kiln accident. \\
5 & Poetry duels settle slights before blades do. \\
6 & Tide calendars are household shrines. \\
7 & Tattoo guilds encode loyalty oaths in wave motifs. \\
8 & Storm-offering floats carry names of the missing. \\
\bottomrule
\end{tabularx}

% ---------- Forces ----------
\subsection{Forces \& Methods (d10)}
\begin{tabularx}{\linewidth}{@{}p{0.8cm}X@{}}
\toprule
\textbf{d10} & \textbf{Military Element} \\
\midrule
1 & \textbf{Samurai} combined-arms: bow, spear, blade. \\
2 & \textbf{Ashigaru} spear walls with whistle signals. \\
3 & \textbf{Shinobi} smoke, peppers, false banners. \\
4 & \textbf{Matchlocks} adopted in volley lines along levees. \\
5 & \textbf{Coastal Wokou} raid tariffs; sometimes deputized. \\
6 & \textbf{Temple Guards} duel to arbitrate feuds. \\
7 & \textbf{Scout Boats} outrun storms by reading bird-lines. \\
8 & \textbf{Blade Saints} sworn to refuse coin, accept rice. \\
9 & \textbf{Armorers} field-test lamellar in rain and ash. \\
10 & \textbf{Storm Drummers} time charges between gusts. \\
\bottomrule
\end{tabularx}

% ---------- Trade ----------
\subsection{Trade \& Exchange (d8)}
\begin{tabularx}{\linewidth}{@{}p{0.8cm}X@{}}
\toprule
\textbf{d8} & \textbf{Exports / Imports} \\
\midrule
1 & \textbf{Exports:} blades, lacquer, ink; \textbf{Imports:} grain, silk, porcelain. \\
2 & Sulphur and silver fund harbor walls. \\
3 & Pirate-chased auctions depress prices (great bargains). \\
4 & Sihai brokers trade jade for sword commissions. \\
5 & Ayokhan spices traded for shipwright plans. \\
6 & Storm insurance scrip issued by shrine treasuries. \\
7 & Gun barrels proofed in temple pits. \\
8 & Pilgrim tourism to volcano shrines. \\
\bottomrule
\end{tabularx}

% ---------- Faith ----------
\subsection{Spirits \& Paths (d8)}
\begin{tabularx}{\linewidth}{@{}p{0.8cm}X@{}}
\toprule
\textbf{d8} & \textbf{Practice} \\
\midrule
1 & \textbf{Kannagara}: offerings to rock, tree, and tide kami. \\
2 & \textbf{Empty Self}: no-mind training in sword halls. \\
3 & Ancestral boats set adrift at equinox. \\
4 & Purity rites before duels; salt circles the ring. \\
5 & Volcano appeasement dances for ash-safe winds. \\
6 & Sea-kami ordain fishing limits by lot. \\
7 & Shrine arbiters stamp contracts with wind-ink. \\
8 & Zen gardens double as map puzzles for students. \\
\bottomrule
\end{tabularx}

% ---------- Relations ----------
\subsection{Relations (Parity Lens, d8)}
\begin{tabularx}{\linewidth}{@{}p{0.8cm}X@{}}
\toprule
\textbf{d8} & \textbf{Current State} \\
\midrule
1 & With \textbf{Sihai}: competitive emulation; naval patrol standoffs. \\
2 & With \textbf{Sihai}: scholar exchanges; tariff disputes. \\
3 & With \textbf{Ayokha}: monsoon ship swaps; dockside duels. \\
4 & With \textbf{Ayokha}: joint anti-piracy cruises. \\
5 & Western gun traders under shrine licensing. \\
6 & Neutral islands serve as truce markets. \\
7 & Mixed Sihai–Nihon craft guild in a border port. \\
8 & Disputed lighthouse fees spark legal bout. \\
\bottomrule
\end{tabularx}

% ---------- Hooks ----------
\subsection{Adventure Hooks (d8)}
\begin{tabularx}{\linewidth}{@{}p{0.8cm}X@{}}
\toprule
\textbf{d8} & \textbf{Hook} \\
\midrule
1 & The Broken Sword: a murdered smith; rival clans claim the masterpiece. \\
2 & Ronin’s Code: master betrayed; walk the ash-road for honor. \\
3 & Ghost of the Fire-Mountain: appease the waking kami or evacuate the valley. \\
4 & The Black Ship: salvage or secrecy? gunpowder bids escalate. \\
5 & Audience with the Sovereign: covert message through enemy provinces. \\
6 & Shrine Storm-Bell stolen; typhoon omens unheeded. \\
7 & Pirate Amnesty debate turns riotous at harbor court. \\
8 & Trial by Poetry averts war—unless someone cheats. \\
\bottomrule
\end{tabularx}

% =========================================================
\section{Ayokha — The Monsoon Throne}
\label{sec:ayokha}

\begin{tcolorbox}[title={\textbf{Ayokha --- The Monsoon Throne, River of Heaven}},colback=white,colframe=black!70]
\textbf{Tagline.} A temple-mandala of river, jungle, and sea—Ayokha blends what comes on the monsoon into something unmistakably its own, anchoring trade between Sihai, Nihon, and the West.
\end{tcolorbox}

\subsection*{Overview}
Ayokha measures borders in loyalty, harbors, and wind. A god-king sits at the center; influence radiates through vassal ports, priest-chieftains, and Sea-Lords who ride the seasonal breath of the world.

% ---------- Geography ----------
\subsection{Geography \& Waterworks (d12)}
\begin{longtable}{@{}p{0.8cm}p{12.2cm}@{}}
\toprule
\textbf{d12} & \textbf{Feature} \\
\midrule
1 & Sona River flood-ladders feeding terrace mosaics. \\
2 & Jade Coast mangroves hiding stilt markets. \\
3 & Tide gates carved with celestial calendars. \\
4 & Delta labyrinths navigated by drum code. \\
5 & Step-pyramid temples aligned to monsoon stars. \\
6 & Inland karst spires riddled with spirit caverns. \\
7 & Bronze gong towers relay harbor warnings. \\
8 & Batik guild towns with dye canals. \\
9 & River palaces on barge foundations. \\
10 & Jungle bridges woven of living roots. \\
11 & Pearl-silt basins taxed by shell weight. \\
12 & Cyclone refuges marked by mirrored tiles. \\
\bottomrule
\end{longtable}

% ---------- Mandala ----------
\subsection{Mandala of Power (d10)}
\begin{tabularx}{\linewidth}{@{}p{0.8cm}X@{}}
\toprule
\textbf{d10} & \textbf{Node} \\
\midrule
1 & \textbf{Devaraja} health omens steer policy. \\
2 & \textbf{Vassal Kings} bid for canal grants. \\
3 & \textbf{Priest-Chieftains} license jungle rites. \\
4 & \textbf{Sea-Lords} tithe for storm-writs and letters of marque. \\
5 & \textbf{Celestial Bureaucrats} set auspicious sailing days. \\
6 & \textbf{Astrologer Courts} arbitrate trade disputes by star. \\
7 & \textbf{Festival Charters} open tax-free weeks for pilgrims. \\
8 & \textbf{Harbor Syndics} standardize weights, seals, and scripts. \\
9 & \textbf{River Wardens} control floodgates and ferry rights. \\
10 & \textbf{Temple Auditors} inspect sacred treasuries and relic loans. \\
\bottomrule
\end{tabularx}

% ---------- Forces ----------
\subsection{Forces \& Fleet (d10)}
\begin{tabularx}{\linewidth}{@{}p{0.8cm}X@{}}
\toprule
\textbf{d10} & \textbf{Military Element} \\
\midrule
1 & Royal Guard in jeweled lamellar (ceremony and steel). \\
2 & War Elephants break lines; skirmishers screen flanks. \\
3 & Monsoon Junks with outrigger scouts. \\
4 & Sea-Lords’ boarding crews with javelins and hooks. \\
5 & Jungle archers versed in toxin governance. \\
6 & River mines disguised as lotus floats. \\
7 & Temple Engineers staff floodworks and siege ramps. \\
8 & Harbor Fire Corps drill for ship blazes. \\
9 & Astrolabe spotters for long-range semaphore. \\
10 & Neutral convoy flags sold to foreigners (limited protection). \\
\bottomrule
\end{tabularx}

% ---------- Culture ----------
\subsection{Culture \& Craft (d8)}
\begin{tabularx}{\linewidth}{@{}p{0.8cm}X@{}}
\toprule
\textbf{d8} & \textbf{Art / Practice} \\
\midrule
1 & Temple bas-reliefs retold with living actors at dusk. \\
2 & Dance-dramas certify guild oaths before gods and crowds. \\
3 & Textiles with gold thread denote tide-rights. \\
4 & Cuisine codified by wind-season: sour for \emph{south}, spice for \emph{west}. \\
5 & River weddings consecrate ferry monopolies. \\
6 & Pilgrim tattoos serve as toll passes. \\
7 & Merchant astrology almanacs predict price tides. \\
8 & Spirit ladders of bells appease nat in storm years. \\
\bottomrule
\end{tabularx}

% ---------- Hooks ----------
\subsection{Adventure Hooks (d8)}
\begin{tabularx}{\linewidth}{@{}p{0.8cm}X@{}}
\toprule
\textbf{d8} & \textbf{Hook} \\
\midrule
1 & The Silent Monsoon: find the rain’s offended spirit. \\
2 & The Sunken Temple: dive a shifting river maze. \\
3 & The Spice Captain’s Gambit: outrun rivals before the wind turns. \\
4 & Elephant Arsenal: hold a pass with pachyderm tactics. \\
5 & The Usurper’s Whisper: unmask a court poisoner. \\
6 & Floodgate Saboteurs threaten harvest tithes. \\
7 & Star-Taxed Market sparks riot over unlucky day. \\
8 & Ghost Barge bears unpaid dead to the capital. \\
\bottomrule
\end{tabularx}

% =========================================================
\section{Sihai — The Central Kingdom}
\label{sec:sihai}

\begin{tcolorbox}[title={\textbf{Sihai --- The Central Kingdom, The Ordered Land}},colback=white,colframe=black!70]
\textbf{Tagline.} An ancient, populous empire of law, letters, and logistics. Sihai projects influence by system and scale—yet continuously adapts through frontier contact and maritime exchange.
\end{tcolorbox}

\subsection*{Overview (Balanced Framing)}
Sihai is continuity given institution: examinations, canals, and corps that turn harvest into fleets and scrolls into strategy. Its power often persuades before it conquers—and learns as it persuades.

% ---------- Geography ----------
\subsection{Geography \& Works (d12)}
\begin{longtable}{@{}p{0.8cm}p{12.2cm}@{}}
\toprule
\textbf{d12} & \textbf{Feature} \\
\midrule
1 & Sihon River basin: levees, fish-ladders, terrace miles. \\
2 & Himadri passes with beacon towers and tea hostels. \\
3 & Grand canals moving rice, troops, and poets. \\
4 & Deep seaports with foreign quarters and coin-mints. \\
5 & Altan Plains forts facing steppe cavalry. \\
6 & Jade quarries sworn to ancestral cults. \\
7 & Paper towns where mills sing like rain. \\
8 & Salt pans governed by imperial stewards. \\
9 & “Sky-Spine” snowmelt reservoirs feeding clockwork gates. \\
10 & Silk districts with mulberry oaths. \\
11 & Porcelain kilns planned on geomantic lines. \\
12 & Maritime colonies ruled by mixed councils. \\
\bottomrule
\end{longtable}

% ---------- Mandate ----------
\subsection{Mandate \& Bureaucracy (d10)}
\begin{tabularx}{\linewidth}{@{}p{0.8cm}X@{}}
\toprule
\textbf{d10} & \textbf{Mechanism} \\
\midrule
1 & \textbf{Mandate of Heaven}: disasters prompt audits and reforms. \\
2 & \textbf{Exam Halls}: anonymous scripts, public rankings. \\
3 & \textbf{Censorate}: roaming inspectors with seal-breaking powers. \\
4 & \textbf{Salt-Iron Monopolies}: fund fleets and roads. \\
5 & \textbf{Canal Corps}: corvée mitigated by meal guarantees. \\
6 & \textbf{Frontier Commands}: civilian-military joint rule. \\
7 & \textbf{Ancestral Temples}: civic rites as tax calendars. \\
8 & \textbf{Merchant Boards}: issue credit scrip and dispute letters. \\
9 & \textbf{Scholar-Generals}: strategy treatises guide deployments. \\
10 & \textbf{Maritime Code}: convoy law for the Hintara and Nasan. \\
\bottomrule
\end{tabularx}

% ---------- Forces ----------
\subsection{Forces \& Methods (d10)}
\begin{tabularx}{\linewidth}{@{}p{0.8cm}X@{}}
\toprule
\textbf{d10} & \textbf{Military Element} \\
\midrule
1 & Mass infantry with crossbows and shield carts. \\
2 & Riverine marines trained for lock combat. \\
3 & Warrior Monks: shock teams and bodyguards. \\
4 & Siege math schools attached to armies. \\
5 & Cavalry auxiliaries recruited from steppe allies. \\
6 & Fire-lance arsenals in coastal forts. \\
7 & Navy junks as floating bastions with tower rigs. \\
8 & Intelligence couriers on poetry circuits. \\
9 & Logistics priests bless storehouses and ration ledgers. \\
10 & Border scouts mapping by star and water-chant. \\
\bottomrule
\end{tabularx}

% ---------- Culture ----------
\subsection{Society, Letters \& Table (d8)}
\begin{tabularx}{\linewidth}{@{}p{0.8cm}X@{}}
\toprule
\textbf{d8} & \textbf{Culture} \\
\midrule
1 & \textbf{Way of Harmony}: physicians/astrologers balance humors and seasons. \\
2 & \textbf{Path of Duty}: etiquette courts arbitrate insults into reparations. \\
3 & \textbf{Inner Reflection}: quiet halls open to commoners at dusk. \\
4 & Family registers entwine tax and ancestor rites. \\
5 & Tea guilds sponsor river poetry contests. \\
6 & Jade carvers hold “silent auctions” by gesture. \\
7 & Porcelain painters encode border news in patterns. \\
8 & Canal operas celebrate engineers as folk heroes. \\
\bottomrule
\end{tabularx}

% ---------- Trade ----------
\subsection{Trade \& Currency (d8)}
\begin{tabularx}{\linewidth}{@{}p{0.8cm}X@{}}
\toprule
\textbf{d8} & \textbf{Flow} \\
\midrule
1 & Exports: silk, porcelain, tea, paper; Imports: horses, gems, grain. \\
2 & Silk Route convoys insured by temple bonds. \\
3 & Coastal colonies mint bilingual coin. \\
4 & Price edicts issued by flood forecast. \\
5 & Joint ventures with Nihon smiths in neutral ports. \\
6 & Ayokhan spice quotas swap for canal timber rights. \\
7 & Examination stipends tied to harvest shares. \\
8 & Merchant \emph{huizi} (notes) accepted across three realms. \\
\bottomrule
\end{tabularx}

% ---------- Relations ----------
\subsection{Relations (Parity Lens, d8)}
\begin{tabularx}{\linewidth}{@{}p{0.8cm}X@{}}
\toprule
\textbf{d8} & \textbf{Current State} \\
\midrule
1 & With \textbf{Nihon}: naval exercises and joint anti-piracy patrols. \\
2 & With \textbf{Nihon}: exam scholars tour sword monasteries. \\
3 & With \textbf{Ayokha}: convoy calendars aligned to monsoon charts. \\
4 & With \textbf{Ayokha}: harbor law harmonization talks. \\
5 & With steppe states: hostage exchange for peace seasons. \\
6 & With westerners: limited concessions, strict tariff boards. \\
7 & Tri-realm craft symposium on steel and ceramics. \\
8 & Lighthouse standardization summit averts wrecks. \\
\bottomrule
\end{tabularx}

% ---------- Hooks ----------
\subsection{Adventure Hooks (d8)}
\begin{tabularx}{\linewidth}{@{}p{0.8cm}X@{}}
\toprule
\textbf{d8} & \textbf{Hook} \\
\midrule
1 & Failed Exam conspiracy reaches into the Censorate. \\
2 & Lost Legion standard haunts a Himadri pass. \\
3 & Tea and Poison at a coastal etiquette court. \\
4 & Monkey King’s Tomb maps hidden in opera costumes. \\
5 & Colony’s Cry: hold a mixed-port against a warlord’s fleet. \\
6 & Canal breach sabotage during flood season. \\
7 & Jade tax revolt mediated by ancestor-oracle. \\
8 & Maritime code forged—or broken—at a tri-realm summit. \\
\bottomrule
\end{tabularx}

\begin{tcolorbox}[title={\textbf{Crowd Mood Track [6]}},colback=white,colframe=black!70]
  \begin{tabularx}{\linewidth}{@{}lX@{}}
  \textbf{0--2} & Curious admiration; cheers, ritual silence, respectful onlookers. \\
  \textbf{3--4} & Partisan chants; thrown petals; accusations of bias; reporters swarm. \\
  \textbf{5} & Projectiles, panic, scuffles; Position --1 on all social checks in the ring. \\
  \textbf{6} & Riot or stampede: duel halted, collateral harm; Political Tensions +2. \\
  \end{tabularx}
  
  \medskip
  \textbf{Influencing the Crowd:}
  \begin{itemize}[leftmargin=1.5em]
    \item Respectful bow, ritual, poetic challenge: Crowd --1
    \item Public mercy or restraint: Crowd --1
    \item Brutality, insults, cheating, sacred bloodshed: Crowd +1
  \end{itemize}
  \end{tcolorbox}

  \begin{tcolorbox}[title={\textbf{Honor \& Pragmatism Tokens}},colback=white,colframe=black!70]
    Characters may earn:
    \begin{itemize}[leftmargin=1.5em]
      \item \textbf{Honor Tokens} for restraint, mercy, proper etiquette, protecting civilians.
      \item \textbf{Pragmatism Tokens} for decisive action, clever shortcuts, bold risk-taking.
    \end{itemize}
    
    \textbf{Spend 1 Token to:}
    \begin{itemize}[leftmargin=1.5em]
      \item Improve Position by +1 \emph{or}
      \item Cancel a single SB spend
    \end{itemize}
    
    If spent in public before the wrong audience, increase \textbf{Crowd Mood +1}.
    \end{tcolorbox}

    \begin{tcolorbox}[title={\textbf{Chain-of-Custody Evidence}},colback=white,colframe=black!70]
      Each item of physical evidence has a \textbf{Custody Clock [4]}. Write each handler’s name per segment.
      
      \textbf{If all segments filled:} evidence is clean and admissible. \\
      \textbf{If gaps exist:} conspirators challenge legitimacy; \textbf{DV +2} to present in court.
      
      PCs may seal evidence with shrine-wax. Breaking a seal without a judge present:
      \begin{itemize}
        \item Evidence becomes inadmissible
        \item Lose 1 Bell from the Seven Bell Court
      \end{itemize}
      \end{tcolorbox}

      \begin{tcolorbox}[title={\textbf{Chain-of-Custody Evidence}},colback=white,colframe=black!70]
        Each item of physical evidence has a \textbf{Custody Clock [4]}. Write each handler’s name per segment.
        
        \textbf{If all segments filled:} evidence is clean and admissible. \\
        \textbf{If gaps exist:} conspirators challenge legitimacy; \textbf{DV +2} to present in court.
        
        PCs may seal evidence with shrine-wax. Breaking a seal without a judge present:
        \begin{itemize}
          \item Evidence becomes inadmissible
          \item Lose 1 Bell from the Seven Bell Court
        \end{itemize}
        \end{tcolorbox}
        \begin{tcolorbox}[title={\textbf{The Seven Bell Court}},colback=white,colframe=black!70]
          A neutral Ayokhan judiciary that evaluates evidence, disputes, and rulings.
          PCs may earn \textbf{Bells} for:
          \begin{itemize}[leftmargin=1.5em]
            \item Respectful testimony or proper ritual
            \item Clear evidence with unbroken custody
            \item Public acts of mercy or restraint
          \end{itemize}
          
          \textbf{3 Bells} allows PCs to \emph{overturn} any one corrupt decision or expose a falsified verdict.
          Disrespect, falsified claims, or broken seals \emph{remove} Bells.
          \end{tcolorbox}

          \begin{tcolorbox}[title={\textbf{Monsoon Clock [8]}},colback=white,colframe=black!70]
            Tracks seasonal weather affecting investigation and duels.
            
            \textbf{1--2:} Dry --- stealth easier; outdoor surveillance +1d \\
            \textbf{3--4:} Squalls --- ranged attacks --1d; rooftops provide cover \\
            \textbf{5--6:} Peak Monsoon --- travel DV +1; evidence may wash away \\
            \textbf{7--8:} Clearing --- crowds return; ideal time for public confrontation
            \end{tcolorbox}

            \begin{tcolorbox}[title={\textbf{Truth Duel (Non-Lethal Resolution)}} ,colback=white,colframe=black!70]
              If both champions accept mediation, replace lethal combat with a \textbf{Three-Round Skills Duel}:
              
              \begin{itemize}[leftmargin=1.5em]
                \item \textbf{Form:} Display mastery (Melee, Acrobatics, Discipline)
                \item \textbf{Spirit:} Withstand insult or provocation (Resolve, Sway, Meditation)
                \item \textbf{Intent:} Defend one’s cause (Speechcraft, Honor, Logic)
              \end{itemize}
              
              Win \textbf{any 2 of 3} to avert war while preserving face and the legitimacy of the tournament.
              \end{tcolorbox}

              \begin{tcolorbox}[title={\textbf{Riot-Stop Setpiece}},colback=white,colframe=black!70]
                When \textbf{Crowd Mood = 5 or 6}, PCs may attempt a culturally resonant display:
                
                \begin{itemize}[leftmargin=1.5em]
                  \item Sihai: Tea ceremony of de-escalation
                  \item Nihon: Poetry exchange before blades
                  \item Ayokha: Bell procession invoking peace
                \end{itemize}
                
                \textbf{Presence + Insight vs. DV 4}:
                \begin{itemize}[leftmargin=1.5em]
                  \item \textit{Success:} Crowd Mood --2, PCs gain 1 Bell
                  \item \textit{Partial:} Crowd Mood --1, conspirators exploit distraction
                  \item \textit{Miss:} Riot worsens; Tournament Integrity +1
                \end{itemize}
                \end{tcolorbox}


                \begin{tcolorbox}[title={\textbf{Judges’ Scorecards}},colback=white,colframe=black!70]
                  Each formal duel is evaluated by an Ayokhan panel. After each major exchange or round:
                  \begin{center}
                  \begin{tabularx}{0.95\linewidth}{@{}l|c|c|c@{}}
                  \textbf{Category} & \textbf{0 Points} & \textbf{1 Point} & \textbf{2 Points} \\
                  \hline
                  \textbf{Form} & Sloppy, reckless & Competent & Exemplary technique, discipline \\
                  \textbf{Spirit} & Cruelty, dishonor & Restraint & Mercy, respect, cultural resonance \\
                  \textbf{Intent} & Selfish motive & Neutral duty & Protection of others, justice \\
                  \end{tabularx}
                  \end{center}
                  
                  At duel's end:
                  \begin{itemize}[leftmargin=1.5em]
                    \item Highest total earns the judges’ ruling.
                    \item A fighter may lose physically yet win socially and avert humiliation.
                    \item PCs may influence \textbf{one category per scene} via ritual, oath, or testimony.
                  \end{itemize}
                  
                  If judges are bribed or pressured, increase \textbf{Tournament Integrity +1}.
                  Exposing tampering grants PCs \textbf{+1 Bell} from the Seven Bell Court.
                  \end{tcolorbox}
                  \begin{center}
                    \begin{tcolorbox}[title={\textbf{Official Tournament Scorecard}},colback=white,colframe=black!70]
                    \textbf{Match:} \hfill \textbf{Date:} \\
                    \textbf{Competitor A:} \hfill \textbf{Competitor B:} \\[6pt]
                    
                    \begin{tabularx}{\linewidth}{@{}l|c|c|c|c|c@{}}
                     & \textbf{Round} & \textbf{Form} & \textbf{Spirit} & \textbf{Intent} & \textbf{Total} \\
                    \hline
                    \textbf{A} & 1 & \_\_ & \_\_ & \_\_ & \_\_ \\
                    \textbf{B} & 1 & \_\_ & \_\_ & \_\_ & \_\_ \\
                    \hline
                    \textbf{A} & 2 & \_\_ & \_\_ & \_\_ & \_\_ \\
                    \textbf{B} & 2 & \_\_ & \_\_ & \_\_ & \_\_ \\
                    \hline
                    \textbf{A} & 3 & \_\_ & \_\_ & \_\_ & \_\_ \\
                    \textbf{B} & 3 & \_\_ & \_\_ & \_\_ & \_\_ \\
                    \end{tabularx}
                    \vspace{6pt}
                    
                    \textbf{Final Totals:} \\
                    Competitor A: \_\_\_\_ \hfill Competitor B: \_\_\_\_
                    
                    \end{tcolorbox}
                    \end{center}

                    \section*{Appendix: Cultural Sensitivity \& Best Practices}
                    This adventure is inspired by the philosophies, aesthetics, and martial traditions of East and Southeast Asia. While the cultures presented here are \emph{fictional}, they draw from real-world analogues. This appendix provides guidance for respectful, conscientious presentation at the table.
                    
                    \subsection*{Core Design Principles}
                    \begin{itemize}
                        \item \textbf{Celebration, not exploitation.} Cultural elements are sources of wisdom, beauty, and complex philosophy—not exotic spectacle.
                        \item \textbf{Individuals act, cultures do not.} Corruption, conspiracy, or villainy are the choices of specific people, not entire nations.
                        \item \textbf{Multiple truths can coexist.} Sihai, Nihon, and Ayokha each possess internally coherent philosophies worthy of respect; none is presented as inferior or “primitive.”
                        \item \textbf{Differences produce story, not stereotypes.} Philosophical tension drives narrative without reducing cultures to caricature.
                    \end{itemize}
                    
                    \subsection*{Guidance for GMs}
                    When presenting culturally inspired characters, customs, or beliefs, prioritize:
                    \begin{itemize}
                        \item \textbf{Respectful tone} – No mocking accents, no comedic stereotypes.
                        \item \textbf{Cultural nuance} – Show internal disagreement: traditionalists, pragmatists, reformers, skeptics.
                        \item \textbf{Agency} – Every NPC (even commoners) has opinions, priorities, and personal stakes.
                        \item \textbf{Avoid monoliths} – No culture speaks with one voice; factions interpret tradition differently.
                    \end{itemize}
                    
                    A simple GM mantra:
                    \begin{center}
                    \emph{“Portray dignity, curiosity, and complexity.”}
                    \end{center}
                    
                    \subsection*{Language, Ritual, and Names}
                    Names, honorifics, and ceremonial acts should be:
                    \begin{itemize}
                        \item \textbf{Evocative but not appropriated} directly from real-language templates
                        \item \textbf{Consistent within each culture’s logic} (Sihai -> bureaucratic/formal; Nihon -> ritual precision; Ayokha -> syncretic sacred cyclicality)
                        \item \textbf{Used respectfully} without parody or exaggerated stereotypes
                    \end{itemize}
                    
                    \textbf{Optional Tool:} If players are unfamiliar with honorifics or rituals, treat them as invitations—\emph{explain their meaning rather than turning them into obstacles or “gotchas.”}
                    
                    \subsection*{Historical Inspiration}
                    While this adventure draws inspiration from:
                    \begin{itemize}
                        \item Imperial examination systems and ancestor veneration
                        \item Samurai ethics and ritualized martial arts
                        \item Monsoon-driven maritime trade networks
                        \item Temple cosmology and spirit veneration
                    \end{itemize}
                    
                    …all cultural elements are intentionally fictionalized to avoid misrepresenting any real-world tradition. The goal is homage, not imitation.
                    
                    \subsection*{Cultural Evolution in Play}
                    Cultures change. To reinforce authenticity:
                    \begin{itemize}
                        \item Allow characters to challenge tradition from within
                        \item Show generational differences in belief
                        \item Let cultural exchange influence customs and technology
                    \end{itemize}
                    
                    For example:
                    \begin{itemize}
                        \item A Sihai scholar fascinated by Ayokhan dance
                        \item A Nihon ronin studying Sihai philosophy
                        \item Ayokhan monks debating merchant influence on ritual purity
                    \end{itemize}
                    
                    Small details like these highlight cultural dynamism without disrespect.
                    
                    \subsection*{Player Engagement \& Consent}
                    If the table is unfamiliar with these cultural spaces:
                    \begin{itemize}
                        \item Offer a \textbf{Session Zero} to align tone and expectations
                        \item Use a \textbf{Lines/Veils/X-Card} framework for comfort
                        \item Allow players to ask cultural questions without embarrassment
                    \end{itemize}
                    
                    A respectful table should always feel safe saying:
                    \begin{center}
                    \emph{“I’m not sure—can you explain that part?”}
                    \end{center}
                    
                    \subsection*{Recommended Reading (Optional)}
                    Not required for play, but helpful for tone and cultural appreciation:
                    \begin{itemize}
                        \item \textbf{The Book of Five Rings} (strategy, discipline)
                        \item \textbf{Tao Te Ching} (balance, paradox, non-coercive ethics)
                        \item \textbf{The Ramayana} and \textbf{Mahabharata} (epic duty, sacred duty)
                        \item \textbf{The Romance of the Three Kingdoms} (statecraft, loyalty, consequence)
                    \end{itemize}
                    
                    These are not canonical to the setting—just inspiration for respectful tone.
                    
                    \subsection*{Final Note}
                    This module approaches cultural worldbuilding with humility and admiration. If at any point a player feels uncomfortable, confused, or concerned, the correct response is simple:
                    \begin{center}
                    \textbf{Pause, listen, adjust, continue with respect.}
                    \end{center}
                    
                    The goal is storytelling that honors the philosophical beauty, ritual depth, and human drama that inspired these cultures.

                    % ---------- Drop-in Section: 12 Ready-to-Run Adventures ----------
% Requires: \usepackage{tcolorbox,tabularx,xcolor}
% Optional: \usepackage{fontawesome5} for icons (commented below)

% ---------- Styling ----------
\definecolor{FeInk}{HTML}{1E293B}
\definecolor{FeAccent}{HTML}{3F6E8C}
\definecolor{FePale}{HTML}{EEF2F6}
\definecolor{FeWarn}{HTML}{8B1E3F}
\definecolor{FeGood}{HTML}{176B34}

\tcbset{
  fe/adventure/.style={
    enhanced,
    colback=FePale,
    colframe=FeAccent,
    coltitle=FeInk,
    fonttitle=\bfseries\large,
    boxrule=0.6pt,
    sharp corners,
    left=8pt,right=8pt,top=8pt,bottom=8pt,
    title filled,
    before skip=\baselineskip, after skip=\baselineskip
  },
  fe/callout/.style={
    enhanced,
    colback=white,
    colframe=FeAccent,
    sharp corners,
    boxrule=0.5pt,
    left=8pt,right=8pt,top=6pt,bottom=6pt,
    before skip=0.5\baselineskip, after skip=0.5\baselineskip
  }
}

% ---------- Micro-macros ----------
\newcommand{\Clock}[2]{\textbf{#1}~\([#2]\)}
\newcommand{\Tag}[1]{\fcolorbox{FeAccent}{FePale}{\footnotesize\textsf{#1}}}
\newcommand{\MiniStat}[2]{\textbf{#1:}~#2\par}
\newcommand{\Patron}[1]{\textit{#1}}

\section*{Twelve Drop-In Adventures}
\noindent Each module below includes a one-line premise, a distinctive mechanical twist, three scene-ready clocks, likely patrons, and two to three cinematic set-pieces. Use as one-shots or stitch into a season arc.

% ================================================================
\subsection*{1) Silk, Salt, and Shadows (Trade Heist Noir)}
\begin{tcolorbox}[fe/adventure,title={Silk, Salt, and Shadows}]
\MiniStat{Premise}{A Sihai customs vault hosts a contraband auction while \emph{three} rival crews run parallel heists. PCs can steal, stop, or triangulate.}
\MiniStat{Twist}{\Tag{Split-Screen Timer}: three parallel clocks advance at different rates during scenes.}
\MiniStat{Clocks}{\Clock{Security Alert}{6} \quad \Clock{Rival Crew Progress}{8} \quad \Clock{Crowd Panic}{4}}
\MiniStat{Likely Patrons}{\Patron{Maelstraeus} (bargains), \Patron{The Witness} (truth), \Patron{Ikasha} (shadow sects).}
\MiniStat{Set-Pieces}{Paper-lantern market stampede; bamboo catwalk chase; silent-code bidding war.}
\end{tcolorbox}

% ================================================================
\subsection*{2) The Emperor's Quiet (Political Thriller)}
\begin{tcolorbox}[fe/adventure,title={The Emperor's Quiet}]
\MiniStat{Premise}{Seven days of ritual silence: factions shove illegal edicts through sleepy bureaus. PCs decide precedent.}
\MiniStat{Twist}{\Tag{Mandate Ledger}: track \textbf{Legitimacy} vs. \textbf{Expediency}; end-state alters imperial doctrine.}
\MiniStat{Clocks}{\Clock{Legitimacy}{6} \quad \Clock{Conspiracy}{8} \quad \Clock{Public Unrest}{6}}
\MiniStat{Likely Patrons}{\Patron{The Mandate}, \Patron{Mykkiel} (writ), \Patron{Sacred Geometry} (order).}
\MiniStat{Set-Pieces}{Night ink-trial with living calligraphy; red-seal room puzzle; clerks’ strike sit-in.}
\end{tcolorbox}

% ================================================================
\subsection*{3) The Ronin's Winter (Duel Arc)}
\begin{tcolorbox}[fe/adventure,title={The Ronin's Winter}]
\MiniStat{Premise}{A famed Nihon duelist refuses battle until someone proves \emph{what victory is for}. PCs must heal the feud beneath the duel.}
\MiniStat{Twist}{\Tag{Seven Bell Court}: score \textbf{Form/Spirit/Intent} for social and martial exchanges.}
\MiniStat{Clocks}{\Clock{Blood Debt}{6} \quad \Clock{Clan Honor}{6} \quad \Clock{Snowmelt (Season)}{8}}
\MiniStat{Likely Patrons}{\Patron{Bushido Code}, \Patron{Gallows Bell/Death Poets}, \Patron{Inaea} (hearth).}
\MiniStat{Set-Pieces}{Tea ceremony as negotiation; moonlit bridge standoff; cutting the last snowfall.}
\end{tcolorbox}

% ================================================================
\subsection*{4) Monsoon of Knives (Navy Caper)}
\begin{tcolorbox}[fe/adventure,title={Monsoon of Knives}]
\MiniStat{Premise}{Ayokhan Sea-Lords race to convoy spices through pirates, reefs, and a saboteur navigator.}
\MiniStat{Twist}{\Tag{Weather Dice}: scene winds shift Position and tag zones (gust, squall, dead calm).}
\MiniStat{Clocks}{\Clock{Monsoon Window}{8} \quad \Clock{Pirate Coalition}{6} \quad \Clock{Hull Integrity (per ship)}{4}}
\MiniStat{Likely Patrons}{\Patron{Monsoon Lords/Raéyn}, \Patron{The Traveler}, \Patron{Gold Temples}.}
\MiniStat{Set-Pieces}{Reef-threading at dusk; rope-bridge boarding; storm-eye parley.}
\end{tcolorbox}

% ================================================================
\subsection*{5) The City That Forgets (Mystic Mystery)}
\begin{tcolorbox}[fe/adventure,title={The City That Forgets}]
\MiniStat{Premise}{Each dawn, the city loses one kind of memory—ledgers, vows, names. PCs hunt the mnemonic keystone.}
\MiniStat{Twist}{\Tag{Memory Tokens}: spend a memory for dice; restore truths to regain.}
\MiniStat{Clocks}{\Clock{Civic Amnesia}{10} \quad \Clock{Archivist Cult Influence}{6} \quad \Clock{Personal Loss (per PC)}{4}}
\MiniStat{Likely Patrons}{\Patron{The Witness}, \Patron{Varnek Karn}, \Patron{Clockwork Monad}.}
\MiniStat{Set-Pieces}{Train of nameless mourners; shifting-street map; court with missing crime.}
\end{tcolorbox}

% ================================================================
\subsection*{6) Festival of Ten Thousand Lanterns (Social Sandbox)}
\begin{tcolorbox}[fe/adventure,title={Festival of Ten Thousand Lanterns}]
\MiniStat{Premise}{A truce festival hides ten agendas across three delegations; lantern honors reroute reputations.}
\MiniStat{Twist}{\Tag{Lantern Placement}: each dedication flips a faction tag; tally modifies DCs and clocks.}
\MiniStat{Clocks}{\Clock{Truce Stability}{6} \quad \Clock{Scandal Heat}{6} \quad \Clock{Lantern Favor (per faction)}{4}}
\MiniStat{Likely Patrons}{\Patron{Temple Flames}, \Patron{Courtesan’s Guild/Mab}, \Patron{Ancestor Cults}.}
\MiniStat{Set-Pieces}{River of lights; poetry duel; masked midnight confession.}
\end{tcolorbox}

% ================================================================
\subsection*{7) The Jade Locust (Plague Investigation)}
\begin{tcolorbox}[fe/adventure,title={The Jade Locust}]
\MiniStat{Premise}{A “locust blessing” swells harvests—and bones. Miracle, weapon, or both?}
\MiniStat{Twist}{\Tag{Two Truth Tracks}: \textbf{Boon} vs. \textbf{Bane}; the one that fills locks the end-state.}
\MiniStat{Clocks}{\Clock{Proliferation}{8} \quad \Clock{Panic}{6} \quad \Clock{Quarantine Breach}{4}}
\MiniStat{Likely Patrons}{\Patron{Carrion King}, \Patron{Sacred Geometry}, \Patron{Justice Shrines}.}
\MiniStat{Set-Pieces}{Shrine that bleeds sap; singing field; rib-idol extraction.}
\end{tcolorbox}

% ================================================================
\subsection*{8) Ash at Sunrise (War-Elephant Tactics)}
\begin{tcolorbox}[fe/adventure,title={Ash at Sunrise}]
\MiniStat{Premise}{Hold a river fort with three half-trained elephants and conscripts.}
\MiniStat{Twist}{\Tag{Command Pips}: assign limited orders; each elephant tracks \textbf{Temperament}.}
\MiniStat{Clocks}{\Clock{Enemy Ladders}{8} \quad \Clock{Elephant Temperament (per beast)}{4} \quad \Clock{Civilians Evacuated}{6}}
\MiniStat{Likely Patrons}{\Patron{Sea-Lords}, \Patron{Threshold Guardians}, \Patron{Inaea}.}
\MiniStat{Set-Pieces}{Mud embankment gambit; fire-rafts by night; panic recovery drill.}
\end{tcolorbox}

% ================================================================
\subsection*{9) The Thousand-Step Court (Paradox Dungeon)}
\begin{tcolorbox}[fe/adventure,title={The Thousand-Step Court}]
\MiniStat{Premise}{A mountain temple demands a paradox per terrace; break your word, descend.}
\MiniStat{Twist}{\Tag{Declared Tenets}: players set vows as constraints—keep for boons, break to give GM SB.}
\MiniStat{Clocks}{\Clock{Pilgrim Resolve}{8} \quad \Clock{Stormfront}{6} \quad \Clock{Rival Pilgrims}{6}}
\MiniStat{Likely Patrons}{\Patron{Inner Reflection}, \Patron{Sacred Masons}, \Patron{Thunder-Speaker}.}
\MiniStat{Set-Pieces}{Hall with false echo; unseen bridge; lightning-struck bell.}
\end{tcolorbox}

% ================================================================
\subsection*{10) Black Powder, White Paper (Tech vs. Tradition)}
\begin{tcolorbox}[fe/adventure,title={Black Powder, White Paper}]
\MiniStat{Premise}{A joint foundry prototypes matchlocks; artisans vanish; a purist manifesto spreads.}
\MiniStat{Twist}{\Tag{Innovation \(\leftrightarrow\) Orthodoxy}: GM slides pressure to swing DCs, fallout, NPC lean.}
\MiniStat{Clocks}{\Clock{Sabotage}{6} \quad \Clock{Patent Theft}{6} \quad \Clock{Guild Schism}{8}}
\MiniStat{Likely Patrons}{\Patron{Mechanist Guilds/Clockwork Monad}, \Patron{Bushido Code}, \Patron{Magistrate Temples}.}
\MiniStat{Set-Pieces}{Proofing-ground blast; water-clock blueprint heist; duel at the drafting table.}
\end{tcolorbox}

% ================================================================
\subsection*{11) Court of Perfumed Knives (Palace Intrigue)}
\begin{tcolorbox}[fe/adventure,title={Court of Perfumed Knives}]
\MiniStat{Premise}{The Devaraja’s “favorite dancer” is three siblings rotating; someone frames them for treason.}
\MiniStat{Twist}{\Tag{Mask Exchanges}: PCs earn costumes/gesture lexicons to swap identities for Scenes.}
\MiniStat{Clocks}{\Clock{Palace Suspicion}{6} \quad \Clock{Sibling Trust}{4} \quad \Clock{Usurper’s Net}{8}}
\MiniStat{Likely Patrons}{\Patron{Dance Temples/Mab}, \Patron{Celestial Vow}, \Patron{Justice Shrines}.}
\MiniStat{Set-Pieces}{Mirror-maze rehearsal; incense truth-trial; rooftop sarabande.}
\end{tcolorbox}

% ================================================================
\subsection*{12) The House that Eats Oaths (Haunted Law)}
\begin{tcolorbox}[fe/adventure,title={The House that Eats Oaths}]
\MiniStat{Premise}{An abandoned magistracy devours broken promises; lies become walls.}
\MiniStat{Twist}{\Tag{Oathcraft}: spoken promises bind as scene tags—boon if kept, Harm if broken.}
\MiniStat{Clocks}{\Clock{House Hunger}{8} \quad \Clock{Case Dossier}{6} \quad \Clock{Personal Complicity}{4}}
\MiniStat{Likely Patrons}{\Patron{The Oath/Mandate}, \Patron{Mykkiel}, \Patron{Threshold Guardians}.}
\MiniStat{Set-Pieces}{Whispering files; judge-less courtroom; corridor of your last lie.}
\end{tcolorbox}

% ================================================================
\subsection*{Options \& Dials (Bolt-On Systems)}
\begin{tcolorbox}[fe/callout,title={Truth Duel}]
Replace any climax with a \textbf{Seven Bell Court} resolution: score \emph{Form/Spirit/Intent} across three exchanges. On a tie, highest \emph{Intent} wins.
\end{tcolorbox}

\begin{tcolorbox}[fe/callout,title={Patron Debt Slider}]
Each Patron Favor spent advances a hidden \Clock{Corruption}{6} that surfaces later as a hard choice. Reveal one segment whenever PCs gain a major boon.
\end{tcolorbox}

\begin{tcolorbox}[fe/callout,title={Cultural Edge Tokens}]
Earn 1 \emph{Edge} by honoring a local custom; cash for \(\!+\!1\)d or Position upshift once/scene. Disrespect flips 1 \emph{Edge} to GM SB.
\end{tcolorbox}

\begin{tcolorbox}[fe/callout,title={Dual Clocks: Civic \& Personal}]
Pair a \emph{civic} clock (unrest, famine, censorship) with a \emph{personal} clock (reputation, family, debt). Progress in one pressures the other; offer explicit trade-offs.
\end{tcolorbox}

\begin{tcolorbox}[fe/callout,title={Travel as Trial (Wayfinding Montage)}]
On journeys, each PC frames one \emph{hazard} and one \emph{grace}. Roll; on success, add a boon tag to the next scene; on miss, GM gains 1 SB and advances \Clock{Way\;Weariness}{4}.
\end{tcolorbox}

% ---------- End Section ----------
\end{document}