\documentclass[11pt]{article}
\usepackage[utf8]{inputenc}
\usepackage[T1]{fontenc}
\usepackage{geometry}
\usepackage{array}
\usepackage{fancyhdr}
\usepackage{titlesec}
\usepackage{enumitem}
\usepackage{tabularx}
\usepackage{booktabs}
\usepackage{tcolorbox}
\usepackage{multicol}
\usepackage{graphicx}
\usepackage{amssymb}
\usepackage{lipsum}

\geometry{a4paper, margin=1in}

\titleformat{\section}{\large\bfseries\scshape}{\thesection}{1em}{}
\titleformat{\subsection}{\bfseries}{\thesubsection}{1em}{}

\pagestyle{fancy}
\fancyhf{}
\lhead{Theona}
\rhead{The Ninth Bell}
\cfoot{\thepage}

\newtcolorbox{npcbox}[1]{
    colback=gray!5,
    colframe=gray!50,
    fonttitle=\bfseries,
    title=#1,
    boxrule=0.5pt
}

\newtcolorbox{locationbox}[1]{
    colback=blue!5,
    colframe=blue!50,
    fonttitle=\bfseries,
    title=#1,
    boxrule=0.5pt
}

\newtcolorbox{mechanicsbox}[1]{
    colback=green!5,
    colframe=green!50,
    fonttitle=\bfseries,
    title=#1,
    boxrule=0.5pt
}

\newtcolorbox{patronbox}[1]{
    colback=purple!5,
    colframe=purple!50,
    fonttitle=\bfseries,
    title=#1,
    boxrule=0.5pt
}

\begin{document}

\begin{center}
    \textbf{\LARGE The Ninth Bell} \\
    \vspace{0.2cm}
    \textbf{A Theona Adventure for Low-Fantasy and Creepy Elements} \\
    \vspace{0.1cm}
    \small A Fate's Edge Adventure for Seasoned (Tier II) Characters
\end{center}

\section*{Adventure Overview}
\begin{tabular}{@{}ll}
    \textbf{Setting:} & Theona \\
    \textbf{Theme:} & The weight of broken oaths, cultural taboos, the land remembers \\
    \textbf{Focus:} & Low-fantasy horror, social tension, cultural consequence \\
    \textbf{Combat:} & Minimal (the true threat cannot be fought directly) \\
    \textbf{Length:} & 2-3 Sessions \\
\end{tabular}

\section*{Premise \& Tone}
In Theona, where the land remembers what you forget, a cultural taboo has been broken. The Ninth Law states: "The ninth word is never spoken, the ninth path is hidden, and the ninth name is erased from history." A bell has begun ringing where no bell should ring, marking the countdown to the Green Host's arrival.

This is not a story about monsters but about consequences. The horror is in the inevitability of the ninth chime, the weight of the land's memory, and the knowledge that no one can outrun what the land has witnessed. The bell rings for the dead who were not given their due, and the ninth chime will come at dawn.

\section*{The Ninth Law}
Theona's most sacred cultural principle is the Ninth Law. It governs everything from how many words should be spoken in a vow to how many names should be listed on a memorial stone. The Ninth Law is not just a tradition—it is woven into the fabric of the land itself.

\begin{mechanicsbox}{The Ninth Law Manifestations}
\begin{itemize}
    \item \textbf{The Ninth Word:} Never spoken in a vow; if spoken, the vow is void and consequences follow
    \item \textbf{The Ninth Path:} Never taken; leads to places the land does not remember
    \item \textbf{The Ninth Name:} Erased from history; spoken only in the darkest moments
    \item \textbf{The Ninth Bell:} Never rung; its ringing means someone broke the Ninth Law
\end{itemize}
\end{mechanicsbox}

\section*{The Hook}
The adventure begins with one of these scenarios:
\begin{itemize}
    \item \textbf{The Ringing Bell:} Traveling through Theona's hills, you hear a bell ringing where no bell should be. The handprints in the snow lead to the bell but stop where no human could stand.
    \item \textbf{The Silent Village:} Arriving at a village that should be welcoming, you find the people huddled in their homes, refusing to speak of the bell that rings in the mist.
    \item \textbf{The Dying Scholar:} A dying traveler clutches your arm: "The ninth name... I spoke the ninth name... the bell is ringing for me..."
\end{itemize}

\section*{Key NPCs}

\begin{npcbox}{Kaelen, the Bell-Warden}
    \textbf{Role:} Keeper of the Silent Bell \\
    \textbf{Demeanor:} Haunted, exhausted, but resolute \\
    \textbf{Knowledge:} The history of the Ninth Law, the Green Host's nature \\
    \textbf{Complication:} His son spoke the ninth name of a dead man \\
    \textbf{Secret:} He has been hearing the bell for seven days, counting down to the ninth chime \\
    \textbf{Quote:} "The land remembers what we forget. The bell rings for what should have been."
\end{npcbox}

\begin{npcbox}{Elyra, the Stone-Mouth}
    \textbf{Role:} Village elder and law-keeper \\
    \textbf{Demeanor:} Stern, unyielding, but not unkind \\
    \textbf{Knowledge:} Cultural traditions, how to properly mourn the dead \\
    \textbf{Complication:} She knows who broke the Ninth Law but cannot speak their name \\
    \textbf{Secret:} Her silence is a punishment for not preventing the taboo \\
    \textbf{Quote:} "The ninth name is not spoken. Not by the living, not by the dead."
\end{npcbox}

\begin{npcbox}{The Green Host}
    \textbf{Type:} Manifestation of Cultural Law \\
    \textbf{Description:} Not a physical entity but a force of consequence \\
    \textbf{Appearance:} Only perceived as a cold in the air, a distortion in vision, a sound like wind through dead trees \\
    \textbf{Nature:} The land's enforcement mechanism for broken oaths \\
    \textbf{Weakness:} Cannot be fought; only appeased through proper cultural ritual \\
    \textbf{Effect:} The ninth chime will mark the moment when the land claims what was denied
\end{npcbox}

\section*{Locations}

\begin{locationbox}{The Silent Bell}
    \textbf{Description:} A bell hanging from a single iron post on a windswept hill. No rope, no striker—yet it rings. \\
    \textbf{Supernatural Features:}
    \begin{itemize}
        \item Handprints in the snow lead to the bell but end where no human could stand
        \item Each chime echoes one more time than it should
        \item The ground is unnaturally cold around the bell
    \end{itemize}
    \textbf{Cultural Significance:} This bell only rings when the Ninth Law is broken; each chime represents a life that was not properly mourned
\end{locationbox}

\begin{locationbox}{The Village of Stone-Silent}
    \textbf{Description:} A village where no doors are locked but no voices are heard \\
    \textbf{Supernatural Features:}
    \begin{itemize}
        \item Villagers communicate only through gestures and written notes
        \item Children are kept indoors; they point toward the hills and weep
        \item Every household has a ninth stone placed in a hidden corner
    \end{itemize}
    \textbf{Cultural Significance:} The village is observing the ritual silence that follows a broken Ninth Law
\end{locationbox}

\begin{locationbox}{The Unmarked Grave}
    \textbf{Description:} A fresh grave in a field, with no marker, no name \\
    \textbf{Supernatural Features:}
    \begin{itemize}
        \item The earth has been turned but no footprints surround it
        \item A ninth flower grows in the center of the grave
        \item The soil is unnaturally cold to the touch
    \end{itemize}
    \textbf{Cultural Significance:} This is the grave of the person whose ninth name was spoken; it was dug by the land itself
\end{locationbox}

\section*{The Clocks}

\begin{mechanicsbox}{The Bell's Countdown [9]}
The bell rings once every hour, counting down to dawn.
\begin{itemize}
    \item \textbf{1-3 Chimes:} First signs of disturbance; villagers become anxious
    \item \textbf{4-6 Chimes:} The land reacts; cold spots, distorted sounds
    \item \textbf{7-8 Chimes:} The Green Host is near; people feel watched, haunted
    \item \textbf{9 Chimes:} Dawn arrives; the Green Host claims its due
\end{itemize}
\end{mechanicsbox}

\begin{mechanicsbox}{The Ninth Law Broken}
\begin{itemize}
    \item \textbf{1:} Someone spoke the ninth name of a dead person
    \item \textbf{2:} The person who spoke it is known but unnamed
    \item \textbf{3:} The dead person's death was not properly mourned
    \item \textbf{4:} The dead person was denied a proper name
\end{itemize}
\end{mechanicsbox}

\begin{mechanicsbox}{The Land's Memory}
The land remembers what you forget. As the Ninth Law is broken, the land's memory becomes more active.
\begin{itemize}
    \item \textbf{1:} Small disturbances (cold spots, distorted sounds)
    \item \textbf{2:} Visual distortions (seeing things that aren't there)
    \item \textbf{3:} The land actively hinders the living (paths disappear, landmarks shift)
    \item \textbf{4:} The Green Host is manifesting
\end{itemize}
\end{mechanicsbox}

\section*{Core Mechanics}

\begin{mechanicsbox}{The Ninth Name}
The ninth name is the true name of the dead, never to be spoken by the living.
\begin{itemize}
    \item Speaking the ninth name breaks the Ninth Law
    \item The name is only known by those who properly mourned the dead
    \item The land remembers the ninth name even if no living person does
    \item The bell rings for every ninth name that is spoken
\end{itemize}
\end{mechanicsbox}

\begin{mechanicsbox}{The Green Host}
The Green Host is not a monster but the cultural consequence of breaking the Ninth Law.
\begin{itemize}
    \item \textbf{It cannot be fought:} It is the land's enforcement of cultural law
    \item \textbf{It cannot be hidden from:} The land knows where you are
    \item \textbf{It can only be appeased:} Through proper cultural ritual
    \item \textbf{It claims what was denied:} The ninth chime marks when the land takes what should have been given
\end{itemize}
\end{mechanicsbox}

\begin{mechanicsbox}{Cultural Rituals}
Proper cultural ritual can prevent or mitigate the Green Host's arrival.
\begin{itemize}
    \item \textbf{Naming the Dead:} Speaking all nine names properly (including the ninth)
    \item \textbf{The Silent Vigil:} A period of silence and mourning
    \item \textbf{The Stone Offering:} Placing nine stones in a specific pattern
    \item \textbf{The Ninth Path:} Taking the path that is never taken to make things right
\end{itemize}
\end{mechanicsbox}

\section*{Session Structure}

\subsection*{Session 1: The Bell Rings}
\begin{itemize}
    \item \textbf{The Hook:} Players encounter the ringing bell or its consequences
    \item \textbf{Investigation:} Learn about the Ninth Law, discover someone spoke the ninth name
    \item \textbf{Cultural Tension:} Experience the village's ritual silence and fear
    \item \textbf{First Clues:} Find the unmarked grave, learn about the person whose ninth name was spoken
    \item \textbf{First Chime:} Experience the bell's first unnatural echo
\end{itemize}

\subsection*{Session 2: The Land Remembers}
\begin{itemize}
    \item \textbf{Rising Tension:} The land's memory becomes more active (cold spots, visual distortions)
    \item \textbf{Cultural Barrier:} Villagers communicate only through gestures and written notes
    \item \textbf{The Seventh Chime:} The Green Host is near; people feel watched
    \item \textbf{Discovery:} Learn the truth about who spoke the ninth name and why
    \item \textbf{The Eighth Chime:} The moment of decision—what ritual will you perform?
\end{itemize}

\subsection*{Session 3: The Ninth Chime}
\begin{itemize}
    \item \textbf{Dawn Approaches:} The final hours before the ninth chime
    \item \textbf{The Ritual:} Perform the proper cultural ritual to appease the land
    \item \textbf{The Green Host:} The land's consequence manifests
    \item \textbf{The Ninth Chime:} The moment when the land claims what was denied
    \item \textbf{Resolution:} The land returns to balance, or the consequences continue
\end{itemize}

\section*{The Ritual of Appeasement}

\begin{mechanicsbox}{The Proper Ritual}
To prevent the Green Host from claiming its due, the following ritual must be performed:
\begin{itemize}
    \item \textbf{Step 1:} Speak all nine names of the dead person (including the ninth)
    \item \textbf{Step 2:} Perform the Silent Vigil from dusk to dawn
    \item \textbf{Step 3:} Place nine stones in the pattern of the Ninth Path
    \item \textbf{Step 4:} Take the Ninth Path (a specific, hidden path known only to the elders)
\end{itemize}
\end{mechanicsbox}

\begin{mechanicsbox}{Ritual Failure Consequences}
\begin{itemize}
    \item \textbf{1-2 failed steps:} The Green Host will claim one person instead of many
    \item \textbf{3-4 failed steps:} The Green Host claims multiple people; the bell continues to ring
    \item \textbf{Complete failure:} The Green Host claims the entire village; the bell rings eternally
\end{itemize}
\end{mechanicsbox}

\section*{Possible Resolutions}

\begin{mechanicsbox}{Resolution Paths}
\begin{itemize}
    \item \textbf{The Proper Ritual:} The land is appeased; the bell stops ringing. The village returns to normal, but the memory remains. PCs gain a cultural boon but also a responsibility to remember.
    
    \item \textbf{Partial Ritual:} The Green Host claims only the person who broke the law. The bell stops, but the land is less forgiving. The village will remember the PC's role, for better or worse.
    
    \item \textbf{Failed Ritual:} The Green Host claims multiple people. The bell stops, but the land's memory is wounded. The Ninth Law becomes stricter in this region; future PCs will face harsher consequences for breaking it.
    
    \item \textbf{The Ninth Path:} The PCs take the Ninth Path themselves, becoming guardians of the Ninth Law. They gain deep understanding of Theona's cultural traditions but are forever marked by the land.
\end{itemize}
\end{mechanicsbox}

\section*{GM Guidance}

\begin{mechanicsbox}{Creating Creepy Atmosphere}
\begin{itemize}
    \item Describe the cold in the air that has no source
    \item Have the bell ring at unexpected moments, even when no one is near it
    \item Use the villagers' silence to create tension (no doors, but no voices)
    \item Describe visual distortions: "The path seems to stretch unnaturally long"
    \item Emphasize the inevitability of the ninth chime: "Dawn is coming"
\end{itemize}
\end{mechanicsbox}

\begin{mechanicsbox}{Low-Fantasy Approach}
\begin{itemize}
    \item The horror comes from cultural consequence, not from magic
    \item Avoid explaining how the bell rings; the "how" is less important than the "why"
    \item The Green Host is not a monster to fight but a cultural truth made manifest
    \item The land itself is the active force, not any supernatural entity
    \item The true terror is in knowing that no one can outrun what the land remembers
\end{itemize}
\end{mechanicsbox}

\begin{mechanicsbox}{Pacing the Countdown}
\begin{itemize}
    \item The first three chimes should feel like curiosity
    \item The next three should create growing unease
    \item The seventh and eighth should be moments of dread
    \item The ninth chime must feel inevitable
\end{itemize}
\end{mechanicsbox}

\begin{mechanicsbox}{The Land as a Character}
\begin{itemize}
    \item The land has memory and consequence
    \item It enforces cultural traditions through natural phenomena
    \item It is neither good nor evil, but it is absolute
    \item The land remembers what you forget
\end{itemize}
\end{mechanicsbox}

\section*{Quick Start for PCs}

\begin{itemize}
    \item \textbf{Scholar:} Research the history of the Ninth Law, find the proper ritual
    \item \textbf{Cantor:} Use song to communicate with the land, or to break the ritual silence
    \item \textbf{Runekeeper:} Understand the weight of oaths and cultural agreements
    \item \textbf{Healer:} Tend to those affected by the land's memory, though the true cure is ritual
    \item \textbf{Wanderer:} Know the hidden paths, including possibly the Ninth Path
\end{itemize}

\begin{center}
    \textit{"The ninth word is never spoken, the ninth path is hidden, and the ninth name is erased from history."} \\
    \textit{"The land remembers what you forget."} \\
    \textit{— The Ninth Law of Theona}
\end{center}

\end{document}