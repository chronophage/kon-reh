\documentclass[11pt,letterpaper]{article}

% Language and Input Encoding
\usepackage[utf8]{inputenc}
\usepackage[T1]{fontenc}
\usepackage[english]{babel}

% Page Layout and Geometry
\usepackage[margin=1in]{geometry}

% Graphics and Colors (for minor visual elements if needed)
\usepackage{graphicx}
\usepackage{xcolor}

% Headers and Footers
\usepackage{fancyhdr}
\pagestyle{fancy}
\fancyhf{} % Clear all header and footer fields
\renewcommand{\headrulewidth}{0pt} % No header rule
\renewcommand{\footrulewidth}{0pt} % No footer rule
% Define a simple footer if desired (page number centered)
% \fancyfoot[C]{\thepage}

% Section Title Formatting
\usepackage{titlesec}
% Customize section titles (e.g., remove numbering, add space)
\titleformat{\section}{\large\bfseries}{}{0em}{}
\titleformat{\subsection}{\bfseries}{}{0em}{}
\titlespacing*{\section}{0pt}{\baselineskip}{\baselineskip}
\titlespacing*{\subsection}{0pt}{\baselineskip}{0.5\baselineskip}

% Paragraph and List Spacing
\usepackage{parskip} % Add space between paragraphs, remove indent
\setlength{\parindent}{0pt}
\setlength{\parskip}{0.5\baselineskip}

% Hyperlinks (for PDF, optional)
\usepackage{hyperref}
\hypersetup{
    colorlinks=true,
    linkcolor=black,
    filecolor=magenta,
    urlcolor=cyan,
    pdftitle={Carol of the Cursed Holly},
    pdfauthor={Fate's Edge One-Shot},
}

% Metadata
\title{\textbf{Carol of the Cursed Holly}\\ A Fate's Edge One-Shot Adventure}
\author{}
\date{}

\begin{document}

% Title Page
\maketitle
\thispagestyle{empty} % No page number on title page

% Table of Contents (optional, can be commented out)
% \newpage
% \tableofcontents
% \newpage

% --- Content Starts Here ---

\section{Premise}

The remote mountain village of Frosthollow, nestled in a valley perpetually shrouded in winter twilight, is preparing for its annual ``Feast of the Endless Night'' -- a solstice celebration meant to ward off the dark. However, the Cantor traditionally leading the festivities, Elara Nightweaver, has gone mad. Influenced by the Patron \textbf{Thrysos, King of Revels}, her carols now carry an infectious, compulsive joy that's driving the villagers into a dangerous, ecstatic frenzy. They work themselves to exhaustion decorating, feasting non-stop, and dancing. Worse, whispers suggest the \textbf{Pale Shepherd} (thresholds, guidance) is claiming souls drawn by this false revelry, and the \textbf{Silent Choir} (mercy, silence) seeks to end the cacophony permanently, perhaps by silencing the village entirely.

\section{Setting}

Frosthollow is a collection of timber-framed houses blanketed in snow, surrounding a central stone chapel and a large, open square where the Feast is held. The air is crisp, and the perpetual twilight casts long shadows. The village is isolated, a day's hard travel from the nearest settlement.

\section{Player Characters (PCs)}

Assume a mixed group of 3-4 players, ideally with varied skill sets (combat, magic, social, stealth). They might be:
\begin{itemize}
    \item A sellsword seeking shelter or a contract.
    \item A Runekeeper investigating strange magical disturbances.
    \item A local hunter or guide with deep knowledge of the area.
    \item An exiled minor noble or scholar fleeing something (or someone).
\end{itemize}

\section{Hook}

The PCs arrive in Frosthollow seeking shelter from a sudden, fierce winter storm or are drawn by a specific request (a bounty on the ``mad Cantor,'' a plea for help from a surviving villager who fled, a need for supplies, or perhaps they are investigators from a nearby town responding to reports of strange lights/sounds). Upon arrival (or shortly after), they witness the effects of the ``Cheer'': villagers caroling off-key with manic grins, decorating trees with their own hair or clothing, or dancing frantically until they collapse.

\section{Key NPCs}

\subsection*{Elara Nightweaver (Mad Cantor - Thrysos' Influence)}

Once beloved, now gaunt and wild-eyed, her voice carries an otherworldly resonance. She wears a crown of holly that seems to grow into her scalp. She believes she's bringing true joy and light to the world, but it's a joy that consumes.

\subsection*{Thrysos (Patron - Ecstasy \& Excess)}

Manifests subtly through Elara and the revelry. His influence makes resistance feel wrong, makes the revelry feel \emph{necessary}. His goal is to sustain and grow the revel until it consumes the village and perhaps spreads.

\subsection*{The Pale Shepherd (Patron - Thresholds \& Liminality)}

Drawn by the unnatural energy and the souls teetering on the brink due to exhaustion and madness. The Shepherd seeks to guide \emph{some} souls peacefully away, but also sees this as a potential ``corral'' for lost memories or unwanted truths (perhaps related to the village's past). Its presence might manifest as fleeting shadows, a lost lamb, or a sense of being watched by something benevolent yet distant.

\subsection*{The Silent Choir (Patron - Mercy \& Silence)}

Perceives the revelry as a cacophony of false joy masking underlying pain and desperation. The Choir seeks to impose silence, to end the suffering by ending the noise. This could manifest through a local priest who becomes obsessed with ``silencing the heresy'' or through direct supernatural effects like objects becoming impossible to speak near.

\subsection*{Greta Frostwhisper (Survivor/Villager)}

An elderly woman who hid in the chapel's bell tower when the madness began. She's terrified but sane, knows the village's history, and can warn the PCs. She believes the holly crown is the source.

\subsection*{Father Markus (Possibly Influenced)}

The village priest, struggling against the revelry. He might be helpful or an obstacle depending on which Patron influences him.

\section{Structure}

\subsection*{1. Introduction (Establishing the Situation)}

\begin{itemize}
    \item \textbf{Scene:} PCs arrive in Frosthollow during/just before the storm. They encounter the initial signs of madness (the revelry).
    \item \textbf{Key NPCs Introduced:} Elara (briefly, singing), Villagers (affected), Greta (if found early), Father Markus (conflicted).
    \item \textbf{Major Objective:} Understand the source of the madness and stop it before the Feast of the Endless Night (set for tonight/soon).
    \item \textbf{Story Beats Generated:} Initial confusion, witnessing the effects of the ``Cheer,'' potential minor conflict with frenzied villagers.
\end{itemize}

\subsection*{2. Development (Challenges and Investigation)}

\begin{itemize}
    \item \textbf{Challenge 1: Surviving the Revelry:} Simply navigating the village is difficult. PCs must resist the urge to join in (Resolve tests, perhaps gaining Fatigue or a ``Revelry'' Condition if they fail). Helping affected villagers without getting pulled in is a challenge.
    \item \textbf{Challenge 2: Investigating the Source:} PCs need to learn about Elara, the holly crown, and the strange influences. This involves:
    \begin{itemize}
        \item Talking to Greta (if found) for history/context.
        \item Investigating Elara's home/lair (the chapel or a decorated grove) for clues about Thrysos' influence.
        \item Dealing with Father Markus, who might be helpful or an obstacle.
        \item Possibly encountering manifestations of the Pale Shepherd or the Silent Choir.
    \end{itemize}
    \item \textbf{Challenge 3: Countering the Influences:} Direct action against the supernatural forces.
    \begin{itemize}
        \item Dealing with Thrysos: Disrupt revelry, confront Elara, break his hold (ritual, opposing magic).
        \item Dealing with the Pale Shepherd: Protect souls, negotiate, understand its intent.
        \item Dealing with the Silent Choir: Understand its motive, stop its agent, find peaceful resolution.
    \end{itemize}
    \item \textbf{Key Scenes:} The chaotic village square, Elara's lair (the chapel), Greta's hiding place (bell tower), confrontations with patrons/villagers.
\end{itemize}

\subsection*{3. Climax (Major Confrontation)}

\begin{itemize}
    \item \textbf{Scene:} The Feast of the Endless Night. Elara, at the height of her power, leads the final revel. Patron influences are strongest.
    \item \textbf{Objective:} Stop Elara and break the spell of the ``Cheer.'' This likely involves:
    \begin{itemize}
        \item A social/mental challenge to resist/counter the joy (Wits+Resolve vs. Thrysos).
        \item A potential physical/magical confrontation with Elara.
        \item A crucial act to destroy/remove the holly crown (ritual, magic, trickery).
        \item Managing Patron interactions (leveraging one against another?).
    \end{itemize}
\end{itemize}

\subsection*{4. Resolution (Consequences)}

\begin{itemize}
    \item \textbf{If Successful:} The ``Cheer'' is broken. Elara is freed/defeated. Villagers recover. Decide Patron fates and the village's future.
    \item \textbf{If Partial Success:} Revelry dampened, core problem remains. Setup for future threat.
    \item \textbf{If Unsuccessful:} Revelry consumes, Choir silences, Shepherd claims souls. Dark ending.
\end{itemize}

\section{GM Tools \& Dials}

\begin{itemize}
    \item \textbf{Patron Prominence:} Adjust how overtly the Patrons act. Thrysos should be most obvious. Shepherd and Choir can be subtle.
    \item \textbf{Revelry Mechanic:} Represent the compulsive ``Cheer'' with a clock (e.g., ``Village Hysteria [6]''). Actions feeding it advance it. Countering it slows/reduces it.
    \item \textbf{Temptation:} Make the revelry \emph{feel} good initially. PCs need rolls to resist joining. Offer minor benefits at the cost of advancing the Hysteria clock.
    \item \textbf{Environmental Hazards:} The winter storm, perpetual twilight, potential avalanches or structural damage.
    \item \textbf{Deck Usage:} Draw from Wilds/Dungeon generator for unexpected complications (hidden cellar, mad animal, structural damage).
\end{itemize}

\section{Conclusion}

This one-shot provides a mix of social investigation, potential combat (with frenzied villagers or the Cantor), magical problem-solving, and dealing with the complex, morally ambiguous influences of multiple Patrons, all wrapped in a wintry, folk-horror atmosphere.

\section*{Duel to the Death: Miniatures Optional Extension}
\addcontentsline{toc}{section}{Duel to the Death: Miniatures Optional Extension}

\vspace{0.5em}

\noindent
This module enhances the martial arts tournament scenario with optional miniatures-based combat.
It preserves Fate's Edge narrative focus while adding tactical depth, cultural nuance, and opportunities
for dramatic spectacle worthy of the Seven Bell Court. Groups may use only the sections they enjoy;
nothing here is required for play.

% =====================================================
% 1. PREAMBLE — WHY MINIATURES?
% =====================================================

\begin{tcolorbox}[title=The Purpose of the Arena]
In Sihai, mastery is precision.
In Nihon, mastery is lethality.
In Ayokha, mastery is spectacle.

When the world watches, every motion has meaning.

Miniatures combat is not merely about striking an opponent.
It is about honor, form, spirit, and intent. The Seven Bell Court judges every bout,
and the crowd shapes the battlefield through awe, fear, and fury.

Use minis when blows are more than violence---when the duel itself could shape nations.
\end{tcolorbox}

\vspace{1em}

% =====================================================
% 2. OPTIONS & DIALS
% =====================================================

\section*{Options \& Dials}

\begin{tcolorbox}
Choose your table's desired level of tactical complexity. All options preserve narrative focus.
\end{tcolorbox}

\begin{center}
\begin{tabularx}{\linewidth}{|X|X|X|X|}
\hline
\textbf{Play Style} & \textbf{Use Minis For} & \textbf{Skip Minis For} & \textbf{Recommended Rules} \\
\hline
Narrative Focused & Final duel only & Everything else &
Abstract zones; Seven Bell Scoring only \\ \hline
Tactical Light & Qualifiers, ambushes, riots &
Diplomatic scenes &
Movement + flanking only \\ \hline
Full Tactical & All major conflicts &
--- &
Facing, reach, terrain effects \\ \hline
\end{tabularx}
\end{center}

\vspace{1em}

\subsection*{Visual Tone Dials}

\begin{tabularx}{\linewidth}{|X|X|X|}
\hline
\textbf{Culture} & \textbf{Style} & \textbf{Mechanical Expression} \\
\hline
Sihai & Wuxia precision & Long leaps, redirect attacks, bonus to Form \\
\hline
Nihon & Lethal minimalism & Hidden weapons, stealth zones, bonus to Spirit under pressure \\
\hline
Ayokha & Ceremonial spectacle & Crowd interaction, environmental effects, bonus to Intent \\
\hline
\end{tabularx}

\vspace{1em}

% =====================================================
% 3. CINEMATIC MANEUVERS
% =====================================================

\section*{Cinematic Maneuvers}

\begin{tcolorbox}
Any combatant---PC or NPC---may use these maneuvers in miniature-based scenes.
Each one encourages motion, spectacle, and cultural style.
\end{tcolorbox}

\begin{tabularx}{\linewidth}{|>{\raggedright}p{3cm}|X|X|}
\hline
\textbf{Maneuver} & \textbf{Trigger} & \textbf{Effect} \\
\hline
Heaven-Shaking Leap & Start turn with full movement available & Jump 3 hexes ignoring terrain; +1d on first strike \\
\hline
Flow Like Silk & Use Withdraw instead of Attack & +2 Defense until next turn \\
\hline
Iron Lotus Counter & Enemy misses melee attack & Immediate counterattack at +1d \\
\hline
Tiger's Pounce & Charge 3+ hexes in a straight line & +1 Harm \emph{or} +2 Crowd Mood \\
\hline
Honor Bind & Spare helpless foe & +2 Spirit points on Seven Bell Scoring \\
\hline
Shadow Step (if stealthy) & Begin turn Hidden & Teleport 2 hexes to new cover; +1d next attack \\
\hline
\end{tabularx}

\vspace{1em}

% =====================================================
% 4. ARENA FEATURES
% =====================================================

\section*{The Seven Bell Arena}

\begin{tabularx}{\linewidth}{|X|X|}
\hline
\textbf{Arena Feature} & \textbf{Mechanical Effect} \\
\hline
Judges' Platform (elevated) & +1d to Notice; +1 Spirit if bowing before judges \\
\hline
Crowd Sections & Influence Crowd Mood clock (cheers, jeers) \\
\hline
Ceremonial Gates & Entering through proper gate grants +1 Intent \\
\hline
Temple Banners & On defense, gain +1 Form when fighting near banners \\
\hline
\end{tabularx}

\vspace{1em}

% =====================================================
% 5. CHAMPIONS (MINIS STATS)
% =====================================================

\section*{Champion Miniatures Profiles}

\subsection*{Master Li Wei (Sihai)}
Size Medium; Speed 4; Moves Close→Near in one action.

\begin{itemize}
  \item \textbf{Melee:} +3d (Elemental Strike), +2d (Unarmed)
  \item \textbf{Special:} Five Elements Flow (change element once/round),
  Celestial Balance (ignore first Fear each scene),
  Harmony Shield (+2 Armor vs. elemental attacks)
  \item \textbf{Weakness:} Emotionally Compromised when rivalry escalates
\end{itemize}

\subsection*{Shinobi Kage (Nihon)}
Size Medium; Speed 5; Can Dash as movement.

\begin{itemize}
  \item \textbf{Melee:} +3d (Kusarigama), +2d (Tanto)
  \item \textbf{Special:} Shadow Step (teleport 2 hexes once/scene),
  Multi-Weapon (swap with no penalty),
  Psychological Warfare (-1d to foes when Crowd Mood $\ge$ 3)
  \item \textbf{Weakness:} Exposed after Shadow Step
\end{itemize}

\vspace{1em}

% =====================================================
% 6. TERRAIN BONUSES
% =====================================================

\section*{Cultural Terrain Bonuses}

\begin{center}
\begin{tabular}{|c|c|c|c|}
\hline
\textbf{Terrain Feature} & \textbf{Sihai} & \textbf{Nihon} & \textbf{Ayokha} \\
\hline
Temple Banner & +1 Spirit & +1 Form & +1 Intent \\
\hline
Shadowed Corners & --- & +2 Stealth & +1 Spirit \\
\hline
Water Pools & +1 Form & --- & +2 Intent \\
\hline
Open Floor & +1 Intent & --- & -1 Spirit \\
\hline
\end{tabular}
\end{center}

\vspace{1em}

% =====================================================
% 7. SEVEN BELL COURT SCORING (IN MINIS PLAY)
% =====================================================

\section*{Seven Bell Court Scoring}

\begin{tabularx}{\linewidth}{|X|X|}
\hline
\textbf{Event} & \textbf{Seven Bell Score} \\
\hline
Exemplary Technique (perfect attack + ideal position) & +2 Form \\
\hline
Mercy Shown (decline killing blow) & +2 Spirit \\
\hline
Protect Innocents / Defend Honored Guests & +2 Intent \\
\hline
Shameful Excess (cruelty, humiliation) & -1 Spirit \\
\hline
Cowardice or Dishonor in view of crowd & -2 Intent \\
\hline
\end{tabularx}

\vspace{1em}

% =====================================================
% 8. CONSPIRACY FIGHT CLOCKS
% =====================================================

\section*{Conspiracy Fight Clocks}

\subsection*{Sabotage Clock [4]}
Each time a PC rolls a \emph{Miss} in combat:
\begin{itemize}
  \item collapsing scaffolds
  \item poisoned darts
  \item hidden explosives
\end{itemize}
When full: \textbf{arena catastrophe} and civilians endangered.

\subsection*{Crowd Panic Clock [6]}
Violence, dishonor, or chaos fill segments.
When full: stampede, riot, and diplomatic disaster.

\vspace{1em}

% =====================================================
% 9. SAMPLE MINIATURE ENCOUNTER
% =====================================================

\section*{Sample Encounter: The Dojo Defense}

\begin{tcolorbox}[title=Setup]
Scale: 12$\times$12 hexes.  
Saboteurs attempt to burn or sabotage the training grounds.
\end{tcolorbox}

\textbf{Objectives:}
\begin{itemize}
  \item Prevent further sabotage
  \item Capture conspirators alive
  \item Protect students and evidence
\end{itemize}

\textbf{Threats:}
\begin{itemize}
  \item Falling beams (DV 3 to avoid, Harm 2)
  \item Oil slicks (risk falling Prone)
  \item Saboteurs using shadows for +1d attacks
\end{itemize}

\textbf{Seven Bell Scoring:}
\begin{itemize}
  \item +2 Spirit: Rescue civilians
  \item +2 Form: Perfect defense with no casualties
  \item +2 Intent: Preserve tournament integrity
\end{itemize}

\vspace{1em}

% =====================================================
% 10. WHY MINIATURES MATTER
% =====================================================

\section*{Why Miniatures Matter}

\begin{tabularx}{\linewidth}{|X|X|}
\hline
\textbf{Without Minis} & \textbf{With Minis} \\
\hline
Duel is abstract & Duel becomes a wuxia set-piece \\
\hline
Crowd is background & Crowd affects Position, dice, politics \\
\hline
Champions feel similar & Cultural styles emerge in motion \\
\hline
Conspiracy is hidden & Sabotage is visible and explosive \\
\hline
PCs watch the duel & PCs intervene, protect innocents, sway judges \\
\hline
\end{tabularx}

\vspace{1em}

% =====================================================
% END
% =====================================================
\section{Eastern Patron Translation Table}
\label{sec:eastern-patrons}

\subsection{Re-Theming Western Patrons for Eastern Adventures}

This translation table adapts Western-style patrons to the spiritual, philosophical, and cultural frameworks of \textbf{Sihai}, \textbf{Nihon}, and \textbf{Ayokha}. Each Patron maintains their metaphysical identity and game mechanics, but their symbols, rites, and cultural expectations are re-expressed through Eastern idioms.

\begin{tcolorbox}[colback=black!2,colframe=black!40,title=\textbf{Design Philosophy}]
\begin{itemize}
    \item \textbf{Maintain Core Themes:} The Patron's nature and rites never change.
    \item \textbf{Cultural Reskinning:} Names, symbols, and rituals change to match the region.
    \item \textbf{Mechanical Integrity:} All game effects and Rites function identically.
    \item \textbf{Regional Interpretation:} The same Patron may be feared in one land and revered in another.
\end{itemize}
\end{tcolorbox}

%---------------------------------------------------------------
% Patron Translation Table
%---------------------------------------------------------------
\subsection{Translation Table}

\begin{longtable}{@{}p{3.3cm}p{3.1cm}p{3.1cm}p{3.2cm}p{3cm}@{}}
\toprule
\textbf{Western Patron} & \textbf{Sihai Equivalent} & \textbf{Nihon Equivalent} & \textbf{Ayokha Equivalent} & \textbf{Core Theme} \\
\midrule
The Oath & The Mandate & The Bushido Code & The Celestial Vow & Binding Promises \\
Sealed Gate & The Great Wall & The Barrier Kami & The Threshold Guardians & Boundaries / Closure \\
Raéyn & The Celestial Bureaucracy & The Tide Masters & The Monsoon Lords & Storms / Tides \\
Khemesh & The Sunken Palace & The Deep Kami & The Abyssal Nat & Abyssal Pressure \\
The Witness & The Imperial Historians & The Chroniclers & The Memory Keepers & Truth / Revelation \\
Mab & The Courtesan's Guild & The Geisha Houses & The Dance Temples & Glamour / Courts \\
Sacred Geometry & The Feng Shui Masters & The Architect Monks & The Sacred Masons & Perfect Forms \\
Clockwork Monad & The Canal Engineers & The Mechanist Guilds & The Water Clock Keepers & Mechanism / Process \\
Varnek Karn & The Ancestor Cults & The Death Shrines & The Bone Temples & Necromancy / Dominion \\
Nidhoggr & The Primordial Forests & The Ancient Kami & The Root Spirits & Deep Earth / Rot \\
The Traveler & The Silk Road Merchants & The Wandering Monks & The Trade Winds & Roads / Ways \\
Oath of Flame \& Light & The Celestial Court & The Shrine Keepers & The Temple Flames & Dawn / Vows \\
Carrion King & The Decomposition Cycle & The Scavenger Kami & The Rot Spirits & Renewal / Carrion \\
Gallows Bell & The Executioner's Code & The Death Poets & The Final Judgment & Doom / Last Rites \\
Old Man of the Black Forest & The Primal Spirits & The Wild Kami & The Untamed Nat & Primal Humanity \\
Ikasha & The Shadow Sects & The Ninja Clans & The Night Spirits & Shadow / Potential \\
Inaea & The Hearth Temples & The Family Shrines & The Ancestral Fires & Mercy / Hearth \\
Mykkiel & The Law Courts & The Magistrate Temples & The Justice Shrines & Judgment / Writ \\
Maelstraeus & The Merchant Lords & The Trading Houses & The Gold Temples & Infernal Bargaining \\
Livaea & The Seduction Arts & The Temptation Scrolls & The Desire Spirits & Temptation / Desire \\
Aliyah & The Cursed Saints & The Bound Kami & The Chained Spirits & Curses / Corruption \\
\bottomrule
\end{longtable}

%---------------------------------------------------------------
% Detailed Cultural Variants
%---------------------------------------------------------------

\subsection{Detailed Cultural Variations}

\subsubsection{The Oath $\rightarrow$ The Mandate / Bushido / Celestial Vow}

\paragraph{Sihai: The Mandate (\textit{Tianming})}
\begin{itemize}
    \item \textbf{Symbol:} Golden seal upon red lacquer.
    \item \textbf{Patron’s Gift:} \textit{Imperial Voice} — Gain +1d to Command when invoking official authority.
    \item \textbf{Favored Rites:} Oaths of service, binding decrees, magistrate contracts.
    \item \textbf{Corruption Tell:} Skin becomes papery and stamped with red seals; speech becomes legalistic, emotionless.
\end{itemize}

\paragraph{Nihon: The Bushido Code (\textit{Bushidō})}
\begin{itemize}
    \item \textbf{Symbol:} Paired katana and inkbrush.
    \item \textbf{Patron’s Gift:} \textit{Warrior’s Rectitude} — +1d Melee when defending honor or clan.
    \item \textbf{Favored Rites:} Duels, testimony, loyalty oaths, ritual confessions.
    \item \textbf{Corruption Tell:} User hallucinates dishonor everywhere; bleeding ink from pores.
\end{itemize}

\paragraph{Ayokha: The Celestial Vow}
\begin{itemize}
    \item \textbf{Symbol:} Conch-shell etched with sigils.
    \item \textbf{Patron’s Gift:} \textit{Heaven-Backed Speech} — +1d Sway when speaking a sacred vow.
    \item \textbf{Favored Rites:} Temple contracts, marriage oaths, diplomatic binding.
    \item \textbf{Corruption Tell:} Voice becomes too beautiful; others must Save or obey.
\end{itemize}

%---------------------------------------------------------------

\subsubsection{Sealed Gate $\rightarrow$ Great Wall / Barrier Kami / Threshold Guardians}

\paragraph{Sihai: The Great Wall}
\begin{itemize}
    \item \textbf{Symbol:} Miniature stone brick wrapped in red cord.
    \item \textbf{Patron’s Gift:} \textit{Bastion of Empire} — +1 Armor when protecting others.
    \item \textbf{Corruption Tell:} Masonry patterns appear beneath the skin; paranoia and xenophobia intensify.
\end{itemize}

\paragraph{Nihon: The Barrier Kami}
\begin{itemize}
    \item \textbf{Symbol:} Paper wards (ofuda) sealed in wax.
    \item \textbf{Patron’s Gift:} \textit{Purifying Seal} — +1d Arcana to repel spirits or demons.
    \item \textbf{Corruption Tell:} Locked joints; speech comes only in ritual language.
\end{itemize}

\paragraph{Ayokha: The Threshold Guardians}
\begin{itemize}
    \item \textbf{Symbol:} Carved jade key.
    \item \textbf{Patron’s Gift:} \textit{Gatekeeper’s Sight} — +1d Insight to read intentions.
    \item \textbf{Corruption Tell:} Eyes resemble padlocks; obsession with control escalates.
\end{itemize}

%---------------------------------------------------------------

\subsubsection{Raéyn $\rightarrow$ Celestial Bureaucracy / Tide Masters / Monsoon Lords}

\paragraph{Sihai: The Celestial Bureaucracy}
\begin{itemize}
    \item \textbf{Symbol:} Silver brush pen.
    \item \textbf{Patron’s Gift:} \textit{Storm Decree} — Once per scene, amplify or calm wind/rain.
    \item \textbf{Corruption Tell:} Paper birds swarm the character; storms whisper their name.
\end{itemize}

\paragraph{Nihon: The Tide Masters}
\begin{itemize}
    \item \textbf{Symbol:} Shell inlaid with gold.
    \item \textbf{Patron’s Gift:} \textit{Salt-Sense} — +1d Survival for navigation and sea prediction.
    \item \textbf{Corruption Tell:} Hair floats as if underwater; breath smells of seawater.
\end{itemize}

\paragraph{Ayokha: The Monsoon Lords}
\begin{itemize}
    \item \textbf{Symbol:} Knotted wind-chime of bone and pearl.
    \item \textbf{Patron’s Gift:} \textit{Wind-Reader} — +1d Notice to detect incoming threats.
    \item \textbf{Corruption Tell:} Skin cracks like parched earth; voice becomes thunderous.
\end{itemize}

%---------------------------------------------------------------
\subsection{Rivalries in Eastern Context}

\begin{longtable}{@{}p{3cm}p{3cm}p{3cm}p{3cm}@{}}
\toprule
\textbf{Patron} & \textbf{Sihai Rival} & \textbf{Nihon Rival} & \textbf{Ayokha Rival} \\
\midrule
The Mandate & Ancestor Cults & Death Poets & Abyssal Nat \\
The Bushido Code & Celestial Bureaucracy & Shadow Sects & Rot Spirits \\
The Celestial Vow & Merchant Lords & Ninja Clans & Justice Shrines \\
The Great Wall & Silk Road Merchants & Wandering Monks & Threshold Guardians \\
The Barrier Kami & Trade Winds & Ancient Kami & Night Spirits \\
\bottomrule
\end{longtable}

%---------------------------------------------------------------
\subsection{Corruption Themes by Region}

\paragraph{Sihai Corruption}
\begin{itemize}
    \item Loss of face, rigid legalism, obsession with hierarchy.
    \item Bureaucratic stamps or seals spread across the skin.
\end{itemize}

\paragraph{Nihon Corruption}
\begin{itemize}
    \item Honor becomes suicidal pride.
    \item Beautiful black rot spreads like ink beneath veins.
\end{itemize}

\paragraph{Ayokha Corruption}
\begin{itemize}
    \item Ritual without meaning; divine perfection eclipses empathy.
    \item Voice becomes too compelling; mortals obey without question.
\end{itemize}

%---------------------------------------------------------------
\subsection{Sample Patron Encounters (Eastern Expression)}

\paragraph{The Mandate (Sihai)}
A courier kneels, delivering a jade-sealed edict: investigate corruption. Refusing brings dishonor—and perhaps divine punishment.

\paragraph{The Bushido Code (Nihon)}
A formal challenge scroll appears at dawn. Resolve a feud between samurai houses. Failure stains everyone’s honor.

\paragraph{The Celestial Vow (Ayokha)}
Dream-message from a forgotten temple flame: restore worship—or the spirits will claim the shrine entirely.

%---------------------------------------------------------------
\subsection{Integration with \textit{Duel to the Death}}

\begin{itemize}
    \item \textbf{The Crimson Merchants’ Guild} = Sihai Merchant Lords / Nihon Trading Houses / Ayokha Gold Temples.
    \item \textbf{Master Li Wei} may serve the Celestial Bureaucracy or Temple Flames.
    \item \textbf{Shinobi Kage} may serve the Night Spirits or Shadow Sects.
\end{itemize}

%---------------------------------------------------------------
\subsection{Conclusion}

These cultural translations preserve mechanical identity while enriching play with meaningful, respectful cultural differences. A Patron remains a Patron—whether worshipped by monk, samurai, or spice-merchant.

\end{document}
