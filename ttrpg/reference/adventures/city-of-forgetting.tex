\documentclass[11pt]{article}

\usepackage[margin=1in]{geometry}
\usepackage{parskip}
\usepackage{hyperref}
\usepackage{xcolor}
\usepackage{tcolorbox}
\usepackage{enumitem}
\usepackage{booktabs}
\usepackage{tabularx}

\hypersetup{
  colorlinks=true,
  linkcolor=blue!40!black,
  urlcolor=blue!60!black
}

\newcommand{\Tag}[1]{\textsc{#1}}
\newcommand{\Term}[1]{\textbf{#1}}
\newcommand{\Place}[1]{\textit{#1}}
\newcommand{\NPC}[1]{\textsc{#1}}

\title{\vspace{-2em}\textbf{The City of Forgetting}\\[0.3em]
\large A Siege Adventure for Fate's Edge}
\author{}
\date{}

\begin{document}
\maketitle
\vspace{-2em}

\begin{center}
\textit{``Some cities are built on stone.\\
Heugen is built on the things people would kill to forget.''}
\end{center}

\tableofcontents
\newpage

%====================================================
\section{Introduction}
\label{sec:intro}

\subsection*{Premise}

The city of \Place{Heugen} stands where the \Term{Paladian River} meets the deserts of Galanina --- a bustling port of exiles, runaways, and those who have bargained away their shame.

Two centuries ago, \Term{Temple of Light} crusaders founded a fortified camp here, sworn to bring ``radiant order'' to the desert tribes. They failed catastrophically: their campaign collapsed into scandal, atrocity, and defeat. Their martyrs were denied, their narrative broken.

In the ruins of their faith, something answered.

They turned to \Term{Aveh}, the Veiled Patron of Release, who promised them life without the weight of their past. The camp became a haven for the disgraced. Wells were blessed, lotus-fields planted, and rites woven to \emph{soften} memory and \emph{unhook} old vows.

Now \Place{Heugen} is a thriving port-city of:
\begin{itemize}[leftmargin=*]
  \item exiled nobles and disowned heirs,
  \item deserters and oath-breakers,
  \item victims fleeing their abusers,
  \item witches and hedge-priests seeking peace.
\end{itemize}

Its \emph{lotus-tinctured wells} and \emph{forgetting rites} give it its nickname:

\begin{center}
  \textbf{The City of Forgetting.}
\end{center}

To followers of \Term{Mykkiel}, Angel of Law, this is an abomination: a wound in the fabric of justice where crimes go unpunished, oaths lie fallow, and names are deliberately blurred.

An army marches under Mykkiel's chained-halo banners. Their goal is simple: \emph{erase Heugen from the desert and restore memory to its fugitives}.

Can the adventurers save a city built on forgetting? Should they?

%----------------------------------------------------
\subsection*{Adventure Tier and Scope}

This adventure is tuned for:
\begin{itemize}[leftmargin=*]
  \item \textbf{Tier:} III--IV characters (experienced to veteran)
  \item \textbf{Tone:} Moral drama, siege horror, political intrigue, memory magic
  \item \textbf{Structure:} 4--6 sessions (or more, if expanded)
\end{itemize}

\noindent
It can be used as:
\begin{itemize}[leftmargin=*]
  \item a standalone mini-campaign centered on the siege of \Place{Heugen}, or
  \item a major arc in a longer Fate's Edge campaign.
\end{itemize}

%----------------------------------------------------
\subsection*{Themes and Safety}

This adventure touches on themes of:
\begin{itemize}[leftmargin=*]
  \item shame, trauma, and the desire to forget;
  \item justice vs.\ mercy;
  \item institutional abuse and religious zealotry;
  \item memory manipulation and loss of self.
\end{itemize}

Use safety tools appropriate for your group:
\begin{itemize}[leftmargin=*]
  \item Lines and Veils (what is off-limits vs.\ faded-to-black),
  \item X-Card or Script Change,
  \item open table check-ins during intense scenes.
\end{itemize}

\begin{tcolorbox}[title=\textbf{Player-Facing Summary},colback=blue!5!white,colframe=blue!60!black]
\textbf{Hook:} An army of fanatical law-bringers marches to erase the City of Forgetting, where lotus-blessed wells and secret rites let people leave their past behind. You must choose: defend the city, expose its buried truths, bargain between gods --- or burn it all down.

\textbf{Core Questions:}
\begin{itemize}
  \item Is forgetting mercy, cowardice, or both?
  \item Who deserves a second life?
  \item What happens if memory itself becomes a battlefield?
\end{itemize}
\end{tcolorbox}

%====================================================
\section{Heugen at a Glance}
\label{sec:heugen}

\subsection*{City Identity}

\Place{Heugen} is:
\begin{itemize}[leftmargin=*]
  \item a \textbf{river-port} on the Paladian, wedged between dunes and marsh;
  \item a \textbf{trade hub} for salt, glass, lotus-ink, and contraband;
  \item a \textbf{refuge city-state} ruled by a Council of Masks (names optional);
  \item a \textbf{ritual engine} built atop wells that \emph{store} memory instead of erasing it.
\end{itemize}

Lotus motifs are everywhere:
\begin{itemize}[leftmargin=*]
  \item stylised lotus sigils on doors (``I do not ask your past''),
  \item lotus petals in wine and tea,
  \item lotus pattern tilework in bathhouses and shrines,
  \item lotus-ink used for anonymous contracts.
\end{itemize}

\subsection*{District Sketches}

Use these briefly sketched districts as flexible locations for scenes and encounters.

\paragraph{Lotus Wells Quarter}
\begin{itemize}[leftmargin=*]
  \item Narrow lanes, shaded balconies, public wells watched by lotus-wardens.
  \item People come at dusk for ``forgetting draughts'' and confession-like rites.
  \item \textbf{Hook:} Fights often break out when someone \emph{refuses} to drink.
\end{itemize}

\paragraph{Riverfront and Docks}
\begin{itemize}[leftmargin=*]
  \item Flat-bottom barges, desert skiffs, and foreign galleys.
  \item Smugglers, runaway crews, desert tribesfolk, cloaked pilgrims.
  \item \textbf{Hook:} First place Mykkiel's vanguard blockades with river-chains.
\end{itemize}

\paragraph{Mask Market}
\begin{itemize}[leftmargin=*]
  \item Stalls selling masks, veils, false papers, and crafted new identities.
  \item Artisans who weave ``lotus-names'' --- public aliases bound to a charm.
  \item \textbf{Hook:} A PC may already have a lotus-name they forgot taking.
\end{itemize}

\paragraph{The Cistern-Depths}
\begin{itemize}[leftmargin=*]
  \item Flooded tunnels under the city where old water and older memories pool.
  \item Rumors of whispering echoes that call out your \emph{true name}.
  \item \textbf{Hook:} The heart of the final choice: keep the memories buried or release them.
\end{itemize}

%====================================================
\section{Major Factions}
\label{sec:factions}

\subsection*{Mykkiel's Host (\Tag{SIEGE} • \Tag{LAW})}

An army of auditors, paladins, and penitents under Mykkiel's chained-halo icon.

\begin{itemize}[leftmargin=*]
  \item \textbf{Goal:} Destroy Heugen, restore memories to its people, purge ``stolen law.''
  \item \textbf{Methods:} Siege, ritual audits, public ``naming trials'' where forgotten deeds are read aloud.
  \item \textbf{Face:} \NPC{Prelate-Seneschal Radan}, who genuinely believes he is saving souls by forcing them to remember.
\end{itemize}

\subsection*{The Lotus Council (\Tag{CITY} • \Tag{AMNESTY})}

Heugen's shifting government: masked elders, exiled generals, and lotus-priestesses.

\begin{itemize}[leftmargin=*]
  \item \textbf{Goal:} Preserve the city's function as a refuge and its pact with Aveh.
  \item \textbf{Tension:} Some want open war; others prefer bargaining or evacuation.
  \item \textbf{Face:} \NPC{Speaker-of-Petals}, a veiled leader whose identity may tie closely to a PC.
\end{itemize}

\subsection*{Aveh, the Veiled Patron (\Tag{MERCY} • \Tag{OBLIVION})}

A subtle, quiet patron whose miracles always \emph{cost} something.

\begin{itemize}[leftmargin=*]
  \item \textbf{Goal:} Protect those who cannot bear their past, blur the sharp edges of memory.
  \item \textbf{Secret:} The memories are not destroyed; they are stored in the cisterns beneath Heugen.
  \item \textbf{Manifestation:} Lotus petals falling in still rooms, the taste of river-water like half-remembered lullabies, dreams where your younger self negotiates with a shadow.
\end{itemize}

\subsection*{Optional: Local Witchcraft Circles (\Tag{COMMUNITY} • \Tag{THRESHOLD})}

Neighborhood circles of threshold-witches (see Threshold magic section if using that module):
\begin{itemize}[leftmargin=*]
  \item maintain warded doorways and street-corners;
  \item shelter victims of both Temple and Mykkiel abuses;
  \item can become powerful allies or targets during the siege.
\end{itemize}

%====================================================
\section{Adventure Structure}
\label{sec:structure}

Run this adventure in four broad acts. Each act can hold one or more sessions.

\begin{enumerate}[leftmargin=*]
  \item \textbf{Act I: The Approach} --- Arrival in Heugen as Mykkiel's host closes in.
  \item \textbf{Act II: Under Siege} --- Street-level missions, hard choices, and lotus bargains.
  \item \textbf{Act III: The Wells Below} --- Descent into the cistern-depths and memory vaults.
  \item \textbf{Act IV: Judgment of Heugen} --- A ritual, a siege climax, and the fate of memory itself.
\end{enumerate}

\begin{tcolorbox}[title=\textbf{Light Mechanic: Memory Pressure},colback=purple!5!white,colframe=purple!60!black]
Track a simple \textbf{Memory Pressure [4]} clock for the group.

Mark a segment when:
\begin{itemize}
  \item the PCs exploit the city's forgetting for easy gain;
  \item they use lotus-draughts to dodge emotional consequences;
  \item they bargain away someone else's memories;
  \item they openly side with total erasure or total exposure.
\end{itemize}

When it fills, introduce a major twist:
\begin{itemize}
  \item a PC's forgotten past resurfaces at the worst time;
  \item Aveh demands a collective price;
  \item Mykkiel's host gains a tactical advantage through revealed secrets.
\end{itemize}
\end{tcolorbox}

%====================================================
\section{Act I: The Approach}
\label{sec:act1}

\subsection*{Goals}

\begin{itemize}[leftmargin=*]
  \item Introduce Heugen, its lotus culture, and its contradictions.
  \item Present Mykkiel's Host as a looming but not-yet-inevitable threat.
  \item Tie at least one PC personally to the city's past or its wells.
\end{itemize}

\subsection*{Opening Hooks}

Choose one or mix several:

\begin{itemize}[leftmargin=*]
  \item \textbf{Lotus Contract:} The PCs are hired by the Lotus Council to act as outside negotiators or troubleshooters --- plausible deniability when dealing with Mykkiel's emissaries.
  \item \textbf{Pursued Exile:} One PC (with consent) is secretly from Heugen, having once drunk from the wells. Mykkiel's vanguard has their \emph{true name} on a writ.
  \item \textbf{Lost Pilgrims:} The PCs escort refugees toward Heugen as rumors of the oncoming crusade spread along the river.
\end{itemize}

\subsection*{Key Scenes}

\paragraph{1. The Lotus Gate}

Arrival at Heugen's river-gate:
\begin{itemize}[leftmargin=*]
  \item guards wear lotus sigils and mirrored veils;
  \item new arrivals may choose to \emph{drink} or not;
  \item a tense moment when an old enemy of a PC fails to recognise them.
\end{itemize}

\textbf{Choices:}
\begin{itemize}[leftmargin=*]
  \item Drinking gains short-term social safety (fewer questions) but risks memory complications later.
  \item Refusing marks PCs as ``anchored'' --- useful allies, but suspect to some locals.
\end{itemize}

\paragraph{2. Rumors of Chains}

In taverns, bathhouses, and markets:
\begin{itemize}[leftmargin=*]
  \item talk of an army marching under chained-halo banners;
  \item former Temple of Light veterans grow quiet or leave the room;
  \item a lotus-priestess asks the PCs what \emph{they} would do if their worst deed were read aloud in public.
\end{itemize}

\paragraph{3. The First Writ}

Mykkiel's Host sends a small delegation under a white flag:
\begin{itemize}[leftmargin=*]
  \item they deliver a \textbf{Writ of Abrogation}: Heugen is to be vacated, its wells sealed, its records surrendered;
  \item they publicly read one anonymous but horrifying past act tied to someone in the crowd --- proving their power;
  \item the PCs can intervene to protect, expose, or bargain.
\end{itemize}

\subsection*{Escalation Trigger}

End Act I when:
\begin{itemize}[leftmargin=*]
  \item the Lotus Council formally asks: \emph{``Will you help defend Heugen, or help us evacuate/appease Mykkiel's host?''}
  \item or when the PCs commit to a strong stance (defense, compromise, or betrayal).
\end{itemize}

\newpage

%====================================================
\section{Act II: Under Siege}
\label{sec:act2}

\subsection*{Act Tone and Function}

Act II shifts from rumor and positioning to \textbf{immediacy and consequence}.

Mykkiel’s Host begins siege operations:
\begin{itemize}[leftmargin=*]
  \item river blockades tighten,
  \item outlying wells collapse under holy bombardment,
  \item masked elders debate surrender behind closed doors,
  \item exiles panic as old identities begin resurfacing.
\end{itemize}

The PCs navigate moral territory:
\begin{itemize}[leftmargin=*]
  \item Do we defend the city that hides sins?
  \item Is forgetting a mercy --- or a refusal to heal?
  \item Should some memories be restored?
\end{itemize}

This act is about \textbf{choosing sides while discovering what those sides really are}.

\bigskip

%----------------------------------------------------
\subsection*{Key Siege Clocks}

Introduce these as the scaffolding of Act II:

\begin{itemize}[leftmargin=*]
  \item \Term{Siege Line [6]} --- when full, the Host breaches the riverfront.
  \item \Term{Council Schism [4]} --- when full, the Lotus Council fractures, spawning rival factions.
  \item \Term{Memory Surge [6]} --- when full, forgotten pasts erupt everywhere.
\end{itemize}

PC actions can fill or reduce these.

\begin{tcolorbox}[title=\textbf{Siege Pressure Effects},colback=red!5!white,colframe=red!70!black]
Whenever a clock ticks up:
\begin{itemize}
  \item increase ambient tension,
  \item show patrols, barricades, ration lines, and water-rites at the wells,
  \item reveal small tragedies --- families drinking to forget, or refusing and breaking.
\end{itemize}
\end{tcolorbox}

\bigskip

%----------------------------------------------------
\subsection*{Major Scenes and Missions}

Pick 2--4 for your group.

%=====================
\paragraph{1. The First Breach Attempt}
Mykkiel’s siege engineers deploy \emph{halo-chains} across the river.
\begin{itemize}[leftmargin=*]
  \item PCs respond to a call for aid by lotus-wardens.
  \item A ritual duel of \Term{Naming vs.\ Forgetting} plays out on bridges.
  \item PCs can:
  \begin{itemize}[leftmargin=*]
    \item sabotage chains,
    \item parley with zealots,
    \item or sacrifice a well’s power for temporary reprieve.
  \end{itemize}
\end{itemize}

%=====================
\paragraph{2. Masks Fall from Faces}
Someone important to the PCs is \emph{recognized} by a former foe or victim.
\begin{itemize}[leftmargin=*]
  \item Optional twist: the PC does not remember who this person is.
  \item PCs can mediate, deny, or demand a lotus working.
  \item This is where personal stakes crystallize.
\end{itemize}

%=====================
\paragraph{3. Council of Masks Fractures}
The Lotus Council calls the PCs into its chamber.
\begin{itemize}[leftmargin=*]
  \item one faction favors \textbf{surrender},
  \item one advocates \textbf{total erasure} (collapse the cistern-depths),
  \item one proposes \textbf{sanctioned remembrance}: return pasts ‒ selectively.
\end{itemize}

PCs may tip the scales.

\begin{tcolorbox}[title=\textbf{GM Prompt: Cost of Influence},colback=blue!5,colframe=blue!50]
Any PC-recommended course becomes a public expectation:
\begin{itemize}
  \item Success? They are praised as saviors.
  \item Failure? They are blamed as traitors or fools.
\end{itemize}
\end{tcolorbox}

%=====================
\paragraph{4. The Lotus Refineries Burn}
Mykkiel’s Host uses sanctified fire to destroy the lotus-drying towers.
\begin{itemize}[leftmargin=*]
  \item Players must save workers,
  \item or protect lotus knowledge,
  \item or steal fire-secrets from the attackers.
\end{itemize}

This is a set-piece battle with lotus petals burning in violet haze.

%=====================
\paragraph{5. The Memory Plague Emerges}
As siege pressure rises, memory instability hits Heugen:
\begin{itemize}[leftmargin=*]
  \item Crowds forget children,
  \item lovers forget grievances,
  \item zealots suddenly remember atrocities and crack.
\end{itemize}

The PCs must stabilize outbreaks with:
\begin{itemize}[leftmargin=*]
  \item lotus rites,
  \item threshold-magic,
  \item or brutal Mykkiel audits \emph{if they ally with the Host}.
\end{itemize}

\bigskip

%----------------------------------------------------
\subsection*{Act II NPC Gallery}

Use these living chess pieces:

\paragraph{\NPC{Speaker-of-Petals}}
Veiled negotiator for the Lotus Council; pragmatic, haunted, might be tied to a PC’s forgotten past.

\paragraph{\NPC{Marshal Cruciant}}
Siege tactician of Mykkiel’s Host. Secretly doubts his cause but fears consequence.

\paragraph{\NPC{The Aster Keeper}}
A witch who runs the lotus wells. Protective, motherly, and terrifying when the water is threatened.

\paragraph{\NPC{An Unnamed Child}}
Shows up repeatedly, follows the PCs, calls one of them \emph{``mother''} or \emph{``father''}.  
GM decides truth later.

\bigskip

%----------------------------------------------------
\subsection*{Faction Shifts}

Let faction alignment evolve:

\begin{itemize}[leftmargin=*]
  \item \textbf{PCs aid the city:} Mykkiel’s Host escalates earlier.
  \item \textbf{PCs strike a bargain:} Lotus Council schism advances.
  \item \textbf{PCs betray a well:} Aveh withdraws, manifest wrath, or offers new pacts.
\end{itemize}

\bigskip

%----------------------------------------------------
\subsection*{Act II Mechanical Twist: Lotus Tokens}

Introduce \textbf{Lotus Tokens}:

\begin{itemize}[leftmargin=*]
  \item PCs earn 1 token when they defend someone’s right to forget.
  \item PCs earn 1 token when they uncover painful hidden truth.
\end{itemize}

At 3 tokens, each PC receives a one-time benefit:
\begin{itemize}[leftmargin=*]
  \item \Term{Lotus Mercy:} Remove a consequence \emph{or}
  \item \Term{Lotus Revelation:} regain/restore a forgotten bond.
\end{itemize}

But each token spent fills \Term{Memory Pressure}.  
Lotus power always has a cost.

\bigskip

%----------------------------------------------------
\subsection*{Transition to Act III: The Wells Below}

Trigger Act III when:
\begin{itemize}[leftmargin=*]
  \item the PCs demand to know what lies beneath Heugen; or
  \item the lotus wells become unstable; or
  \item Mykkiel’s Host breaches the outer wall.
\end{itemize}

A lotus-priestess (or Aveh itself) says:

\begin{center}
  \emph{``To save a city of forgetting, you must remember what lies beneath it.''}
\end{center}

Lead the party into:
\begin{itemize}[leftmargin=*]
  \item the cistern-depths,
  \item vaults of stored memory,
  \item lotus-root catacombs,
  \item and perhaps the city’s buried origin.
\end{itemize}

\newpage

%====================================================
\section{Act III: The Wells Below}
\label{sec:act3}

\subsection*{Act Tone and Function}

Act III shifts the siege narrative inward.

The city’s surface burns, starves, and fractures --- but beneath Heugen lies the
\textbf{cistern-labyrinth} where Aveh’s blessing first took root.

This act is:
\begin{itemize}[leftmargin=*]
  \item \textbf{fantastic archaeology,}
  \item \textbf{metaphysical infiltration,}
  \item and \textbf{personal crisis}.
\end{itemize}

The PCs descend past lotus wells, aqueduct-veins, and memory-galleries into
\emph{the buried truth of Heugen.}

\bigskip

%----------------------------------------------------
\subsection*{Environment: The Lotus Catacombs}

The cisterns are not silent.

\paragraph{Features:}
\begin{itemize}[leftmargin=*]
  \item Black, still water canals reflecting unreal versions of the PCs.
  \item Wall carvings showing pilgrims drinking the water and erasing names.
  \item Petal-falls drifting like snow, forming echoes of forgotten events.
  \item Lotus roots like ivory ribs arching overhead and underfoot.
\end{itemize}

\paragraph{Travel Dynamics:}
Treat the descent as a hazardous exploration:
\begin{itemize}[leftmargin=*]
  \item \Term{Root Collapse [4]}~— instability increases with noise or violence.
  \item \Term{Memory Echo [4]}~— hallucinations tempt PCs to alter their past.
\end{itemize}

\bigskip

%----------------------------------------------------
\subsection*{Key Scenes and Revelations}

Choose or remix these for pacing:

%=====================
\paragraph{1. The Flooded Gallery}
Murals depict the founding crusaders kneeling not before light ---
but before a lotus-robed figure offering water.

\begin{itemize}[leftmargin=*]
  \item Perception / Lore tests reveal Aveh was once human.
  \item PCs may glimpse their own pasts in the water (GM selects).
\end{itemize}

%=====================
\paragraph{2. The Vault of Unwritten Sins}
A spiral chamber of suspended clay masks.

\begin{itemize}[leftmargin=*]
  \item Each mask corresponds to a forgotten wrong.
  \item Touching one summons the event for judgement:
  \begin{itemize}[leftmargin=*]
    \item resolve it peacefully,
    \item or accept a Mark of Memory (Condition),
    \item or shatter it --- evolving the \emph{Memory Surge}.
  \end{itemize}
\end{itemize}

%=====================
\paragraph{3. The Sacrarium of Names}
An underground tribunal where lotus-priestesses once weighed guilt.

\begin{itemize}[leftmargin=*]
  \item PCs learn the \textbf{Lotus Sacrament of Consent}:
    no one was forced to forget --- they \emph{asked} to.
  \item Reveal that some fled justice by choosing erasure.
\end{itemize}

This is the moral hinge of the campaign.

%=====================
\paragraph{4. Aveh’s Chrysalis Chamber}
A chrysalis of lotus fibre and ossified bone.

\begin{itemize}[leftmargin=*]
  \item Aveh’s presence speaks in divided voices:
    \emph{``I gave them mercy; they called it escape.''}
  \item PCs confront that \emph{forgetting is not healing}.
\end{itemize}

Aveh may offer:
\begin{itemize}[leftmargin=*]
  \item boon,
  \item bargain,
  \item indictment,
  \item or unsettling truth.
\end{itemize}

\bigskip

%====================================================
\section*{The Wells of Heugen: Memory Economy}

\index{Heugen!Wells}
\index{Memory Economy}
\index{Forgetting Mechanics}

Drinking the Waters removes burdens—at a price.

\subsection*{Memory Tokens}
Every PC begins with:
\begin{itemize}
  \item \textbf{Contacts} (1–3 people they matter to)
  \item \textbf{Histories} (coping scars, debts, vows)
  \item \textbf{Identity Pillars} (titles, roles, reputations)
\end{itemize}

Treat these as \emph{Memory Tokens}.

They may be \textbf{spent} during a scene to:
\begin{itemize}
  \item Gain \textbf{Position +1} for a roll
  \item Reduce a DV by 1
  \item Convert a Miss into a Partial
  \item Add \textbf{+2d} to a plea, duel, or rite
\end{itemize}

\paragraph{But:} When you spend one, it is \textbf{erased from the character}—
until restored by story or ritual.

\subsection*{City Cap and Power Scaling}
The \textbf{City Cap} (Heugen’s Fate) scales the benefit:
\begin{description}
  \item[Cap 1:] +1d or DV–1 only.
  \item[Cap 2:] as above plus heal a Condition.
  \item[Cap 3:] spend 1 Token to \textbf{cancel a scene clock tick}.
  \item[Cap 4+] spend 1 Token to \textbf{rewrite fictional positioning:}
  create an exit, expose a lie, force a truce.
\end{description}

When the crusade escalates, Cap increases. PCs can accelerate this—at terrible prices.

\subsection*{When Tokens Run Out}
If a PC reaches zero Tokens:
\textit{They become a Child of the Wells.}
\begin{itemize}
  \item Contacts forget them.
  \item Background and reputation dissolve.
  \item They can no longer use Aveh rites that require identity.
\end{itemize}

To reverse this, they must:
\begin{enumerate}
  \item publicly name what was forgotten, and
  \item be recognized by another as true.
\end{enumerate}

%====================================================
%----------------------------------------------------
\subsection*{Encounters and Adversaries}

\paragraph{Lotus Husk-Guardians}
Forgotten sentients whose identities dissolved.

\begin{itemize}[leftmargin=*]
  \item Attack with lotus-laced touch that forces Memory Echo tests.
  \item Can be calmed by naming rituals or Lotus Tokens.
\end{itemize}

\paragraph{Unremembered Priests}
They offer rites with unknown cost, unable to recall their own vows.

\begin{itemize}[leftmargin=*]
  \item Bargain for safe passage,
  \item reveal lore,
  \item or collapse into hostile guilt-spirals.
\end{itemize}

\paragraph{Siege Intrusion}
Mykkiel’s zealots breach into the cisterns.

\begin{itemize}[leftmargin=*]
  \item PCs may face exorcists who seek to \emph{purify the wells},
  \item triggering breach fights among root-chambers.
\end{itemize}

\bigskip

%----------------------------------------------------
\subsection*{Act III Special Mechanic: Memory Trials}

When a PC hears their name in the echoing water,
they must resist revising their history.

\begin{tcolorbox}[title=\textbf{Memory Trial},colback=purple!5!white,
colframe=purple!70!black]
Roll \emph{Spirit + Resolve} against DV determined by guilt-weight (3–6).

\begin{description}
  \item[Success:] You keep your memory; gain 1 Lotus Token.
  \item[Partial:] You lose a detail truthfully; GM may introduce a new past tie.
  \item[Miss:] You forget something important; the GM marks \Term{Memory Surge}.
\end{description}
\end{tcolorbox}

\bigskip

%----------------------------------------------------
\subsection*{NPC Gallery}

\paragraph{\NPC{The Quiet Mask}}
A silent judge of the Sacrarium; offers gestures, not words.

\paragraph{\NPC{The Aster Keeper Returned}}
Appears here as her unforgotten self; reveals contradictions.

\paragraph{\NPC{Aveh (Manifest / Soft-Spoken)}}
Neither god nor ghost --- an ideology that learned to speak.

\bigskip

%----------------------------------------------------
\subsection*{Player Choices and Moral Forks}

By the end of this act, PCs should decide:

\begin{itemize}[leftmargin=*]
  \item \textbf{Does Heugen deserve protection?}
  \item \textbf{Should forgetting be defended or dismantled?}
  \item \textbf{Do we ally with Aveh, Mykkiel, or neither?}
\end{itemize}

\bigskip

%----------------------------------------------------
\subsection*{Transition to Act IV: Revelation and Reckoning}

Trigger Act IV when:
\begin{itemize}[leftmargin=*]
  \item PCs commune directly with Aveh,
  \item or the Memory Surge reaches [6],
  \item or Mykkiel’s Host breaches into the catacombs.
\end{itemize}

Aveh whispers:

\begin{center}
  \emph{``You cannot defend a city of forgetting without remembering what was buried.''}
\end{center}

The way opens deeper --- into the lotus-root vault,
where the \emph{truth origin} of Heugen sleeps.

\newpage

%====================================================
\section{Act IV: Revelation and Reckoning}
\label{sec:act4}

\subsection*{Act Function}

Act IV reveals that the conflict is not between gods, but between
\emph{those who claim to serve them}.

The city stands at a contested threshold:
\begin{itemize}[leftmargin=*]
  \item Aveh’s agents preach belonging without judgment,
  \item Mykkiel’s agents demand truth without exemption,
  \item both insist the Patron wishes their version obeyed.
\end{itemize}

The PCs must navigate competing “mouthpieces” of power.

\bigskip

%----------------------------------------------------
\subsection*{The Root-Vault of Heugen}

\paragraph{Location and Symbolism}
An inverted lotus chamber beneath the dunes:
\begin{itemize}[leftmargin=*]
  \item petals formed of salt and glass,
  \item memory-water flows upward in thin rivulets,
  \item sleeping inscriptions vibrate when truths or lies are spoken.
\end{itemize}

This is where the founders swore the covenant — recorded, but never understood.

\bigskip

%----------------------------------------------------
\subsection*{Factional Arrival}

\paragraph{Aveh’s Delegates}
Three rival emissaries descend:
\begin{itemize}[leftmargin=*]
  \item The \textbf{Mask‐Bearer}, who promises absolution for a price,
  \item The \textbf{Unbound Herald}, who insists belonging means never naming harm,
  \item The \textbf{Archivist of Strays}, who believes belonging demands accountability.
\end{itemize}

Each claims:
\emph{Aveh wills this.}

\bigskip

\paragraph{Mykkiel’s Delegates}
Judicial expedition forces breach the vault:
\begin{itemize}[leftmargin=*]
  \item The \textbf{Sword‐Reader}, pronouncing sentence on the city,
  \item The \textbf{Law‐Singer}, who insists truth must be tempered by survival,
  \item The \textbf{Seal‐Bearer}, who would bind the city instead of destroy it.
\end{itemize}

Each claims:
\emph{Mykkiel decrees this.}

\bigskip

%----------------------------------------------------
\subsection*{Scene Structure}

Run Act IV in three escalating stages:

%=====================
\subsubsection*{1. The Memory Tribunal}

The Lotus stirs, projecting suppressed memories —
but no Patron speaks.

Instead:
\begin{itemize}[leftmargin=*]
  \item emissaries cite scripture differently,
  \item delegates try to recruit PCs as witnesses,
  \item old sins of the city echo without a clear voice.
\end{itemize}

Mechanic: PCs may \\textbf{declare a personal truth} to gain 1 \Term{Truth‐Favor}.

\bigskip

%=====================
\subsubsection*{2. The Schism Court}

The Root‐Vault becomes a court, but no judge presides.

\textbf{Play Priority:} the conflict is doctrinal, not divine.

\begin{itemize}[leftmargin=*]
  \item Aveh’s factions accuse each other of betraying the Patron’s mercy.
  \item Mykkiel’s factions declare each other heretics misreading the Law.
  \item PCs serve as arbitrators or agitators by choice or necessity.
\end{itemize}

The vault reacts not to theology, but to:
\begin{itemize}[leftmargin=*]
  \item stated truths,
  \item acknowledged harms,
  \item reconciled contradictions.
\end{itemize}

\bigskip

%=====================
\subsubsection*{3. The Lotus Fracture}

Competing doctrines destabilize the vault:

\begin{itemize}[leftmargin=*]
  \item memory‐water erupts into sandstorms,
  \item petals crack and bleed light,
  \item siege lines above buckle.
\end{itemize}

This is both literal and symbolic:
\textbf{no Patron intervenes} — their will is inscrutable,
and their agents are the ones tearing reality.

\bigskip

%----------------------------------------------------
\subsection*{Special Mechanic: Truth Arbitration}

\paragraph{Truth‐Favor (Campaign Currency)}

When a PC:
\begin{itemize}[leftmargin=*]
  \item names a painful truth,
  \item bridges opposing doctrines,
  \item refuses to permit doctrinal violence,
\end{itemize}

they earn 1 \Term{Truth‐Favor}.

Truth‐Favor can:
\begin{itemize}[leftmargin=*]
  \item reduce DV of persuasion in the vault,
  \item countermand a doctrinal “sentence,”
  \item prevent a Lotus collapse event.
\end{itemize}

\bigskip

%----------------------------------------------------
\subsection*{Win Conditions of Act IV}

Resolution does not come from:
\begin{itemize}[leftmargin=*]
  \item slaying emissaries,
  \item summoning Patrons,
  \item or proving one creed “right.”
\end{itemize}

Instead, the vault stabilizes when the PCs demonstrate:

\begin{itemize}[leftmargin=*]
  \item \textbf{Synthesis:} belonging acknowledged, truth accounted for.
  \item \textbf{Restraint:} neither delegations nor zealots decide alone.
  \item \textbf{Naming Harm:} the city’s past is spoken instead of swallowed.
\end{itemize}

Trigger Act V when:
\begin{itemize}[leftmargin=*]
  \item emissaries fall silent,
  \item inscriptions react to the PCs instead of agents,
  \item and an ancient voice — not a Patron, but the first founder — asks:

  \begin{center}
     \emph{``Whose will shall we live by — yours, or another’s?”}
  \end{center}
\end{itemize}

%====================================================
\subsection*{Doctrine Conflict Subsystem}
\label{subsec:doctrineconflict}
\index{Doctrine Conflict}

Emissaries of Patrons assert competing interpretations of divine will.
The PCs resolve these conflicts through narrative arbitration rather
than force.

\paragraph{Doctrine Claims}
When two factions assert incompatible readings of a Patron’s will,
create a \textbf{Doctrine Clash [4]} clock.

\begin{itemize}[leftmargin=*]
  \item Fill 1 tick when emissaries escalate rhetoric.
  \item Fill 1 tick when PCs stall, equivocate, or refuse to engage.
  \item Clear 1 tick when PCs articulate a reconciling truth or expose
        contradiction.
\end{itemize}

\paragraph{Clash Resolution}
When Doctrine Clash fills, the factions split violently:
\begin{itemize}[leftmargin=*]
  \item riots, purges, or zealotry erupt,
  \item siege lines fracture,
  \item and the Lotus Vault destabilizes.
\end{itemize}

When Doctrine Clash empties through reconciliation:
\begin{itemize}[leftmargin=*]
  \item emissaries fall silent,
  \item a third path becomes available,
  \item and the vault reacts to PCs as arbiters rather than witnesses.
\end{itemize}

\paragraph{PC Leverage: Truth-Favor}
When PCs:
\begin{itemize}[leftmargin=*]
  \item name painful truths,
  \item spotlight hypocrisy,
  \item or invoke shared stakes,
\end{itemize}
award \textbf{1 Truth-Favor}.  
Spend 1 Truth-Favor to:
\begin{itemize}[leftmargin=*]
  \item clear 1 tick on Doctrine Clash,
  \item or suppress zealot escalation for a scene.
\end{itemize}

\medskip
\noindent
This subsystem rewards narrative mediation while making doctrine tension
tangible without requiring divine intervention.

\subsection{Bestiary}

%====================================================
%====================================================
\begin{creature}[The Wells-Worm — Shame Devourer]
    \tag{BOSS} • \tag{MEMORY} • \tag{DESERT} • \tag{PSYCHIC}
    
    \signs{Sand bulges like lungs; whispering voices underfoot; lotus petals desiccate.}
    
    \etiquette{
    It never speaks first. If someone confesses truth, it recoils. If someone lies,
    the sand stirs.}
    
    %--------------------------- CORE PROFILE
    \subsubsection*{Profile}
    \begin{itemize}
      \item \textbf{Tier:} IV
      \item \textbf{Scale:} Huge (city threat)
      \item \textbf{Presence:} Oppressive, horizon-spanning
      \item \textbf{Position:} Starts Dominant
      \item \textbf{Resonance:} Gains strength from forgotten things
    \end{itemize}
    
    \vspace{6pt}
    
    %--------------------------- BOSSMOVES
    \subsubsection*{Boss Moves}
    \begin{itemize}
      \item \textbf{Consume Memory:} Force a PC to lose a Memory Token and heal a Phase.
      \item \textbf{Sand Maw Drag:} Pull a PC underground into a hallucinated past.
      \item \textbf{False Relief:} Offer to take away a burden—mark Corruption if accepted.
      \item \textbf{Echoed Guilt:} Speak in the voice of someone the PC failed.
      \item \textbf{Burrowing Surge:} Erupt under a group, scattering them across zones.
    \end{itemize}
    
    \vspace{6pt}
    
    %--------------------------- SB MENU
    \subsubsection*{SB Menu (GM Spending)}
    \begin{enumerate}
      \item Target loses 1 Memory Token (unnamed).
      \item The Worm becomes Hidden until provoked.
      \item A PC hears a forgotten voice—mark \emph{Shadowed}.
      \item Crowd or NPCs panic and flee or betray.
      \item A crusader claims “This proves the city must burn.”
    \end{enumerate}
    
    \vspace{6pt}
    
    %--------------------------- PHASES
    \subsection*{Phases of the Wells-Worm}
    The creature cannot be slain—only \emph{forced to disgorge what it has eaten}.
    
    It manifests in three escalating forms:
    
    \paragraph{Phase I: The Burrowed Serpent}
    \begin{itemize}
      \item \textbf{Attack:} Sand Lash (Fatigue 2 + displacement)
      \item \textbf{Trigger:} PCs reveal vulnerability → The Worm surfaces
      \item \textbf{Vulnerability:} Public naming of personal truths lowers its Position
    \end{itemize}
    
    \paragraph{Phase II: The Mirror of Shame}
    \begin{itemize}
      \item \textbf{Attack:} Manifest someone forgotten, forcing confession
      \item \textbf{Trigger:} First Phase defeated or someone Confesses
      \item \textbf{Special:} PCs must speak truth or lose 1 Memory Token
      \item \textbf{Vulnerability:} Collective witness (three or more PCs acknowledge truth)
    \end{itemize}
    
    \paragraph{Phase III: The Voice of Forgetting}
    \begin{itemize}
      \item \textbf{Attack:} The Offer — “Let it go, and walk free.” PCs may:
        \begin{description}
           \item[\emph{Accept:}] Heal a Condition but lose 2 Tokens.
           \item[\emph{Refuse:}] Take Fatigue +1 and become \emph{Shadowed}.
        \end{description}
      \item \textbf{Trigger:} First two shells unraveled
      \item \textbf{Vulnerability:} PCs must confess an unspoken truth — and be witnessed.
    \end{itemize}
    
    \vspace{6pt}
    
    %--------------------------- NEGOTIATION
    \subsection*{Confrontation \& Negotiation}
    Negotiation is possible—at grave risk.
    
    \begin{itemize}
      \item \textbf{DV:} 5 (Desperate)
      \item \textbf{Stake:} On Miss, the Worm eats the negotiator’s \emph{Identity Pillar}
      \item \textbf{Reward:} The Worm releases 1 eaten soul or memory into play
    \end{itemize}
    
    \vspace{6pt}
    
    %--------------------------- WEAKNESSES
    \subsection*{Weaknesses}
    \begin{itemize}
      \item \textbf{Confession.}
      \item \textbf{Collective witness.}
      \item \textbf{Name-speaking:} When three truths are named aloud, reduce its Scale.
    \end{itemize}
    
    \vspace{6pt}
    
    %--------------------------- DEFEAT CONDITIONS
    \subsection*{Defeat Conditions}
    The Worm cannot be slain — but it can be driven back if:
    
    \begin{enumerate}
      \item at least one truth is confessed publicly,
      \item three witnesses acknowledge it, and
      \item someone refuses its Offer at cost.
    \end{enumerate}
    
    If all are met, it \emph{vomits up} a reliquary:  
    a lotus kernel containing memory, soul, or city secret.
    
    \vspace{6pt}
    
    %--------------------------- ESCALATION
    \subsection*{Escalation: The City Under Judgment}
    Each time the Wells-Worm is confronted:
    
    \begin{itemize}
      \item Raise the City’s \textbf{Cap} by 1
      \item Shift one district toward:
        \begin{description}
          \item[\emph{Forgetfulness}] identity dissolves, or
          \item[\emph{Remembrance}] painful truth surfaces
        \end{description}
      \item Crusade Clocks advance: “Proof of Sin” or “Purging Justified”
    \end{itemize}
    
    %====================================================
    \end{creature}

%====================================================
\subsection*{Crusader Propaganda: The Wells-Worm as Proof of Sin}
\index{Propaganda!Crusader}
\index{Crusade!Moral Justification}

The adherents of Mykkiel do not see the Wells-Worm as a horror to
understand---they see it as \textbf{confirmation}.

To them, it is the \emph{inevitable consequence} of a city built on hidden
shame: a desert-born parasite that thrives on unconfessed truths.

Their commanders call it:

\begin{quote}
  ``A scourge fed by secrecy, and thus proof the people are unworthy to rule
  themselves.''
\end{quote}

\paragraph{Messaging Themes}
Crusader preachers and heralds twist sightings into a narrative:

\begin{itemize}
  \item \textbf{The Worm is Aveh's Shadow:} evidence that the city's pact is corrupt.
  \item \textbf{It Feeds on Forgetting:} therefore its existence reveals a population
    too weak to remember righteousness.
  \item \textbf{Only Purification Ends It:} siege, confession, subjugation---or flame.
\end{itemize}

\paragraph{Mechanics: Siege of Narrative}
Treat propaganda as a Weapon:

\begin{description}
  \item[Effect:] Each time the Worm surfaces publicly, the \emph{Crusader Moral Clock}
    ticks +1.
  \item[If full:] The Crusaders' resolve becomes unbreakable; parley DV +2 until the
    city performs a public Rite of Contrition.
\end{description}

\paragraph{Scene Effects}
When crusader agents spread this rhetoric:

\begin{itemize}
  \item \textbf{Civilians fracture:} some demand absolution, some denial.
  \item \textbf{City Guard morale wavers.}
  \item \textbf{Priests of Aveh grow divided:} is the Worm a test or a curse?
\end{itemize}

\paragraph{Rumor Engine}
Whenever the Worm acts or is glimpsed, roll 1d6:

\begin{tabular}{>{\bfseries}c|p{9cm}}
1 & ``The Worm speaks---the city hides something.'' \\
2 & ``Aveh cannot protect us; Mykkiel saw this coming.'' \\
3 & ``The Worm appeared at the well---it smells our guilt.'' \\
4 & ``The crusaders warned us; now they must save us.'' \\
5 & ``The walls are lies, the Worm is the truth.'' \\
6 & ``Only submission will end this horror.'' \\
\end{tabular}

Each rumor advances one clock:

\begin{itemize}
  \item \textbf{Crisis of Faith [4]}
  \item \textbf{Crusader Legitimacy [6]}
\end{itemize}

\paragraph{Counterplay: Truth as Rebuttal}
PCs or Aveh's agents reduce these Clocks by:

\begin{itemize}
  \item \textbf{Public Witness to truth}
  \item \textbf{Naming wounds and sins openly}
  \item \textbf{Leading communal rites of remembrance}
\end{itemize}

These reduce either Clock by 1--2, depending on crowd size and sincerity.

\paragraph{Design Note for GMs}
Do not overwrite player success with propaganda---\emph{use it as tension}.  
Every sighting of the Wells-Worm increases pressure to either:

\begin{itemize}
  \item \textbf{cleanse} the city, or
  \item \textbf{change} it
\end{itemize}

Both paths are victory conditions---but with radically different endings.

%====================================================

\begin{creature}[Mask-Bearer of Aveh]
    \tag{EMISSARY} • \tag{BELONGING-AS-FLIGHT} • \tag{MERCY-TWISTED}
    \signs{Veiled face; lacquered mask; voice soothing but hollow.}
    \moves{Offer absolution without cost; erase confessed memory; inflame denial of harm.}
    \strings{Mask-Pardon (ignore a Sin clock tick), Forgetting Rite (one shame erased), Sanctuary Under Mask (1 scene).}
    \weaknesses{Unmasked truths; survivors whose harm persists; any demand for accountability.}
    \end{creature}
    \begin{creature}[Archivist of Strays]
        \tag{EMISSARY} • \tag{BELONGING-AS-ACCOUNTING} • \tag{REPARATIVE}
        \signs{Featherless quill; scrolls that bleed ink; calm counting tone.}
        \moves{Name unpaid harms; price restitution; stabilize conflict through witnessing.}
        \strings{Record of Reckoning (DV -1 to mediation), Ledger of Mercy (reduce Harm by 1 if debt named), Witness Seal (1 scene truce).}
        \weaknesses{Silence; lies; crowds demanding clean absolution.}
        \end{creature}
        \begin{creature}[Unbound Herald]
            \tag{EMISSARY} • \tag{BELONGING-AS-DENIAL} • \tag{HONEYED}
            \signs{Open palms; too-smooth voice; always interrupts accountability.}
            \moves{Deflect blame; collapse nuance; turn reconciliation into accusation of division.}
            \strings{Sweetness of Silence (DV -1 to ignore conflict), “We Are One” Mantle (Position +1 vs truth-sayers).}
            \weaknesses{Named specifics; harmed voices; receipts of consequence.}
            \end{creature}

            \subsubsection{Mykkiel's Emissaries}

            \begin{creature}[Sword-Reader of Mykkiel]
                \tag{EMISSARY} • \tag{LAW-AS-BLADE} • \tag{ZEALOT}
                \signs{Razor bound in scripture; no shadow; verdicts spoken as cuts.}
                \moves{Pronounce sentence without context; escalate punishment; mark heresy clocks.}
                \strings{Sentence Seal (one person bound), Purity Look (truth forced), Blade of Measure (Position +1 to dominate parley).}
                \weaknesses{Ambiguity; testimony of nuance; conflicting doctrine.}
                \end{creature}

                \begin{creature}[Law-Singer of Mykkiel]
                    \tag{EMISSARY} • \tag{LAW-AS-TEMPERANCE} • \tag{PRAGMATIC}
                    \signs{Balancing tones; sand-shaped runes; voice like steel smoothed.}
                    \moves{Reinterpret verdicts into mercy; grant amnesty clauses; stabilize crowds.}
                    \strings{Harmonized Mandate (reduce Suffering clock by 1), Tempered Seal (truce scene), Binding Hymn (DV -1 to reconciliation).}
                    \weaknesses{Extremes—either faction freezes them out.}
                    \end{creature}

                    \begin{creature}[Seal-Bearer of Mykkiel]
                        \tag{EMISSARY} • \tag{LAW-AS-BINDING} • \tag{CARCERAL}
                        \signs{Lead scrolls; waxen sigils; keys for unseen locks.}
                        \moves{Bind a region under oath; demand confession for liberation; convert conflict into custody.}
                        \strings{Bond-Seal (bind a district), Confession Rite (DV -1 on interrogation), Safe Custody (protect a target while holding them).}
                        \weaknesses{Ungoverned spaces; refusals to name wrongs; impossible witnesses.}
                        \end{creature}

                        %====================================================
\section{Emissary Clash Encounter Tables}
\label{sec:emissaryclash}
\index{Encounters!Doctrine Conflicts}

Emissary clashes occur when rival interpretations of Patron will collide
in view of mortals. These tables generate scenes, pressure points, and
third-party complications.

Roll 1d6, or select for thematic fit.

\subsection{General Doctrine Flashpoints (1d6)}

\begin{enumerate}[leftmargin=*]
  \item \textbf{The Misheard Miracle:} A small supernatural sign is
        witnessed; two emissaries issue contradictory readings.
  \item \textbf{The Surviving Victim:} A harmed voice interrupts ritual
        narratives, demanding accountability or mercy.
  \item \textbf{The Unclaimed Dead:} A body becomes locus of dispute —
        punishment or absolution?
  \item \textbf{The Found Testament:} A fragment of doctrine surfaces
        but its ambiguity fuels escalation.
  \item \textbf{The Crowd Turns:} Onlookers adopt the most dramatic
        interpretation, driving panic and zealotry.
  \item \textbf{The Public Accusation:} One emissary denounces another
        as heretical; the clash polarizes bystanders.
\end{enumerate}

%----------------------------------------------------
\subsection{Aveh-Sided Challenges (1d6)}
\index{Aveh encounters}

\begin{enumerate}[leftmargin=*]
  \item \textbf{Confession Without Cost:} Mask-Bearers offer absolution
        so freely it destabilizes bonds; reckoners object.
  \item \textbf{The Forgotten Debt:} A victim remembers harm despite
        absolution, triggering repudiation of false mercy.
  \item \textbf{Ledger Crack:} Two Aveh emissaries argue whether
        belonging requires consequence or forgetting.
  \item \textbf{Sweet Denial:} The Unbound Herald interrupts accountability,
        accusing truth-seekers of betrayal.
  \item \textbf{Runaway Refugee:} A supplicant flees into Aveh’s faction —
        was it sanctuary or abduction?
  \item \textbf{Memory Leak:} A forgotten truth resurfaces; Aveh-aligned
        factions recoil, Mykkiel’s escalate.
\end{enumerate}

%----------------------------------------------------
\subsection{Mykkiel-Sided Challenges (1d6)}
\index{Mykkiel encounters}

\begin{enumerate}[leftmargin=*]
  \item \textbf{Sentence Without Context:} Sword-Readers demand punitive
        verdicts; Law-Singers struggle to temper them.
  \item \textbf{Confession Shaping Truth:} A Seal-Bearer demands public
        testimony that distorts narrative for custody.
  \item \textbf{Mercy on Trial:} A Law-Singer grants reprieve; zealots
        call it heresy.
  \item \textbf{Binding the Wrong One:} A Seal-Bearer shackles an
        innocent; Aveh emissaries weaponize outrage.
  \item \textbf{Purity Riot:} Punishers whip a crowd into zealotry;
        PCs must redirect or disperse.
  \item \textbf{Doctrine Split:} Two Mykkiel factions pronounce opposed
        interpretations; schism clock rises.
\end{enumerate}

%----------------------------------------------------
\subsection{Third Path Complications (1d6)}
\index{Third Path encounters}

These scenes appear when PCs clear \emph{Doctrine Clash}
(\S\ref{subsec:doctrineconflict}) or assert reconciliatory truths.

\begin{enumerate}[leftmargin=*]
  \item \textbf{Reconciliation Vision:} A supernatural sign points to
        shared stakes — emissaries hesitate.
  \item \textbf{The Forgotten Clause:} A doctrinal artefact surfaces
        showing both sides once shared a covenant.
  \item \textbf{The Returning Witness:} Someone harmed by both sides
        demands a new framework, not victory.
  \item \textbf{The Lotus Insight:} PCs glimpse a transcendent reading
        — emissaries falter, crowd listens.
  \item \textbf{The Silent Audience:} A supernatural hush forces emissaries
        to hear, not speak.
  \item \textbf{The Ledger Tears:} A metaphysical record splits — PCs
        can rewrite doctrine or let chaos rewrite it.
\end{enumerate}

%----------------------------------------------------
\subsection{Doctrine Escalation Indicator}

Whenever three emissary scenes occur without reconciliation,
increase the \textbf{Doctrine Clash} clock by 1.

Whenever a Third Path scene resolves peacefully, reduce the clock by 1.

%====================================================
\section{Talents and Rites of Mykkiel and Aveh}
\label{sec:patronTalentsAvehMykkiel}

This section provides thematic Talents and usable Rites for two opposed
but intertwined Patrons: \textbf{Mykkiel} (Law, Form, Judgement) and
\textbf{Aveh} (Forgetfulness, Refuge, Unmaking of Shame). These entries
reflect mortal interpretation — Patrons do not act directly.
\index{Patrons!Mykkiel}
\index{Patrons!Aveh}
\index{Rites!Mykkiel}
\index{Rites!Aveh}
%====================================================
\section{Dual-Faith Conversion Mechanics}
\label{sec:dualFaith}
\index{Faith!dual}
\index{Conversion}
\index{Oaths!religious}

Some characters walk between Patrons. This subsystem enables dramatic,
mechanical representation of dual-faith loyalty without complex bookkeeping.

Patrons possess \emph{want and will but no agency}; their followers act in
their names, often at cross-purposes.

%----------------------------------------------------
\subsection{The Devotion Spectrum}
\index{Devotion Spectrum}

Each character may track faith tension on a single spectrum:

\[
-3 \ (\text{Patron A extreme}) \quad
0 \ (\text{Balanced}) \quad
+3 \ (\text{Patron B extreme})
\]

\noindent
Only major revelation, vow, or rite scene shifts the spectrum.

\paragraph*{Mechanical Effects by Position}
\begin{description}
  \item[-2 or lower:] Gain Position +1 when acting in Patron A’s ethos,
  but Position -1 when acting under Patron B’s ethos.
  \item[0:] Gain +1d when mediating faith conflicts, but suffer DV +1 on
  either side when pressed to choose.
  \item[+2 or higher:] Gain Position +1 under Patron B’s ethos, but
  Position -1 under Patron A’s.
\end{description}

%----------------------------------------------------
\subsection{Faith Conflict Scenes}
\index{Scenes!faith conflict}

When a scene revolves around doctrinal choice, the GM places the
character at Controlled, Desperate, or Dominant based on:

\begin{itemize}
  \item which Patron’s ethos the action aligns with,
  \item their current spectrum position,
  \item witness pressure (emissaries, crowds, oathkeepers).
\end{itemize}

%----------------------------------------------------
\subsection{Dual-Oath Resolution}
\index{Oaths!dual}

If a character simultaneously holds vows from two Patrons:

\begin{itemize}
  \item When rolling under either vow, any \textbf{1s} trigger a choice:
  \begin{enumerate}
    \item Offend One Faith --- mark \emph{Spiritual Conflict} (Condition)
    with that Patron; or
    \item Delay Judgment --- increase the Devotion Spectrum by 1 toward
    indecision (toward 0).
  \end{enumerate}
\end{itemize}

\paragraph*{Resolving Spiritual Conflict}
A dramatic scene is required:

\begin{itemize}
  \item \textbf{Public prioritization} of one vow over another, or
  \item \textbf{Syncretic declaration} recognized by witnesses.
\end{itemize}

Clearing the Condition restores normal action.

%----------------------------------------------------
\subsection{Synthesis Tokens}
\index{Tokens!synthesis}

When a character successfully blends rites, customs, or mediates
opposed faithful, they gain a \textbf{Synthesis Token}.

Each token may be spent to:

\begin{itemize}
  \item improve Position +1 on a faith-based roll,
  \item gain +1d on parley between opposed factions,
  \item establish \emph{temporary sanctuary} in contested religious space.
\end{itemize}

Synthesis Tokens reset after significant betrayal or revelation.

%----------------------------------------------------
\subsection{Faith Venue Tags}
\index{Tags!venue}

Scenes may be marked with religious tags:

\begin{description}
  \item[Shrine-Favored:] Those aligned to Patron A gain Position +1;
  Patron B aligned suffer Position -1.
  \item[Temple-Favored:] Reverse of Shrine-Favored.
  \item[Contested Ground:] Both sides Position -1 until someone bridges
  the divide (GM may spend 1 SB to create opportunity).
\end{description}

Venue tags shift only after decisive events or rites.

%----------------------------------------------------
\subsection{Patron Ambassadors}
\index{Talents!Ambassador}

Characters who consistently bridge opposed faiths may earn:

\begin{description}
  \item[Ambassador Status:] Once per session, convert a faith-based
  Position penalty into a bonus.
\end{description}

Ambassador Status is lost if the character publicly betrays one faith.

%----------------------------------------------------
\subsection{Ritual Contamination}
\index{Contamination}

Performing a Patron’s rite in unfavored ground marks a
\textbf{Contamination Token}:

\begin{itemize}
  \item suffers -1d on faith rolls until purified, forgiven, or confessed.
  \item may be \emph{converted to XP} (1 token = 1 XP toward faith talent).
\end{itemize}

%----------------------------------------------------
\subsection{GM Guidance}
\index{Faith!GM guidance}

\begin{itemize}
  \item Faith tension works best when visible to NPCs.
  \item Place opportunity in scenes for syncretic declaration.
  \item Let emissaries disagree over interpretation — Patrons do not
  act in person, only through mortal mandate.
\end{itemize}

\parbox{\linewidth}{
\textbf{Play Intent:} These mechanics frame conversion as
\emph{identity conflict with consequences}, not a statline optimization.
}


%====================================================
\subsection{Rites of Mykkiel, Angel of Law and Form}

\paragraph{Domain Themes}
Oaths, structure, duty, naming, binding, revelation through ordeal.

\begin{itemize}
  \item \textbf{Seal the Ledger} \;\texttt{[BIND][TRUTH][SANCTION]}\\
  Mark a witnessed oath. Until fulfilled or renounced, both sides gain
  +1d when acting in alignment, and suffer +1 DV when acting against it.

  \item \textbf{Sword of Interpretation} \;\texttt{[JUDGE][CUT][CLARIFY]}\\
  Force one truth into prominence. A contested scene must pivot around
  that interpretation; resisting it costs +1 Fatigue.

  \item \textbf{Light of Sentence} \;\texttt{[REVEAL][SHAME][FORM]}\\
  Unmask one concealed intention or guilt. Targets must speak or pay
  1 Harm or Condition (GM discretion).

  \item \textbf{Sanctified Witness} \;\texttt{[SEAL][ORDER][TEST]}\\
  Create a space where actions become binding and logged. Social or
  magical violence carries an added price (Condition: \emph{Marked}).

  \item \textbf{Clause of Mercy} \;\texttt{[REPRIEVE][BALANCE][LEDGER]}\\
  Designate a sin forgiven — but bind the forgiven to one duty that,
  if failed, doubles consequences.

  \item \textbf{Sword-Quiet} \;\texttt{[SILENCE][COMPULSION][LAW]}\\
  Still a riot or argument long enough for judgment. Lasts one scene;
  backlash creates rivals with \emph{Just Cause}.
\end{itemize}

%====================================================
\subsection{Talents of Mykkiel}
\index{Talents!Mykkiel}

\begin{description}
  \item[Doctrine-Bearer (4 XP):]
  Once per session, declare one outcome as \emph{precedent}. Allies gain
  +1d when invoking your ruling; enemies do so too in appeal.

  \item[Witness of the Unnamed Price (6 XP Prestige):]
  When you mark someone guilty (fictionally), once per session you may
  shift one failed roll into a partial — but mark 1 Fatigue for the
  burden of judgment.

  \item[Mercy as Blade (4 XP):]
  When you spare someone, gain Position +1 against them until they
  betray or defy your mercy.
\end{description}

\vspace{5mm}

%====================================================
\subsection{Rites of Aveh, Keeper of Unburdening and Estranged Ways}

\paragraph{Domain Themes}
Forgetting, refuge, belonging, shedding past harm, dissolving identity
into new masks.

\begin{itemize}
  \item \textbf{Drift the Ledger} \;\texttt{[FORGET][MASK][BREAK]}\\
  Remove one witnessed shame or debt for a scene. Afterward, roll DV~3:
  on Miss, it resurfaces twisted.

  \item \textbf{Mask Right of Sanctuary} \;\texttt{[REFUGE][ROLE][COST]}\\
  Grant someone absolution from past name or crime; they take on a new
  mask until dawn or crisis. This creates debts the GM may invoke.

  \item \textbf{Lotus Sip} \;\texttt{[FORGET][CALM][SUBLIME]}\\
  Ease Harm or Fatigue in a target, but they temporarily lose an
  important memory or drive.

  \item \textbf{Unmake the Chain} \;\texttt{[SUNDER][ESCAPE][UNMAP]}\\
  Break a declared consequence, bond, or pursuit. Alarmingly popular
  with fugitives.

  \item \textbf{The Sweet Lie} \;\texttt{[VEIL][COMFORT][HOOK]}\\
  Offer a truth as they wish it to be. For one scene, they gain +1d when
  acting under the lie; at scene end, mark a complication.

  \item \textbf{Breath of the Nameless Feast} \;\texttt{[ABSOLVE][FADE][COST]}\\
  Consume a small shame or misdeed — gain 1 Boon, but someone else
  experiences its consequence.
\end{itemize}

%====================================================
\subsection{Talents of Aveh}
\index{Talents!Aveh}

\begin{description}
  \item[Mask-Shift Initiate (4 XP):]
  Once per session, remove a social or narrative stigma from yourself
  or an ally — it returns later in altered form.

  \item[Keeper of Lost Doors (6 XP Prestige):]
  You can declare “There was always another way out.”
  Spend 1 Fatigue to rewrite one exit or solution into the fiction.

  \item[Sap of Forgetting (2 XP Minor):]
  When you comfort someone, you may ease their Condition — but they
  lose a small truth or detail they valued.
\end{description}

%====================================================
\section{Conversion Rites (Opposed Faiths)}
\label{sec:conversionRites}
\index{Rites!conversion}
\index{Faith!conversion}

Conversion between Patrons is never simple.
These rites represent ritualized thresholds that mark allegiance,
temptation, conflict, or synthesis between opposed faiths.

All conversion rites carry risk: they may shift the Devotion Spectrum
(\S\ref{sec:dualFaith}) or mark \emph{Spiritual Conflict}.

%----------------------------------------------------
\subsection*{Rite of First Leaning}
\tag{LOW} \quad \texttt{[DECLARATION] [OATH]}

\paragraph{Effect:}
The supplicant publicly voices doubt in their current Patron or expresses
aspiration toward another. Gain:
\begin{itemize}
  \item +1 \textbf{Synthesis Token} if done with witness
  \item OR shift the Devotion Spectrum one step toward the desired Patron
\end{itemize}

\paragraph{Cost:}
Mark \textbf{+1 Exposure} to emissary scrutiny.
If secrecy is kept, gain no benefit.

%----------------------------------------------------
\subsection*{Rite of Renunciation}
\tag{STANDARD} \quad \texttt{[OATH] [CONFLICT] [TRIAL]}

\paragraph{Effect:}
A symbolic cutting of ties: burning a token, unmaking a vow, or shaming
a past title.

Shift Devotion Spectrum by 2 toward the new faith,
but mark \textbf{Spiritual Conflict} unless a witness validates the act.

\paragraph{Push It:}
If the rite is performed in unfavored ground without permission,
gain +1d on the shift but mark a \textbf{Contamination Token}.

%----------------------------------------------------
\subsection*{Rite of the Two Lamps (or Two Wells)}
\tag{STANDARD} \quad \texttt{[MEDIATION] [BALANCE]}

\paragraph{Effect:}
A ritual of dual affirmation: one hand to each altar, lamp, or well.
Roll Spirit + Resolve to maintain balance.
\begin{itemize}
  \item \textbf{Success:} Gain \emph{Ambassador Status} for this scene.
  \item \textbf{Partial:} Mark 1 \emph{Synthesis Token} but suffer DV +1 vs either side.
  \item \textbf{Miss:} Trigger a doctrinal challenge scene immediately.
\end{itemize}

%----------------------------------------------------
\subsection*{Rite of Invocation by Doubt}
\tag{HIGH} \quad \texttt{[TEST] [REVELATION]}

\paragraph{Effect:}
The supplicant petitions a Patron through uncertainty:
“I do not know; show me.”

GM frames a vision or emissary encounter.
Resolve a DV 4–5 \emph{Insight} test.

\begin{itemize}
  \item \textbf{Success:} Shift devotion one step AND gain a hidden name,
        phrase, or rite fragment from the target Patron.
  \item \textbf{Partial:} Shift devotion but mark \emph{Spiritual Conflict}.
  \item \textbf{Miss:} Emissaries contradict — gain +2 Suspicion in the opposing faction.
\end{itemize}

%----------------------------------------------------
\subsection*{Rite of Severance and Binding}
\tag{EXTENDED} \quad \texttt{[OATH] [SACRIFICE] [WITNESS] [SCAR]}

\paragraph{Effect:}
A profound vow: the supplicant binds themselves to a Patron’s ethos while
severing their former identity.

Choose:
\begin{itemize}
  \item Sacrifice a Resource
  \item Sacrifice Harm 2
  \item Sacrifice a Reputation tag
\end{itemize}

Shift devotion 2–3 steps and clear \emph{Spiritual Conflict}.

\paragraph{Obligation:}
Mark an \textbf{Obligation Clock [4]}:
\begin{itemize}
  \item fulfill a service,
  \item spread doctrine,
  \item restore something wronged.
\end{itemize}

If unfulfilled, the spectrum swings back by 2.

%----------------------------------------------------
\subsection*{Rite of Syncretic Crown (Final Rite)}
\tag{LEGENDARY} \quad \texttt{[UNITY] [NARRATIVE] [JUDGMENT]}

\paragraph{Effect:}
This rite may only occur once the supplicant has:
\begin{itemize}
  \item touched both faiths meaningfully,
  \item mediated a conflict scene,
  \item survived emissary judgment.
\end{itemize}

The supplicant declares their synthesis — a principle that both Patrons
\textit{could} accept.

\paragraph{Mechanical Resolution:}
\begin{itemize}
  \item Roll Spirit + Command or Spirit + Lore (DV 5+)
  \item Spend 1--3 \emph{Synthesis Tokens}
\end{itemize}

\paragraph{Success:}
\begin{itemize}
  \item Character moves to Devotion 0 (Balanced)
  \item Gain permanent \textbf{Ambassador Talent}
        (ignore first Position penalty in faith scenes per session)
  \item Either faith may claim them as a righteous example
\end{itemize}

\paragraph{Partial:}
\begin{itemize}
  \item Balanced devotion, but mark \textbf{Spiritual Conflict}
  \item One faction becomes hostile or demands a price
\end{itemize}

\paragraph{Miss:}
The rite is rejected publicly — shift devotion to the extreme of whichever
faith dominates the scene and mark \textbf{+2 Suspicion}.

%====================================================
\section*{Possession \& Exorcism}

\index{Possession}
\index{Exorcism}

A soul may be seized by:
\begin{itemize}
  \item guilt too large to bear,
  \item a memory denied,
  \item or a hungry emissary’s doctrine.
\end{itemize}

\subsection*{Possession Procedure}
When a PC or NPC breaks under pressure, mark:
\begin{description}
  \item[Shadowed:] they hear whispers of their buried truth.
  \item[Claimed:] they act upon it without consent.
  \item[Lost:] the entity speaks through them.
\end{description}

\subsection*{Exorcism as Rite or Duel}

\paragraph{The Mykkiel Method: Purge by Confession}
\begin{itemize}
  \item DV 4 (Desperate)
  \item spend 1 Memory Token (named)
  \item roll Spirit + Resolve
  \item On Success: entity expelled but shame remains as scar
\end{itemize}

\paragraph{The Aveh Method: Relinquish and Rewrite}
\begin{itemize}
  \item DV 4 (Controlled)
  \item PCs or target \textbf{rewrite} the memory:
  change “what happened” into “what it means now”
  \item spend 1 Contact or Pillar
  \item On Success: entity dissolves; target heals but forgets
\end{itemize}

\subsection*{Failure Effects}
\begin{itemize}
  \item Entity becomes external (spawn “Shame-Worm spawn”)
  \item PCs gain Corruption or lose Token involuntarily
  \item Crusaders gain proof the city must burn
\end{itemize}

\subsection*{Dual-Faith Exorcism}
If both rites are used:
\begin{itemize}
  \item PCs spend 2 Tokens combined
  \item Target is freed \emph{and} remembers—\textbf{but must now choose a patron}
\end{itemize}
%====================================================

\end{document}