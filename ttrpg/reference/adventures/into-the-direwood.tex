\documentclass[11pt]{article}

%=========
% Encoding & Layout
%=========
\usepackage[T1]{fontenc}
\usepackage[utf8]{inputenc}
\usepackage[margin=1in]{geometry}
\usepackage{microtype}
\usepackage{parskip}

%=========
% Fonts & Formatting
%=========
\usepackage{titlesec}
\usepackage{enumitem}
\usepackage{multicol}
\usepackage{booktabs}
\usepackage{tabularx}
\usepackage{longtable}
\usepackage{array}

%=========
% Color & Boxes
%=========
\usepackage{xcolor}
\usepackage{tcolorbox}

\definecolor{mistgrey}{RGB}{180,190,200}
\definecolor{diregreen}{RGB}{23,66,48}
\definecolor{bloodrust}{RGB}{112,32,32}

\tcbset{
  colback=mixwhite!98!mistgrey,
  colframe=diregreen,
  sharp corners,
  boxrule=0.7pt
}

%=========
% Hyperlinks
%=========
\usepackage[hidelinks]{hyperref}

% Optional: quick tag macro if you’re using [TAGS]
\newcommand{\Tag}[1]{\textsf{\textbf{[}#1\textbf{]}}}

%=========
% Title
%=========
\title{\vspace{-1.5cm}\bfseries\Huge Into the Direwood\\[4pt]
\Large A Tier~III--IV Survival Horror Saga}
\author{\large A Fate's Edge Adventure Module}
\date{\small Version 0.9 -- \today}

\begin{document}
\maketitle
\thispagestyle{empty}

% Optional mood blurb on title page
\begin{tcolorbox}
\centering
\emph{Beyond the last bell-line, where the mist never lifts and the roots drink only memory,\\
an ancient spear pins a dead druid to a fortress of bone.\\
Pull it free, and the forest remembers how to scream.}
\end{tcolorbox}

\clearpage
\tableofcontents
\clearpage

%=== Adventure text begins here ===

\section{Into the Direwood (Tier III--IV Adventure)}
\label{sec:into-direwood}

\subsection*{Premise and Tone}

\begin{itemize}
  \item \textbf{Recommended Tier:} III--IV (experienced crews)
  \item \textbf{Atmosphere:} Half cosmic horror, half survival horror
  \item \textbf{Themes:} Lost names, corrupted guardianship, identity erased by history, truth vs.\ myth
\end{itemize}

The Direwood is a blighted frontier forest where an ancient fortress vomits mist and undead into the world. Long ago, a nameless champion impaled a corrupted druid with a \textbf{gilden spear}---but the act unleashed the Mist that devoured her, erased her name, and warped her soul into a tragic antagonist.

Deep within the Direwood lies the fortress heart where she and the druid remain bound in a cycle of torment. Heroes must enter, survive, and uncover the truth:  
\textbf{The curse persists not because the druid lives, but because the heroine’s name was erased.}

Only by revealing her name and restoring her saga can the curse be broken.

%====================================================
\subsection{Setup and Hooks}

This adventure is \textbf{setting-agnostic}. Any region bordering cursed wilderness can use it.

\subsubsection*{Universal Hooks}

\begin{itemize}
  \item \textbf{The Mist Thickens:} Something wakes in the Direwood. Fog breaches warding lines—new horrors stalk settlements.
  \item \textbf{A Fragment of Saga:} Scholars, oracles, or ravings speak of a “nameless champion” and a spear of gilded dawn.
  \item \textbf{Faith or Faction Interest:} Reformers, pilgrims, war cults, or occult orders want the fortress breached or sanctified.
  \item \textbf{A Bargain:} A ghostly figure visits a character in a dream:  
        \emph{“Find my name, free me… or drown with me.”}
\end{itemize}

You may replace factions with whatever religions, orders, or political blocs exist in your campaign.

%====================================================
\subsection{Historical Truth}

Once, a powerful druid served as the Direwood’s warden.  
When an exiled necromancer befriended him, their work twisted root and bone into death-magic. The druid raised undead armies—Shades, Mistwraiths, Shadewalkers, and beast-things.

A champion—her culture forgotten—defeated him using a \textbf{gilden spear} of unknown origin.

But victory was a lie.

\begin{itemize}
  \item The druid’s death ruptured a dimensional wound.
  \item Mist poured out, consuming the heroine and her companions.
  \item She became a bound shade—hero and warden both—now twisted into a \textbf{Guardian of Silence}.
\end{itemize}

The world remembered the druid as the villain but \emph{forgot her name}.  
That erasure is the core of the curse.

%====================================================
\subsection{Adventure Structure}

Run this as a 3--5 act ordeal:

%----------------------------------------------------
%====================================================
\subsubsection*{Act I: Payden’s Port and the Forgotten Chord}

Tone: political gothic + creeping horror

\paragraph{Premise:}
The warding bell-line falters. Bells toll out of sequence,  
Mist incursions grow bolder, and refugees whisper of shapes in fog.

The PCs arrive in Payden’s Port — a sea-garrison city of cracked statues,  
storm-beaten docks, and chanting wardens — and are pulled into crisis.

\paragraph{The Hook:}
Only one thing can repair the failing bell-line —  
\textbf{the Chord of the First Curse}:  
a strip of woven gold and root, a relic that was  
\emph{torn away and lost in the Direwood when the first curse was sealed}.

Legends say it once bridged sound and spirit, harmonizing  
the ward with the heroine’s true name.

\paragraph{Act Beats}

\begin{itemize}
  \item \textbf{Factional pressure}:  
  The Aeler garrison, Mistlander priests, and Guild factors all demand different outcomes; each tries to recruit or manipulate the PCs.
  
  \item \textbf{Living omens}:  
  Dead gulls arrange themselves in votive circles, fog-children whisper names not spoken.
  
  \item \textbf{The first breach:}  
  The city’s outer bell-line sputters; Mistwraiths claw at the docks. PCs must assist the wardens or improvise defenses — gaining first taste of the Mist.
  
  \item \textbf{Call to action:}  
  Divination, confession, or coercion leads to one truth:
  \emph{the Chord lies deep in the Direwood,  
  near the fortress where the heroine fell.}
  
  \item \textbf{The Charge}:  
  “Retrieve the Chord and restore the bell-line —  
  or the Port will drown in white.”
  
  The PCs are sent through the breached ward with blessings, suspicions,  
  or threats at their back.
\end{itemize}

\paragraph{Dread Clock}
% moved unchanged, still applicable
Mark ticks when:
\begin{itemize}
    \item the party ignores port omens or visions
    \item they mishandle the dead or desecrate wards
    \item they underestimate or bargain with Mist
\end{itemize}

If filled early, Act II becomes harsher and  
the heroine initially appears as a hostile nemesis.

\vspace{0.5em}

\noindent\textbf{Act I Exit Conditions:}
\begin{itemize}
    \item the party crosses into the Direwood,
    \item an alliance (or enmity) with at least one faction is set,
    \item the first imprint of the nameless heroine haunts one PC.
\end{itemize}

\subsubsection*{Act II: Through the Direwood}

\textbf{Tone:} Survival horror, mythic fever-dream, predatory landscape.

Crossing the tree-line is not entering a forest—it is passing into a place shaped by memory, undeath, and betrayal. The Direwood resists mapping; it \emph{reacts} to intrusion.

\paragraph{Environmental Themes}
\begin{itemize}
  \item \textbf{Twisting Trails:} Paths repeat; distances elongate. A day's march may loop back to a familiar tree, now adorned with new corpses.
  \item \textbf{Living Landmarks:} Hollow oaks bleed pale sap, stone cairns whisper warnings, ravens speak in half-remembered curses.
  \item \textbf{Memory Echoes:} Scenes replay—villagers fleeing, army banners burning, lovers separated—yet the faces blur and distort.
  \item \textbf{Dream Fatigue:} Rest is possible, but sleep is disturbed. PCs wake with soil under their nails, leaves in their hair, or frost upon their cheeks.
\end{itemize}

\paragraph{Adversaries and Threats}
\begin{description}
  \item[Mistwraith Packs:] \emph{Tier III} predators that sense rising fatigue; they do not eat—\emph{they erase}.  
    PCs marked by fear or exhaustion attract them.
  \item[Shadewalkers:] \emph{Tier II-III} revenant infantry—stone-cold tacticians reenacting endless patrols until interrupted.
  \item[Hunger Wolves:] \emph{Tier III beast-spirits}, sinew stretched over bone, jaws unhinging to swallow light itself.
  \item[The Forgotten Chorus:] \emph{Tier IV psychic hazard}—mourning voices that harmonize grief into psychic drowning.
  \item[The Heroine’s Phantom:] A \emph{shifting Tier IV entity}—sometimes guide, sometimes executioner. Her mood tracks the party’s reverence for lost history.
\end{description}

\paragraph{Nightly Visitations}
Each night, one PC dreams of the nameless heroine:
\begin{itemize}
    \item Meeting her mother beneath a moonlit birch.
    \item Riding with oath-sworn shield-sisters.
    \item Walking toward fire knowing she will die.
\end{itemize}

\textbf{GM Note:} Fragments arrive out of order, wrong names inserted—only when her true name is spoken do the visions align.

\paragraph{Direwood Survival Procedures}
\begin{enumerate}
  \item At each travel segment, roll the \textbf{Direwood Consequence Table}.
  \item Reduce safe camp options—firelight attracts attention, cold invites wraiths.
  \item Food spoils overnight unless blessed or sealed.
\end{enumerate}

\bigskip

%==========================================================
\subsection*{Direwood Encounter Generator}
Roll once when entering a new sector or when the party loses direction.

\paragraph{Step 1: Terrain Distortion (d6)}
\begin{enumerate}
  \item Fog walls—movement halved, perception impaired.
  \item Reverse stream—water flows upward, whispers of drowned souls.
  \item Forest of empty armor—hollow suits clatter as wind passes.
  \item Birch grove—trees lean inward as if listening.
  \item Corpse orchard—fruit grows from ribcages (\emph{dangerous temptation}).
  \item Silent clearing—sound dies, even thoughts echo.
\end{enumerate}

\paragraph{Step 2: Threat Manifestation (d6)}
\begin{enumerate}
  \item Mistwraith Pack hunting fear scent.
  \item Shadewalker patrol marching in silent formation.
  \item Hunger Wolves circling camp perimeter.
  \item Memory echo forcing PCs into historical reenactments.
  \item Voice of the Heroine offering cryptic aid (or condemnation).
  \item \emph{No creature}—but the woods rearrange behind them.
\end{enumerate}

\paragraph{Step 3: Psychological Pressure (d6)}
\begin{enumerate}
  \item One PC hears their name whispered from a corpse.
  \item A random PC dreams while awake—vision interrupts action.
  \item Someone’s equipment rots or tarnishes instantly.
  \item PCs lose track of time—night becomes day without transition.
  \item PCs forget a small fact—names, directions, purpose.
  \item PCs find something belonging to them—but aged or broken.
\end{enumerate}

\paragraph{Step 4: Strange Reward or Revelation (optional d6)}
\begin{enumerate}
  \item A black feather giving +1d to bravery tests.
  \item A shard of gilt metal humming near the heart of the forest.
  \item A whisper of her true name—wrong syllables, but directionally useful.
  \item A bone talisman—\emph{Ignore fear once, then crack under stress.}
  \item A half-remembered lullaby that soothes undead spirits.
  \item A map that leads in circles unless the party honors the dead.
\end{enumerate}

\textbf{GM Guidance:} Use 1--2 results per segment. Combine for escalating horror.

\paragraph{Escalation Triggers}
Increase threat when:
\begin{itemize}
  \item PCs desecrate corpses or mock legends.
  \item Someone tries to rationalize the supernatural.
  \item A vision is ignored or laughed off.
\end{itemize}

At high escalation, the Heroine’s Phantom appears—not as ally, but as judge.

\paragraph{Role of the Heroine in Act II}
She is not a quest-giver—she is \emph{the wound}. PCs must:
\begin{itemize}
  \item earn her attention through reverence or boldness,
  \item learn fragments of her forgotten name,
  \item survive her tests.
\end{itemize}

Only by \textbf{speaking her true name} does her attitude shift from hunter to guide.

\subsection*{The Name That Was Forgotten}

The Direwood will not allow the heroine's name to be spoken lightly. Her identity shattered when she died unremembered; the party must \textbf{gather fragments} through visions, echoes, relics, and mercy.

Each revelation provides one \textbf{Name Fragment}. When all are gathered and spoken with intention, her wrath becomes sorrow—and the way forward opens.

\bigskip

\begin{center}
\begin{tabular}{cll}
\toprule
\textbf{d8} & \textbf{Fragment Source} & \textbf{What the PCs Receive} \\
\midrule
1 & Whisper in dream & A syllable spoken by a weeping voice \\
2 & Relic unearthed & Rune-symbol etched in gilt metal \\
3 & Mercy to dead & A spirit forms a phoneme in frost or ash \\
4 & Heroine's phantom & She screams it in rage, misheard or inverted \\
5 & Shadewalker patrol & Captain repeats her battle-cry fragment \\
6 & Shrine remnant & A carved votive holding half her naming-mark \\
7 & Mistwraith hunt & One wraith mimics her dying breath-sound \\
8 & The wood itself & Wind across hollow birches shapes the final tone \\
\bottomrule
\end{tabular}
\end{center}

\bigskip

\paragraph{Guidance}

\begin{itemize}
  \item \textbf{Fragments are ambiguous:} each is a syllable, rune, or phonetic gesture.
  \item Players must \emph{interpret}—wrong assembly draws her fury.
  \item If spoken prematurely, increase the \textbf{Dread Clock} by +1 and attract a spectral assault.
  \item Once all are gathered, speaking the true name pacifies her and reshapes Act IV.
\end{itemize}

\paragraph{Optional Twist}
The final fragment cannot be heard—only \emph{remembered}. A PC must sacrifice a cherished memory to speak it aloud.


%----------------------------------------------------
\subsubsection*{Act III: The Fortress Heart}

The fortress is not built — it is \textit{grown}.  
Petrified roots coil into walls, bone-columns hold roofs of fungus stone, and halls pulse faintly as if remembering breath.

Tone: grim revelation + predatory architecture.

\begin{itemize}
  \item \textbf{Shades} recite campaigns that never resolved.
  \item The \textbf{druid’s corpse} still hangs impaled by the gilden spear, a font of eternal fog.
  \item The \textbf{nameless heroine} prowls silently — testing, haunting, or confronting.
\end{itemize}

\noindent
If the \emph{Dread Clock} is high, she manifests as a lethal nemesis.  
If the PCs earned reverence or curiosity, she delivers riddles, cryptic lore, or half-truth bargains.

\bigskip

%----------------------------------------------------
\paragraph{Threats of the Fortress}

\begin{itemize}
  \item \textbf{Mistwraith Sentinels:} absorb Fatigue; attack as whispers that drain resolve.
  \item \textbf{Bone Choir Shades:} animate when disturbed, chanting fragments of battle oaths.
  \item \textbf{Shadewalker Wardens:} undead knights commanded by instinct only — bodyguards without a master.
  \item \textbf{Grasping Roots:} attempt to root intruders in place; test Body + Athletics to break free.
  \item \textbf{Reflected Self:} mirrors show PCs as future undead — risk mental Fatigue on failed Resolve.
\end{itemize}

\bigskip

%----------------------------------------------------
\paragraph{Fortress Traps and Wards}

\textbf{Magical Traps}
\begin{itemize}
  \item \textbf{Bone-Sigil Wards:} trigger phantom spear volleys.
  \item \textbf{Echo Brands:} mark targets — nearby undead treat branded PCs as hated traitors.
  \item \textbf{Mist Spore Bursts:} cause hallucinations and misplaced action declarations.
  \item \textbf{Remembrance Locks:} doors open only when spoken with a name fragment.
\end{itemize}

\textbf{Mundane Traps}
\begin{itemize}
  \item Root-pit collapse leading to a crypt nursery of Shadewalkers.
  \item Impaling stakes grown from bone growths under leaf cover.
  \item Hollow floors that drop PCs into memory-echo chambers.
  \item Glyph-etched bells that summon a Shade patrol if rung.
\end{itemize}

\bigskip

%----------------------------------------------------
\paragraph{Room Generator: The Dire Stronghold}

Roll d10 for each chamber or turn deeper:

\begin{center}
\begin{tabular}{cl}
\toprule
\textbf{d10} & \textbf{Room / Encounter} \\
\midrule
1 & Hall of Hollow Shields — spectral infantry drill endlessly \\
2 & Bone Spindle Stair — roots twist upward; false steps activate impaling branches \\
3 & The Echo Choir — Shades lament their failed charge; listening risks Fatigue \\
4 & The Memory Pit — visions drag one PC into a relived massacre \\
5 & Oath Archive — bone tablets with names scored out; fragments hidden here \\
6 & The Bloodless Barracks — Shadewalker knights frozen until disturbed \\
7 & Grove of Still Hearts — petrified Druid seeds containing wraith embryos \\
8 & Hall of the Betrayed — mirrors show the PCs as corrupted avatars \\
9 & Root-Chamber Gate — door opens only to a true fragment spoken aloud \\
10 & The Spear Hall — the druid’s husk nailed to reality; the heroine watches \\
\bottomrule
\end{tabular}
\end{center}

\bigskip

%----------------------------------------------------
\paragraph{Adversary Table: Fortress Foes}

\begin{center}
\begin{tabular}{cl}
\toprule
\textbf{d6} & \textbf{Adversary} \\
\midrule
1 & Shade Cohort (3--5 spectral infantry, DV 3 vs Resolve) \\
2 & Mistwraith Pack (absorbs Fatigue before Harm) \\
3 & Shadewalker Knight (Cap 3 undead bodyguard wielding rusted rites) \\
4 & Echo Sorcerer (fragment of the druid, casts DV 4 control spells) \\
5 & The Forgotten Squire (undead child bearing a name rune fragment) \\
6 & Her Shrine-Guardian Self — a phantom of the heroine in life \\
\bottomrule
\end{tabular}
\end{center}

\bigskip

%----------------------------------------------------
\paragraph{Environmental Effects (1d6 per room)}

\begin{itemize}
  \item \textbf{1: Breath Leech} — PCs lose 1 Fatigue entering.
  \item \textbf{2: Time Loop} — first action each turn repeats.
  \item \textbf{3: Whispering Winds} — disadvantage on stealth.
  \item \textbf{4: Grief Pulse} — one PC makes a sorrow check.
  \item \textbf{5: Bone Bloom} — undead roots sprout mid-combat.
  \item \textbf{6: Heroine’s Watch} — her phantom judges the action; aid or hinder.
\end{itemize}

\bigskip

%----------------------------------------------------
\paragraph{GM Guidance: The Heroine as Dynamic Antagonist}

She is not merely a boss — she is a \emph{mirror}.

\begin{itemize}
  \item When PCs show \textbf{oath-keeping}, she grants a name fragment.
  \item When they show \textbf{hubris or desecration}, treat her as a deadly hunter.
  \item If her name is assembled, the fortress reorganises —  
  root-walls peel back, the mist recoils, and the druid’s corpse becomes vulnerable.
\end{itemize}

\noindent
This act should end with a revelation, not a combat — unless the party refuses to remember.

%----------------------------------------------------

%====================================================
\subsection{Name Quest Subsystem: Remembering the Forgotten}
\label{subsec:name_quest}
\index{Name Quests}

\paragraph*{Premise}
Some truths cannot be killed — only forgotten.  
The Heroine of the Mist can only be freed by remembering her.

\subsubsection*{Memory Track}
A shared party track with 6 boxes.

Mark one when:
\begin{itemize}
    \item uncovering a name fragment or true deed,
    \item fulfilling an oath tied to her story,
    \item suffering hardship to defend the innocent,
    \item being recognized by shades as her lost companions.
\end{itemize}

\paragraph*{Effects at Thresholds}
\begin{description}
    \item[2 Marks:] Shades hesitate; mist retreats briefly.
    \item[4 Marks:] The Heroine shifts from Nemesis to Trial.
    \item[6 Marks:] PCs may speak the Name — final resolution triggers.
\end{description}

%----------------------------------------------------
\subsubsection*{Name Fragments}
Fragments may be discovered, traded, stolen, or revealed through:
\begin{itemize}
    \item prophetic dreams,
    \item battlefield memories,
    \item ghost-songs,
    \item the Golden Thorn's revelations.
\end{itemize}

Each Name Fragment must be \textbf{correctly placed} in her saga — requiring Roleplay + Lore + Presence.

%----------------------------------------------------
\subsubsection*{Final Rite: Restoration of Name}
When all fragments fit:
\begin{description}
    \item[Her Shade kneels,] ceasing hostility.
    \item[The Mist lashes,] then withdraws.
    \item[The Heroine ascends,] becoming patron or guide.
\end{description}

The party must swear:
\textit{her saga shall be sung,  
her deeds recorded,  
her name remembered.}

Breaking this oath:
\begin{itemize}
    \item summons her wrath,
    \item marks all PCs as \textbf{oathbreakers},
    \item the Mist becomes a nemesis for the rest of the campaign.
\end{itemize}

%====================================================
%====================================================
\subsubsection*{Act IV: Descent into the Root-Vault}

Tone: metaphysical pursuit + sacrificial myth

\paragraph{Premise}
The Root-Vault is an inverted under-world where:
\begin{itemize}
  \item roots are ossified memory,
  \item words become walls,
  \item and oaths have gravity.
\end{itemize}

Here, sealed at the breach-point of the First Curse,
rests the \textbf{Resonant Keystone} —  
a crystalline bell-core wrapped in the heroine’s erasure.

\paragraph{Why It Matters}
Without it, the bell-line can never be repaired —
the Mist spreads until shore cities fall.

With it, PCs can:
\begin{itemize}
  \item re-tune the warding line,
  \item awaken the quiet bells,
  \item and name the heroine back into existence.
\end{itemize}

\paragraph{Vault Threats}

\begin{itemize}
  \item \textbf{Name-Eaters} —  
        spectral archivists that rip identity pages from memory.

  \item \textbf{Mist Stalkers} —  
        ragged emanations of the druid’s will,
        trying to seize the Keystone to unbind the curse.

  \item \textbf{Root-Engines} —  
        biomechanical heart nodes that reorganize corridors,  
        threatening to erase the path behind PCs.
\end{itemize}

\paragraph{Trial of Worth}  
To claim the Keystone, PCs must:

\begin{enumerate}
  \item \textbf{Speak or fully assemble the heroine’s name.}
  \item \textbf{Acknowledge her sacrifice — and its injustice.}
  \item \textbf{Swear an oath to restore her saga publicly.}
\end{enumerate}

Oath refusal?  
The Keystone remains inert — or lashes out violently.

\paragraph{Backlash Possibilities}
If PCs try to take the Keystone without the oath:
\begin{itemize}
  \item the heroine becomes their eternal Nemesis,
  \item the Resonant Keystone fractures (marked with their names),
  \item and Act V begins with Payden’s Port collapsing.
\end{itemize}

\paragraph{Act IV Ends When:}

\begin{itemize}
  \item the PCs physically hold the \textbf{Resonant Keystone}, and
  \item their bond — or enmity — with the heroine is sealed.
\end{itemize}
%====================================================

%====================================================
\subsection*{Act V: The Last Saga — The Seal and the Spear}
\label{act5:last_saga}

Tone: mythic horror, tragic revelation, ritual triumph

\noindent
The Heart-Fortress shudders as roots split, mist curls in reverse,  
and the pinned druid’s body flexes against the Golden Thorn.  
The party stands where the curse began and will end — or repeat.

%----------------------------------------------------
\subsubsection*{The Awakening of the Husk}

If the PCs disturbed the Thorn, or if the Name remains incomplete:

\begin{itemize}
  \item The \textbf{Druid of the Mist} reanimates.
  \item Root-bone walls shatter into a \textbf{spiral arena}.
  \item The husk becomes a \textbf{multi-phasic boss} (see Stat Block).
\end{itemize}

The party cannot kill it — only \emph{subdue} or \emph{pin} it again.  
Its flesh reforms, its bones knit; it is the curse given shape.

\paragraph*{Final Objective:}  
Hold long enough to force it beneath the Thorn.

%----------------------------------------------------
\subsubsection*{Intervention of the Heroine Shade}

If the PCs completed the Name Quest (Section~\ref{subsec:name_quest}):

\begin{itemize}
  \item The Heroine Shade manifests in full Valkyric radiance.
  \item She strides through the mist untouched, spear alight.
  \item Her presence weakens the husk’s regeneration.
\end{itemize}

\paragraph{Resolution Trigger}
A PC must proclaim her Name aloud as the Shade strikes —  
only then does the cycle break.

\medskip
If done, the Shade impales the Druid again with:
\begin{itemize}
  \item golden light,
  \item cascading roots,
  \item and the sighs of a thousand dead.
\end{itemize}

The arena collapses inward, and the mist recoils violently.

%----------------------------------------------------
\subsubsection*{If the Name Was Not Restored}

The husk overwhelms the fortress’s structure:

\begin{itemize}
  \item Root corridors twist and collapse.
  \item Mist seals passages behind the party.
  \item They must flee through the \textbf{Root-Vault escape path}.
\end{itemize}

Survival is victory — but the curse remains, stronger than before.  
The Golden Thorn becomes inert until her Name is remembered in future play.

%----------------------------------------------------
\subsubsection*{Loot, Legacy, and Consequences}

\paragraph{If the Name Was Restored}
\begin{itemize}
  \item PCs gain \textbf{Saga Reputation +1}.
  \item The Heroine becomes a \textbf{Patron of Valor, Memory, and Resurrection}.
  \item The Mist withdraws for a generation — bell-lines strengthen.
  \item The Golden Thorn ascends in power (see Artifact Upgrades).
\end{itemize}

\paragraph{If the PCs Only Re-Pinned the Druid}
\begin{itemize}
  \item The fortress stabilises.
  \item The Mist is contained, not ended.
  \item PCs become \textit{known} — but without honor or clarity.
  \item The Thorn is unchanged, awaiting remembrance.
\end{itemize}

\paragraph{If the PCs Failed or Fled}
\begin{itemize}
  \item Mist incursions increase for months.
  \item Bell-lines must be rebuilt at great cost.
  \item PCs are viewed with suspicion — or blame.
\end{itemize}

%----------------------------------------------------
\subsubsection*{Epilogue: The Heroine’s Choice}

If restored:
\begin{quote}
``My saga is yours — now yours is mine.
Go — and let the living remember.''
\end{quote}

\noindent
She dissolves into starlight — but may return as:
\begin{itemize}
  \item a Patron,
  \item a guide spirit,
  \item or a PC boon (once/arc miracle of defense or intervention).
\end{itemize}

If not restored, her Shade remains bound:
\begin{itemize}
  \item sometimes an ally,
  \item more often a Nemesis in later arcs,
  \item forever seeking one who will speak her name.
\end{itemize}

%====================================================

\paragraph{GM Guidance: Multi-Phasic Boss Confrontation}

Removing the tuning-spike or attempting to destroy the husk awakens the \textbf{First Druid} in a catastrophic resurrection state.  
This is not a standard fight — it is a ritualized horror battle with shifting phases and survival focus.

\subparagraph{Phase I: The Husk Wakes}
\begin{itemize}
  \item The petrified body reanimates, still impaled.
  \item It lashes mist tendrils and summons \emph{root-shade guardians}.
  \item Attacks ignore armor unless the PCs invoke sacred names or oaths.
\end{itemize}

\noindent % spacing
\textbf{Objective:} Break its stabilization — the spike needs to be removed or driven \emph{deeper}.

\subparagraph{Phase II: The Unbound Druid}
When the spike shifts:
\begin{itemize}
  \item The husk splits open along its old wounds.
  \item The druid’s true form — a writhing tangle of bone, antler, and liquid mist — pulls free.
  \item Mist thickens; PCs must navigate collapsing terrain and hallucinations.
\end{itemize}

\noindent
\textbf{Abilities:}
\begin{itemize}
  \item \emph{Corrupted Command} — rerolls PC successes into failures in scenes of desecration.
  \item \emph{Mist Rebirth} — dismissed enemies reform unless sanctified.
\end{itemize}

\noindent
\textbf{Objective:} Force the druid back to the sanctum-heart — where the spike can restrain it again.

\subparagraph{Phase III: The Heart Rift}
Push the druid to the spike altar — the battlefield becomes a circular pit of oaths, death echoes, and sagas written in air.

\begin{itemize}
  \item PCs must endure mist storms, identity erosion, and memory theft.
  \item The druid attempts to \emph{rewrite one PC into its replacement host}.
\end{itemize}

\noindent
\textbf{Objective:} Impale the druid again — but only through a sacrifice, bargain, or declared deed.

\subparagraph{Conditional Intervention: The Heroinic Shade}
If the party has restored the heroine's name:
\begin{itemize}
  \item Her shade manifests behind the druid as a golden silhouette.
  \item She seizes the spike herself and drives it home through the druid's heart-core.
  \item Mist recoils; servants disintegrate; the sanctum stabilizes.
\end{itemize}

\noindent
This ending:
\begin{itemize}
  \item restores the status quo,
  \item seals the husk back into slumber,
  \item and confirms the heroine as a protector rather than nemesis.
\end{itemize}

\subparagraph{If Her Name Is Not Restored}
The fight must continue through three additional rounds of escalating horror:

\begin{itemize}
  \item the druid's antler mass becomes a crown of thorns,
  \item a PC must willingly offer an oath and blood to bind the spike,
  \item otherwise the fortress collapses and the Mist erupts outward.
\end{itemize}

\noindent
If the fight ends without the heroine intervening, the victory is pyrrhic —  
the druid is bound, but the Mist threshold expands, warping the wider region.

\subparagraph{Reward for the True Name Path}
If her true saga is spoken before or during the battle:
\begin{itemize}
  \item the fight ends early in dramatic fashion,
  \item the heroine becomes a continuing NPC patron or ally,
  \item PCs may take her boon or blessing as a campaign-long tag.
\end{itemize}

\noindent
This version preserves high tension while allowing the name-restoration path to feel mythic and consequential.

\paragraph{Act III Ends When:}
\begin{itemize}
  \item PCs understand that true salvation lies below,
  \item their relationship to the heroine is set (Nemesis, Test, or Maybe Ally),
  \item and the path to the vault opens — willingly or through collapse.
\end{itemize}
%----------------------------------------------------

%====================================================
\subsection{Key Threats}

\subsubsection*{Mistwraiths (Tier IV Horrors)}

Secrets made flesh. Feed on awareness and fear.  
Appear where the party hesitates or falters.

\subsubsection*{Shadewalkers (Tier III Undead)}

Corpse-puppets bound in root and bone.  
Slow, relentless, territorial.

\subsubsection*{Direwood Beasts}

Animals twisted by necromantic saturation.  
Symptoms: extra limbs, eyeless faces, echoing screams.

\subsubsection*{The Nameless Heroine}

Shade-queen, tragic antagonist.  
If encountered before her name is found, treat her as a boss encounter.  
If her name is revealed mid-battle, she collapses into grief rather than rage.

%====================================================
\subsection{Environmental Horror}

\paragraph{Core Tags:}
\begin{itemize}
  \item \textbf{[MIST-CHOKED]} — Disorienting and oppressive.
  \item \textbf{[HAUNTED]} — Reality shows seams; time loops.
  \item \textbf{[STARVED-WORLD]} — No food, no warmth, whispers that sap Resolve.
\end{itemize}

Consider starvation, dehydration, and will-despair as slow pressures.  
The Direwood wants them to give up—or accept its bargain.

%====================================================
\subsection{Rewards and Consequences}

\subsubsection*{Curse-Breaker Outcomes}

\begin{itemize}
  \item \textbf{Saga-Bearer:} PCs gain renown as keepers of a forgotten truth.
  \item \textbf{Artifact Wardens:} They inherit the gilden spear—now dormant unless wielded in her name.
  \item \textbf{Map Change:} The Mistline shifts; local politics scramble to react.
\end{itemize}

\subsubsection*{Failure or Refusal}

\begin{itemize}
  \item The Mist expands or mutates.
  \item The heroine becomes a recurring nemesis.
  \item PCs are marked by the Direwood—hunted by nightmares or undead emissaries.
\end{itemize}

%====================================================
\subsection{Using With Any Setting}

Plug this adventure into:

\begin{itemize}
  \item a Viking-flavored frontier,
  \item a Slavic cursed forest,
  \item a Romanized empire borderland,
  \item a gothic or Celtic wilderness,
  \item or a colonial frontier horror setting.
\end{itemize}

Swap religions, factions, and cultural icons as needed—the core dynamics remain:

\begin{itemize}
  \item \textbf{Forgotten hero as antagonist}
  \item \textbf{Curse anchored in erasure of identity}
  \item \textbf{The truth is the weapon, not the sword}
\end{itemize}

This keeps the module stand-alone, portable, and emotionally powerful.

\begin{center}
    {\Large\bfseries Tales of the Direwood}\\[4pt]
    \textit{A Traveler's Briefing for Those Who Walk Beyond the Bell-Line}
    \end{center}
    
    \vspace{0.5em}
    
    \noindent
    \textbf{What is the Direwood?}\\
    A blighted forest where fog is older than the trees.\
    No map draws the same twice.\
    Beasts walk on borrowed bones.\
    Time forgets to move correctly.
    
    \vspace{0.6em}
    
    \noindent
    \textbf{What the Elders Say:}
    \begin{itemize}
        \item Once it guarded the borderlands.
        \item Now it breathes death instead.
        \item Something within remembers us.
    \end{itemize}
    
    \vspace{0.4em}
    
    \noindent
    \textbf{Safe Conduct Within the Mist}
    \begin{itemize}
        \item Travel in pairs; loneliness draws attention.
        \item Firelight helps, but only if you believe it will.
        \item Bells keep the lesser spirits away—until they learn you.
        \item Do not chase echoes; if someone calls your name, check if their feet touch the ground.
    \end{itemize}
    
    \vspace{0.4em}
    
    \noindent
    \textbf{Signs of Corruption}
    \begin{itemize}
        \item Roots that bleed when cut.
        \item Faces where bark should be.
        \item Whispers in languages you almost know.
    \end{itemize}
    
    \vspace{0.6em}
    
    \noindent
    \textbf{Things Seen Near the Heartwood}
    \begin{itemize}
        \item Towers of petrified wood and bone.
        \item A spearhead of gold hammered through stone.
        \item A phantom warrior in ruined mail watching from the mist.
    \end{itemize}
    
    \vspace{0.6em}
    
    \noindent
    \textbf{The Nameless Hero}
    \begin{itemize}
        \item Locals speak of the “Warden Without Name.”
        \item Some call her guardian, others curse her as betrayer.
        \item Legend holds she fought a great evil—and died forgotten.
    \end{itemize}
    
    \vspace{0.6em}
    
    \noindent
    \textbf{Rumors and Warnings}
    \begin{itemize}
        \item \textit{“If you speak her name, she might listen.”}
        \item \textit{“If you mock her, she will find you.”}
        \item \textit{“The spear keeps him dead—but keeps her bound.”}
    \end{itemize}
    
    \vspace{0.6em}
    
    \noindent
    \textbf{Why Venture In?}
    \begin{itemize}
        \item To stop the creeping Mist.
        \item To find lost knowledge.
        \item To free or defeat the Warden.
        \item To discover her true name.
    \end{itemize}
    
    \vspace{0.8em}
    
    \noindent
    \textbf{Advice from Survivors}
    \begin{itemize}
        \item Bring rope, salt, bells, and flame.
        \item Do not trust your memories inside.
        \item When in doubt—speak truth, not courage.
    \end{itemize}
    
    \vspace{1.0em}
    
    \noindent
    \begin{center}
    \textit{The Direwood does not hate you.}\\
    \textit{It simply remembers before you do.}
    \end{center}

    %====================================================
\subsubsection{Rites of the Druid of the Mist}
\label{rites:mistrdruid}
\index{Rites!Druid of the Mist}

\noindent
These rites reflect the patron’s ethos: \emph{erode, unmake, devour, dissolve}.

Followers learn that every invocation is a transaction --- power traded for identity.

%----------------------------------------------------
\paragraph*{Low Rites (4--6 XP)}

\begin{description}

  \item[Mist’s Caress] \emph{Scene; Zone; No.} \\
  \texttt{[VEIL] [SENSE] [DRAIN]} \\
  Call thin fog across Near. Gain +1 die to detect fear, fatigue, or spiritual imbalance.  
  Living foes suffer unease; undead gain interest.

  \item[Whisper of Hollow Roots] \emph{Scene; Self; No.} \\
  \texttt{[SENSE] [GUIDE]} \\
  Hear paths where boundaries blur. Once per scene, bypass mundane impediment (wall, guard, protocol) narratively; GM provides complication.

  \item[Sap of Forgetting] \emph{Scene; Touch; Yes.} \\
  \texttt{[DRAIN] [WEAKEN]} \\
  Inflict \(-1\) die to resistance rolls for one exchange as the target's name wavers.  
  Push It: Target forgets one small truth or detail temporarily.

  \item[Bitter Blossom Mark] \emph{Extended; Close; Yes.} \\
  \texttt{[CURSE] [DECAY]} \\
  Plant a sigil that corrupts soil, masonry, or a social oath.  
  Start a \textbf{Creeping Ruin [4]} clock; when full, something breaks.
  
\end{description}

%----------------------------------------------------
\paragraph*{Standard Rites (7--9 XP)}

\begin{description}

  \item[Root of Unmaking] \emph{Scene; Zone; Yes.} \\
  \texttt{[DECAY] [BIND] [CORRUPT]} \\
  Infect a structure, body, or agreement with invisible rot.  
  Allies gain +1 die to exploit fractures; order-based actions suffer +1 DV.

  \item[Shroud of Dissolution] \emph{Scene; Self; Yes.} \\
  \texttt{[VEIL] [CONCEAL]} \\
  Become indistinct: sound muffles, form blurs. Gain +1 die on evasion/stealth; fail with 1s risks name-fragment loss (GM chooses).

  \item[Bone Orchard Bloom] \emph{Scene; Zone; Yes.} \\
  \texttt{[ANIMATE] [AREA] [CORRUPT]} \\
  Corpses or dead roots animate as spectral vines/limbs that hinder foes.  
  Create \textbf{Difficult Terrain} and start a \textbf{Gravegarden [4]} clock.

  \item[Harvester’s Breath] \emph{Scene; Zone; Yes.} \\
  \texttt{[DRAIN] [FEED] [AREA]} \\
  Draw fatigue, sorrow, or fear from all in Near.  
  Convert one such condition into +1 die for yourself or a chosen ally.

\end{description}

%----------------------------------------------------
\paragraph*{High Rites (10--14 XP)}

\begin{description}

  \item[Reclaim the Boundary] \emph{Scene; Zone; Yes.} \\
  \texttt{[DOMAIN] [CURSE] [REWRITE]} \\
  Select a place or law; for the scene, its borders cease functioning.  
  Walls don’t divide, ranks don’t compel, names are unreliable.  
  Resistance causes sensory fragmentation (Position -1 next exchange).

  \item[Pale Menagerie] \emph{Scene; Near; Yes.} \\
  \texttt{[SUMMON] [UNDEAD] [COMMAND]} \\
  Call forth 1--3 mist-echoes (Cap 2) that obey simple urges.  
  They dissolve when named or sanctified.

  \item[Walk of Unbeing] \emph{Scene; Self; Yes.} \\
  \texttt{[PHASE] [VEIL] [MOVE]} \\
  Step through boundaries, walls, or social constraints.  
  Treat any Blocked route as Open but mark \textbf{1 Name Erosion}.

  \item[Famine of the Living Word] \emph{Scene; Far; Yes.} \\
  \texttt{[CURSE] [SILENCE] [BLIGHT]} \\
  Silence a leader, liturgy, or institution.  
  For one scene, oaths, commands, or blessings falter.  
  GM marks \textbf{Narrative Ruin} — consequences linger.

\end{description}

%----------------------------------------------------
\paragraph*{Apex Rite (15--20 XP)}

\begin{description}

  \item[The Mist Devours the Story] \emph{Extended; Domain; Yes.} \\
  \texttt{[UNMAKE] [DOMAIN] [APOCALYPSE]} \\
  Declare a rooted structure — a keep, dynasty, covenant, religion, or name.  
  Begin \textbf{The Unmaking [6+]} clock.  
  When full, that thing loses identity:  
  histories rewrite, memories fracture, its power-base dissolves.  
  Survivors may cling to fragments, but it never becomes the same again.  
  \emph{Cost:} Mark \textbf{3 Erosion}, lose one personal truth, risk becoming an avatar.

\end{description}

%----------------------------------------------------
\subsubsection*{Corruption \& Erosion Mechanics}
\index{Corruption}
\index{Identity Loss}

Followers of the Mist do not track \emph{Obligation}; they track \textbf{Erosion}.

Each time a rite produces backlash or a 1 is rolled:
\begin{itemize}
  \item erase a memory,
  \item distort a feature,
  \item or lose a social bond.
\end{itemize}

At \textbf{3+ Erosion}, the GM may enforce:
\begin{itemize}
  \item shadow-speech,
  \item hostile mist manifestations,
  \item or NPCs forgetting the PC’s name.
\end{itemize}

At \textbf{6+ Erosion}, the patron attempts apotheosis —  
the character becomes a \emph{Mist-Thing} unless named and anchored by others.

%====================================================

%====================================================
\subsubsection{Spellwheel: The Druid of the Mist}
\label{spellwheel:mistrdruid}

\begin{center}
\begin{tabular}{|c|c|c|p{6cm}|}
\hline
\textbf{Sigil} & \textbf{Aspect} & \textbf{Tags} & \textbf{Invocation Theme} \\
\hline
Root & Boundary & [BIND][DECAY] &
Unmake structure, rot identity, bind paths or truths \\
\hline
Bloom & Hunger & [DRAIN][FEED] &
Harvest fear/fatigue; convert suffering into momentum \\
\hline
Mist & Veil & [VEIL][SENSE] &
Cloud meaning; detect cracks, sorrow, forgotten names \\
\hline
Bone & Unlife & [ANIMATE][CORRUPT] &
Raise echoes; twist flesh or soil with old death \\
\hline
Hollow & Dominion & [CURSE][REWRITE] &
bend will, rewrite laws, fracture names and memories \\
\hline
Rift & Tide of Unbeing & [PHASE][UNMAKE] &
erase borders, dissolve authority, or step through reality \\
\hline
\end{tabular}
\end{center}

\paragraph*{Usage Notes}
Cantors assemble invocations by chaining:
\[
\textit{Aspect} + \textit{Tags} + \textit{Channel}
\]
Invoker-style play treats the wheel as a \emph{menu}:
\begin{enumerate}
    \item Choose 1--2 Aspects
    \item Add 1--3 matching Tags
    \item State the narrative form (fog, vines, sorrow, whispers)
\end{enumerate}

\noindent
The more the invocation erodes boundaries or identity,  
the higher the DV and the greater the Erosion risk.

%====================================================

%====================================================
\subsection{Patron: The Heroine of the Mist}
\label{patron:mist_heroine}
\index{Patrons!Heroine of the Mist}

\paragraph*{Aspect}
\textit{Memory, Oath, Bloodshed in Defense}

Once mortal — now half-Valkyrie, half-Mourn-Warden.  
She embodies:
\begin{itemize}
    \item valor without recognition,
    \item sacrifice without audience,
    \item and the terrible cost of defending those who never remember you.
\end{itemize}

\paragraph*{Sphere of Influence}
\begin{itemize}
    \item names,
    \item sagas,
    \item identity persistence,
    \item righteous vengeance,
    \item guardianship of the innocent,
    \item oaths that bind but uplift.
\end{itemize}

\paragraph*{Devotional Archetypes}
\begin{description}
    \item[Oath-Keepers:] sworn blades, bodyguards, shieldbearers
    \item[Saga-Wardens:] skalds, lorekeepers, the memory-obsessed
    \item[The Forgotten:] outcasts, orphans, traumatised veterans
\end{description}

\paragraph*{Taboos}
\begin{itemize}
    \item abandoning a sworn charge,
    \item erasing someone’s rightful name,
    \item cowardice in duty.
\end{itemize}

%----------------------------------------------------
\subsubsection{Miracles and Major Rites}
Devotional power flows through remembrance and sacrifice.

\paragraph*{Rites (General Access)}
\begin{description}
    \item[Namebinding:] Mark 1 Fatigue — grant an ally \textbf{+1 Position} for defending or reclaiming identity.
    \item[Saga’s Spark:] Once per scene, declare a past victory to gain \textbf{+1 die} for an action — must narrate the remembered deed.
    \item[Shield of the Forgotten:] Allies under your protection count as \textbf{one tier higher} for resisting fear or domination.
    \item[Blood Oath:] Swear protection over a person or truth; gain a boon whenever fulfilling that oath through hardship.
    \item[Witness to Glory:] Convert suffering (Fatigue/Harm) into +1 SB when defending others.
\end{description}

%----------------------------------------------------
\subsubsection{Higher Mysteries (Prestige/Epic Rites)}
\begin{description}
    \item[Reclaim the Name:] Restore a forgotten truth or identity; remove Mist penalties for the scene.
    \item[Mist-Splitting Spear:] Manifest spectral spear; +2 dice vs undead and corrupted oathbreakers.
    \item[Saga Ascent:] Once per session, prevent death of one ally — they rise with 1 Harm and a memory of you they cannot shake.
    \item[The Last Watch:] Bind your spirit to a place or charge; if abandoned, you become a shade that hunts oathbreakers.
\end{description}

%----------------------------------------------------
\subsubsection{Spellwheel: The Heroine of the Mist}
\label{spellwheel:mist_heroine}

\begin{center}
\begin{tabular}{|c|c|c|p{6cm}|}
\hline
\textbf{Sigil} & \textbf{Aspect} & \textbf{Tags} & \textbf{Invocation Theme} \\
\hline
Spear & Valor & [SMITE][PURIFY] & punish undead, strike corruption, reveal truth \\
\hline
Shield & Oath & [GUARD][ENDURE] & intercept harm, stand immovable, sanctify ground \\
\hline
Torch & Memory & [REVEAL][NAME] & expose hidden identity; illuminate forgotten stories \\
\hline
Tears & Sacrifice & [BIND][MEND] & heal through cost; bind wounds, take suffering \\
\hline
Crown & Remembrance & [SAGA][INSPIRE] & uplift allies, declare deeds into reality, rewrite legacy \\
\hline
Bell & Requiem & [BANISH][SEAL] & end hauntings; sever curses; close breaches \\
\hline
\end{tabular}
\end{center}

\paragraph*{Cantor/Invoker Themes}
\begin{itemize}
    \item songs of valor,
    \item whispered sagas,
    \item spear-light metaphors,
    \item radiant mist,
    \item oath-fire halos.
\end{itemize}

%----------------------------------------------------
\subsubsection{Devotional Gifts (Mechanical Boons)}
\begin{itemize}
    \item \textbf{Once per session:} treat any Desperate defense as Dominant.
    \item When defending others, gain \textbf{+1d} to resist fear, torment, or loss.
    \item When fulfilling sworn duty under hardship, \textbf{mark 1 SB}.
\end{itemize}

%----------------------------------------------------
\subsubsection{Narrative Role in the Direwood Adventure}
If her true Name is restored:
\begin{itemize}
    \item she manifests at Act V,
    \item impales the Druid with the \textit{Golden Thorn},
    \item and offers one PC her blessing — or burden.
\end{itemize}

If her Name remains lost:
\begin{itemize}
    \item she remains a Nemesis Shade,
    \item hunting oathbreakers and defilers,
    \item until someone speaks her back into the world.
\end{itemize}

%====================================================

%====================================================
\subsection{Nemesis Profile: The Heroine Shade}
\label{nemesis:mist_heroine}
\index{Nemesis!Heroine of the Mist}

\paragraph*{Role}
\textit{Guardian of memory, judge of oaths, hunter of the unworthy.}

A spectral shieldmaiden formed of mist and radiant sorrow.  
She is tragedy weaponised.

\paragraph*{Tier}
IV (Legendary Adversary)

\paragraph*{Concept Tags}
[VALKYRIE] [OATH] [MOURN] [JUDGE] [SAGA]

%----------------------------------------------------
\subsubsection*{Core Traits}
\begin{description}
    \item[Spirit:] 6
    \item[Resolve:] 5
    \item[Body:] 4
    \item[Speed:] 3
    \item[Presence:] 6
\end{description}

\noindent
\textbf{Harm Capacity:} 6 (Cannot be killed until her Name is restored)

\textbf{Armor:} 2 (Spectral Plate)

%----------------------------------------------------
\subsubsection*{Phase Structure}

\paragraph*{Phase I: Silent Hunt}
\textit{She watches, tests, strikes symbols not flesh.}

\begin{itemize}
    \item Attacks inflict Conditions rather than Harm:\\
    [REVERENCE], [FEAR], [UNWORTHY]
    \item If PCs perform \textbf{oath-keeping}, she withdraws for 1 exchange.
\end{itemize}

\paragraph*{Phase II: The Shieldmaiden Revealed}
\textit{Triggered when she is struck for 3+ Harm in one scene.}

\begin{itemize}
    \item Gains +2 dice to melee attacks.
    \item Her spear manifests: Harm 2, ignores Armor.
    \item She utters fragments of her lost Name.
\end{itemize}

\paragraph*{Phase III: The Ascendant Shade}
\textit{Triggered when PCs hold three Name Fragments or swear an oath to restore her saga.}

\begin{itemize}
    \item She switches from Nemesis to Trial.
    \item PCs may \textbf{roll to remember} (Lore + Presence).
    \item If spoken true, she kneels — and joins Act V.
\end{itemize}

%----------------------------------------------------
\subsubsection*{Abilities}

\paragraph*{Mist-Splitting Spear}
+2 dice vs corrupted or undead foes; on 6s, pins enemy spirits.

\paragraph*{Witness of Deeds}
Once per scene, invert Position against oathbreakers.

\paragraph*{Saga Binding}
When PCs narrate past deeds tied to sacrifice, grant +1d to them.

\paragraph*{Oathscour}
When someone abandons a charge, immediately strike Harm 2 ignoring Armor.

%----------------------------------------------------
\subsubsection*{Weaknesses}
\begin{itemize}
    \item Cannot be harmed until her Name is known.
    \item Will not strike the innocent unless manipulated.
    \item Spear shattered by betrayal or false remembrance.
\end{itemize}

%====================================================

%====================================================
\subsection{Artifact: The Golden Thorn}
\label{artifact:golden_thorn}
\index{Artifacts!Golden Thorn}

\paragraph*{Origin}
Forged by an unknown power to pierce immortal spirits and bind mist-born corruption.

\paragraph*{Appearance}
A meter-long gilded pin, root-etched, warm to the touch, humming faintly with invoked names.

\paragraph*{Properties}
\begin{itemize}
    \item \textbf{Harm 2;} ignores Armor on undead or cursed entities.
    \item When used to impale, creates a \textbf{Banishment Seal}.
    \item If wielded by one who knows the Heroine’s true Name, treat as \textbf{Harm 3}.
\end{itemize}

%----------------------------------------------------
\subsubsection*{Golden Thorn Invocation Rites}
These may be used by Cantors, Invokers, or Runekeepers.

\paragraph*{Thorn Seal}
\texttt{[BANISH][SEAL][MIST]} — impale a spirit, pinning it to reality.

\paragraph*{Memory Piercing}
\texttt{[NAME][REVEAL]} — the impaled target utters a name or truth unwillingly.

\paragraph*{Vigilant Thorn}
\texttt{[GUARD][WARD]} — mark a threshold; undead must test to cross.

\paragraph*{Radiant Impalement}
\texttt{[PURIFY][LIGHT][SPEAR]} — destroy spectral extensions or illusions.

\noindent
\textbf{Restriction:}  
May only be used at full potency by one who has sworn an oath to restore the forgotten.

%====================================================
\appendix
\chapter{Running Direwood: GM Tips and Appendix}

\section{Preparation Checklist}

\subsection{Session Zero Considerations}
\begin{itemize}
    \item Confirm player comfort with horror themes (psychological horror, identity loss, existential dread)
    \item Establish safety tools (X-Card, Script Change) for intense scenes
    \item Review character backgrounds for integration hooks
    \item Set expectations for investigative vs. combat focus
\end{itemize}

\subsection{Pre-Session Preparation}
\begin{enumerate}
    \item Map key faction NPCs with clear motivations
    \item Prepare 3-4 name fragment discovery scenarios
    \item Review Patron mechanics for The Heroine and The Druid
    \item Set up tracking sheets for:
        \begin{itemize}
            \item Dread Clock
            \item Memory Track
            \item Faction attitudes
            \item Environmental consequences
        \end{itemize}
    \item Prepare key location descriptions (Payden's Port, Direwood sectors, Fortress Heart)
\end{enumerate}

\section{Pacing and Flow Management}

\subsection{Act I: Payden's Port}
\begin{itemize}
    \item \textbf{Focus}: Political tension + creeping horror
    \item \textbf{Key Scenes}: Faction meetings, first breach, departure
    \item \textbf{Pacing}: 2-3 sessions
    \item \textbf{GM Tips}: 
        \begin{itemize}
            \item Use omens liberally (dead gulls, fog children)
            \item Let factions compete for PC attention
            \item Don't rush the departure - build dread
        \end{itemize}
\end{itemize}

\subsection{Act II: Through the Direwood}
\begin{itemize}
    \item \textbf{Focus}: Survival horror + mythic revelation
    \item \textbf{Key Mechanics}: Environmental table, name fragments, Heroine's phantom
    \item \textbf{Pacing}: 3-4 sessions
    \item \textbf{GM Tips}:
        \begin{itemize}
            \item Vary the environmental effects - don't use them every scene
            \item Let the Heroine's phantom shift based on PC actions
            \item Name fragments should feel earned, not random
        \end{itemize}
\end{itemize}

\subsection{Act III: The Fortress Heart}
\begin{itemize}
    \item \textbf{Focus}: Grim revelation + predatory architecture
    \item \textbf{Key Mechanics}: Room generator, adversary table, Heroine's judgment
    \item \textbf{Pacing}: 2-3 sessions
    \item \textbf{GM Tips}:
        \begin{itemize}
            \item The fortress should feel alive and reactive
            \item Use the Heroine as a dynamic antagonist, not static boss
            \item Let the PCs' Memory Track influence her behavior
        \end{itemize}
\end{itemize}

\section{Key Mechanical Reminders}

\subsection{Story Beat Economy}
\begin{table}[h]
\centering
\begin{tabular}{|c|l|l|}
\hline
\textbf{SB} & \textbf{Direwood Use} & \textbf{Impact} \\
\hline
1 SB & Minor environmental shift & Path loops, fog thickens \\
2 SB & Moderate complication & Wraith attraction, memory loss \\
3 SB & Serious threat & Shadewalker patrol, trap activation \\
4+ SB & Major escalation & Heroine's wrath, fortress restructuring \\
\hline
\end{tabular}
\caption{Direwood Story Beat Applications}
\end{table}

\subsection{Environmental Horror Tags}
\begin{itemize}
    \item \textbf{[MIST-CHOKED]}: Disorientation, impaired perception, fear generation
    \item \textbf{[HAUNTED]}: Reality seams, time loops, memory distortion
    \item \textbf{[STARVED-WORLD]}: Resource scarcity, will-draining, identity erosion
\end{itemize}

\section{Character Archetype Support}

\subsection{Tank/Controller Players}
\begin{itemize}
    \item Emphasize defensive positioning opportunities
    \item Provide clear crowd control scenarios
    \item Reward proactive threat management
\end{itemize}

\subsection{Striker/DPS Players}
\begin{itemize}
    \item Offer tactical advantage positioning
    \item Include mobile combat scenarios
    \item Provide clear damage-dealing opportunities
\end{itemize}

\subsection{Support/Buffer Players}
\begin{itemize}
    \item Create party-cohesion challenges
    \item Include social manipulation opportunities
    \item Provide healing/resource management moments
\end{itemize}

\subsection{Utility/Scout Players}
\begin{itemize}
    \item Design puzzle/problem-solving scenarios
    \item Include exploration-focused challenges
    \item Reward specialized skill usage
\end{itemize}

\section{Thematic Reinforcement}

\subsection{Consistency Tips}
\begin{itemize}
    \item Always tie mechanical effects to narrative causes
    \item Let the forest react to player choices thematically
    \item Use recurring motifs (bones, mist, names, memory)
    \item Maintain the tension between heroism and tragedy
\end{itemize}

\subsection{Horror Pacing}
\begin{enumerate}
    \item \textbf{Establish Normalcy}: Start with recognizable elements
    \item \textbf{Introduce Wrongness}: Small, subtle violations
    \item \textbf{Escalate Gradually}: Build to major violations
    \item \textbf{Provide Relief}: Moments of safety or clarity
    \item \textbf{Return Stronger}: Escalate beyond initial wrongness
\end{enumerate}

\section{Troubleshooting Common Issues}

\subsection{Players Moving Too Slowly}
\begin{itemize}
    \item Increase environmental pressure (Dread Clock ticks)
    \item Introduce time-sensitive threats
    \item Use faction pressure from Payden's Port
\end{itemize}

\subsection{Players Rushing Ahead}
\begin{itemize}
    \item Enforce consequences for missed investigation
    \item Use environmental barriers that require preparation
    \item Let important clues become inaccessible
\end{itemize}

\subsection{Combat Dominance}
\begin{itemize}
    \item Emphasize non-combat solutions to threats
    \item Use adversaries that are difficult to defeat conventionally
    \item Introduce social or psychological consequences
\end{itemize}

\subsection{Investigation Stalling}
\begin{itemize}
    \item Provide multiple paths to information
    \item Use active investigation (visions, dreams, encounters)
    \item Let the environment provide clues organically
\end{itemize}

\section{Optional Variations}

\subsection{Shortened Campaign}
\begin{itemize}
    \item Focus on core mystery (name restoration)
    \item Reduce faction complexity
    \item Streamline environmental challenges
    \item Condense fortress exploration
\end{itemize}

\subsection{Increased Horror Focus}
\begin{itemize}
    \item Emphasize psychological effects
    \item Increase memory distortion elements
    \item Add more personal horror connections
    \item Reduce combat options
\end{itemize}

\subsection{Mythic Heroism Focus}
\begin{itemize}
    \item Emphasize the Heroine's redemption arc
    \item Provide more opportunities for legendary actions
    \item Focus on saga and legacy themes
    \item Reduce horror elements
\end{itemize}

\section{Tracking Sheets and Handouts}

\subsection{Player Tracking}
\begin{itemize}
    \item Memory Track shared sheet
    \item Name Fragment collection log
    \item Faction relationship tracker
    \item Direwood survival log
\end{itemize}

\subsection{GM Tracking}
\begin{itemize}
    \item Dread Clock status
    \item Heroine's attitude tracker
    \item Environmental effect log
    \item Faction pressure indicators
\end{itemize}

\section{Conclusion}
Remember: The Direwood responds to the players' approach. Respect their choices, escalate meaningfully, and let the tragedy of the forgotten heroine drive both horror and hope.


\end{document}
