\documentclass[11pt]{article}
\usepackage[utf8]{inputenc}
\usepackage[T1]{fontenc}
\usepackage[a4paper, margin=1in]{geometry}
\usepackage{titlesec}
\usepackage{enumitem}
\usepackage{fancyhdr}
\usepackage{hyperref}
\usepackage{lmodern}
\usepackage{setspace}
\usepackage{mdframed}
\usepackage{xcolor}

% Set up headers and footers
\pagestyle{fancy}
\fancyhf{}
\rhead{The Memory Merchant's Labyrinth}
\lhead{Fate's Edge Adventure}
\rfoot{\thepage}

% Section formatting
\titleformat{\section}{\Large\bfseries\color{blue}}{\thesection}{1em}{}
\titleformat{\subsection}{\large\bfseries\color{black}}{\thesubsection}{1em}{}
\titleformat{\subsubsection}{\normalsize\bfseries\color{black}}{\thesubsubsection}{1em}{}

% Paragraph spacing
\setlength{\parindent}{0pt}
\setlength{\parskip}{6pt}

% Custom box for Core Innovation
\newenvironment{coreinnovation}{%
  \begin{mdframed}[backgroundcolor=gray!10, linewidth=0pt]
  \textbf{Core Innovation Box:}
}{%
  \end{mdframed}
}

% Title
\title{
  \textbf{The Memory Merchant's Labyrinth} \\
  \large A Fate's Edge Adventure for Tiers II--III
}
\author{}
\date{}

\begin{document}

\maketitle

\section*{Core Concept Framework}

\subsection*{Adventure Identity Statement}
\textbf{What makes this adventure sing?} A heist adventure where memories are the ultimate currency, and players must steal from a merchant who trades in stolen experiences while navigating a maze that shifts based on collective memory.

\textbf{What is the central tension?} The players must balance their need for powerful memories against the risk of losing their own identities in the process.

\textbf{What will players remember?} The moment they had to choose between a memory that could save a friend or one that could damn an innocent person.

\subsection*{Thematic Pillars}
\begin{itemize}[leftmargin=*]
  \item \textbf{Horror by Consent} -- Players willingly risk their memories for power
  \item \textbf{Social by Manipulation} -- Memory trading creates complex webs of obligation
  \item \textbf{Exploration by Discovery} -- The labyrinth reveals itself through remembered truths
\end{itemize}

\section*{Structural Template}

\subsection*{Act I: Entry \& Engagement}

\textbf{Hook (Opening Scene)} \\
The PCs wake in a memory-auction house with no recollection of how they arrived. Their belongings are gone, replaced with ``debt markers'' -- memories they supposedly owe to the Memory Merchant. A holo-projection of a smiling woman's face explains: ``Welcome to your payment plan. You have seven days to retrieve seven memories, or you'll wake up as someone else entirely.''

\textbf{Establishment Beats}
\begin{enumerate}[leftmargin=*]
  \item \textbf{The Memory Market}: Players navigate the bizarre auction house where memories are traded like commodities. They witness a man literally forget how to love after selling that memory, and a woman who bought ``courage under fire'' but now panics at the sight of her own reflection.
  \item \textbf{First Debt}: The Merchant's assistant explains one of their debts -- they must steal ``The Last Sunrise Memory'' from the labyrinth. The memory belongs to an elderly man who's the last person to see the sun before the Great Dimming.
  \item \textbf{The Labyrinth Revealed}: Players enter the memory-maze, which immediately shifts based on what they remember about their missing memories. The walls are made of crystallized experiences, and touching them reveals fragments of other people's lives.
\end{enumerate}

\subsection*{Act II: Complications \& Choices (Continued)}

\textbf{Rising Tension}
\begin{enumerate}[leftmargin=*]
  \item \textbf{Competing Collectors}: Another group is also hunting for memories -- ``Memory Purists'' who believe memories shouldn't be traded. They're willing to kill to ``free'' memories, creating a three-way conflict.
  \item \textbf{The Price of Remembering}: Each memory the players acquire changes them slightly. They gain abilities tied to those memories but also inherit the emotional baggage. A memory of ``perfect revenge'' makes them more aggressive; ``first love'' makes them more empathetic but also more vulnerable.
  \item \textbf{The Merchant's True Debt}: Discovery that their ``debt'' was manufactured -- they were targeted because they possess a memory the Merchant desperately needs: witnessing a crime that never officially happened.
\end{enumerate}

\textbf{Midpoint Crisis} \\
The labyrinth begins to collapse as the players' memories start bleeding into the maze itself. Walls show their childhood fears, corridors echo with voices of people they've forgotten. They realize the labyrinth is feeding on their confusion to grow stronger. The elderly man who owns ``The Last Sunrise'' offers to help, but his price is one of their own precious memories.

\subsection*{Act III: Climax \& Resolution}

\textbf{Approach to Climax} \\
Players must choose: steal the Merchant's ultimate memory (which could destroy the memory trade forever) or find a way to expose her crimes without becoming like her. The labyrinth grows more unstable as their own memories become unreliable.

\textbf{Climactic Encounter} \\
\textbf{The Memory Throne Room}: A battle of wits and wills where players must navigate not just physical space but mental space. The Merchant can weaponize memories against them, forcing them to experience traumatic events or tempting them with perfect alternate lives.

\textbf{Resolution \& Aftermath}
\begin{itemize}[leftmargin=*]
  \item \textbf{If they destroy the memory trade}: Memories return to their original owners, but some people lose abilities they'd come to depend on. The players gain a permanent ability to sense when memories are false.
  \item \textbf{If they expose the Merchant}: She's arrested, but the memory market goes underground, becoming more dangerous. Players gain connections in the legitimate memory recovery business.
  \item \textbf{If they join the Merchant}: They become memory merchants themselves, gaining incredible power but slowly losing their original identities.
\end{itemize}

\section*{Mechanical Integration Framework}

\begin{coreinnovation}
\textbf{Signature System: Memory Resonance} -- Players can ``tune'' into memories left in objects/locations
\begin{itemize}[leftmargin=*]
  \item \textbf{Purpose}: Make memory exploration tactile and risky
  \item \textbf{Integration}: Uses existing skill system but adds memory-specific consequences
  \item \textbf{Player Agency}: Choose which memories to absorb and which risks to take
  \item \textbf{Sample Uses}:
    \begin{itemize}
      \item Touching a murder weapon to see the crime (but risk absorbing the killer's rage)
      \item Holding a wedding ring to experience the marriage (but inherit relationship baggage)
      \item Reading a diary page to gain knowledge (but the writer's neuroses)
    \end{itemize}
\end{itemize}
\end{coreinnovation}

\subsection*{Resource Management Layer}

\textbf{Adventure-Specific Resources:}
\begin{itemize}[leftmargin=*]
  \item \textbf{Memory Debt Tracker} [7]: Players must acquire 7 memories or lose their identities
  \item \textbf{Identity Stability} (0--10): How much of themselves players retain -- drops when absorbing powerful memories
  \item \textbf{Memory Echoes}: Temporary abilities gained from absorbed memories (1 session duration)
\end{itemize}

\textbf{Asset Building Opportunities:}
\begin{itemize}[leftmargin=*]
  \item \textbf{Recovered Memories}: Permanent abilities tied to significant memories
  \item \textbf{Memory Anchors}: Items that help maintain identity stability
  \item \textbf{Debt Forgiveness}: Reduces required memories through moral choices
  \item \textbf{Memory Network}: Contacts who can provide/sell memories
\end{itemize}

\subsection*{Custom Mechanics Integration}

\textbf{New Resolution Method: Memory Infiltration}
\begin{itemize}[leftmargin=*]
  \item \textbf{Approach}: Wits + Insight or Presence + Sway
  \item \textbf{Position}: Based on how personal the memory is to the target
  \item \textbf{Success}: Gain memory + temporary ability
  \item \textbf{Partial}: Gain memory but suffer emotional consequence
  \item \textbf{Miss}: Memory backfires, target's trauma becomes yours
\end{itemize}

\textbf{Clock Architecture:}
\begin{itemize}[leftmargin=*]
  \item \textbf{Labyrinth Stability} [8]: Maze becomes more dangerous as players destabilize it
  \item \textbf{Identity Dissolution} [6]: Players lose themselves the longer they stay
  \item \textbf{Merchant's Plan} [10]: Her scheme to acquire the forbidden memory progresses
  \item \textbf{Memory Market Corruption} [6]: Underground trade becomes more dangerous
\end{itemize}

\section*{GM Support Systems}

\subsection*{Session Preparation Checklist}
\begin{itemize}[leftmargin=*]
  \item [ ] Core memories needed for each debt clearly defined
  \item [ ] Memory Merchant's true goal and backup plans
  \item [ ] Labyrinth layout that shifts based on player memories
  \item [ ] Emotional consequences for each major memory
  \item [ ] Competing faction motivations and tactics
  \item [ ] Identity stability effects at different thresholds
\end{itemize}

\subsection*{Player Agency Reminders}
\textbf{At Choice Points:}
\begin{itemize}[leftmargin=*]
  \item ``What piece of yourself are you willing to risk for this memory?''
  \item ``Who suffers if you take this memory from them?''
  \item ``How will this memory change who you are tomorrow?''
  \item ``What would your past self think of this trade?''
\end{itemize}

\textbf{When Tension Lags:}
\begin{itemize}[leftmargin=*]
  \item Introduce memory echoes that conflict with current situation
  \item Have the labyrinth shift to reflect player fears
  \item Force identity stability checks during crucial moments
  \item Bring in Memory Purists with extreme solutions
\end{itemize}

\subsection*{Complication Generator}
\begin{itemize}[leftmargin=*]
  \item \textbf{Mild (1 SB)}: Memory echo conflicts with current task; brief identity confusion
  \item \textbf{Moderate (2 SB)}: Labyrinth shifts to traumatic memory; temporary skill loss
  \item \textbf{Serious (3 SB)}: Someone else claims the memory first; identity stability drops
  \item \textbf{Major (4+ SB)}: Memory contains a hidden curse; player must choose which memory to lose permanently
\end{itemize}

\section*{Thematic Consistency Tools}

\subsection*{Tone Maintenance}
\textbf{Opening Description:} ``The air tastes of forgotten dreams and the metallic tang of crystallized thoughts. Every surface whispers with the ghost-impressions of experiences that no longer belong to their owners.''

\textbf{NPC Voices:}
\begin{itemize}[leftmargin=*]
  \item \textbf{Memory Merchant}: Speaks in perfect, calculated sentences that feel like they're reading from a script
  \item \textbf{Elderly Man}: Voice cracks with age and the weight of too many sunrises remembered
  \item \textbf{Memory Purist Leader}: Passionate but fanatical, speaking in absolutes about the sanctity of experience
\end{itemize}

\textbf{Environmental Storytelling:} Walls show fragmented scenes from memories -- a child's first steps, a lover's betrayal, a parent's death -- all playing simultaneously like a broken film reel.

\subsection*{Theme Reinforcement}
\textbf{Each Scene Should:}
\begin{itemize}[leftmargin=*]
  \item Challenge players' understanding of what makes a memory ``theirs''
  \item Present choices between power and authenticity
  \item Reveal how memories shape identity through concrete examples
  \item Connect to larger themes about commodification of experience
\end{itemize}

\subsection*{Moral Complexity Integration}
\textbf{Gray Choices:}
\begin{itemize}[leftmargin=*]
  \item Steal a memory that's destroying its owner vs. respecting their right to keep it
  \item Trade a friend's embarrassing memory for a crucial ability
  \item Help the Memory Purists destroy the trade vs. letting people make their own bad choices
  \item Take a memory that could prevent a crime but was gained through torture
\end{itemize}

\section*{Resolution Path Framework}

\subsection*{Multiple Valid Approaches}
\begin{enumerate}[leftmargin=*]
  \item \textbf{Direct Confrontation}: Face the Merchant in her memory throne room
  \item \textbf{Market Manipulation}: Work within the memory trade to bankrupt or expose her
  \item \textbf{Purist Alliance}: Join forces with Memory Purists to destroy the labyrinth entirely
  \item \textbf{Identity Sacrifice}: Become the new Memory Merchant to control the trade ethically
\end{enumerate}

\subsection*{Outcome Matrix}
\begin{itemize}[leftmargin=*]
  \item \textbf{Success}: Gain powerful memories, but at the cost of some personal identity
  \item \textbf{Compromise}: Partial victory that leaves the memory trade in a new but uncertain state
  \item \textbf{Failure}: Become memory debt slaves or lose crucial parts of who they are
  \item \textbf{Transformation}: Players fundamentally change their approach to knowledge and experience
\end{itemize}

\subsection*{Consequence Types}
\begin{itemize}[leftmargin=*]
  \item \textbf{Immediate}: Memory echoes that last the session
  \item \textbf{Ongoing}: Permanent changes to skills/abilities based on absorbed memories
  \item \textbf{Character}: Identity stability becomes a persistent resource to manage
  \item \textbf{World}: Memory trade either reformed, destroyed, or driven underground
\end{itemize}

\section*{Scalability Framework}

\subsection*{Tier Adaptation}
\textbf{Lower Tiers:}
\begin{itemize}[leftmargin=*]
  \item Reduce Memory Debt Tracker to [5]
  \item Simplify labyrinth to fewer shifting elements
  \item Focus on 2--3 core memories instead of 7
  \item Make consequences less severe (temporary vs. permanent changes)
\end{itemize}

\textbf{Higher Tiers:}
\begin{itemize}[leftmargin=*]
  \item Add political factions who use memory trade for espionage
  \item Include other memory merchants with competing schemes
  \item Expand to multiple labyrinth locations across different realms
  \item Add time pressure as memory market collapse threatens reality
\end{itemize}

\subsection*{Session Modularity}
\textbf{Can Break Into:}
\begin{itemize}[leftmargin=*]
  \item \textbf{3-session arc}: Entry \& first debts, Labyrinth exploration \& Merchant's scheme, Final confrontation
  \item \textbf{2-session adventure}: Market introduction \& first 3 debts, Labyrinth climax
  \item \textbf{1-session encounter}: Single memory heist with identity consequences
\end{itemize}

\section*{Documentation Standards}

\subsection*{Player-Facing Materials}
\begin{itemize}[leftmargin=*]
  \item \textbf{Memory Debt Card}: Track owed memories and acquired ones
  \item \textbf{Identity Stability Tracker}: Visual representation of self vs. absorbed memories
  \item \textbf{Labyrinth Map Sheet}: Dynamic map that players can mark as it shifts
  \item \textbf{Memory Echo Cards}: Temporary ability cards with emotional consequences
\end{itemize}

\subsection*{GM Quick Reference}
\begin{itemize}[leftmargin=*]
  \item \textbf{Memory Merchant's True Motivation}: She witnessed a crime that would destroy a major patron
  \item \textbf{Key Memory Descriptions}: 10--15 memories with abilities and consequences
  \item \textbf{Labyrinth Shift Triggers}: When to change the maze based on player actions
  \item \textbf{Identity Stability Effects}: What happens at different stability levels
\end{itemize}

\subsection*{Adventure Scaling Notes}
\textbf{Fast Play:} Focus on 3 core memories and simplified labyrinth \\
\textbf{Deep Dive:} Include all factions, memory market politics, and long-term consequences \\
\textbf{Character Spotlights:} Each PC's background memories become part of the debt \\
\textbf{Campaign Integration:} Memory trade becomes ongoing source of quests and complications

\section*{Quality Assurance Checklist}

\subsection*{Essential Elements}
\begin{itemize}[leftmargin=*]
  \item [x] Clear adventure identity statement about memory commodification
  \item [x] Meaningful choices between power and identity
  \item [x] Integration with core Fate's Edge mechanics through custom Memory Resonance
  \item [x] Custom mechanics that serve the memory theme
  \item [x] Multiple valid approaches to the central conflict
  \item [x] Clear escalation from market exploration to labyrinth climax
  \item [x] Memorable NPCs with distinct motivations
  \item [x] Atmospheric location descriptions focused on memory themes
  \item [x] Meaningful resource management through Identity Stability
  \item [x] Consequences that ripple forward into future sessions
\end{itemize}

\subsection*{Excellence Indicators}
\begin{itemize}[leftmargin=*]
  \item [x] Mechanical innovation (Memory Resonance) serves narrative about experience commodification
  \item [x] Player agency reinforced through memory choice consequences
  \item [x] Thematic consistency maintained through identity vs. power tension
  \item [x] Multiple resolution paths with different philosophical implications
  \item [x] Consequences feel earned and tie to player choices
  \item [x] Scalability options clearly marked for different tiers
  \item [x] GM support materials comprehensive for memory management
  \item [x] Safety considerations around identity loss and memory manipulation
\end{itemize}

\vspace{1em}
\noindent\rule{\textwidth}{0.4pt}

\textit{The Memory Merchant's Labyrinth} challenges players to confront what makes them who they are, while providing concrete mechanical systems for exploring the intangible nature of memory and identity. The adventure's strength lies in making abstract concepts tactile through the Memory Resonance system and Identity Stability resource, creating a unique experience that could only exist in Fate's Edge's narrative-first framework.

\end{document}
