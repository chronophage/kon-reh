\documentclass{article}
\usepackage[utf8]{inputenc}
\usepackage{geometry}
\geometry{a4paper, margin=1in}
\usepackage{graphicx}
\usepackage{fancyhdr}
\usepackage{titling}
\usepackage{xcolor}
\usepackage{tcolorbox}
\usepackage{array}
\usepackage{longtable}
\usepackage{enumitem}
\usepackage{framed}
\usepackage{multicol}

\definecolor{adventureblue}{RGB}{40, 80, 120}
\definecolor{mechanicgreen}{RGB}{40, 120, 80}
\definecolor{storyred}{RGB}{120, 40, 40}

\newtcolorbox{adventurebox}{
    colback=adventureblue!10,
    colframe=adventureblue,
    boxrule=0.5pt,
    arc=3mm,
    title=\textbf{Adventure Element}
}

\newtcolorbox{mechanicbox}{
    colback=mechanicgreen!10,
    colframe=mechanicgreen,
    boxrule=0.5pt,
    arc=3mm,
    title=\textbf{Mechanics Integration}
}

\newtcolorbox{storybox}{
    colback=storyred!10,
    colframe=storyred,
    boxrule=0.5pt,
    arc=3mm,
    title=\textbf{Story Beat}
}

\pagestyle{fancy}
\fancyhf{}
\rhead{The Shifting City of Chantelune}
\lhead{Fate's Edge Adventure}
\rfoot{\thepage}

\title{The Shifting City of Chantelune}
\author{Fate's Edge Adventure for Warhammer 40k Fans}
\date{}

\begin{document}

\maketitle

\begin{center}
\textit{"The stones remember what the living forget. In Chantelune, the city remembers too well."}
\end{center}

\begin{abstract}
The city of Chantelune, jewel of the Vhasian realm, is unraveling. During the Blood Moon Rising, the city shifts at night—buildings move, streets rearrange, and the dead walk the streets. The city is being reshaped by something that should not be, a breach in reality that threatens to consume the entire realm. This adventure is designed for Fate's Edge and specifically targets fans of Warhammer 40k, preserving grimdark atmosphere and tactical combat while emphasizing narrative consequences. The city itself is alive and hostile, with a "City Mood Clock" tracking its state from Stable to Collapsing. Players must survive the shifting city, find the source of the shifting, and seal the breach before the city is lost forever.
\end{abstract}

\section{Adventure Overview}

\begin{adventurebox}
\textbf{Setting:} The city of Chantelune, in the Vhasian realm \\
\textbf{Tier:} II--III (can be adjusted for lower tiers) \\
\textbf{Themes:} Grimdark, survival horror, tactical combat \\
\textbf{Player Count:} 3--5 \\
\textbf{Estimated Play Time:} 3--4 sessions
\end{adventurebox}

Chantelune is the crown jewel of the Vhasian realm, a city of ancient spires and cobbled streets that has stood for centuries. But during the Blood Moon Rising, the city has begun to shift—buildings move at night, streets rearrange, and the dead walk the streets. The city is being reshaped by something that should not be, a breach in reality that threatens to consume the entire realm.

The adventure is structured around three major beats, each with its own clocks and challenges. The city itself is alive and hostile, with a "City Mood Clock" tracking its state from Stable to Collapsing. As the mood worsens, the city's architecture becomes more dangerous and unpredictable.

\begin{storybox}
\textbf{The Hook:} The PCs arrive in Chantelune during the Blood Moon Rising. The city gates are locked from within, and no one is answering. As they approach, they see buildings shifting position and hear the moans of the dead. A desperate courier emerges from a side alley, begging for help: "The city is eating itself! Find the source before it's too late!"
\end{storybox}

\section{City Mood Clock}

The City Mood Clock tracks the city's state, affecting Position and Difficulty Values throughout the adventure.

\begin{center}
\begin{longtable}{|c|c|c|c|}
\hline
\textbf{Stable} & \textbf{Troubled} & \textbf{Unraveling} & \textbf{Collapsing} \\
\hline
(0--2) & (3--5) & (6--8) & (9--10) \\
\hline
\end{longtable}
\end{center}

\begin{itemize}
    \item \textbf{Stable (0--2 segments):} The city is still recognizable. Position is typically Dominant or Controlled for navigation. DV for navigation checks is -1.
    \item \textbf{Troubled (3--5 segments):} The city shows signs of shifting. Streets change slowly. Position is typically Controlled. DV for navigation checks is normal.
    \item \textbf{Unraveling (6--8 segments):} The city is actively shifting. Streets change position each scene. Position is typically Controlled or Desperate. DV for navigation checks is +1.
    \item \textbf{Collapsing (9--10 segments):} The city is falling apart. Streets shift during scenes. Position is typically Desperate. DV for navigation checks is +2.
\end{itemize}

\textbf{How the Mood Changes:}
\begin{itemize}
    \item \textbf{Worsens by 1:} On a miss when navigating the city, when the players cause damage to the city, when the players ignore a "shifting warning."
    \item \textbf{Improves by 1:} On a clean success when finding a stable path, when the players successfully interact with a "city spirit," when the players perform a ritual to stabilize the city.
\end{itemize}

\section{Adventure Beats}

\subsection{Beat 1: Survive the Night}

\begin{mechanicbox}
\textbf{City Mood Clock:} Begins at Troubled (3 segments) \\
\textbf{Primary Clock:} Night Survival [6] \\
\textbf{Key Mechanics:} Position, Story Beats, Tactical Clocks
\end{mechanicbox}

The first challenge is simply surviving the first night in Chantelune. The city shifts at night, and the dead walk the streets. The players must find shelter before dawn.

\subsubsection{The Shifting Streets}

As night falls, the streets of Chantelune begin to shift. Buildings move slowly, streets rearrange, and familiar landmarks disappear. The players must navigate the shifting city to find safe shelter.

\begin{itemize}
    \item \textbf{Position:} Controlled (the players have time to react)
    \item \textbf{DV:} 3 (pressured; the city is shifting unpredictably)
    \item \textbf{Skill:} Survival (to navigate) or Lore (to understand the city's patterns)
    \item \textbf{Effect:} Standard (finding a safe path)
\end{itemize}

\begin{storybox}
\textbf{Story Beat:} If the players fail, the streets shift violently. The City Mood worsens by 1, and the GM spends 1 SB to create a complication (e.g., a dead end that requires backtracking, a street that narrows dangerously, or a section of the street that collapses into darkness).
\end{storybox}

\subsubsection{The Dead Walk}

The dead of Chantelune rise during the Blood Moon Rising. They are not traditional zombies but fragmented spirits, trapped between life and death, drawn to the shifting city.

\begin{itemize}
    \item \textbf{Position:} Desperate (the dead are relentless)
    \item \textbf{DV:} 4 (hard; the dead are numerous and persistent)
    \item \textbf{Skill:} Melee or Ranged (to fight) or Presence (to drive them back)
    \item \textbf{Effect:} Standard (repelling the dead)
\end{itemize}

The dead have a \textbf{Dead Tide Clock} [4]:

\begin{itemize}
    \item \textbf{-1 per successful repulsion}
    \item \textbf{+1 per failed repulsion}
    \item \textbf{+1 per player harmed by the dead}
\end{itemize}

When the clock reaches 4, the dead overwhelm the players' position, forcing them to flee.

\begin{storybox}
\textbf{Story Beat:} On a partial success when fighting the dead, the City Mood worsens by 1 as the players' violence disturbs the city's balance. The dead are not inherently hostile—they are lost and confused—and harming them may make things worse.
\end{storybox}

\subsubsection{Finding Shelter}

The players must find shelter before dawn. Possible options include:

\begin{itemize}
    \item \textbf{The Grand Cathedral:} A massive stone structure that seems to resist the shifting. However, it is guarded by a fanatical sect of monks.
    \item \textbf{The Catacombs:} An underground network that is less affected by the shifting. However, it is filled with older, more dangerous spirits.
    \item \textbf{The Iron Foundry:} A place of industry and order that resists the chaos. However, it is controlled by a ruthless guild.
\end{itemize}

\begin{itemize}
    \item \textbf{Position:} Controlled (if they choose wisely) or Desperate (if they choose poorly)
    \item \textbf{DV:} 3 (pressured; time is running out)
    \item \textbf{Skill:} Survival (to find the best option) or Insight (to understand the implications)
    \item \textbf{Effect:} Standard (reaching shelter before dawn)
\end{itemize}

\begin{storybox}
\textbf{Story Beat:} If the players reach shelter before dawn, the City Mood improves by 1 as the city settles for a brief moment. However, the shelter is not safe forever—the shifting will resume at the next Blood Moon.
\end{storybox}

\subsection{Beat 2: Find the Source}

\begin{mechanicbox}
\textbf{City Mood Clock:} Depends on previous actions (starts at Troubled or Unraveling) \\
\textbf{Primary Clock:} Shifting Source [8] \\
\textbf{Key Mechanics:} Position Shifts, Clocks, Tactical Movement
\end{mechanicbox}

The next challenge is to find the source of the shifting. The players must navigate the unstable city during the day (when it is less volatile but still unpredictable) to track down the origin of the breach.

\subsubsection{The Shifting City During the Day}

The city shifts less dramatically during the day, but it is still unstable. Streets may rearrange, but more slowly. The players must navigate the city while it is in a state of flux.

\begin{itemize}
    \item \textbf{Position:} Controlled (the city is less volatile during the day)
    \item \textbf{DV:} 3 (pressured; the city is still unpredictable)
    \item \textbf{Skill:} Survival (to navigate) or Insight (to notice patterns)
    \item \textbf{Effect:} Standard (finding a path through the city)
\end{itemize}

\begin{storybox}
\textbf{Story Beat:} If the players fail, the City Mood worsens by 1 and the GM spends 1 SB to create a complication (e.g., a building they need to enter has shifted to an inaccessible location, or a crucial path has been blocked).
\end{storybox}

\subsubsection{The City Spirits}

The city of Chantelune is filled with spirits—fragments of the city's history that have taken on a life of their own. Some are helpful, while others are hostile.

\begin{itemize}
    \item \textbf{Position:} Controlled (the spirits are neutral)
    \item \textbf{DV:} 3 (pressured; the spirits are wary)
    \item \textbf{Skill:} Presence (to communicate) or Lore (to understand their purpose)
    \item \textbf{Effect:} Standard (gaining the spirits' trust)
\end{itemize}

If the players successfully communicate with the spirits, they may learn valuable information about the source of the shifting. If the players attack the spirits, they will become hostile and the City Mood worsens by 2.

\subsubsection{The Shifting Source Clock}

The Shifting Source Clock [8] tracks the players' progress toward finding the source:

\begin{itemize}
    \item \textbf{+1 per successful navigation check}
    \item \textbf{+2 per failed navigation check (GM spends 1 SB)}
    \item \textbf{+1 per successful interaction with city spirits}
    \item \textbf{+0 if using a city map or guide}
\end{itemize}

When the clock fills, the players have located the source of the shifting—a massive breach in reality at the heart of the city.

\begin{storybox}
\textbf{Story Beat:} When the Shifting Source Clock fills, the players discover the breach, but the City Mood worsens by 1 as their presence disturbs the unstable reality. The breach is guarded by the \textbf{Shifting Guardian}, a creature born from the breach itself.
\end{storybox}

\subsection{Beat 3: The Broken Gate}

\begin{mechanicbox}
\textbf{City Mood Clock:} Depends on previous actions (starts at Troubled or Unraveling) \\
\textbf{Primary Clock:} Reality Breach [10] \\
\textbf{Key Mechanics:} Sealed Gate Runekeeper, Position, Tactical Clocks
\end{mechanicbox}

The final challenge is sealing the breach before the city is consumed. The breach is a wound in reality at the heart of Chantelune, and it is growing larger with each Blood Moon.

\subsubsection{The Reality Breach}

The breach is a swirling vortex of unstable reality, where time and space are distorted. It is the source of the city's shifting and the dead that walk the streets.

\begin{itemize}
    \item \textbf{Position:} Desperate (the breach is actively destabilizing reality)
    \item \textbf{DV:} 5 (extreme; the breach is a cosmic threat)
    \item \textbf{Skill:} Lore (to understand the breach) or Arcana (if a Caster) or Runekeeper (if a Runekeeper)
    \item \textbf{Effect:} Standard (assessing the breach)
\end{itemize}

\begin{storybox}
\textbf{Story Beat:} On a success, the players understand how to seal the breach but trigger a defensive mechanism—the City Mood worsens by 1 and the GM spends 1 SB to create a complication (e.g., the breach spews forth a wave of unstable reality, or the Shifting Guardian attacks).
\end{storybox}

\subsubsection{The Shifting Guardian}

The Shifting Guardian is a manifestation of the breach itself—a creature of shifting reality that defends the breach.

\begin{center}
\textbf{Shifting Guardian}
\end{center}

\begin{itemize}
    \item \textbf{Position:} Desperate (the Guardian is actively attacking)
    \item \textbf{DV:} 4 (hard; the Guardian is powerful)
    \item \textbf{Skills:} Spirit (4), Wits (3), Body (2)
    \item \textbf{Harm:} 2 (reality distortion)
    \item \textbf{Special:} \textbf{Shifting Form}—each time it is attacked, the Guardian can shift position, forcing the attacker to re-roll one die (desperate effect)
\end{itemize}

The Shifting Guardian has a \textbf{Guardian Health Clock} [4]:

\begin{itemize}
    \item \textbf{-1 per successful attack}
    \item \textbf{-2 if using a Sealed Gate symbol to disrupt its form}
    \item \textbf{+1 per failed attack (the Guardian feeds on the players' energy)}
\end{itemize}

\subsubsection{Sealing the Breach}

While fighting the Guardian, the players can attempt to seal the breach. This requires:

\begin{itemize}
    \item \textbf{Position:} Controlled or Desperate (depending on the fight)
    \item \textbf{DV:} 5 (extreme; the breach is a cosmic threat)
    \item \textbf{Skill:} Lore or Runekeeper (if a Runekeeper of the Sealed Gate)
    \item \textbf{Effect:} Great (if successful, the breach begins to close)
\end{itemize}

\begin{storybox}
\textbf{Story Beat:} Each time the players successfully work on sealing the breach, the City Mood improves by 1. On a miss, the City Mood worsens by 1 and the GM spends 2 SB to create a major complication (e.g., a section of reality tears away, requiring a replacement, or the breach grows larger).
\end{storybox}

\subsubsection{The Reality Breach Clock}

The Reality Breach Clock [10] tracks the state of the breach:

\begin{itemize}
    \item \textbf{-1 per successful sealing attempt}
    \item \textbf{+1 per failed sealing attempt or Guardian attack}
    \item \textbf{-2 if the Guardian is defeated}
\end{itemize}

When the clock reaches 0, the breach is sealed and the city stabilizes. When it reaches 10, the city is consumed by the breach, taking the realm with it.

\section{NPCs}

\subsection{The Shifting Monks}

\begin{itemize}
    \item \textbf{Description:} A fanatical sect of monks who guard the Grand Cathedral. They believe the shifting is divine punishment and resist any attempt to stop it.
    \item \textbf{Role:} They control the Grand Cathedral, one of the few stable places in the city.
    \item \textbf{Secret:} They know how to temporarily stabilize the city using ancient rites, but they believe the city must be destroyed to be reborn.
\end{itemize}

\subsection{The Shifting Guardian}

\begin{itemize}
    \item \textbf{Description:} A creature of shifting reality, with no fixed form. It appears as a swirling vortex of broken architecture and unstable energy.
    \item \textbf{Role:} The primary antagonist of the final beat. It defends the breach and is a manifestation of the shifting city.
    \item \textbf{Secret:} It is not inherently evil—it is a natural defense mechanism of reality itself, drawn to the breach. It can be reasoned with or banished, but not destroyed.
\end{itemize}

\subsection{The City Spirits}

\begin{itemize}
    \item \textbf{Description:} Fragments of the city's history that have taken on a life of their own. They appear as translucent figures of people, buildings, or objects from Chantelune's past.
    \item \textbf{Role:} They provide information about the city and the breach, but they are confused and often contradictory.
    \item \textbf{Secret:} Some spirits are aware of the breach and want to help seal it, while others are part of the problem and resist any attempt to stop the shifting.
\end{itemize}

\section{Adventure Conclusion}

The adventure has three possible conclusions, depending on the players' actions:

\subsection{Success: The Breach Sealed}

If the players seal the breach and defeat the Shifting Guardian, the city stabilizes. The shifting stops, and the dead return to their graves. The people of Chantelune are saved, and the realm is preserved.

The players receive:
\begin{itemize}
    \item A \textbf{Seal of Stability} (a minor asset that can temporarily stabilize reality in a small area)
    \item A \textbf{City Blessing} (a one-time +2 to any roll)
    \item A \textbf{Rune of the Sealed Gate} (a minor talent that allows the player to create a small area of stable reality)
\end{itemize}

The City Mood Clock resets to Stable, and Chantelune becomes a safe haven for the players in future adventures.

\subsection{Partial Success: The Breach Contained}

If the players defeat the Guardian but fail to fully seal the breach, the shifting is reduced but not eliminated. The city stabilizes somewhat, but it remains vulnerable to future Blood Moon Risings.

The players receive:
\begin{itemize}
    \item A \textbf{Seal of Stability} (a minor asset that can temporarily stabilize reality in a small area)
    \item A \textbf{City Blessing} (a one-time +2 to any roll)
    \item A \textbf{Breach Fragment} (a minor asset that can be used to manipulate reality, but with a risk of worsening the City Mood)
\end{itemize}

The City Mood Clock resets to Troubled, and Chantelune remains a place of danger that requires regular attention.

\subsection{Failure: The City Consumed}

If the City Mood Clock reaches Collapsing and the Reality Breach Clock fills, the city is consumed by the breach. The players must make a final Survival check (DV 5) to escape the collapsing city. If they succeed, they escape with their lives but Chantelune is lost forever. If they fail, they are consumed by the breach, becoming part of the shifting city themselves.

\section{Fate's Edge Mechanics Integration}

\begin{mechanicbox}
\textbf{Sealed Gate Runekeeper Integration:} The adventure is designed to showcase the Sealed Gate Runekeeper path. Only a Runekeeper of the Sealed Gate can create "sanctuary" zones that resist the shifting.
\end{mechanicbox}

\subsection{Sealed Gate Runekeeper}

A Runekeeper of the Sealed Gate has special abilities related to the breach:

\begin{itemize}
    \item \textbf{Sanctuary Zone:} Can create a small area of stable reality (DV 3, Controlled/Standard), improving Position for the party and preventing the City Mood from worsening in that area.
    \item \textbf{Sealing Ritual:} Can perform a ritual to close the breach (DV 4, Controlled/Standard), reducing the Reality Breach Clock by 1.
    \item \textbf{Breach Interface:} Can directly interact with the breach (DV 3, Controlled/Standard), understanding it with greater effect.
\end{itemize}

\textbf{Obligation:} Each time a Runekeeper uses their abilities, they mark 1 Obligation to the Sealed Gate. If they exceed their Obligation Capacity (Spirit + Presence), they immediately take 1 Fatigue.

\subsection{Position and Effect}

The shifting city creates dynamic Position changes:

\begin{itemize}
    \item \textbf{Dominant:} When the players have a moment to prepare or find a stable position (e.g., during the day or in a sanctuary zone).
    \item \textbf{Controlled:} The default state in most areas of the city.
    \item \textbf{Desperate:} When the city is shifting violently or the players are in danger (e.g., at night or near the breach).
\end{itemize}

\subsection{Story Beats and Boons}

\begin{itemize}
    \item \textbf{Story Beats:} The city generates SB through its shifting. Each time the City Mood worsens, the GM gains 1 SB. The GM should spend SB on complications related to the city's instability.
    \item \textbf{Boons:} Players earn Boons for meaningful failures (e.g., failing to navigate the city but learning its patterns). They can spend Boons to improve their Position or to stabilize the city.
\end{itemize}

\begin{tcolorbox}[colback=blue!5,colframe=blue!50!black,title=Design Note]
This adventure was specifically designed to preserve the grimdark atmosphere of Warhammer 40k while emphasizing Fate's Edge's narrative flexibility. The shifting city creates dynamic battlefields that require tactical thinking, while the City Mood Clock and city spirits add narrative consequences that go beyond mere combat. The adventure is designed so that players who focus on understanding and stabilizing the city (rather than fighting) will have the best chance of success, reflecting Fate's Edge philosophy of "narrative first, mechanics serving the story."
\end{tcolorbox}

\section{Adaptation Notes for GMs}

\subsection{For Warhammer 40k Fans}

This adventure preserves the grimdark atmosphere Warhammer fans love while adding narrative depth through the Position and Effect system. The shifting city creates dynamic battlefields that require tactical thinking, and the adventure includes clear survival mechanics:

\begin{itemize}
    \item \textbf{Grimdark Atmosphere:} Hope is rare, and survival is a daily struggle. The city itself is hostile, and even success comes at a cost.
    \item \textbf{Tactical Combat:} The shifting city creates dynamic battlefields that require players to adapt their tactics each scene.
    \item \textbf{The Warp as Magic:} The shifting is caused by a breach in reality, similar to the Warp in Warhammer 40k.
\end{itemize}

\subsection{Scaling for Higher Tiers}

For Tier III+ characters:

\begin{itemize}
    \item Increase the DVs by 1 for all checks.
    \item Increase the size of the clocks by 2 (e.g., Night Survival [8] instead of [6]).
    \item Add a second Shifting Guardian or a more powerful variant.
    \item The Shifting Guardian gains additional abilities (e.g., can shift reality for a single player, forcing them to re-roll all dice).
\end{itemize}

\section{Conclusion}

The Shifting City of Chantelune is an adventure that captures the grimdark essence of Warhammer 40k while embracing the narrative flexibility of Fate's Edge. The shifting city creates dynamic challenges that require both strategic thinking and narrative creativity, and the City Mood Clock ensures that every action has consequences that ripple through the adventure.

The adventure ends with a moment of triumph—not through defeating a monster, but through sealing a breach in reality and restoring stability to a broken city. This reflects Fate's Edge's core philosophy: that narrative comes first, and mechanics exist to serve the story.

\begin{center}
\textit{"In the city that shifts, the only constant is the will to endure. Even when the streets move beneath your feet, you keep walking."}
\end{center}

\end{document}