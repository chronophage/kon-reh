\documentclass[12pt]{article}
\usepackage[margin=1in]{geometry}
\usepackage{enumitem}
\usepackage{tabularx}
\usepackage{fancyhdr}
\usepackage{titlesec}
\usepackage{xcolor}
\usepackage{multicol}
\usepackage{tikz}
\usepackage{fancyvrb}

\pagestyle{fancy}
\fancyhf{}
\rhead{Fate's Edge}
\lhead{Monster Manual}
\rfoot{\thepage}

\titleformat{\section}{\large\bfseries\color{red}}{\thesection}{1em}{}
\titleformat{\subsection}{\bfseries}{}{0em}{}

\newcommand{\dice}[1]{\texttt{#1}}
\newcommand{\dv}[1]{\textbf{DV #1}}
\newcommand{\cp}[1]{\textbf{SB: #1}}
\newcommand{\harm}[1]{\textbf{Harm #1}}
\newcommand{\threat}[1]{\textbf{Threat: #1}}

\begin{document}

\begin{center}
{\LARGE \textbf{FATE'S EDGE MONSTER MANUAL}} \\
{\large \textbf{Creatures of the Amaranthine Sea Region}}
\end{center}

\section*{USING THIS MONSTER MANUAL}

In Fate's Edge, monsters and creatures are not mere stat blocks - they are story elements with motivations, weaknesses, and consequences. Each entry includes:
\begin{itemize}
\item \textbf{Threat Level}: Indicates relative danger (Minor, Moderate, Major, Epic)
\item \textbf{Harm Rating}: What injury they inflict
\item \textbf{Story Beats}: What narrative complications they generate
\item \textbf{Motivations}: Why they act as they do
\item \textbf{Weaknesses}: How clever players can overcome them
\end{itemize}

\section{BEASTS AND PREDATORS}

\subsection*{Blackwood Wolf}

\threat{Moderate} \\
\harm{=} \\
\cp{Pack tactics generate 1 SB per additional wolf}

\vspace{0.5em}
\textbf{Description:} Cunning predators of Acasia's cursed forests. Larger than common wolves, with eyes that gleam with unnatural intelligence. They hunt in coordinated packs and seem to communicate through low, rumbling growls that echo strangely.

\textbf{Motivations:} Territory defense, hunger, protecting young

\textbf{Abilities:}
\begin{itemize}
\item \textbf{Pack Hunters}: +1 die when 2+ wolves attack the same target
\item \textbf{Cursed Senses}: Can track cursed or supernatural beings with +2 dice
\item \textbf{Flanking}: Gain start Controlled when attacking from opposite sides
\end{itemize}

\textbf{Weaknesses:} 
\begin{itemize}
\item Fear fire (start Desperate in burning areas)
\item Distracted by bright lights or loud noises
\item Pack breaks if alpha is killed or driven off
\end{itemize}

\textbf{Sample Encounter:}
A pack of 3 Blackwood Wolves stalks the PCs through the cursed Blackwood. The GM can spend their SB to:
\begin{itemize}
\item 1 SB: Wolves coordinate attack (+1 die next round)
\item 2 SB: One wolf circles behind PCs (flanking advantage)
\item 3 SB: Howling alerts more wolves (reinforcements arrive)
\end{itemize}

\subsection*{Mist Wraith}

\threat{Major} \\
\harm{>} \\
\cp{Uncanny presence generates 2 SB per scene}

\vspace{0.5em}
\textbf{Description:} Semi-corporeal spirits that emerge from the deep mists of the Mistlands. They appear as translucent humanoid figures with hollow eyes and mouths that seem to whisper in forgotten tongues. Often guardians of ancient burial sites or places of great sorrow.

\textbf{Motivations:} Protecting sacred sites, seeking justice for past wrongs, feeding on life force

\textbf{Abilities:}
\begin{itemize}
\item \textbf{Incorporeal}: Cannot be harmed by non-magical weapons
\item \textbf{Life Drain}: Successful attack inflicts Fatigue 1 in addition to harm
\item \textbf{Mist Form}: Can become gaseous to avoid physical attacks
\item \textbf{Terrifying Presence}: Opponents start Desperate vs. social/mental actions
\end{itemize}

\textbf{Weaknesses:}
\begin{itemize}
\item Vulnerable to blessed/clean weapons
\item Sunlight forces solid form (Harm >> becomes =)
\item Cannot cross running water
\item Specific unfinished business can be resolved
\end{itemize}

\textbf{Sample Encounter:}
A Mist Wraith guards an ancient Aeler burial chamber. The GM can spend their SB to:
\begin{itemize}
\item 1 SB: Thick mist reduces visibility (-1 die to ranged attacks)
\item 2 SB: Whispered voices cause confusion (players must reroll one success)
\item 3 SB: Life drain affects multiple targets
\item 4+ SB: Wraith merges with mist, becomes nearly impossible to target
\end{itemize}

\subsection*{Dire Boar}

\threat{Moderate} \\
\harm{>} \\
\cp{Territorial rage generates 1 SB when provoked}

\vspace{0.5em}
\textbf{Description:} Massive wild boars found in the deep forests and marshes of the region. Twice the size of common boars, with tusks like curved daggers and a temperament to match. Often found near ancient ruins or sacred sites.

\textbf{Motivations:} Territory defense, mating season aggression, protecting food sources

\textbf{Abilities:}
\begin{itemize}
\item \textbf{Charge}: +2 effect on first attack if has clear run
\item \textbf{Thick Hide}: Reduce all harm by one level (severe becomes moderate, etc.)
\item \textbf{Relentless}: Cannot be deterred by normal fear effects
\item \textbf{Trample}: Can attack multiple adjacent targets in one action
\end{itemize}

\textbf{Weaknesses:}
\begin{itemize}
\item Poor eyesight (easily distracted by movement)
\item Vulnerable in soft underbelly
\item Can be led away from territory with food
\item Loud noises can startle them
\end{itemize}

\textbf{Sample Encounter:}
A Dire Boar has made its lair in an ancient shrine. The GM can spend their SB to:
\begin{itemize}
\item 1 SB: Boar charges, knocking opponent prone
\item 2 SB: Tusks gore armor/weapons (gear damage)
\item 3 SB: Boar's roar alerts other wildlife (environmental complications)
\end{itemize}

\section{HUMANOID THREATS}

\subsection*{Bandit Skirmisher}

\threat{Minor} \\
\harm{=} \\
\cp{Criminal cunning generates 1 SB in urban environments}

\vspace{0.5em}
\textbf{Description:} Desperate outlaws who prey on travelers and merchants. Poorly equipped but numerous and desperate. Often found in the lawless regions of Acasia or along dangerous trade routes.

\textbf{Motivations:} Survival, greed, revenge against authority

\textbf{Abilities:}
\begin{itemize}
\item \textbf{Numbers}: +1 die when fighting in groups of 3+
\item \textbf{Ambush}: Start Controlled when attacking from surprise
\item \textbf{Dirty Fighting}: Can generate 1 SB even on successful defense
\item \textbf{Run Away}: Will flee if clearly outmatched
\end{itemize}

\textbf{Weaknesses:}
\begin{itemize}
\item Poor quality weapons/armor
\item Low morale (flee at first serious injury)
\item Often have bounty or criminal history
\item Divided loyalties within groups
\end{itemize}

\textbf{Sample Encounter:}
A group of 4 Bandit Skirmishers attempts to rob the PCs on a lonely road. The GM can spend their SB to:
\begin{itemize}
\item 1 SB: One bandit flanks from unexpected direction
\item 2 SB: Hidden 5th bandit joins the fight
\item 3 SB: Bandits use environment (rocks, mud) against PCs
\end{itemize}

\subsection*{Ykrul Raider}

\threat{Moderate} \\
\harm{>} \\
\cp{Steppe tactics generate 2 SB in open terrain}

\vspace{0.5em}
\textbf{Description:} Nomadic warriors from the Ykrul steppes. Expert horsemen and archers, they raid settlements and trade caravans. Painted with ritual markings and carrying curved weapons of exceptional quality.

\textbf{Motivations:} Honor, plunder, proving martial prowess, clan loyalty

\textbf{Abilities:}
\begin{itemize}
\item \textbf{Mounted Combat}: +2 dice when fighting from horseback
\item \textbf{Archery}: +1 effect on ranged attacks at long distance
\item \textbf{Steppe Survival}: +2 dice on survival/tracking in grasslands
\item \textbf{Wolfskin Cloak}: +1 die on intimidation/social actions
\end{itemize}

\textbf{Weaknesses:}
\begin{itemize}
\item Disadvantaged in close quarters or urban environments
\item Honor-bound (won't attack from complete surprise)
\item Superstitious about certain omens/signs
\item Dependent on horse for mobility advantage
\end{itemize}

\textbf{Sample Encounter:}
A band of 3 Ykrul Raiders harasses a caravan. The GM can spend their SB to:
\begin{itemize}
\item 1 SB: Raider circles around to flank
\item 2 SB: Volley of arrows from multiple attackers
\item 3 SB: War cries demoralize caravan guards
\item 4+ SB: Reinforcements arrive from nearby ridge
\end{itemize}

\section{SUPERNATURAL ENTITIES}

\subsection*{Shadow Stalker}

\threat{Major} \\
\harm{>} \\
\cp{Unnatural presence generates 2-3 SB per scene}

\vspace{0.5em}
\textbf{Description:} Creatures of living darkness that hunt in places where light fears to go. They appear as shifting pools of shadow that move with purpose and intelligence. Often found in ancient ruins, deep caves, or the darkest hours of night.

\textbf{Motivations:} Feeding on fear, protecting dark places, serving ancient masters

\textbf{Abilities:}
\begin{itemize}
\item \textbf{Shadow Form}: Can pass through small openings, ignore non-magical barriers
\item \textbf{Fear Feed}: Grows stronger as opponents become afraid (start Desperate)
\item \textbf{Darkness Manipulation}: Can extinguish lights, create areas of shadow
\item \textbf{Silent Movement}: Cannot be detected by normal hearing
\end{itemize}

\textbf{Weaknesses:}
\begin{itemize}
\item Vulnerable to bright light (start Controlled vs. illuminated targets)
\item Holy symbols/blessed items cause them harm
\item Cannot cross consecrated ground
\item Often bound to specific locations or tasks
\end{itemize}

\textbf{Sample Encounter:}
A Shadow Stalker hunts the PCs through an abandoned temple. The GM can spend their SB to:
\begin{itemize}
\item 1 SB: Shadows deepen, reducing visibility
\item 2 SB: Multiple stalkers emerge from darkness
\item 3 SB: Target's shadow turns against them
\item 4+ SB: Stalker merges with darkness, becomes nearly invisible
\end{itemize}

\subsection*{Curse Echo}

\threat{Moderate} \\
\harm{=} \\
\cp{Paradoxical nature generates 1-2 SB unpredictably}

\vspace{0.5em}
\textbf{Description:} Manifestations of Acasian curses - repeated moments of trauma or tragedy that play out endlessly. They appear as ghostly reenactments of past events, unable to perceive the present but affecting it nonetheless.

\textbf{Motivations:} Repeating their final moments, seeking resolution, protecting something they died for

\textbf{Abilities:}
\begin{itemize}
\item \textbf{Temporal Loop}: Actions repeat in predictable patterns
\item \textbf{Curse Resonance}: Other supernatural effects in area gain +1 die
\item \textbf{Unfinished Business}: Cannot be permanently defeated until their purpose is fulfilled
\item \textbf{Echo Sight}: Can perceive other supernatural entities clearly
\end{itemize}

\textbf{Weaknesses:}
\begin{itemize}
\item Follows strict behavioral patterns (predictable)
\item Cannot adapt to new situations
\item Specific actions can break their cycle
\item Often vulnerable during key moments of their loop
\end{itemize}

\textbf{Sample Encounter:}
A Curse Echo of a murdered merchant replays his final journey through a marketplace. The GM can spend their SB to:
\begin{itemize}
\item 1 SB: Echo's presence causes others to repeat past mistakes
\item 2 SB: Environmental objects become cursed (doors jam, coins turn to leaves)
\item 3 SB: Multiple echoes manifest (entire scene repeats)
\end{itemize}

\section{GIANTS AND MONSTROSITIES}

\subsection*{Stone Giant Elder}

\threat{Epic} \\
\harm{>>} \\
\cp{Ancient power generates 3-4 SB per action}

\vspace{0.5em}
\textbf{Description:} Ancient beings of living stone who dwell in the deepest mountain halls. Towering over humans, with skin like weathered granite and eyes that glow with inner fire. They remember the world's first days and speak in voices like grinding stone.

\textbf{Motivations:} Guarding ancient secrets, maintaining geological balance, testing worthiness of mortals

\textbf{Abilities:}
\begin{itemize}
\item \textbf{Massive Strength}: +3 effect on physical actions
\item \textbf{Stone Shape}: Can manipulate earth and stone as easily as clay
\item \textbf{Ageless Wisdom}: +2 dice on all knowledge/lore rolls
\item \textbf{Earth Sense}: Cannot be surprised underground, detect all movement
\item \textbf{Crushing Grip}: Successful grapple inflicts ongoing Harm = per round
\end{itemize}

\textbf{Weaknesses:}
\begin{itemize}
\item Extremely slow (only one action per two rounds in normal combat)
\item Vulnerable to sonic attacks (thunder damage)
\item Can be distracted by philosophical debates
\item Bound by ancient oaths and geasa
\end{itemize}

\textbf{Sample Encounter:}
A Stone Giant Elder blocks the PCs' path to an ancient dwarven vault. The GM can spend their SB to:
\begin{itemize}
\item 1 SB: Earth tremor destabilizes footing (all start Desperate)
\item 2 SB: Stone projectiles rain down from ceiling
\item 3 SB: Walls shift to trap intruders
\item 4+ SB: Elder calls upon mountain's will (area hazard clock advances)
\end{itemize}

\subsection*{Deep Drake}

\threat{Major} \\
\harm{>} \\
\cp{Predatory cunning generates 2-3 SB in confined spaces}

\vspace{0.5em}
\textbf{Description:} Massive serpentine predators that dwell in underground lakes and cavern systems. With scales like black glass and eyes that reflect no light, they are apex predators of the deep places. Their lairs are littered with treasure and bones.

\textbf{Motivations:} Hoarding treasure, defending territory, hunger for rare prey

\textbf{Abilities:}
\begin{itemize}
\item \textbf{Aquatic Mastery}: +2 dice in water environments
\item \textbf{Crushing Bite}: Successful attack ignores 1 point of armor
\item \textbf{Constrict}: Grapple +1 ongoing harm per round
\item \textbf{Treasure Sense}: Can detect precious metals and gems within 100 feet
\item \textbf{Amphibious}: Can survive indefinitely on land or water
\end{itemize}

\textbf{Weaknesses:}
\begin{itemize}
\item Vulnerable on land (half movement speed)
\item Sensitive to bright lights (start Desperate in well-lit areas)
\item Greed can be exploited with offered treasure
\item Cold temperatures slow their metabolism
\end{itemize}

\textbf{Sample Encounter:}
A Deep Drake guards an underwater passage in a flooded dwarven mine. The GM can spend their SB to:
\begin{itemize}
\item 1 SB: Drake dives, attacking from unexpected angle
\item 2 SB: Water currents hinder PC movement
\item 3 SB: Drake's roar echoes, alerting other underground dwellers
\item 4+ SB: Cave-in blocks escape routes
\end{itemize}

\section{FAE AND OTHERWORLDLY}

\subsection*{Redcap}

\threat{Moderate} \\
\harm{=} \\
\cp{Fae malevolence generates 2 SB through trickery}

\vspace{0.5em}
\textbf{Description:} Malevolent fey creatures from the Valewood, easily recognized by their distinctive red caps soaked in the blood of their victims. Small in stature but vicious in nature, they delight in causing suffering and breaking mortal oaths.

\textbf{Motivations:} Spreading chaos, breaking promises, causing pain for sport

\textbf{Abilities:}
\begin{itemize}
\item \textbf{Fae Speed}: Two actions per round in their favored environments
\item \textbf{Blood Magic}: Grows stronger when blood is spilled
\item \textbf{Oath Breaking}: Can sense and exploit mortal promises
\item \textbf{Illusion Craft}: Can create convincing false images
\item \textbf{Small Stature}: Can hide in spaces others cannot reach
\end{itemize}

\textbf{Weaknesses:}
\begin{itemize}
\item Bound by their own twisted sense of "fair play"
\item Iron weapons cause them severe harm
\item Cannot cross running water without invitation
\item Obsessed with specific taboos (often color, number, or action related)
\end{itemize}

\textbf{Sample Encounter:}
A Redcap has been hired to sabotage the PCs' mission in the Valewood. The GM can spend their SB to:
\begin{itemize}
\item 1 SB: Illusion makes PCs attack each other
\item 2 SB: Redcap exploits a previously made promise
\item 3 SB: More redcaps join the fight (they travel in murderous gangs)
\item 4+ SB: Ancient geas forces PCs to act against their interests
\end{itemit

\section{CREATING CUSTOM MONSTERS}

\subsection*{Threat Level Guidelines}

\begin{center}
\begin{tabularx}{\textwidth}{|l|X|X|}
\hline
\textbf{Level} & \textbf{Typical Harm} & \textbf{SB Generation} \\
\hline
Minor & Harm - or = & 1 SB per significant action \\
\hline
Moderate & Harm = or > & 1-2 SB per action \\
\hline
Major & Harm > or >> & 2-3 SB per action \\
\hline
Epic & Harm >> or † & 3-4+ SB per action \\
\hline
\end{tabularx}
\end{center}

\subsection*{Monster Design Template}

When creating custom creatures, consider these elements:

\textbf{Core Identity:}
\begin{itemize}
\item What is this creature's essential nature?
\item Where does it fit in the world's ecosystem?
\item What makes it unique or memorable?
\end{itemize}

\textbf{Narrative Function:}
\begin{itemize}
\item What role does it serve in the story?
\item How does it challenge player assumptions?
\item What themes does it represent?
\end{itemize}

\textbf{Mechanical Balance:}
\begin{itemize}
\item Is its threat level appropriate for the PCs?
\item Do its abilities create interesting tactical choices?
\item Does it generate SB in thematically appropriate ways?
\end{itemize}

\subsection*{SB Spending Philosophy}

Monsters should spend SB to:
\begin{itemize}
\item Escalate tension and stakes
\item Introduce new complications
\item Reflect their nature and motivations
\item Create interesting choices for players
\item Push the story forward
\end{itemize}

Avoid spending SB to:
\begin{itemize}
\item Simply make monsters "stronger"
\item Punish players for playing well
\item Create boring or repetitive complications
\item Ignore the fiction and creature's nature
\end{itemize}

\section{TACTICAL ENCOUNTERS}

\subsection*{Environmental Hazards}

Many monsters work best when combined with environmental threats:

\textbf{Cave-In Hazards:}
\begin{itemize}
\item Rockfall: Harm > to random target per round
\item Blocked exits: Escape clock advances +1
\item Dust cloud: -1 die to all ranged attacks
\end{itemize}

\textbf{Fire Hazards:}
\begin{itemize}
\item Spreading flames: Area hazard clock (6 segments)
\item Smoke: -1 die to perception, Fatigue 1 per round
\item Heat: Gear damage, -1 die to physical actions
\end{itemize}

\textbf{Water Hazards:}
\begin{itemize}
\item Rising water: Swimming required or Harm =
\item Strong current: -2 die to movement actions
\item Cold: Fatigue accumulation, reduced effectiveness
\end{itemize}

\subsection*{Mass Combat Creatures}

For large-scale battles, creatures can be grouped:

\textbf{Wolf Pack (6-10 individuals):}
\begin{itemize}
\item Treat as single Cap 3 unit
\item Harm =, but can flank effectively
\item SB: 1 per 3 wolves lost, 2 for pack coordination
\end{itemize}

\textbf{Bandit Company (20-30 individuals):}
\begin{itemize}
\item Treat as single Cap 4 unit
\item Mixed weapons, moderate armor
\item SB: 1 for morale effects, 2 for tactical maneuvers
\end{itemize}

\section{SAMPLE ENCOUNTER TABLES}

\subsection*{Road Encounter Table (d10)}
\begin{enumerate}
\item Bandit Skirmishers (2-4) - Minor threat
\item Merchant Caravan under attack - Social encounter
\item Dire Boar - Moderate physical threat
\item Broken bridge - Environmental hazard
\item Ykrul Raider patrol (1-2) - Moderate threat
\item Traveling merchant with valuable goods - Social opportunity
\item Curse Echo of previous traveler - Supernatural complication
\item Refugee family seeking help - Moral choice
\item Storm approaches - Environmental pressure
\item Nothing of note - Safe travel
\end{enumerate}

\subsection*{Forest Encounter Table (d10)}
\begin{enumerate}
\item Blackwood Wolf pack (3-5) - Moderate threat
\item Lost traveler - Social encounter
\item Ancient shrine guarded by curse - Supernatural
\item Poisonous plants - Environmental hazard
\item Bandit ambush - Minor-Moderate threat
\item Rare herbs/valuables to discover - Opportunity
\item Mist Wraith - Major supernatural threat
\item Fallen tree blocking path - Environmental obstacle
\item Fae circle performing ritual - Otherworldly encounter
\item Clearing with good camping spot - Beneficial
\end{enumerate}

\subsection*{Underground Encounter Table (d10)}
\begin{enumerate}
\item Cave-in - Environmental hazard
\item Deep Drake - Major threat
\item Lost dwarven patrol - Social encounter
\item Ancient vault with traps - Opportunity/Threat
\item Stone Giant Elder (rare) - Epic threat
\item Underground river - Environmental feature
\item Mineral deposit - Valuable resource
\item Fungal garden tended by creatures - Supernatural
\item Echoes of ancient battle - Historical mystery
\item Safe chamber for rest - Beneficial
\end{enumerate}

\subsection*{Urban Encounter Table (d10)}
\begin{enumerate}
\item Pickpockets/Bandit Skirmishers - Minor threat
\item Noble seeking assistance - Social opportunity
\item Guild dispute turning violent - Moderate threat
\item Festival/crowd scene - Social complexity
\item Fire breaking out - Environmental hazard
\item Official investigation - Legal complications
\item Curse manifestation - Supernatural threat
\item Merchant competition - Economic challenge
\item Political assassination attempt - Major threat
\item Nothing unusual - Normal city activity
\end{enumerate}

\section{MONSTER GROUPS AND ORGANIZATIONS}

\subsection*{The Pale Court}

A cabal of undead nobles who rule from a hidden realm between life and death. They appear as beautiful, ageless aristocrats but their touch brings cold and despair.

\textbf{Threat Level:} Major to Epic (individual members) \\
\textbf{Harm:} = to > \\
\textbf{SB Generation:} 2-4 SB per action

\textbf{Abilities:}
\begin{itemize}
\item \textbf{Undead Resilience}: Immune to disease, poison, and fatigue
\item \textbf{Charm Person}: Start Controlled on social actions vs. mortals
\item \textbf{Life Drain}: Successful attacks inflict Fatigue 1 + normal harm
\item \textbf{Realm Shifting}: Can transport willing subjects to their domain
\end{itemize}

\textbf{Weaknesses:}
\begin{itemize}
\item Cannot cross consecrated ground
\item Vulnerable to blessed weapons
\item Obsessed with mortal customs and etiquette
\item Cannot create without mortal cooperation
\end{itemize}

\subsection*{The Bone Merchants}

A guild of necromancers and grave-robbers who traffic in death-related services. They operate openly in certain cities and maintain a complex network of suppliers and customers.

\textbf{Threat Level:} Moderate to Major \\
\textbf{Harm:} = to > \\
\textbf{SB Generation:} 1-3 SB per encounter

\textbf{Abilities:}
\begin{itemize}
\item \textbf{Corpse Network}: Can animate dead bodies as servants
\item \textbf{Death Sense}: Can detect recent deaths within 1 mile
\item \textbf{Bargain Craft}: Skilled in contracts and legal loopholes
\item \textbf{Preservation Expertise}: Can maintain bodies/objects indefinitely
\end{itemize}

\textbf{Weaknesses:}
\begin{itemize}
\item Bound by contracts they write
\item Vulnerable to fire and holy magic
\item Obsessive about proper procedures
\item Cannot operate where death is forbidden
\end{itemize}

\section{REGIONAL BESTIARY INDEX}

\subsection*{Acasia - Broken Marches}
\begin{itemize}
\item Blackwood Wolf
\item Curse Echo
\item Bandit Skirmisher
\item Dire Boar
\item Shadow Stalker (in ruins)
\end{itemize}

\subsection*{Mistlands - Bells and Breath}
\begin{itemize}
\item Mist Wraith
\item Deep Drake
\item Bell-Spirit (minor)
\item Salt-Wight
\item Pall Guardian
\end{itemize}

\subsection*{Valewood - Empire Under Leaves}
\begin{itemize}
\item Redcap
\item Thorn Beast
\item Echo-Legionary
\item Fox-Headed Courier
\item Ancient Tree-Spirit
\end{itemize}

\subsection*{Aeler - Crowns and Under-Vaults}
\begin{itemize}
\item Stone Giant Elder
\item Vault Warden (undead)
\item Deep Drake
\item Geomantic Construct
\item Ancestral Guardian
\end{itemize}

\subsection*{Ykrul - Wolf Standards}
\begin{itemize}
\item Dire Wolf
\item Steppe Warg
\item Bone-Singer
\item Sky-Spirit Manifestation
\item Kurgan Wight
\end{itemize}

\section{USING MONSTERS IN CAMPAIGN PLAY}

\subsection*{Scaling Threats}

Adjust monster difficulty based on PC capabilities:
\begin{itemize}
\item \textbf{Rookies (0-40 XP)}: 1-2 Minor threats, avoid Major/Epic
\item \textbf{Seasoned (41-90 XP)}: Mix of Minor/Moderate, occasional Major
\item \textbf{Veterans (91-150 XP)}: Moderate/Major mix, Epic as climactic threats
\item \textbf{Paragon (151-220 XP)}: Major threats common, Epic as ongoing rivals
\item \textbf{Mythic (221+ XP)}: Epic threats as regular opponents
\end{itemize}

\subsection*{Narrative Integration}

Monsters should serve story purposes:
\begin{itemize}
\item \textbf{Obstacle}: Preventing progress toward goals
\item \textbf{Revelation}: Providing information through defeat/capture
\item \textbf{Transformation}: Changing PCs through encounter
\item \textbf{Relationship}: Creating ongoing connections (allies, rivals, patrons)
\item \textbf{Theme}: Reinforcing campaign themes and atmosphere
\end{itemize}

\subsection*{Consequence Management}

Every monster encounter should have lasting effects:
\begin{itemize}
\item \textbf{Victory}: What do PCs gain? Information, resources, reputation?
\item \textbf{Defeat}: What are the consequences? Pursuit, debt, injury?
\item \textbf{Bargain}: What compromises were made? Oaths, payment, favors?
\item \textbf{Escape}: What follows them? Pursuit, curse, reputation?
\end{itemize}

\section{QUICK REFERENCE CARDS}

\subsection*{Monster Creation Quick Sheet}

\textbf{Step 1: Define Core Concept}
\begin{itemize}
\item What makes this creature unique?
\item What role does it serve in encounters?
\item What themes does it represent?
\end{itemize}

\textbf{Step 2: Set Mechanical Parameters}
\begin{itemize}
\item Assign Threat Level (Minor/Moderate/Major/Epic)
\item Determine Harm Rating
\item Establish SB Generation Pattern
\end{itemize}

\textbf{Step 3: Create Signature Abilities}
\begin{itemize}
\item 2-3 key abilities that define the creature
\item Tie abilities to creature's nature
\item Balance power with interesting choices
\end{itemize}

\textbf{Step 4: Identify Weaknesses}
\begin{itemize}
\item Give players ways to overcome the threat
\item Tie weaknesses to creature's nature
\item Make weaknesses discoverable through play
\end{itemize}

\subsection*{Combat Encounter Checklist}

\begin{itemize}
\item \textbf{Setup}: Establish position, environment, and initial tension
\item \textbf{Motivation}: Why is this creature here? What does it want?
\item \textbf{Escalation Plan}: How will the encounter intensify?
\item \textbf{Player Agency}: What choices do players have?
\item \textbf{Consequences}: What happens for victory, defeat, or escape?
\item \textbf{SB Budget}: Plan 3-6 SB for a typical encounter
\item \textbf{Narrative Hook}: How does this connect to larger story?
\end{itemize}

\subsection*{SB Spending Guidelines}

\textbf{1 SB - Minor Complications:}
\begin{itemize}
\item Environmental disadvantage
\item Tactical positioning shift
\item Minor gear damage
\item Simple reinforcements
\end{itemize}

\textbf{2-3 SB - Moderate Complications:}
\begin{itemize}
\item New threat introduction
\item Significant environmental change
\item Ally in danger
\item Tactical disadvantage
\item Ongoing condition
\end{itemize}

\textbf{4+ SB - Major Complications:}
\begin{itemize}
\item Scene transformation
\item Major reinforcement arrival
\item Critical ally incapacitation
\item Fundamental tactical shift
\item Campaign-altering consequence
\end{itemize}

\section{MAGICAL BEASTS INDEX}

\subsection*{Aetherial Stalker}

\threat{Major} \\
\harm{=} \\
\cp{Phase shifting generates 2-3 SB unpredictably}

\vspace{0.5em}
\textbf{Description:} Creatures of pure magical energy that exist partially outside normal reality. They appear as shifting, translucent humanoids with stars for eyes and voices like distant thunder.

\textbf{Motivations:} Feeding on magical energy, protecting magical sites, hunting spellcasters

\textbf{Abilities:}
\begin{itemize}
\item \textbf{Phase Shift}: Can become intangible, avoiding physical attacks
\item \textbf{Magic Drain}: Successful attack reduces target's magical energy by 1 Boon
\item \textbf{Reality Distortion}: Can alter local physics (gravity, time, space)
\item \textbf{Invisible to Mundane}: Cannot be detected by non-magical senses
\end{itemize}

\textbf{Weaknesses:}
\begin{itemize}
\item Vulnerable during phase shift (1 round after becoming tangible)
\item Iron disrupts their phase-shifting ability
\item Cannot affect blessed/consecrated items
\item Must feed regularly or weaken
\end{itemize}

\subsection*{Chimera Construct}

\threat{Epic} \\
\harm{>} \\
\cp{Unstable creation generates 3-4 SB through malfunction}

\vspace{0.5em}
\textbf{Description:} Massive artificial creatures created by combining different magical beasts. They vary wildly in form but typically have multiple heads, mismatched limbs, and an aura of barely contained magical chaos.

\textbf{Motivations:} Following creator's commands, seeking destruction, protecting creation site

\textbf{Abilities:}
\begin{itemize}
\item \textbf{Multi-Attack}: Can attack multiple targets per round
\item \textbf{Elemental Breath}: Different heads breathe different elements
\item \textbf{Regeneration}: Heal 1 Harm per round unless damaged by specific weakness
\item \textbf{Magical Overflow}: Spells cast nearby may backfire or empower the construct
\end{itemize}

\textbf{Weaknesses:}
\begin{itemize}
\item Unstable magical core can be disrupted
\item Specific creation rituals can shut them down
\item Vulnerable to anti-magic fields
\item Parts may be individually targetable
\end{itemize}

\section{ECOSYSTEM INTERACTIONS}

\subsection*{Predator-Prey Relationships}

Understanding how creatures interact helps create realistic encounters:

\textbf{Blackwood Wolves} hunt \textbf{Dire Boars} but avoid \textbf{Mist Wraiths}

\textbf{Bandit Skirmishers} prey on \textbf{Merchants} but fear \textbf{Ykrul Raiders}

\textbf{Deep Drakes} compete with \textbf{Stone Giant Elders} for territory

\textbf{Redcaps} manipulate \textbf{Bandits} but cannot control \textbf{Curse Echoes}

\subsection*{Seasonal Variations}

Creature behavior changes with seasons:

\textbf{Spring}:
\begin{itemize}
\item Dire Boars more aggressive (mating season)
\item Bandit activity increases (roads clear)
\item Mist Wraiths less common (less fog)
\end{itemize}

\textbf{Summer}:
\begin{itemize}
\item Ykrul Raiders more active (good weather for raids)
\item Blackwood Wolves hunt at night (avoid heat)
\item Shadow Stalkers weaker (more daylight)
\end{itemize}

\textbf{Autumn}:
\begin{itemize}
\item All predators more dangerous (winter preparation)
\item Curse activity increases (approaching dark season)
\item Bandit raids become desperate (harvest time)
\end{itemize}

\textbf{Winter}:
\begin{itemize}
\item Creatures cluster around resources
\item Supernatural activity peaks
\item Survival becomes primary motivation
\end{itemize}

\section{MONSTER LORE AND BACKGROUND}

\subsection*{Creating Monster Histories}

Every monster should have a story:

\textbf{Origin}: How did this creature come to be?
\begin{itemize}
\item Natural evolution
\item Magical experimentation
\item Curse transformation
\item Ancient remnant
\item Imported from elsewhere
\end{itemize}

\textbf{Current State}: What drives this creature now?
\begin{itemize}
\item Survival instinct
\item Unfinished business
\item Obedience to master
\item Territorial defense
\item Hunger for specific thing
\end{itemize}

\textbf{Future Potential}: What could change this creature?
\begin{itemize}
\item Redemption/defeat of master
\item Environmental change
\item Player intervention
\item Natural cycle completion
\item Magical evolution
\end{itemize}

\subsection*{Sample Monster Backstory}

\textbf{The Weeping Knight} - Mist Wraith

Once Sir Aldric Brightshield, a champion of Ecktoria who died defending refugees during a Ykrul raid. His body was never recovered from the Mistlands, and his spirit, unable to accept death, became bound to his armor. Now he endlessly reenacts his final stand, protecting anyone who matches the refugees he failed to save.

\textbf{Current Motivation}: Protect the innocent, especially children and families

\textbf{Weakness}: His vow to protect refugees can be fulfilled by players who take up that cause

\textbf{Potential Growth}: Could become an ally if PCs honor his sacrifice appropriately

\section{ENCOUNTER DESIGN TIPS}

\subsection*{Building Tension}

\begin{itemize}
\item \textbf{Foreshadowing}: Drop hints before the encounter
\item \textbf{Escalation}: Start simple, add complications
\item \textbf{Stakes}: Make it clear what's at risk
\item \textbf{Choices}: Give players meaningful options
\item \textbf{Pacing}: Vary action and dialogue
\end{itemize}

\subsection*{Balancing Challenge}

\begin{itemize}
\item \textbf{Threat Assessment}: Match monster power to PC capabilities
\item \textbf{Environmental Factors}: Consider terrain and circumstances
\item \textbf{Resource Management}: Account for PC resources (Boons, Fatigue, Gear)
\item \textbf{Fallback Options}: Plan for different player approaches
\item \textbf{Consequence Variety}: Not every encounter needs combat
\end{itemize}

\subsection*{Narrative Integration}

\begin{itemize}
\item \textbf{Campaign Connections}: Link to ongoing storylines
\item \textbf{Character Development}: Provide growth opportunities
\item \textbf{World Building}: Reveal setting details
\item \textbf{Thematic Reinforcement}: Support campaign themes
\item \textbf{Future Hooks}: Create seeds for later encounters
\end{itemize}

\section{APPENDIX: MONSTER CREATION WORKSHEETS}

\subsection*{Basic Monster Profile}

\begin{tabularx}{\textwidth}{|l|X|}
\hline
\textbf{Name:} & \\
\hline
\textbf{Threat Level:} & Minor / Moderate / Major / Epic \\
\hline
\textbf{Harm Rating:} & - / = / > / >> / † \\
\hline
\textbf{SB Generation:} & \\
\hline
\textbf{Description:} & \\
\hline
\textbf{Motivations:} & \\
\hline
\textbf{Abilities:} & \\
\hline
\textbf{Weaknesses:} & \\
\hline
\end{tabularx}

\subsection*{Encounter Planning Sheet}

\begin{tabularx}{\textwidth}{|l|X|}
\hline
\textbf{Setup:} & \\
\hline
\textbf{Initial Position:} & Dominant/Controlled/Desperate \\
\hline
\textbf{Environmental Factors:} & \\
\hline
\textbf{Player Options:} & \\
\hline
\textbf{SB Budget:} & \\
\hline
\textbf{Escalation Plan:} & \\
\hline
\textbf{Resolution Paths:} & \\
\hline
\textbf{Consequences:} & \\
\hline
\end{tabularx}

\end{document}

