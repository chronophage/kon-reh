\documentclass[11pt]{article}
\usepackage[margin=1in]{geometry}
\usepackage{titlesec}
\usepackage{enumitem}
\usepackage{fancyhdr}
\usepackage{hyperref}
\usepackage{graphicx}
\usepackage{tabularx}
\usepackage{booktabs}
\usepackage{multicol}
\usepackage{tikz}
\usepackage{framed}
\usepackage{sectsty}

\sectionfont{\large\bfseries\MakeUppercase}
\subsectionfont{\bfseries}

\pagestyle{fancy}
\fancyhf{}
\fancyhead[L]{Solo Fate's Edge}
\fancyhead[R]{\thepage}
\fancyfoot[C]{}

\hypersetup{
    colorlinks=true,
    linkcolor=black,
    filecolor=magenta,      
    urlcolor=cyan,
}

\newcommand{\dice}[1]{\textbf{[1d6: #1]}}

\begin{document}

\begin{titlepage}
\centering
\vspace*{2cm}
{\Huge\bfseries Solo Fate's Edge: A Comprehensive Guide\par}
\vspace{1cm}
{\Large Playing Fate's Edge Solo with Deck-Based Generation\par}
\vspace{2cm}
{\large A Supplement for Solo Adventurers\par}
\vspace{3cm}
{\large \today\par}
\end{titlepage}

\tableofcontents
\newpage

\section{Introduction}

Welcome to the world of \textbf{Fate's Edge}, where every choice carries weight, every spell risks backlash, and every legend is written in the shadow of consequence. This guide provides everything you need to play Fate's Edge as a compelling solo experience using the game's built-in deck-based generation systems and simple dice mechanics.

\subsection{What You Need}

To play Fate's Edge solo, you will need:
\begin{itemize}[leftmargin=*]
    \item Standard deck of playing cards (52 cards)
    \item Six-sided dice (d6) - multiple dice recommended
    \item Character sheet and pencils
    \item Regional generator decks (Thepyrgos, Mistlands, etc.)
    \item Deck of Consequences
    \item Optional: Regional reference materials
\end{itemize}

\section{Core Solo Mechanics}

\subsection{Position Determination}

Every significant action requires determining your position. Roll \dice{1-2: Desperate, 3-4: Risky, 5-6: Controlled}.

\subsection{Complication Management}

When you roll 1s and generate Story Beats (SB):
\dice{1-2: Must draw from Deck of Consequences, 3-4: Optional draw, 5-6: Avoid drawing}

\subsection{Follower Risk}

When spending 2+ SB on assisted actions:
\dice{1: Follower endangered, 2-3: Follower at risk, 4-6: No follower risk}

\section{Character Creation}

\subsection{Solo-Specific Considerations}

As a solo player, consider these character development principles:

\begin{framed}
\textbf{Balanced Investment Path:} Since you're playing alone, diversify your investments across all three paths:
\begin{itemize}
    \item \textbf{Enhance Self (50-65\%):} Core combat/social skills
    \item \textbf{Acquire Assets (15-25\%):} 1-2 key assets for off-screen leverage
    \item \textbf{Learn Talents (15-25\%):} Cultural abilities that enhance solo play
\end{itemize}
\end{framed}

\subsection{Recommended Starting Builds}

\subsubsection{The Versatile Explorer}
\begin{itemize}
    \item Wits 3, Body 3, Spirit 2
    \item Survival 3, Stealth 2, Lore 2
    \item Assets: Minor Safehouse, Herbal Garden
    \item Talents: Versatile, Route Whisper
\end{itemize}

\subsubsection{The Oath-Bound Blade}
\begin{itemize}
    \item Body 4, Spirit 3
    \item Melee 4, Endurance 2
    \item Assets: Signature Weapon, Minor Shrine
    \item Talents: Battle Instincts, Iron Stomach
\end{itemize}

\subsubsection{The Shadow Operative}
\begin{itemize}
    \item Wits 4, Presence 2
    \item Skullduggery 3, Stealth 3
    \item Assets: Safehouse Network, Courier Network
    \item Talents: Silver Tongue, Hand Signals
\end{itemize}

\section{Deck-Based Solo Play}

\subsection{Regional Exploration}

Use regional generator decks to seed your adventures. For each travel leg:

\begin{enumerate}
    \item Draw Spade (Place) and Heart (Actor) from destination deck
    \item Draw Club from Wilds deck or destination deck
    \item Draw Diamond from controlling authority deck
    \item Set clock size by highest rank
\end{enumerate}

\subsection{The Deck of Consequences}

When complications arise, draw from the Deck of Consequences:
\begin{itemize}
    \item Hearts: Social/emotional fallout
    \item Spades: Physical harm/escalation
    \item Clubs: Material/resource cost
    \item Diamonds: Magical/spiritual disturbance
\end{itemize}

\section{Solo Adjudication Systems}

\subsection{Position Table}

\begin{center}
\begin{tabular}{cl}
\toprule
\textbf{Roll} & \textbf{Position} \\
\midrule
1-2 & Desperate (disadvantaged, severe consequences) \\
3-4 & Risky (even odds, moderate consequences) \\
5-6 & Controlled (advantageous, minor consequences) \\
\bottomrule
\end{tabular}
\end{center}

\subsection{Complication Engagement}

\dice{1-2: Must draw complication, 3-4: Optional draw, 5-6: Avoid drawing}

\subsection{Follower Risk Management}

\dice{1: Follower endangered, 2-3: Follower at risk, 4-6: No risk}

\subsection{Campaign Clock Management}

At end of major scenes:
\dice{1: Crisis +1, 2: Mandate +1, 3: Both +1, 4: No change, 5: Mandate +1, 6: Crisis +1}

\section{Resource Management}

\subsection{Supply Clock}

The Supply Clock tracks your access to food, water, and gear:
\begin{itemize}
    \item 0 filled: Full Supply (no penalties)
    \item 2 filled: Low Supply (minor complications)
    \item 3 filled: Dangerously Low (Fatigue +1 to all)
    \item 4 filled: Out of Supply (severe penalties)
\end{itemize}

Supply changes:
\dice{1-2: +1 segment, 3-4: No change, 5-6: -1 segment (if possible)}

\subsection{Fatigue Management}

Fatigue represents exhaustion and strain:
\begin{itemize}
    \item 1 Fatigue: Re-roll one success
    \item 2 Fatigue: Re-roll one success (cumulative)
    \item 3 Fatigue: Re-roll two successes
    \item 4 Fatigue: Collapse/KO
\end{itemize}

Fatigue accumulation:
\dice{1: +1 Fatigue, 2-5: No change, 6: -1 Fatigue}

\subsection{Asset and Follower Conditions}

Assets and followers have three conditions:
\begin{itemize}
    \item \textbf{Maintained:} Full capability
    \item \textbf{Neglected:} -1 die penalty
    \item \textbf{Compromised:} Unavailable
\end{itemize}

Neglect risk:
\dice{1-2: Becomes Neglected, 3-4: Maintained, 5-6: Improves condition}

\section{Combat Systems}

\subsection{Solo Combat Adjudication}

Combat positions are determined the same as other actions:
\dice{1-2: Desperate, 3-4: Risky, 5-6: Controlled}

Position dynamics:
\dice{1-2: Position worsens, 3-4: No change, 5-6: Position improves}

\subsection{Tactical Clocks}

Use clocks to track persistent combat conditions:
\begin{itemize}
    \item Mob Overwhelm (6): Enemy numbers become advantage
    \item Fatigue Spiral (4): Exhaustion affects performance
    \item Morale Collapse (6): Fear undermines effectiveness
    \item Environmental Collapse (8): Terrain/fire/building failure
\end{itemize}

Clock advancement:
\dice{1-2: +1 segment, 3-4: No change, 5-6: -1 segment}

\section{Magic and Backlash}

\subsection{Solo Spellcasting}

The Casting Loop works the same in solo play:
\begin{enumerate}
    \item Channel: Roll Wits + Arcana
    \item Weave: Roll Wits + (Art) on next turn
    \item Backlash: Resolve SB through dice systems
\end{enumerate}

\subsection{Backlash Severity}

When Backlash SB are generated:
\dice{1-2: Minor nuisance, 3-4: Noticeable setback, 5-6: Major turn}

\section{Travel Framework}

\subsection{Solo Travel Procedure}

For each travel leg:
\begin{enumerate}
    \item Draw cards to seed the journey
    \item Set travel clock by highest rank
    \item Travel complications:
    \dice{1-2: Draw Club complication, 3-4: Draw Wilds Club, 5: Smooth travel, 6: Advantageous travel}
\end{enumerate}

\subsection{Travel Hazards}

Supply depletion during travel:
\dice{1-2: +1 Supply segment, 3-4: No change, 5-6: -1 Supply segment}

\section{Campaign Management}

\subsection{The Crown Spread}

At campaign start, draw the Crown Spread:
\begin{itemize}
    \item Spade: Crown Site (where the monument is decided)
    \item Heart: Crown Rival (who can still stop it)
    \item Club: Crown Pressure (the rail that will bite)
    \item Diamond: Crown Leverage (the payoff)
    \item Wild: Hidden force (Face = patron, Ace = site becomes 10-clock)
\end{itemize}

\subsection{Mandate and Crisis Clocks}

Track your influence and opposition:
\begin{itemize}
    \item Mandate (0-6): Public legitimacy and buy-in
    \item Crisis (0-6): Opposition engine and pressure
\end{itemize}

Finale conditions:
\begin{itemize}
    \item Player-Called: Mandate 6 and Crisis 3
    \item Forced: Crisis 6 (regardless of Mandate)
\end{itemize}

\section{Advanced Solo Techniques}

\subsection{Multi-Character Campaigns}

Manage multiple characters by:
\begin{itemize}
    \item Rotating focus between characters
    \item Using followers as secondary protagonists
    \item Creating character relationships and conflicts
\end{itemize}

\subsection{Faction Play}

Run campaigns from different faction perspectives:
\begin{itemize}
    \item Track faction relationships with loyalty scales
    \item Use dice to determine faction reactions
    \item Create faction-specific goals and challenges
\end{itemize}

\subsection{Legacy Games}

Use epilogue mechanics for long-term character evolution:
\begin{itemize}
    \item Convert major assets to institutions
    \item Promote followers to stationed NPCs
    \item Create lasting world changes
\end{itemize}

\section{Sample Solo Session}

\subsection{Setup}

Character: The Versatile Explorer

Region: Mistlands

Goal: Investigate strange bell-line failures

\subsection{Scene Framing}

1. Position Roll: \dice{4 = Risky} approach to Pall Watch-tower

2. Investigation Action: Wits + Investigation

3. Result: 2 successes, 1 SB generated

4. Complication Check: \dice{3 = Optional draw} $\rightarrow$ Choose to investigate further

5. Deck Draw: Club - "Bell-line failure on the levee; a wraith steps across like it owns the road"

6. New Position: Risky encounter with wraith

\subsection{Resolution}

1. Combat Position: \dice{3 = Risky}

2. Melee Action: Body + Melee

3. Result: 3 successes, 0 SB

4. Clean Success: Wraith defeated, bell-line mystery deepened

5. Campaign Clock: \dice{1 = Crisis +1} - Opposition notices your investigation

\section{Troubleshooting Common Issues}

\subsection{Over-Powerment}

\begin{framed}
\textbf{Solutions:}
\begin{itemize}
    \item Use higher DV for solo challenges
    \item Embrace complications more actively
    \item Maintain strict resource management
    \item Allow followers to be compromised more readily
\end{itemize}
\end{framed}

\subsection{Narrative Stagnation}

\begin{framed}
\textbf{Solutions:}
\begin{itemize}
    \item Use multiple regional decks for variety
    \item Create personal character arcs
    \item Introduce recurring NPCs with dice-driven reactions
    \item Embrace unexpected deck results rather than rerolling
\end{itemize}
\end{framed}

\subsection{Mechanical Gaming}

\begin{framed}
\textbf{Solutions:}
\begin{itemize}
    \item Reward descriptive play with Boons
    \item Use dice to force engagement with complications
    \item Create personal stakes that go beyond mechanical optimization
    \item Maintain campaign clocks to ensure long-term consequences
\end{itemize}
\end{framed}

\section{Solo-Specific House Rules}

\subsection{Enhanced Boon Economy}

Solo players may:
\begin{itemize}
    \item Convert 1 Boon $\rightarrow$ 1 XP (instead of 2:1 ratio)
    \item Gain bonus Boon for particularly challenging solo scenes
    \item Earn Boons for creative problem-solving without dice rolls
\end{itemize}

\subsection{Risk Engagement Bonuses}

Players earn bonus XP for:
\begin{itemize}
    \item Choosing Risky over Controlled positions
    \item Accepting meaningful complications
    \item Engaging with generated pressures rather than avoiding them
\end{itemize}

\subsection{Narrative Investment Rewards}

Bonus resources for:
\begin{itemize}
    \item Detailed world-building descriptions
    \item Consistent character voice/behavior
    \item Meaningful interaction with generated elements
    \item Creative integration of deck results
\end{itemize}

\section{Conclusion}

Solo Fate's Edge offers a unique opportunity to explore the setting's rich mechanical and narrative systems at your own pace. The deck-based generation ensures consistent thematic content while the dice-based adjudication systems maintain mechanical integrity without requiring external oversight.

The key to successful solo play lies in:
\begin{enumerate}
    \item \textbf{Honest mechanical resolution} - Let the dice decide when appropriate
    \item \textbf{Embracing generated complications} - They drive the story forward
    \item \textbf{Maintaining resource management} - Supply, Fatigue, and asset conditions matter
    \item \textbf{Investing in the narrative} - Your character's story is what makes this compelling
\end{enumerate}

Whether you're exploring the mist-shrouded bell-lines of the Mistlands, navigating the political intrigue of Thepyrgos, or delving into the ancient mysteries of Valewood, Fate's Edge provides a rich, engaging solo experience that rewards both mechanical skill and narrative creativity.

The world is watching. What are you willing to risk to reshape the world around you?

\end{document}
