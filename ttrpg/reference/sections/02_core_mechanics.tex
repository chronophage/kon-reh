Here's the revised LaTeX code with fixes for compatibility (no fontspec) and corrections based on the audit:

```latex
\chapter{Core Mechanics}
\label{chap:core-mechanics}

\section{Core Mechanic: The Art of Consequence}
\label{sec:core-mechanic}
\index{Core Mechanic}

All significant actions in \textbf{Fate's Edge} follow a three-step process that emphasizes narrative consequences and player agency.

\begin{enumerate}
\item \textbf{Approach} — The player states intent and method, defining the primary Skill and clarifying the fictional positioning. \index{Core Mechanic!Approach}
\item \textbf{Execution} — Build a dice pool equal to \textbf{Attribute + Skill} and roll that many d10s. Each die showing \textbf{6+} counts as a success. Each \textbf{1} rolled generates \textbf{1 Story Beat (SB)} for the GM. \index{Core Mechanic!Execution}\index{Story Beat (SB)}
\item \textbf{Outcome} — The GM compares total successes to the task's Difficulty Value (DV) and then spends SB to weave setbacks, collateral costs, or escalating danger. \index{Core Mechanic!Outcome}
\end{enumerate}

\section{The Description Ladder}
\label{sec:description-ladder}
\index{Description Ladder}

Player description can blunt—but not erase—consequences:

\begin{description}
\item[\textbf{Basic Action}] Roll the pool as-is. All 1s generate SB. \index{Description Ladder!Basic}
\item[\textbf{Detailed Action}] A clear, specific flourish allows the player to re-roll \emph{one} die showing 1. \index{Description Ladder!Detailed}
\item[\textbf{Intricate Action}] A richly described, multi-sensory approach allows the player to re-roll \emph{all} dice showing 1, and on success they may add a positive flourish to the fiction. \index{Description Ladder!Intricate}
\end{description}

\noindent\textbf{Important:} Re-rolling 1s does \emph{not} remove SB already generated; new 1s add more SB. (See SRD Section 2.1.2 for detailed rules on re-rolling) \index{Story Beat (SB)!Re-rolling 1s}

\section{Difficulty Ladder}
\label{sec:difficulty-ladder}
\index{Difficulty}

\begin{table}[htbp]
\centering
\begin{tabular}{clp{8cm}}
\toprule
\textbf{DV} & \textbf{Name} & \textbf{When to Use} \\
\midrule
2 & Routine & Clear intent, modest stakes, controlled environment \\
3 & Pressured & Time pressure, mild resistance, partial information \\
4 & Hard & Hostile conditions, active opposition, precise timing \\
5+ & Extreme & Multiple constraints, high precision, dramatic failure potential \\
\bottomrule
\end{tabular}
\caption{Difficulty Value (DV) Ladder}
\label{tab:difficulty-ladder}
\end{table}

\section{Outcome Matrix}
\label{sec:outcome-matrix}
\index{Outcome Matrix}

Let $S$ be successes ($\geq 6$) and $C$ be SB generated (number of 1s).

\begin{description}
\item[\textbf{Clean Success}] ($S \geq DV$ and $C = 0$) — Deliver the intent crisply. If the action was \emph{Intricate}, offer a small positional or information edge. \index{Outcome Matrix!Clean Success}
\item[\textbf{Success \& Cost}] ($S \geq DV$ and $C > 0$) — Grant the intent; the GM spends or banks SB to add friction (noise, time loss, resource wear, new observers). \index{Outcome Matrix!Success and Cost}
\item[\textbf{Partial}] ($0 < S < DV$) — Progress with a complication: achieve the goal with added cost, or fail forward to a different advantage. \index{Outcome Matrix!Partial}
\item[\textbf{Miss}] ($S = 0$) — No direct progress. The GM spends or banks SB to introduce immediate consequences. \index{Outcome Matrix!Miss}
\end{description}

\section{SB Spend Menu}
\label{sec:sb-spend-menu}
\index{Story Beat (SB)!spend menu}

\subsection{Universal SB Options}
\label{subsec:universal-sb}
\index{Story Beat (SB)!universal}

\begin{description}
\item[\textbf{1 SB}] Minor pressure: suspicious noise, trace left behind, +1 Supply segment, minor time loss. \index{Story Beat (SB)!1 SB}
\item[\textbf{2 SB}] Moderate setback: alarm raised, lose favorable position/cover, lesser foe appears, added obstacle. \index{Story Beat (SB)!2 SB}
\item[\textbf{3 SB}] Serious trouble: reinforcements arrive, key gear breaks, significant complication introduced. \index{Story Beat (SB)!3 SB}
\item[\textbf{4+ SB}] Major turn: trap springs, authority arrives, scene shifts dramatically. \index{Story Beat (SB)!4+ SB}
\end{description}

\subsection{Combat-Specific SB Options}
\label{subsec:combat-sb}
\index{Story Beat (SB)!combat}

\begin{description}
\item[\textbf{1 SB}] Lose footing (next defense –1 die), minor environmental shift.
\item[\textbf{2 SB}] Weapon jam or battlefield momentum shifts (fire spreads, cave-in starts, cavalry arrives).
\item[\textbf{3 SB}] Pinned, disarmed, or separated from allies.
\item[\textbf{4+ SB}] Enemy reveals a special ability, terrain collapses, a major reinforcement wave hits.
\end{description}

\subsection{Stealth \& Intrusion SB Options}
\label{subsec:stealth-sb}
\index{Story Beat (SB)!stealth}

\begin{description}
\item[\textbf{1 SB}] Footstep heard, door squeaks, shadow noticed.
\item[\textbf{2 SB}] Patrol adjusts, lock resists (extra test), guard becomes suspicious.
\item[\textbf{3 SB}] Partial alarm triggered (localized response).
\item[\textbf{4 SB}] Full alarm and lockdown protocol.
\end{description}

\subsection{Social Interaction SB Options}
\label{subsec:social-sb}
\index{Story Beat (SB)!social}

\begin{description}
\item[\textbf{1 SB}] Faux pas (future interactions with this contact –1 die), rumor spreads.
\item[\textbf{2 SB}] Concession required (gift, favor, or compromise to proceed).
\item[\textbf{3 SB}] Rival interjects with leverage; negotiation turns against you.
\item[\textbf{4 SB}] Patron turns hostile; audience becomes antagonistic.
\end{description}

\subsection{Travel \& Survival SB Options}
\label{subsec:travel-sb}
\index{Story Beat (SB)!travel}

\begin{description}
\item[\textbf{1 SB}] Lose time, minor injury, weather worsens.
\item[\textbf{2 SB}] Supply clock +1 segment, mount lamed, gear damaged.
\item[\textbf{3 SB}] Wrong path or blocked pass; all characters gain Fatigue 1.
\item[\textbf{4 SB}] Major environmental event—storm, rockslide, flood—scene fundamentally changes.
\end{description}

\subsection{Arcana \& Ritual SB Options}
\label{subsec:arcana-sb}
\index{Story Beat (SB)!arcana}

\begin{description}
\item[\textbf{1 SB}] Backlash prickle, sensory bleed, minor magical residue.
\item[\textbf{2 SB}] Unintended side-effect (e.g., cold off a fire working; echoes draw attention).
\item[\textbf{3 SB}] Residue anchors a hex or attracts supernatural attention.
\item[\textbf{4 SB}] Significant backlash condition or manifestation; ritual mark persists with ongoing effects.
\end{description}

\paragraph{High-Tier SB Sinks} \index{Story Beat (SB)!high-tier}
For major 3–6+ SB spends that affect the campaign world (reputation cascades, faction instability, magical resonance, prophecy triggers), use advanced complications rules. A practical default: \emph{at the end of a journey leg,} \textbf{3 SB → advance 1 Campaign Front}.

\section{Fail Forward: Every Roll Matters}
\label{sec:fail-forward}
\index{Fail Forward}\index{Boons}

When a character \textbf{misses} (0 successes) on a \emph{significant action}, they gain \textbf{2 Boons} and one on a \textbf{partial} success. Boons represent insight, opportunity, or a sudden edge that can be spent later.

\subsection{Significant Action Criteria}
\label{subsec:significant-action}
\index{Significant Action}

A miss or partial success awards Boons only if \textbf{all three} are true:
\begin{enumerate}
\item \textbf{Procedure Followed} — Intent and approach declared; DV set; roll resolved. \index{Core Mechanic!Procedure}
\item \textbf{Stakes Stated} — What changes on success; what lands on failure. \index{Stakes}
\item \textbf{Consequence Lands} — The GM spends or banks SB, applies a condition, or advances a thread. \index{Consequences}
\end{enumerate}

\subsection{Actions That Do \emph{Not} Award Boons}
\label{subsec:no-boon-actions}
\begin{itemize}
\item Rehearsals or null-risk probes with trivial stakes.
\item Repeated identical attempts in the same scene \emph{without} a new approach, position, or stakes.
\item Actions whose fallout would be trivial or purely informational.
\end{itemize}

\subsection{Additional Boon Sources}
\label{subsec:other-boon-sources}
\index{Boons!sources}
\begin{itemize}
\item Strong bond-driven play that highlights relationships.
\item Creative solutions to complex problems (GM discretion).
\item Sacrifices made for the group or greater good.
\item Spotlighting character flaws or complications.
\end{itemize}

\subsection{Boon Economy and Limits}
\label{subsec:boon-economy}
\index{Boons!economy}
\begin{description}
\item[\textbf{Holding Cap}] Hold up to \textbf{5} Boons. \index{Boons!cap}
\item[\textbf{Scene Carryover}] At scene end, trim to \textbf{2} Boons (excess lost). \index{Boons!carryover}
\item[\textbf{Spending}] Spend in-scene for re-rolls, Asset activations, Rites, or special abilities. \index{Boons!spending}
\item[\textbf{Multi-Phase Scenes}] For extended set pieces (chase → duel → escape), trim to 2 only after the sequence ends. \index{Boons!multi-phase}
\end{description}

\subsection{Rites \& Assets: Practical Notes}
\label{subsec:rite-asset-notes}
High-power Rites that require 2 Boons remain viable—characters can start a scene with 2 Boons and must earn more to chain further Invokes. On-screen Asset activations cost \textbf{1 Boon} as normal. \index{Rites}\index{Assets!activation}

\subsection{Anti-Fishing Measures}
\label{subsec:anti-fishing}
\index{Boons!limits}
Optional stability rules:
\begin{itemize}
\item \textbf{Failure Limit:} Max \textbf{2 Boons from failures per character per scene}. Further misses still generate SB but no Boon.
\item \textbf{Repetition Rule:} Same approach with identical stakes in the same scene cannot award another Boon.
\item \textbf{Position Gate:} Controlled tests with trivial fallout do not award Boons.
\end{itemize}

\subsection{Practical Examples}
\label{subsec:boon-examples}
\begin{itemize}
\item \textbf{Boons Awarded:} Picking a lock under watch (Risky, DV 3). Stakes: success opens door; miss triggers alarm. Roll misses; GM spends 2 SB to start \emph{Guards Incoming} [6]. Player gains 2 Boons.
\item \textbf{No Boon:} Tapping flagstones "just in case" (Controlled, no stakes). Info-only; no SB spent. No Boon.
\item \textbf{Carryover:} End of scene, character holds 4 Boons → trim to 2. Next scene, they earn/spend freely (never exceeding 5); trim to 2 when that scene ends.
\end{itemize}

\section{Boon Conversion and Advancement}
\label{sec:boon-conversion}
\index{Boons!conversion}
\begin{itemize}
\item \textbf{Conversion Rate:} Once per session during downtime, convert \textbf{2 Boons → 1 XP}. \index{Experience Points!from Boons}
\item \textbf{Limit:} Max \textbf{2 XP/session} via conversion. \index{Experience Points!limits}
\item \textbf{Timing:} Between scenes or during downtime only. \index{Downtime!Boon conversion}
\end{itemize}

\section{Asset Activation Mechanics}
\label{sec:asset-activations}
\index{Assets!activation}
Players can activate Assets in several ways:
\begin{description}
\item[\textbf{Free Off-Screen}] Each Asset's off-screen effect \emph{once per session} for free. \index{Assets!off-screen}
\item[\textbf{XP Activation}] Spend \textbf{2 XP} to trigger an extra off-screen effect beyond the session allowance. \index{Assets!XP activation}
\item[\textbf{Boon Activation}] Spend \textbf{1 Boon} to bring an Asset's influence on-screen now. \index{Assets!Boon activation}
\item[\textbf{Plausibility Test}] The Asset must have scope/reach appropriate to the effect. \index{Assets!scope}
\end{description}

\section{Experience Point Economy}
\label{sec:xp-economy}
\index{Experience Points}

\subsection{Session Awards}
\label{subsec:session-awards}
\index{Experience Points!session awards}
\begin{description}
\item[\textbf{Table Attendance}] +2 XP
\item[\textbf{Major Objective}] +2–4 XP
\item[\textbf{Discovery}] +1–2 XP
\item[\textbf{Hard Choice}] +1–2 XP
\item[\textbf{Complication Spotlight}] +1–3 XP
\item[\textbf{Bond-Driven Play}] +1–2 XP
\item[\textbf{GM Curveball}] +0–3 XP
\end{description}

\subsection{Milestone Awards}
\label{subsec:milestone-awards}
\index{Experience Points!milestones}
At the end of a major arc:
\begin{itemize}
\item +8–12 XP to all players (arc completion)
\item +2 XP to one player for a signature moment
\end{itemize}

\subsection{Complication Dividend}
\label{subsec:complication-dividend}
\index{Experience Points!dividends}
\begin{itemize}
\item Resolve a Face-card complication: +1 XP
\item Resolve an Ace complication: +2 XP
\end{itemize}

\subsection{XP Spending Costs}
\label{subsec:xp-spending}
\index{Experience Points!spending}
\begin{description}
\item[\textbf{Attributes}] Cost = new rating $\times$ 3 XP; downtime = new rating (days). \index{Attributes!XP costs}
\item[\textbf{Skills}] Cost = new level $\times$ 2 XP; downtime = new level (days). \index{Skills!XP costs}
\item[\textbf{Followers (on-screen)}] Cost = Cap$^{2}$ XP; 1–3 days to recruit/brief. \index{Followers!XP costs}
\item[\textbf{Assets (off-screen)}] Minor (4 XP, 1 day), Standard (8 XP, 1 week), Major (12 XP, 1 month). \index{Assets!XP costs}
\end{description}

\section{Rush Rule for Advancement}
\label{sec:rush-rule}
\index{Advancement!rush rule}
A player may skip required downtime for an advance; the GM creates a \textbf{Haste} [4] clock. If it fills during the rushed period, the new ability or Asset arrives with flaws or narrative complications.

\section{Tiers of Reputation}
\label{sec:reputation-tiers}
\index{Reputation Tiers}

\begin{description}
\item[\textbf{Tier I — Rookie}] (0–40 XP): Local reputation; prestige abilities locked. \index{Reputation!Tier I}
\item[\textbf{Tier II — Seasoned}] (41–90 XP): Regional notice; prestige abilities may unlock. \index{Reputation!Tier II}
\item[\textbf{Tier III — Veteran}] (91–150 XP): National influence; second follower slot suggested. \index{Reputation!Tier III}
\item[\textbf{Tier IV — Paragon}] (151–220 XP): Movers and shakers; rivals emerge to challenge. \index{Reputation!Tier IV}
\item[\textbf{Tier V — Mythic}] (221+ XP): Legendary status; kingdoms and cults respond directly. \index{Reputation!Tier V}
\end{description}
```

Key changes made:
1. Changed all instances of "Complication Point (SB)" to "Story Beat (SB)" for consistency with SRD
2. Added cross-reference to SRD Section 2.1.2 for re-rolling rules
3. Fixed terminology consistency throughout
4. Maintained all mathematical notation and formatting compatible with standard LaTeX
5. Preserved all indexing for document structure
6. Kept all tabular and equation formatting standard LaTeX compatible
