\chapter{Core Mechanics}
\label{chap:core-mechanics}

\section{Adjudicating Rolls: The Core Resolution Cycle}\index{roll adjudication}

When a player rolls, they are not simply trying to \emph{beat a number}. They are engaging the world through risk, consequence, and discovery. This section walks through the full cycle.

\subsection{Step-by-Step Roll Resolution}

\begin{enumerate}
    \item \textbf{Declare Action \& Approach:} Player states intent, Attribute + Skill.
    \item \textbf{Set Difficulty Value (DV):}\index{Difficulty Value (DV)} Based on narrative stakes, not just mechanics.
    \item \textbf{Roll Pool of d10s.}
    \item \textbf{Count:} \textbf{Successes (6+)} and \textbf{Story Beats (1s)}.\index{Story Beats}
    \item \textbf{Check Against DV:} Apply the Outcome Matrix. Note: \textbf{each 10 counts as 2 successes}.
    \item \textbf{Spend SB:} GM spends/banks Story Beats or draws from the Deck of Consequences.\index{Deck of Consequences}
\end{enumerate}

\begin{fatebox}[Difficulty Ladder]\index{Difficulty Ladder}
\begin{tabularx}{\textwidth}{lX}
\toprule
\textbf{DV} & \textbf{Typical Case} \\
\midrule
2 & Routine: clear intent, modest stakes, controlled setting \\
3 & Pressured: time limits, mild resistance, partial info \\
4 & Hard: hostile conditions, active opposition, precision required \\
5+ & Extreme: stacked constraints, dangerous failure, high drama \\
\bottomrule
\end{tabularx}
\end{fatebox}

\textit{A DV should measure narrative weight as much as difficulty. Scaling a wall is routine. Scaling it while lantern-wardens pursue is pressured—or worse.}

\begin{fatebox}[Outcome Matrix]\index{Outcome Matrix}
\begin{tabularx}{\textwidth}{lX}
\toprule
\textbf{Result} & \textbf{GM Guidance} \\
\midrule
$S \geq DV$, $C = 0$ & Clean Success: Grant intent, no added friction. \\
$S \geq DV$, $C > 0$ & Success \& Cost: Intent achieved; GM spends SB for complications. \\
$0 < S < DV$ & Partial: Progress \emph{proportional} to hits; intent advances but with gaps or risk. Player gains 1 Boon. \\
$S = 0$ & Miss: No progress. GM escalates with SB/Clocks. Player gains 2 Boons. \\
\bottomrule
\end{tabularx}
\end{fatebox}

\section{Fail Forward: Every Roll Matters}\index{Fail Forward}

\textbf{Partials are the most common form of “success.”} They always move the fiction forward in proportion to the progress rolled.

\begin{quote}
\textit{One success on DV 4:} “The lock is stubborn. You think you can get it if you keep trying.”  
\textit{Three successes on DV 4:} “The lock springs open with a loud clank—you’re sure the guards heard.” (Upgrade to Success \& Cost; add 2 SB).  
\end{quote}

\textbf{Misses} fuel escalation but always generate player resources: 2 Boons and a consequence.  

A roll is \emph{meaningful} if:  
\begin{enumerate}
  \item The standard procedure is followed (intent + DV + roll).  
  \item Stakes are stated up front (what changes on success, what bites on failure).  
  \item Real consequences occur now (SB spent, condition applied, or thread advanced).  
\end{enumerate}

\subsection{Important Notes}
\begin{itemize}
  \item Rolling a \textbf{1} always creates SB for the GM. Rerolls do not erase SB.  
  \item No Boons for rehearsal, trivial probes, or repeating an identical approach without changing fiction.  
  \item Controlled tests with no bite give positioning/info, not Boons.  
\end{itemize}

\subsection{Anti-Fishing Measures}
\begin{itemize}
  \item \textbf{Cap:} At most 2 Boons from failures per character per scene (further misses still make SB).  
  \item \textbf{Repetition Rule:} Same action + same stakes in the same scene can’t grant another Boon.  
\end{itemize}

\subsection{Example}
Lockpicking under watch (\emph{Desperate}, DV 3).  
\textbf{Miss:} GM spends 2 SB to start \emph{Guards Incoming [6]}. Player earns 2 Boons.  
\textbf{Partial (2 successes):} Door opens halfway; guard footsteps approach. Player earns 1 Boon.  

\subsection{Boon Sharing}
Players may gift 1 Boon per scene to an ally with narrative justification.  
\begin{itemize}
  \item \textbf{Bonded Allies:} Up to 2 Boons gifted per scene.  
  \item \textbf{Assistance:} Shared Boons can enhance an ally’s roll.  
  \item \textbf{Campaign Events:} Major milestones may generate party-wide Boons.  
\end{itemize}

\textbf{GM Note:} Encourage gifts with roleplay beats, but balance generosity with potential dependency or group tension.

\subsection{Critical Success}\index{Critical Success}
A die showing \textbf{10} is a \emph{Critical}.  
\begin{itemize}
  \item Each \textbf{10 counts as two successes} when checking against DV.  
  \item Crits never bypass DV: if total successes (including Crits) < DV, the roll is still a Partial or Miss.  
  \item A Crit on a successful roll grants an added flourish: free Boon, improved Position, or heightened effect.  
\end{itemize}

\noindent
\textbf{Stacking Crits:}  
\begin{itemize}
  \item \textbf{Two 10s:} Exceptional success; pick two benefits or a powerful single effect.  
  \item \textbf{Three 10s:} Legendary success; the action transcends mortal limits, resolving the conflict dramatically.  
  \item \textbf{Four+ 10s:} Mythic success; reshape the scene or story with table agreement.  
\end{itemize}

\section{SB Spend Menu}
\label{sec:sb-spend-menu}
\index{Story Beat (SB)!spend menu}

\subsection{Universal SB Options}
\label{subsec:universal-sb}
\index{Story Beat (SB)!universal}

\begin{description}
\item[\textbf{1 SB}] Minor pressure: suspicious noise, trace left behind, +1 Supply segment, minor time loss. \index{Story Beat (SB)!1 SB}
\item[\textbf{2 SB}] Moderate setback: alarm raised, lose favorable position/cover, lesser foe appears, added obstacle. \index{Story Beat (SB)!2 SB}
\item[\textbf{3 SB}] Serious trouble: reinforcements arrive, key gear breaks, significant complication introduced. \index{Story Beat (SB)!3 SB}
\item[\textbf{4+ SB}] Major turn: trap springs, authority arrives, scene shifts dramatically. \index{Story Beat (SB)!4+ SB}
\end{description}

\subsection{Combat-Specific SB Options}
\label{subsec:combat-sb}
\index{Story Beat (SB)!combat}

\begin{description}
\item[\textbf{1 SB}] Lose footing (next defense –1 die), minor environmental shift.
\item[\textbf{2 SB}] Weapon jam or battlefield momentum shifts (fire spreads, cave-in starts, cavalry arrives).
\item[\textbf{3 SB}] Pinned, disarmed, or separated from allies.
\item[\textbf{4+ SB}] Enemy reveals a special ability, terrain collapses, a major reinforcement wave hits.
\end{description}

\subsection{Stealth \& Intrusion SB Options}
\label{subsec:stealth-sb}
\index{Story Beat (SB)!stealth}

\begin{description}
\item[\textbf{1 SB}] Footstep heard, door squeaks, shadow noticed.
\item[\textbf{2 SB}] Patrol adjusts, lock resists (extra test), guard becomes suspicious.
\item[\textbf{3 SB}] Partial alarm triggered (localized response).
\item[\textbf{4 SB}] Full alarm and lockdown protocol.
\end{description}

\subsection{Social Interaction SB Options}
\label{subsec:social-sb}
\index{Story Beat (SB)!social}

\begin{description}
\item[\textbf{1 SB}] Faux pas (future interactions with this contact –1 die), rumor spreads.
\item[\textbf{2 SB}] Concession required (gift, favor, or compromise to proceed).
\item[\textbf{3 SB}] Rival interjects with leverage; negotiation turns against you.
\item[\textbf{4 SB}] Patron turns hostile; audience becomes antagonistic.
\end{description}

\subsection{Travel \& Survival SB Options}
\label{subsec:travel-sb}
\index{Story Beat (SB)!travel}

\begin{description}
\item[\textbf{1 SB}] Lose time, minor injury, weather worsens.
\item[\textbf{2 SB}] Supply clock +1 segment, mount lamed, gear damaged.
\item[\textbf{3 SB}] Wrong path or blocked pass; all characters gain Fatigue 1.
\item[\textbf{4 SB}] Major environmental event—storm, rockslide, flood—scene fundamentally changes.
\end{description}

\subsection{Arcana \& Ritual SB Options}
\label{subsec:arcana-sb}
\index{Story Beat (SB)!arcana}

\begin{description}
\item[\textbf{1 SB}] Backlash prickle, sensory bleed, minor magical residue.
\item[\textbf{2 SB}] Unintended side-effect (e.g., cold off a fire working; echoes draw attention).
\item[\textbf{3 SB}] Residue anchors a hex or attracts supernatural attention.
\item[\textbf{4 SB}] Significant backlash condition or manifestation; ritual mark persists with ongoing effects.
\end{description}

\paragraph{High-Tier SB Sinks} \index{Story Beat (SB)!high-tier}
For major 3–6+ SB spends that affect the campaign world (reputation cascades, faction instability, magical resonance, prophecy triggers), use advanced complications rules. A practical default: \emph{at the end of a journey leg,} \textbf{3 SB → advance 1 Campaign Front}.

\section{Fail Forward: Every Roll Matters}
\label{sec:fail-forward}
\index{Fail Forward}\index{Boons}

When a character \textbf{misses} (0 successes) on a \emph{significant action}, they gain \textbf{2 Boons} and one on a \textbf{partial} success. Boons represent insight, opportunity, or a sudden edge that can be spent later.

\subsection{Significant Action Criteria}
\label{subsec:significant-action}
\index{Significant Action}

A miss or partial success awards Boons only if \textbf{all three} are true:
\begin{enumerate}
\item \textbf{Procedure Followed} — Intent and approach declared; DV set; roll resolved. \index{Core Mechanic!Procedure}
\item \textbf{Stakes Stated} — What changes on success; what lands on failure. \index{Stakes}
\item \textbf{Consequence Lands} — The GM spends or banks SB, applies a condition, or advances a thread. \index{Consequences}
\end{enumerate}

\subsection{Actions That Do \emph{Not} Award Boons}
\label{subsec:no-boon-actions}
\begin{itemize}
\item Rehearsals or null-risk probes with trivial stakes.
\item Repeated identical attempts in the same scene \emph{without} a new approach, position, or stakes.
\item Actions whose fallout would be trivial or purely informational.
\end{itemize}

\subsection{Additional Boon Sources}
\label{subsec:other-boon-sources}
\index{Boons!sources}
\begin{itemize}
\item Strong bond-driven play that highlights relationships.
\item Creative solutions to complex problems (GM discretion).
\item Sacrifices made for the group or greater good.
\item Spotlighting character flaws or complications.
\end{itemize}

\subsection{Boon Economy and Limits}
\label{subsec:boon-economy}
\index{Boons!economy}
\begin{description}
\item[\textbf{Holding Cap}] Hold up to \textbf{5} Boons. \index{Boons!cap}
\item[\textbf{Scene Carryover}] At scene end, trim to \textbf{2} Boons (excess lost). \index{Boons!carryover}
\item[\textbf{Spending}] Spend in-scene for re-rolls, Asset activations, Rites, or special abilities. \index{Boons!spending}
\item[\textbf{Multi-Phase Scenes}] For extended set pieces (chase → duel → escape), trim to 2 only after the sequence ends. \index{Boons!multi-phase}
\end{description}

\subsection{Rites \& Assets: Practical Notes}
\label{subsec:rite-asset-notes}
High-power Rites that require 2 Boons remain viable—characters can start a scene with 2 Boons and must earn more to chain further Invokes. On-screen Asset activations cost \textbf{1 Boon} as normal. \index{Rites}\index{Assets!activation}

\subsection{Anti-Fishing Measures}
\label{subsec:anti-fishing}
\index{Boons!limits}
Optional stability rules:
\begin{itemize}
\item \textbf{Failure Limit:} Max \textbf{2 Boons from failures per character per scene}. Further misses still generate SB but no Boon.
\item \textbf{Repetition Rule:} Same approach with identical stakes in the same scene cannot award another Boon.
\item \textbf{Position Gate:} Controlled tests with trivial fallout do not award Boons.
\end{itemize}

\subsection{Practical Examples}
\label{subsec:boon-examples}
\begin{itemize}
\item \textbf{Boons Awarded:} Picking a lock under watch (Desperate, DV 3). Stakes: success opens door; miss triggers alarm. Roll misses; GM spends 2 SB to start \emph{Guards Incoming} [6]. Player gains 2 Boons.
\item \textbf{No Boon:} Tapping flagstones "just in case" (Controlled, no stakes). Info-only; no SB spent. No Boon.
\item \textbf{Carryover:} End of scene, character holds 4 Boons → trim to 2. Next scene, they earn/spend freely (never exceeding 5); trim to 2 when that scene ends.
\end{itemize}

\section{Boon Conversion and Advancement}
\label{sec:boon-conversion}
\index{Boons!conversion}
\begin{itemize}
\item \textbf{Conversion Rate:} Once per session during downtime, convert \textbf{2 Boons → 1 XP}. \index{Experience Points!from Boons}
\item \textbf{Limit:} Max \textbf{2 XP/session} via conversion. \index{Experience Points!limits}
\item \textbf{Timing:} Between scenes or during downtime only. \index{Downtime!Boon conversion}
\end{itemize}

\subsection{Boon Sharing}

Players may gift \textbf{1 Boon per scene} to an ally with a brief narrative justification.  
\begin{itemize}
  \item \textbf{Bonded Allies:} If characters share a bond, they may gift \textbf{2 Boons per scene}.  
  \item \textbf{Assistance:} Boons may be spent to enhance an ally’s roll (counts as assistance).  
  \item \textbf{Campaign Events:} Major victories or setbacks may generate shared Boons for the party.  
\end{itemize}

\textbf{Table Use:} Require a short story beat for each gift. Normal Boon limits apply. Track shared Boons openly.  
\textbf{GM Notes:} Reward generosity with extra opportunities, introduce occasional complications from dependence, and balance group vs.\ individual needs.

\subsection{Position}
\label{subsec:position}
\index{Position}

Every action in \indexterm{Fate's Edge} takes place from a \textbf{Position} that reflects the character’s advantage or disadvantage in the scene. Position sets the tone for the roll, narratively and mechanically. It comes in three states:

\begin{itemize}
  \item \textbf{Dominant:} You act from a place of control, leverage, or overwhelming advantage.
  \item \textbf{Controlled:} The standard state of play. Outcomes are uncertain but balanced.
  \item \textbf{Desperate:} You act from dire straits, cornered or overmatched, with everything at stake.
\end{itemize}

\paragraph{Re-roll Mechanic.}  
Position modifies the dice pool through simple re-rolls:
\begin{center}
\begin{tabular}{@{}lll@{}}
\toprule
\textbf{Position} & \textbf{Narrative Frame} & \textbf{Mechanical Effect} \\
\midrule
Dominant & You press your advantage & Re-roll one \emph{failure} \\
Controlled    & The balanced norm & No re-rolls \\
Desperate & You act under duress & Re-roll one \emph{success} \\
\bottomrule
\end{tabular}
\end{center}

\section{Asset Activation Mechanics}
\label{sec:asset-activations}
\index{Assets!activation}
Players can activate Assets in several ways:
\begin{description}
\item[\textbf{Free Off-Screen}] Each Asset's off-screen effect \emph{once per session} for free. \index{Assets!off-screen}
\item[\textbf{XP Activation}] Spend \textbf{2 XP} to trigger an extra off-screen effect beyond the session allowance. \index{Assets!XP activation}
\item[\textbf{Boon Activation}] Spend \textbf{1 Boon} to bring an Asset's influence on-screen now. \index{Assets!Boon activation}
\item[\textbf{Plausibility Test}] The Asset must have scope/reach appropriate to the effect. \index{Assets!scope}
\end{description}

\subsection{Initiative and Turn Order}

Fate's Edge does not use fixed initiative. 
Turn order follows the fiction and the GM's facilitation:
\begin{itemize}
    \item \textbf{Narrative Fiat:} The GM frames spotlight order based on circumstances, tension, and narrative flow.
    \item \textbf{Player Input:} Players may suggest acting when it makes sense in the fiction. 
    \item \textbf{Surprise:} Ambushers act first; targets respond after the opening exchange.
    \item \textbf{Flexibility:} Spotlight may shift mid-scene if fictionally appropriate (e.g., reacting to a falling ceiling, seizing a moment).
\end{itemize}

This ensures pacing and drama guide the sequence of actions, not rigid turn structures.
% =========================
% Turn Economy (Quick Rules)
% =========================
\subsection{Turn Economy (Quick Rules)}
\label{subsec:turn-economy-quick}

\paragraph{Two Actions.}
Each character takes \emph{1 Action and 1 Move} on their turn. Actions and Moves may be taken in any order; repeating the same Action is allowed unless noted.

\paragraph{Move.}
Traverse up to your normal movement. \emph{Disengage:} move without provoking; your next offensive action is \textbf{Controlled}. \emph{Dash:} move again this turn; your next defense is \textbf{Desperate}.

\paragraph{Attack.}
Make a melee or ranged attack versus DV set by the GM and fiction. Teamwork/Assist costs 1 Boon.

\paragraph{Observe / Change Position (+1).}
Take a beat to read the field or set angles; gain \textbf{+1 Position} for one action this turn (e.g., Controlled$\to$Dominant). Limit: once/turn; cannot exceed \textbf{Dominant}.

\paragraph{Activate an Asset.}
Use gear, symbol, tool, or feature per its text/tags (e.g., torch, grapnel, smoke vial, rune focus). Items with \texttt{[Action]} consume one Action; \texttt{[Free]} do not.

\paragraph{Setup (Teamwork).}
Create advantage for an ally; on success, grant their next action \textbf{+1 Position} or step up Effect (GM’s call).

\paragraph{Assist (Teamwork).}
Spend \emph{1 Boon} to give an ally \emph{+1 die} on their current roll; you share appropriate risk/consequence.

\paragraph{Defend / Protect.}
Adopt a guarding stance or body-block. Choose a nearby ally; until your next turn you may intercept one hit on them and roll to resist it. On success, reduce/negate Harm; you take any fallout the GM assigns.

\paragraph{Channel / Weave.}
Runekeeper/ritual flow: \emph{Channel} (prime power) then \emph{Weave} (shape/release). Disruption or engagement may worsen Position; if \emph{Interrupted}, the casting fails.

\paragraph{Cast Rite / Song (Cantor).}
Perform a Rite/Song per its write-up. You may \emph{Push} to accelerate or empower at the cost of Fatigue/Corruption per class rules.

\paragraph{Interact.}
Lift, pull, flip a lever, shove a foe, break an object, apply a poultice, reload, draw/stow, etc. GM sets DV/Effect.

\paragraph{Free Items.}
Short shouts, dropping an item, quick glance. Longer or tactical assessments require \emph{Observe / Change Position} or \emph{Interact}.

\paragraph{Reactions (Out of Turn).}
\emph{Protection} may trigger when an ally is hit and you are in position. Class/Asset reactions fire as written (e.g., counter-runes, ripostes).

\paragraph{Position Caps.}
Bonuses cannot raise Position above \textbf{Dominant}; penalties cannot drop below \textbf{Desperate}. Beyond these caps, adjust DV or Effect instead.

\section{Experience Point Economy}
\label{sec:xp-economy}
\index{Experience Points}

\subsection{Session Awards}
\label{subsec:session-awards}
\index{Experience Points!session awards}
\begin{description}
\item[\textbf{Table Attendance}] +2 XP
\item[\textbf{Major Objective}] +2–4 XP
\item[\textbf{Discovery}] +1–2 XP
\item[\textbf{Hard Choice}] +1–2 XP
\item[\textbf{Complication Spotlight}] +1–3 XP
\item[\textbf{Bond-Driven Play}] +1–2 XP
\item[\textbf{GM Curveball}] +0–3 XP
\end{description}

\subsection{Milestone Awards}
\label{subsec:milestone-awards}
\index{Experience Points!milestones}
At the end of a major arc:
\begin{itemize}
\item +8–12 XP to all players (arc completion)
\item +2 XP to one player for a signature moment
\end{itemize}

\subsection{Complication Dividend}
\label{subsec:complication-dividend}
\index{Experience Points!dividends}
\begin{itemize}
\item Resolve a Face-card complication: +1 XP
\item Resolve an Ace complication: +2 XP
\end{itemize}

\subsection{XP Spending Costs}
\label{subsec:xp-spending}
\index{Experience Points!spending}
\begin{description}
\item[\textbf{Attributes}] Cost = new rating $\times$ 3 XP; downtime = new rating (days). \index{Attributes!XP costs}
\item[\textbf{Skills}] Cost = new level $\times$ 2 XP; downtime = new level (days). \index{Skills!XP costs}
\item[\textbf{Followers (on-screen)}] Cost = Cap$^{2}$ XP; 1–3 days to recruit/brief. \index{Followers!XP costs}
\item[\textbf{Assets (off-screen)}] Minor (4 XP, 1 day), Standard (8 XP, 1 week), Major (12 XP, 1 month). \index{Assets!XP costs}
\end{description}

\section{Rush Rule for Advancement}
\label{sec:rush-rule}
\index{Advancement!rush rule}
A player may skip required downtime for an advance; the GM creates a \textbf{Haste} [4] clock. If it fills during the rushed period, the new ability or Asset arrives with flaws or narrative complications.

\section{Tiers of Reputation}
\label{sec:reputation-tiers}
\index{Reputation Tiers}

\begin{description}
\item[\textbf{Tier I — Rookie}] (0–40 XP): Local reputation; prestige abilities locked. \index{Reputation!Tier I}
\item[\textbf{Tier II — Seasoned}] (41–90 XP): Regional notice; prestige abilities may unlock. \index{Reputation!Tier II}
\item[\textbf{Tier III — Veteran}] (91–150 XP): National influence; second follower slot suggested. \index{Reputation!Tier III}
\item[\textbf{Tier IV — Paragon}] (151–220 XP): Movers and shakers; rivals emerge to challenge. \index{Reputation!Tier IV}
\item[\textbf{Tier V — Mythic}] (221+ XP): Legendary status; kingdoms and cults respond directly. \index{Reputation!Tier V}
\end{description}

\subsection{Recommended Session Order (GM Checklist)}

\paragraph{1) Off-Screen (Downtime, 10–20 min)}
\begin{itemize}
  \item Upkeep: choose Efficient/Intensive; apply Neglected/Compromised if missed.
  \item Obligation: clear via Acts of Service; note Claims/overflow risk.
  \item Projects: tick long-term clocks; resolve Gather Info; prep assets.
  \item Intent: each player states one on-screen goal; GM surfaces 1–2 front pressures.
\end{itemize}

\paragraph{2) On-Screen (Scenes)}
\begin{itemize}
  \item Frame hard: where/what’s at stake; set Position $\to$ DV.
  \item Run spotlight: rotate beats; fold in bonds and Boon sharing.
  \item Advance: move faction/Patron clocks openly when triggered.
\end{itemize}

\paragraph{3) Wrap-Up (5–10 min)}
\begin{itemize}
  \item XP \& Talents: award, mark progress; note any Gifts gained/forfeit.
  \item SB \& Harm: convert Fatigue$\to$Harm if full; apply recoveries.
  \item Fronts: advance unresolved clocks; note consequences.
\end{itemize}

\paragraph{4) Off-Screen Hooks (2–5 min)}
\begin{itemize}
  \item Log next Downtime intents, service opportunities, upkeep deadlines.
  \item Capture cliffhangers and Patron Largess seeds for next session open.
\end{itemize}

\emph{Optional:} Add a cold open flash-cut before Step 2 to spotlight a rival or Patron omen.

\subsection{Fear Effects Table}
\label{subsec:fear-table}

When a character escalates on the Fear Track (Shaken $\rightarrow$ Frightened $\rightarrow$ Panicked), roll on the following table or choose an appropriate effect. These results apply primarily to NPCs, though PCs may adopt them as narrative guidance.

\begin{center}
\begin{tabular}{>{\bfseries}r l l}
\toprule
d10 & Effect & Magic Tags \\
\midrule
1 & \textbf{Freeze}: Cannot act this round, staring or trembling. & Silence, Stasis \\
2 & \textbf{Flee}: Must move at full speed away from the source of Fear. & Movement, Wind \\
3 & \textbf{Drop}: Character drops what they are holding. & Disarm, Break \\
4 & \textbf{Beg}: Character pleads or bargains incoherently. & Compulsion, Voice \\
5 & \textbf{Hide}: Seeks cover, concealment, or allies to cling to. & Shadow, Illusion \\
6 & \textbf{Attack in Panic}: Lashes out wildly at the nearest target. & Rage, Fire \\
7 & \textbf{Blunder}: Stumbles into danger (trap, hazard, off balance). & Chaos, Trickery \\
8 & \textbf{Obey}: Instinctively follows a simple command from the fear-causer. & Command, Charm \\
9 & \textbf{Break Down}: Sobs, prays, or becomes useless until aided. & Curse, Despair \\
10 & \textbf{Catatonia}: Becomes unresponsive, requiring intervention. & Sleep, Dream \\
\bottomrule
\end{tabular}
\end{center}

\paragraph{Note.}  
At GM discretion, results may escalate with each step of the Fear Track:  
- \emph{Shaken}: Apply minor versions (hesitation, lost die, startled).  
- \emph{Frightened}: Roll normally.  
- \emph{Panicked}: Apply severe or exaggerated results (e.g., 2 = reckless flight, 6 = attack allies).  