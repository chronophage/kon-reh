\chapter{Core Mechanics}
\label{chap:core-mechanics}

\section{Adjudicating Rolls: The Core Resolution Cycle}\index{roll adjudication}

When a player rolls, they are not simply trying to \emph{beat a number}. They are engaging the world through risk, consequence, and discovery. This section walks through the full cycle.

\subsection{Step-by-Step Roll Resolution}

\begin{enumerate}
    \item \textbf{Declare Action \& Approach:} Player states intent, Attribute + Skill.
    \item \textbf{Set Difficulty Value (DV):}\index{Difficulty Value (DV)} Based on narrative stakes, not just mechanics.
    \item \textbf{Roll Pool of d10s.}
    \item \textbf{Count:} \textbf{Successes (6+)} and \textbf{Story Beats (1s)}.\index{Story Beats}
    \item \textbf{Check Against DV:} Apply the Outcome Matrix. Note: \textbf{each 10 counts as 2 successes}.
    \item \textbf{Spend SB:} GM spends/banks Story Beats or draws from the Deck of Consequences.\index{Deck of Consequences}
\end{enumerate}

\begin{fatebox}[Difficulty Ladder]\index{Difficulty Ladder}
\begin{tabularx}{\textwidth}{lX}
\toprule
\textbf{DV} & \textbf{Typical Case} \\
\midrule
3 & Routine: clear intent, modest stakes, controlled setting \\
4 & Pressured: time limits, mild resistance, partial info \\
5 & Hard: hostile conditions, active opposition, precision required \\
6+ & Extreme: stacked constraints, dangerous failure, high drama \\
\bottomrule
\end{tabularx}
\end{fatebox}

\begin{tcolorbox}[title=\textbf{Difficulty Values (DV) by Tier},colback=white!97!gray,
colframe=black!80!gray,sharp corners,boxrule=0.4pt]
\textbf{Guideline.}
The base Difficulty Value (DV) for an opposed or environmental test scales with Tier:
\[
\boxed{\text{DV} = \text{Tier} + 2 + \text{Modifiers}}
\]

\textbf{Typical DVs.}
\begin{center}
\renewcommand{\arraystretch}{1.1}
\begin{longtable}{l c c}
\toprule
\textbf{Tier} & \textbf{Base DV} & \textbf{Example Challenge}\\
\midrule
I & 5 & Local threat / novice test\\
II & 6 & Veteran foe or skilled task\\
III & 7 & Elite / magical challenge\\
IV & 8 & Mythic or cosmic threat\\
\bottomrule
\end{longtable}
\end{center}

\textbf{Positional Modifiers.}
\begin{itemize}[leftmargin=*]
  \item \textbf{Desperate:} +2  \textbf{Controlled:} +1  \textbf{Controlled:} +0  \textbf{Dominant:} –1
\end{itemize}
Use \textit{DV = Tier + 2} as the default; adjust for environment, advantage, or narrative pressure.
\end{tcolorbox}

\textit{A DV should measure narrative weight as much as difficulty. Scaling a wall is routine. Scaling it while lantern-wardens pursue is pressured—or worse.}

\begin{fatebox}[Outcome Matrix]\index{Outcome Matrix}
\begin{tabularx}{\textwidth}{lX}
\toprule
\textbf{Result} & \textbf{GM Guidance} \\
\midrule
$S \geq DV$, $C = 0$ & Clean Success: Grant intent, no added friction. \\
$S \geq DV$, $C > 0$ & Success \& Cost: Intent achieved; GM spends SB for complications. \\
$0 < S < DV$ & Partial: Progress \emph{proportional} to hits; intent advances but with gaps or risk. Player gains 1 Boon. \\
$S = 0$ & Miss: No progress. GM escalates with SB/Clocks. Player gains 2 Boons. \\
\bottomrule
\end{tabularx}
\end{fatebox}

\section{Fail Forward: Every Roll Matters}\index{Fail Forward}

\textbf{Partials are the most common form of “success.”} They always move the fiction forward in proportion to the progress rolled.

\begin{quote}
\textit{One success on DV 4:} “The lock is stubborn. You think you can get it if you keep trying.”  
\textit{Three successes on DV 4:} “The lock springs open with a loud clank—you’re sure the guards heard.” (Upgrade to Success \& Cost; add 2 SB).  
\end{quote}

\textbf{Misses} fuel escalation but always generate player resources: 2 Boons and a consequence.  

A roll is \emph{meaningful} if:  
\begin{enumerate}
  \item The standard procedure is followed (intent + DV + roll).  
  \item Stakes are stated up front (what changes on success, what bites on failure).  
  \item Real consequences occur now (SB spent, condition applied, or thread advanced).  
\end{enumerate}

\subsection{Important Notes}
\begin{itemize}
  \item Rolling a \textbf{1} always creates SB for the GM. Rerolls do not erase SB.  
  \item No Boons for rehearsal, trivial probes, or repeating an identical approach without changing fiction.  
  \item Controlled tests with no bite give positioning/info, not Boons.  
\end{itemize}

\subsection{Anti-Fishing Measures}
\begin{itemize}
  \item \textbf{Cap:} At most 2 Boons from failures per character per scene (further misses still make SB).  
  \item \textbf{Repetition Rule:} Same action + same stakes in the same scene can’t grant another Boon.  
\end{itemize}

\subsection{Example}
Lockpicking under watch (\emph{Desperate}, DV 3).  
\textbf{Miss:} GM spends 2 SB to start \emph{Guards Incoming [6]}. Player earns 2 Boons.  
\textbf{Partial (2 successes):} Door opens halfway; guard footsteps approach. Player earns 1 Boon.  

% --- SRD CORE RULE ---

\section{Standard Rule: Player-Managed Modules}
\label{sec:player-managed-modules}

This rule makes each player the primary steward of their character-facing trackers (\emph{modules}). It keeps table pace high, reduces hidden bookkeeping, and clarifies when mechanical thresholds trigger. The GM retains authority over world-facing clocks, faction fronts, and all major narrative consequences.

\subsection{Scope (\emph{What Counts as a Module})}
Player-managed modules are any \textbf{character-facing} clocks, counters, or discrete states that sit on a single character sheet:
\begin{itemize}
  \item \textbf{Obligation} (per Patron or Symbol).
  \item \textbf{Corruption Clock} (e.g., Cantor).
  \item \textbf{Leash} (Summoned spirit strain) and \textbf{Spirit Bond Clock}.
  \item \textbf{Repertoire Clock} (Cantor) or similar progression clocks.
  \item \textbf{Asset States} (e.g., Symbol: Maintained / Neglected / \textsc{Compromised} / \textsc{Shattered}).
  \item \textbf{Scene Counters} explicitly tied to a PC (e.g., Exposure on that PC, personal Buff/Debuff durations).
\end{itemize}
\textit{Not included:} GM story resources (global \textbf{Story Beats}), location/faction clocks, and mystery/doom fronts.

\begin{tcolorbox}[colback=black!3,colframe=black!40!white,title={What Players Track (at a Glance)}]
\begin{tabularx}{\textwidth}{l l X}
\toprule
\textbf{Module} & \textbf{Owner} & \textbf{Tick / Change Triggers (examples)} \\
\midrule
Obligation (by Patron) & Player & Invoke/Push/ritual text; Invoker \emph{Borrowed Grace}; cracking a Symbol; bargain costs. \\
Corruption Clock & Player & Cantor Push; Resonant Rite; GM spends a Beat tied to the PC’s occult actions. \\
Leash (Summoning) & Player & Harm to spirit; commands against nature; split focus; crossing \texttt{[WARD]} (DV = Cap). \\
Spirit Bond [4] & Player & Shared victories, mutual aid, meaningful attempts (\emph{near-miss progress} once/session/type). \\
Repertoire [6] & Player & Learn a new unique Song/rite-as-song; practice milestones. \\
Asset State (Symbol) & Player & Maintenance/downtime checks; \emph{Crack the Seal} \(\rightarrow\) \textsc{Compromised}; breakage \(\rightarrow\) \textsc{Shattered}. \\
\bottomrule
\end{tabularx}
\end{tcolorbox}

\subsection{Core Principle}
Players \textbf{immediately} mark their own modules when a rule says ``mark $+X$'' or a trigger fires. Threshold effects resolve as soon as they are reached.

\subsection{Player Duties}
\begin{enumerate}
  \item \textbf{Mark Increases/Decreases on Cue.} When you Invoke a Rite, Push, spend/clear per rules text, or a trigger fires, update your module \emph{now}, not later.
  \item \textbf{Declare Thresholds.} If marking fills a clock or crosses capacity, say so aloud; thresholds resolve before the scene proceeds.
  \item \textbf{State Ownership.} Keep per-Patron Obligation tallies distinct; track each Symbol’s state if you use Symbols.
  \item \textbf{Keep It Visible.} Use a tracker the GM and table can see (sheet boxes, index cards, or shared digital).
\end{enumerate}

\subsection{GM Duties}
\begin{enumerate}
  \item \textbf{Spot-Check.} At need, ask any player: current Obligation by Patron, Corruption segments, Leash state, Asset states.
  \item \textbf{Enforce Thresholds.} When a player reports a threshold, apply the standard effects below \emph{immediately}.
  \item \textbf{Own the Fallout.} Patron intrusions, faction reactions, front clocks, and major narrative consequences remain GM authority.
\end{enumerate}

\subsection{Standard Thresholds \& Effects}
\paragraph{Obligation Capacity}
\label{sec:obligation-capacity}
\[
\textbf{Obligation Capacity} \;=\; \textit{Spirit} + \textit{Presence}
\]
\begin{itemize}
  \item \textbf{Over Capacity:} Immediately mark \textbf{+1 Fatigue} per segment over capacity.
  \item \textbf{Over \(\mathbf{2\times}\) Capacity:} Immediately clear all Fatigue, mark \textbf{+1 Harm (Stress)}, and a \textbf{Patron Intrusion} occurs (GM frames on-theme demand/complication).
\end{itemize}

\paragraph{Corruption Full}
When a \textbf{Corruption Clock} fills:
\begin{itemize}
  \item Apply the last-Patron \textbf{benefit \& burden} (per Patron table or setting guidance) to the PC (and any listed followers/retainers).
  \item \textbf{Reset} the Corruption Clock to empty.
  \item If the player chooses \textbf{Embrace Corruption}, convert the current Patron theme into a permanent boon/curse per \S\ref{subsec:corruption-fading}.
\end{itemize}

\paragraph{Leash Full (Summoning)}
When the \textbf{Leash} fills:
\begin{itemize}
  \item The spirit acts once to its nature, then \textbf{departs} (or turns hostile at GM discretion and fiction).
\end{itemize}
\textbf{Leash Capacity:} \(\textit{Cap} + \textit{Spirit}\) segments. (\textit{Cap} is the outsider’s tier: Cap~1 for Lesser, Cap~3 for Greater.)

\paragraph{Symbol State (Invoker)}
\begin{itemize}
  \item \textbf{Maintained} \(\rightarrow\) normal function. \quad
        \textbf{Neglected} \(\rightarrow\) GM may impose $+1$ DV to related rites.
  \item \textbf{\textsc{Compromised}} (e.g., \emph{Crack the Seal}) \(\rightarrow\) instant resolution per rules; mark extra Obligation; repair in Downtime or pay 1 XP.
  \item \textbf{\textsc{Shattered}} \(\rightarrow\) unusable until replaced or ritually restored per fiction.
\end{itemize}

\subsection{Table Procedure (90-Second Loop)}
\paragraph{Start of Session}
Players read out: per-Patron \textbf{Obligation} totals, \textbf{Corruption} segments, standing \textbf{Asset States}, and any personal clocks at 3+.

\paragraph{End of Scene}
Quick pass: ``\emph{Any marks?}'' Players tick modules from scene events. If a threshold triggers, resolve now.

\paragraph{Downtime}
Players apply clears (service, contrition, purification, study) to their own modules. GM verifies any costs or fiction.

\subsection{Disputes \& Order of Operations}
If two marks would land simultaneously, apply them in the \textbf{least advantageous order for the acting character}, unless a rule specifies otherwise. The GM is final arbiter.

\subsection{Accessibility \& Tools}
Use highly visible trackers: bold boxes on sheets, poker chips for segments, or a shared table of per-Patron Obligation. Keep modules at-a-glance to minimize interruption.

\subsection{Worked Micro-Examples}
\begin{itemize}
  \item \textbf{Invoker Rites Twice:} Vessa Invokes two different Patrons. She marks each Patron’s \textbf{Obligation} separately. Hitting capacity with Patron A causes Fatigue; Patron B remains below capacity.
  \item \textbf{Cantor Pushes:} Jorel Pushes a Song (mark +1 Corruption). That fill triggers the last-Patron boon/burden immediately; then he resets to 0.
  \item \textbf{Summoner Clash:} Kestra’s Cap~3 elemental takes Harm and crosses a \texttt{[WARD]}; she ticks her \textbf{Leash} twice. On fill, the elemental flares once and departs.
\end{itemize}

\subsection{Boon Sharing}
Players may gift 1 Boon per scene to an ally with narrative justification.  
\begin{itemize}
  \item \textbf{Bonded Allies:} Up to 2 Boons gifted per scene.  
  \item \textbf{Assistance:} Shared Boons can enhance an ally’s roll.  
  \item \textbf{Campaign Events:} Major milestones may generate party-wide Boons.  
\end{itemize}

\textbf{GM Note:} Encourage gifts with roleplay beats, but balance generosity with potential dependency or group tension.

\subsection{Critical Success}

Rolling a \textbf{10} on any die indicates a critical tier of success. Each 10 adds weight to the outcome:

\begin{itemize}
  \item \textbf{One 10:} Strong success with a free boon, improved Position, or other narrative flourish.
  \item \textbf{Two 10s:} Exceptional success; choose two benefits or a single powerful effect.
  \item \textbf{Three 10s:} Legendary success; resolve the conflict dramatically and progress or clear 1 segment on a secondary clock (generally, a clock tied to the scene, not the overarching campaign).
  \item \textbf{Four+ 10s:} Mythic success; progress or clear 1--2 segments from a secondary clock or create a significant story development.
\end{itemize}

\noindent If no 10s are rolled, resolve the action normally by the highest die result.

\noindent \textbf{10s are never re-rolled by Position effects or other mechanics. Critical hit effects always take place if the roll is successful, despite any SB rolled. Critical successes may reduce Backlash/Obligation/Corruption severity by one tier.}
\section{SB Spend Menu}
\label{sec:sb-spend-menu}
\index{Story Beat (SB)!spend menu}

\subsection{Universal SB Options}
\label{subsec:universal-sb}
\index{Story Beat (SB)!universal}

\begin{description}
\item[\textbf{1 SB}] Minor pressure: suspicious noise, trace left behind, +1 Supply segment, minor time loss. \index{Story Beat (SB)!1 SB}
\item[\textbf{2 SB}] Moderate setback: alarm raised, lose favorable position/cover, lesser foe appears, added obstacle. \index{Story Beat (SB)!2 SB}
\item[\textbf{3 SB}] Serious trouble: reinforcements arrive, key gear breaks, significant complication introduced. \index{Story Beat (SB)!3 SB}
\item[\textbf{4+ SB}] Major turn: trap springs, authority arrives, scene shifts dramatically. \index{Story Beat (SB)!4+ SB}
\end{description}

\subsection{Combat-Specific SB Options}
\label{subsec:combat-sb}
\index{Story Beat (SB)!combat}

\begin{description}
\item[\textbf{1 SB}] Lose footing (next defense –1 die), minor environmental shift.
\item[\textbf{2 SB}] Weapon jam or battlefield momentum shifts (fire spreads, cave-in starts, cavalry arrives).
\item[\textbf{3 SB}] Pinned, disarmed, or separated from allies.
\item[\textbf{4+ SB}] Enemy reveals a special ability, terrain collapses, a major reinforcement wave hits.
\end{description}

\subsection{Stealth \& Intrusion SB Options}
\label{subsec:stealth-sb}
\index{Story Beat (SB)!stealth}

\begin{description}
\item[\textbf{1 SB}] Footstep heard, door squeaks, shadow noticed.
\item[\textbf{2 SB}] Patrol adjusts, lock resists (extra test), guard becomes suspicious.
\item[\textbf{3 SB}] Partial alarm triggered (localized response).
\item[\textbf{4 SB}] Full alarm and lockdown protocol.
\end{description}

\subsection{Social Interaction SB Options}
\label{subsec:social-sb}
\index{Story Beat (SB)!social}

\begin{description}
\item[\textbf{1 SB}] Faux pas (future interactions with this contact –1 die), rumor spreads.
\item[\textbf{2 SB}] Concession required (gift, favor, or compromise to proceed).
\item[\textbf{3 SB}] Rival interjects with leverage; negotiation turns against you.
\item[\textbf{4 SB}] Patron turns hostile; audience becomes antagonistic.
\end{description}

\subsection{Travel \& Survival SB Options}
\label{subsec:travel-sb}
\index{Story Beat (SB)!travel}

\begin{description}
\item[\textbf{1 SB}] Lose time, minor injury, weather worsens.
\item[\textbf{2 SB}] Supply clock +1 segment, mount lamed, gear damaged.
\item[\textbf{3 SB}] Wrong path or blocked pass; all characters gain Fatigue 1.
\item[\textbf{4 SB}] Major environmental event—storm, rockslide, flood—scene fundamentally changes.
\end{description}

\subsection{Arcana \& Ritual SB Options}
\label{subsec:arcana-sb}
\index{Story Beat (SB)!arcana}

\begin{description}
\item[\textbf{1 SB}] Backlash prickle, sensory bleed, minor magical residue.
\item[\textbf{2 SB}] Unintended side-effect (e.g., cold off a fire working; echoes draw attention).
\item[\textbf{3 SB}] Residue anchors a hex or attracts supernatural attention.
\item[\textbf{4 SB}] Significant backlash condition or manifestation; ritual mark persists with ongoing effects.
\end{description}

\paragraph{High-Tier SB Sinks} \index{Story Beat (SB)!high-tier}
For major 3–6+ SB spends that affect the campaign world (reputation cascades, faction instability, magical resonance, prophecy triggers), use advanced complications rules. A practical default: \emph{at the end of a journey leg,} \textbf{3 SB → advance 1 Campaign Front}.

\section{Fail Forward: Every Roll Matters}
\label{sec:fail-forward}
\index{Fail Forward}\index{Boons}

When a character \textbf{misses} (0 successes) on a \emph{significant action}, they gain \textbf{2 Boons} and one on a \textbf{partial} success. Boons represent insight, opportunity, or a sudden edge that can be spent later.

\subsection{Significant Action Criteria}
\label{subsec:significant-action}
\index{Significant Action}

A miss or partial success awards Boons only if \textbf{all three} are true:
\begin{enumerate}
\item \textbf{Procedure Followed} — Intent and approach declared; DV set; roll resolved. \index{Core Mechanic!Procedure}
\item \textbf{Stakes Stated} — What changes on success; what lands on failure. \index{Stakes}
\item \textbf{Consequence Lands} — The GM spends or banks SB, applies a condition, or advances a thread. \index{Consequences}
\end{enumerate}

\subsection{Actions That Do \emph{Not} Award Boons}
\label{subsec:no-boon-actions}
\begin{itemize}
\item Rehearsals or null-risk probes with trivial stakes.
\item Repeated identical attempts in the same scene \emph{without} a new approach, position, or stakes.
\item Actions whose fallout would be trivial or purely informational.
\end{itemize}

\subsection{Additional Boon Sources}
\label{subsec:other-boon-sources}
\index{Boons!sources}
\begin{itemize}
\item Strong bond-driven play that highlights relationships.
\item Creative solutions to complex problems (GM discretion).
\item Sacrifices made for the group or greater good.
\item Spotlighting character flaws or complications.
\end{itemize}

\subsection{Boon Economy and Limits}
\label{subsec:boon-economy}
\index{Boons!economy}
\begin{description}
\item[\textbf{Holding Cap}] Hold up to \textbf{5} Boons. \index{Boons!cap}
\item[\textbf{Scene Carryover}] At scene end, trim to \textbf{2} Boons (excess lost). \index{Boons!carryover}
\item[\textbf{Spending}] Spend in-scene for re-rolls, Asset activations, Rites, or special abilities. \index{Boons!spending}
\item[\textbf{Multi-Phase Scenes}] For extended set pieces (chase → duel → escape), trim to 2 only after the sequence ends. \index{Boons!multi-phase}
\end{description}

\subsection{Rites \& Assets: Practical Notes}
\label{subsec:rite-asset-notes}
High-power Rites that require 2 Boons remain viable—characters can start a scene with 2 Boons and must earn more to chain further Invokes. On-screen Asset activations cost \textbf{1 Boon} as normal. \index{Rites}\index{Assets!activation}

\subsection{Anti-Fishing Measures}
\label{subsec:anti-fishing}
\index{Boons!limits}
Optional stability rules:
\begin{itemize}
\item \textbf{Failure Limit:} Max \textbf{2 Boons from failures per character per scene}. Further misses still generate SB but no Boon.
\item \textbf{Repetition Rule:} Same approach with identical stakes in the same scene cannot award another Boon.
\item \textbf{Position Gate:} Controlled tests with trivial fallout do not award Boons.
\end{itemize}

\subsection{Practical Examples}
\label{subsec:boon-examples}
\begin{itemize}
\item \textbf{Boons Awarded:} Picking a lock under watch (Desperate, DV 3). Stakes: success opens door; miss triggers alarm. Roll misses; GM spends 2 SB to start \emph{Guards Incoming} [6]. Player gains 2 Boons.
\item \textbf{No Boon:} Tapping flagstones "just in case" (Controlled, no stakes). Info-only; no SB spent. No Boon.
\item \textbf{Carryover:} End of scene, character holds 4 Boons → trim to 2. Next scene, they earn/spend freely (never exceeding 5); trim to 2 when that scene ends.
\end{itemize}

\section{Boon Conversion and Advancement}
\label{sec:boon-conversion}
\index{Boons!conversion}
\begin{itemize}
\item \textbf{Conversion Rate:} Once per session during downtime, convert \textbf{2 Boons → 1 XP}. \index{Experience Points!from Boons}
\item \textbf{Limit:} Max \textbf{2 XP/session} via conversion. \index{Experience Points!limits}
\item \textbf{Timing:} Between scenes or during downtime only. \index{Downtime!Boon conversion}
\end{itemize}

\subsection{Boon Sharing}

Players may gift \textbf{1 Boon per scene} to an ally with a brief narrative justification.  
\begin{itemize}
  \item \textbf{Bonded Allies:} If characters share a bond, they may gift \textbf{2 Boons per scene}.  
  \item \textbf{Assistance:} Boons may be spent to enhance an ally’s roll (counts as assistance).  
  \item \textbf{Campaign Events:} Major victories or setbacks may generate shared Boons for the party.  
\end{itemize}

\textbf{Table Use:} Require a short story beat for each gift. Normal Boon limits apply. Track shared Boons openly.  
\textbf{GM Notes:} Reward generosity with extra opportunities, introduce occasional complications from dependence, and balance group vs.\ individual needs.

\subsection{Position}
\label{subsec:position}
\index{Position}

Every action in \indexterm{Fate's Edge} takes place from a \textbf{Position} that reflects the character’s advantage or disadvantage in the scene. Position sets the tone for the roll, narratively and mechanically. It comes in three states:

\begin{itemize}
  \item \textbf{Dominant:} You act from a place of control, leverage, or overwhelming advantage.
  \item \textbf{Controlled:} The standard state of play. Outcomes are uncertain but balanced.
  \item \textbf{Desperate:} You act from dire straits, cornered or overmatched, with everything at stake.
\end{itemize}

\paragraph{Re-roll Mechanic.}  
Position modifies the dice pool through simple re-rolls:
\begin{center}
\begin{longtable}{@{}lll@{}}
\toprule
\textbf{Position} & \textbf{Narrative Frame} & \textbf{Mechanical Effect} \\
\midrule
Dominant & You press your advantage & Re-roll one \emph{failure} \\
Controlled    & The balanced norm & No re-rolls \\
Desperate & You act under duress & Re-roll one \emph{success} \\
\bottomrule
\end{longtable}
\end{center}

\section{Asset Activation Mechanics}
\label{sec:asset-activations}
\index{Assets!activation}
Players can activate Assets in several ways:
\begin{description}
\item[\textbf{Free Off-Screen}] Each Asset's off-screen effect \emph{once per session} for free. \index{Assets!off-screen}
\item[\textbf{XP Activation}] Spend \textbf{2 XP} to trigger an extra off-screen effect beyond the session allowance. \index{Assets!XP activation}
\item[\textbf{Boon Activation}] Spend \textbf{1 Boon} to bring an Asset's influence on-screen now. \index{Assets!Boon activation}
\item[\textbf{Plausibility Test}] The Asset must have scope/reach appropriate to the effect. \index{Assets!scope}
\end{description}

\subsection{Initiative and Turn Order}

Fate's Edge does not use fixed initiative. 
Turn order follows the fiction and the GM's facilitation:
\begin{itemize}
    \item \textbf{Narrative Fiat:} The GM frames spotlight order based on circumstances, tension, and narrative flow.
    \item \textbf{Player Input:} Players may suggest acting when it makes sense in the fiction. 
    \item \textbf{Surprise:} Ambushers act first; targets respond after the opening exchange.
    \item \textbf{Flexibility:} Spotlight may shift mid-scene if fictionally appropriate (e.g., reacting to a falling ceiling, seizing a moment).
\end{itemize}

This ensures pacing and drama guide the sequence of actions, not rigid turn structures.
% =========================
% Turn Economy (Quick Rules)
% =========================
\subsection{Turn Economy (Quick Rules)}
\label{subsec:turn-economy-quick}

\paragraph{Two Actions.}
Each character takes \emph{1 Action and 1 Move} on their turn. Actions and Moves may be taken in any order; repeating the same Action is not allowed unless noted. A character may use a Boon to re-roll their action at the expense of their move if they still have it available. Some weapon tempos effect whether you can take an attack and a move.

\paragraph{Move.}
Traverse up to your normal movement. \emph{Disengage:} move without provoking; your next offensive action is \textbf{Controlled}. \emph{Dash:} move again this turn; your next defense is \textbf{Desperate}.

\paragraph{Attack.}
Make a melee or ranged attack versus DV set by the GM and fiction. Teamwork/Assist costs 1 Boon.

\paragraph{Observe / Change Position (+1).}
Take a beat to read the field or set angles; gain \textbf{+1 Position} for one action this turn (e.g., Controlled$\to$Dominant). Limit: once/turn; cannot exceed \textbf{Dominant}.

\paragraph{Activate an Asset.}
Use gear, symbol, tool, or feature per its text/tags (e.g., torch, grapnel, smoke vial, rune focus). Items with \texttt{[Action]} consume one Action; \texttt{[Free]} do not.

\paragraph{Setup (Teamwork).}
Create advantage for an ally; on success, grant their next action \textbf{+1 Position} or step up Effect (GM’s call).

\paragraph{Assist (Teamwork).}
Spend \emph{1 Boon} to give an ally \emph{+1 die} on their current roll; you share appropriate risk/consequence.

\paragraph{Defend / Protect.}
Adopt a guarding stance or body-block. Choose a nearby ally; until your next turn you may intercept one hit on them and roll to resist it. On success, reduce/negate Harm; you take any fallout the GM assigns.

\paragraph{Channel / Weave.}
Runekeeper/ritual flow: \emph{Channel} (prime power) then \emph{Weave} (shape/release). Disruption or engagement may worsen Position; if \emph{Interrupted}, the casting fails.

\paragraph{Cast Rite / Song (Cantor).}
Perform a Rite/Song per its write-up. You may \emph{Push} to accelerate or empower at the cost of Fatigue/Corruption per class rules.

\paragraph{Interact.}
Lift, pull, flip a lever, shove a foe, break an object, apply a poultice, reload, draw/stow, etc. GM sets DV/Effect.

\paragraph{Free Items.}
Short shouts, dropping an item, quick glance. Longer or tactical assessments require \emph{Observe / Change Position} or \emph{Interact}.

\paragraph{Reactions (Out of Turn).}
\emph{Protection} may trigger when an ally is hit and you are in position. Class/Asset reactions fire as written (e.g., counter-runes, ripostes).

\paragraph{Position Caps.}
Bonuses cannot raise Position above \textbf{Dominant}; penalties cannot drop below \textbf{Desperate}. Beyond these caps, adjust DV or Effect instead.

\section{Experience Point Economy}
\label{sec:xp-economy}
\index{Experience Points}

\subsection{Session Awards}
\label{subsec:session-awards}
\index{Experience Points!session awards}
\begin{description}
\item[\textbf{Table Attendance}] +2 XP
\item[\textbf{Major Objective}] +2–4 XP
\item[\textbf{Discovery}] +1–2 XP
\item[\textbf{Hard Choice}] +1–2 XP
\item[\textbf{Complication Spotlight}] +1–3 XP
\item[\textbf{Bond-Driven Play}] +1–2 XP
\item[\textbf{GM Curveball}] +0–3 XP
\end{description}

\subsection{Milestone Awards}
\label{subsec:milestone-awards}
\index{Experience Points!milestones}
At the end of a major arc:
\begin{itemize}
\item +8–12 XP to all players (arc completion)
\item +2 XP to one player for a signature moment
\end{itemize}

\subsection{Complication Dividend}
\label{subsec:complication-dividend}
\index{Experience Points!dividends}
\begin{itemize}
\item Resolve a Face-card complication: +1 XP
\item Resolve an Ace complication: +2 XP
\end{itemize}

\subsection{XP Spending Costs}
\label{subsec:xp-spending}
\index{Experience Points!spending}
\begin{description}
\item[\textbf{Attributes}] Cost = new rating $\times$ 3 XP; downtime = new rating (days). \index{Attributes!XP costs}
\item[\textbf{Skills}] Cost = new level $\times$ 2 XP; downtime = new level (days). \index{Skills!XP costs}
\item[\textbf{Followers (on-screen)}] Cost = Cap$^{2}$ XP; 1–3 days to recruit/brief. \index{Followers!XP costs}
\item[\textbf{Assets (off-screen)}] Minor (4 XP, 1 day), Standard (8 XP, 1 week), Major (12 XP, 1 month). \index{Assets!XP costs}
\end{description}

\section{Rush Rule for Advancement}
\label{sec:rush-rule}
\index{Advancement!rush rule}
A player may skip required downtime for an advance; the GM creates a \textbf{Haste} [4] clock. If it fills during the rushed period, the new ability or Asset arrives with flaws or narrative complications.

\section{Tiers of Reputation}
\label{sec:reputation-tiers}
\index{Reputation Tiers}

\begin{description}
\item[\textbf{Tier I — Rookie}] (0–40 XP): Local reputation; prestige abilities locked. \index{Reputation!Tier I}
\item[\textbf{Tier II — Seasoned}] (41–90 XP): Regional notice; prestige abilities may unlock. \index{Reputation!Tier II}
\item[\textbf{Tier III — Veteran}] (91–150 XP): National influence; second follower slot suggested. \index{Reputation!Tier III}
\item[\textbf{Tier IV — Paragon}] (151–220 XP): Movers and shakers; rivals emerge to challenge. \index{Reputation!Tier IV}
\item[\textbf{Tier V — Mythic}] (221+ XP): Legendary status; kingdoms and cults respond directly. \index{Reputation!Tier V}
\end{description}

\subsection{Recommended Session Order (GM Checklist)}

\paragraph{1) Off-Screen (Downtime, 10–20 min)}
\begin{itemize}
  \item Upkeep: choose Efficient/Intensive; apply Neglected/Compromised if missed.
  \item Obligation: clear via Acts of Service; note Claims/overflow risk.
  \item Projects: tick long-term clocks; resolve Gather Info; prep assets.
  \item Intent: each player states one on-screen goal; GM surfaces 1–2 front pressures.
\end{itemize}

\paragraph{2) On-Screen (Scenes)}
\begin{itemize}
  \item Frame hard: where/what’s at stake; set Position $\to$ DV.
  \item Run spotlight: rotate beats; fold in bonds and Boon sharing.
  \item Advance: move faction/Patron clocks openly when triggered.
\end{itemize}

\paragraph{3) Wrap-Up (5–10 min)}
\begin{itemize}
  \item XP \& Talents: award, mark progress; note any Gifts gained/forfeit.
  \item SB \& Harm: convert Fatigue$\to$Harm if full; apply recoveries.
  \item Fronts: advance unresolved clocks; note consequences.
\end{itemize}

\paragraph{4) Off-Screen Hooks (2–5 min)}
\begin{itemize}
  \item Log next Downtime intents, service opportunities, upkeep deadlines.
  \item Capture cliffhangers and Patron Largess seeds for next session open.
\end{itemize}

\emph{Optional:} Add a cold open flash-cut before Step 2 to spotlight a rival or Patron omen.

\subsection{Fatigue}
\label{subsec:fatigue}
\index{Fatigue}

\textbf{Track:} Each character has a Fatigue track equal to \textbf{Body}. Mark Fatigue for exertion, strain, or backlash.

\textbf{In Play:} Each Fatigue step worsens your \textbf{Position} by one level 
(Dominant $\rightarrow$ Controlled $\rightarrow$ Desperate). 
If you are already \textbf{Desperate}, instead apply a \textbf{--1 die} penalty per Fatigue to that roll.

\textbf{Overflow:} When your Fatigue track fills, immediately increase \textbf{Harm by 1 step} and clear all Fatigue to 0. 
If this raises Harm to a level that incapacitates you, you fall out of the scene as normal for Harm.

\textbf{Recovery:} Short rest clears 1--2 Fatigue; a full night's rest clears all Fatigue.


\subsection{Fear Effects Table}
\label{subsec:fear-table}

When a character escalates on the Fear Track (Shaken $\rightarrow$ Frightened $\rightarrow$ Panicked), roll on the following table or choose an appropriate effect. These results apply primarily to NPCs, though PCs may adopt them as narrative guidance.

\begin{center}
\begin{longtable}{>{\bfseries}r l l}
\toprule
d10 & Effect & Magic Tags \\
\midrule
1 & \textbf{Freeze}: Cannot act this round, staring or trembling. & Silence, Stasis \\
2 & \textbf{Flee}: Must move at full speed away from the source of Fear. & Movement, Wind \\
3 & \textbf{Drop}: Character drops what they are holding. & Disarm, Break \\
4 & \textbf{Beg}: Character pleads or bargains incoherently. & Compulsion, Voice \\
5 & \textbf{Hide}: Seeks cover, concealment, or allies to cling to. & Shadow, Illusion \\
6 & \textbf{Attack in Panic}: Lashes out wildly at the nearest target. & Rage, Fire \\
7 & \textbf{Blunder}: Stumbles into danger (trap, hazard, off balance). & Chaos, Trickery \\
8 & \textbf{Obey}: Instinctively follows a simple command from the fear-causer. & Command, Charm \\
9 & \textbf{Break Down}: Sobs, prays, or becomes useless until aided. & Curse, Despair \\
10 & \textbf{Catatonia}: Becomes unresponsive, requiring intervention. & Sleep, Dream \\
\bottomrule
\end{longtable}
\end{center}

\paragraph{Note.}  
At GM discretion, results may escalate with each step of the Fear Track:  
- \emph{Shaken}: Apply minor versions (hesitation, lost die, startled).  
- \emph{Frightened}: Roll normally.  
- \emph{Panicked}: Apply severe or exaggerated results (e.g., 2 = reckless flight, 6 = attack allies).  

\subsection{Maximum die pool}

An individual can have a max die pool of 10d10. All extra are converted to auto-successes. 

\section{Skills}
\label{sec:skills}

\subsection*{How Skills Work}
An action roll pairs an \textbf{Attribute} with a \textbf{Skill} to reflect what you do and how you do it (e.g., \emph{Wits + Subterfuge}, \emph{Body + Athletics}). The Keeper sets \textbf{Position} and \textbf{DV} (difficulty value) from the fiction; your hits determine \textbf{Effect}, with \textbf{SB} (setback) generated on low dice as usual.

\textbf{Fiction-first handles.} Obstacles should present at least two plausible “handles” (different Skills/approaches) so players can choose a method that fits their build and the scene. Assistance uses the helper’s Attribute+Skill; tools, tags, Strings, and Diamonds modify Position/DV/Effect as normal.

\subsection*{Core Skill List (A–Z)}
Each entry lists what the Skill covers and common Attribute pairings. These are examples, not limits.

\subsubsection*{Arcana}
\textbf{What:} Magical theory, sigils, wards, occult correspondences, ritual praxis.\\
\textbf{Pairs:} \emph{Wits} (analyze a sigil), \emph{Spirit} (sustain a rite), \emph{Presence} (lead a chorus).

\subsubsection*{Athletics}
\textbf{What:} Running, jumping, climbing, swimming, balance under strain.\\
\textbf{Pairs:} \emph{Body} (vault a gap), \emph{Wits} (time a leap), \emph{Spirit} (push through fatigue).

\subsubsection*{Brawl}
\textbf{What:} Unarmed strikes, grapples, improvised holds, close scrums.\\
\textbf{Pairs:} \emph{Body} (tackle), \emph{Wits} (feint), \emph{Spirit} (fight on while dazed).

\subsubsection*{Command}
\textbf{What:} Directing allies, drilling troops, battlefield orders, keeping cohesion.\\
\textbf{Pairs:} \emph{Presence} (rally), \emph{Wits} (issue smart orders), \emph{Spirit} (hold the line).

\subsubsection*{Craft}
\textbf{What:} Making and mending—smithing, carpentry, weaving, cooking, alchemy set-up.\\
\textbf{Pairs:} \emph{Wits} (plan), \emph{Body} (execute heavy work), \emph{Spirit} (long, careful work).

\subsubsection*{Deception}
\textbf{What:} Direct lies, misstatements, bluffing in conversation.\\
\textbf{Pairs:} \emph{Presence} (sell a lie), \emph{Wits} (keep stories straight), \emph{Spirit} (lie under pressure).

\subsubsection*{Diplomacy}
\textbf{What:} Formal negotiation, etiquette, treaties, court protocol, “Bowl before Board.”\\
\textbf{Pairs:} \emph{Presence} (host a parley), \emph{Wits} (read concessions), \emph{Spirit} (stay courteous under fire).

\subsubsection*{Endurance}
\textbf{What:} Marches, exposure, pain tolerance, poison, disease, holding breath.\\
\textbf{Pairs:} \emph{Spirit} (resist), \emph{Body} (carry load), \emph{Wits} (ration effort).

\subsubsection*{Insight}
\textbf{What:} Read emotions, motives, tells; spot a con at the \emph{person} level.\\
\textbf{Pairs:} \emph{Wits} (parse signals), \emph{Presence} (mirror, probe), \emph{Spirit} (keep your center).

\subsubsection*{Investigation}
\textbf{What:} Structured inquiry—interviews, paper trails, scene reconstruction.\\
\textbf{Pairs:} \emph{Wits} (deduce), \emph{Presence} (question), \emph{Body} (methodical canvass).

\subsubsection*{Lore}
\textbf{What:} History, cultures, laws, faiths, bestiaries, ancient sites.\\
\textbf{Pairs:} \emph{Wits} (recall), \emph{Presence} (cite), \emph{Spirit} (keep taboo rites correctly).

\subsubsection*{Medicine}
\textbf{What:} First aid, surgery, leechcraft, epidemics, long-term care.\\
\textbf{Pairs:} \emph{Wits} (diagnose), \emph{Body} (operate), \emph{Spirit} (steady hands under stress).

\subsubsection*{Melee}
\textbf{What:} Armed close combat—blades, axes, staves, shields.\\
\textbf{Pairs:} \emph{Body} (strike), \emph{Wits} (footwork), \emph{Spirit} (press the advantage).

\subsubsection*{Nature}
\textbf{What:} Wilds knowledge—tracks, foraging, animal signs, weather sense.\\
\textbf{Pairs:} \emph{Wits} (read terrain), \emph{Spirit} (respect dangers), \emph{Body} (set snares).

\subsubsection*{Notice}
\textbf{What:} Situational awareness—perceive, scan, spot ambushes and tells in \emph{places}.\\
\textbf{Pairs:} \emph{Wits} (observe), \emph{Body} (react), \emph{Spirit} (keep calm perceptions).

\subsubsection*{Performance}
\textbf{What:} Acting, music, dance, oratory, crowd-working.\\
\textbf{Pairs:} \emph{Presence} (captivate), \emph{Wits} (timing), \emph{Spirit} (stage nerve).

\subsubsection*{Ranged}
\textbf{What:} Bows, crossbows, thrown weapons, firearms (by setting).\\
\textbf{Pairs:} \emph{Body} (shoot), \emph{Wits} (lead), \emph{Spirit} (hold the shot).

\subsubsection*{Stealth}
\textbf{What:} Move unseen, silence, shadowing, hide-and-evade.\\
\textbf{Pairs:} \emph{Body} (sneak), \emph{Wits} (choose routes), \emph{Spirit} (stay still under pressure).

\subsubsection*{Streetwise}
\textbf{What:} Underworld culture—contacts, fences, black markets, rumor webs.\\
\textbf{Pairs:} \emph{Presence} (work a contact), \emph{Wits} (vet info), \emph{Spirit} (walk bad streets).

\subsubsection*{Subterfuge}
\textbf{What:} Criminal craft and social deception: casing, impersonation, forgery, palming/planting, short cons, engineered distractions. Subterfuge tricks \emph{people and systems}, not mechanisms.\\
\textbf{Pairs:} \emph{Wits} (case routines), \emph{Presence} (talk past checkpoints), \emph{Body} (sleight of hand), \emph{Spirit} (sustain a cover).

\subsubsection*{Tactics}
\textbf{What:} Small-unit plans, flanking, formations, reading the field, pursuit/evasion.\\
\textbf{Pairs:} \emph{Wits} (plan), \emph{Presence} (coordinate), \emph{Spirit} (execute under fire).

\subsubsection*{Tinker}
\textbf{What:} Mechanisms—locks, traps, engines, devices, jury-rigs, sabotage.\\
\textbf{Pairs:} \emph{Wits} (diagnose), \emph{Body} (delicate work), \emph{Spirit} (keep steady during failure modes).

\bigskip
\noindent\textbf{Locks \& Wards (clarity note).} Bypass \emph{mechanical} locks/traps with \textbf{Tinker + Attribute}. Bypass \emph{arcane} seals/wards with \textbf{Arcana/Lore + Attribute}. \textbf{Subterfuge} gets you \emph{to} the door and past the people, not \emph{through} the mechanism.

\subsection*{Optional \& Mode Skills}
Tables may enable additional Skills by mode:
\begin{itemize}[leftmargin=*]
  \item \textbf{Psionics} (Psionics module): psychic arts, mental strain, disciplines.
  \item \textbf{Technology} (Modern Noir): digital systems, intrusion software, electronics.
  \item \textbf{Perception/Insight merge}: Some tables collapse \emph{Notice} and \emph{Insight} into one \emph{Perception}; if so, keep the above niches visible in examples.
\end{itemize}

\subsection*{Adding a New Skill (Guidance)}
Define the gap (one line on what it does that others don’t), list 3–5 common Attribute pairings, and provide 6–8 typical actions. Do \emph{not} delete existing handles from procedures—add your Skill where the fiction justifies it, keeping niches crisp.

\section*{Core DV Philosophy}

Difficulty Values (DV) in Fate's Edge represent \textbf{narrative weight}, not simulationist challenge. The DV system should answer: "How much does this matter to the story right now?"

\section{The Standard DV Ladder}

\begin{center}
\begin{longtable}{clp{3.5in}}
\toprule
\textbf{DV} & \textbf{Category} & \textbf{When to Use} \\
\midrule
2 & Routine & Clear intent, modest stakes, controlled environment \\
3 & Easy & Minor challenge, familiar task, slight pressure \\
4 & Moderate & Notable challenge, active opposition, time limits \\
5 & Hard & Significant challenge, hostile conditions, precision required \\
6 & Very Hard & Exceptional challenge, multiple constraints, high drama \\
7+ & Extreme & Mythic challenge, campaign-defining, near-impossible \\
\bottomrule
\end{longtable}
\end{center}

\section{DV Setting by Narrative Context}

\subsection{Character Capability Baseline}

Start with the character's Tier and adjust based on the specific challenge:

\begin{center}
\begin{longtable}{cll}
\toprule
\textbf{Tier} & \textbf{Baseline DV} & \textbf{Example Character} \\
\midrule
I & 3-4 & Rookie, local threat \\
II & 4-5 & Seasoned, regional threat \\
III & 5-6 & Veteran, national threat \\
IV & 6-7 & Paragon, legendary threat \\
V & 7-8 & Mythic, world-changing threat \\
\bottomrule
\end{longtable}
\end{center}

\subsection{Position Modifiers}

\begin{center}
\begin{longtable}{cl}
\toprule
\textbf{Position} & \textbf{DV Modifier} \\
\midrule
Dominant & -1 \\
Controlled & +0 \\
Desperate & +1 \\
\bottomrule
\end{longtable}
\end{center}

\section{Contextual DV Modifiers}

\subsection{Environmental Factors}

\begin{itemize}
\item \textbf{Favorable Conditions:} -1 DV (good lighting, stable ground, clear weather)
\item \textbf{Neutral Conditions:} +0 DV (typical environment)
\item \textbf{Challenging Conditions:} +1 DV (dim light, uneven ground, light wind)
\item \textbf{Hostile Conditions:} +2 DV (darkness, slippery surfaces, heavy rain)
\item \textbf{Extreme Conditions:} +3 DV (blizzard, earthquake, magical storm)
\end{itemize}

\subsection{Time Pressure}

\begin{itemize}
\item \textbf{No Time Pressure:} -1 DV (deliberate, careful approach)
\item \textbf{Standard Timing:} +0 DV (normal pace)
\item \textbf{Moderate Pressure:} +1 DV (limited time, but manageable)
\item \textbf{Severe Pressure:} +2 DV (countdown, immediate consequences)
\item \textbf{Critical Timing:} +3 DV (split-second timing, life-or-death)
\end{itemize}

\subsection{Character Condition}

\begin{itemize}
\item \textbf{Well-rested, Focused:} -1 DV (clear mind, full attention)
\item \textbf{Normal Condition:} +0 DV (typical state)
\item \textbf{Fatigued (1-2):} +1 DV (minor exhaustion, distraction)
\item \textbf{Fatigued (3-4):} +2 DV (significant strain, impaired focus)
\item \textbf{Harm 1-2:} +1-2 DV (injury effects, pain penalties)
\item \textbf{Harm 3+:} +3 DV (severe injury, near incapacity)
\end{itemize}

\section{Skill and Attribute Considerations}

\subsection{Skill Mastery Modifiers}

\begin{itemize}
\item \textbf{Skill 0:} +2 DV (untrained attempt)
\item \textbf{Skill 1-2:} +0 DV (basic competence)
\item \textbf{Skill 3-4:} -1 DV (skilled practitioner)
\item \textbf{Skill 5+:} -2 DV (mastery level)
\end{itemize}

\subsection{Attribute Relevance}

When the primary Attribute is exceptionally high or low:

\begin{itemize}
\item \textbf{Attribute 5:} -1 DV (exceptional natural talent)
\item \textbf{Attribute 1:} +2 DV (significant natural limitation)
\end{itemize}

\section{Group Actions and Assistance}

\subsection{Assistance Modifiers}

\begin{itemize}
\item \textbf{One Competent Helper:} -1 DV (relevant expertise)
\item \textbf{Two Helpers:} -1 DV (combined assistance, diminishing returns)
\item \textbf{Three+ Helpers:} -1 DV (maximum assistance benefit)
\item \textbf{Unhelpful Environment:} +1-2 DV (crowded, chaotic, obstructive)
\end{itemize}

\subsection{Group vs. Individual Challenges}

\begin{itemize}
\item \textbf{Individual Task:} Standard DV
\item \textbf{Group Coordination Required:} +1-2 DV (communication complexity)
\item \textbf{Massive Scale:} +2-3 DV (beyond individual scope)
\item \textbf{Specialized Roles Needed:} +1 DV per missing expertise
\end{itemize}

\section{Equipment and Tools}

\subsection{Tool Quality Modifiers}

\begin{itemize}
\item \textbf{Superior Tools:} -1 DV (specialized, well-maintained)
\item \textbf{Adequate Tools:} +0 DV (standard equipment)
\item \textbf{Poor Tools:} +1 DV (worn, improvised, inadequate)
\item \textbf{Wrong Tools:} +2-3 DV (completely inappropriate)
\item \textbf{Magical/Advanced Tools:} -1 to -2 DV (depending on power)
\end{itemize}

\subsection{Tool Condition}

\begin{itemize}
\item \textbf{Maintained:} +0 DV
\item \textbf{Neglected:} +1 DV
\item \textbf{Compromised:} +2 DV
\item \textbf{Broken:} Task impossible without repair
\end{itemize}

\section{Opposition and Resistance}

\subsection{Opposition Level}

\begin{itemize}
\item \textbf{No Active Opposition:} -1 DV (unopposed action)
\item \textbf{Passive Resistance:} +0 DV (natural resistance, no active counter)
\item \textbf{Active Opposition:} +1-2 DV (opponent actively countering)
\item \textbf{Skilled Opposition:} +2-3 DV (opponent with relevant expertise)
\item \textbf{Superior Opposition:} +3-4 DV (opponent significantly more capable)
\end{itemize}

\section{Scenario-Specific DV Guidelines}

\subsection{Combat DV Modifiers}

\begin{itemize}
\item \textbf{Target Size:} -1 to +2 DV (tiny to huge)
\item \textbf{Cover:} +1-2 DV (partial to full cover)
\item \textbf{Range:} +0 to +2 DV (Close to Far)
\item \textbf{Mobility:} +1-2 DV (moving target)
\item \textbf{Illumination:} +1-2 DV (dim to darkness)
\end{itemize}

\subsection{Social DV Modifiers}

\begin{itemize}
\item \textbf{Relationship:} -2 to +2 DV (close ally to bitter enemy)
\item \textbf{Social Distance:} +0 to +2 DV (intimate to formal/professional)
\item \textbf{Cultural Familiarity:} -1 to +2 DV (native customs to foreign protocols)
\item \textbf{Stakes Clarity:} -1 to +2 DV (clear, mutual benefit to ambiguous/harmful)
\item \textbf{Time Pressure:} +0 to +2 DV (leisurely discussion to immediate deadline)
\end{itemize}

\section{Calculating Final DV}

To determine the final DV for any action:

\begin{enumerate}
\item Start with the \textbf{Base DV} from the Standard Ladder (2-7+)
\item Add the character's \textbf{Tier Modifier}: DV = Base DV + (Character Tier - 1)
\item Apply relevant \textbf{Contextual Modifiers} from previous sections
\item Consider \textbf{Position Effects}: Dominant (-1), Controlled (±0), Desperate (+1)
\item Adjust for \textbf{Environmental and Circumstantial Factors}
\end{enumerate}

\textbf{Minimum DV:} No roll can have a DV lower than 2. If modifiers would reduce it further, treat the action as automatic success with narrative description of the easy victory.

\textbf{Maximum DV:} For extremely challenging tasks, DV may exceed 7. Consider using clocks or extended challenges for DV 8+ tasks rather than single rolls.

\section{Special DV Considerations}

\subsection{Group Actions}

When multiple characters act together on a single goal:

\begin{itemize}
\item One character leads the action (sets main DV)
\item Helpers provide assistance (typically -1 DV or +1 Effect)
\item Each helper accepts shared risk from complications
\item Complex coordination may increase DV by +1
\end{itemize}

\subsection{Extended Challenges}

For tasks requiring multiple successes over time:

\begin{itemize}
\item Set a \textbf{Challenge Clock} (4-8 segments)
\item Each successful roll advances the clock
\item Complications may tick the clock backward
\item Partial successes may advance clock slowly
\end{itemize}

\subsection{Contested Actions}

When two parties oppose each other directly:

\begin{itemize}
\item Both parties roll against the same DV
\item Higher successes win the contest
\item Tie results favor the defender or status quo
\item Story Beats generated by either side may be spent by the GM
\end{itemize}

\section{DV Quick Reference}

For rapid gameplay, use these guidelines:

\begin{center}
\begin{longtable}{cl}
\toprule
\textbf{Situation} & \textbf{Quick DV} \\
\midrule
Clear, no pressure & 2 \\
Standard challenge & 3 \\
Notable opposition & 4 \\
Serious danger & 5 \\
Extreme circumstances & 6 \\
Mythic challenge & 7+ \\
\bottomrule
\end{longtable}
\end{center}

Remember: DV represents \textbf{narrative weight}, not simulationist difficulty. Adjust based on story importance, not just mechanical challenge.
