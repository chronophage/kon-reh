\chapter{Combat and Conflict}
\label{chap:combat-conflict}

\section{Core Philosophy}
\label{sec:combat-philosophy}
\index{Combat!Core Philosophy}

Combat in \textbf{Fate's Edge} uses the same consequence-forward mechanics as all other challenges. Every combat action can produce triumph and complication, and outcomes cascade through Story Beats (SB), clocks, and position changes. The focus is on narrative positioning, tactical choices, and meaningful risk—not granular simulation.

\section{Combat Structure}
\label{sec:combat-structure}
\index{Combat!Structure}

\subsection{Rounds and Turns}
\label{subsec:rounds-turns}
\index{Combat!Rounds}\index{Combat!Turns}

\begin{itemize}
  \item \textbf{Rounds:} A few seconds of simultaneous action.
  \item \textbf{Turns:} Each participant takes one significant action per round.
  \item \textbf{Scenes:} A battle is usually one scene unless the fiction splits it.
  \item \textbf{Initiative:} Act in the order that makes sense fictionally; the GM adjudicates flow based on Position and established threats. \index{Combat!Initiative}
\end{itemize}

\subsection{Action Declaration}
\label{subsec:action-declaration}
\index{Combat!Action Declaration}

\begin{enumerate}
  \item \textbf{Approach:} Describe intent and method.
  \item \textbf{Position:} GM sets \textit{Controlled}, \textit{Risky}, or \textit{Desperate}. \index{Position}
  \item \textbf{Pool:} Build dice (Attribute + Skill + situational/modifiers).
  \item \textbf{Roll \& Resolve:} Use DV and the Outcome Matrix; any \textbf{1}s generate SB. \index{Story Beats (SB)}\index{Difficulty Value (DV)}
\end{enumerate}

\section{Position States}
\label{sec:position-states}
\index{Combat!Position States}

\begin{description}
  \item[\indexterm{Controlled}] Advantage: cover, flank, surprise, elevation. Failure leaves options; consequences are minor. \index{Combat!Controlled}
  \item[\indexterm{Risky}] Even footing; typical exchange of blows. Failure has teeth; moderate consequences. \index{Combat!Risky}
  \item[\indexterm{Desperate}] Bad footing, outnumbered, wounded. Failure is severe; success might unlock extra narrative rewards. \index{Combat!Desperate}
\end{description}

\section{Range Bands and Movement}
\label{sec:range-bands}
\index{Combat!Range Bands}\index{Combat!Movement}

\subsection{Range Band Definitions}
\label{subsec:range-definitions}
\index{Combat!Range Bands!Definitions}

\begin{description}
  \item[\indexterm{Close}] Arm’s length; grappling; only melee is practical. \index{Combat!Close Range}
  \item[\indexterm{Near}] Same room/zone; most actions occur here. \index{Combat!Near Range}
  \item[\indexterm{Far}] Same site but distant; requires movement to engage. \index{Combat!Far Range}
  \item[\indexterm{Absent}] Off-screen/another area; requires significant effort/time. \index{Combat!Absent Range}
\end{description}

\subsection{Movement Rules}
\label{subsec:movement-rules}
\index{Combat!Movement Rules}

\begin{itemize}
  \item \textbf{1 Move:} Shift one band (Close$\leftrightarrow$Near or Near$\leftrightarrow$Far).
  \item \textbf{Dash (action):} Shift two bands (Close$\rightarrow$Far or Far$\rightarrow$Close).
  \item \textbf{Engage:} Entering Close from Near usually costs a Move.
  \item \textbf{Disengage:} Leaving Close may require a test if threatened.
\end{itemize}

\section{Combat Actions}
\label{sec:combat-actions}
\index{Combat!Actions}

\subsection{Standard Actions}
\label{subsec:standard-actions}
\index{Combat!Standard Actions}

\begin{description}
  \item[\indexterm{Attack}] Strike with appropriate Skill (e.g., Combat/Melee/Ranged by your list). \index{Combat!Attack}
  \item[\indexterm{Defend}] Active defense against incoming harm (parry, block, roll aside). \index{Combat!Defend}
  \item[\indexterm{Maneuver}] Change Position, create advantage, or set up an ally. \index{Combat!Maneuver}
  \item[\indexterm{Use Object}] Interact with doors, levers, lanterns, terrain, or gear. \index{Combat!Use Object}
  \item[\indexterm{Cast Spell}] Perform magical actions (see Chapter~\ref{chap:magic-system}). \index{Combat!Cast Spell}
\end{description}

\subsection{Special Actions}
\label{subsec:special-actions}
\index{Combat!Special Actions}

\begin{description}
  \item[\indexterm{Aid}] Provide assistance to another’s action (costs as per Assist rules). \index{Combat!Aid}
  \item[\indexterm{Ready}] Prepare an action with a clear trigger. \index{Combat!Ready}
  \item[\indexterm{Withdraw}] Attempt to disengage safely. \index{Combat!Withdraw}
  \item[\indexterm{Sprint}] Spend your action to Dash (two-band shift). \index{Combat!Sprint}
\end{description}

% =========================
% HEALTH, FATIGUE, & HARM (REVISED)
% =========================
\section{Health, Fatigue, \& Harm} 
\label{sec:health-fatigue-harm-rev}

\subsection*{Tracks \& Caps}
\begin{itemize}
  \item \textbf{Fatigue Track}: boxes equal to \textbf{Body}.
  \item \textbf{Harm Levels}: as defined elsewhere in the SRD (\textbf{Harm 1}, \textbf{Harm 2}, \textbf{Harm 3}).
\end{itemize}

\subsection*{Fatigue $\rightarrow$ Harm Conversion}
Whenever you would mark Fatigue and your Fatigue Track \emph{fills} (all boxes marked):
\begin{enumerate}
  \item \textbf{Increase} your \textbf{Harm} by one level (e.g., 0$\rightarrow$Harm~1, Harm~1$\rightarrow$Harm~2, Harm~2$\rightarrow$Harm~3).
  \item \textbf{Clear all Fatigue} (erase the Fatigue Track back to 0).
\end{enumerate}
This conversion can occur multiple times in a scene. Effects of Harm tier (disadvantage, action limits, incapacitation at Harm~3, etc.) follow your existing SRD.

\subsection*{Taking Fatigue}
Mark Fatigue for strain, exertion, travel, magic costs, or \S\ref{sec:obligation-overflow-rev} overflow. Fatigue can exceed remaining boxes only to \emph{trigger} conversion; any excess is ignored after the Harm increase and Fatigue clear.

\subsection*{Recovering Fatigue}
\begin{itemize}
  \item \textbf{Short Rest} (quiet watch, food/water): remove \textbf{2 Fatigue}.
  \item \textbf{Full Night}: remove \textbf{all Fatigue}.
\end{itemize}
\emph{Fatigue recovery does not remove Harm.} Recover Harm via your normal medical/ritual rules in the SRD.

\subsection*{Mitigation (Optional Dials)}
\begin{itemize}
  \item \textbf{Soak/Ward}: Before marking Fatigue, reduce it by 1--2 (to a minimum of 0) if protected by armor/boons/rites.
  \item \textbf{Convert}: Some effects may convert incoming \textbf{Harm 1} to \textbf{2 Fatigue}; if this \emph{fills} the track, convert as normal.
\end{itemize}

\subsection{Harm and Consequences}
\label{sec:harm-consequences}
\index{Combat!Harm}\index{Combat!Consequences}

\subsubsection*{Harm Levels}
\label{subsec:harm-levels}
\index{Combat!Harm Levels}

\begin{description}
  \item[\indexterm{Minor}] $-1$ die to related actions; GM may flag narrative nuisance. \index{Combat!Minor Harm}
  \item[\indexterm{Moderate}] $-1$ die to most actions; obvious impairment. \index{Combat!Moderate Harm}
  \item[\indexterm{Severe}] $-2$ dice to most actions; immediate danger, may force tests to act. \index{Combat!Severe Harm}
  \item[\indexterm{Critical}] Incapacitated/dying; requires rescue or intervention. \index{Combat!Critical Harm}
\end{description}

\paragraph{Story Beats from Harm}\index{Story Beats (SB)!From Harm}
At the GM’s discretion, fresh harm can immediately grant the GM 1--2 SB to reflect chaos, panic, or collateral danger in the scene.

\subsubsection*{Resisting Harm}
\label{subsec:resisting-harm}
\index{Combat!Resisting Harm}

Attempt to blunt or avoid harm with a relevant Attribute test (typical DV~3):
\begin{itemize}
  \item \textbf{Success:} Reduce harm by one level.
  \item \textbf{Partial:} Reduce or transform the consequence (GM offers options).
  \item \textbf{Miss:} Full harm applies.
  \item Any \textbf{1}s rolled still generate SB. \index{Story Beats (SB)}
\end{itemize}

\section{Teamwork in Combat}
\label{sec:teamwork}
\index{Combat!Teamwork}

\subsection{Assistance}
\label{subsec:assistance}
\index{Combat!Assistance}\index{Assists!Maximum}

\begin{itemize}
  \item \textbf{Cost:} 1 Boon (or a defined stress-like resource if used in your table). \index{Boons}
  \item \textbf{Effect:} +1 die to the assisted roll.
  \item \textbf{Limit:} Total assist dice from all sources are capped at +3 (unless a Talent says otherwise).
  \item \textbf{Exception:} \emph{Exceptional Coordination} allows one follower to grant +4 by itself.
\end{itemize}

\subsection{Setup Actions}
\label{subsec:setup-actions}
\index{Combat!Setup Actions}

\begin{itemize}
  \item Create cover, draw fire, threaten flanks, or reposition foes.
  \item On success, grant +1 Position \emph{or} step up Effect for the next allied action.
  \item Must be fictionally justified by space, timing, and method.
\end{itemize}

\subsection{Protection}
\label{subsec:protection}
\index{Combat!Protection}

\begin{itemize}
  \item Interpose to take harm intended for an ally.
  \item You must be in a plausible Position to intervene.
  \item Resolve as a defense or resist, per fiction.
\end{itemize}

\section{Tactical Clocks}
\label{sec:tactical-clocks}
\index{Combat!Tactical Clocks}

Use clocks to track persistent pressures and battlefield states.

\subsection{Common Combat Clocks}
\label{subsec:combat-clocks}
\index{Combat!Clocks}

\begin{description}
  \item[\indexterm{Mob Overwhelm [6]}] Numbers begin to swamp the PCs. \index{Combat!Mob Overwhelm}
  \item[\indexterm{Fatigue Spiral [4]}] Exhaustion degrades performance. \index{Combat!Fatigue Spiral}
  \item[\indexterm{Morale Collapse [6]}] A side is on the brink of routing. \index{Combat!Morale Collapse}
  \item[\indexterm{Environmental Collapse [8]}] Fire, flood, or structure failure escalates. \index{Combat!Environmental Collapse}
  \item[\indexterm{Reinforcement Arrival [4]}] Additional foes or allies appear. \index{Combat!Reinforcements}
\end{description}

\section{Position Dynamics}
\label{sec:position-dynamics}
\index{Combat!Position Dynamics}

\subsection{GM-Initiated Shifts}
\label{subsec:gm-shifts}
\index{Combat!Position Shifts}

\begin{itemize}
  \item \textbf{Spend 1 SB:} Worsen a character’s Position by one step. \index{Story Beats (SB)}
  \item \textbf{Narrative Events:} Reinforcements, collapsing cover, \emph{Dolmis} gale, etc.
  \item \textbf{Environment:} Weather, lighting, footing, smoke, crowding.
\end{itemize}

\subsection{Player-Initiated Shifts}
\label{subsec:player-shifts}
\index{Combat!Player Shifts}

\begin{itemize}
  \item \textbf{Spend 1 Boon:} Improve Position by one step for the current action. \index{Boons}
  \item \textbf{Maneuvers:} Flank, gain elevation, break a shield wall.
  \item \textbf{Assets:} Trigger tools, terrain features, or followers to alter Position.
\end{itemize}

\section{Magic in Combat}
\label{sec:magic-combat}
\index{Combat!Magic Integration}

\subsection{Casting Actions}
\label{subsec:casting-actions}
\index{Magic!Combat Casting}

\begin{description}
  \item[\indexterm{Standard Casting}] \textit{Channel} then \textit{Weave}: 1 action each, in order. \index{Magic!Standard Casting}
  \item[\indexterm{Rushed Casting}] Combine phases at \textit{Risky} with harsher consequences. \index{Magic!Rushed Casting}
  \item[\indexterm{Rites Invocation}] 1 action; may \textit{Push} for +1 Obligation. \index{Magic!Rites in Combat}\index{Obligation}
  \item[\indexterm{Invoker Rituals}] Usually too slow; \textit{Crack the Seal} for instant effect at cost. \index{Magic!Invoker Combat}
\end{description}

\subsection{Combat Spell Effects}
\label{subsec:spell-effects}
\index{Magic!Combat Effects}

\begin{itemize}
  \item Shift Position for multiple combatants.
  \item Create or advance tactical clocks.
  \item Spawn hazards (smoke, grease, quake) or advantages (light, ward, barrier).
  \item Grant offensive/defensive edges, with Backlash risks. \index{Backlash}
\end{itemize}

\section{Social Conflict}
\label{sec:social-conflict}
\index{Social Conflict}

\subsection{Social Skills}
\label{subsec:social-skills}
\index{Social Conflict!Skills}

\begin{description}
  \item[\indexterm{Sway}] Persuasion, negotiation, formal discourse. \index{Skills!Sway}
  \item[\indexterm{Deception}] Lies, misdirection, manipulation. \index{Skills!Deception}
  \item[\indexterm{Performance}] Oratory, entertainment, emotional appeal. \index{Skills!Performance}
  \item[\indexterm{Insight}] Reading people, spotting tells, motives. \index{Skills!Insight}
  \item[\indexterm{Command}] Leadership, intimidation, asserting authority. \index{Skills!Command}
\end{description}

\subsection{Social Position}
\label{subsec:social-position}
\index{Social Conflict!Position}

\begin{description}
  \item[\indexterm{Controlled}] You hold leverage, information, or status. \index{Social Conflict!Controlled}
  \item[\indexterm{Risky}] Even footing; standard negotiation. \index{Social Conflict!Risky}
  \item[\indexterm{Desperate}] You lack leverage; they hold the cards. \index{Social Conflict!Desperate}
\end{description}

\subsection{Social Consequences}
\label{subsec:social-consequences}
\index{Social Conflict!Consequences}

SB often manifest as:
\begin{itemize}
  \item Rumors, scandal, or damaged reputation. \index{Social Conflict!Reputation}
  \item Allies turning wary or distant. \index{Social Conflict!Allies}
  \item Concessions owed: favors, payments, or oaths. \index{Social Conflict!Obligations}
  \item Lost access, revoked privilege, or closed doors. \index{Social Conflict!Standing}
  \item Strained or broken relationships. \index{Social Conflict!Relationships}
\end{itemize}

\section{Mass Combat}
\label{sec:mass-combat}
\index{Combat!Mass Combat}

Treat armies as high-Cap followers with domain tags and clocks.

\subsection{Army as Followers}
\label{subsec:army-followers}
\index{Combat!Mass Combat!Armies}

\begin{itemize}
  \item \textbf{Cost:} Cap$^2$ XP to raise and maintain. \index{Followers!Cost}
  \item \textbf{Types:} Infantry, cavalry, archers, engineers, fleets.
  \item \textbf{Capabilities:} Provide large assist dice to war-scale actions (still subject to caps unless a rule overrides). \index{Assists!Maximum}
  \item \textbf{Risks:} Supply, morale, command/control, terrain.
\end{itemize}

\subsection{War Clocks}
\label{subsec:war-clocks}
\index{Combat!Mass Combat!War Clocks}

\begin{description}
  \item[\indexterm{Supply Lines [8]}] Logistics and provisioning. \index{Combat!Supply Lines}
  \item[\indexterm{Army Morale [6]}] Cohesion and willingness to fight. \index{Combat!Army Morale}
  \item[\indexterm{Strategic Position [8]}] Control of passes, ports, bridges. \index{Combat!Strategic Position}
  \item[\indexterm{Alliance Stability [6]}] Political support and coalition strain. \index{Combat!Alliance Stability}
\end{description}

\section{Environmental Combat}
\label{sec:environmental-combat}
\index{Combat!Environmental}

\subsection{Environmental Hazards}
\label{subsec:environmental-hazards}
\index{Combat!Environmental!Hazards}

\begin{description}
  \item[\indexterm{Fire [6]}] Spreading flames limit movement and cause harm. \index{Combat!Fire Hazard}
  \item[\indexterm{Flood [8]}] Rising water creates difficult terrain and drowning risk. \index{Combat!Flood Hazard}
  \item[\indexterm{Collapse [6]}] Structural failure; falling debris, blocked routes. \index{Combat!Collapse Hazard}
  \item[\indexterm{Weather [4]}] Storms, fog, glare reduce visibility/accuracy. \index{Combat!Weather Hazard}
\end{description}

\subsection{Terrain Effects}
\label{subsec:terrain-effects}
\index{Combat!Environmental!Terrain}

\begin{itemize}
  \item \textbf{Choke Points:} Favor defenders, constrain numbers.
  \item \textbf{Elevation:} Bonuses to ranged/oversight, harder to assault.
  \item \textbf{Cover:} Improves Position and reduces consequence severity.
  \item \textbf{Difficult Terrain:} Consumes movement; may worsen Position.
\end{itemize}

\section{Quick Reference}
\label{sec:combat-quick-ref}
\index{Combat!Quick Reference}

\subsection{Position Effects}
\label{subsec:position-quick-ref}
\index{Combat!Quick Reference!Position}

\begin{center}
\begin{tabular}{lll}
\toprule
\textbf{Position} & \textbf{Typical Edge} & \textbf{Consequence Severity} \\
\midrule
Controlled & Better options, easier withdraw & Minor \\
Risky      & Standard options               & Moderate \\
Desperate  & High reward potential           & Severe \\
\bottomrule
\end{tabular}
\end{center}

\subsection{Harm Quick Reference}
\label{subsec:harm-quick-ref}
\index{Combat!Quick Reference!Harm}

\begin{center}
\begin{tabular}{llll}
\toprule
\textbf{Harm} & \textbf{Penalty} & \textbf{Typical SB Grant} & \textbf{Recovery} \\
\midrule
Minor   & $-1$ die (related) & 0--1 (GM option) & Rest/basic care \\
Moderate& $-1$ die (most)    & 0--1 (GM option) & Treatment \\
Severe  & $-2$ dice (most)   & 1--2 (GM option) & Extended care \\
Critical& Incapacitated      & 2+ (GM option)   & Major intervention \\
\bottomrule
\end{tabular}
\end{center}

\subsection{Common Action DVs}
\label{subsec:action-dvs}
\index{Combat!Quick Reference!DVs}

\begin{center}
\begin{tabular}{lll}
\toprule
\textbf{Action Type} & \textbf{Typical DV} & \textbf{Notes} \\
\midrule
Basic Attack    & 2   & Standard melee/ranged in even footing \\
Maneuver        & 2--3& Create advantage, change Position \\
Active Defense  & 3   & Parry, block, evade under pressure \\
Complex Action  & 4   & Big swing, multi-target, field control \\
High-Risk       & 5+  & Desperate gambit, extreme precision \\
\bottomrule
\end{tabular}
\end{center}

\section{Combat Examples}
\label{sec:combat-examples}
\index{Combat!Examples}

\subsection{Melee Combat Example}
\label{subsec:melee-example}
\index{Combat!Examples!Melee}

Kael strikes a cultist with an imbued blade (\textit{Risky}, DV~2):
\begin{itemize}
  \item Roll 5d10 $\rightarrow$ 9, 7, 5, 2, 1 $\Rightarrow$ 3 successes, 1~SB.
  \item \textbf{Success with cost:} The cultist falls.
  \item GM spends 1~SB: \emph{Blood spatters the ritual circle; the summoning clock advances 1.}
\end{itemize}

\subsection{Ranged Combat Example}
\label{subsec:ranged-example}
\index{Combat!Examples!Ranged}

Lyra shoots at a distant archer (\textit{Desperate}, DV~3):
\begin{itemize}
  \item Roll 4d10 $\rightarrow$ 10, 6, 3, 1 $\Rightarrow$ 2 successes, 1~SB.
  \item \textbf{Partial:} The archer is hit but dives for cover.
  \item GM offers choice: shift Lyra to \textit{Risky} (stay exposed) \emph{or} take Minor harm from return fire.
\end{itemize}

\subsection{Magic Combat Example}
\label{subsec:magic-example}
\index{Combat!Examples!Magic}

Theron raises a defensive ward (\textit{Controlled}, DV~3):
\begin{itemize}
  \item \textbf{Channel:} 2 successes, 0~SB.
  \item \textbf{Weave:} 3 successes, 2~SB.
  \item \textbf{Success with cost:} Ward holds; GM spends 2~SB to start \emph{Ward Strain [4]}.
\end{itemize}

\section{Tags \& States}
\label{sec:tags-states-ref}

\begin{description}
  \item[\textbf{[HEALED]}] Remove \textbf{all Fatigue}. (Does not remove Harm.)
  \item[\textbf{[RALLIED]}] Remove \textbf{2 Fatigue}; gain +1 die on your next action this scene.
  \item[\textbf{[FORTIFIED]}] Until scene end, first incoming \textbf{Harm 1} becomes \textbf{2 Fatigue} instead.
  \item[\textbf{[STABILIZED]}] As per SRD: end bleed/burn/poison; does not remove Harm.
  \item[\textbf{[MENDED]}] As per SRD: remove Harm (follows your existing recovery procedure).
  \item[\textbf{[REVIVED]}] As per SRD: stand a fallen ally; follow your Harm 3/KO rules.
\end{description}
