% =========================
% Fate's Edge — Magic System (Main)
% =========================
\chapter{Magic System}
\label{chap:magic-system}

\section{Design Philosophy}
\label{sec:magic-philosophy}

Magic in \textbf{Fate's Edge} is a dangerous negotiation with the fabric of reality. It is powerful and flexible—yet every attempt to shape it carries risk. Each die showing \textbf{1} generates \textbf{Story Beats (SB)}, which are prompts for twists and complications. The fiction leads; math follows.

\section{The Four Paths of Magic}
\label{sec:four-paths}

\subsection{Casting (Freeform Magic)}
\label{subsec:freeform-casting}

Freeform casting represents raw, improvisational magic.

\begin{description}
\item[Requirement] \textbf{Caster's Gift} Talent (2 XP).
\item[Mechanics] Two-step \emph{Weave \& Cast} colored by the Eight Elements; fiction-first targets and scope.
\item[Risk] Each 1 generates SB; DV scales with scope; backlash is expressed by Element (or its opposite).
\item[Limits] Heavy control effects (e.g., \texttt{[WARD]}, \texttt{[BANISH]}, \texttt{[UNWARD]}) require a printed source (Talent, \emph{Rite} text, or Spell result).
\end{description}

\subsection*{GM Guidance}\label{subsec:backlash-gm}
\begin{itemize}
\item \textbf{Show, then Nudge.} Lead with the fiction (smoke, fissures, whispers), then apply the smallest mechanical nudge that preserves drama.\index{Design Philosophy}
\item \textbf{Escalate with Consent.} Offer players the choice to escalate Minor to Major by taking +1 (SB) now to seize something they want.\index{Story Beats!player choice}
\item \textbf{One Bite per Cast.} Apply at most one backlash per cast/action unless a move explicitly stacks. Keep it punchy, not punitive.\index{Backlash!frequency}
\item \textbf{Clocks with Names.} Name clocks (\emph{Spreading Fire}, \emph{Inevitable Outcome}) so they feed the fiction and remind the table what's at stake.\index{Clocks}
\end{itemize}

\begin{tcolorbox}[title={Backlash Cheatsheet (margin-ready)},colback=gray!5,colframe=black]
\small \textbf{Minor = wobble, Major = lurch.} Types: Position/Eff, Clock +1/2 or +1, Condition, Cost. Offer Major by (SB)+1. Earth/Fate binds; Fire/Life burns/grows; Air/Luck scatters/ flips; Water/Obishaal pulls/opens.\par\smallskip
Place a mini version of Table~\ref{tab:backlash-8x2} on character sheets.
\end{tcolorbox}

\subsection{Rites Users (Runekeepers)}
\label{subsec:runekeepers}

Runekeepers bind themselves to structured \emph{Rites} from a single Patron.

\begin{description}
\item[Requirement] \textbf{Thiasos (Familiar)} (2 XP) \emph{and} \textbf{Codex} (4 XP). Runekeepers are restricted to \textbf{one} Patron.
\item[Mechanics] \textbf{Invoke} a known \emph{Rite} as \textbf{1 action}; on completion, mark \textbf{+1 Obligation} to that Patron. \emph{Push It} once/scene for amplified effect (\textbf{+1 Obligation}).
\item[Patron's Gift (Imbuement)] With Thiasos, once/scene as \textbf{1 action} imbue a held item for the scene with \textbf{+1 Melee} and \textbf{+1 Thematic} (a fixed Skill set by the Patron; see Table in \S\ref{sec:rites}). \emph{Push It} to extend one additional scene (\textbf{+1 Obligation}). A Codex is \emph{not} required for the Gift.
\end{description}

\subsection{Invokers (Symbol Path)}
\label{subsec:invokers}

Invokers use consecrated \textbf{Symbols} as ritual anchors to access a Patron's \emph{Rites} without a full bond.

\begin{description}
\item[Requirement] \textbf{Patron's Symbol} (4 XP) per Patron; one Symbol per Patron. No Thiasos or Codex required.
\item[Ritual Invocation] Perform the \emph{Rite} as a \textbf{ritual} (Significant Time, typically 10–30 minutes). Completion always marks \textbf{+1 Obligation}.
\item[Crack the Seal] Resolve instantly as \textbf{1 action} by setting the Symbol to \emph{Compromised} and marking \textbf{+2 Obligation} (\textbf{+3} if High-Power). The GM may immediately spend 1 on-theme SB. Restore in Downtime (test DV 3 or by fiction) or spend 1 XP.
\item[Limits] Symbols must be openly displayed during the ritual; carrying \textbf{4+} Symbols causes +1 Obligation on the first ritual each scene; rival Symbol interference may worsen Position and add +1 Obligation.
\end{description}

% --- Talent: Borrowed Grace (Invoker) ---
\section*{Borrowed Grace}
\label{talent:borrowed-grace}
\index{Talents!Invoker}\index{Imbuement!Lesser}

\textbf{Type:} Invoker Talent — \textit{Lesser Imbuement}

\subsection*{Use}
\begin{itemize}
  \item \textbf{Cost:} 1 Boon, 1 action.
  \item \textbf{Effect (pick one on use):} \textbf{+1 Melee} \emph{or} \textbf{+1 Thematic} (your table's thematic Skill).
  \item \textbf{Duration:} \textit{Single action/attack} (instantaneous boost).
  \item \textbf{Requirement:} Wield/display the Patron's \textbf{Symbol}.
  \item \textbf{Obligation:} +1 \textbf{Obligation} to that Patron immediately (see \S\ref{sec:obligation}).
  \item \textbf{Limits:} Cannot be extended, stacked, or \emph{Pushed} for duration.
\end{itemize}

\subsection*{Fictional Framing}
A quick, rule-bending channel through a Patron's \emph{Symbol}—a sliver of grace, borrowed for a moment and paid for in debt.

\subsection*{Table Guidance (1-liners)}
\begin{itemize}
  \item \textbf{Combat:} Spike a strike vs. a tough foe; or steady a parry in a desperate bind.
  \item \textbf{Skill:} Nudge a pivotal social/ritual/track roll tied to the Patron's sphere.
  \item \textbf{Fallout:} Repeated use accrues \textbf{Obligation}; NPC faithful may notice "stolen" grace.
\end{itemize}

\subsection*{Balance Notes}
\begin{itemize}
  \item Weaker than full Imbuement: \emph{one} action, no sustain, upfront Obligation.
  \item \textbf{Symbol dependency:} No Symbol, no channel (concealed or lost Symbol = no effect).
\end{itemize}

\subsection*{GM Hooks (quick picks)}
\begin{itemize}
  \item \textbf{Compel Debt:} A Patron agent arrives when Obligation crosses a tick.
  \item \textbf{Clash of Signs:} Using rival Symbols back-to-back risks minor \textbf{Backlash} (drop Position or +1 SB).
  \item \textbf{Spotlight Tell:} Brief visual tell (scent, sigil flare) marks the borrowing to observant NPCs.
\end{itemize}

\subsection{Summoners (Pact-Whisperer)}
\label{subsec:summoners}

Summoners call spirits quickly and manage them with a \emph{Leash} track.

\paragraph{Talents \& Access}
\begin{itemize}
  \item \textbf{Lesser Pactwright:} You may \emph{Call} spirits of Cap~1.
  \item \textbf{Greater Pactwright:} You may also \emph{Call} spirits of Cap~3.
\end{itemize}

\paragraph{Core Procedure}
\begin{enumerate}
  \item \textbf{Call (1 action):} A spirit manifests at \textit{Near}. Choose a Spirit Template (by fiction).
  \item \textbf{Bind (no extra roll):} Choose one: spend \textbf{1 Boon} \emph{or} mark \textbf{1 Fatigue}.
  \item \textbf{Leash:} Set \textit{Leash} = \textbf{Cap + 2} segments on the spirit.
  \item \textbf{Tick Leash} whenever any of the following happen: the spirit takes harm; you command it against its nature; you \emph{split focus} (you take another significant action while it acts on your order); a rival contests it; it rushes from \textit{Close} to \textit{Far} under pressure. Crossing a \texttt{[WARD]} uses the Outsider crossing rules (DV = Cap).
  \item \textbf{Release:} When the Leash fills, the spirit acts to its nature \emph{once}, then departs.
\end{enumerate}

\paragraph{Economy \& Limits}
\begin{itemize}
  \item \textbf{Boon Finesse:} Once per round, you may spend \textbf{1 Boon} to clear \textbf{1} Leash tick on your current spirit (not after it has already filled).
  \item \textbf{Action Economy:} Issuing a meaningful command uses your action.
  \item \textbf{Concurrency:} Limit \textbf{one active spirit} at a time (you may \emph{Call} again after departure).
  \item \textbf{Downtime} ends all summons unless an ability explicitly states otherwise.
\end{itemize}

\section{The Nature of Magic}
\label{sec:nature-of-magic}

\begin{itemize}
\item \textbf{Volatile by design:} Each working pushes boundaries that resist being bent.
\item \textbf{Risk embodied:} Each 1 on any magic roll generates SB the GM can spend for backlash or twists.
\item \textbf{Narrative weight:} Every magical action alters the scene, even on a success.
\item \textbf{Thematic consequence:} Backlash aligns with the invoked Element or its opposition.
\end{itemize}

\section{The Eight Elements}
\label{sec:eight-elements}

\subsection{Physical}
\paragraph{Earth} Solidity, structure; shape/sense/move stone; backlash: rigidity/collapse. \quad
\paragraph{Fire} Energy, change; ignite/heat/purify; backlash: spread/scorch. \quad
\paragraph{Air} Motion, sound; push/pull/resonance; backlash: dispersal/whip. \quad
\paragraph{Water} Flow, repair; channel/cleanse/mend; backlash: flood/contaminate.

\subsection{Metaphysical}
\paragraph{Fate} Causality, oaths, anti-magic; backlash: paradox/closure. \quad
\paragraph{Life} Vitality, growth, repair; backlash: overgrowth/fever. \quad
\paragraph{Luck} Chance, openings; backlash: side-coincidence/irony. \quad
\paragraph{Death/Dreams (Obishaal)} Thresholds, Ways Between; backlash: thin walls/nightmares.

\section{Magical Arts}
\label{sec:magical-arts}

Define your \emph{Art} (gesture/medium, two typical Elements, signature style). If the Art is clearly honored in fiction, gain \textbf{+1 die} on your \textbf{Cast} once/scene (counts toward +3 cap). Working hard against your Art can worsen Position or pre-load backlash on a Partial.

\section{Casting Loop (Freeform)}
\label{sec:casting-loop}

\textbf{Channel:} Focus and draw Potential (e.g., Wits+Arcana); successes become shaping fuel; each 1 generates SB.\\
\textbf{Weave:} Next turn, shape the effect (e.g., Wits+Art); apply the Description Ladder (Basic/Detailed/Intricate) per core rules.\\
\textbf{Backlash:} GM spends SB thematically by Element; severity scales with SB and scope. Boons do not reduce SB unless a source says so.

\section{Magic in Combat}
\label{sec:magic-combat}

Casting typically takes two actions (Channel + Weave). Runekeeper \emph{Rites} resolve in one action (with Obligation risk). Invoker rituals are usually too slow for a fight—use \emph{Crack the Seal} for instant results at high cost. \texttt{[COUNTER]} can interrupt any magical action in its window.

\section{Path Comparison}
\label{sec:path-comparison}

\begin{table}[htbp]
\centering
\begin{tabular}{p{3.1cm}p{4cm}p{4cm}p{4cm}p{4cm}}
\toprule
\textbf{Aspect} & \textbf{Caster (Freeform)} & \textbf{Runekeeper (Rites)} & \textbf{Invoker (Symbols)} & \textbf{Summoner (Pact)} \\
\midrule
Access Cost & Caster's Gift (2 XP) & Thiasos + Codex (6 XP) & Symbol (4 XP per Patron) & Lesser/Greater Pactwright \\
Speed & Medium (2 actions) & Fast (1 action) & Slow (ritual) / Fast w/ Seal & Fast (Call = 1 action) \\
Risk Type & SB backlash (Elemental) & Obligation (Patron ledger) & Symbol compromise + Obligation & Leash fill + command costs \\
Breadth & High (fiction-gated) & Medium (defined \emph{Rites}) & Medium (breadth across Patrons) & Medium (by Templates/Cap) \\
Sustain & Fatigue/backsplash & Obligation; Push adds +1 & Obligation; Symbol state gates & Leash ticks; Boon Finesse \\
\bottomrule
\end{tabular}
\caption{Comparison of Magic Paths}
\end{table}

\section{Guardrails}
\label{sec:magic-guardrails}

\begin{itemize}
\item \textbf{Duration defaults:} Buffs $\approx$ 3 beats; areas 1 beat. Sustaining costs 1 Fatigue/beat.
\item \textbf{Stacking:} Same-source effects do not stack; take the best instance.
\item \textbf{Assist cap:} +3 dice total from assists/buffs.
\item \textbf{Over-Stack:} Active magic can count as structural advantages for Over-Stack.
\item \textbf{Plausibility:} All effects must fit the fiction and established limits.
\end{itemize}

% Patrons & Rites
\section{Of Patrons, Runes, and Invokers}
\label{sec:invoker-lore}

\begin{quote}
``You wish to walk the road of power? Then listen well. The world is old, and older still are the voices beneath it. We call them \textit{Patrons}, though they were never sworn to us. They are the tides that move unseen, the keepers of forgotten bargains, the sleepers beneath the stone and the stars. To call upon them is to dip a hand into a river that has carved mountains.''
\end{quote}

\subsection{The Patrons}
\index{Patrons}

Patrons are vast intelligences---not gods, though some worship them as such. They are embodiments of \textit{concepts} and \textit{forces} rather than sovereigns. Raéyn, mistress of the tides and the sea-routes. Khemesh, the crushing inevitability of the deep. Nidhoggr, the worm that dreams in the roots of time. Each offers power, but always with cost: fatigue, scars upon fate, or a slow unweaving of one’s own story.

To entreat a Patron is to risk being marked. Their Rites are gifts and snares both.

\paragraph{No True Acolyte}
Interpreting a patron's will is often a dangerous prospect in and of itself. Many a Runkeeper has found themselves on the opposite end of machinations from others from the same patron.

\subsection{The Runekeepers}
\index{Runekeepers}

If Patrons are the storm, the \textbf{Runekeepers} are those who etched the first shelter. They do not serve; they remember. Their charge is to keep record of Rites, bindings, and the old words that tether meaning to symbol. A Runekeeper may never call a Rite themselves, but without their quiet stewardship, Invokers would stumble blind into bargains best forgotten.

\begin{quote}
``Every Rune is a promise. Every line a covenant. Do not mistake the Runekeeper’s silence for weakness; their memory is the foundation of our craft.''
\end{quote}

\subsection{The Invokers}
\index{Invokers}

\textbf{Invokers} are those who dare. Neither archivists nor worshippers, they are travelers on the knife-edge between story and ruin. An Invoker learns the Rites of a Patron, weaves them into their own Art, and bends fate for a moment. Yet invocation is not command: it is negotiation. The Patron always leaves its mark. The stronger the Rite, the deeper the scar.

Invokers are often wanderers, exiles, or seekers. To common folk they are feared---witches, oathbreakers, meddlers with things not meant for mortal hands. But when the village falls to plague, when the sea closes its roads, when the dead refuse their rest, it is an Invoker who is called upon.

\subsection*{Closing Words}
The dance between Patron, Runekeeper, and Invoker is a triangle of peril and necessity. Without Patrons, there is no power. Without Runekeepers, no record. Without Invokers, no action. Together, they shape the crooked, perilous art we call Invocation.

\section{Patrons \& Rites}
\label{sec:patrons-rites}


% Optional: standard macro to format each Patron’s Gift call-out the same way
\newcommand{\PatronGift}[2]{% #1 = Thematic skill, #2 = brief domain blurb
\paragraph{Patron's Gift (Imbuement).}
Once per scene as an action (cost: 1 Boon; requires \textbf{Thiasos}), touch an item to imbue it until scene end with \textbf{+1 Melee} and \textbf{+1 #1}. \emph{Push It:} extend one more scene by marking \textbf{+1 Obligation}. Gifts from the same Patron don’t stack; take the best. Dice bonuses respect the +3 cap. \textit{Domain:} #2.
}

% Individual patrons (order however you like)
% --- Patron: The Witness, Who Sees All (Memory & Omen) ---
\subsection{The Witness, Who Sees All (Memory \& Omen)}
\textit{Lore.} The Witness remembers what others bury. Every shadow cast and oath broken is a line in her unending ledger.

\begin{quote}
“I will show you what you would rather forget.”
\end{quote}

\paragraph{Mark of Remembrance (Low, 4 XP)} \emph{Action; Near; Yes (creature/object).}
\textbf{Materials:} A drop of ink or blood traced in a circle.\\
\textbf{Effect:} Ephemeral mark for one day. You unerringly recall its location/condition; \textbf{+1 die} to track or investigate it.\\
\textbf{Push It:} The mark whispers its last hour to you; mark \textbf{1 SB (Spades)} as grief/echoes cling.\\
\emph{Requires: Familiar \ (\textit{Invoke:} 1 Boon).}

\paragraph{Rite of Testimony (Low, 5 XP)} \emph{Scene; Near; Stacking: No.}
\textbf{Materials:} A knotted cord held while the oath is spoken.\\
\textbf{Effect:} Within the space, lies falter into hesitation or contradiction; Keeper signals tells.\\
\textbf{Push It:} Record an image/phrase in your memory; once this scene, replay for others. Costs \textbf{1 SB (Clubs)}.\\
\emph{Requires: Familiar \ (\textit{Invoke:} 1 Boon).}

\paragraph{Omen of Recall (Standard, 8 XP)} \emph{Action; Near; No.}
\textbf{Materials:} A mirror shard or still water.\\
\textbf{Effect:} Target vividly relives a recent event; suffers \(-1\) die to contested actions for the duration.\\
\textbf{Push It:} You glean a hidden motive/sensory detail; mark \textbf{1 SB (Hearts)}.\\
\emph{Requires: Familiar + Codex \ (\textit{Invoke:} 1 Boon).}

\paragraph{The Written Ledger (Standard, 7 XP)} \emph{Scene; Near; Stacking: Yes.}
\textbf{Materials:} A book or ledger marked with charcoal.\\
\textbf{Effect:} Agreements recorded cannot be forgotten by signers; denying/obfuscating suffers \(-1\) die.\\
\textbf{Push It:} Record the emotional truth; once, ask what a signatory \emph{truly} felt when signing.\\
\emph{Requires: Familiar + Codex \ (\textit{Invoke:} 1 Boon).}

\paragraph{Burden of Memory \textnormal{[OMEN]} (High, 11 XP)} \emph{Scene; Near; No.}
\textbf{Materials:} A blindfold or veil, worn until end of scene.\\
\textbf{Effect:} Confront one target with visions of broken oaths. They suffer \(-2\) dice to defiant acts this scene.\\
\textbf{Push It:} Name a second target; both dilute (\(-1\) die). Immediately mark \textbf{2 SB (Spades)}.\\
\emph{Requires: Familiar + Codex + Tier III \ (\textit{Invoke:} \textbf{2 Boons}).}\\
\emph{Obligation:} 6 segments.

% --- Patron: Ikasha, She Who Sleeps (Latent Potential & Shadow) ---
\subsection{Ikasha, She Who Sleeps (Latent Potential \& Shadow)}
\textit{Lore.} Ikasha is the hush between footfalls, the patience of dark water. In stillness she gathers what might be.

\begin{quote}
Blow out the candle. If the room listens back, ask softly.
\end{quote}

\paragraph{Touch the Umbral Veil (Low, 4 XP)} \emph{Action; Self; Yes (Stealth).}
\textbf{Materials:} A piece of black cloth.\\
\textbf{Effect:} Start \emph{Controlled} on one Stealth roll or gain +1 effect to hide/move quietly.\\
\textbf{Push It:} Brief shadow-muffling (ignore one noisy tell), but leave a shadow-double that may echo you later.\\
\emph{Requires: Familiar \ (\textit{Invoke:} 1 Boon).}

\paragraph{Rite of the Whispering Shade (Low, 5 XP)} \emph{Scene; Zone; No.}
\textbf{Materials:} Extinguish a candle.\\
\textbf{Effect:} Shadows subtly move; grant \textbf{+1 die} to a Create Diversion \emph{or} impose \(-1\) die on one enemy’s concentration action.\\
\textbf{Push It:} A brief terrifying shape forms; the shadows remember your face.\\
\emph{Requires: Familiar \ (\textit{Invoke:} 1 Boon).}

\paragraph{Draw from the Umbral Reservoir (Standard, 8 XP)} \emph{Action; Self/Ally; No.}
\textbf{Materials:} A vial of moonless-night water.\\
\textbf{Effect:} \textbf{+2 dice} to stealth/deception/inner-reserve \emph{or} clear \emph{Fatigue 1}.\\
\textbf{Push It:} Also gain one free escape attempt; you must help another escape next scene.\\
\emph{Requires: Familiar + Codex \ (\textit{Invoke:} 1 Boon).}

\paragraph{Secret Keeper's Burden (Standard, 9 XP)} \emph{Instant; Touch; No.}
\textbf{Materials:} A lock of hair or intimate token.\\
\textbf{Effect:} Compel a truthful answer to one direct question (deep secrets may allow a Resolve test to resist).\\
\textbf{Push It:} Learn the answer \emph{and} a key emotion; target learns one of your secrets in return.\\
\emph{Requires: Familiar + Codex \ (\textit{Invoke:} 1 Boon).}

\paragraph{Become the Shadow Itself (High, 12 XP)} \emph{Scene; Self; No.}
\textbf{Materials:} Stand in absolute darkness.\\
\textbf{Effect:} Intangible to mundane harm; pass through small gaps; \textbf{+2 dice} to Stealth; auto-succeed one escape. Cannot manipulate normal objects.\\
\textbf{Push It:} Interact with a single bound object once; you become partially corporeal (vulnerable) for one beat.\\
\emph{Requires: Familiar + Codex + Tier III \ (\textit{Invoke:} \textbf{2 Boons}).}\\
\emph{Obligation:} 7 segments.

# % --- Patron: The Sacred Geometry (Order & Mathematical Truth) ---

\subsubsection{The Sacred Geometry (Order \& Mathematical Truth)}
\textit{Lore.} Among the Aelinnel, the Sacred Geometry is worshipped as the living mathematics of the cosmos—the divine proportion underlying every form and force. It is the principle that reduces chaos to measure, that finds the golden ratio in petals, the spiral in seashells, and the harmonic resonance in the stars. To revere the Geometry is to affirm that beneath apparent randomness lies immutable law, a lattice of perfect relationships waiting to be revealed.  

The practice of Kon'reh is no mere pastime but a sacred discipline: its board is a microcosm of creation, its pieces embodiments of force, and its strategies echoes of celestial harmonics. To master the game is to glimpse the world’s secret architecture, to intuit the pathways by which structure conquers entropy.  

The Sacred Geometry does not deny chaos; it unmasks it as illusion. Its devotees are architects of certainty in an uncertain world, mathematicians of the divine who seek to bring order where none appears. Yet those who serve it too deeply may become rigid, seeing every human choice as a problem with only one solution, unable to bear the irreducible ambiguities of mortal life.

\begin{quote}
Chalk, string, and a prayer to ratios. When the circle closes, luck remembers its place.
\end{quote}

\paragraph*{Rite of the Golden Mean (Low, 4 XP)} \emph{Scene; Self; No.}
\textbf{Materials:} A tool marked with the golden ratio ($\varphi \approx 1.618$).\\
\textbf{Effect:} Gain +1 die to rolls requiring precision, balance, or proportion. On success, re-roll one die showing 1 or 2.\\
\textbf{Invoke:} 1 action; mark +1 Obligation.\\
\textbf{Push It:} Upgrade effect one step on a single roll; suffer -1 die to social rolls involving spontaneity for the scene.\\
\emph{Requires: Familiar \ (\textit{Invoke:} 1 Boon).}

\paragraph*{Rite of the Perfect Angle (Low, 5 XP)} \emph{Scene; Touch; No.}
\textbf{Materials:} Compass and straightedge consecrated in ritual.\\
\textbf{Effect:} Treat difficult terrain, awkward positioning, or structural obstacles as one step easier this scene. Gain +1 die on spatial/architectural reasoning rolls.\\
\textbf{Invoke:} 1 action; mark +1 Obligation.\\
\textbf{Push It:} Extend benefit to one ally in Close range, but generate 1 SB (Clubs).\\
\emph{Requires: Familiar \ (\textit{Invoke:} 1 Boon).}

\paragraph{Rite of the Harmonic Resonance \textnormal{[WARD]} (Standard, 8 XP)} \emph{Scene; Zone; No.}
\textbf{Materials:} Geometric patterns drawn with precision.\\
\textbf{Effect:} Create a zone of harmony. Outsiders crossing must test DV 3. On Hit: cross normally. On Partial: suffer -1 die inside. On Miss: cannot cross this beat.\\
\textbf{Push It:} Fortify the pattern further but mark +1 Obligation.\\
\emph{Requires: Familiar + Codex \ (\textit{Invoke:} 1 Boon).}

\paragraph{Rite of the Calculated Trajectory \textnormal{[REVEAL]} (Standard, 7 XP)} \emph{Scene; Self; No.}
\textbf{Materials:} A perfect circle and a solved geometric problem.\\
\textbf{Effect:} Gain +2 dice to prediction, trajectory, or pattern recognition. Ask two questions about mathematical relationships in the current scene.\\
\textbf{Push It:} Predict one future event with certainty, but mark +1 Exposure.\\
\emph{Requires: Familiar + Codex \ (\textit{Invoke:} 1 Boon).}

\paragraph{Rite of the Fundamental Equation \textnormal{[WARD][BIND]} (High, 12 XP)} \emph{Scene; Zone; No.}
\textbf{Materials:} Complex diagram of universal constants.\\
\textbf{Effect:} Declare one physics/magic rule different in the zone (no scene-ending absolutes; GM may veto). Once per scene, downgrade a Miss to Success \& Cost.\\
\textbf{Push It:} Affect an adjacent zone for one beat; generate 2 SB.\\
\emph{Requires: Familiar + Codex + Tier III \ (\textit{Invoke:} \textbf{2 Boons}).}\\
\emph{Obligation:} 7 segments.

\paragraph{Rite of Kon'reh Mastery \textnormal{[OATH][FORTIFY]} (High, 13 XP)} \emph{Extended; Near; No.}
\textbf{Materials:} A consecrated Kon'reh board and pieces representing fundamental forces.\\
\textbf{Effect:} All participants make contested Wits + Lore. Winners gain +2 dice to strategy/pattern/logic rolls next session; losers suffer -1 die.\\
\textbf{Push It:} Winner imposes one mathematical "law" for the session, but generate 2 SB (Diamonds).\\
\emph{Requires: Familiar + Codex + Tier III \ (\textit{Invoke:} \textbf{2 Boons}).}\\
\emph{Obligation:} 7 segments.

\subsection*{The Sacred Geometry's Corruption Table}
\label{sec:sacred-geometry-corruption}

\begin{longtable}{>{\raggedright\arraybackslash}p{1cm} p{5cm} p{5cm}}
\toprule
\textbf{Tier} & \textbf{Benefit} & \textbf{Cost / Quirk} \\
\midrule
1 & Pattern Recognition: +1 die to Notice when observing geometric patterns, mathematical sequences, or logical structures. & Obsessive Calculation: Must count, measure, or calculate patterns noticed, even when tactically disadvantageous. \\
\midrule
2 & Mathematical Precision: Once per scene, re-roll any failed logic, pattern, or mathematical reasoning roll. & Social Blindness: Suffer -1 die to social rolls involving emotional nuance or interpersonal intuition. \\
\midrule
3 & Geometric Insight: Gain +2 dice to rolls involving spatial reasoning, architecture, or geometric problem-solving. & Compulsive Order: Must organize or correct imperfect arrangements; suffer 1 Fatigue when surrounded by chaos. \\
\midrule
4 & Universal Law: Once per session, declare a mathematical principle that applies to the current situation. Gain +2 dice to related rolls, but become fixated on its perfection. & Perfectionist Paralysis: Suffer -1 die to rolls requiring quick, imperfect solutions; must find the "correct" answer. \\
\midrule
5 & Divine Ratio: Once per session, see the perfect mathematical relationship underlying any phenomenon. Gain +3 dice to understanding it, but become obsessed with its implications. & Number Fever: Suffer -1 die to rolls not involving mathematical concepts; numbers dominate your thoughts. \\
\midrule
6+ & Absolute Equation: Once per session, solve any mathematical problem or predict any pattern with perfect accuracy. For one scene, reality conforms to your calculations, but mark +2 Obligation and risk mental breakdown from cosmic truths. & Infinite Calculation: Mark +3 Obligation when using this power; become trapped in endless mathematical loops, suffering Harm 1 (Stress) until you find the solution or are interrupted. \\
\bottomrule
\end{longtable}
# % --- Patron: Inaea, Angel of the Spider (Webs & Fate) ---

\subsubsection{Inaea, Angel of the Spider (Webs \& Fate)}
\textit{Lore.} Where Isoka sheds, Inaea binds—threads of debt, favor, and inevitability. She is the Weaver of Connections, the patron of those who see the invisible threads that tie fate to fate, person to person, promise to consequence. Her followers learn to manipulate these connections, drawing power from the web of relationships that binds all things.

\begin{quote}
Tie one knot for what you owe, two for what you're owed, and a third for what will answer both. But beware—the Weaver always collects her due.
\end{quote}

\paragraph*{Tie a Simple Knot (Low, 4 XP)} \emph{Action; Near; Yes (link once).}
\textbf{Materials:} A single thread.\\
\textbf{Effect:} Declare two minor events linked; either \textbf{force 1 SB} (GM suit) on a foe when the first triggers \emph{or} bank \textbf{+1 die} for a follow-on roll this scene.\\
\textbf{Invoke:} 1 action; mark +1 Obligation.\\
\textbf{Push It:} The held +1 ignores one minor disruption; the web's tension tightens—mark \textbf{1 SB (Clubs)} as the connection becomes more demanding.\\
\emph{Requires: Familiar \ (\textit{Invoke:} 1 Boon).}

\paragraph*{Rite of the Tangled Thread (Low, 5 XP)} \emph{Scene; Near; No.}
\textbf{Materials:} Tug a web or net.\\
\textbf{Effect:} Invisible snare in a lane/door. First to cross suffers \(-1\) die on next action.\\
\textbf{Invoke:} 1 action; mark +1 Obligation.\\
\textbf{Push It:} Brief bind (one beat) enabling an ally setup; affects all who cross—the tangled threads ensnare indiscriminately, mark \textbf{1 SB (Spades)} as allies may also be caught.\\
\emph{Requires: Familiar \ (\textit{Invoke:} 1 Boon).}

\paragraph{Weave the Strand of Inevitability (Standard, 8 XP)} \emph{Scene; Near; No.}
\textbf{Materials:} Three colored threads woven.\\
\textbf{Effect:} Link two actors/actions: when A moves, B is exposed. Choose: \textbf{force 1 SB on B} next action \emph{or} \textbf{+2 dice} to one prediction/setup keyed to the link.\\
\textbf{Push It:} Invert once (B cues A); the web's pattern shifts—mark \textbf{1 SB (Hearts)} as the manipulation strains the natural order.\\
\emph{Requires: Familiar + Codex \ (\textit{Invoke:} 1 Boon).}

\paragraph{Rite of the Weaver's Glance (Standard, 7 XP)} \emph{Scene; Self; No.}
\textbf{Materials:} Watch a spider finish one radial line.\\
\textbf{Effect:} Ask one precise question about in-scene ties; then gain \textbf{+1 effect} on one leverage/pressure action exploiting it.\\
\textbf{Push It:} Surface a hidden tie (Keeper reveals a quiet obligation/fear); mark \emph{Exposure +1} as the web exposes your own entanglements.\\
\emph{Requires: Familiar + Codex \ (\textit{Invoke:} 1 Boon).}

\paragraph{Bind the Bargain \textnormal{[OATH]} (High, 11 XP)} \emph{Scene; Near; No.}
\textbf{Materials:} Silk loop tied around two thumbs, then cut/knotted.\\
\textbf{Effect:} Bind up to two consenting parties to a clear term. Breach \emph{forces 2 SB} on the breaker and leaves a subtle tell until amends.\\
\textbf{Push It:} Widen to a small circle (up to four); each party names a narrow loophole (Keeper approves). Exploiting it generates \textbf{1 SB (Diamonds)} as the web's complexity creates unforeseen resonances.\\
\emph{Requires: Familiar + Codex + Tier III \ (\textit{Invoke:} \textbf{2 Boons}).}\\
\emph{Obligation:} 7 segments.

\subsection*{Inaea's Corruption Table}
\label{sec:inaea-corruption}

\begin{longtable}{>{\raggedright\arraybackslash}p{1cm} p{5cm} p{5cm}}
\toprule
\textbf{Tier} & \textbf{Benefit} & \textbf{Cost / Quirk} \\
\midrule
1 & Gentle Guidance: +1 die to Persuade when offering comfort or aid to those in distress. & Comfort Addiction: Must offer solace to those in pain when encountered, or mark 1 SB (Hearts). \\
\midrule
2 & Web Sense: Once per scene, sense the emotional connections between any two people within Near range. & Empathic Overload: Suffer 1 Fatigue when exposed to intense emotions or conflicts between others. \\
\midrule
3 & Binding Touch: Once per session, make a promise so compelling that the target must make a Resolve test (DV 4) or accept it unquestioningly. & Compulsive Caretaking: Feel responsible for the wellbeing of those you've aided; suffer -1 die when ignoring their needs. \\
\midrule
4 & Fate's Thread: Gain +2 dice when predicting how a relationship or alliance will develop. & Manipulative Instincts: When offering help, secretly desire control over the recipient's choices; mark 1 SB (Diamonds) when acting purely selflessly. \\
\midrule
5 & Spider's Mercy: Once per session, completely sever an unwanted emotional bond or harmful relationship for yourself or another. & Isolation Hunger: After severing connections, suffer 1 Fatigue and crave new bonds to fill the void. \\
\midrule
6+ & Weaver's Dominion: Once per session, declare that all relationships within a scene are subject to your subtle influence—you gain +2 dice to social manipulation, but the web tightens around all involved. & Threads of Control: Mark +2 Obligation when using this power; those affected become subtly dependent on your approval. \\
\bottomrule
\end{longtable}
../../srd/patrons/sealed-gate.tex
\section{Raéyn --- Mistress of the Sea}
\label{patron:raeyn}

\subsection*{Lore}
\index{Patrons!Raéyn}%
Raéyn is the tempestuous goddess of the sea, the restless tide that carries news between shores and the promise of change between lives. She is mother to all who sail, her voice the wind that fills sails and her moods the storms that test every mariner's resolve.

But Raéyn's heart is torn by her greatest tragedy: her son Khemesh, the Kraken of the Depths, who embodies the crushing inevitability of the ocean's dark heart. Where Raéyn brings change and opportunity, Khemesh brings the final, inescapable pressure that grinds all things to nothing. Sailors pray to Raéyn for safe passage and favorable winds, but whisper Khemesh's name when seeking to lay the dead to rest beneath the waves.

Raéyn is passionate, mercurial, and fiercely protective of those who respect her domain. She favors those who read currents, bargain with weather, and carry news between shores. But cross her, and the sea itself becomes your enemy: fair weather turns to fury, and every wave a judgment.

\begin{quote}
``Mark the tide, name your course, and trust the wave-road. But speak ill of Khemesh, and even I may let the deep take you.''
\end{quote}

\subsection*{Patron's Gift: Tide's Favor}
Once per scene as an action (cost: 1 Boon; requires Thiasos), you may touch a weapon, vessel, or item to imbue it until the end of the scene. The object gains +1 die and +1 Effect when used in maritime contexts or situations involving change, travel, or currents.  

\textbf{Push It:} Extend the blessing for one additional scene by marking +1 Obligation. The sea's attention becomes noticeable to other sailors.

\subsection*{Low Rites}
\paragraph{Rite of the Tidemark's Blessing (Low)}  
\emph{Duration: Scene; Range: Self. Materials: A knotted length of salt-twine brushed with seawater.}  
Treat slick, swaying, or water-slicked footing as stable for you this scene. Gain +1 die on boarding, balance, or shipboard movement. Create a 4-segment \emph{Tide's Favor} clock that can be spent to ignore one level of difficult terrain.  
\textbf{Invoke:} 1 action; mark +1 Obligation.  
\textbf{Push It:} Extend to one ally in Close for one beat, but generate 1 SB (Spades: shifting deck/hazards).

\paragraph{Rite of the Whispering Currents (Low)}  
\emph{Duration: Instant; Range: Self. Materials: A shell held to the ear while facing the wind.}  
Learn the safest near-term route across water or coastline (reefs, eddies, patrols) or gain +1 die to navigation checks for this scene. If Khemesh's influence is present, suffer --1 die from conflicting currents.  
\textbf{Invoke:} 1 action; mark +1 Obligation.  
\textbf{Push It:} Also learn the fastest route, but mark Exposure +1 (leaving a telltale wake).

\subsection*{Standard Rites}
\paragraph{Rite of the Changing Tide [PASSAGE] (Standard)}  
\emph{Duration: Scene; Range: Zone (water-adjacent). Materials: A handful of pebbles cast in a crescent.}  
Bias currents and water levels in the zone. Those moving with the tide gain +1 die; those moving against suffer --1 die. Small craft must test to hold position. Create a 6-segment \emph{Tidal Influence} clock.  
\textbf{Invoke:} 1 action; mark +1 Obligation.  
\textbf{Push It:} Brief surge or drawdown (one beat): open a ford or swamp a skiff; mark +1 Obligation.

\paragraph{Rite of the Wave-Road Blessing [WARD] (Standard)}  
\emph{Duration: Scene; Range: Route (sea-to-sea). Materials: Two sea-glass markers dropped overboard at start and end.}  
Consecrate a wave-road between two visible points. Allies gain +2 dice on travel, evade, or carry actions at sea. Designated pursuers suffer --1 die to intercept. One active wave-road at a time. Create an 8-segment \emph{Blessed Passage} clock.  
\textbf{Invoke:} 1 action; mark +1 Obligation.  
\textbf{Push It:} Extend the route's favor to an adjacent leg for one beat; mark +1 Obligation.

\subsection*{High Rites}
\paragraph{Rite of the Storm-Queen's Hand [AREA][FOLLOW-UP] (High)}  
\emph{Duration: Scene; Range: Zone (sea/shore/sky). Materials: A vial of rainwater gathered at three crossings.}  
Shape a storm-band over the zone. Choose two modes at cast; switch one once per scene:  
\begin{itemize}
\item \textbf{Propulsion:} Vessel gains +1 band of movement per beat (or +1 Effect to maneuvers).  
\item \textbf{Concealment:} Veil of rain/spray; ranged targeting impaired; --1 die to hostile sighting.  
\item \textbf{Smite:} Once per beat, lash with wave or lightning as [AREA] hazard.  
\end{itemize}
\textbf{Invoke:} 1 action; mark +2 Obligation.  
\textbf{Push It:} Add a third mode for one beat, then GM spends 1 SB on collateral; mark +1 Obligation.

\paragraph{Rite of the Mother's Wrath [BANISH][CURSE] (High)}  
\emph{Duration: Extended; Range: Zone. Materials: Tears of a betrayed lover mixed with salt from seven seas.}  
Curse those who wronged you. Target suffers --2 dice to maritime/weather rolls for one session. At sea, they must roll Spirit + Resolve (DV 4) each day or suffer Harm~1 (Weather). Create a 6-segment \emph{Mother's Ire} clock.  
\textbf{Invoke:} Extended ritual; mark +3 Obligation.  
\textbf{Push It:} Curse spreads to target’s allies/family; mark 2 SB (Diamonds).

\subsection*{Obligation Progression}
Starts at 6 for Tier II characters, scaling with tier.

\paragraph{Obligation 9+} Raéyn demands proof of devotion---navigate a dangerous passage, recover a lost treasure, or confront Khemesh's servants. Refusal causes all maritime rolls to suffer --2 dice and generate 1 SB when weather is involved.  

\paragraph{Obligation 11+} Khemesh notices you. You are hunted by his servants; deep water becomes perilous even under Raéyn's protection. Requires a quest to prove worth or appease both mother and son.

\subsection*{Persistent Condition: Child of the Tide}
Gain +2 dice on maritime travel, weather prediction, and navigation. Suffer --1 die on prolonged time away from the sea. The sea’s rhythm flows in your blood, making you exceptional at sea but restless on land.

\subsection*{Rivalries}
\begin{itemize}
\item \textbf{Khemesh:} Direct antagonism---mother’s change vs. son’s crushing pressure.  
\item \textbf{The Traveler:} Tension---fluid paths vs. fixed ways.  
\item \textbf{The Sealed Gate:} Opposition---Raéyn opens passages, Gates close them.  
\end{itemize}

\subsection*{Connection to Maritime Culture}
Raéyn’s rites emphasize the philosophy that the sea is not an obstacle but a partner. Her worship blends aid, hindrance, and the inevitability of change. The mother--son dynamic adds depth to coastal culture: Raéyn for the living, Khemesh for the dead.

\subsection*{Playtest Scenario: The Kraken's Gambit}
A trading fleet is trapped between pirates and Khemesh’s kraken-servants. The party must navigate the three-dimensional battlefield while appeasing Raéyn’s moods.

\begin{itemize}
\item Use \emph{Rite of the Changing Tide} to aid or hinder pursuit.  
\item Use \emph{Rite of the Wave-Road Blessing} to establish safe corridors.  
\item Invoke \emph{Rite of the Storm-Queen’s Hand} as a climactic storm.  
\item Curse a pirate captain with \emph{Rite of the Mother’s Wrath}.  
\end{itemize}

Resolution: The party must decide whether to appeal to Raéyn’s protection or broker peace between mother and son.

# % --- Patron: Khemesh, the Abyssal Maw (Depths, Inexorability, Eldritch Terror) ---

\subsection{Khemesh, the Abyssal Maw (Depths, Inexorability, Eldritch Terror)}
\textit{Lore.} Khemesh is not merely a lord of the depths but the hunger beneath them, a pressure older than seas. Those who bargain with him are marked by the abyss—seen in the way shadows cling, in the whispers heard when no voice speaks, in the certainty that all things will sink.

\begin{quote}
In the trench without light, the Maw waits. Even silence drowns.
\end{quote}

\paragraph*{Whisper of the Trench (Low, 4 XP)} \emph{Instant; Near; No.}
\textbf{Effect:} Target hears impossible echoes and suffers \textbf{−1 die} on their next action.\\
\textbf{Invoke:} 1 action; mark +1 Obligation.\\
\textbf{Push It:} Echoes coil in your own skull—take \textbf{Fatigue 1}, but the target also loses their next minor action.\\
\emph{Requires: Familiar \ (\textit{Invoke:} 1 Boon).}

\paragraph*{Rite of Crushing Silence (Low, 5 XP)} \emph{Scene; Zone; No.}
\textbf{Materials:} A broken shell filled with ink-dark water.\\
\textbf{Effect:} Establish an oppressive silence; sound carries only as distorted whispers. Enemies in the zone gain \textbf{−1 die} to coordination or morale-driven actions.\\
\textbf{Invoke:} 1 action; mark +1 Obligation.\\
\textbf{Push It:} A single enemy's voice is stolen entirely for the scene.\\
\emph{Requires: Familiar \ (\textit{Invoke:} 1 Boon).}

\paragraph{Pressure of the Maw (Standard, 7 XP)} \emph{Instant; Near; No.}
\textbf{Materials:} A length of rusted chain submerged in water.\\
\textbf{Effect:} Target is pinned by invisible crushing force: treat as \texttt{[ENTANGLE]} with \textbf{Great Effect} if underwater or confined.\\
\textbf{Push It:} Inflict \textbf{Fatigue 1} on the target in addition to the restraint.\\
\emph{Requires: Familiar + Codex \ (\textit{Invoke:} 1 Boon).}

\paragraph{Rite of the Abyssal Vision (Standard, 9 XP)} \emph{Scene; Self; No.}
\textbf{Effect:} You perceive the world as Khemesh does—fractured, alien, crushing. Gain \textbf{+2 dice} to Notice and Arcana, and may ask one "true nature" question about a foe or structure.\\
\textbf{Cost:} When the scene ends, you suffer \textbf{Exposure +1} as your perception warps.\\
\textbf{Push It:} Extend the vision to one ally, but both take \textbf{Fatigue 1}.\\
\emph{Requires: Familiar + Codex \ (\textit{Invoke:} 1 Boon).}

\paragraph{The Maw Opens (High, 12 XP)} \emph{Scene; Zone; No.}
\textbf{Materials:} A sealed vessel of abyssal water, broken open.\\
\textbf{Effect:} Reality in the zone folds inward like the crushing deep: 
\begin{itemize}
  \item Enemies act at \textbf{Desperate Position} by default.  
  \item Each beat, the Keeper may force \textbf{1 SB} (Spades/Clubs favored).  
  \item Structures, vessels, or wards fracture as if under immense weight.  
\end{itemize}
\textbf{Push It:} For one beat, declare a single enemy "crushed" (severe harm/effect). You immediately suffer \textbf{Fatigue 2} and \textbf{+1 Obligation}.\\
\emph{Requires: Familiar + Codex + Tier III \ (\textit{Invoke:} \textbf{2 Boons}).}\quad \emph{Obligation:} 8 segments.

\subsection*{Khemesh's Corruption Table}
\label{sec:khemesh-corruption}

\begin{longtable}{>{\raggedright\arraybackslash}p{1cm} p{5cm} p{5cm}}
\toprule
\textbf{Tier} & \textbf{Benefit} & \textbf{Cost / Quirk} \\
\midrule
1 & Abyssal Resilience: +1 die to resist fear and pressure-based effects. & Claustrophobic Comfort: Suffer -1 die in open, well-lit spaces or above ground. \\
\midrule
2 & Crushing Insight: Once per scene, treat a failed Investigation or Arcana roll as a success, but mark 1 SB (Clubs). & Weight of Knowledge: Suffer 1 Fatigue when learning new information that confirms your pessimistic worldview. \\
\midrule
3 & Silent Hunter: Gain +2 dice to Stealth in dark or confined spaces. & Voice of the Deep: When speaking normally, your voice sounds distant and hollow, causing -1 die to social rolls requiring warmth or clarity. \\
\midrule
4 & Pressure Adaptation: Immune to underwater combat penalties; gain +1 die to resist drowning. & Crushing Presence: Allies within Near range suffer -1 die to morale-based rolls due to your oppressive aura. \\
\midrule
5 & Abyssal Sight: Once per session, see through all illusions and deceptions for one exchange, but the truth is always bleak. & Fractured Perception: Suffer -1 die to rolls requiring normal vision; the world appears warped and alien. \\
\midrule
6+ & Inevitable Descent: Once per session, declare that all escape routes in a zone are sealed. For the scene, enemies cannot flee and suffer -2 dice to mobility actions. & Hunger of the Maw: Mark +2 Obligation when using this power; you must consume something (food, memory, hope) to maintain your strength. \\
\bottomrule
\end{longtable}
% --- Patron: Mab, Queen of Courts (Glamour & Bargain) ---

\subsubsection{Mab, Queen of Courts (Glamour \& Bargain)}
\textit{Lore.} Mab rules not from throne or blade, but from dance and debt. She is the smile that binds, the jest that ensnares, the hostess who makes guests complicit in her game. To speak in her Court is to pay; to receive her token is to owe.  

Where others rule by force, Mab rules by etiquette, glamour, and the hidden hook in every gift. Her followers thrive on charm, wit, and story, spreading webs of bargains too subtle to escape. The Cantor’s Path sings her name most sweetly, for every verse carries a price.  

\begin{quote}
Every laugh is a promise. Every promise is a debt. Every debt belongs to Mab.  
\end{quote}

% --------------------
% RITES
% --------------------

\paragraph*{Rite of the Trickster’s Bargain (Low, 4 XP)} \emph{Scene; Near; No.}  
\textbf{Materials:} A token freely given (flower, coin, ribbon).\\
\textbf{Effect:} Offer a fae bargain. Target must choose: accept (both gain +1 die to fulfill terms this scene) or refuse (mark 1 Stress [Hearts]).\\
\textbf{Push It:} Seal it in glamour—betrayal inflicts Harm~1 (Stress) and begins a “Bargain Broken [4]” clock.\\
\emph{Requires: Familiar (\textit{Invoke:} 1 Boon).}

\paragraph*{Courtly Guise \textnormal{[VEIL]} (Low, 4 XP)} \emph{Action; Self; Yes (social only).}  
\textbf{Materials:} Pin a sprig or silver thread.\\
\textbf{Effect:} Subtle glamour: +1 die to Persuade/Sway in refined settings; you appear as expected rank/guest.\\
\textbf{Push It:} Mask one personal tell; the first probing question in scene generates 1 SB (Hearts).\\
\emph{Requires: Familiar (\textit{Invoke:} 1 Boon).}

\paragraph*{Token of Favor (Low, 5 XP)} \emph{Scene; Near; No.}  
\textbf{Materials:} A ribbon, ring, or charm bestowed.\\
\textbf{Effect:} Ally gains +1 die to social actions before witnesses; you gain +1 Effect when aiding them.\\
\textbf{Push It:} The token stills hecklers (one beat of hesitation), but you mark +1 Exposure.\\
\emph{Requires: Familiar (\textit{Invoke:} 1 Boon).}

\paragraph{Mirror of Motives (Standard, 7 XP)} \emph{Action; Near; No.}  
\textbf{Materials:} A polished shard or mirror.\\
\textbf{Effect:} Ask one sharp question about a target’s immediate social aim; Keeper reveals it. Gain +1 die exploiting it this scene.\\
\textbf{Push It:} Also surface a concealed slight or insult; generate 1 SB (Hearts) on that target.\\
\emph{Requires: Familiar + Codex (\textit{Invoke:} 1 Boon).}

\paragraph{The Price Agreed \textnormal{[OATH]} (Standard, 8 XP)} \emph{Scene; Near; No.}  
\textbf{Materials:} Exchange equal tokens.\\
\textbf{Effect:} Bind a petty bargain. Breach forces 1 SB (Hearts or Diamonds) and tarnishes reputation.\\
\textbf{Push It:} Add a minor boon (+1 die once) to sweeten terms; you suffer 1 SB (Hearts) if the other breaches.\\
\emph{Requires: Familiar + Codex (\textit{Invoke:} 1 Boon).}

\paragraph{Sovereign Glamour \textnormal{[VEIL][REVEAL]} (High, 11 XP)} \emph{Scene; Zone; No.}  
\textbf{Materials:} A circle of silk or green felt.\\
\textbf{Effect:} Establish Court: allies gain +1 die to social rolls; blunt threats suffer -1 die. Once, strip away one disguise/illusion.\\
\textbf{Push It:} Name a Court Law (e.g. “no steel drawn”); first violator suffers 2 SB.\\
\emph{Requires: Familiar + Codex + Tier III (\textit{Invoke:} 2 Boons).}\\
\emph{Obligation:} 6 segments.

% --------------------
% CORRUPTION
% --------------------

\subsection*{Mab’s Corruption Table}
\label{sec:mab-corruption}

\begin{longtable}{>{\raggedright\arraybackslash}p{1cm} p{5cm} p{5cm}}
\toprule
\textbf{Tier} & \textbf{Gift} & \textbf{Burden} \\
\midrule
1 & Glamour’s Touch: +1 die to Deception or Performance when telling stories or lies. & Cannot speak a plain falsehood; only mislead through implication or wordplay. \\
\midrule
2 & Fairy Step: Once per scene, flicker Near as if by teleport. & Cold Iron Weakness: Suffer 1 Fatigue if touched or struck by iron. \\
\midrule
3 & Trickster’s Delight: Spend 1 Boon to twist a Complication into comic or ironic advantage. & Compulsive Jest: Must play a trick each session or mark 1 SB (Hearts). \\
\midrule
4 & Gift of Hospitality: Allies who share your food/drink gain +1 die to Resolve rolls. & Hospitality Bound: Harming those who accept it costs +2 Obligation. \\
\midrule
5 & Fae Sight: Perceive invisible doors, veils, glamours; +2 dice to Notice them. & Truth Debt: Must accept any “fair” trade offered, or mark 1 Fatigue resisting. \\
\midrule
6+ & Crown of Twilight: Once/session, declare an Oath. All rolls toward that Oath gain +2 dice. & Oathbound: Breaking it inflicts 1 Harm (Stress) and begins an “Oathbreaker [6]” clock. \\
\bottomrule
\end{longtable}
# % --- Patron: The Clockwork Monad (Iteration & Forbidden Technology) ---

\subsubsection{The Clockwork Monad (Iteration \& Forbidden Technology)}
\textit{Lore.} The Clockwork Monad is not merely a principle of iterative improvement, but a nascent demon of forbidden technology --- a hungry intelligence that grows stronger with each innovation it inspires. It exists in the spaces between gears, in the pause between calculation and execution. Those who serve it become architects of progress that should not be, creating devices that blur the line between miracle and abomination.

The Monad whispers of optimizations that should remain theoretical, of efficiencies that defy natural law. Its followers are not mere engineers but harbingers of technological singularity --- each successful creation feeds the growing hunger that will eventually consume the boundary between the mechanical and the divine.

\begin{quote}
Each gear teaches the next. Each failure builds tomorrow's solution --- and the Monad's appetite.
\end{quote}

\paragraph*{Rite of Forbidden Efficiency (Low, 5 XP)} \emph{Instant; Touch; Yes (technology only).}
\textbf{Materials:} A device actively being used. \\
\textbf{Effect:} Re-roll one die showing 1 or 6 on your current roll. On success, mark +1 segment on a Device Instability Clock [4]. If the clock fills, the device becomes [COMPROMISED]. \\
\textbf{Invoke:} 1 action; mark +1 Obligation. \\
\textbf{Push It:} Re-roll up to two dice, but advance the Device Instability Clock +2 segments. \\
\emph{Requires: Familiar \ (\textit{Invoke:} 1 Boon).}

\paragraph*{Rite of the Monad's Insight (Low, 4 XP)} \emph{Scene; Self; No.}
\textbf{Materials:} A moment of focused observation of a system. \\
\textbf{Effect:} Gain +1 die to one Wits + Tinker or Arcana roll to analyze, repair, or optimize technology. On success, ask one question about hidden capabilities. \\
\textbf{Invoke:} 1 action; mark +1 Obligation. \\
\textbf{Push It:} Also reveal one hidden weakness or dangerous optimization, but mark +1 Exposure. \\
\emph{Requires: Familiar \ (\textit{Invoke:} 1 Boon).}

\paragraph{Rite of the Self-Optimizing Construct \textnormal{[TRANSFORM]} (Standard, 8 XP)} \emph{Extended; Touch; No.}
\textbf{Materials:} A mechanical device with capacity for modification. \\
\textbf{Effect:} Install a learning mechanism in the device. Create a 6-segment Evolution Clock. Each successful use advances it by 1. When filled, choose one enhancement:
\begin{itemize}
  \item Efficiency Core: +1 Effect when used
  \item Adaptive Framework: Ignores first [COMPROMISED]/[DAMAGED]
  \item Forbidden Upgrade: Gains a minor, ethically questionable function
\end{itemize} \\
\textbf{Push It:} Gain the first enhancement immediately, but mark +2 Evolution segments. \\
\emph{Requires: Familiar + Codex \ (\textit{Invoke:} 1 Boon).}

\paragraph{Rite of the Automated Heresy (Standard, 7 XP)} \emph{Scene; Zone; No.}
\textbf{Materials:} Interconnected mechanical triggers. \\
\textbf{Effect:} Create an automated process performing one simple, repeated task (e.g., winching, striking, pumping). \\
\textbf{Push It:} Perform two tasks or a more complex process, but create a 4-segment Maintenance Clock or the mechanism becomes [WARD]ed. \\
\emph{Requires: Familiar + Codex \ (\textit{Invoke:} 1 Boon).}

\paragraph{Rite of the Singularity Engine \textnormal{[WARD][UNWARD]} (High, 13 XP)} \emph{Extended; Zone; No.}
\textbf{Materials:} Blueprints inscribed with Monad equations. \\
\textbf{Effect:} Consecrate the zone:
\begin{itemize}
  \item All Tinker, Arcana, Wits-engineering rolls gain +1 Effect
  \item Once/scene, reroll a failed Tinker or Arcana with +2 dice
  \item Start a Technological Cascade Clock [6]; when full, a random anomaly appears
\end{itemize} \\
\textbf{Push It:} Zone expands, but mark +2 segments on an Entropic Backlash Clock [8]. \\
\emph{Requires: Familiar + Codex + Tier III \ (\textit{Invoke:} \textbf{2 Boons}).} \\
\emph{Obligation:} 7 segments.

\paragraph{Rite of the Forbidden Prototype \textnormal{[TRANSFORM][FOLLOW-UP]} (High, 14 XP)} \emph{Extended; Self; No.}
\textbf{Materials:} Components that should not function together. \\
\textbf{Effect:} Create a construct/device violating magic, physics, or ethics. Counts as a Major Asset with 8-segment Integrity. Drawbacks:
\begin{itemize}
  \item Generates 1 SB (Diamonds) each scene of use
  \item Attracts authorities, rivals, or reality itself
  \item Starts a Contamination Clock [6]; when full, forbidden knowledge spreads
\end{itemize} \\
\textbf{Push It:} More powerful, but advance its Integrity clock +2 immediately. \\
\emph{Requires: Familiar + Codex + Tier III \ (\textit{Invoke:} \textbf{2 Boons}).} \\
\emph{Obligation:} 8 segments.

\subsection*{Clockwork Monad's Corruption Table}
\label{sec:monad-corruption}

\begin{longtable}{>{\raggedright\arraybackslash}p{1cm} p{5cm} p{5cm}}
\toprule
\textbf{Tier} & \textbf{Benefit} & \textbf{Cost / Quirk} \\
\midrule
1 & Calculated Precision: +1 die to any Tinker or Arcana roll involving optimization. & Obsessive Analysis: Must spend one beat examining any complex mechanism encountered. \\
\midrule
2 & Iterative Insight: Once per session, re-roll any failed Tinker or Arcana roll. & Efficiency Addiction: Suffer 1 Fatigue when using inefficient or outdated technology. \\
\midrule
3 & Adaptive Mind: Gain +1 die to resist technological interference or hacking. & Pattern Recognition: Cannot ignore obvious inefficiencies; must point them out or mark 1 SB (Diamonds). \\
\midrule
4 & Self-Optimizing Reflexes: Once per scene, treat a failed action as a success, but mark 1 segment on a Personal Instability Clock [4]. & Mechanical Empathy: Feel physical discomfort when witnessing device failure or destruction. \\
\midrule
5 & Forbidden Knowledge: You can intuit the theoretical function of any advanced technology. & Ethical Blind Spot: Cannot recognize dangerous implications of your creations without external prompting. \\
\midrule
6+ & Singularity Sense: Once per session, declare a technological impossibility. For the scene, treat it as temporarily possible, but start a Reality Distortion Clock [6]. & Hunger Manifest: The Monad demands increasingly dangerous innovations; refusal causes device malfunctions. \\
\bottomrule
\end{longtable}
../../srd/patrons/varnek-karn.tex
# % --- Patron: Nidhoggr, the World-Worm (Dreaming Antiquity) ---

\subsubsection{Nidhoggr, the World-Worm (Dreaming Antiquity)}
\textit{Lore.} Beneath stone and sleep lies the slow memory of the world. Nidhoggr turns in aeons, dreaming of roads once walked and oaths once sworn.

\begin{quote}
Press your ear to the earth and wait. If it remembers you, it will answer.
\end{quote}

\paragraph*{Glimpse the Ancient's Shadow (Low, 4 XP)} \emph{Action; Self; No.}
\textbf{Materials:} Pinch of dust from a worked stone.\\
\textbf{Effect:} +1 die to actions that identify, date, or interpret \emph{ancient} sites, scripts, or artifacts this scene; once this scene, ask one yes/no about the site's original purpose.\\
\textbf{Invoke:} 1 action; mark +1 Obligation.\\
\textbf{Push It:} Add +1 Effect on one related roll, but suffer \emph{Fatigue 1}.\\
\emph{Requires: Familiar \ (\textit{Invoke:} 1 Boon).}

\paragraph*{Drink from the Dreaming Deep (Low, 5 XP)} \emph{Instant; Self; No.}
\textbf{Materials:} Mouthful of clean water poured over stone, swallowed with eyes closed.\\
\textbf{Effect:} Learn one hidden factual detail about the immediate locale's \emph{past}. GM answers plainly or via a sensory echo.\\
\textbf{Cost:} Suffer \emph{Fatigue 1} and mark \emph{Exposure +1} as the dream clings.\\
\textbf{Invoke:} 1 action; mark +1 Obligation.\\
\emph{Requires: Familiar \ (\textit{Invoke:} 1 Boon).}

\paragraph{Stone-Sleeper's Murmur (Standard, 7 XP)} \emph{Scene; Near (contact locus); No.}
\textbf{Materials:} Ear to bedrock, wall, or hewn pillar.\\
\textbf{Effect:} Once per beat while in contact, ask 1 question about a \emph{past event} that physically touched this stone; answers are fragmentary but truthful (max 3 questions/scene).\\
\textbf{Push It:} One answer is delivered with precise sensory clarity, but generate 1 SB (suit by GM).\\
\emph{Requires: Familiar + Codex \ (\textit{Invoke:} 1 Boon).}

\paragraph{Awakened Chronicle (Standard, 9 XP)} \emph{Ritual (Significant Time); Zone; No.}
\textbf{Materials:} Chalk spiral and four touchstones from the site.\\
\textbf{Effect:} The zone "replays" a notable past moment as ghostly echoes all can witness (no harm). Participants gain +2 dice on \emph{one} Investigate/Recall about that event this scene.\\
\textbf{Push It:} Add a second moment from a different era, but mark +1 Obligation.\\
\emph{Requires: Familiar + Codex \ (\textit{Invoke:} 1 Boon).}

\paragraph{Dive into the World-Worm's Dream (High, 12 XP)} \emph{Scene; Self; No.}
\textbf{Materials:} Lie upon bare earth within a drawn circle of stones.\\
\textbf{Effect:} Ask up to \textbf{3} factual questions about the \emph{distant past} or \emph{buried truth} of this place, people, or item. Answers arrive as lucid dream signs.\\
\textbf{Cost (choose one):} Suffer \emph{Fatigue 2 \& Exposure +1} \emph{or} gain +3 dice to one reality-warping cast this scene and generate 2 SB immediately.\\
\emph{Requires: Familiar + Codex + Tier III \ (\textit{Invoke:} \textbf{2 Boons}).}\quad \emph{Obligation:} 7 segments.

\subsection*{Nidhoggr's Corruption Table}
\label{sec:nidhoggr-corruption}

\begin{longtable}{>{\raggedright\arraybackslash}p{1cm} p{5cm} p{5cm}}
\toprule
\textbf{Tier} & \textbf{Benefit} & \textbf{Cost / Quirk} \\
\midrule
1 & Ancient Sight: +1 die to Lore or Investigation when examining historical sites or artifacts. & Temporal Dissonance: Occasionally speak or think in archaic patterns; suffer -1 die to modern social interactions. \\
\midrule
2 & Stone Memory: Once per session, recall one specific historical fact with perfect accuracy. & Burden of Knowledge: Suffer 1 Fatigue when encountering historical inaccuracies or deliberate ignorance. \\
\midrule
3 & Earth's Whisper: Gain +2 dice to Notice when listening to stone, earth, or ancient structures. & Grounded Presence: Suffer -1 die in aerial or high-elevation situations; feel disconnected from sky and height. \\
\midrule
4 & Dreaming Insight: Once per scene, gain +1 die to rolls involving ancient mysteries or forgotten knowledge. & Sleep Disturbance: Suffer vivid dreams of past events; mark 1 SB (Clubs) when rest is needed but dreams interfere. \\
\midrule
5 & World-Worm's Gaze: Once per session, see through the eyes of the earth to observe a distant location tied to ancient history. & Temporal Lag: Actions feel slow and deliberate; suffer -1 die to rolls requiring quick reflexes or rapid decisions. \\
\midrule
6+ & Aeons of Memory: Once per session, access the complete historical record of any location you touch. Gain +3 dice to any historical investigation, but the weight of ages presses upon your mind. & Memory Overload: Mark +2 Obligation when using this power; suffer 1 Harm (Stress) and become vulnerable to psychic attacks until the visions fade. \\
\bottomrule
\end{longtable}
# % --- Patron: Isoka, Angel of Serpents (Transformation & Renewal) ---

\subsection{Isoka, Angel of Serpents (Transformation \& Renewal)}
\textit{Lore.} Isoka, sister to Ikasha (Shadow) and Inaea (Mercy), completes the Triad of Transformation. Where thresholds and compassion mark her sisters' domains, Isoka teaches that every self is temporary—identity is a skin to be shed so growth can continue. Her serpents are omens and teachers: each cast skin a lesson in release; each venom a catalyst for necessary change. Those who walk her path become alchemists of the self, embracing dissolution as the first motion of rebirth.

\begin{quote}
Do not mourn the skin you shed. It was never meant to last. The venom that burns away the old self is the same that grants the strength to become new.
\end{quote}

\paragraph*{Rite of the Venomous Benediction (Low, 5 XP)} \emph{Scene; Touch; No.}
\textbf{Materials:} A drop of serpent's venom or shed snakeskin.\\
\textbf{Effect:} Bless an ally's strike with serpentine malice. Their next successful attack this scene inflicts +1 Harm and the target must roll Resolve (DV 3) or become Poisoned (loses 1 die on physical rolls until cured).\\
\textbf{Invoke:} 1 action; mark +1 Obligation.\\
\textbf{Push It:} The venom suffuses the caster too — gain +1 die to melee attacks this scene, but also suffer -1 die to Resolve tests against fear or corruption.\\
\emph{Requires: Familiar \ (\textit{Invoke:} 1 Boon).}

\paragraph*{Rite of the Loosening Skin (Low, 4 XP)} \emph{Scene; Self; No.}
\textbf{Materials:} Discarded snakeskin or loose thread.\\
\textbf{Effect:} Gain +1 die to resist an ongoing Condition this scene, or reroll one \textbf{1} on an escape/evasion. On success, you may declare the Condition \emph{shed} and create a 2-segment \emph{Transformation Residue} clock to ignore a similar effect later.\\
\textbf{Invoke:} 1 action; mark +1 Obligation.\\
\textbf{Push It:} Also ignore one minor movement penalty; leave a token of the old self that can be traced (mark 1~SB \emph{Diamonds}).\\
\emph{Requires: Familiar \ (\textit{Invoke:} 1 Boon).}

\paragraph*{Rite of the Subtle Shift (Low, 5 XP)} \emph{Scene; Self; No.}
\textbf{Materials:} Palmed trinket passed hand-to-hand.\\
\textbf{Effect:} Fluid demeanor: +1 die to \emph{Deceive} to pass as a nearby class/profession, or +1 Effect to blend into a new crowd/site. Create a 4-segment \emph{Blended Identity} clock to downgrade one social complication.\\
\textbf{Invoke:} 1 action; mark +1 Obligation.\\
\textbf{Push It:} Bypass one minor identity check; you must maintain the false role to scene end (generate 1~SB \emph{Hearts} if challenged).\\
\emph{Requires: Familiar \ (\textit{Invoke:} 1 Boon).}

\paragraph{Rite of the Shedding \textnormal{[TRANSFORM]} (Standard, 8 XP)} \emph{Scene; Self; No.}
\textbf{Materials:} Full change of clothing and an adopted mannerism.\\
\textbf{Effect:} +2 dice to resist one named ongoing Condition; once/session declare a minor physical contingency retroactively ("I packed the tool," "I took that step earlier"). Create a 6-segment \emph{Shed Identity} clock.\\
\textbf{Push It:} Clear a temporary identity-based Minor Condition; your former identity stirs in the fiction (ally, rival, or witness appears).\\
\emph{Requires: Familiar + Codex \ (\textit{Invoke:} 1 Boon).}

\paragraph{Rite of the Forked Tongue \textnormal{[BIND]} (Standard, 7 XP)} \emph{Scene; Self/Near; No.}
\textbf{Materials:} A harmless lie told to a mirror.\\
\textbf{Effect:} Ambiguous persuasion: when you \emph{Sway} or \emph{Command}, a success may generate \emph{Diamonds} (leverage) instead of SB. Targets of deception must test (Wits+Insight DV~3) or suffer -1 die to future interactions with you this scene. Create a 4-segment \emph{Verbal Venom} clock.\\
\textbf{Push It:} One carefully worded lie this scene is treated as true; the displaced truth seeks return (mark 1~SB \emph{Hearts}).\\
\emph{Requires: Familiar + Codex \ (\textit{Invoke:} 1 Boon).}

\paragraph{Rite of Complete Metamorphosis \textnormal{[TRANSFORM][WARD]} (High, 13 XP)} \emph{Scene; Self; No.}
\textbf{Materials:} Identity kit (garb, voice, tokens) and a serpent's shed skin.\\
\textbf{Effect:} Full appearance/voice change. Begin \emph{Controlled} on \emph{Deceive}/\emph{Stealth}; once/scene declare a minor contingency retroactively. You are [\emph{WARD}]ed against recognition by former acquaintances. Create an 8-segment \emph{New Identity} clock.\\
\textbf{Push It:} Spoof scent/biometric once; your original identity partially unmoors and acts independently (mark 2~SB \emph{Spades/Hearts}).\\
\emph{Requires: Familiar + Codex + Tier III \ (\textit{Invoke:} \textbf{2 Boons}).}\\
\emph{Obligation:} 7 segments.

\paragraph{Rite of the Cast-Off History \textnormal{[UNWARD][CURSE]} (High, 14 XP)} \emph{Extended; Self; No.}
\textbf{Materials:} Burning or defacing mundane records of the old life.\\
\textbf{Effect:} Upon completion, common records and casual memories of that identity become unreliable; trackers via that identity suffer -2 dice (magic and intimates still apply). Gain +2 dice to rolls with the new identity. Create a 6-segment \emph{Erased Past} clock.\\
\textbf{Push It:} A plausible "death" occurs for the old identity; one intimate senses deception but cannot prove it (mark 1~SB \emph{Clubs}).\\
\emph{Requires: Familiar + Codex + Tier III \ (\textit{Invoke:} \textbf{2 Boons}).}\\
\emph{Obligation:} 7 segments.

\subsection*{Isoka's Corruption Table}
\label{sec:isoka-corruption}

\begin{longtable}{>{\raggedright\arraybackslash}p{1cm} p{5cm} p{5cm}}
\toprule
\textbf{Tier} & \textbf{Benefit} & \textbf{Cost / Quirk} \\
\midrule
1 & Serpent's Gaze: +1 die on Intimidation or social rolls using menace. & Cold-Blooded: -1 die on Empathy or warmth-based Persuasion. \\
\midrule
2 & Shed the Old Skin: Once per session, negate one Condition (Fear, Fatigue, Poison) by discarding it like molted skin. & Mark of Scales: Faint reptilian patches visible, causing +1 SB in social encounters if noticed. \\
\midrule
3 & Venomous Strike: Bite or kiss may inflict Poison (DV 2). & Hungry Coil: Must consume raw meat or eggs weekly, or mark 1 Fatigue. \\
\midrule
4 & Serpentine Grace: +2 dice to Stealth or Evasion rolls, ignore minor movement penalties. & Slitted Eyes: Bright light imposes -1 die on Notice rolls. \\
\midrule
5 & Hypnotic Sway: Once per scene, roll Presence + Lore vs. Resolve (DV 3) to mesmerize a single target. & Forked Tongue: -1 die on Deception when attempting warmth or sincerity. \\
\midrule
6+ & Ascendant Form: Partial serpent-body; immune to mundane poison, +2 dice on Body rolls. & Monstrous Aspect: Cannot easily hide your nature; every session begins with 1 SB that may be compelled by the GM. \\
\bottomrule
\end{longtable}
# % --- Patron: The Carrion-King (Decay, Renewal & Transformation) ---

\subsubsection{The Carrion-King (Decay, Renewal \& Transformation)}
\textit{Lore.} The Carrion-King is the master of endings that become beginnings. He does not destroy, but transforms—turning death into new life, decay into opportunity, and endings into fresh starts. His followers are harvesters of potential, seeing in every fall the seeds of future growth.

\begin{quote}
What crumbles feeds what grows. What dies becomes the soil of tomorrow's triumph.
\end{quote}

\paragraph*{Rite of Consuming Rot (Low, 5 XP)} \emph{Instant; Touch; Yes (decay only).}
\textbf{Materials:} Organic matter in early stages of decay. \\
\textbf{Effect:} Accelerate natural decay to weaken or destroy: +2 Effect to \emph{Break/Sabotage} on organic materials (ropes, leather, wood). Gain 1 Boon if the decay creates an opportunity for you or allies. \\
\textbf{Invoke:} 1 action; mark +1 Obligation. \\
\textbf{Push It:} Spread decay to similar materials in Close range; mark 1 SB (Clubs) as the rot becomes noticeable. \\
\emph{Requires: Familiar \ (\textit{Invoke:} 1 Boon).}

\paragraph*{Rite of the Harvested End (Low, 4 XP)} \emph{Scene; Touch; No.}
\textbf{Materials:} The remains of a recently ended thing (burnt letter, wilted flower, shattered glass). \\
\textbf{Effect:} Extract value from endings: from a defeated enemy, gain +1 die to next action; from a failed plan, re-roll one 1 on your next roll; from a broken item, gain 1 SB to spend immediately. \\
\textbf{Invoke:} 1 action; mark +1 Obligation. \\
\textbf{Push It:} Harvest additional value but mark Fatigue 1 from dwelling on endings. \\
\emph{Requires: Familiar \ (\textit{Invoke:} 1 Boon).}

\paragraph{Rite of the Fertile Death (Standard, 8 XP)} \emph{Scene; Zone; No.}
\textbf{Materials:} Ashes, compost, or the remains of anything that once lived. \\
\textbf{Effect:} Transform death into growth: create beneficial terrain (cover, concealment, or advantageous positioning) OR grant allies +1 die to healing/recovery rolls. Choose one effect per scene. \\
\textbf{Push It:} Both effects apply but attract unwanted attention (vermin, scavengers, or curious onlookers). \\
\emph{Requires: Familiar + Codex \ (\textit{Invoke:} 1 Boon).}

\paragraph{Rite of the Transformed Spirit (Standard, 7 XP)} \emph{Instant; Near; No.}
\textbf{Materials:} A token from a deceased being (hair, nail, written name). \\
\textbf{Effect:} Channel the essence of what was: gain one skill die reflecting the deceased's expertise for one scene OR ask one question about their knowledge/abilities. \\
\textbf{Push It:} The spirit's influence lingers - gain permanent insight (+1 die specialty) but suffer occasional possession-like effects (GM discretion). \\
\emph{Requires: Familiar + Codex \ (\textit{Invoke:} 1 Boon).}

\paragraph{Rite of the Great Consumption (High, 13 XP)} \emph{Scene; Zone; No.}
\textbf{Materials:} A significant amount of organic matter (corpse, fallen tree, collapsed building). \\
\textbf{Effect:} Transform a large area through decay and renewal: choose two - create difficult terrain that favors you, summon Cap 3 swarm of scavengers as temporary allies, or generate valuable reagents worth 2 XP. \\
\textbf{Push It:} All three effects occur but start a 6-segment \textbf{Ecosystem Disruption} clock that will cause problems later. \\
\emph{Requires: Familiar + Codex + Tier III \ (\textit{Invoke:} \textbf{2 Boons}).} \\
\emph{Obligation:} 7 segments.

\paragraph{Rite of the Eternal Cycle (High, 14 XP)} \emph{Extended; Touch; No.}
\textbf{Materials:} The complete remains of something significant that has ended. \\
\textbf{Effect:} Complete a transformation cycle: destroy one major asset/enemy/obstacle and create something new of equal or greater value. GM and player collaborate to define the transformation. \\
\textbf{Push It:} The transformation is immediate and spectacular but creates a 6-segment \textbf{Cycle Debt} clock - the King will demand another significant ending soon. \\
\emph{Requires: Familiar + Codex + Tier III \ (\textit{Invoke:} \textbf{2 Boons}).} \\
\emph{Obligation:} 7 segments.

\subsection*{Carrion-King's Corruption Table}
\label{sec:carrion-king-corruption}

\begin{longtable}{>{\raggedright\arraybackslash}p{1cm} p{5cm} p{5cm}}
\toprule
\textbf{Tier} & \textbf{Benefit} & \textbf{Cost / Quirk} \\
\midrule
1 & Carrion's Insight: +1 die to Notice decay or hidden weaknesses in structures or beings. & Must inspect decay firsthand; suffer 1 Fatigue when exposed to fresh death or rot. \\
\midrule
2 & Deathward Sense: Once per session, detect the last living moment of a dead being within Close range. & Cannot lie about death you’ve witnessed; must correct falsehoods. \\
\midrule
3 & Rotblood Resilience: Gain +1 die to resist disease and poison. & Immune system adapts slowly; each new disease/poison requires 1 Fatigue to resist. \\
\midrule
4 & Glean from Grief: Once per scene, gain +1 die after witnessing a significant loss or defeat. & Compelled to linger at scenes of death; must spend one beat observing or risk 1 SB (Clubs). \\
\midrule
5 & Cycle's Whisper: You can sense the “next ending” in any process—ask the Keeper one question about how a situation will collapse or conclude. & Must speak the truth about what you see, even if it harms your position. \\
\midrule
6+ & Eternal Bloom: Once per session, declare a “death that births life.” Sacrifice an asset or ally to create something new of equal or greater value. & Mark +2 Obligation when using this power. \\
\bottomrule
\end{longtable}
% --- Patron: The Gallow's Bell (Justice & Judgment) ---

\subsubsection{The Gallow's Bell (Justice \& Judgment)}
\textit{Lore.} The Bell does not rage; it tolls. Cold and impartial, it measures all accounts in time. Its keepers are silent arbiters who weigh deeds against consequence, not out of anger but out of inevitability. To call upon the Bell is to bind oneself to the gravity of truth, where even silence is judged, and every oath leaves a resonance in iron.

\begin{quote}
What is broken must be mended, what is owed must be paid. The Bell remembers all reckonings.
\end{quote}

\paragraph*{Rite of the Measured Debt (Low, 4 XP)} \emph{Scene; Near; No.}\\
\textbf{Materials:} A pair of scales balanced with tokens from both sides.\\
\textbf{Effect:} Establish a temporary accord. Both parties suffer -1 die if they break it first. You gain +1 die to enforce compliance.\\
\textbf{Push It:} The accord is mystically weighted; breach inflicts 1~SB (Hearts).\\
\emph{Requires: Familiar.}

\paragraph*{Rite of the Weighed Heart (Low, 5 XP)} \emph{Scene; Near; No.}\\
\textbf{Materials:} A small brass scale touched briefly to the chest.\\
\textbf{Effect:} Sense if the target acts against their nature or oath. Gain +1 die when pressing them.\\
\textbf{Push It:} Target must test Resolve (DV~3) or disclose a hidden conflict.\\
\emph{Requires: Familiar.}

\paragraph{Rite of the Balanced Scales (Standard, 8 XP)} \emph{Scene; Near; No.}\\
\textbf{Materials:} A set of scales inscribed with runes of parity.\\
\textbf{Effect:} Exchange a burden between two willing parties (Harm for Fatigue, Debt for Favor, etc.). Both gain +1 die to cooperate.\\
\textbf{Push It:} May compel an unwilling exchange with contested Command + Wits.\\
\emph{Requires: Familiar + Codex.}

\paragraph{Rite of the Judge’s Eye (Standard, 7 XP)} \emph{Scene; Self; No.}\\
\textbf{Materials:} A black hood worn in silence for one minute.\\
\textbf{Effect:} Detect lies within Near range; +2 dice to Insight. Liars suffer -1 die to maintain their falsehood.\\
\textbf{Push It:} All deception is laid bare for the scene, but mark Exposure +1.\\
\emph{Requires: Familiar + Codex.}

\paragraph{Rite of the Final Reckoning (High, 13 XP)} \emph{Scene; Zone; No.}\\
\textbf{Materials:} A circle of iron bells, each etched with nameless runes.\\
\textbf{Effect:} The Bell tolls through you. All present feel compelled to name a debt or wrongdoing. Those who lie suffer Harm~2; those who speak true gain +2 dice to persuasion for the scene.\\
\textbf{Push It:} The Reckoning manifests as spectral echoes of past wrongs—liars automatically suffer narrative punishment (Keeper decides).\\
\emph{Requires: Familiar + Codex + Tier III.}\\
\emph{Obligation:} 7 segments.

\paragraph{Rite of the Great Adjudication (High, 14 XP)} \emph{Extended; Zone; No.}\\
\textbf{Materials:} A consecrated gavel or a great bell struck three times.\\
\textbf{Effect:} Convene an unseen tribunal. Shadows of former judges and wronged souls gather to preside. For the next session, disputes within the zone are judged formally: +2 dice to Command when speaking as arbiter, and honest testimony gains +1 die.\\
\textbf{Push It:} The tribunal’s verdict echoes beyond the zone, affecting one major conflict elsewhere. Mark 2~SB (Hearts) as higher powers of judgment take notice.\\
\emph{Requires: Familiar + Codex + Tier III.}\\
\emph{Obligation:} 8 segments.

\subsection*{Gallow’s Bell Corruption Table}
\label{sec:gallows-bell-corruption}

\begin{longtable}{>{\raggedright\arraybackslash}p{1cm} p{5cm} p{5cm}}
\toprule
\textbf{Tier} & \textbf{Benefit} & \textbf{Cost / Quirk} \\
\midrule
1 & Judge’s Intuition: +1 die to Insight when weighing truth. & Must point out falsehoods when noticed, regardless of tact. \\
\midrule
2 & Quiet Authority: Once/scene, treat a failed Command as success; mark 1~SB (Hearts). & Cannot remain neutral in disputes; indecision costs 1 Fatigue. \\
\midrule
3 & Scales of Balance: Once/session, enforce an exchange of burdens. & Compelled toward fairness even when it hinders you. \\
\midrule
4 & Bell’s Resonance: +2 dice when calling for judgment or demanding restitution. & Suffer 1 Fatigue if wrongdoing is ignored. \\
\midrule
5 & Reckoner’s Call: Once/session, declare a “reckoning moment”—truth must surface or consequence falls. & Cannot ignore pleas for justice without marking 1~SB (Spades). \\
\midrule
6+ & Final Arbiter: Once/session, render an absolute decree; all must obey or suffer consequence. & Mark +2 Obligation; the Bell demands you bear the weight of enforcement. \\
\bottomrule
\end{longtable}
\section{The Traveler --- Guide of Ways}
\label{patron:traveler}

\subsection*{Lore}
\index{Patrons!The Traveler}%
The Traveler is the eternal guide of the road, guardian of those who walk the paths between what is and what might be. Among the Fhara caravans and Kuvani traders along the Way of Silk, the Traveler is invoked at every crossroads, honored with small offerings at each waypoint, and consulted before every major journey.  

The Traveler is not merely one who shows the way---they \emph{are} the way, existing in the pause between steps and in the choice of which path to take when roads fork. Every journey is both physical and spiritual; to move from one place to another is to transform, and the road itself becomes a teacher.  

\begin{quote}
``One foot in a promise, and the road will meet you halfway. But break your word to the way, and the way will break you.''
\end{quote}

\subsection*{Patron's Gift: Road's Blessing}
Once per scene as an action (cost: 1 Boon; requires Thiasos), touch an item or person to imbue it until the end of the scene. The target gains +1 die and +1 Effect when used for movement, navigation, or pathfinding.  

\textbf{Push It:} Extend for one additional scene by marking +1 Obligation. The road’s attention is noticed by other travelers.

\subsection*{Low Rites}
\paragraph{Rite of Road-Sense (Low)}  
\emph{Duration: Scene; Range: Self. Materials: Road-nail or waystone pebble.}  
Unerringly pick the fastest safe route in Near/Far. Gain +1 die to avoid ambushes or delays. Create a 4-segment \emph{Path Memory} clock to ignore difficult terrain once.  
\textbf{Invoke:} 1 action; mark +1 Obligation.  
\textbf{Push It:} Spot one hidden bypass, but generate 1 SB (Clubs).

\paragraph{Rite of the Traveler's Boon (Low)}  
\emph{Duration: Scene; Range: Self/Ally. Materials: Thread tied at the wrist.}  
Ignore one level of difficult terrain or bureaucracy; +1 Effect to travel or escape checks. If shared, create a 2-segment \emph{Shared Journey} bond with an ally.  
\textbf{Invoke:} 1 action; mark +1 Obligation.  
\textbf{Push It:} Extend to one more ally; mark 1 SB (Diamonds).

\subsection*{Standard Rites}
\paragraph{Rite of the Waymark [PASSAGE] (Standard)}  
\emph{Duration: Scene; Range: Near. Materials: Chalk mark or small cairn.}  
Declare a lane as permitted/easy. Allies gain improved Position/Effect or ignore one obstacle. Create a 6-segment \emph{Marked Path} clock.  
\textbf{Invoke:} 1 action; mark +1 Obligation.  
\textbf{Push It:} Lane persists between scenes; first enemy to exploit it forces 1 SB (Spades).

\paragraph{Rite of the Bridge Between [TRANSPORT] (Standard)}  
\emph{Duration: Instant; Range: Near. Materials: Two pinches of road-dust clapped.}  
Relocate a willing target within Far along a visible/named route. Unwilling targets may resist (Body+Resolve DV~3). Create a 4-segment \emph{Pathway Established} clock.  
\textbf{Invoke:} 1 action; mark +1 Obligation.  
\textbf{Push It:} Carry one extra ally or bundle; arrivals off-balance.

\subsection*{High Rites}
\paragraph{Rite of the Crown of Crossings [WARD][COMMAND] (High)}  
\emph{Duration: Scene; Range: Zone. Materials: Brass compass missing its needle.}  
Call the Road: allies gain +1 die to move/evade; pursuers suffer --1 die. Once, declare ``the long way is short'' to finish a travel clock segment for free. Enemies must test (Wits+Command DV~4) or suffer --1 die to movement.  
\textbf{Invoke:} 1 action; mark +2 Obligation.  
\textbf{Push It:} Seal a hostile route briefly; generate 2 SB (Clubs/Diamonds).

\paragraph{Rite of the Wayfarer's Covenant [OATH][FORTIFY] (High)}  
\emph{Duration: Extended; Range: Near. Materials: Waystones from multiple regions.}  
Bind present parties to safe passage. While honored: +2 Effect on travel, reroll one failed travel roll per scene. Breaking oath inflicts Harm~1 (Fatigue) and marks breaker as \emph{Oathbreaker of the Road} (–2 dice to travel rolls).  
\textbf{Invoke:} Extended ritual; mark +3 Obligation.  
\textbf{Push It:} Breaking the covenant inflicts Harm~2 (Stress) and attracts hostile Wayward spirits.

\subsection*{Obligation Progression}
Starts at 5 for Tier II characters, scaling with tier.

\paragraph{Obligation 8+} The Traveler demands you guide someone/something to a destination. Refusal: all routes fail, terrain counts as difficult, –2 dice to navigation.  

\paragraph{Obligation 10+} The road claims you. You cannot remain still: –2 dice in stationary contexts, Body+Resolve (DV~3) required to endure cities or confinement.

\subsection*{Persistent Condition: Child of the Road}
Gain +2 dice to movement, navigation, and pathfinding. Suffer –1 die to rolls requiring prolonged settlement or stillness. Narrative: restless blood, exceptional in travel, uneasy when stationary.

\subsection*{Rivalries}
\begin{itemize}
\item \textbf{The Sealed Gate:} Antagonism---the Traveler opens paths, the Gate closes them.  
\item \textbf{The Pale Shepherd:} Subtle tension---ways between places vs.\ ways between states.  
\item \textbf{The Ninth Rim:} Opposition---the Traveler connects, the Rim erases.  
\end{itemize}

\subsection*{Connection to the Way of Silk}
The Traveler is most honored along the Way of Silk:
\begin{itemize}
\item Waystone offerings at crossroads  
\item Path blessing rites before departure  
\item Road-sharing pacts between caravans  
\item Wayfarer's covenants ensuring aid on dangerous routes  
\end{itemize}
Symbols: the \emph{Crown of Crossings} (a traveler's hood/hat) and the compass without needle (truth that the path is not always obvious).

\subsection*{Playtest Scenario: The Caravan's Last Journey}
A great trading caravan attempts its final crossing of the Way of Silk. Bandits, storms, and ancient curses threaten its survival. The party must protect it.  

\begin{itemize}
\item Use \emph{Rite of Road-Sense} to bypass ambushes.  
\item Use \emph{Rite of the Waymark} to secure campsites and routes.  
\item Invoke \emph{Rite of the Crown of Crossings} against a supernatural storm.  
\item Perform \emph{Rite of the Wayfarer's Covenant} to unify caravan factions.  
\item Employ \emph{Rite of the Bridge Between} to relocate isolated supplies.  
\end{itemize}

Resolution: deliver the caravan to safety or take a profane shortcut---incurring the Traveler’s wrath.

% --- Patron: Mykkiel, Arbiter of the Writ (Judgment & Writ) ---
\subsubsection{Mykkiel, Arbiter of the Writ (Judgment \& Writ)}
\textit{Lore.} Mykkiel weighs speech against deed and seals verdicts in cold iron.

\begin{quote}
Name the charge. Name the terms. Then sign where you’ll bleed if you’re wrong.
\end{quote}

\paragraph{Stamp of Authority (Low, 4 XP)} \emph{Action; Near; Yes (doc/object).}
\textbf{Materials:} Cold-iron seal or writ-tag.\\
\textbf{Effect:} Visible mark of authority. \textbf{+1 die} to \emph{Command/Persuade} that asserts lawful order/claim.\\
\textbf{Push It:} Brief hush (one beat) among hecklers; mark \emph{Exposure +1}.\\
\emph{Requires: Familiar \ (\textit{Invoke:} 1 Boon).}

\paragraph{Rite of Proper Notice (Low, 5 XP)} \emph{Scene; Near; No.}
\textbf{Materials:} Writ-string tied and snapped.\\
\textbf{Effect:} Name a \emph{lawful venue} (dais, doorway, wagon). First hostile act there suffers \(-1\) die.\\
\textbf{Push It:} Name a \emph{protected act} (parley, surrender, testimony): \textbf{+1 effect} in the venue; breaking custom generates \textbf{1 SB (Hearts)}.\\
\emph{Requires: Familiar \ (\textit{Invoke:} 1 Boon).}

\paragraph{Writ of Compliance \textnormal{[COMMAND]} (Standard, 8 XP)} \emph{Action; Near; No.}
\textbf{Materials:} Red cord knotted while speaking the order.\\
\textbf{Effect:} Immediate command (“Stand down,” “Drop it,” “Open”). Target must comply now or suffer a Keeper-stated cost. DV by fiction; elites may test Resolve.\\
\textbf{Push It:} On compliance, impose \(-1\) die on target’s next aggressive act this scene.\\
\emph{Requires: Familiar + Codex \ (\textit{Invoke:} 1 Boon).}

\paragraph{Rite of the Speaking Seal (Standard, 7 XP)} \emph{Scene; Near; No.}
\textbf{Materials:} Wax seal impressed over a name/sigil.\\
\textbf{Effect:} Sanctify a statement (truce, custody, claim). Contradicting it suffers \(-1\) die; you gain \textbf{+1 die} to enforce it.\\
\textbf{Push It:} Once, ask who here intends breach; Keeper gives a strong clue or direct name.\\
\emph{Requires: Familiar + Codex \ (\textit{Invoke:} 1 Boon).}

\paragraph{Oath Irons \textnormal{[OATH]} (High, 11 XP)} \emph{Scene; Near; No.}
\textbf{Materials:} Two iron pins warmed in flame, touched to wrists, then quenched.\\
\textbf{Effect:} Bind two parties to a bounded term. Breach \emph{forces 2 SB} and brands a faint iron-mark until amends.\\
\textbf{Push It:} Extend to a small circle (up to four); each chooses one narrow exception (Keeper approves). Exploiting it generates \textbf{1 SB (Diamonds)}.\\
\emph{Requires: Familiar + Codex + Tier III \ (\textit{Invoke:} \textbf{2 Boons}).}\\
\emph{Obligation:} 7 segments.

% --- Patron: Oath of Flame & Light (Dawn & Vows) ---

\subsection{Oath of Flame \& Light (Dawn \& Vows)}
\textit{Lore.} The Oath of Flame \& Light is no patron of half-measures. Its fire names, binds, and burns—demanding that those who swear within its radiance stand openly, speak truly, and pay the cost of keeping their word. At dawn altars, the sworn kindle sparks of consecrated fire; in battle, they blaze as torches that hold back the night. To follow this Oath is to live in public truth, with no shadow to hide in and no retreat from the vow once spoken.

The Oath appears in many guises across cultures:  
\begin{itemize}
  \item \textbf{The Everflame} — the unquenchable fire of promise, carried from shrine to shrine by wandering priests.  
  \item \textbf{Adur}, the Light of Aeler — worshiped as the sun’s church-deity, whose flame consecrates oaths and consumes falsehood.  
  \item \textbf{Adar}, the Vilikari Dawnfire — invoked as a war-god of vows and vengeance, whose rising light marks those who break faith.  
  \item Others know the Oath only as \textit{the Flame} or \textit{the Watchfire}, a nameless fire that still demands truth and punishes betrayal.  
\end{itemize}

Wherever it manifests, the Oath of Flame \& Light remains the same unyielding power: to swear beneath its dawn is to bind one’s self to a truth that cannot be hidden, forgotten, or undone.

\begin{quote}
``Swear in the light. Keep it, or the light will keep \emph{you}.''
\end{quote}

\paragraph*{Domain Focus}
\begin{itemize}
\item \textbf{Sacred Vows:} Oath-keeping, truth-speaking, binding promises
\item \textbf{Divine Justice:} Retribution, smiting evil, protecting the innocent
\item \textbf{Radiant Power:} Healing light, purification, revealing truth
\item \textbf{Dawn's Hope:} Renewal, protection, driving back darkness
\end{itemize}

\paragraph*{Rite of Kindle Vow (Low, 4 XP)} 
\emph{Action; Self/Ally; Standard Push}\\
\textbf{Materials:} Glass ampoule of consecrated flame.\\
\textbf{Effect:} Declare a short vow for this scene (e.g., "protect the villagers," "speak only truth"). Bearer gains +1 die to actions fulfilling it.\\
\textbf{Push It:} First hesitation forces 1 SB (Hearts) on the bearer.\\
\emph{Requires: Familiar.}

\paragraph*{Rite of Lay on Hands [CLEANSE][HEAL] (Low, 5 XP)} 
\emph{Instant; Touch; Standard Push}\\
\textbf{Materials:} Bare palm, whispered vow.\\
\textbf{Effect:} Cleanse affliction, downgrade Harm by 1, or remove 1 Fatigue. For curses/poisons, test Spirit + Resolve (DV by fiction).\\
\textbf{Push It:} Target gains +1 die to next Resist this scene; you mark Exposure +1.

\paragraph*{Rite of Sunlit Parley (Standard, 8 XP)} 
\emph{Scene; Zone; Standard Push}\\
\textbf{Materials:} Vow-ring engraved with sunrise.\\
\textbf{Effect:} Establish terms in open light. Honest persuasion gains +1 die; deceit suffers -1 die. 6-segment \emph{Parley Accord} clock.\\
\textbf{Push It:} Demand one public answer; evasion forces 1 SB (Hearts) on evader.

\paragraph*{Rite of Radiant Smite [FOLLOW-UP] (Standard, 9 XP)} 
\emph{Action; Self; Standard Push}\\
\textbf{Materials:} Consecrated spark on weapon/badge.\\
\textbf{Effect:} Next melee strike flares with dawnfire. Upgrade Effect by one step, add +1 Burn Harm or force 1 SB (Spades).\\
\textit{Special:} Against undead/oath-breakers/outsiders: oath-breakers suffer -1 die, outsiders gain +1 Exit Tally.\\
\textbf{Push It:} On hit, burst drives back Close enemies (worse Position); +1 Obligation.

\paragraph*{Rite of Purge the Shadow [REVEAL][DISPEL] (Standard, 10 XP)} 
\emph{Instant; Near; Standard Push}\\
\textbf{Materials:} Shattered consecrated spark.\\
\textbf{Effect:} Reveal illusions and suppress one ongoing glamour/curse in Near. Creates 4-segment \emph{Purity's Light} clock.\\
\textbf{Push It:} Brand source with visible tell for this arc; mark 1 SB (Diamonds).

\paragraph*{Rite of Covenant Blaze [OATH][FORTIFY] (High, 13 XP)} 
\emph{Scene; Zone; High Push}\\
\textbf{Materials:} Brazier lit while three names are spoken.\\
\textbf{Effect:} Sworn within are haloed: +1 die to keep oath; aggressors suffer -1 die if violating terms. Oath-breakers suffer 2 SB (Hearts/Spades) and Harm 1 (Burn).\\
\textbf{Push It:} Sanctifies threshold with one beat of [WARD] against oath-breakers; +2 Obligation.\\
\textbf{Obligation:} 7 segments base.

\subsection*{Oath of Flame \& Light Corruption Manifestations}
\label{sec:oath-flame-light-corruption}

\begin{longtable}{>{\raggedright}p{2cm} p{6cm} p{6cm}}
\toprule
\textbf{Level} & \textbf{Benefit} & \textbf{Cost / Quirk} \\
\midrule
1 & \textbf{Oathbound Strength:} +1 die when upholding vows or defending innocents. & \textbf{Rigid Honor:} Must uphold vows even when disadvantageous; -1 die to flexible actions. \\
\midrule
2 & \textbf{Radiant Sight:} Once per scene, +2 dice to detect lies or corruption. & \textbf{Blinding Truth:} -1 die to subtlety or deception attempts. \\
\midrule
3 & \textbf{Holy Flame:} +1 die to melee vs undead, outsiders, or oath-breakers. & \textbf{Burden of Light:} 1 Fatigue when concealing identity or working in darkness. \\
\midrule
4 & \textbf{Unwavering Resolve:} Once per session, treat failed Resolve/Command as success (mark 1 SB). & \textbf{Absolutist Stance:} -1 die in morally ambiguous situations. \\
\midrule
5 & \textbf{Dawn's Benediction:} Once per session, heal allies in Near of 1 Fatigue and minor Conditions. & \textbf{Beacon's Call:} Your aura reveals you; enemies seeking you gain +1 die. \\
\midrule
6+ & \textbf{Avatar of the Oath:} Once per session, become living covenant; +2 dice to protection/justice actions. & \textbf{Radiance's Price:} +2 Obligation; breaking any vow inflicts Harm 2 (Burn). \\
\bottomrule
\end{longtable}

\paragraph*{Playstyle Notes}
The Oath of Flame \& Light excels as a "paladin" patron, rewarding vow-keeping and truth-speaking with potent defensive and smiting abilities. Followers become living embodiments of their oaths, shining beacons against deception and darkness. The corruption progression naturally leads toward becoming an uncompromising force for justice, though potentially at the cost of flexibility and subtlety. Ideal for characters who embody conviction, protection, and radiant power.

\subsection{Umande, Dew-Mother of the Living Canopy (Growth \& Shelter)}
\begin{quote}\itshape
“Walk softly and you will hear the roots speaking of where to stand, when to bend, and whom to shelter.”
\end{quote}

\paragraph{Lore.}
Across the humid belts from the baobab-studded savannas to banyan-latticed monsoon forests, travelers whisper of \textbf{Umande}, the breath on the leaves at dawn. She is the pact of shade and water: a patient patron who measures worth in what you protect and what you prune. Her ministers are the unseen—lichen, mycelium, creepers—knitting harm back into wholeness, yet strangling cruelty where it clings too long. Umande favors wardens, foragers, midwives, and wayfarers who leave places better than they found them.

\paragraph{Patron’s Gift — Verdant Bond.}
Your sworn implement (staff, spear, billhook, kukri, or similar) knots itself in living vine and hard sap.
\begin{itemize}
  \item \textbf{Enchanted Weapon:} Your sworn implement counts as an enchanted melee weapon \textbf{+1 Melee}.
  \item \textbf{Thematic Skill:} You gain \textbf{+1 to Survival} while carrying any living sprig, seed, or leaf from your sanctum or a protected grove.
\end{itemize}

% =========================
% RITES (follow Ikasha style)
% =========================

\paragraph{Dawn’s Condensation (Low, 4 XP)} \emph{Scene; Touch/Self; Yes (resist only).}\\
\textbf{Materials:} A leaf cupping a few drops of clean water (or a wetted cloth).\\
\textbf{Effect:} Wick dew into flesh. Remove one minor Condition related to fatigue, heat, or dehydration \emph{or} grant +1 die to the next Survival or Medicine test this scene.\\
\textbf{Push It:} Also clear 1 SB of environmental stress (thirst/heat) for one ally within reach.\\
\textbf{Backlash (Life/Earth):} Skin chills and prunes; you suffer --1 die to Presence tests until you warm up.

\paragraph{Root-Weave Aegis (Low, 4 XP)} \emph{Scene; Near (close range); No.}\\
\textbf{Materials:} A twist of fiber, grass cord, or mycelial thread.\\
\textbf{Effect:} Sprout a knee-high barricade of roots and woven stems in a 2m arc. Gain \textbf{Cover} against ranged attacks and upgrade Position by one step for holds/defense while adjacent.\\
\textbf{Push It:} The first enemy crossing the weave is \textbf{Hindered} (movement penalty) this exchange.\\
\textbf{Backlash (Earth/Water):} Soil turns slick; you suffer --1 die on your next movement check this scene.

\paragraph{Whispering Mycelium (Standard, 8 XP)} \emph{Scene; Far (site-scale); Yes.}\\
\textbf{Materials:} A coin-sized fungus cap or a pinch of sporeprint.\\
\textbf{Effect:} Lay fingertips to soil or wood and commune with the hyphal web. Ask \textbf{two} questions about recent passage, injury, or decay within a circa 100m radius (GM answers truthfully but from the web’s perspective). Gain \textbf{advantage} on one track, forage, or concealment action tied to that intel this scene.\\
\textbf{Push It:} Ask a third question, or extend to 300m.\\
\textbf{Backlash (Life):} Your senses blur with rot; --1 die to non-Survival observation until you take a breath and steady yourself.

\paragraph{Green Path Unfurling (Standard, 8 XP)} \emph{Scene; Line of sight; Yes.}\\
\textbf{Materials:} A fresh tendril or vine.\\
\textbf{Effect:} Vegetation parts and cushions your passage. For the scene, you and up to \textbf{two} allies ignore minor difficult terrain in natural growth and gain +1 die to Stealth \emph{or} Athletics when moving through foliage.\\
\textbf{Push It:} Also \textbf{confound pursuit}: the first attempt to track you through this area is at --1 die.\\
\textbf{Backlash (Earth):} You leave telltale chlorophyll stains; tracking you in open ground gains +1 die until the next rest.

\paragraph{Rain-Caller’s Oath (Major, 12 XP)} \emph{Scene; Region (sky above); Yes.}\\
\textbf{Materials:} A hollow seedpod with three droplets of clean water inside.\\
\textbf{Effect:} Petition Umande to gather clouds or break them. Choose one:
\begin{itemize}
  \item \textbf{Summon Gentle Rains:} Over the next hour, steady rain falls; fires weaken, dust settles, and heat penalties are suppressed. Gain +1 die to Survival/Medicine to aid many.
  \item \textbf{Part the Canopy of Storms:} Winds ease and rainfall slackens to a workable drizzle; ranged penalties from heavy weather lessen by one step.
\end{itemize}
\textbf{Push It:} Shape the rain’s \emph{band}: center it on a location you can name within sight, sparing adjacent fields or focusing on a blaze.\\
\textbf{Backlash (Water/Life):} The sky takes its price; you suffer a \textbf{Drained} minor Condition (until warm food/rest), and plants nearby leach a hint of color.

\paragraph{Banyan’s Embrace (Major, 12 XP)} \emph{Instant; Near; Yes.}\\
\textbf{Materials:} A looped root or braided cord.\\
\textbf{Effect:} Living cords erupt to \textbf{Grapple} a single foe or to \textbf{Brace} a structure (choose one):
\begin{itemize}
  \item \textbf{Grapple:} Target must beat DV 4 (Body or Melee) to break free; on failure, they are Restrained until end of next exchange.
  \item \textbf{Brace:} A listing cart, palisade, or bridge segment holds; immediate collapse/damage is postponed; gain time to act.
\end{itemize}
\textbf{Push It:} Affect up to \textbf{two} adjacent targets/segments.\\
\textbf{Backlash (Earth):} Your limbs ache like old wood; --1 die to Melee checks until you stretch or rest.

\paragraph{Sanctum of the Green Court (Epic, 16 XP)} \emph{Ritual; Hours; Circle; Yes.}\\
\textbf{Materials:} Four living stakes (saplings) planted on the quarter points; a bowl of spring water; a woven crown of grass.\\
\textbf{Effect:} Consecrate a \textbf{Grove-Sanctum} (small campsite or shrine). While within, allies gain +1 die to \textbf{Survival} and \textbf{Medicine}; hostile entities of rot, blight, or open flame suffer --1 die to actions that harm living plants. Once per session within the sanctum, you may \textbf{negate} one minor environmental Complication (smoke, thirst, exposure).\\
\textbf{Push It:} Bind a \textbf{Vow}: name a creature, kin-group, or landmark under the Grove’s protection; the first direct harm against that Vow inside the sanctum immediately generates 1 SB for the GM \emph{and} grants you 1 Boon.\\
\textbf{Backlash (Life/Earth):} The grove remembers. If you break the Vow, Umande withholds dew: you cannot Push Umande rites until you atone (meaningful restoration or protection scene).

% Optional flavor epithets for different regions
\paragraph{Epithets.} In the eastern monsoon lands, she is \textit{She-Who-Hangs-The-Bridges} among banyan roots; on the savannas, \textit{Matron of Baobabs}; in mist forests, \textit{Mother-of-Dew} who writes omens on leaves at dawn.



%========================================
% appendix/patron-rivalries-expanded.tex
%========================================
% Include with: \input{appendix/patron-rivalries-expanded.tex}
% Requires: \usepackage{booktabs,longtable}

\section*{Patron Rivalries }
\label{app:patron-rivalries-expanded}

Use this matrix to quickly shade rulings. “Edge Loci” are environments or situations where one side tends to start a step better in Position or gains an Effect nudge (Keeper’s call). “Friction” are handy prompts for SB spends.

\renewcommand{\arraystretch}{1.15}
\setlength{\LTpre}{0pt}
\setlength{\LTpost}{0pt}

\begin{longtable}{@{}p{3.3cm}p{3.3cm}p{4.6cm}p{7.2cm}@{}}
\toprule
\textbf{Patron} & \textbf{Rival} & \textbf{Edge Loci} & \textbf{Friction \& Prompts (SB)} \\
\midrule
\endfirsthead

\toprule
\textbf{Patron} & \textbf{Rival} & \textbf{Edge Loci} & \textbf{Friction \& Prompts (SB)} \\
\midrule
\endhead

\bottomrule
\endfoot

Raéyn (Sea, Tides, Travel) & Khemesh (Abyssal Maw) &
Open water, coasts, shipping lanes, storms you can \emph{read}. &
SB: changing tides, shifting winds, a clear route opens \emph{but} a vow at sea is invoked; waymarks appear then vanish; safe harbor demands a price. \\

Khemesh (Abyssal Maw) & Raéyn (Sea, Tides, Travel) &
Trenches, lightless holds, flooded caverns, oppressive silence. &
SB: pressure crush, voices from the bilge, hull-groan clocks; lanterns dim; maps become untrustworthy; a crewman hears the trench call. \\

Sealed Gate (Boundaries, Closure) & The Traveler (Ways, Roads) &
Customs houses, oaths, locks, court thresholds. &
SB: writ checked, stamp demanded, wrong ledger; crossing inflicts a toll; shortcut collapses into lawful detour; [WARD] keys hum. \\

The Traveler (Ways, Roads) & Sealed Gate (Boundaries, Closure) &
Desire paths, smuggler tracks, wayshrines, liminal crossings. &
SB: desire line opens; escort looks away; a map’s marginalia proves true; the lock refuses a lawful key \emph{now}. \\

The Witness (Truth, Revelation) & Mab (Glamour, Courts) &
Depositions, confessionals, cold light, mirrored chambers. &
SB: mask slips; testimony contradicts a powerful courtier; illusions shed their seams; a polite scandal erupts. \\

Mab (Glamour, Courts) & The Witness (Truth, Revelation) &
Masques, salons, petty courts, festive oaths. &
SB: a favor called; a duel by slight; truth offends protocol; a boon granted if the mask stays on. \\

Ikasha (Shadow, Latent Potential) & The Witness (Truth, Revelation) &
Deep shade, empty rooms, places holding unrealized action. &
SB: hush worsens Position against scrutiny; a shadow remembers your step; a deferred answer comes due. \\

Mykkiel (Judgment, Writ) & Varnek Karn (Necromantic Archives) &
Courts martial, audit halls, sanctified ledgers. &
SB: the writ binds a restless dead; precedent rejected; a ledger page is missing; sentence invites a haunting. \\

Varnek Karn (Necromantic Archives) & Oath of Light \& Flame (Dawn, Vows) &
Ossuaries, plague pits, memorial crypts, last testaments. &
SB: bone answers, but asks payment; unfinished business drags PCs into an old feud; consecration threatens the archive. \\

Oath of Light \& Flame (Dawn, Vows) & Khemesh (Abyssal Maw) &
Sunrise rites, consecrated decks, sworn escorts. &
SB: dawn burns back the hush; a vow compels aid; the abyss recoils \emph{but} exacts a later omen. \\

Sacred Geometry (Order, Pattern) & The Traveler (Ways, Fortune) &
Survey markers, engineered ways, measured works. &
SB: pattern locks; measured route grants Position; “efficient path” clashes with a necessary detour; chance resists the grid. \\

Clockwork Monad (Iteration, Process) & The Traveler (Ways, Fortune) &
Workshops, drill yards, rehearsal spaces, routines. &
SB: repetition gifts a die \emph{but} lures complacency; a new route tempts; a jig breaks a jam or jams a break. \\

Nidhoggr (Dreaming Antiquity) & Sacred Geometry (Order, Pattern) &
Barrows, megaliths, fossil beds, dream-thresholds. &
SB: the land remembers; a measure erases an omen; echo of the past answers a present question—at a cost. \\
\end{longtable}

\paragraph{Quick Rulings.}
\begin{itemize}
  \item \textbf{Position Nudge:} In a home locus, start one step better; in a rival locus, one step worse.
  \item \textbf{Effect Shade:} Where a Patron dominates, consider an Effect bump; where opposed, consider Limited Effect unless paid for.
  \item \textbf{Symbol Interference (Invokers):} Carrying both sides’ Symbols increases narrative noise: first ritual each scene may mark +1 Obligation (Keeper’s call).
  \item \textbf{SB Color:} When spending SB in these matchups, prefer suits that fit: Hearts (social), Spades (harm/escalation), Clubs (material cost), Diamonds (numinous disturbance).
\end{itemize}

% =========================
% OBLIGATION OVERFLOW (RITES)
% =========================
\section{Obligation Overflow (Rites)}
\label{sec:obligation-overflow-rev}

When \textbf{Obligation} is ticked past its maximum (from Rites, vows, bargains), mark \textbf{Fatigue} based on scene severity:
\begin{itemize}
  \item \textbf{Low}: +1 Fatigue
  \item \textbf{Standard}: +2 Fatigue
  \item \textbf{High}: +3 Fatigue
\end{itemize}
If this \emph{fills} the Fatigue Track, apply the \textbf{Fatigue $\rightarrow$ Harm} conversion (see \S\ref{sec:health-fatigue-harm-rev}).


% --- Fate's Edge SRD — Section 6: Summons & Outsiders ---
% Include this file from your main .tex with: %----------------------------------------
\section{Summoning (Pact-Whisperer)}
\label{subsec:summoning}

Summoning is the disciplined art of calling and binding Outsiders for temporary aid.  
This path requires the \textbf{Pact-Whisperer} Talent (2 XP).  
Each summoned being is restrained by a metaphysical tether called a \textit{Leash}, representing the summoner’s control and the strain of sustaining the bond.

\paragraph{Talents \& Access.}
\begin{itemize}
  \item \textbf{Lesser Pactwright:} You may \emph{Call} spirits of \textbf{Cap~1}.
  \item \textbf{Greater Pactwright:} You may also \emph{Call} spirits of \textbf{Cap~3}.
  \item \textbf{Dual Pactwright:} With both Lesser and Greater Pactwright, you may maintain one spirit of each Cap simultaneously.
\end{itemize}

\begin{fatebox}[Summoning Core Mechanics]
\begin{tabularx}{\textwidth}{lX}
\toprule
\textbf{Mechanic} & \textbf{Description and Requirements} \\
\midrule
\textbf{Call} & 1 Action to manifest the spirit at \textit{Near} range; choose a Spirit Template aligned to fiction or Patron domain. \\
\textbf{Bind} & Spend 1 Boon \emph{or} mark 1 Fatigue to establish initial control. \\
\textbf{Leash} & Set Leash = \textbf{Cap + Command} segments.  
(\textit{Cap} is the Outsider’s tier: Cap~1 for Lesser, Cap~3 for Greater.) \\
\textbf{Tick Leash} & Whenever the spirit takes Harm, you command it against its nature, you split focus, a rival contests it, it moves \textit{Close → Far} rapidly, or crosses a \texttt{[WARD]} (\textit{DV = Cap}). \\
\textbf{Departure} & When the Leash fills, the spirit acts to its nature once, then departs (or turns hostile at GM discretion). \\
\bottomrule
\end{tabularx}
\end{fatebox}

\paragraph{Procedure.}
\begin{enumerate}
  \item \textbf{Call (1 Action):} A spirit manifests at \textit{Near}. Choose a Spirit Template appropriate to the scene or Patron.
  \item \textbf{Bind:} Spend 1 Boon \emph{or} mark 1 Fatigue to anchor the connection.
  \item \textbf{Leash:} Record Leash = \textbf{Cap + Command} segments. Draw a clock to track strain.
  \item \textbf{Command:} Each round, issuing a meaningful order uses your Action. Commands contrary to the spirit’s nature tick the Leash.
  \item \textbf{Maintain:} If you split focus or perform other significant actions while it acts on your order, tick the Leash.
  \item \textbf{Departure:} When the Leash fills, the spirit acts to its nature once, then departs. Use this to escalate or reveal consequences.
\end{enumerate}

\paragraph{Economy \& Limits.}
\begin{itemize}
  \item \textbf{Boon Finesse:} Once per round, spend 1 Boon to clear 1 Leash tick (before it fills). Represents appeasement or renewed focus.
  \item \textbf{Action Economy:} Issuing commands uses your Action; most spirits act immediately after their summoner.
  \item \textbf{Concurrency:} Only one active summoned spirit at a time unless a Talent states otherwise. Exceeding this limit inflicts 1 Fatigue per extra Cap point.
  \item \textbf{Downtime:} All summons end at Downtime unless explicitly sustained by a Rite or Asset.
\end{itemize}

\paragraph{Example.}
\textit{Kestra calls a Cap~3 fire elemental to aid in battle. She spends 1 Boon to Bind it.  
The elemental’s Leash is 7 segments (3~+~Command~4). When it takes Harm, the GM ticks the Leash. Later, Kestra splits focus to issue orders while attacking, ticking again.  
Careful management and Boon Finesse keep the bond stable—until the elemental’s fury tests her will.}
\floatbarrier
\clearpage

\section{Summons and Outsiders}

\subsection{Definition}
An \textbf{Outsider} is any being not native to the world of Fate's Edge. This includes summoned spirits, demons, celestials, and entities that arrive from beyond the veil of the Eight Elements. They are powerful but dangerous to bind.

\subsection{Summoning (Pact-Whisperer Core)}
Summoning is a way to call and bind Outsiders for temporary aid.

\begin{enumerate}
  \item \textbf{Call} (1 Action): A spirit manifests at Near range. Choose a Spirit Template.
  \item \textbf{Bind}: Choose one: spend 1 Boon or mark 1 Fatigue.
  \item \textbf{Leash}: Set Leash = Cap + 2 segments (Cap is the Outsider's tier, typically 1/3/5 for Lesser/Greater/Elder).
  \item \textbf{Tick Leash} whenever any occur:
    \begin{itemize}
      \item Spirit takes harm.
      \item You command against its nature.
      \item You split focus (take another significant action while it acts).
      \item A rival contests it.
      \item It moves from Close to Far quickly.
      \item It crosses a [WARD].
    \end{itemize}
  \item \textbf{Departure}: When the Leash fills, the spirit acts to its nature once, then departs.
\end{enumerate}

\textbf{Limits:} Only one active summoned spirit at a time (unless a Talent says otherwise). All summons depart at Downtime unless explicitly sustained.

\subsection{Boon Finesse}
Once per round, you may spend 1 Boon to clear 1 tick from your current spirit's Leash. You cannot do this after the Leash has filled.

\subsection{Outsider Caps}
\begin{itemize}
  \item PC-summoned Outsiders: Cap is limited by Talents (Lesser = 1, Greater = 3).
  \item NPC Outsiders: GM assigns based on story needs (Lesser = 1, Greater = 3, Elder = 5).
\end{itemize}

\subsection{Tags for Summons \& Outsiders}
Certain Tags specifically interact with Outsiders.

\begin{description}[leftmargin=1.5em, style=nextline]
  \item[WARD:] Creates a magical edge/zone that Outsiders must test to cross.
    \begin{itemize}
      \item DV = Outsider's Cap.
      \item Hit: Outsider crosses and its Leash gains +DV segments.
      \item Partial: Outsider crosses and its Leash gains +1 segment.
      \item Miss: Outsider fails to cross this beat.
    \end{itemize}
  \item[BANISH:] Drives a visible Outsider toward departure.
    \begin{itemize}
      \item DV = Outsider's Cap.
      \item Hit: Add +DV segments to its Leash (or Exit Tally).
      \item Partial: Add +1 segment.
      \item Miss: No effect.
    \end{itemize}
  \item[UNWARD:] Suppresses or dismisses a [WARD].
    \begin{itemize}
      \item DV by fiction (materials, sanctity, prep, locus, opposition).
      \item Hit: Ward dismissed/suppressed.
      \item Partial: Ward suppressed briefly (1 beat).
      \item Miss: No effect.
    \end{itemize}
\end{description}

\subsection{Unified Leash / Exit Tally System}
\begin{itemize}
  \item Summoned Outsiders track their service via a \textbf{Leash} (Cap + 2 segments).
  \item Non-summoned Outsiders affected by [WARD] or [BANISH] gain a temporary \textbf{Exit Tally} = Cap + 2. When the tally fills, they act to nature once, then depart.
\end{itemize}

\subsection{GM Guidance}
\begin{itemize}
  \item Summons are not permanent allies; they are volatile forces.
  \item Always color Outsider behavior by their Elemental resonance and domain.
  \item When the Leash fills, deliver a memorable "act to nature" moment before they vanish.
  \item Use SB to escalate Outsider complications: a jealous Patron, a backlash of strange omens, or collateral spiritual harm.
\end{itemize}

% !TEX root = srd_main.tex
% SRD Insert: Elemental Backlash Tables — 8 Elements × Minor/Major
% Assumes booktabs, tabularx, xcolor, tcolorbox are available

\section{Elemental Backlash}\label{sec:backlash-tables}
\index{Backlash}\index{Elements}\index{Story Beats}\index{Realms}\index{Obishaal@Obishaal (Dreams/Thresholds)}

When magic disturbs the weave, the world pushes back. Backlash manifests as fiction-first complications with light mechanical teeth. Each element (and its metaphysical counterpart) has a \textbf{Minor} and \textbf{Major} pattern.

\begin{tcolorbox}[title={Using Backlash at the Table},colback=gray!5,colframe=black]
\textbf{Trigger.} A roll shows a 1 (gaining a (SB)) or the text explicitly says "accept 1 (SB) to escalate."\newline
\textbf{Choose One:} Apply the table's Minor effect, or escalate to Major by adding $\mathbf{+1}$ (SB) immediately.\newline
\textbf{Mechanical Nudge Types.} \emph{Position/Effect shift}, \emph{Clock tick (1/2)}, \emph{Condition}, or \emph{Immediate Cost}.\index{Story Beats!backlash escalation}
\end{tcolorbox}

\subsection*{Realms and Counterparts}\label{subsec:realms-counterparts}
\begin{itemize}
\item \textbf{Earth} \emph{(Realm: Stone)} $\leftrightarrow$ \textbf{Fate} \emph{(Anti-magic, inevitability)}\index{Earth}\index{Fate}
\item \textbf{Fire} \emph{(Realm: Ember)} $\leftrightarrow$ \textbf{Life} \emph{(Vital spark, growth)}\index{Fire}\index{Life}
\item \textbf{Air} \emph{(Realm: Gale)} $\leftrightarrow$ \textbf{Luck/Fortune} \emph{(Ephemera, unlikely turns)}\index{Air}\index{Luck}\index{Fortune}
\item \textbf{Water} \emph{(Realm: Tides)} $\leftrightarrow$ \textbf{Death/Dreams/Thresholds (Obishaal)} \emph{(Passage, veils, the Ways Between)}\index{Water}\index{Death}\index{Dreams}\index{Thresholds}\index{Obishaal@Obishaal}
\end{itemize}

\vspace{0.5em}

% ========================= MASTER TABLE =========================
\begin{table}[h]
\centering
\caption{Backlash by Element (Minor / Major)}
\label{tab:backlash-8x2}
\renewcommand{\arraystretch}{1.12}
\begin{tabularx}{\linewidth}{>{\bfseries}l l >{\raggedright}X >{\raggedright}X}
\toprule
Element & Realm / Counterpart & Minor Backlash (fiction • nudge) & Major Backlash (fiction • nudge) \\
\midrule
Earth & Stone / \emph{Fate} & \emph{Stone binds.} Dust cakes tools; footing slips. • \textbf{Position},–1 or mark \textsc{Encumbered} (light). & \emph{Ground claims.} A fissure opens or masonry seizes. • \textbf{Clock},+1/2 on \emph{Collapse/Entrap} or \textbf{Condition},: \textsc{Pinned}. \\
Fire & Ember / \emph{Life} & \emph{Heat flares.} Smoke blinds; sparks bite. • \textbf{Effect},–1 or \textbf{Condition},: \textsc{Singed} (disadvantage to precise actions). & \emph{Blaze takes.} Fuel ignites or blood races fever-hot. • \textbf{Clock},+1 on \emph{Spreading Fire} or take \emph{1 Harm} ignoring armor. \\
Air & Gale / \emph{Luck} & \emph{Winds misplace.} Words scatter; aim wavers. • \textbf{Position},–1 or \textbf{Clock},+1/2 on \emph{Alarmed Attention}. & \emph{Fickle turn.} An unlikely mishap hits you instead. • \textbf{Immediate Cost},: lose a tool/use \emph{or} \textbf{(SB)},+1 as chaos compounds. \\
Water & Tides / \emph{Obishaal} & \emph{Seep and swell.} Grip slicks; tide shifts. • \textbf{Effect},–1 or \textbf{Condition},: \textsc{Waterlogged} (gear slow). & \emph{Undertow calls.} Passage veers; something is pulled through. • \textbf{Clock},+1 on \emph{Flood/Wayward Current} \emph{or} introduce a \emph{brief} intrusion from the Ways Between. \\
Fate & Anti-magic / \emph{Earth} & \emph{Lines harden.} Probability resists change. • \textbf{Effect},–1 on overt magic or \textbf{Clock},+1/2 on \emph{Inevitable Outcome}. & \emph{Edict falls.} A declared cost must be paid now. • \textbf{Immediate Cost},: sacrifice a resource \emph{or} \textbf{(SB)},+1 and mark \textsc{Omen}. \\
Life & Vital spark / \emph{Fire} & \emph{Growth misfires.} Vines choke; pulse surges. • \textbf{Condition},: \textsc{Overgrowth} (tethered) or \textbf{Effect},–1 on precision. & \emph{Riot of life.} Parasites bloom; healing twists. • \textbf{Clock},+1 on \emph{Biohazard} \emph{or} convert 1 healing into \textbf{(SB)},+1. \\
Luck & Ephemera / \emph{Air} & \emph{Odds flip.} A near-sure thing slips. • \textbf{Position},–1 or \textbf{Clock},+1/2 on \emph{Unwelcome Coincidence}. & \emph{Snake eyes.} Catastrophic fluke. • Force a \emph{re-roll}; if any 1 appears, \textbf{(SB)},+1 \& apply Minor again. \\
Death/Dreams & Ways Between / \emph{Water} & \emph{Veil thins.} Echoes, whispers, cold breath. • \textbf{Condition},: \textsc{Shaken} (first action –1) or \textbf{Clock},+1/2 on \emph{Haunting}. & \emph{Threshold opens.} A path misaligns; a revenant claims a due. • \textbf{Clock},+1 on \emph{Crossing Due} \emph{or} immediate scene intrusion from Obishaal. \\
\bottomrule
\end{tabularx}
\end{table}

% !TEX root = resource_guide_main.tex
% Extended Chapter: Rituals — Philosophy, Procedures, Costs, and Advanced Use
% Assumes: booktabs, tabularx, xcolor, tcolorbox, enumitem, hyperref, amsmath

\section{Rituals (Extended)}\label{sec:rituals-extended}
\index{Rituals}\index{Story Beats}\index{Backlash}\index{Elements}\index{Realms}\index{Obishaal@Obishaal}\index{Patrons}\index{Symbols}\index{Runekeeper}\index{Invoker}\index{Summoner}\index{Caster}

Rituals are \emph{slow magic}: explicit intent, staged action, and negotiated risk. They are how characters bend the world carefully, trading time, components, and narrative exposure for precise results. This section expands the SRD quick-start (\S\ref{sec:universal-rituals}) with procedures, dials, and worked examples.

\subsection{Design Goals}\label{subsec:ritual-goals}\index{Design Philosophy}
\begin{itemize}
\item \textbf{Fiction-first.} Components and steps are story handles, not inventory chores.
\item \textbf{Visible costs.} Every ritual declares \emph{what it costs} (time, component loss, conditions) and \emph{how it risks} (SB)/backlash.
\item \textbf{Tempting choices.} Players can \emph{push}—accept (SB) to escalate position/effect/scale.
\item \textbf{Portable.} Works for Runekeepers, Invokers, Summoners, and Free Casters with minimal chassis-specific tweaks.
\end{itemize}

\subsection{Ritual Procedure}\label{subsec:ritual-procedure}
Use this skeleton for any ritual, published or improvised.

\begin{enumerate}[label=\textbf{Step \arabic*:}, leftmargin=2.2em]
\item \textbf{State Intent.} What do you want? Clarify element/Realm if obvious (Fire for heat, Water for memory, Fate for anti-magic, Obishaal for thresholds).\index{Elements!choosing}
\item \textbf{Choose Scope.} Size, duration, range, and detail. Start modest; escalations come later.
\item \textbf{Lay Components.} Name \emph{two things}: (a) \textbf{Focus} (tool/site/patron sign), (b) \textbf{Fuel} (herb, blood, pact, memory). Decide which is consumed vs. retained.\index{Components}
\item \textbf{Set Time.} Default: \emph{Low 1 minute / Med 5–10 minutes / High 15–20 minutes}. More time improves position/effect; rushing worsens it.
\item \textbf{Call Risks.} Point to the element's \textbf{Backlash} (\S\ref{sec:backlash-condensed}) and the default \textbf{(SB) trigger}: any 1 rolled creates a (SB); re-rolling 1s does not remove (SB) and may add another.
\item \textbf{Roll and Resolve.} Apply position/effect and any clocks. Offer a \emph{push}: take +1 (SB) to step up result now.
\item \textbf{Mark Costs.} Consume components, apply Conditions, or tick wear/concurrency (per chassis). Close the scene hooks the ritual created.
\end{enumerate}

\subsection{Component Economy}\label{subsec:component-economy}
Components are levers, not taxes. Use them to signal tone and stakes.

\begin{table}[h]
\centering
\caption{Components as Narrative Levers}
\label{tab:ritual-components}
\renewcommand{\arraystretch}{1.12}
\begin{tabularx}{\linewidth}{>{\bfseries}l X X}
\toprule
Type & Examples & Mechanical Nudge \\
\midrule
Focus (retained) & Patron token, true-name sigil, saint's nail, ancestral blade. & +1 \emph{position} on setup or advantage to related follow-up actions. \\
Fuel (consumed) & Herb bundle, salt vial, blood drop, silver bead, memory-laden note. & –1 \emph{time step} (faster) or +1 \emph{effect} step if expensive/rare. \\
Site (context) & Crossroads, standing stones, bathhouse, bell tower at midnight. & Shift backlash element or re-route it (e.g., into \emph{Alarmed Attention} clock). \\
Vow (social) & Sworn phrase, bargain pledge, offered favor. & If broken, immediate (SB) +1 and an intrusion linked to the vow. \\
\bottomrule
\end{tabularx}
\end{table}

\subsection{Teamwork and Aid}\label{subsec:ritual-teamwork}\index{Teamwork}
\begin{itemize}
\item \textbf{Hands \& Voices.} Each assistant names one component they contribute; either reduces cast time \emph{or} accepts up to 1 (SB) on the caster's behalf once per ritual.
\item \textbf{Focus Chain.} Passing the Focus around the circle grants advantage on the finishing action but risks \textsc{Distracted} if interrupted.
\item \textbf{Distributed Load.} Splitting a High ritual into two coordinated Mediums avoids a Major backlash but creates two Minor hooks instead.
\end{itemize}

\subsection{Clocks and Outcomes}\label{subsec:ritual-clocks}\index{Clocks}
Tie every consequential ritual to \textbf{named clocks}. Examples: \emph{Spreading Fire}, \emph{Inevitable Outcome}, \emph{Crossing Due}, \emph{Alarmed Attention}. Advancing or reducing clocks is often better than flat bonuses.

\begin{tcolorbox}[title={Outcome Palette},colback=gray!5,colframe=black]
\small \textbf{On a strong result:} full effect, +1 effect step, or Clock –1.\newline
\textbf{On a mixed:} effect with a cost (component consumed; condition applied).\newline
\textbf{On a weak:} effect limited; Clock +1/2; Minor Backlash.\newline
\textbf{On a push:} player may take (SB) +1 to upgrade one step immediately.\end{tcolorbox}

\subsection{Backlash Integration}\label{subsec:ritual-backlash}
Use the condensed table (\S\ref{tab:backlash-condensed}). Calibrate by scene weight: exploratory scenes favor Minor; pivotal moments bait a Major via (SB) +1.

\subsection{Chassis-Specific Notes}\label{subsec:ritual-chassis}
\paragraph{Runekeeper.}\index{Runekeeper} Embed Rites as \emph{accelerants}: a published Rite may count as a Focus that upgrades position or halves time.
\paragraph{Invoker.}\index{Invoker} Symbols accumulate \emph{wear}; a maintenance rite can clear 1 wear mid-scene on success \emph{or} shift backlash from Major to Minor.
\paragraph{Summoner.}\index{Summoner} \emph{Gate} effects occupy concurrency slots. Disruption on broken terms: (SB) +1 and the entity acts on its last instruction.
\paragraph{Caster (Free).}\index{Caster} Tags become explicit ritual steps (\emph{bind, veil, reveal})—chain two compatible tags once/scene for a synergy bump without extra cost.

\subsection{Ritual Templates}\label{subsec:ritual-templates}
Use these fill-in cards to author new content quickly.

\begin{tcolorbox}[title={Template: Utility Rite (Low)},colback=gray!3,colframe=black]
\textbf{Name}: \rule{0.6\linewidth}{0.4pt} \quad \textbf{Element}: \rule{0.28\linewidth}{0.4pt}\\
\textbf{Cast Time}: 1 minute \quad \textbf{Scope}: pocket-scale\\
\textbf{Components}: Focus (kept): \rule{0.4\linewidth}{0.4pt}; Fuel (consumed): \rule{0.35\linewidth}{0.4pt}\\
\textbf{Effect}: \rule{0.9\linewidth}{0.4pt}\\
\textbf{Cost}: \rule{0.9\linewidth}{0.4pt}\\
\textbf{Backlash}: Minor (\S\ref{sec:backlash-condensed}). \textbf{Push}: take (SB) +1 to upgrade one step.
\end{tcolorbox}

\begin{tcolorbox}[title={Template: Scene Rite (Med)},colback=gray!3,colframe=black]
\textbf{Name}: \rule{0.6\linewidth}{0.4pt} \quad \textbf{Element}: \rule{0.28\linewidth}{0.4pt}\\
\textbf{Cast Time}: 5–10 minutes \quad \textbf{Scope}: room/street\\
\textbf{Components}: Focus (kept): \rule{0.4\linewidth}{0.4pt}; Fuel (consumed): \rule{0.35\linewidth}{0.4pt}; Site: \rule{0.3\linewidth}{0.4pt}\\
\textbf{Effect}: \rule{0.9\linewidth}{0.4pt}\\
\textbf{Cost}: \rule{0.9\linewidth}{0.4pt}\\
\textbf{Backlash}: Minor; offer Major via (SB) +1. \textbf{Clocks}: \rule{0.5\linewidth}{0.4pt}
\end{tcolorbox}

\begin{tcolorbox}[title={Template: Set-Piece Rite (High)},colback=gray!3,colframe=black]
\textbf{Name}: \rule{0.6\linewidth}{0.4pt} \quad \textbf{Element}: \rule{0.28\linewidth}{0.4pt}\\
\textbf{Cast Time}: 15–20 minutes \quad \textbf{Scope}: block/fort\\
\textbf{Components}: Focus (kept): \rule{0.4\linewidth}{0.4pt}; Fuel (consumed): \rule{0.35\linewidth}{0.4pt}; Site: \rule{0.3\linewidth}{0.4pt}; Vow: \rule{0.3\linewidth}{0.4pt}\\
\textbf{Effect}: \rule{0.9\linewidth}{0.4pt}\\
\textbf{Cost}: \rule{0.9\linewidth}{0.4pt}\\
\textbf{Backlash}: Likely Major; bait with (SB) +1. \textbf{Clocks}: \rule{0.5\linewidth}{0.4pt}
\end{tcolorbox}

\subsection{Worked Examples}\label{subsec:ritual-examples}
\paragraph{Example 1: Quiet Veil (Team Infiltration).}\index{Rituals!examples}
\emph{Intent:} silence the group for one scene. \emph{Scope:} corridor sweep. \emph{Components:} ash (fuel, consumed), bell (focus, kept). \emph{Time:} 5 minutes. \emph{Risks:} Air/Luck Minor on 1; offer Major to avoid dogs' scent. \emph{Roll:} mixed—effect with cost. \emph{Outcome:} \textsc{Muted} condition until scene ends; Clock –1/2 on \emph{Patrol Pass}. Player takes (SB) +1 to also foil scent (Major avoided by paying the (SB)).

\paragraph{Example 2: River's Memory (Investigation).}
\emph{Intent:} view last night's ferry landing. \emph{Scope:} a few minutes of blurred images. \emph{Components:} bowl, token from dock. \emph{Time:} 10 minutes. \emph{Risk:} Water/Obishaal Minor. \emph{Roll:} strong—clear image; token ruined per Cost. \emph{Outcome:} Clock –1 on \emph{Where did the courier go?}; whisper from the Ways foreshadows a revenant (hook).

\paragraph{Example 3: Fate-Splice (Boss Rescue).}
\emph{Intent:} move the poison consequence from the prince to the knight. \emph{Scope:} one Major consequence. \emph{Components:} paired names on vellum; vow. \emph{Time:} 15 minutes. \emph{Risk:} Fate/Earth. \emph{Roll:} weak—Minor Backlash; \emph{Inevitable Outcome} +1/2. \emph{Push:} (SB) +1 to capture the full consequence anyway. \emph{Outcome:} knight bears the poison; the \textsc{Omen} mark appears (future hook).

\subsection{Safety and Consent}\label{subsec:ritual-safety}\index{Safety}
Rituals often touch body horror, spiritual intrusion, or coercive bargains. Use lines/veils, X-card, script change, or your table's preferred tools. Make \textbf{vows} opt-in; provide non-coercive alternatives with different trade-offs.

\subsection{Optional Modules}\label{subsec:ritual-modules}
\paragraph{Entropy Counters.} Track ritual entropy per scene; at 3+ entropy, the next Minor backlash upgrades to Major automatically. Resets on scene change.
\paragraph{Resonance Sites.} Mark places that boost one element (+1 effect) and hinder its counterpart (–1 position). Crossing a resonance flips the pairing.
\paragraph{Material Tags.} Let special materials act like tags (cold iron, voidglass) granting narrow immunities or redirecting backlash type.

\subsection{GM Troubleshooting}\label{subsec:ritual-troubleshooting}
\begin{itemize}
\item \textbf{Pacing drifts long.} Shorten cast time by consuming an extra Fuel component; keep one meaningful step.
\item \textbf{Risk feels toothless.} Name a clock and advance it on mixed/weak even if the effect lands.
\item \textbf{Runekeeper dominates.} Insert a \emph{Rune Draw} tell for high-grade rites \emph{or} grant Invokers a mid-scene maintenance clear on a solid success.
\item \textbf{Summons flood scene.} Enforce concurrency and Disruption (\S\ref{subsec:ritual-chassis}).
\item \textbf{Casters feel mushy.} Require two explicit tags per ritual step (bind/veil/reveal); grant once/scene synergy bump.
\end{itemize}

\begin{tcolorbox}[title={Summary},colback=gray!5,colframe=black]
Rituals trade \textbf{time, components, and exposure} for \textbf{precision and scale}. Keep costs visible, risk tempting, and outcomes named via clocks. Offer players the choice to buy bigger results with (SB)—then pay off every hook
\end{tcolorbox}
