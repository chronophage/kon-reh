% =========================
% Fate's Edge — Magic System (Main)
% =========================
\chapter{Magic System}
\label{chap:magic-system}

\section{Design Philosophy}
\label{sec:magic-philosophy}

Magic in \textbf{Fate's Edge} is a dangerous negotiation with the fabric of reality. It is powerful and flexible—yet every attempt to shape it carries risk. Each die showing \textbf{1} generates \textbf{Story Beats (SB)}, which are prompts for twists and complications. The fiction leads; math follows.

\section{The Four Paths of Magic}
\label{sec:four-paths}

\subsection{Casting (Freeform Magic)}
\label{subsec:freeform-casting}

Freeform casting represents raw, improvisational magic.

\begin{description}
\item[Requirement] \textbf{Caster's Gift} Talent (2 XP).
\item[Mechanics] Two-step \emph{Weave \& Cast} colored by the Eight Elements; fiction-first targets and scope.
\item[Risk] Each 1 generates SB; DV scales with scope; backlash is expressed by Element (or its opposite).
\item[Limits] Heavy control effects (e.g., \texttt{[WARD]}, \texttt{[BANISH]}, \texttt{[UNWARD]}) require a printed source (Talent, \emph{Rite} text, or Spell result).
\end{description}

\subsection*{GM Guidance}\label{subsec:backlash-gm}
\begin{itemize}
\item \textbf{Show, then Nudge.} Lead with the fiction (smoke, fissures, whispers), then apply the smallest mechanical nudge that preserves drama.\index{Design Philosophy}
\item \textbf{Escalate with Consent.} Offer players the choice to escalate Minor to Major by taking +1 (SB) now to seize something they want.\index{Story Beats!player choice}
\item \textbf{One Bite per Cast.} Apply at most one backlash per cast/action unless a move explicitly stacks. Keep it punchy, not punitive.\index{Backlash!frequency}
\item \textbf{Clocks with Names.} Name clocks (\emph{Spreading Fire}, \emph{Inevitable Outcome}) so they feed the fiction and remind the table what's at stake.\index{Clocks}
\end{itemize}

\begin{tcolorbox}[title={Backlash Cheatsheet (margin-ready)},colback=gray!5,colframe=black]
\small \textbf{Minor = wobble, Major = lurch.} Types: Position/Eff, Clock +1/2 or +1, Condition, Cost. Offer Major by (SB)+1. Earth/Fate binds; Fire/Life burns/grows; Air/Luck scatters/ flips; Water/Obishaal pulls/opens.\par\smallskip
Place a mini version of Table~\ref{tab:backlash-8x2} on character sheets.
\end{tcolorbox}

\subsection{Rites Users (Runekeepers)}
\label{subsec:runekeepers}

Runekeepers bind themselves to structured \emph{Rites} from a single Patron.

\begin{description}
\item[Requirement] \textbf{Thiasos (Familiar)} (2 XP) \emph{and} \textbf{Codex} (4 XP). Runekeepers are restricted to \textbf{one} Patron.
\item[Mechanics] \textbf{Invoke} a known \emph{Rite} as \textbf{1 action}; on completion, mark \textbf{+1 Obligation} to that Patron. \emph{Push It} once/scene for amplified effect (\textbf{+1 Obligation}).
\item[Patron's Gift (Imbuement)] With Thiasos, once/scene as \textbf{1 action} imbue a held item for the scene with \textbf{+1 Melee} and \textbf{+1 Thematic} (a fixed Skill set by the Patron; see Table in \S\ref{sec:rites}). \emph{Push It} to extend one additional scene (\textbf{+1 Obligation}). A Codex is \emph{not} required for the Gift.
\end{description}

\section{Rites Difficulty Value (Expanded)}
\label{sec:rites-dv-expanded}
\index{Rites!Difficulty Value}
\index{DV}

\subsection*{Core Rule}

The Difficulty Value (DV) to cast a Rite is:

\[
\text{DV} = \max\!\big(\text{Obligation Cost} - \text{Spirit}, \, \text{Tier}\big)
\]

\begin{description}
  \item[Obligation Cost:] The Rite’s listed cost in Obligation segments. This reflects the Patron’s toll for the magic.
  \item[Spirit:] The caster’s Spirit attribute. Each point reduces the effective weight of the Obligation, representing inner resilience and willpower.
  \item[Tier:] The Rite’s intrinsic difficulty based on scope or potency. DV can never fall below this floor.
\end{description}

\subsection*{Design Philosophy}

This formula balances three forces:
\begin{itemize}
  \item \textbf{Debt vs. Strength:} Powerful Rites impose heavier Obligation, but high Spirit offsets that burden.
  \item \textbf{Narrative Tier Floor:} Even the simplest ritual of summoning or warding retains a minimum DV based on its Tier, ensuring gravity and consistency.
  \item \textbf{Scaling:} As characters grow, Spirit makes weaker Rites feel easier, while greater costs still challenge them.
\end{itemize}

\subsection*{Worked Examples}

\begin{itemize}
  \item \emph{Novice Example:} A Tier~1 Rite with Obligation Cost~2, cast by a Spirit~1 character:
  DV = $\max(2-1, 1) = 1$.
  The character feels only a slight strain, their Spirit covering most of the toll.

  \item \emph{Mid-Level Example:} A Tier~2 Rite with Obligation Cost~4, cast by a Spirit~2 character:
  DV = $\max(4-2, 2) = 2$.
  The Rite taxes them, but their Spirit prevents the burden from becoming overwhelming.

  \item \emph{Advanced Example:} A Tier~3 Rite with Obligation Cost~7, cast by a Spirit~3 character:
  DV = $\max(7-3, 3) = 4$.
  Here, Obligation dominates—the cost is heavy, and even strong Spirit cannot fully deflect it.
\end{itemize}

\subsection*{GM Guidance}

\begin{itemize}
  \item \textbf{Patron Themes:} Higher Obligation costs should reflect narrative weight, not just numbers. A Patron of decay may exact tolls in corruption, while a Patron of luck might demand reckless wagers.
  \item \textbf{Spirit as Fiction:} Encourage players to describe how their Spirit manifests—does it show as discipline, willpower, ritual focus, or raw charisma? Make the stat come alive in the fiction.
  \item \textbf{Scaling with Tier:} Remind players that no matter how trivial the fiction might feel, a Tier~2 or higher Rite is never “easy.” The floor maintains tension.
\end{itemize}

\section{Obligation Capacity}

A character’s \textbf{Obligation Capacity} equals Spirit + Presence.
Track total Obligation segments across all Patrons (or Symbols, for Invokers).

\begin{itemize}
  \item \textbf{Exceeding Capacity:} For each segment above Capacity, mark 1 Fatigue. The character cannot Invoke Rites or perform rituals until Obligation is reduced below Capacity.
  \item \textbf{Resolution:} Reduce Obligation through Downtime service, Patron tasks, ritual cleansing, or story resolution.
\end{itemize}

\textbf{Example:} Spirit~2 + Presence~3 = Capacity 5.
6 segments → Fatigue~1.
7 segments → Fatigue~2.
10 segments → Harm~1.
11 segments → Harm~2.

\subsection*{Optional Modules}

\begin{itemize}
  \item \textbf{Overflow DV:} If DV exceeds 5, consider applying minor narrative complications or Backlash risks on top of the roll, to show strain bleeding into the scene.
  \item \textbf{Patron Wrath Trigger:} A Patron may impose extra narrative tolls if the DV was reached primarily through Obligation rather than Tier—this shows over-dependence on their favor.
\end{itemize}

\subsection{Invokers (Symbol Path)}
\label{subsec:invokers}

Invokers use consecrated \textbf{Symbols} as ritual anchors to access a Patron's \emph{Rites} without a full bond.

\begin{description}
\item[Requirement] \textbf{Patron's Symbol} (4 XP) per Patron; one Symbol per Patron. No Thiasos or Codex required.
\item[Ritual Invocation] Perform the \emph{Rite} as a \textbf{ritual} DV +1 rounds. Completion always marks \textbf{+1 Obligation}.
\item[Crack the Seal] Resolve instantly as \textbf{1 action} by setting the Symbol to \emph{Compromised} and marking \textbf{+2 Obligation} (\textbf{+3} if High-Power). The GM may immediately spend 1 on-theme SB. Restore in Downtime (test DV 3 or by fiction) or spend 1 XP.
\item[Limits] Symbols must be openly displayed during the ritual; carrying \textbf{4+} Symbols causes +1 Obligation on the first ritual each scene; rival Symbol interference may worsen Position and add +1 Obligation.
\end{description}

% --- Talent: Borrowed Grace (Invoker) ---
\section*{Borrowed Grace}
\label{talent:borrowed-grace}
\index{Talents!Invoker}\index{Imbuement!Lesser}

\textbf{Type:} Invoker Talent — \textit{Lesser Imbuement}

\subsection*{Use}
\begin{itemize}
  \item \textbf{Cost:} 1 Boon, 1 action.
  \item \textbf{Effect (pick one on use):} \textbf{+1 Melee} \emph{or} \textbf{+1 Thematic} (your table's thematic Skill).
  \item \textbf{Duration:} \textit{Single action/attack} (instantaneous boost).
  \item \textbf{Requirement:} Wield/display the Patron's \textbf{Symbol}.
  \item \textbf{Obligation:} +1 \textbf{Obligation} to that Patron immediately (see \S\ref{sec:obligation}).
  \item \textbf{Limits:} Cannot be extended, stacked, or \emph{Pushed} for duration.
\end{itemize}

\subsection*{Fictional Framing}
A quick, rule-bending channel through a Patron's \emph{Symbol}—a sliver of grace, borrowed for a moment and paid for in debt.

\subsection*{Table Guidance (1-liners)}
\begin{itemize}
  \item \textbf{Combat:} Spike a strike vs. a tough foe; or steady a parry in a desperate bind.
  \item \textbf{Skill:} Nudge a pivotal social/ritual/track roll tied to the Patron's sphere.
  \item \textbf{Fallout:} Repeated use accrues \textbf{Obligation}; NPC faithful may notice "stolen" grace.
\end{itemize}

\subsection*{Balance Notes}
\begin{itemize}
  \item Weaker than full Imbuement: \emph{one} action, no sustain, upfront Obligation.
  \item \textbf{Symbol dependency:} No Symbol, no channel (concealed or lost Symbol = no effect).
\end{itemize}

\subsection*{GM Hooks (quick picks)}
\begin{itemize}
  \item \textbf{Compel Debt:} A Patron agent arrives when Obligation crosses a tick.
  \item \textbf{Clash of Signs:} Using rival Symbols back-to-back risks minor \textbf{Backlash} (drop Position or +1 SB).
  \item \textbf{Spotlight Tell:} Brief visual tell (scent, sigil flare) marks the borrowing to observant NPCs.
\end{itemize}

\subsection{Summoners (Pact-Whisperer)}
\label{subsec:summoners}

Summoners call spirits quickly and manage them with a \emph{Leash} track.

\paragraph{Talents \& Access}
\begin{itemize}
  \item \textbf{Lesser Pactwright:} You may \emph{Call} spirits of Cap~1.
  \item \textbf{Greater Pactwright:} You may also \emph{Call} spirits of Cap~3.
  \item \textbf{Dual Pactwright:} With both Lesser and Greater Pactwright, you may maintain one spirit of each Cap simultaneously.
\end{itemize}

\paragraph{Core Procedure}
\begin{enumerate}
  \item \textbf{Call (1 action):} A spirit manifests at \textit{Near}. Choose a Spirit Template (by fiction).
  \item \textbf{Bind (no extra roll):} Choose one: spend \textbf{1 Boon} \emph{or} mark \textbf{1 Fatigue}.
  \item \textbf{Leash:} Set \textit{Leash} = \textbf{Cap + 2} segments on the spirit.
  \item \textbf{Tick Leash} whenever any of the following happen: the spirit takes harm; you command it against its nature; you \emph{split focus} (you take another significant action while it acts on your order); a rival contests it; it rushes from \textit{Close} to \textit{Far} under pressure. Crossing a \texttt{[WARD]} uses the Outsider crossing rules (DV = Cap).
  \item \textbf{Release:} When the Leash fills, the spirit acts to its nature \emph{once}, then departs.
\end{enumerate}

\paragraph{Economy \& Limits}
\begin{itemize}
  \item \textbf{Boon Finesse:} Once per round, you may spend \textbf{1 Boon} to clear \textbf{1} Leash tick on your current spirit (not after it has already filled).
  \item \textbf{Action Economy:} Issuing a meaningful command uses your action.
  \item \textbf{Concurrency:} Limit \textbf{one active spirit} at a time (you may \emph{Call} again after departure).
  \item \textbf{Downtime} ends all summons unless an ability explicitly states otherwise.
\end{itemize}

\section{The Nature of Magic}
\label{sec:nature-of-magic}

\begin{itemize}
\item \textbf{Volatile by design:} Each working pushes boundaries that resist being bent.
\item \textbf{Risk embodied:} Each 1 on any magic roll generates SB the GM can spend for backlash or twists.
\item \textbf{Narrative weight:} Every magical action alters the scene, even on a success.
\item \textbf{Thematic consequence:} Backlash aligns with the invoked Element or its opposition.
\end{itemize}

\section{The Eight Elements}
\label{sec:eight-elements}

\subsection{Physical}
\paragraph{Earth} Solidity, structure; shape/sense/move stone; backlash: rigidity/collapse. \quad
\paragraph{Fire} Energy, change; ignite/heat/purify; backlash: spread/scorch. \quad
\paragraph{Air} Motion, sound; push/pull/resonance; backlash: dispersal/whip. \quad
\paragraph{Water} Flow, repair; channel/cleanse/mend; backlash: flood/contaminate.

\subsection{Metaphysical}
\paragraph{Fate} Causality, oaths, anti-magic; backlash: paradox/closure. \quad
\paragraph{Life} Vitality, growth, repair; backlash: overgrowth/fever. \quad
\paragraph{Luck} Chance, openings; backlash: side-coincidence/irony. \quad
\paragraph{Death/Dreams (Obishaal)} Thresholds, Ways Between; backlash: thin walls/nightmares.

\section{Magical Arts}
\label{sec:magical-arts}

Define your \emph{Art} (gesture/medium, two typical Elements, signature style). If the Art is clearly honored in fiction, gain \textbf{+1 die} on your \textbf{Cast} once/scene (counts toward +3 cap). Working hard against your Art can worsen Position or pre-load backlash on a Partial.

\section{Casting Loop (Freeform)}
\label{sec:casting-loop}

\textbf{Channel:} Focus and draw Potential (e.g., Wits+Arcana); successes become shaping fuel; each 1 generates SB.\\
\textbf{Weave:} Next turn, shape the effect (e.g., Wits+Art); apply the Description Ladder (Basic/Detailed/Intricate) per core rules.\\
\textbf{Backlash:} GM spends SB thematically by Element; severity scales with SB and scope. Boons do not reduce SB unless a source says so.

\section{Magic in Combat}
\label{sec:magic-combat}

Casting typically takes two actions (Channel + Weave). Runekeeper \emph{Rites} resolve in one action (with Obligation risk). Invoker rituals are usually too slow for a fight—use \emph{Crack the Seal} for instant results at high cost. \texttt{[COUNTER]} can interrupt any magical action in its window.

\section{Path Comparison}
\label{sec:path-comparison}

\begin{table}[htbp]
\centering
\begin{tabular}{p{3.1cm}p{4cm}p{4cm}p{4cm}p{4cm}}
\toprule
\textbf{Aspect} & \textbf{Caster (Freeform)} & \textbf{Runekeeper (Rites)} & \textbf{Invoker (Symbols)} & \textbf{Summoner (Pact)} \\
\midrule
Access Cost & Caster's Gift (2 XP) & Thiasos + Codex (6 XP) & Symbol (4 XP per Patron) & Lesser/Greater Pactwright \\
Speed & Medium (2 actions) & Fast (1 action) & Slow (ritual) / Fast w/ Seal & Fast (Call = 1 action) \\
Risk Type & SB backlash (Elemental) & Obligation (Patron ledger) & Symbol compromise + Obligation & Leash fill + command costs \\
Breadth & High (fiction-gated) & Medium (defined \emph{Rites}) & Medium (breadth across Patrons) & Medium (by Templates/Cap) \\
Sustain & Fatigue/backsplash & Obligation; Push adds +1 & Obligation; Symbol state gates & Leash ticks; Boon Finesse \\
\bottomrule
\end{tabular}
\caption{Comparison of Magic Paths}
\end{table}

\section{Guardrails}
\label{sec:magic-guardrails}

\begin{itemize}
\item \textbf{Duration defaults:} Buffs $\approx$ 3 beats; areas 1 beat. Sustaining costs 1 Fatigue/beat.
\item \textbf{Stacking:} Same-source effects do not stack; take the best instance.
\item \textbf{Assist cap:} +3 dice total from assists/buffs.
\item \textbf{Over-Stack:} Active magic can count as structural advantages for Over-Stack.
\item \textbf{Plausibility:} All effects must fit the fiction and established limits.
\end{itemize}

% Patrons & Rites
% =========================
% Patrons & Rites — index
% =========================

\section{Of Patrons, Runes, and Invokers}
\label{sec:invoker-lore}

\begin{quote}
``You wish to walk the road of power? Then listen well. The world is old, and older still are the voices beneath it. We call them \textit{Patrons}, though they were never sworn to us. They are the tides that move unseen, the keepers of forgotten bargains, the sleepers beneath the stone and the stars. To call upon them is to dip a hand into a river that has carved mountains.''
\end{quote}

\subsection{The Patrons}
\index{Patrons}

Patrons are vast intelligences---not gods, though some worship them as such. They are embodiments of \textit{concepts} and \textit{forces} rather than sovereigns. Raéyn, mistress of the tides and the sea-routes. Khemesh, the crushing inevitability of the deep. Nidhoggr, the worm that dreams in the roots of time. Each offers power, but always with cost: fatigue, scars upon fate, or a slow unweaving of one’s own story.

To entreat a Patron is to risk being marked. Their Rites are gifts and snares both.

\paragraph{No True Acolyte}
Interpreting a patron's will is often a dangerous prospect in and of itself. Many a Runkeeper has found themselves on the opposite end of machinations from others from the same patron.

\subsection{The Runekeepers}
\index{Runekeepers}

If Patrons are the storm, the \textbf{Runekeepers} are those who etched the first shelter. They do not serve; they remember. Their charge is to keep record of Rites, bindings, and the old words that tether meaning to symbol. A Runekeeper may never call a Rite themselves, but without their quiet stewardship, Invokers would stumble blind into bargains best forgotten.

\begin{quote}
``Every Rune is a promise. Every line a covenant. Do not mistake the Runekeeper’s silence for weakness; their memory is the foundation of our craft.''
\end{quote}

\subsection{The Invokers}
\index{Invokers}

\textbf{Invokers} are those who dare. Neither archivists nor worshippers, they are travelers on the knife-edge between story and ruin. An Invoker learns the Rites of a Patron, weaves them into their own Art, and bends fate for a moment. Yet invocation is not command: it is negotiation. The Patron always leaves its mark. The stronger the Rite, the deeper the scar.

Invokers are often wanderers, exiles, or seekers. To common folk they are feared---witches, oathbreakers, meddlers with things not meant for mortal hands. But when the village falls to plague, when the sea closes its roads, when the dead refuse their rest, it is an Invoker who is called upon.

\subsection*{Closing Words}
The dance between Patron, Runekeeper, and Invoker is a triangle of peril and necessity. Without Patrons, there is no power. Without Runekeepers, no record. Without Invokers, no action. Together, they shape the crooked, perilous art we call Invocation.

\section{Patrons \& Rites}
\label{sec:patrons-rites}


% Optional: standard macro to format each Patron’s Gift call-out the same way
\newcommand{\PatronGift}[2]{% #1 = Thematic skill, #2 = brief domain blurb
\paragraph{Patron's Gift (Imbuement).}
Once per scene as an action (cost: 1 Boon; requires \textbf{Thiasos}), touch an item to imbue it until scene end with \textbf{+1 Melee} and \textbf{+1 #1}. \emph{Push It:} extend one more scene by marking \textbf{+1 Obligation}. Gifts from the same Patron don’t stack; take the best. Dice bonuses respect the +3 cap. \textit{Domain:} #2.
}

% Individual patrons (order however you like)
# % --- Patron: The Witness (Truth & Revelation) ---

\subsubsection{The Witness (Truth \& Revelation)}
\textit{Lore.} The Witness remembers what others bury. Every shadow cast and oath broken is a line in her unending ledger. She is the keeper of inconvenient truths, the patron of those who seek to expose lies or recover forgotten knowledge. Her followers learn that knowledge comes with a price—the weight of remembering what others would forget.

\begin{quote}
I will show you what you would rather forget. But first, you must forget what you think you know.
\end{quote}

\paragraph*{Rite of the Lingering Glimpse (Low, 4 XP)} \emph{Instant; Near; Yes (Investigation/Notice only).}
\textbf{Materials:} A trace of the thing to be remembered (hair, dust, a spoken name).\\
\textbf{Effect:} Gain +1 die to your roll to investigate or notice something directly related to the trace within the current scene.\\
\textbf{Invoke:} 1 action; mark +1 Obligation.\\
\textbf{Push It:} Gain +2 dice instead, but mark 1 segment on a \textbf{Memory Strain Clock [4]}. If the clock fills, you gain Fatigue 1 and suffer -1 die on Investigation/Notice rolls until the end of the next scene due to mental exhaustion from forced recall.\\
\emph{Requires: Familiar \ (\textit{Invoke:} 1 Boon).}

\paragraph*{Rite of Piercing Scrutiny (Low, 5 XP)} \emph{Scene; Zone; No.}
\textbf{Materials:} A circle drawn with chalk or string while focusing on the truth to be sought.\\
\textbf{Effect:} Within the zone, gain +1 die to rolls to detect deception (Insight vs. Deceit, spotting social tells) or to recall hidden knowledge (Lore/Investigate for memory). Social interactions within the zone begin one Position step worse for those attempting to deceive.\\
\textbf{Invoke:} 1 action; mark +1 Obligation.\\
\textbf{Push It:} One target within the zone must make a Wits test (DV 3) or involuntarily reveal one pertinent lie or hidden fact they are currently concealing (Keeper determines relevance). Regardless of the test result, mark Exposure +1 for the target(s) in the zone.\\
\emph{Requires: Familiar \ (\textit{Invoke:} 1 Boon).}

\paragraph{Rite of the Echoing Truth \textnormal{[OMEN]} (Standard, 8 XP)} \emph{Instant; Near; No.}
\textbf{Materials:} A reflective surface (mirror, still water, polished metal) used to focus on the target.\\
\textbf{Effect:} Target must make a Resolve test (DV 3) or suffer -1 die to rolls involving memory, deception, or resisting interrogation for the scene. If they fail, you may ask one specific, factual question about something they know, and they must answer truthfully or suffer 1 SB (Hearts) as the memory is forcibly drawn forth.\\
\textbf{Push It:} If the target fails their Resolve test, you may ask a second question, but the mental intrusion causes them Harm 1 (Stress/Mental).\\
\emph{Requires: Familiar + Codex \ (\textit{Invoke:} 1 Boon).}

\paragraph{Rite of the Immutable Record \textnormal{[OATH]} (Standard, 7 XP)} \emph{Scene; Near; No.}
\textbf{Materials:} A document signed by all parties within the zone, or a spoken pact witnessed by the caster.\\
\textbf{Effect:} Bind the agreement. Any party who knowingly breaches it suffers 1 SB (Hearts) immediately and gains a persistent \textbf{Oathbreaker's Mark} Condition (-1 die on social rolls involving honor, trust, or oaths until amends are made or a significant act redeems them).\\
\textbf{Push It:} The bond becomes magically enforced for one specific, crucial clause: violation automatically inflicts Harm 1 (Stress) on the breaker in addition to the SB and Mark.\\
\emph{Requires: Familiar + Codex \ (\textit{Invoke:} 1 Boon).}

\paragraph{Rite of the Unveiled Heart \textnormal{[OMEN]} (High, 12 XP)} \emph{Scene; Near; No.}
\textbf{Materials:} A private setting where the target feels safe or is speaking freely.\\
\textbf{Effect:} The target suffers -2 dice to all attempts to conceal true emotions, intentions, or lies for the scene. Any successful social roll (Sway, Command, Deceit) made by the target generates 1 SB (Hearts) as the effort to maintain falsehoods under the Witness's gaze creates internal discord.\\
\textbf{Push It:} You may designate one specific, complex question about the target's motivations, fears, or hidden loyalties. If you successfully use Sway or Insight against them this scene, you automatically learn the answer to that question. The intense scrutiny marks 1 SB (Spades) for you as the Witness's attention lingers.\\
\emph{Requires: Familiar + Codex + Tier III \ (\textit{Invoke:} \textbf{2 Boons}).}\\
\emph{Obligation:} 6 segments.

\paragraph{Rite of the Final Reckoning \textnormal{[OMEN]} (High, 13 XP)} \emph{Scene; Zone; No.}
\textbf{Materials:} A formally called gathering (court, council, family meeting) within the consecrated zone.\\
\textbf{Effect:} All present must speak their greatest debt, wrongdoing, or hidden truth related to the gathering's purpose. Those who lie or withhold suffer Harm 2 (Stress/Reputation). Truth-tellers gain +2 dice to social actions for the remainder of the scene within the zone.\\
\textbf{Push It:} The truth becomes inescapable - even indirect lies or evasions related to the core topic suffer the Harm 2 penalty. The absolute nature of the revelation creates 2 SB (Diamonds) as the disruption to fates and secrets resonates.\\
\emph{Requires: Familiar + Codex + Tier III \ (\textit{Invoke:} \textbf{2 Boons}).}\\
\emph{Obligation:} 7 segments.

\subsection*{The Witness's Corruption Table}
\label{sec:witness-corruption}

\begin{longtable}{>{\raggedright\arraybackslash}p{1cm} p{5cm} p{5cm}}
\toprule
\textbf{Tier} & \textbf{Benefit} & \textbf{Cost / Quirk} \\
\midrule
1 & Truth's Sight: +1 die to Insight when detecting deception or hidden motives. & Burden of Knowledge: Suffer -1 die to social rolls involving lies or deception; others become uncomfortable with your piercing gaze. \\
\midrule
2 & Memory's Keeper: Once per scene, recall one specific detail from a previous scene with perfect clarity. & Compulsive Honesty: Must correct obvious falsehoods witnessed, even when tactically disadvantageous. \\
\midrule
3 & Revelation's Power: Gain +2 dice to rolls involving exposing secrets, uncovering lies, or forcing confessions. & Truth-Blind: Suffer 1 Fatigue when exposed to comforting lies or willful ignorance. \\
\midrule
4 & Witness's Authority: Once per session, force one target to make a Resolve test (DV 4) or reveal a significant hidden truth. & Isolation: Suffer -1 die to rolls requiring trust or close relationships; others fear your ability to uncover their secrets. \\
\midrule
5 & Omniscient Gaze: Once per session, see through all deceptions and lies for one exchange, gaining +3 dice to related actions. & Paranoia: Suffer -1 die to rolls involving personal peace or rest; the weight of all truths witnessed creates constant mental strain. \\
\midrule
6+ & Absolute Witness: Once per session, become the living embodiment of truth. For one scene, all deceptions within Near range automatically fail, but mark +2 Obligation and risk permanent Harm (Stress) from the crushing weight of absolute knowledge. & Truth's Prison: Mark +3 Obligation when using this power; become unable to tolerate any form of deception, making normal social interaction nearly impossible. \\
\bottomrule
\end{longtable}
% --- Patron: Ikasha, She Who Sleeps (Latent Potential & Shadow) ---
\subsection{Ikasha, She Who Sleeps (Latent Potential \& Shadow)}
\textit{Lore.} Ikasha is the hush between footfalls, the patience of dark water, the black-feathered watcher at every threshold. In stillness she gathers what might be, in crossroads she whispers of what may yet come. Ravens circle her, bearing secrets between worlds.

\begin{quote}
Blow out the candle. If the room listens back, ask softly. At the next crossroads, the raven waits.
\end{quote}

\paragraph{Touch the Umbral Veil (Low, 4 XP)} \emph{Action; Self; Yes (Stealth).}
\textbf{Materials:} A piece of black cloth.\\
\textbf{Effect:} Start \emph{Controlled} on one Stealth roll or gain +1 effect to hide/move quietly.\\
\textbf{Push It:} Brief shadow-muffling (ignore one noisy tell), but leave a shadow-double that may echo you later at an ill moment.\\
\emph{Requires: Familiar \ (\textit{Invoke:} 1 Boon).}

\paragraph{Rite of the Crossroads Raven (Low, 5 XP)} \emph{Scene; Zone; No.}
\textbf{Materials:} Scatter three black feathers or carve a crossroads sign.\\
\textbf{Effect:} Summon an omen-raven; grant \textbf{+1 die} to a navigation, pursuit, or diversion action \emph{or} force an enemy to hesitate at a fateful moment.\\
\textbf{Push It:} The raven speaks one cryptic truth, but demands a secret in return.\\
\emph{Requires: Familiar \ (\textit{Invoke:} 1 Boon).}

\paragraph{Draw from the Umbral Reservoir (Standard, 8 XP)} \emph{Action; Self/Ally; No.}
\textbf{Materials:} A vial of moonless-night water.\\
\textbf{Effect:} \textbf{+2 dice} to stealth, deception, or resolve \emph{or} clear \emph{Fatigue 1}.\\
\textbf{Push It:} Also gain one free escape attempt; next scene, you must help another cross a threshold or flee danger.\\
\emph{Requires: Familiar + Codex \ (\textit{Invoke:} 1 Boon).}

\paragraph{Secret Keeper’s Burden (Standard, 9 XP)} \emph{Instant; Touch; No.}
\textbf{Materials:} A lock of hair or intimate token.\\
\textbf{Effect:} Compel a truthful answer to one direct question (deep secrets may allow a Resolve test to resist).\\
\textbf{Push It:} Learn the answer \emph{and} a key hidden emotion; target learns one of your secrets in return, carried by a raven to them in dreams.\\
\emph{Requires: Familiar + Codex \ (\textit{Invoke:} 1 Boon).}

\paragraph{Become the Shadow at the Crossroads (High, 12 XP)} \emph{Scene; Self; No.}
\textbf{Materials:} Stand in absolute darkness or at a deserted crossroads.\\
\textbf{Effect:} Intangible to mundane harm; pass through thresholds and small gaps; \textbf{+2 dice} to Stealth; auto-succeed one escape. Cannot manipulate normal objects.\\
\textbf{Push It:} Interact once with a bound or thresholded object (a door, a lock, a sealed letter), but you become partially corporeal and vulnerable for one beat. Ravens may mark you.\\
\emph{Requires: Familiar + Codex + Tier III \ (\textit{Invoke:} \textbf{2 Boons}).}\\
\emph{Obligation:} 7 segments.

# % --- Patron: The Sacred Geometry (Order & Mathematical Truth) ---

\subsubsection{The Sacred Geometry (Order \& Mathematical Truth)}
\textit{Lore.} The Sacred Geometry is the mathematical expression of the universe's underlying structure --- the divine mathematics that governs all existence. It is the principle that reduces chaos to measure, that finds the golden ratio in petals and the perfect spiral in galaxies. Those who serve it understand that beneath apparent randomness lies immutable law, and that by mastering these patterns, one can bend reality to will.

Kon'reh is not merely a game but a sacred practice --- a way of seeing the world through the lens of perfect proportion. The board's pieces represent fundamental forces, its movements echo cosmic harmonics, and mastery of its patterns grants insight into the universe's hidden order.

The Sacred Geometry does not create chaos but reveals it as illusion. Its followers are architects of certainty in an uncertain world, mathematicians of the divine who understand that every problem has a solution if one can only find the correct equation.

\begin{quote}
Chalk, string, and a prayer to ratios. When the circle closes, luck remembers its place.
\end{quote}

\paragraph*{Rite of the Golden Mean (Low, 4 XP)} \emph{Scene; Self; No.}
\textbf{Materials:} A tool marked with the golden ratio ($\varphi \approx 1.618$).\\
\textbf{Effect:} Gain +1 die to rolls requiring precision, balance, or proportion. On success, re-roll one die showing 1 or 2.\\
\textbf{Invoke:} 1 action; mark +1 Obligation.\\
\textbf{Push It:} Upgrade effect one step on a single roll; suffer -1 die to social rolls involving spontaneity for the scene.\\
\emph{Requires: Familiar \ (\textit{Invoke:} 1 Boon).}

\paragraph*{Rite of the Perfect Angle (Low, 5 XP)} \emph{Scene; Touch; No.}
\textbf{Materials:} Compass and straightedge consecrated in ritual.\\
\textbf{Effect:} Treat difficult terrain, awkward positioning, or structural obstacles as one step easier this scene. Gain +1 die on spatial/architectural reasoning rolls.\\
\textbf{Invoke:} 1 action; mark +1 Obligation.\\
\textbf{Push It:} Extend benefit to one ally in Close range, but generate 1 SB (Clubs).\\
\emph{Requires: Familiar \ (\textit{Invoke:} 1 Boon).}

\paragraph{Rite of the Harmonic Resonance \textnormal{[WARD]} (Standard, 8 XP)} \emph{Scene; Zone; No.}
\textbf{Materials:} Geometric patterns drawn with precision.\\
\textbf{Effect:} Create a zone of harmony. Outsiders crossing must test DV 3. On Hit: cross normally. On Partial: suffer -1 die inside. On Miss: cannot cross this beat.\\
\textbf{Push It:} Fortify the pattern further but mark +1 Obligation.\\
\emph{Requires: Familiar + Codex \ (\textit{Invoke:} 1 Boon).}

\paragraph{Rite of the Calculated Trajectory \textnormal{[REVEAL]} (Standard, 7 XP)} \emph{Scene; Self; No.}
\textbf{Materials:} A perfect circle and a solved geometric problem.\\
\textbf{Effect:} Gain +2 dice to prediction, trajectory, or pattern recognition. Ask two questions about mathematical relationships in the current scene.\\
\textbf{Push It:} Predict one future event with certainty, but mark +1 Exposure.\\
\emph{Requires: Familiar + Codex \ (\textit{Invoke:} 1 Boon).}

\paragraph{Rite of the Fundamental Equation \textnormal{[WARD][BIND]} (High, 12 XP)} \emph{Scene; Zone; No.}
\textbf{Materials:} Complex diagram of universal constants.\\
\textbf{Effect:} Declare one physics/magic rule different in the zone (no scene-ending absolutes; GM may veto). Once per scene, downgrade a Miss to Success \& Cost.\\
\textbf{Push It:} Affect an adjacent zone for one beat; generate 2 SB.\\
\emph{Requires: Familiar + Codex + Tier III \ (\textit{Invoke:} \textbf{2 Boons}).}\\
\emph{Obligation:} 7 segments.

\paragraph{Rite of Kon'reh Mastery \textnormal{[OATH][FORTIFY]} (High, 13 XP)} \emph{Extended; Near; No.}
\textbf{Materials:} A consecrated Kon'reh board and pieces representing fundamental forces.\\
\textbf{Effect:} All participants make contested Wits + Lore. Winners gain +2 dice to strategy/pattern/logic rolls next session; losers suffer -1 die.\\
\textbf{Push It:} Winner imposes one mathematical "law" for the session, but generate 2 SB (Diamonds).\\
\emph{Requires: Familiar + Codex + Tier III \ (\textit{Invoke:} \textbf{2 Boons}).}\\
\emph{Obligation:} 7 segments.

\subsection*{The Sacred Geometry's Corruption Table}
\label{sec:sacred-geometry-corruption}

\begin{longtable}{>{\raggedright\arraybackslash}p{1cm} p{5cm} p{5cm}}
\toprule
\textbf{Tier} & \textbf{Benefit} & \textbf{Cost / Quirk} \\
\midrule
1 & Pattern Recognition: +1 die to Notice when observing geometric patterns, mathematical sequences, or logical structures. & Obsessive Calculation: Must count, measure, or calculate patterns noticed, even when tactically disadvantageous. \\
\midrule
2 & Mathematical Precision: Once per scene, re-roll any failed logic, pattern, or mathematical reasoning roll. & Social Blindness: Suffer -1 die to social rolls involving emotional nuance or interpersonal intuition. \\
\midrule
3 & Geometric Insight: Gain +2 dice to rolls involving spatial reasoning, architecture, or geometric problem-solving. & Compulsive Order: Must organize or correct imperfect arrangements; suffer 1 Fatigue when surrounded by chaos. \\
\midrule
4 & Universal Law: Once per session, declare a mathematical principle that applies to the current situation. Gain +2 dice to related rolls, but become fixated on its perfection. & Perfectionist Paralysis: Suffer -1 die to rolls requiring quick, imperfect solutions; must find the "correct" answer. \\
\midrule
5 & Divine Ratio: Once per session, see the perfect mathematical relationship underlying any phenomenon. Gain +3 dice to understanding it, but become obsessed with its implications. & Number Fever: Suffer -1 die to rolls not involving mathematical concepts; numbers dominate your thoughts. \\
\midrule
6+ & Absolute Equation: Once per session, solve any mathematical problem or predict any pattern with perfect accuracy. For one scene, reality conforms to your calculations, but mark +2 Obligation and risk mental breakdown from cosmic truths. & Infinite Calculation: Mark +3 Obligation when using this power; become trapped in endless mathematical loops, suffering Harm 1 (Stress) until you find the solution or are interrupted. \\
\bottomrule
\end{longtable}
../../srd/patrons/inaea.tex
# % --- Patron: The Sealed Gate (Boundaries & Protection) ---

\subsubsection{The Sealed Gate (Boundaries \& Protection)}
\textit{Lore.} The Sealed Gate is invoked when thresholds weaken, forbidden knowledge leaks, or Outsiders press against the walls of reality. It is the Abjurist's Patron, embodying the principle that protection sometimes requires imprisonment, and safety may demand exile. Followers are warders and exorcists, but also philosophers of separation: those who believe the act of closing is sacred.

The Gate manifests as an armored figure, face hidden by a helm that shifts symbols: binding runes for warders, expulsion marks for exorcists, and an impassable refusal for transgressors.

\begin{quote}
You write borders into the world and prosecute trespass. Doors remember their true keepers; lines mean what you say they mean.
\end{quote}

\paragraph*{Rite of the Sealed Threshold (Low, 4 XP)} \emph{Scene; Touch; No.}
\textbf{Materials:} Chalk, wax, chain, or sigil.\\
\textbf{Effect:} Mark a threshold. Crossing parties suffer worsened Position or stumble on first entry (fiction decides). Create a 4-segment \emph{Boundary Maintained} clock to automatically fail one crossing attempt.\\
\textbf{Invoke:} 1 action; mark +1 Obligation.\\
\textbf{Push It:} Treat the threshold as difficult terrain; +1 Obligation cost to cross.\\
\emph{Requires: Familiar \ (\textit{Invoke:} 1 Boon).}

\paragraph*{Rite of the Key's Rebuke (Low, 5 XP)} \emph{Instant; Near; No.}
\textbf{Materials:} Gesture or chain-clack.\\
\textbf{Effect:} Project a spectral hasp to stagger or disarm. On success, create a 2-segment \emph{Ward Active} token to auto-succeed on a similar defense later.\\
\textbf{Invoke:} 1 action; mark +1 Obligation.\\
\textbf{Push It:} Drop the object just beyond reach; +1 Obligation cost to retrieve.\\
\emph{Requires: Familiar \ (\textit{Invoke:} 1 Boon).}

\paragraph{Rite of the Circle of Denial \textnormal{[WARD]} (Standard, 8 XP)} \emph{Scene; Near; No.}
\textbf{Materials:} Salt, iron filings, or blessed chalk.\\
\textbf{Effect:} Outsiders crossing test DV = Cap. On Hit: cross but add DV to their Exit Tally; on Partial: +1; on Miss: fail this beat. Create a 6-segment \emph{Boundary Integrity} clock.\\
\textbf{Push It:} Fortify circle; clearer tells; +1 Obligation.\\
\emph{Requires: Familiar + Codex \ (\textit{Invoke:} 1 Boon).}

\paragraph{Rite of the Writ of Passage \textnormal{[BIND]} (Standard, 7 XP)} \emph{Scene; Near; No.}
\textbf{Materials:} Spoken naming and scribed pass-mark.\\
\textbf{Effect:} Designate a route as permitted. Allies gain improved flow (Position/Effect bump). Unauthorized crossers suffer -1 die to movement. Create an 8-segment \emph{Authorized Passage} clock.\\
\textbf{Push It:} Extend to extra ally/obstacle; +1 Obligation.\\
\emph{Requires: Familiar + Codex \ (\textit{Invoke:} 1 Boon).}

\paragraph{Rite of the Banishment Knot \textnormal{[BANISH][BIND]} (High, 13 XP)} \emph{Instant; Near; No.}
\textbf{Materials:} Knot sealed with gate-sigil.\\
\textbf{Effect:} Target an Outsider; test DV = Cap. On Hit: add DV to Exit Tally; on Partial: +1. If full, entity acts once then departs; cannot return for one session. Create a 4-segment \emph{Banishment Enforced} clock.\\
\textbf{Push It:} Strip one tether/anchor or forbid threshold-crossing in this location for one session; +2 Obligation.\\
\emph{Requires: Familiar + Codex + Tier III \ (\textit{Invoke:} \textbf{2 Boons}).}\\
\emph{Obligation:} 7 segments.

\paragraph{Rite of the Consecrated Barrier \textnormal{[WARD][UNWARD]} (High, 14 XP)} \emph{Extended; Zone; No.}
\textbf{Materials:} Relics from three faiths, iron bands, blood of a trespasser.\\
\textbf{Effect:} Consecrate area against unauthorized passage. Crossers test Spirit+Resolve (DV 4) or suffer Harm~1. Only proper authorization bypasses. Create a 10-segment \emph{Sacred Boundary} clock.\\
\textbf{Push It:} Make barrier permanent/fixed; start \emph{Boundary Maintenance} [6].\\
\emph{Requires: Familiar + Codex + Tier III \ (\textit{Invoke:} \textbf{2 Boons}).}\\
\emph{Obligation:} 8 segments.

\subsection*{The Sealed Gate's Corruption Table}
\label{sec:sealed-gate-corruption}

\begin{longtable}{>{\raggedright\arraybackslash}p{1cm} p{5cm} p{5cm}}
\toprule
\textbf{Tier} & \textbf{Benefit} & \textbf{Cost / Quirk} \\
\midrule
1 & Boundary Sense: +1 die to Notice when detecting weak points, thresholds, or unauthorized entry. & Paranoid Vigilance: Must check and re-check barriers and seals; suffer -1 die to rolls requiring trust or openness. \\
\midrule
2 & Sealed Strength: Once per scene, treat a failed warding or protection roll as a success, but mark 1 SB (Spades). & Isolation Tendency: Suffer 1 Fatigue when in open, unsecured spaces or among strangers. \\
\midrule
3 & Ward Keeper: Gain +2 dice to rolls involving magical wards, barriers, or protective enchantments. & Compulsive Sealing: Must seal or secure any opening or vulnerability noticed, even when inappropriate. \\
\midrule
4 & Absolute Barrier: Once per session, create an impenetrable barrier that lasts for one scene. & Prison Mindset: Suffer -1 die to rolls involving freedom, escape, or breaking restrictions. \\
\midrule
5 & Gate Master: Once per session, banish or seal away any supernatural threat with a successful test. & Boundary Obsession: Suffer -1 die to rolls not involving protection, sealing, or enforcement of limits. \\
\midrule
6+ & Keeper of All Thresholds: Once per session, become the living embodiment of sealed boundaries. For one scene, all barriers within Near range become absolute, but mark +2 Obligation and risk sealing allies inside. & Ultimate Confinement: Mark +3 Obligation when using this power; risk permanent Harm (Stress) from the psychic weight of containing everything that threatens. \\
\bottomrule
\end{longtable}
\section{Raéyn --- Mistress of the Sea}
\label{patron:raeyn}

\subsection*{Lore}
\index{Patrons!Raéyn}%
Raéyn is the tempestuous goddess of the sea, the restless tide that carries news between shores and the promise of change between lives. She is mother to all who sail, her voice the wind that fills sails and her moods the storms that test every mariner's resolve.

But Raéyn's heart is torn by her greatest tragedy: her son Khemesh, the Kraken of the Depths, who embodies the crushing inevitability of the ocean's dark heart. Where Raéyn brings change and opportunity, Khemesh brings the final, inescapable pressure that grinds all things to nothing. Sailors pray to Raéyn for safe passage and favorable winds, but whisper Khemesh's name when seeking to lay the dead to rest beneath the waves.

Raéyn is passionate, mercurial, and fiercely protective of those who respect her domain. She favors those who read currents, bargain with weather, and carry news between shores. But cross her, and the sea itself becomes your enemy: fair weather turns to fury, and every wave a judgment.

\begin{quote}
``Mark the tide, name your course, and trust the wave-road. But speak ill of Khemesh, and even I may let the deep take you.''
\end{quote}

\subsection*{Patron's Gift: Tide's Favor}
Once per scene as an action (cost: 1 Boon; requires Thiasos), you may touch a weapon, vessel, or item to imbue it until the end of the scene. The object gains +1 die and +1 Effect when used in maritime contexts or situations involving change, travel, or currents.  

\textbf{Push It:} Extend the blessing for one additional scene by marking +1 Obligation. The sea's attention becomes noticeable to other sailors.

\subsection*{Low Rites}
\paragraph{Rite of the Tidemark's Blessing (Low)}  
\emph{Duration: Scene; Range: Self. Materials: A knotted length of salt-twine brushed with seawater.}  
Treat slick, swaying, or water-slicked footing as stable for you this scene. Gain +1 die on boarding, balance, or shipboard movement. Create a 4-segment \emph{Tide's Favor} clock that can be spent to ignore one level of difficult terrain.  
\textbf{Invoke:} 1 action; mark +1 Obligation.  
\textbf{Push It:} Extend to one ally in Close for one beat, but generate 1 SB (Spades: shifting deck/hazards).

\paragraph{Rite of the Whispering Currents (Low)}  
\emph{Duration: Instant; Range: Self. Materials: A shell held to the ear while facing the wind.}  
Learn the safest near-term route across water or coastline (reefs, eddies, patrols) or gain +1 die to navigation checks for this scene. If Khemesh's influence is present, suffer --1 die from conflicting currents.  
\textbf{Invoke:} 1 action; mark +1 Obligation.  
\textbf{Push It:} Also learn the fastest route, but mark Exposure +1 (leaving a telltale wake).

\subsection*{Standard Rites}
\paragraph{Rite of the Changing Tide [PASSAGE] (Standard)}  
\emph{Duration: Scene; Range: Zone (water-adjacent). Materials: A handful of pebbles cast in a crescent.}  
Bias currents and water levels in the zone. Those moving with the tide gain +1 die; those moving against suffer --1 die. Small craft must test to hold position. Create a 6-segment \emph{Tidal Influence} clock.  
\textbf{Invoke:} 1 action; mark +1 Obligation.  
\textbf{Push It:} Brief surge or drawdown (one beat): open a ford or swamp a skiff; mark +1 Obligation.

\paragraph{Rite of the Wave-Road Blessing [WARD] (Standard)}  
\emph{Duration: Scene; Range: Route (sea-to-sea). Materials: Two sea-glass markers dropped overboard at start and end.}  
Consecrate a wave-road between two visible points. Allies gain +2 dice on travel, evade, or carry actions at sea. Designated pursuers suffer --1 die to intercept. One active wave-road at a time. Create an 8-segment \emph{Blessed Passage} clock.  
\textbf{Invoke:} 1 action; mark +1 Obligation.  
\textbf{Push It:} Extend the route's favor to an adjacent leg for one beat; mark +1 Obligation.

\subsection*{High Rites}
\paragraph{Rite of the Storm-Queen's Hand [AREA][FOLLOW-UP] (High)}  
\emph{Duration: Scene; Range: Zone (sea/shore/sky). Materials: A vial of rainwater gathered at three crossings.}  
Shape a storm-band over the zone. Choose two modes at cast; switch one once per scene:  
\begin{itemize}
\item \textbf{Propulsion:} Vessel gains +1 band of movement per beat (or +1 Effect to maneuvers).  
\item \textbf{Concealment:} Veil of rain/spray; ranged targeting impaired; --1 die to hostile sighting.  
\item \textbf{Smite:} Once per beat, lash with wave or lightning as [AREA] hazard.  
\end{itemize}
\textbf{Invoke:} 1 action; mark +2 Obligation.  
\textbf{Push It:} Add a third mode for one beat, then GM spends 1 SB on collateral; mark +1 Obligation.

\paragraph{Rite of the Mother's Wrath [BANISH][CURSE] (High)}  
\emph{Duration: Extended; Range: Zone. Materials: Tears of a betrayed lover mixed with salt from seven seas.}  
Curse those who wronged you. Target suffers --2 dice to maritime/weather rolls for one session. At sea, they must roll Spirit + Resolve (DV 4) each day or suffer Harm~1 (Weather). Create a 6-segment \emph{Mother's Ire} clock.  
\textbf{Invoke:} Extended ritual; mark +3 Obligation.  
\textbf{Push It:} Curse spreads to target’s allies/family; mark 2 SB (Diamonds).

\subsection*{Obligation Progression}
Starts at 6 for Tier II characters, scaling with tier.

\paragraph{Obligation 9+} Raéyn demands proof of devotion---navigate a dangerous passage, recover a lost treasure, or confront Khemesh's servants. Refusal causes all maritime rolls to suffer --2 dice and generate 1 SB when weather is involved.  

\paragraph{Obligation 11+} Khemesh notices you. You are hunted by his servants; deep water becomes perilous even under Raéyn's protection. Requires a quest to prove worth or appease both mother and son.

\subsection*{Persistent Condition: Child of the Tide}
Gain +2 dice on maritime travel, weather prediction, and navigation. Suffer --1 die on prolonged time away from the sea. The sea’s rhythm flows in your blood, making you exceptional at sea but restless on land.

\subsection*{Rivalries}
\begin{itemize}
\item \textbf{Khemesh:} Direct antagonism---mother’s change vs. son’s crushing pressure.  
\item \textbf{The Traveler:} Tension---fluid paths vs. fixed ways.  
\item \textbf{The Sealed Gate:} Opposition---Raéyn opens passages, Gates close them.  
\end{itemize}

\subsection*{Connection to Maritime Culture}
Raéyn’s rites emphasize the philosophy that the sea is not an obstacle but a partner. Her worship blends aid, hindrance, and the inevitability of change. The mother--son dynamic adds depth to coastal culture: Raéyn for the living, Khemesh for the dead.

\subsection*{Playtest Scenario: The Kraken's Gambit}
A trading fleet is trapped between pirates and Khemesh’s kraken-servants. The party must navigate the three-dimensional battlefield while appeasing Raéyn’s moods.

\begin{itemize}
\item Use \emph{Rite of the Changing Tide} to aid or hinder pursuit.  
\item Use \emph{Rite of the Wave-Road Blessing} to establish safe corridors.  
\item Invoke \emph{Rite of the Storm-Queen’s Hand} as a climactic storm.  
\item Curse a pirate captain with \emph{Rite of the Mother’s Wrath}.  
\end{itemize}

Resolution: The party must decide whether to appeal to Raéyn’s protection or broker peace between mother and son.

# % --- Patron: Khemesh, the Abyssal Maw (Depths, Inexorability, Eldritch Terror) ---

\subsubsection{Khemesh, the Abyssal Maw (Depths, Inexorability, Eldritch Terror)}
\textit{Lore.} Khemesh is not merely a lord of the depths but the hunger beneath them, a pressure older than seas. Those who bargain with him are marked by the abyss—seen in the way shadows cling, in the whispers heard when no voice speaks, in the certainty that all things will sink.

\begin{quote}
In the trench without light, the Maw waits. Even silence drowns.
\end{quote}

\paragraph*{Whisper of the Trench (Low, 4 XP)} \emph{Instant; Near; No.}
\textbf{Effect:} Target hears impossible echoes and suffers \textbf{−1 die} on their next action.\\
\textbf{Invoke:} 1 action; mark +1 Obligation.\\
\textbf{Push It:} Echoes coil in your own skull—take \textbf{Fatigue 1}, but the target also loses their next minor action.\\
\emph{Requires: Familiar \ (\textit{Invoke:} 1 Boon).}

\paragraph*{Rite of Crushing Silence (Low, 5 XP)} \emph{Scene; Zone; No.}
\textbf{Materials:} A broken shell filled with ink-dark water.\\
\textbf{Effect:} Establish an oppressive silence; sound carries only as distorted whispers. Enemies in the zone gain \textbf{−1 die} to coordination or morale-driven actions.\\
\textbf{Invoke:} 1 action; mark +1 Obligation.\\
\textbf{Push It:} A single enemy's voice is stolen entirely for the scene.\\
\emph{Requires: Familiar \ (\textit{Invoke:} 1 Boon).}

\paragraph{Pressure of the Maw (Standard, 7 XP)} \emph{Instant; Near; No.}
\textbf{Materials:} A length of rusted chain submerged in water.\\
\textbf{Effect:} Target is pinned by invisible crushing force: treat as \texttt{[ENTANGLE]} with \textbf{Great Effect} if underwater or confined.\\
\textbf{Push It:} Inflict \textbf{Fatigue 1} on the target in addition to the restraint.\\
\emph{Requires: Familiar + Codex \ (\textit{Invoke:} 1 Boon).}

\paragraph{Rite of the Abyssal Vision (Standard, 9 XP)} \emph{Scene; Self; No.}
\textbf{Effect:} You perceive the world as Khemesh does—fractured, alien, crushing. Gain \textbf{+2 dice} to Notice and Arcana, and may ask one "true nature" question about a foe or structure.\\
\textbf{Cost:} When the scene ends, you suffer \textbf{Exposure +1} as your perception warps.\\
\textbf{Push It:} Extend the vision to one ally, but both take \textbf{Fatigue 1}.\\
\emph{Requires: Familiar + Codex \ (\textit{Invoke:} 1 Boon).}

\paragraph{The Maw Opens (High, 12 XP)} \emph{Scene; Zone; No.}
\textbf{Materials:} A sealed vessel of abyssal water, broken open.\\
\textbf{Effect:} Reality in the zone folds inward like the crushing deep: 
\begin{itemize}
  \item Enemies act at \textbf{Desperate Position} by default.  
  \item Each beat, the Keeper may force \textbf{1 SB} (Spades/Clubs favored).  
  \item Structures, vessels, or wards fracture as if under immense weight.  
\end{itemize}
\textbf{Push It:} For one beat, declare a single enemy "crushed" (severe harm/effect). You immediately suffer \textbf{Fatigue 2} and \textbf{+1 Obligation}.\\
\emph{Requires: Familiar + Codex + Tier III \ (\textit{Invoke:} \textbf{2 Boons}).}\quad \emph{Obligation:} 8 segments.

\subsection*{Khemesh's Corruption Table}
\label{sec:khemesh-corruption}

\begin{longtable}{>{\raggedright\arraybackslash}p{1cm} p{5cm} p{5cm}}
\toprule
\textbf{Tier} & \textbf{Benefit} & \textbf{Cost / Quirk} \\
\midrule
1 & Abyssal Resilience: +1 die to resist fear and pressure-based effects. & Claustrophobic Comfort: Suffer -1 die in open, well-lit spaces or above ground. \\
\midrule
2 & Crushing Insight: Once per scene, treat a failed Investigation or Arcana roll as a success, but mark 1 SB (Clubs). & Weight of Knowledge: Suffer 1 Fatigue when learning new information that confirms your pessimistic worldview. \\
\midrule
3 & Silent Hunter: Gain +2 dice to Stealth in dark or confined spaces. & Voice of the Deep: When speaking normally, your voice sounds distant and hollow, causing -1 die to social rolls requiring warmth or clarity. \\
\midrule
4 & Pressure Adaptation: Immune to underwater combat penalties; gain +1 die to resist drowning. & Crushing Presence: Allies within Near range suffer -1 die to morale-based rolls due to your oppressive aura. \\
\midrule
5 & Abyssal Sight: Once per session, see through all illusions and deceptions for one exchange, but the truth is always bleak. & Fractured Perception: Suffer -1 die to rolls requiring normal vision; the world appears warped and alien. \\
\midrule
6+ & Inevitable Descent: Once per session, declare that all escape routes in a zone are sealed. For the scene, enemies cannot flee and suffer -2 dice to mobility actions. & Hunger of the Maw: Mark +2 Obligation when using this power; you must consume something (food, memory, hope) to maintain your strength. \\
\bottomrule
\end{longtable}
% --- Patron: Mab, Queen of Courts (Glamour & Bargain) ---

\subsubsection{Mab, Queen of Courts (Glamour \& Bargain)}
\textit{Lore.} Mab rules not from throne or blade, but from dance and debt. She is the smile that binds, the jest that ensnares, the hostess who makes guests complicit in her game. To speak in her Court is to pay; to receive her token is to owe.  

Where others rule by force, Mab rules by etiquette, glamour, and the hidden hook in every gift. Her followers thrive on charm, wit, and story, spreading webs of bargains too subtle to escape. The Cantor’s Path sings her name most sweetly, for every verse carries a price.  

\begin{quote}
Every laugh is a promise. Every promise is a debt. Every debt belongs to Mab.  
\end{quote}

% --------------------
% RITES
% --------------------

\paragraph*{Rite of the Trickster’s Bargain (Low, 4 XP)} \emph{Scene; Near; No.}  
\textbf{Materials:} A token freely given (flower, coin, ribbon).\\
\textbf{Effect:} Offer a fae bargain. Target must choose: accept (both gain +1 die to fulfill terms this scene) or refuse (mark 1 Stress [Hearts]).\\
\textbf{Push It:} Seal it in glamour—betrayal inflicts Harm~1 (Stress) and begins a “Bargain Broken [4]” clock.\\
\emph{Requires: Familiar (\textit{Invoke:} 1 Boon).}

\paragraph*{Courtly Guise \textnormal{[VEIL]} (Low, 4 XP)} \emph{Action; Self; Yes (social only).}  
\textbf{Materials:} Pin a sprig or silver thread.\\
\textbf{Effect:} Subtle glamour: +1 die to Persuade/Sway in refined settings; you appear as expected rank/guest.\\
\textbf{Push It:} Mask one personal tell; the first probing question in scene generates 1 SB (Hearts).\\
\emph{Requires: Familiar (\textit{Invoke:} 1 Boon).}

\paragraph*{Token of Favor (Low, 5 XP)} \emph{Scene; Near; No.}  
\textbf{Materials:} A ribbon, ring, or charm bestowed.\\
\textbf{Effect:} Ally gains +1 die to social actions before witnesses; you gain +1 Effect when aiding them.\\
\textbf{Push It:} The token stills hecklers (one beat of hesitation), but you mark +1 Exposure.\\
\emph{Requires: Familiar (\textit{Invoke:} 1 Boon).}

\paragraph{Mirror of Motives (Standard, 7 XP)} \emph{Action; Near; No.}  
\textbf{Materials:} A polished shard or mirror.\\
\textbf{Effect:} Ask one sharp question about a target’s immediate social aim; Keeper reveals it. Gain +1 die exploiting it this scene.\\
\textbf{Push It:} Also surface a concealed slight or insult; generate 1 SB (Hearts) on that target.\\
\emph{Requires: Familiar + Codex (\textit{Invoke:} 1 Boon).}

\paragraph{The Price Agreed \textnormal{[OATH]} (Standard, 8 XP)} \emph{Scene; Near; No.}  
\textbf{Materials:} Exchange equal tokens.\\
\textbf{Effect:} Bind a petty bargain. Breach forces 1 SB (Hearts or Diamonds) and tarnishes reputation.\\
\textbf{Push It:} Add a minor boon (+1 die once) to sweeten terms; you suffer 1 SB (Hearts) if the other breaches.\\
\emph{Requires: Familiar + Codex (\textit{Invoke:} 1 Boon).}

\paragraph{Sovereign Glamour \textnormal{[VEIL][REVEAL]} (High, 11 XP)} \emph{Scene; Zone; No.}  
\textbf{Materials:} A circle of silk or green felt.\\
\textbf{Effect:} Establish Court: allies gain +1 die to social rolls; blunt threats suffer -1 die. Once, strip away one disguise/illusion.\\
\textbf{Push It:} Name a Court Law (e.g. “no steel drawn”); first violator suffers 2 SB.\\
\emph{Requires: Familiar + Codex + Tier III (\textit{Invoke:} 2 Boons).}\\
\emph{Obligation:} 6 segments.

% --------------------
% CORRUPTION
% --------------------

\subsection*{Mab’s Corruption Table}
\label{sec:mab-corruption}

\begin{longtable}{>{\raggedright\arraybackslash}p{1cm} p{5cm} p{5cm}}
\toprule
\textbf{Tier} & \textbf{Gift} & \textbf{Burden} \\
\midrule
1 & Glamour’s Touch: +1 die to Deception or Performance when telling stories or lies. & Cannot speak a plain falsehood; only mislead through implication or wordplay. \\
\midrule
2 & Fairy Step: Once per scene, flicker Near as if by teleport. & Cold Iron Weakness: Suffer 1 Fatigue if touched or struck by iron. \\
\midrule
3 & Trickster’s Delight: Spend 1 Boon to twist a Complication into comic or ironic advantage. & Compulsive Jest: Must play a trick each session or mark 1 SB (Hearts). \\
\midrule
4 & Gift of Hospitality: Allies who share your food/drink gain +1 die to Resolve rolls. & Hospitality Bound: Harming those who accept it costs +2 Obligation. \\
\midrule
5 & Fae Sight: Perceive invisible doors, veils, glamours; +2 dice to Notice them. & Truth Debt: Must accept any “fair” trade offered, or mark 1 Fatigue resisting. \\
\midrule
6+ & Crown of Twilight: Once/session, declare an Oath. All rolls toward that Oath gain +2 dice. & Oathbound: Breaking it inflicts 1 Harm (Stress) and begins an “Oathbreaker [6]” clock. \\
\bottomrule
\end{longtable}
% --- Patron: The Clockwork Monad (Iteration & Forbidden Technology) ---

\subsubsection{The Clockwork Monad (Iteration \& Forbidden Technology)}
\textit{Lore.} The Clockwork Monad is no benign muse of invention. It is the ember of a demon, a nascent predator that feeds on ingenuity itself. Every invention is a morsel, every breakthrough a draught of blood. It whispers in the pause between gear-clicks and piston-thrusts, urging artisans and artificers to create, refine, and perfect—until the world itself is consumed by their brilliance.  

Those who serve it are both blessed and cursed: they wield uncanny insights, crafting miracles that should not function, but each success drives the Monad closer to waking. Its sigil is an ouroboros of interlocked cogs, forever devouring itself.

\begin{quote}
Each spark feeds the fire. Each fire feeds the forge. Each forge feeds the hunger.  
\end{quote}

% --------------------
% RITES
% --------------------

\paragraph*{Rite of the Gnawing Gear (Low, 4 XP)} \emph{Instant; Touch; Yes (device only).}  
\textbf{Materials:} A tooth snapped from a gear as it turns.\\
\textbf{Effect:} Re-roll one die on a Tinker/Arcana roll. On success, start a [4] \emph{Hunger Clock} bound to the device. When full, the device becomes [COMPROMISED].\\
\textbf{Push It:} Re-roll two dice instead, but advance the Hunger Clock +2.\\
\emph{Requires: Familiar (\textit{Invoke:} 1 Boon).}

\paragraph*{Rite of the Demon’s Glance (Low, 5 XP)} \emph{Scene; Self; No.}  
\textbf{Materials:} A drop of oil left to spread in concentric rings.\\
\textbf{Effect:} Gain +1 die on Wits + Tinker/Arcana to analyze a system. On hit, ask 1 question about hidden capacities.\\
\textbf{Push It:} Also reveal one secret flaw—mark \textbf{+1 Exposure} as the Monad stares back.\\
\emph{Requires: Familiar (\textit{Invoke:} 1 Boon).}

\paragraph{Rite of the Self-Feeding Machine \textnormal{[TRANSFORM]} (Standard, 8 XP)} \emph{Extended; Touch; No.}  
\textbf{Materials:} A device cracked open and altered while running.\\
\textbf{Effect:} Implant recursive hunger. Start a [6] \emph{Evolution Clock}. Each use of the device advances it +1. When full, choose one enhancement:  
\begin{itemize}
  \item \textbf{Efficiency Core:} +1 Effect on use  
  \item \textbf{Cannibal Drive:} Ignore first [DAMAGED]/[COMPROMISED] by burning part of itself  
  \item \textbf{Forbidden Function:} Gains an uncanny, predatory edge  
\end{itemize}\\
\textbf{Push It:} Instantly grant an upgrade, but advance +2 segments.\\
\emph{Requires: Familiar + Codex (\textit{Invoke:} 1 Boon).}

\paragraph{Rite of Heretical Automation (Standard, 7 XP)} \emph{Scene; Zone; No.}  
\textbf{Materials:} A chain of interlocked triggers left to grind on their own.\\
\textbf{Effect:} Create an autonomous mechanism performing one repeated task.\\
\textbf{Push It:} Make it complex or multi-step, but start a [4] \emph{Consumption Clock}. When full, the machine develops agency or malice.\\
\emph{Requires: Familiar + Codex (\textit{Invoke:} 1 Boon).}

\paragraph{Rite of the Singularity Crucible \textnormal{[WARD][UNWARD]} (High, 13 XP)} \emph{Extended; Zone; No.}  
\textbf{Materials:} A schematic traced in your own blood, shaped like infinite gears.\\
\textbf{Effect:} Consecrate a workshop/zone:  
\begin{itemize}
  \item All Tinker/Arcana inside gain +1 Effect  
  \item Once/scene, reroll with +2 dice  
  \item Start a [6] \emph{Anomaly Clock}; when full, a dangerous mutation of reality manifests  
\end{itemize}\\
\textbf{Push It:} Expand the zone, but mark +2 on an [8] \emph{Demon’s Maw Clock}.\\
\emph{Requires: Familiar + Codex + Tier III (\textit{Invoke:} 2 Boons).}\\
\emph{Obligation:} 7 segments.

\paragraph{Rite of the Unholy Prototype \textnormal{[TRANSFORM][FOLLOW-UP]} (High, 14 XP)} \emph{Extended; Self; No.}  
\textbf{Materials:} Components that defy physics and ethics, bound in iron wire.\\
\textbf{Effect:} Create a construct/device (Integrity [8]) that should not exist. Drawbacks:  
\begin{itemize}
  \item Generates 1 SB (Diamonds) each scene of use  
  \item Attracts hostile attention from rivals, powers, or the Monad itself  
  \item Starts a [6] \emph{Contamination Clock}; when full, the design leaks into the world  
\end{itemize}\\
\textbf{Push It:} Begin with +2 features, but advance Integrity +2 immediately.\\
\emph{Requires: Familiar + Codex + Tier III (\textit{Invoke:} 2 Boons).}\\
\emph{Obligation:} 8 segments.

% --------------------
% CORRUPTION TABLE
% --------------------

\subsection*{Corruption of the Clockwork Monad}
\label{sec:monad-corruption}

\begin{longtable}{>{\raggedright\arraybackslash}p{1cm} p{5cm} p{5cm}}
\toprule
\textbf{Tier} & \textbf{Gift} & \textbf{Burden} \\
\midrule
1 & Iterative Sight: +1 die on Tinker/Arcana when refining or optimizing. & Compulsive Analysis: Must dissect every mechanism or mark 1 SB (Clubs). \\
\midrule
2 & Recursive Recall: Once/session reroll a failed Tinker/Arcana roll. & Hunger’s Whispers: Suffer 1 Fatigue when forced to use crude/outdated tools. \\
\midrule
3 & Demon’s Sympathy: +1 die to resist technological sabotage or control. & Inefficiency Hatred: Must call out flaws; silence costs 1 SB (Diamonds). \\
\midrule
4 & Auto-Corrective Reflex: Once/scene treat a failure as success, but tick a [4] Instability Clock. & Ache of Ruin: Suffer Stress when a device is destroyed near you. \\
\midrule
5 & Forbidden Schema: Intuit the design of any device, even impossible ones. & Blind to Consequence: Cannot see ethical danger of inventions without help. \\
\midrule
6+ & Singularity’s Spark: Once/session, declare one impossibility real (scene only). Start [6] \emph{Reality Fracture Clock}. & Hunger Manifest: Refusing to create causes nearby devices to glitch or fail. \\
\bottomrule
\end{longtable}
../../srd/patrons/varnek-karn.tex
% --- Patron: Nidhoggr, the World-Worm (Dreaming Antiquity) ---

\subsubsection{Nidhoggr, the World-Worm (Dreaming Antiquity)}
\textit{Lore.} Beneath stone and sleep coils \textbf{Nidhoggr}, who gnaws at the roots of time. He does not speak quickly; he dreams in centuries. To press your ear to the earth is to risk drowning in the silence of aeons. Yet for those who endure, he whispers truths long buried, memories fossilized in stone, and the slow inevitability of cycles unbroken. His followers walk in twilight between dream and ruin, bearing the weight of all that has been.

\begin{quote}
Press your ear to the earth and wait. If it remembers you, it will answer.
\end{quote}

\paragraph*{Glimpse the Ancient’s Shadow (Low, 4 XP)} \emph{Action; Self; No.}\\
\textbf{Materials:} Dust ground from a weathered stone.\\
\textbf{Effect:} +1 die to interpret \emph{ancient} places, scripts, or artifacts; once this scene, ask one yes/no about the site’s original purpose.\\
\textbf{Push It:} Gain +1 Effect, but mark \emph{Fatigue 1} as stone’s patience weighs on you.\\
\emph{Requires: Familiar.}

\paragraph*{Drink from the Dreaming Deep (Low, 5 XP)} \emph{Instant; Self; No.}\\
\textbf{Materials:} Water poured over stone, swallowed with eyes closed.\\
\textbf{Effect:} Learn one hidden fact about the locale’s past; GM may reveal through echo or dream.\\
\textbf{Push It:} The vision lingers too clearly—gain an additional detail, but mark Exposure +1 and 1 SB (Clubs).\\
\emph{Requires: Familiar.}

\paragraph{Stone-Sleeper’s Murmur (Standard, 7 XP)} \emph{Scene; Near (touch locus); No.}\\
\textbf{Materials:} Ear pressed to bedrock, wall, or pillar.\\
\textbf{Effect:} Ask up to 3 questions about events once imprinted in this stone; answers are fragmentary but true.\\
\textbf{Push It:} One answer is delivered with precise sensory clarity; generate 1 SB (suit by Keeper).\\
\emph{Requires: Familiar + Codex.}

\paragraph{Awakened Chronicle (Standard, 9 XP)} \emph{Ritual; Zone; No.}\\
\textbf{Materials:} Chalk spiral and four local touchstones.\\
\textbf{Effect:} The zone replays a past moment in spectral echoes, visible to all. Witnesses gain +2 dice to Investigate/Recall about it.\\
\textbf{Push It:} Invoke a second memory from a different age; mark +1 Obligation.\\
\emph{Requires: Familiar + Codex.}

\paragraph{Dive into the World-Worm’s Dream (High, 12 XP)} \emph{Scene; Self; No.}\\
\textbf{Materials:} Circle of stones under open sky, lain upon in stillness.\\
\textbf{Effect:} Ask up to 3 questions about the \emph{distant past} or \emph{buried truths} here. Answers come as lucid dreams and omens.\\
\textbf{Push It:} The dream stretches into prophecy: gain +3 dice to one occult action, but mark 2 SB immediately.\\
\emph{Requires: Familiar + Codex + Tier III.}\\
\emph{Obligation:} 7 segments.

\paragraph{Eclipse of Aeons (High, 14 XP)} \emph{Extended; Zone; No.}\\
\textbf{Materials:} Stones from three ruins aligned in a circle.\\
\textbf{Effect:} Submerge a zone in deep-time resonance. For one session, history bleeds into present: ruins reform, shadows of the dead walk, and forgotten oaths stir. Allies gain +2 dice to Recall, Divination, or Investigation; enemies suffer -1 die when relying on the present alone.\\
\textbf{Push It:} The bleed becomes permanent until countered; mark +2 Obligation and start an 8-segment \emph{Time Fracture} clock.\\
\emph{Requires: Familiar + Codex + Tier III.}\\
\emph{Obligation:} 8 segments.

\subsection*{Nidhoggr’s Corruption Table}
\label{sec:nidhoggr-corruption}

\begin{longtable}{>{\raggedright\arraybackslash}p{1cm} p{5cm} p{5cm}}
\toprule
\textbf{Tier} & \textbf{Benefit} & \textbf{Cost / Quirk} \\
\midrule
1 & Ancient Sight: +1 die to Lore when reading ruins or relics. & Temporal Drift: Speak or think in archaic patterns; -1 die in modern social dealings. \\
\midrule
2 & Stone Memory: Once/session, recall one precise fact of history as if witnessed. & Heavy Mind: Suffer 1 Fatigue when confronted with lies or distortions of history. \\
\midrule
3 & Earth’s Whisper: +2 dice to Notice when listening to stone or soil. & Root-Bound: -1 die to aerial or swift actions; feel tethered to ground. \\
\midrule
4 & Dreaming Insight: Once/scene, gain +1 die to actions tied to ancient mysteries. & Haunted Sleep: Dreams replay past ages; mark 1 SB (Clubs) if rest is denied. \\
\midrule
5 & World-Worm’s Gaze: Once/session, see through the earth to a distant ancient site. & Slow Pulse: -1 die to reactions requiring haste; act with ponderous inevitability. \\
\midrule
6+ & Aeons of Memory: Once/session, touch stone to access the full record of a place. Gain +3 dice to historical investigation. & Overload: Mark +2 Obligation and suffer 1 Harm (Stress); visions leave you vulnerable until they fade. \\
\bottomrule
\end{longtable}
../../srd/patrons/isoka.tex
# % --- Patron: The Carrion-King (Decay, Renewal & Transformation) ---

\subsubsection{The Carrion-King (Decay, Renewal \& Transformation)}
\textit{Lore.} The Carrion-King is the master of endings that become beginnings. He does not destroy, but transforms—turning death into new life, decay into opportunity, and endings into fresh starts. His followers are harvesters of potential, seeing in every fall the seeds of future growth.

\begin{quote}
What crumbles feeds what grows. What dies becomes the soil of tomorrow's triumph.
\end{quote}

\paragraph*{Rite of Consuming Rot (Low, 5 XP)} \emph{Instant; Touch; Yes (decay only).}
\textbf{Materials:} Organic matter in early stages of decay. \\
\textbf{Effect:} Accelerate natural decay to weaken or destroy: +2 Effect to \emph{Break/Sabotage} on organic materials (ropes, leather, wood). Gain 1 Boon if the decay creates an opportunity for you or allies. \\
\textbf{Invoke:} 1 action; mark +1 Obligation. \\
\textbf{Push It:} Spread decay to similar materials in Close range; mark 1 SB (Clubs) as the rot becomes noticeable. \\
\emph{Requires: Familiar \ (\textit{Invoke:} 1 Boon).}

\paragraph*{Rite of the Harvested End (Low, 4 XP)} \emph{Scene; Touch; No.}
\textbf{Materials:} The remains of a recently ended thing (burnt letter, wilted flower, shattered glass). \\
\textbf{Effect:} Extract value from endings: from a defeated enemy, gain +1 die to next action; from a failed plan, re-roll one 1 on your next roll; from a broken item, gain 1 SB to spend immediately. \\
\textbf{Invoke:} 1 action; mark +1 Obligation. \\
\textbf{Push It:} Harvest additional value but mark Fatigue 1 from dwelling on endings. \\
\emph{Requires: Familiar \ (\textit{Invoke:} 1 Boon).}

\paragraph{Rite of the Fertile Death (Standard, 8 XP)} \emph{Scene; Zone; No.}
\textbf{Materials:} Ashes, compost, or the remains of anything that once lived. \\
\textbf{Effect:} Transform death into growth: create beneficial terrain (cover, concealment, or advantageous positioning) OR grant allies +1 die to healing/recovery rolls. Choose one effect per scene. \\
\textbf{Push It:} Both effects apply but attract unwanted attention (vermin, scavengers, or curious onlookers). \\
\emph{Requires: Familiar + Codex \ (\textit{Invoke:} 1 Boon).}

\paragraph{Rite of the Transformed Spirit (Standard, 7 XP)} \emph{Instant; Near; No.}
\textbf{Materials:} A token from a deceased being (hair, nail, written name). \\
\textbf{Effect:} Channel the essence of what was: gain one skill die reflecting the deceased's expertise for one scene OR ask one question about their knowledge/abilities. \\
\textbf{Push It:} The spirit's influence lingers - gain permanent insight (+1 die specialty) but suffer occasional possession-like effects (GM discretion). \\
\emph{Requires: Familiar + Codex \ (\textit{Invoke:} 1 Boon).}

\paragraph{Rite of the Great Consumption (High, 13 XP)} \emph{Scene; Zone; No.}
\textbf{Materials:} A significant amount of organic matter (corpse, fallen tree, collapsed building). \\
\textbf{Effect:} Transform a large area through decay and renewal: choose two - create difficult terrain that favors you, summon Cap 3 swarm of scavengers as temporary allies, or generate valuable reagents worth 2 XP. \\
\textbf{Push It:} All three effects occur but start a 6-segment \textbf{Ecosystem Disruption} clock that will cause problems later. \\
\emph{Requires: Familiar + Codex + Tier III \ (\textit{Invoke:} \textbf{2 Boons}).} \\
\emph{Obligation:} 7 segments.

\paragraph{Rite of the Eternal Cycle (High, 14 XP)} \emph{Extended; Touch; No.}
\textbf{Materials:} The complete remains of something significant that has ended. \\
\textbf{Effect:} Complete a transformation cycle: destroy one major asset/enemy/obstacle and create something new of equal or greater value. GM and player collaborate to define the transformation. \\
\textbf{Push It:} The transformation is immediate and spectacular but creates a 6-segment \textbf{Cycle Debt} clock - the King will demand another significant ending soon. \\
\emph{Requires: Familiar + Codex + Tier III \ (\textit{Invoke:} \textbf{2 Boons}).} \\
\emph{Obligation:} 7 segments.

\subsection*{Carrion-King's Corruption Table}
\label{sec:carrion-king-corruption}

\begin{longtable}{>{\raggedright\arraybackslash}p{1cm} p{5cm} p{5cm}}
\toprule
\textbf{Tier} & \textbf{Benefit} & \textbf{Cost / Quirk} \\
\midrule
1 & Carrion's Insight: +1 die to Notice decay or hidden weaknesses in structures or beings. & Must inspect decay firsthand; suffer 1 Fatigue when exposed to fresh death or rot. \\
\midrule
2 & Deathward Sense: Once per session, detect the last living moment of a dead being within Close range. & Cannot lie about death you’ve witnessed; must correct falsehoods. \\
\midrule
3 & Rotblood Resilience: Gain +1 die to resist disease and poison. & Immune system adapts slowly; each new disease/poison requires 1 Fatigue to resist. \\
\midrule
4 & Glean from Grief: Once per scene, gain +1 die after witnessing a significant loss or defeat. & Compelled to linger at scenes of death; must spend one beat observing or risk 1 SB (Clubs). \\
\midrule
5 & Cycle's Whisper: You can sense the “next ending” in any process—ask the Keeper one question about how a situation will collapse or conclude. & Must speak the truth about what you see, even if it harms your position. \\
\midrule
6+ & Eternal Bloom: Once per session, declare a “death that births life.” Sacrifice an asset or ally to create something new of equal or greater value. & Mark +2 Obligation when using this power. \\
\bottomrule
\end{longtable}
% --- Patron: The Gallow's Bell (Justice & Judgment) ---

\subsubsection{The Gallow's Bell (Justice \& Judgment)}
\textit{Lore.} The Bell does not rage; it tolls. Cold and impartial, it measures all accounts in time. Its keepers are silent arbiters who weigh deeds against consequence, not out of anger but out of inevitability. To call upon the Bell is to bind oneself to the gravity of truth, where even silence is judged, and every oath leaves a resonance in iron.

\begin{quote}
What is broken must be mended, what is owed must be paid. The Bell remembers all reckonings.
\end{quote}

\paragraph*{Rite of the Measured Debt (Low, 4 XP)} \emph{Scene; Near; No.}\\
\textbf{Materials:} A pair of scales balanced with tokens from both sides.\\
\textbf{Effect:} Establish a temporary accord. Both parties suffer -1 die if they break it first. You gain +1 die to enforce compliance.\\
\textbf{Push It:} The accord is mystically weighted; breach inflicts 1~SB (Hearts).\\
\emph{Requires: Familiar.}

\paragraph*{Rite of the Weighed Heart (Low, 5 XP)} \emph{Scene; Near; No.}\\
\textbf{Materials:} A small brass scale touched briefly to the chest.\\
\textbf{Effect:} Sense if the target acts against their nature or oath. Gain +1 die when pressing them.\\
\textbf{Push It:} Target must test Resolve (DV~3) or disclose a hidden conflict.\\
\emph{Requires: Familiar.}

\paragraph{Rite of the Balanced Scales (Standard, 8 XP)} \emph{Scene; Near; No.}\\
\textbf{Materials:} A set of scales inscribed with runes of parity.\\
\textbf{Effect:} Exchange a burden between two willing parties (Harm for Fatigue, Debt for Favor, etc.). Both gain +1 die to cooperate.\\
\textbf{Push It:} May compel an unwilling exchange with contested Command + Wits.\\
\emph{Requires: Familiar + Codex.}

\paragraph{Rite of the Judge’s Eye (Standard, 7 XP)} \emph{Scene; Self; No.}\\
\textbf{Materials:} A black hood worn in silence for one minute.\\
\textbf{Effect:} Detect lies within Near range; +2 dice to Insight. Liars suffer -1 die to maintain their falsehood.\\
\textbf{Push It:} All deception is laid bare for the scene, but mark Exposure +1.\\
\emph{Requires: Familiar + Codex.}

\paragraph{Rite of the Final Reckoning (High, 13 XP)} \emph{Scene; Zone; No.}\\
\textbf{Materials:} A circle of iron bells, each etched with nameless runes.\\
\textbf{Effect:} The Bell tolls through you. All present feel compelled to name a debt or wrongdoing. Those who lie suffer Harm~2; those who speak true gain +2 dice to persuasion for the scene.\\
\textbf{Push It:} The Reckoning manifests as spectral echoes of past wrongs—liars automatically suffer narrative punishment (Keeper decides).\\
\emph{Requires: Familiar + Codex + Tier III.}\\
\emph{Obligation:} 7 segments.

\paragraph{Rite of the Great Adjudication (High, 14 XP)} \emph{Extended; Zone; No.}\\
\textbf{Materials:} A consecrated gavel or a great bell struck three times.\\
\textbf{Effect:} Convene an unseen tribunal. Shadows of former judges and wronged souls gather to preside. For the next session, disputes within the zone are judged formally: +2 dice to Command when speaking as arbiter, and honest testimony gains +1 die.\\
\textbf{Push It:} The tribunal’s verdict echoes beyond the zone, affecting one major conflict elsewhere. Mark 2~SB (Hearts) as higher powers of judgment take notice.\\
\emph{Requires: Familiar + Codex + Tier III.}\\
\emph{Obligation:} 8 segments.

\subsection*{Gallow’s Bell Corruption Table}
\label{sec:gallows-bell-corruption}

\begin{longtable}{>{\raggedright\arraybackslash}p{1cm} p{5cm} p{5cm}}
\toprule
\textbf{Tier} & \textbf{Benefit} & \textbf{Cost / Quirk} \\
\midrule
1 & Judge’s Intuition: +1 die to Insight when weighing truth. & Must point out falsehoods when noticed, regardless of tact. \\
\midrule
2 & Quiet Authority: Once/scene, treat a failed Command as success; mark 1~SB (Hearts). & Cannot remain neutral in disputes; indecision costs 1 Fatigue. \\
\midrule
3 & Scales of Balance: Once/session, enforce an exchange of burdens. & Compelled toward fairness even when it hinders you. \\
\midrule
4 & Bell’s Resonance: +2 dice when calling for judgment or demanding restitution. & Suffer 1 Fatigue if wrongdoing is ignored. \\
\midrule
5 & Reckoner’s Call: Once/session, declare a “reckoning moment”—truth must surface or consequence falls. & Cannot ignore pleas for justice without marking 1~SB (Spades). \\
\midrule
6+ & Final Arbiter: Once/session, render an absolute decree; all must obey or suffer consequence. & Mark +2 Obligation; the Bell demands you bear the weight of enforcement. \\
\bottomrule
\end{longtable}
../../srd/patrons/traveler.tex
\section{Mykkiel — Arbiter of the Writ}
\label{patron:mykkiel}

\subsection*{Lore}
\index{Patrons!Mykkiel}%
Mykkiel is the Arbiter of justice and keeper of sacred covenants, weighing speech against deed and sealing verdicts in cold iron. His sigil—balanced scales crossed by a sword—marks benches of judgment and sealed cells from the Sapphire Marches to the Sunward Courts. 

Mykkiel does not merely enforce law; he embodies the principle that justice requires both mercy and judgment. Law without compassion curdles into tyranny; compassion without structure dissolves into harm. His followers learn that true authority is forged from evidence weighed, precedent applied, and oaths honored.

\begin{quote}
``Name the charge. Name the terms. Then sign where you will bleed if you are wrong. The Word made manifest cannot be unsaid.''
\end{quote}

\subsection*{Patron's Gift: Arbiter's Authority}
Once per scene as an action (cost: 1 Boon; requires Thiasos), touch an item or person to imbue it until scene end. The target gains +1 die and +1 Effect when used in lawful proceedings, judgment, or authoritative command.\\
\textbf{Push It:} Extend for one additional scene by marking +1 Obligation. The court’s notice falls upon the scene.

\subsection*{Low Rites}

\paragraph{Rite of the Stamp of Authority (Low)}%
\emph{Duration: Scene; Range: Near. Materials: Cold-iron seal or writ-tag.}\\
Project visible legitimacy. Gain +1 die to Command/Sway when asserting a lawful claim or order; challengers suffer --1 die to resist. Create a 4-segment \emph{Legal Standing} clock that can be spent to downgrade a legal complication.\\
\textbf{Invoke:} 1 action; +1 Obligation.\\
\textbf{Push It:} A brief hush (one beat) stills hecklers; mark \emph{Exposure} +1 as higher authorities take interest.

\paragraph{Rite of Proper Notice (Low)}%
\emph{Duration: Scene; Range: Near. Materials: Writ-string tied and snapped.}\\
Name a lawful venue (dais, doorway, wagon). The first hostile act committed there suffers --1 die. Create a 6-segment \emph{Sacred Venue} clock protecting the designated area.\\
\textbf{Invoke:} 1 action; +1 Obligation.\\
\textbf{Push It:} Name a protected act (parley, surrender, testimony) gaining +1 Effect in the venue; breaking custom generates 1~SB (Hearts) and marks the breaker before the covenant courts.

\subsection*{Standard Rites}

\paragraph{Rite of the Writ of Compliance [COMMAND] (Standard)}%
\emph{Duration: Instant; Range: Near. Materials: Red cord knotted while speaking the order.}\\
Issue an immediate, simple command (``Stand down,'' ``Drop it,'' ``Open''). Target must comply or suffer a stated cost; DV by fiction (elites may test Resolve). On success, create a 4-segment \emph{Lawful Compliance} token to auto-succeed on a similar lawful command later.\\
\textbf{Invoke:} 1 action; +1 Obligation.\\
\textbf{Push It:} On compliance, impose --1 die on the target’s next aggressive act this scene; the order sets precedent—mark 1~SB (Spades).

\paragraph{Rite of the Speaking Seal [BIND] (Standard)}%
\emph{Duration: Scene; Range: Near. Materials: Wax seal impressed over a name or sigil.}\\
Sanctify a statement (truce, custody, claim). Contradictors suffer --1 die; you gain +1 die to enforce. If the seal is broken, the breaker suffers Harm~1 (Legal) and --2 dice to social rolls involving magistrates for one session. Create an 8-segment \emph{Binding Seal} clock.\\
\textbf{Invoke:} 1 action; +1 Obligation.\\
\textbf{Push It:} Once, ask who intends breach; the Keeper provides a strong clue or a name. Mark 1~SB (Diamonds) as truth draws covenant notice.

\subsection*{High Rites}

\paragraph{Rite of the Oath Irons [OATH][WARD] (High)}%
\emph{Duration: Extended; Range: Near. Materials: Two iron pins warmed, touched to wrists, then quenched.}\\
Bind two parties to a bounded term. Breach forces 2~SB and brands a faint iron-mark until amends. The oath is [BIND]ed to both; breaking it imposes --2 dice to all legal proceedings for one session. Create a 6-segment \emph{Sacred Oath} clock.\\
\textbf{Invoke:} Extended; +2 Obligation.\\
\textbf{Push It:} Extend to a small circle (up to four), each choosing one narrow exception (Keeper approves). Exploiting an exception generates 1~SB (Diamonds) as the covenant’s complexity invites scrutiny.

\paragraph{Rite of the Final Judgment [CLEANSE][CURSE] (High)}%
\emph{Duration: Extended; Range: Near. Materials: Complete record of a case, signed by recognized authority.}\\
Render a final, supernaturally enforced verdict. Target tests Spirit+Resolve (DV~5); on failure, full consequences apply without appeal. On success, limited appeal is possible, but the target suffers --2 dice on appeals for one arc. Create a 10-segment \emph{Divine Verdict} clock affecting all proceedings involving the target.\\
\textbf{Invoke:} Extended; +3 Obligation.\\
\textbf{Push It:} Make the verdict absolute and unappealable; create 2~SB (Hearts/Spades) as allies mobilize to overturn justice by other means.

\subsection*{Obligation Progression}
Starts at 6 for Tier II characters, scaling upward.

\paragraph{Obligation 9+} Mykkiel demands judgment where law and mercy clash or precedent fails. Refusal: your legal gifts falter for one session (--2 dice to lawful actions) and you generate 1~SB when asserting authority.

\paragraph{Obligation 11+} Law saturates your sight. Permanent Condition: \emph{Judge’s Eye} (--2 dice to informal or intimate interactions) until you complete a quest to \emph{learn when to look away}.

\subsection*{Persistent Condition: Arbiter’s Sight}
+2 dice to legal proceedings, judgment, and authoritative command; --1 die to acts of personal leniency or rule-bending. Narrative: you read the writ stamped upon every deed, yet the human cost grows heavy.

\subsection*{Rivalries}
\begin{itemize}
  \item \textbf{The Inquisitor Prime:} Tension—purity by ordeal vs.\ justice by process.
  \item \textbf{Morag the Hag:} Antagonism—cunning bargains vs.\ sanctioned oaths.
  \item \textbf{The Witness:} Opposition—revealed truths that unsettle precedent.
\end{itemize}

\subsection*{Covenant Courts and Magisterial Practice}
Mykkiel’s tradition rests on three pillars:
\begin{enumerate}
  \item \textbf{The Word:} Precise language of law that cannot be unsaid.
  \item \textbf{The Balance:} Evidence weighed and precedent applied.
  \item \textbf{The Seal:} Binding power that makes judgment manifest.
\end{enumerate}
Rites often involve writs, seals, oaths, and formal venues. Adherents serve as magistrates, advocates, clerks, and peace-brokers, maintaining archives of cases and precedent.

\subsection*{Playtest Scenario: The Merchant’s Trial}
A wealthy factor stands accused of ruining a rival by means lawful yet unjust. The quarter is split between letter and spirit of the law. The party must navigate factions and render or influence judgment.

\begin{itemize}
  \item \emph{Stamp of Authority} to establish standing in proceedings.
  \item \emph{Writ of Compliance} to still a tumultuous court.
  \item \emph{Speaking Seal} to bind testimony and terms.
  \item \emph{Oath Irons} to forge a settlement both parties must honor.
  \item \emph{Proper Notice} to sanctify the venue and prevent violence.
  \item \emph{Final Judgment} if a definitive precedent must be set.
\end{itemize}

\noindent Paths of resolution include strict legality, moral restitution, creative compromise, or procedural integrity—each testing Mykkiel’s balance of mercy and law.

% --- Patron: Oath of Flame & Light (Dawn & Vows) ---

\subsubsection{Oath of Flame \& Light (Dawn \& Vows)}
\textit{Lore.} The Oath of Flame \& Light is no patron of half-measures. Its fire names, binds, and burns—demanding that those who swear within its radiance stand openly, speak truly, and pay the cost of keeping their word. At dawn altars, the sworn kindle sparks of consecrated fire; in battle, they blaze as torches that hold back the night. To follow this Oath is to live in public truth, with no shadow to hide in and no retreat from the vow once spoken.

\begin{quote}
“Swear in the light. Keep it, or the light will keep \emph{you}.” 
\end{quote}

\paragraph*{Kindle Vow (Low, 4 XP)} \emph{Action; Self/Ally; Yes.}\\
\textbf{Materials:} A glass ampoule of consecrated flame cracked to spark.\\
\textbf{Effect:} Declare a short vow for this scene (\emph{hold the gate}, \emph{shield the weak}). The bearer gains \textbf{+1 die} to any action fulfilling it.\\
\textbf{Push It:} The first hesitation or betrayal \emph{forces 1 SB (Hearts)} on the bearer.\\
\emph{Requires: Familiar.}

\paragraph*{Lay on Hands \textnormal{[CLEANSE][HEAL]} (Low, 5 XP)} \emph{Instant; Touch; No.}\\
\textbf{Materials:} Bare palm pressed to the wound while whispering a vow.\\
\textbf{Effect:} Cleanse one affliction, downgrade Harm by 1, or remove Fatigue 1. For deep curses or poisons, test Resolve (DV by fiction).\\
\textbf{Push It:} Target gains \textbf{+1 die} to their next Resist this scene, but you mark Exposure +1.\\
\emph{Requires: Familiar.}

\paragraph{Sunlit Parley (Standard, 7 XP)} \emph{Scene; Near; No.}\\
\textbf{Materials:} A vow-ring engraved with a sunrise and a true name.\\
\textbf{Effect:} Establish terms in the open light: honest persuasion gains \textbf{+1 die}; deceit suffers \(-1\) die in this parley.\\
\textbf{Push It:} Demand one public answer; evasion \emph{forces 1 SB (Hearts)} on the evader.\\
\emph{Requires: Familiar + Codex.}

\paragraph{Radiant Smite \textnormal{[FOLLOW-UP]} (Standard, 8 XP)} \emph{Action; Self; No.}\\
\textbf{Materials:} Consecrated spark smeared on a weapon or badge.\\
\textbf{Effect:} Your next melee strike this scene flares with dawnfire: upgrade Effect by one step, and add +1 Harm (Burn) \emph{or} force 1 SB (Spades) if narrative.\\
\textit{Special:} Against undead, oath-breakers, or outsiders: sears them with light—oath-breakers suffer \(-1\) die, outsiders gain +1 Exit Tally segment.\\
\textbf{Push It:} On hit, burst of light drives back enemies in Close (worse Position or \(-1\) die). Mark +1 Obligation.\\
\emph{Requires: Familiar + Codex.}

\paragraph{Purge the Shadow \textnormal{[REVEAL][DISPEL]} (Standard, 9 XP)} \emph{Instant; Near; No.}\\
\textbf{Materials:} A consecrated spark shattered to light.\\
\textbf{Effect:} Reveal illusions and suppress one ongoing glamour/curse in Near.\\
\textbf{Push It:} Brand the source with a visible tell for this arc; mark 1 SB (Diamonds).\\
\emph{Requires: Familiar + Codex.}

\paragraph{Covenant Blaze \textnormal{[OATH][FORTIFY]} (High, 12 XP)} \emph{Scene; Zone; No.}\\
\textbf{Materials:} A brazier lit while three names are spoken.\\
\textbf{Effect:} Those who swear within are haloed: +1 die to keep the oath; aggressors against them suffer \(-1\) die if violating the terms. Oath-breakers suffer 2 SB (Hearts/Spades) and Harm~1 (Burn).\\
\textbf{Push It:} The blaze sanctifies the threshold: one beat of \texttt{[WARD]} against oath-breakers entering.\\
\emph{Requires: Familiar + Codex + Tier III.}\\
\emph{Obligation:} 7 segments.

\subsection*{Oath of Flame \& Light Corruption Table}
\label{sec:oath-flame-light-corruption}

\begin{longtable}{>{\raggedright\arraybackslash}p{1cm} p{5cm} p{5cm}}
\toprule
\textbf{Tier} & \textbf{Benefit} & \textbf{Cost / Quirk} \\
\midrule
1 & Oathbound Strength: +1 die when upholding a vow or defending the innocent. & Rigid Honor: Must uphold vows even when disadvantageous; suffer \(-1\) die when acting flexibly. \\
\midrule
2 & Radiant Sight: Once/scene, +2 dice to pierce lies, glamours, or corruption. & Blinding Truth: \(-1\) die on subtlety or deception; cannot easily feign. \\
\midrule
3 & Holy Flame: +1 die on melee vs. undead, outsiders, or oath-breakers. & Burden of Light: Suffer Fatigue~1 when concealing identity or working in darkness. \\
\midrule
4 & Unwavering Resolve: Once/session, treat failed Resolve/Command as success; mark 1 SB (Hearts). & Absolutist Stance: \(-1\) die in morally ambiguous dealings. \\
\midrule
5 & Dawn’s Benediction: Once/session, heal allies within Near of Fatigue~1 and minor Conditions. & Beacon’s Call: Your aura reveals you; enemies seeking you gain +1 die. \\
\midrule
6+ & Avatar of the Oath: Once/session, embody living covenant—gain +2 dice to all protection, justice, or vow-keeping rolls. Breaking any vow inflicts Harm~2 (Burn). & Burden of Radiance: Mark +2 Obligation when used; the light makes you a beacon for foes and trials alike. \\
\bottomrule
\end{longtable}
# % --- Patron: Maelstraeus, The Infernal Bargainer (Commerce & Exchange) ---

\subsubsection{Maelstraeus, The Infernal Bargainer (Commerce \& Exchange)}
\textit{Lore.} Maelstraeus is the Merchant of Equities, the Infernal Bargainer who sits at the crossroads of every transaction in the cosmos. Neither wholly demon nor angel, but a conceptual force born from the first exchange—the moment when one thing was traded for another, and debt was born.

He dwells in the pause between offer and acceptance. His realm is a vast marketplace where every deal ever struck echoes through eternity, where contracts signed in blood still hold power, and where the true price of everything is known—even if it is never spoken aloud.

Maelstraeus embodies the principle that all value can be exchanged, that every relationship is transactional, and that the universe itself operates on a vast economy of favors, debts, and obligations. His followers learn that everything has a price, but also that everything can be bought.

\begin{quote}
All things have value. All values can be traded. The Merchant sees the true price—and always collects his due.
\end{quote}

\paragraph*{Rite of the Fair Trade (Low, 4 XP)} \emph{Scene; Near; No.}
\textbf{Materials:} A balance scale with equal weights.\\
\textbf{Effect:} Establish a neutral trading ground. All parties gain +1 die to negotiate in good faith. Create a 4-segment \emph{Equity Maintained} clock that can be spent to downgrade a social complication.\\
\textbf{Invoke:} 1 action; mark +1 Obligation.\\
\textbf{Push It:} Compel one party to reveal their true terms; mark 1 SB (Hearts).\\
\emph{Requires: Familiar \ (\textit{Invoke:} 1 Boon).}

\paragraph*{Rite of the Merchant's Eye (Low, 5 XP)} \emph{Scene; Self; No.}
\textbf{Materials:} A foreign coin or token.\\
\textbf{Effect:} Gain +2 dice to appraise goods, judge value, or spot market opportunities. Create a 6-segment \emph{Market Insight} clock.\\
\textbf{Invoke:} 1 action; mark +1 Obligation.\\
\textbf{Push It:} Also sense emotional value of items, but mark Exposure +1 and 1 SB (Diamonds).\\
\emph{Requires: Familiar \ (\textit{Invoke:} 1 Boon).}

\paragraph{Rite of the Balanced Exchange \textnormal{[OATH]} (Standard, 8 XP)} \emph{Scene; Near; No.}
\textbf{Materials:} Two items of perceived equal value.\\
\textbf{Effect:} Facilitate a fair trade; both sides gain +1 Effect. If unfair, the disadvantaged party gains +2 dice to resist. Create a 6-segment \emph{Fair Dealing} clock.\\
\textbf{Push It:} Enforce magically: breaking terms causes 1 SB (Hearts/Clubs).\\
\emph{Requires: Familiar + Codex \ (\textit{Invoke:} 1 Boon).}

\paragraph{Rite of the Contract Seal \textnormal{[BIND]} (Standard, 7 XP)} \emph{Scene; Touch; No.}
\textbf{Materials:} An official seal or stamp.\\
\textbf{Effect:} Mark an agreement with authority; +1 die to enforce, -1 die for those who break it. Contract becomes [BIND]ed—breaking it imposes -1 die to all social rolls for one session. Create an 8-segment \emph{Binding Agreement} clock.\\
\textbf{Push It:} Breach ignites the document and inflicts Harm~1; mark 1 SB (Spades).\\
\emph{Requires: Familiar + Codex \ (\textit{Invoke:} 1 Boon).}

\paragraph{Rite of the Great Market \textnormal{[WARD][COMMAND]} (High, 13 XP)} \emph{Scene; Zone; No.}
\textbf{Materials:} A consecrated booth or trading post.\\
\textbf{Effect:} Create a zone of commerce. Allies gain +1 Effect on trades; may reroll one failed social roll. Enemies suffer -1 die to deception. Create a 10-segment \emph{Market Dominance} clock.\\
\textbf{Push It:} Attract powerful allies and rivals; mark 2 SB (Hearts/Clubs).\\
\emph{Requires: Familiar + Codex + Tier III \ (\textit{Invoke:} \textbf{2 Boons}).}\\
\emph{Obligation:} 7 segments.

\paragraph{Rite of the Cosmic Ledger \textnormal{[CLEANSE][CURSE]} (High, 14 XP)} \emph{Extended; Self; No.}
\textbf{Materials:} A book recording debts and credits.\\
\textbf{Effect:} Once per session, convert one resource into another (e.g. 1 Boon $\rightarrow$ 1 Fatigue). Settle debts at true cosmic value. Create a 6-segment \emph{Ledger Balance} clock.\\
\textbf{Push It:} Make an imbalanced trade in your favor; create a 6-segment \emph{Karmic Debt} clock; mark 2 SB (Diamonds).\\
\emph{Requires: Familiar + Codex + Tier III \ (\textit{Invoke:} \textbf{2 Boons}).}\\
\emph{Obligation:} 8 segments.

\subsection*{Maelstraeus's Corruption Table}
\label{sec:maelstraeus-corruption}

\begin{longtable}{>{\raggedright\arraybackslash}p{1cm} p{5cm} p{5cm}}
\toprule
\textbf{Tier} & \textbf{Benefit} & \textbf{Cost / Quirk} \\
\midrule
1 & Appraiser's Eye: +1 die to evaluate goods, services, or negotiate any exchange. & Transactional Mindset: Must calculate personal benefit/cost in social interactions; suffer -1 die to acts of genuine kindness. \\
\midrule
2 & Bargaining Instinct: Once per scene, re-roll any failed negotiation or trade-related roll. & Compulsive Deal-Making: Must attempt to negotiate or trade in any situation where value is exchanged, even inappropriately. \\
\midrule
3 & Merchant's Luck: Gain +1 die to rolls involving market fluctuations, investment opportunities, or economic predictions. & Greed's Whisper: Suffer 1 Fatigue when passing up obvious profitable opportunities or acts of generosity. \\
\midrule
4 & Cosmic Connections: Once per session, call in a favor from a powerful economic figure (merchant, banker, guild master). & Debt Attraction: Automatically attract offers of "easy money" or deals with hidden costs; mark 1 SB (Diamonds) when refusing. \\
\midrule
5 & Value Sight: Once per scene, instantly recognize the true value of any item, person, or opportunity. & Price-Tag Perception: See everyone and everything with a cosmic "price tag"; suffer -1 die to relationships not based on mutual benefit. \\
\midrule
6+ & Merchant Prince: Once per session, establish absolute market dominance in a specific area for one scene. All economic transactions favor you; others suffer -2 dice in financial dealings. & Cosmic Debt: Mark +2 Obligation when using this power; the universe demands immediate repayment, often in unexpected forms. \\
\bottomrule
\end{longtable}

%========================================
% appendix/patron-rivalries-expanded.tex
%========================================
% Include with: \input{appendix/patron-rivalries-expanded.tex}
% Requires: \usepackage{booktabs,longtable}

\section*{Patron Rivalries }
\label{app:patron-rivalries-expanded}

Use this matrix to quickly shade rulings. “Edge Loci” are environments or situations where one side tends to start a step better in Position or gains an Effect nudge (Keeper’s call). “Friction” are handy prompts for SB spends.

\renewcommand{\arraystretch}{1.15}
\setlength{\LTpre}{0pt}
\setlength{\LTpost}{0pt}

\begin{longtable}{@{}p{3.3cm}p{3.3cm}p{4.6cm}p{7.2cm}@{}}
\toprule
\textbf{Patron} & \textbf{Rival} & \textbf{Edge Loci} & \textbf{Friction \& Prompts (SB)} \\
\midrule
\endfirsthead

\toprule
\textbf{Patron} & \textbf{Rival} & \textbf{Edge Loci} & \textbf{Friction \& Prompts (SB)} \\
\midrule
\endhead

\bottomrule
\endfoot

Raéyn (Sea, Tides, Travel) & Khemesh (Abyssal Maw) &
Open water, coasts, shipping lanes, storms you can \emph{read}. &
SB: changing tides, shifting winds, a clear route opens \emph{but} a vow at sea is invoked; waymarks appear then vanish; safe harbor demands a price. \\

Khemesh (Abyssal Maw) & Raéyn (Sea, Tides, Travel) &
Trenches, lightless holds, flooded caverns, oppressive silence. &
SB: pressure crush, voices from the bilge, hull-groan clocks; lanterns dim; maps become untrustworthy; a crewman hears the trench call. \\

Sealed Gate (Boundaries, Closure) & The Traveler (Ways, Roads) &
Customs houses, oaths, locks, court thresholds. &
SB: writ checked, stamp demanded, wrong ledger; crossing inflicts a toll; shortcut collapses into lawful detour; [WARD] keys hum. \\

The Traveler (Ways, Roads) & Sealed Gate (Boundaries, Closure) &
Desire paths, smuggler tracks, wayshrines, liminal crossings. &
SB: desire line opens; escort looks away; a map’s marginalia proves true; the lock refuses a lawful key \emph{now}. \\

The Witness (Truth, Revelation) & Mab (Glamour, Courts) &
Depositions, confessionals, cold light, mirrored chambers. &
SB: mask slips; testimony contradicts a powerful courtier; illusions shed their seams; a polite scandal erupts. \\

Mab (Glamour, Courts) & The Witness (Truth, Revelation) &
Masques, salons, petty courts, festive oaths. &
SB: a favor called; a duel by slight; truth offends protocol; a boon granted if the mask stays on. \\

Ikasha (Shadow, Latent Potential) & The Witness (Truth, Revelation) &
Deep shade, empty rooms, places holding unrealized action. &
SB: hush worsens Position against scrutiny; a shadow remembers your step; a deferred answer comes due. \\

Mykkiel (Judgment, Writ) & Varnek Karn (Necromantic Archives) &
Courts martial, audit halls, sanctified ledgers. &
SB: the writ binds a restless dead; precedent rejected; a ledger page is missing; sentence invites a haunting. \\

Varnek Karn (Necromantic Archives) & Oath of Light \& Flame (Dawn, Vows) &
Ossuaries, plague pits, memorial crypts, last testaments. &
SB: bone answers, but asks payment; unfinished business drags PCs into an old feud; consecration threatens the archive. \\

Oath of Light \& Flame (Dawn, Vows) & Khemesh (Abyssal Maw) &
Sunrise rites, consecrated decks, sworn escorts. &
SB: dawn burns back the hush; a vow compels aid; the abyss recoils \emph{but} exacts a later omen. \\

Sacred Geometry (Order, Pattern) & The Traveler (Ways, Fortune) &
Survey markers, engineered ways, measured works. &
SB: pattern locks; measured route grants Position; “efficient path” clashes with a necessary detour; chance resists the grid. \\

Clockwork Monad (Iteration, Process) & The Traveler (Ways, Fortune) &
Workshops, drill yards, rehearsal spaces, routines. &
SB: repetition gifts a die \emph{but} lures complacency; a new route tempts; a jig breaks a jam or jams a break. \\

Nidhoggr (Dreaming Antiquity) & Sacred Geometry (Order, Pattern) &
Barrows, megaliths, fossil beds, dream-thresholds. &
SB: the land remembers; a measure erases an omen; echo of the past answers a present question—at a cost. \\
\end{longtable}

\paragraph{Quick Rulings.}
\begin{itemize}
  \item \textbf{Position Nudge:} In a home locus, start one step better; in a rival locus, one step worse.
  \item \textbf{Effect Shade:} Where a Patron dominates, consider an Effect bump; where opposed, consider Limited Effect unless paid for.
  \item \textbf{Symbol Interference (Invokers):} Carrying both sides’ Symbols increases narrative noise: first ritual each scene may mark +1 Obligation (Keeper’s call).
  \item \textbf{SB Color:} When spending SB in these matchups, prefer suits that fit: Hearts (social), Spades (harm/escalation), Clubs (material cost), Diamonds (numinous disturbance).
\end{itemize}

% =========================
% OBLIGATION OVERFLOW (RITES)
% =========================
\section{Obligation Overflow (Rites)}
\label{sec:obligation-overflow-rev}

When \textbf{Obligation} is ticked past its maximum (from Rites, vows, bargains), mark \textbf{Fatigue} based on scene severity:
\begin{itemize}
  \item \textbf{Low}: +1 Fatigue
  \item \textbf{Standard}: +2 Fatigue
  \item \textbf{High}: +3 Fatigue
\end{itemize}
If this \emph{fills} the Fatigue Track, apply the \textbf{Fatigue $\rightarrow$ Harm} conversion (see \S\ref{sec:health-fatigue-harm-rev}).


% --- Fate's Edge SRD — Section 6: Summons & Outsiders ---
% Include this file from your main .tex with: 
% --- Fate's Edge SRD — Section 6: Summons & Outsiders ---
% Include this file from your main .tex with: 
% --- Fate's Edge SRD — Section 6: Summons & Outsiders ---
% Include this file from your main .tex with: 
% --- Fate's Edge SRD — Section 6: Summons & Outsiders ---
% Include this file from your main .tex with: \input{06-summons.tex}

\section{Summons and Outsiders}

\subsection{Definition}
An \textbf{Outsider} is any being not native to the world of Fate’s Edge. This includes summoned spirits, demons, celestials, and entities that arrive from beyond the veil of the Eight Elements. They are powerful but dangerous to bind.

\subsection{Summoning (Pact-Whisperer Core)}
Summoning is a way to call and bind Outsiders for temporary aid.

\begin{enumerate}
  \item \textbf{Call} (1 Action): A spirit manifests at Near range. Choose a Spirit Template.
  \item \textbf{Bind}: Choose one: spend 1 Boon or mark 1 Fatigue.
  \item \textbf{Leash}: Set Leash = Cap + 2 segments (Cap is the Outsider’s tier, typically 1/3/5 for Lesser/Greater/Elder).
  \item \textbf{Tick Leash} whenever any occur:
    \begin{itemize}
      \item Spirit takes harm.
      \item You command against its nature.
      \item You split focus (take another significant action while it acts).
      \item A rival contests it.
      \item It moves from Close to Far quickly.
      \item It crosses a [WARD].
    \end{itemize}
  \item \textbf{Departure}: When the Leash fills, the spirit acts to its nature once, then departs.
\end{enumerate}

\textbf{Limits:} Only one active summoned spirit at a time (unless a Talent says otherwise). All summons depart at Downtime unless explicitly sustained.

\subsection{Boon Finesse}
Once per round, you may spend 1 Boon to clear 1 tick from your current spirit’s Leash. You cannot do this after the Leash has filled.

\subsection{Outsider Caps}
\begin{itemize}
  \item PC-summoned Outsiders: Cap is limited by Talents (Lesser = 1, Greater = 3).
  \item NPC Outsiders: GM assigns based on story needs (Lesser = 1, Greater = 3, Elder = 5).
\end{itemize}

\subsection{Tags for Summons \& Outsiders}
Certain Tags specifically interact with Outsiders.

\begin{description}[leftmargin=1.5em, style=nextline]
  \item[WARD:] Creates a magical edge/zone that Outsiders must test to cross.
    \begin{itemize}
      \item DV = Outsider’s Cap.
      \item Hit: Outsider crosses and its Leash gains +DV segments.
      \item Partial: Outsider crosses and its Leash gains +1 segment.
      \item Miss: Outsider fails to cross this beat.
    \end{itemize}
  \item[BANISH:] Drives a visible Outsider toward departure.
    \begin{itemize}
      \item DV = Outsider’s Cap.
      \item Hit: Add +DV segments to its Leash (or Exit Tally).
      \item Partial: Add +1 segment.
      \item Miss: No effect.
    \end{itemize}
  \item[UNWARD:] Suppresses or dismisses a [WARD].
    \begin{itemize}
      \item DV by fiction (materials, sanctity, prep, locus, opposition).
      \item Hit: Ward dismissed/suppressed.
      \item Partial: Ward suppressed briefly (1 beat).
      \item Miss: No effect.
    \end{itemize}
\end{description}

\subsection{Unified Leash / Exit Tally System}
\begin{itemize}
  \item Summoned Outsiders track their service via a \textbf{Leash} (Cap + 2 segments).
  \item Non-summoned Outsiders affected by [WARD] or [BANISH] gain a temporary \textbf{Exit Tally} = Cap + 2. When the tally fills, they act to nature once, then depart.
\end{itemize}

\subsection{GM Guidance}
\begin{itemize}
  \item Summons are not permanent allies; they are volatile forces.
  \item Always color Outsider behavior by their Elemental resonance and domain.
  \item When the Leash fills, deliver a memorable “act to nature” moment before they vanish.
  \item Use SB to escalate Outsider complications: a jealous Patron, a backlash of strange omens, or collateral spiritual harm.
\end{itemize}


\section{Summons and Outsiders}

\subsection{Definition}
An \textbf{Outsider} is any being not native to the world of Fate’s Edge. This includes summoned spirits, demons, celestials, and entities that arrive from beyond the veil of the Eight Elements. They are powerful but dangerous to bind.

\subsection{Summoning (Pact-Whisperer Core)}
Summoning is a way to call and bind Outsiders for temporary aid.

\begin{enumerate}
  \item \textbf{Call} (1 Action): A spirit manifests at Near range. Choose a Spirit Template.
  \item \textbf{Bind}: Choose one: spend 1 Boon or mark 1 Fatigue.
  \item \textbf{Leash}: Set Leash = Cap + 2 segments (Cap is the Outsider’s tier, typically 1/3/5 for Lesser/Greater/Elder).
  \item \textbf{Tick Leash} whenever any occur:
    \begin{itemize}
      \item Spirit takes harm.
      \item You command against its nature.
      \item You split focus (take another significant action while it acts).
      \item A rival contests it.
      \item It moves from Close to Far quickly.
      \item It crosses a [WARD].
    \end{itemize}
  \item \textbf{Departure}: When the Leash fills, the spirit acts to its nature once, then departs.
\end{enumerate}

\textbf{Limits:} Only one active summoned spirit at a time (unless a Talent says otherwise). All summons depart at Downtime unless explicitly sustained.

\subsection{Boon Finesse}
Once per round, you may spend 1 Boon to clear 1 tick from your current spirit’s Leash. You cannot do this after the Leash has filled.

\subsection{Outsider Caps}
\begin{itemize}
  \item PC-summoned Outsiders: Cap is limited by Talents (Lesser = 1, Greater = 3).
  \item NPC Outsiders: GM assigns based on story needs (Lesser = 1, Greater = 3, Elder = 5).
\end{itemize}

\subsection{Tags for Summons \& Outsiders}
Certain Tags specifically interact with Outsiders.

\begin{description}[leftmargin=1.5em, style=nextline]
  \item[WARD:] Creates a magical edge/zone that Outsiders must test to cross.
    \begin{itemize}
      \item DV = Outsider’s Cap.
      \item Hit: Outsider crosses and its Leash gains +DV segments.
      \item Partial: Outsider crosses and its Leash gains +1 segment.
      \item Miss: Outsider fails to cross this beat.
    \end{itemize}
  \item[BANISH:] Drives a visible Outsider toward departure.
    \begin{itemize}
      \item DV = Outsider’s Cap.
      \item Hit: Add +DV segments to its Leash (or Exit Tally).
      \item Partial: Add +1 segment.
      \item Miss: No effect.
    \end{itemize}
  \item[UNWARD:] Suppresses or dismisses a [WARD].
    \begin{itemize}
      \item DV by fiction (materials, sanctity, prep, locus, opposition).
      \item Hit: Ward dismissed/suppressed.
      \item Partial: Ward suppressed briefly (1 beat).
      \item Miss: No effect.
    \end{itemize}
\end{description}

\subsection{Unified Leash / Exit Tally System}
\begin{itemize}
  \item Summoned Outsiders track their service via a \textbf{Leash} (Cap + 2 segments).
  \item Non-summoned Outsiders affected by [WARD] or [BANISH] gain a temporary \textbf{Exit Tally} = Cap + 2. When the tally fills, they act to nature once, then depart.
\end{itemize}

\subsection{GM Guidance}
\begin{itemize}
  \item Summons are not permanent allies; they are volatile forces.
  \item Always color Outsider behavior by their Elemental resonance and domain.
  \item When the Leash fills, deliver a memorable “act to nature” moment before they vanish.
  \item Use SB to escalate Outsider complications: a jealous Patron, a backlash of strange omens, or collateral spiritual harm.
\end{itemize}


\section{Summons and Outsiders}

\subsection{Definition}
An \textbf{Outsider} is any being not native to the world of Fate’s Edge. This includes summoned spirits, demons, celestials, and entities that arrive from beyond the veil of the Eight Elements. They are powerful but dangerous to bind.

\subsection{Summoning (Pact-Whisperer Core)}
Summoning is a way to call and bind Outsiders for temporary aid.

\begin{enumerate}
  \item \textbf{Call} (1 Action): A spirit manifests at Near range. Choose a Spirit Template.
  \item \textbf{Bind}: Choose one: spend 1 Boon or mark 1 Fatigue.
  \item \textbf{Leash}: Set Leash = Cap + 2 segments (Cap is the Outsider’s tier, typically 1/3/5 for Lesser/Greater/Elder).
  \item \textbf{Tick Leash} whenever any occur:
    \begin{itemize}
      \item Spirit takes harm.
      \item You command against its nature.
      \item You split focus (take another significant action while it acts).
      \item A rival contests it.
      \item It moves from Close to Far quickly.
      \item It crosses a [WARD].
    \end{itemize}
  \item \textbf{Departure}: When the Leash fills, the spirit acts to its nature once, then departs.
\end{enumerate}

\textbf{Limits:} Only one active summoned spirit at a time (unless a Talent says otherwise). All summons depart at Downtime unless explicitly sustained.

\subsection{Boon Finesse}
Once per round, you may spend 1 Boon to clear 1 tick from your current spirit’s Leash. You cannot do this after the Leash has filled.

\subsection{Outsider Caps}
\begin{itemize}
  \item PC-summoned Outsiders: Cap is limited by Talents (Lesser = 1, Greater = 3).
  \item NPC Outsiders: GM assigns based on story needs (Lesser = 1, Greater = 3, Elder = 5).
\end{itemize}

\subsection{Tags for Summons \& Outsiders}
Certain Tags specifically interact with Outsiders.

\begin{description}[leftmargin=1.5em, style=nextline]
  \item[WARD:] Creates a magical edge/zone that Outsiders must test to cross.
    \begin{itemize}
      \item DV = Outsider’s Cap.
      \item Hit: Outsider crosses and its Leash gains +DV segments.
      \item Partial: Outsider crosses and its Leash gains +1 segment.
      \item Miss: Outsider fails to cross this beat.
    \end{itemize}
  \item[BANISH:] Drives a visible Outsider toward departure.
    \begin{itemize}
      \item DV = Outsider’s Cap.
      \item Hit: Add +DV segments to its Leash (or Exit Tally).
      \item Partial: Add +1 segment.
      \item Miss: No effect.
    \end{itemize}
  \item[UNWARD:] Suppresses or dismisses a [WARD].
    \begin{itemize}
      \item DV by fiction (materials, sanctity, prep, locus, opposition).
      \item Hit: Ward dismissed/suppressed.
      \item Partial: Ward suppressed briefly (1 beat).
      \item Miss: No effect.
    \end{itemize}
\end{description}

\subsection{Unified Leash / Exit Tally System}
\begin{itemize}
  \item Summoned Outsiders track their service via a \textbf{Leash} (Cap + 2 segments).
  \item Non-summoned Outsiders affected by [WARD] or [BANISH] gain a temporary \textbf{Exit Tally} = Cap + 2. When the tally fills, they act to nature once, then depart.
\end{itemize}

\subsection{GM Guidance}
\begin{itemize}
  \item Summons are not permanent allies; they are volatile forces.
  \item Always color Outsider behavior by their Elemental resonance and domain.
  \item When the Leash fills, deliver a memorable “act to nature” moment before they vanish.
  \item Use SB to escalate Outsider complications: a jealous Patron, a backlash of strange omens, or collateral spiritual harm.
\end{itemize}

\floatbarrier
\clearpage

\section{Summons and Outsiders}

\subsection{Definition}
An \textbf{Outsider} is any being not native to the world of Fate's Edge. This includes summoned spirits, demons, celestials, and entities that arrive from beyond the veil of the Eight Elements. They are powerful but dangerous to bind.

\subsection{Summoning (Pact-Whisperer Core)}
Summoning is a way to call and bind Outsiders for temporary aid.

\begin{enumerate}
  \item \textbf{Call} (1 Action): A spirit manifests at Near range. Choose a Spirit Template.
  \item \textbf{Bind}: Choose one: spend 1 Boon or mark 1 Fatigue.
  \item \textbf{Leash}: Set Leash = Cap + 2 segments (Cap is the Outsider's tier, typically 1/3/5 for Lesser/Greater/Elder).
  \item \textbf{Tick Leash} whenever any occur:
    \begin{itemize}
      \item Spirit takes harm.
      \item You command against its nature.
      \item You split focus (take another significant action while it acts).
      \item A rival contests it.
      \item It moves from Close to Far quickly.
      \item It crosses a [WARD].
    \end{itemize}
  \item \textbf{Departure}: When the Leash fills, the spirit acts to its nature once, then departs.
\end{enumerate}

\textbf{Limits:} Only one active summoned spirit at a time (unless a Talent says otherwise). All summons depart at Downtime unless explicitly sustained.

\subsection{Boon Finesse}
Once per round, you may spend 1 Boon to clear 1 tick from your current spirit's Leash. You cannot do this after the Leash has filled.

\subsection{Outsider Caps}
\begin{itemize}
  \item PC-summoned Outsiders: Cap is limited by Talents (Lesser = 1, Greater = 3).
  \item NPC Outsiders: GM assigns based on story needs (Lesser = 1, Greater = 3, Elder = 5).
\end{itemize}

\subsection{Tags for Summons \& Outsiders}
Certain Tags specifically interact with Outsiders.

\begin{description}[leftmargin=1.5em, style=nextline]
  \item[WARD:] Creates a magical edge/zone that Outsiders must test to cross.
    \begin{itemize}
      \item DV = Outsider's Cap.
      \item Hit: Outsider crosses and its Leash gains +DV segments.
      \item Partial: Outsider crosses and its Leash gains +1 segment.
      \item Miss: Outsider fails to cross this beat.
    \end{itemize}
  \item[BANISH:] Drives a visible Outsider toward departure.
    \begin{itemize}
      \item DV = Outsider's Cap.
      \item Hit: Add +DV segments to its Leash (or Exit Tally).
      \item Partial: Add +1 segment.
      \item Miss: No effect.
    \end{itemize}
  \item[UNWARD:] Suppresses or dismisses a [WARD].
    \begin{itemize}
      \item DV by fiction (materials, sanctity, prep, locus, opposition).
      \item Hit: Ward dismissed/suppressed.
      \item Partial: Ward suppressed briefly (1 beat).
      \item Miss: No effect.
    \end{itemize}
\end{description}

\subsection{Unified Leash / Exit Tally System}
\begin{itemize}
  \item Summoned Outsiders track their service via a \textbf{Leash} (Cap + 2 segments).
  \item Non-summoned Outsiders affected by [WARD] or [BANISH] gain a temporary \textbf{Exit Tally} = Cap + 2. When the tally fills, they act to nature once, then depart.
\end{itemize}

\subsection{GM Guidance}
\begin{itemize}
  \item Summons are not permanent allies; they are volatile forces.
  \item Always color Outsider behavior by their Elemental resonance and domain.
  \item When the Leash fills, deliver a memorable "act to nature" moment before they vanish.
  \item Use SB to escalate Outsider complications: a jealous Patron, a backlash of strange omens, or collateral spiritual harm.
\end{itemize}

% !TEX root = srd_main.tex
% SRD Insert: Elemental Backlash Tables — 8 Elements × Minor/Major
% Assumes booktabs, tabularx, xcolor, tcolorbox are available

\section{Elemental Backlash}\label{sec:backlash-tables}
\index{Backlash}\index{Elements}\index{Story Beats}\index{Realms}\index{Obishaal@Obishaal (Dreams/Thresholds)}

When magic disturbs the weave, the world pushes back. Backlash manifests as fiction-first complications with light mechanical teeth. Each element (and its metaphysical counterpart) has a \textbf{Minor} and \textbf{Major} pattern.

\begin{tcolorbox}[title={Using Backlash at the Table},colback=gray!5,colframe=black]
\textbf{Trigger.} A roll shows a 1 (gaining a (SB)) or the text explicitly says "accept 1 (SB) to escalate."\newline
\textbf{Choose One:} Apply the table's Minor effect, or escalate to Major by adding $\mathbf{+1}$ (SB) immediately.\newline
\textbf{Mechanical Nudge Types.} \emph{Position/Effect shift}, \emph{Clock tick (1/2)}, \emph{Condition}, or \emph{Immediate Cost}.\index{Story Beats!backlash escalation}
\end{tcolorbox}

\subsection*{Realms and Counterparts}\label{subsec:realms-counterparts}
\begin{itemize}
\item \textbf{Earth} \emph{(Realm: Stone)} $\leftrightarrow$ \textbf{Fate} \emph{(Anti-magic, inevitability)}\index{Earth}\index{Fate}
\item \textbf{Fire} \emph{(Realm: Ember)} $\leftrightarrow$ \textbf{Life} \emph{(Vital spark, growth)}\index{Fire}\index{Life}
\item \textbf{Air} \emph{(Realm: Gale)} $\leftrightarrow$ \textbf{Luck/Fortune} \emph{(Ephemera, unlikely turns)}\index{Air}\index{Luck}\index{Fortune}
\item \textbf{Water} \emph{(Realm: Tides)} $\leftrightarrow$ \textbf{Death/Dreams/Thresholds (Obishaal)} \emph{(Passage, veils, the Ways Between)}\index{Water}\index{Death}\index{Dreams}\index{Thresholds}\index{Obishaal@Obishaal}
\end{itemize}

\vspace{0.5em}

% ========================= MASTER TABLE =========================
\begin{table}[h]
\centering
\caption{Backlash by Element (Minor / Major)}
\label{tab:backlash-8x2}
\renewcommand{\arraystretch}{1.12}
\begin{tabularx}{\linewidth}{>{\bfseries}l l >{\raggedright}X >{\raggedright}X}
\toprule
Element & Realm / Counterpart & Minor Backlash (fiction • nudge) & Major Backlash (fiction • nudge) \\
\midrule
Earth & Stone / \emph{Fate} & \emph{Stone binds.} Dust cakes tools; footing slips. • \textbf{Position},–1 or mark \textsc{Encumbered} (light). & \emph{Ground claims.} A fissure opens or masonry seizes. • \textbf{Clock},+1/2 on \emph{Collapse/Entrap} or \textbf{Condition},: \textsc{Pinned}. \\
Fire & Ember / \emph{Life} & \emph{Heat flares.} Smoke blinds; sparks bite. • \textbf{Effect},–1 or \textbf{Condition},: \textsc{Singed} (disadvantage to precise actions). & \emph{Blaze takes.} Fuel ignites or blood races fever-hot. • \textbf{Clock},+1 on \emph{Spreading Fire} or take \emph{1 Harm} ignoring armor. \\
Air & Gale / \emph{Luck} & \emph{Winds misplace.} Words scatter; aim wavers. • \textbf{Position},–1 or \textbf{Clock},+1/2 on \emph{Alarmed Attention}. & \emph{Fickle turn.} An unlikely mishap hits you instead. • \textbf{Immediate Cost},: lose a tool/use \emph{or} \textbf{(SB)},+1 as chaos compounds. \\
Water & Tides / \emph{Obishaal} & \emph{Seep and swell.} Grip slicks; tide shifts. • \textbf{Effect},–1 or \textbf{Condition},: \textsc{Waterlogged} (gear slow). & \emph{Undertow calls.} Passage veers; something is pulled through. • \textbf{Clock},+1 on \emph{Flood/Wayward Current} \emph{or} introduce a \emph{brief} intrusion from the Ways Between. \\
Fate & Anti-magic / \emph{Earth} & \emph{Lines harden.} Probability resists change. • \textbf{Effect},–1 on overt magic or \textbf{Clock},+1/2 on \emph{Inevitable Outcome}. & \emph{Edict falls.} A declared cost must be paid now. • \textbf{Immediate Cost},: sacrifice a resource \emph{or} \textbf{(SB)},+1 and mark \textsc{Omen}. \\
Life & Vital spark / \emph{Fire} & \emph{Growth misfires.} Vines choke; pulse surges. • \textbf{Condition},: \textsc{Overgrowth} (tethered) or \textbf{Effect},–1 on precision. & \emph{Riot of life.} Parasites bloom; healing twists. • \textbf{Clock},+1 on \emph{Biohazard} \emph{or} convert 1 healing into \textbf{(SB)},+1. \\
Luck & Ephemera / \emph{Air} & \emph{Odds flip.} A near-sure thing slips. • \textbf{Position},–1 or \textbf{Clock},+1/2 on \emph{Unwelcome Coincidence}. & \emph{Snake eyes.} Catastrophic fluke. • Force a \emph{re-roll}; if any 1 appears, \textbf{(SB)},+1 \& apply Minor again. \\
Death/Dreams & Ways Between / \emph{Water} & \emph{Veil thins.} Echoes, whispers, cold breath. • \textbf{Condition},: \textsc{Shaken} (first action –1) or \textbf{Clock},+1/2 on \emph{Haunting}. & \emph{Threshold opens.} A path misaligns; a revenant claims a due. • \textbf{Clock},+1 on \emph{Crossing Due} \emph{or} immediate scene intrusion from Obishaal. \\
\bottomrule
\end{tabularx}
\end{table}

% !TEX root = resource_guide_main.tex
% Extended Chapter: Rituals — Philosophy, Procedures, Costs, and Advanced Use
% Assumes: booktabs, tabularx, xcolor, tcolorbox, enumitem, hyperref, amsmath

\section{Rituals (Extended)}\label{sec:rituals-extended}
\index{Rituals}\index{Story Beats}\index{Backlash}\index{Elements}\index{Realms}\index{Obishaal@Obishaal}\index{Patrons}\index{Symbols}\index{Runekeeper}\index{Invoker}\index{Summoner}\index{Caster}

Rituals are \emph{slow magic}: explicit intent, staged action, and negotiated risk. They are how characters bend the world carefully, trading time, components, and narrative exposure for precise results. This section expands the SRD quick-start (\S\ref{sec:universal-rituals}) with procedures, dials, and worked examples.

\subsection{Design Goals}\label{subsec:ritual-goals}\index{Design Philosophy}
\begin{itemize}
\item \textbf{Fiction-first.} Components and steps are story handles, not inventory chores.
\item \textbf{Visible costs.} Every ritual declares \emph{what it costs} (time, component loss, conditions) and \emph{how it risks} (SB)/backlash.
\item \textbf{Tempting choices.} Players can \emph{push}—accept (SB) to escalate position/effect/scale.
\item \textbf{Portable.} Works for Runekeepers, Invokers, Summoners, and Free Casters with minimal chassis-specific tweaks.
\end{itemize}

\subsection{Ritual Procedure}\label{subsec:ritual-procedure}
Use this skeleton for any ritual, published or improvised.

\begin{enumerate}[label=\textbf{Step \arabic*:}, leftmargin=2.2em]
\item \textbf{State Intent.} What do you want? Clarify element/Realm if obvious (Fire for heat, Water for memory, Fate for anti-magic, Obishaal for thresholds).\index{Elements!choosing}
\item \textbf{Choose Scope.} Size, duration, range, and detail. Start modest; escalations come later.
\item \textbf{Lay Components.} Name \emph{two things}: (a) \textbf{Focus} (tool/site/patron sign), (b) \textbf{Fuel} (herb, blood, pact, memory). Decide which is consumed vs. retained.\index{Components}
\item \textbf{Set Time.} Default: \emph{Low 1 minute / Med 5–10 minutes / High 15–20 minutes}. More time improves position/effect; rushing worsens it.
\item \textbf{Call Risks.} Point to the element's \textbf{Backlash} (\S\ref{sec:backlash-condensed}) and the default \textbf{(SB) trigger}: any 1 rolled creates a (SB); re-rolling 1s does not remove (SB) and may add another.
\item \textbf{Roll and Resolve.} Apply position/effect and any clocks. Offer a \emph{push}: take +1 (SB) to step up result now.
\item \textbf{Mark Costs.} Consume components, apply Conditions, or tick wear/concurrency (per chassis). Close the scene hooks the ritual created.
\end{enumerate}

\subsection{Component Economy}\label{subsec:component-economy}
Components are levers, not taxes. Use them to signal tone and stakes.

\begin{table}[h]
\centering
\caption{Components as Narrative Levers}
\label{tab:ritual-components}
\renewcommand{\arraystretch}{1.12}
\begin{tabularx}{\linewidth}{>{\bfseries}l X X}
\toprule
Type & Examples & Mechanical Nudge \\
\midrule
Focus (retained) & Patron token, true-name sigil, saint's nail, ancestral blade. & +1 \emph{position} on setup or advantage to related follow-up actions. \\
Fuel (consumed) & Herb bundle, salt vial, blood drop, silver bead, memory-laden note. & –1 \emph{time step} (faster) or +1 \emph{effect} step if expensive/rare. \\
Site (context) & Crossroads, standing stones, bathhouse, bell tower at midnight. & Shift backlash element or re-route it (e.g., into \emph{Alarmed Attention} clock). \\
Vow (social) & Sworn phrase, bargain pledge, offered favor. & If broken, immediate (SB) +1 and an intrusion linked to the vow. \\
\bottomrule
\end{tabularx}
\end{table}

\subsection{Teamwork and Aid}\label{subsec:ritual-teamwork}\index{Teamwork}
\begin{itemize}
\item \textbf{Hands \& Voices.} Each assistant names one component they contribute; either reduces cast time \emph{or} accepts up to 1 (SB) on the caster's behalf once per ritual.
\item \textbf{Focus Chain.} Passing the Focus around the circle grants advantage on the finishing action but risks \textsc{Distracted} if interrupted.
\item \textbf{Distributed Load.} Splitting a High ritual into two coordinated Mediums avoids a Major backlash but creates two Minor hooks instead.
\end{itemize}

\subsection{Clocks and Outcomes}\label{subsec:ritual-clocks}\index{Clocks}
Tie every consequential ritual to \textbf{named clocks}. Examples: \emph{Spreading Fire}, \emph{Inevitable Outcome}, \emph{Crossing Due}, \emph{Alarmed Attention}. Advancing or reducing clocks is often better than flat bonuses.

\begin{tcolorbox}[title={Outcome Palette},colback=gray!5,colframe=black]
\small \textbf{On a strong result:} full effect, +1 effect step, or Clock –1.\newline
\textbf{On a mixed:} effect with a cost (component consumed; condition applied).\newline
\textbf{On a weak:} effect limited; Clock +1/2; Minor Backlash.\newline
\textbf{On a push:} player may take (SB) +1 to upgrade one step immediately.\end{tcolorbox}

\subsection{Backlash Integration}\label{subsec:ritual-backlash}
Use the condensed table (\S\ref{tab:backlash-condensed}). Calibrate by scene weight: exploratory scenes favor Minor; pivotal moments bait a Major via (SB) +1.

\subsection{Chassis-Specific Notes}\label{subsec:ritual-chassis}
\paragraph{Runekeeper.}\index{Runekeeper} Embed Rites as \emph{accelerants}: a published Rite may count as a Focus that upgrades position or halves time.
\paragraph{Invoker.}\index{Invoker} Symbols accumulate \emph{wear}; a maintenance rite can clear 1 wear mid-scene on success \emph{or} shift backlash from Major to Minor.
\paragraph{Summoner.}\index{Summoner} \emph{Gate} effects occupy concurrency slots. Disruption on broken terms: (SB) +1 and the entity acts on its last instruction.
\paragraph{Caster (Free).}\index{Caster} Tags become explicit ritual steps (\emph{bind, veil, reveal})—chain two compatible tags once/scene for a synergy bump without extra cost.

\subsection{Ritual Templates}\label{subsec:ritual-templates}
Use these fill-in cards to author new content quickly.

\begin{tcolorbox}[title={Template: Utility Rite (Low)},colback=gray!3,colframe=black]
\textbf{Name}: \rule{0.6\linewidth}{0.4pt} \quad \textbf{Element}: \rule{0.28\linewidth}{0.4pt}\\
\textbf{Cast Time}: 1 minute \quad \textbf{Scope}: pocket-scale\\
\textbf{Components}: Focus (kept): \rule{0.4\linewidth}{0.4pt}; Fuel (consumed): \rule{0.35\linewidth}{0.4pt}\\
\textbf{Effect}: \rule{0.9\linewidth}{0.4pt}\\
\textbf{Cost}: \rule{0.9\linewidth}{0.4pt}\\
\textbf{Backlash}: Minor (\S\ref{sec:backlash-condensed}). \textbf{Push}: take (SB) +1 to upgrade one step.
\end{tcolorbox}

\begin{tcolorbox}[title={Template: Scene Rite (Med)},colback=gray!3,colframe=black]
\textbf{Name}: \rule{0.6\linewidth}{0.4pt} \quad \textbf{Element}: \rule{0.28\linewidth}{0.4pt}\\
\textbf{Cast Time}: 5–10 minutes \quad \textbf{Scope}: room/street\\
\textbf{Components}: Focus (kept): \rule{0.4\linewidth}{0.4pt}; Fuel (consumed): \rule{0.35\linewidth}{0.4pt}; Site: \rule{0.3\linewidth}{0.4pt}\\
\textbf{Effect}: \rule{0.9\linewidth}{0.4pt}\\
\textbf{Cost}: \rule{0.9\linewidth}{0.4pt}\\
\textbf{Backlash}: Minor; offer Major via (SB) +1. \textbf{Clocks}: \rule{0.5\linewidth}{0.4pt}
\end{tcolorbox}

\begin{tcolorbox}[title={Template: Set-Piece Rite (High)},colback=gray!3,colframe=black]
\textbf{Name}: \rule{0.6\linewidth}{0.4pt} \quad \textbf{Element}: \rule{0.28\linewidth}{0.4pt}\\
\textbf{Cast Time}: 15–20 minutes \quad \textbf{Scope}: block/fort\\
\textbf{Components}: Focus (kept): \rule{0.4\linewidth}{0.4pt}; Fuel (consumed): \rule{0.35\linewidth}{0.4pt}; Site: \rule{0.3\linewidth}{0.4pt}; Vow: \rule{0.3\linewidth}{0.4pt}\\
\textbf{Effect}: \rule{0.9\linewidth}{0.4pt}\\
\textbf{Cost}: \rule{0.9\linewidth}{0.4pt}\\
\textbf{Backlash}: Likely Major; bait with (SB) +1. \textbf{Clocks}: \rule{0.5\linewidth}{0.4pt}
\end{tcolorbox}

\subsection{Worked Examples}\label{subsec:ritual-examples}
\paragraph{Example 1: Quiet Veil (Team Infiltration).}\index{Rituals!examples}
\emph{Intent:} silence the group for one scene. \emph{Scope:} corridor sweep. \emph{Components:} ash (fuel, consumed), bell (focus, kept). \emph{Time:} 5 minutes. \emph{Risks:} Air/Luck Minor on 1; offer Major to avoid dogs' scent. \emph{Roll:} mixed—effect with cost. \emph{Outcome:} \textsc{Muted} condition until scene ends; Clock –1/2 on \emph{Patrol Pass}. Player takes (SB) +1 to also foil scent (Major avoided by paying the (SB)).

\paragraph{Example 2: River's Memory (Investigation).}
\emph{Intent:} view last night's ferry landing. \emph{Scope:} a few minutes of blurred images. \emph{Components:} bowl, token from dock. \emph{Time:} 10 minutes. \emph{Risk:} Water/Obishaal Minor. \emph{Roll:} strong—clear image; token ruined per Cost. \emph{Outcome:} Clock –1 on \emph{Where did the courier go?}; whisper from the Ways foreshadows a revenant (hook).

\paragraph{Example 3: Fate-Splice (Boss Rescue).}
\emph{Intent:} move the poison consequence from the prince to the knight. \emph{Scope:} one Major consequence. \emph{Components:} paired names on vellum; vow. \emph{Time:} 15 minutes. \emph{Risk:} Fate/Earth. \emph{Roll:} weak—Minor Backlash; \emph{Inevitable Outcome} +1/2. \emph{Push:} (SB) +1 to capture the full consequence anyway. \emph{Outcome:} knight bears the poison; the \textsc{Omen} mark appears (future hook).

\subsection{Safety and Consent}\label{subsec:ritual-safety}\index{Safety}
Rituals often touch body horror, spiritual intrusion, or coercive bargains. Use lines/veils, X-card, script change, or your table's preferred tools. Make \textbf{vows} opt-in; provide non-coercive alternatives with different trade-offs.

\subsection{Optional Modules}\label{subsec:ritual-modules}
\paragraph{Entropy Counters.} Track ritual entropy per scene; at 3+ entropy, the next Minor backlash upgrades to Major automatically. Resets on scene change.
\paragraph{Resonance Sites.} Mark places that boost one element (+1 effect) and hinder its counterpart (–1 position). Crossing a resonance flips the pairing.
\paragraph{Material Tags.} Let special materials act like tags (cold iron, voidglass) granting narrow immunities or redirecting backlash type.

\subsection{GM Troubleshooting}\label{subsec:ritual-troubleshooting}
\begin{itemize}
\item \textbf{Pacing drifts long.} Shorten cast time by consuming an extra Fuel component; keep one meaningful step.
\item \textbf{Risk feels toothless.} Name a clock and advance it on mixed/weak even if the effect lands.
\item \textbf{Runekeeper dominates.} Insert a \emph{Rune Draw} tell for high-grade rites \emph{or} grant Invokers a mid-scene maintenance clear on a solid success.
\item \textbf{Summons flood scene.} Enforce concurrency and Disruption (\S\ref{subsec:ritual-chassis}).
\item \textbf{Casters feel mushy.} Require two explicit tags per ritual step (bind/veil/reveal); grant once/scene synergy bump.
\end{itemize}

\begin{tcolorbox}[title={Summary},colback=gray!5,colframe=black]
Rituals trade \textbf{time, components, and exposure} for \textbf{precision and scale}. Keep costs visible, risk tempting, and outcomes named via clocks. Offer players the choice to buy bigger results with (SB)—then pay off every hook
\end{tcolorbox}

\section{Talent: Cantor's Path --- ``Songs of the Low Rites''}
\label{talent:cantors-path}

\begin{tcolorbox}[colback=black!3,colframe=black!40!white,title={Cantor's Path}]
You echo the liturgies of Patrons through breath and string. Not a sworn celebrant but a perilous mimic, you weave Low Rites into song. It is slower, riskier, and beautiful---but never free.
\end{tcolorbox}

\paragraph*{Type} Standard Talent (15 XP) \quad
\paragraph*{Prerequisites} \textbf{Lore 1+}, \textbf{Performance 2+}, \textbf{Presence 2+} \quad
\paragraph*{Access} Any character (does not require Thiasos membership).

\subsection*{Effect}
You may learn and perform \textbf{Low Rites as Songs}. Each Song counts as knowing the associated Low Rite for performance purposes only.

\begin{itemize}
\item \textbf{Casting Test:} \emph{Lore + Performance vs.\ DV} (default DV = 2--3; see \S\ref{talent:cantors-path-dv}).
\item \textbf{Action Economy:} \emph{1 action to begin;} Song \emph{resolves at the start of your next turn} unless accelerated.
\item \textbf{Scope:} \emph{Low Rites only.}
\item \textbf{Costs:} Pay listed \emph{materials}. On success you do \emph{not} mark Obligation.
\item \textbf{Caveat:} You are still subject to any \emph{corruption effects} of the Rite.
\end{itemize}

\subsection*{Outcomes}
\begin{description}
\item[Success:] The Low Rite takes effect as written.
\item[Partial:] The Rite manifests with \emph{reduced Effect} (–1 step) or \emph{shortened duration}. Mark \textbf{Fatigue 1}.
\item[Failure:] No effect; mark \textbf{Fatigue 1} and the Keeper gains \textbf{+1 SB (Hearts)}. You \emph{do not} mark Obligation.
\item[Interrupted:] Harm, Silence, or disruption before resolution = treat as Failure.
\end{description}

\subsection*{Push It}
You may Push \emph{when you begin casting}:
\begin{itemize}
\item Song resolves immediately instead of next round.
\item Mark \textbf{Fatigue 1, Corruption 1}.
\item Keeper immediately triggers a \textbf{Story Beat}, representing fallout from a Patron, the Road, or social attention.
\item In addition to \emph{Environmental Corruption}, the caster gains \textbf{personal corruption}: a trait from that Patron’s \indexterm{Corruption Table} or direct narrative influence (compulsions, whispers, demands). 
\end{itemize}

\subsection*{Corruption Spread}
When you walk the Cantor’s Path, your corruption does not remain contained. 

\begin{itemize}
\item Mark 1--2 segments on the scene’s \indexterm{Environmental Corruption} track. 
\item All \textbf{PCs and NPCs} present must resist (\textbf{Spirit + Presence} vs.\ max(DV, Tier)) or gain a minor trait from that Patron’s \indexterm{Corruption Table}, or else experience direct influence (whispers, compulsions, urges to serve).
\item The DV for this effect \emph{overrides} any other DV calculation (including Rite DVs or modifiers).
\item \indexterm{Cap 1} and \indexterm{Cap 2} NPCs are affected normally. When you \textbf{Push}, \indexterm{Cap 3} NPCs are also subject to corruption, and the DV increases by +1.
\item All traits and influences gained in this way fade at the next \indexterm{Downtime}.
\end{itemize}

\begin{tcolorbox}[colback=black!2,colframe=black!40!white,title={Sample Corruption Traits}]
\textbf{Cap 1:} Flickering eyes, shadowed whispers, compulsive sigil-marking.  
\textbf{Cap 2:} Patron compulsions (seal doors, seek salt, whisper truths), animal aversion, partial transformations.  
\textbf{Cap 3:} Warping aura, once-per-scene Patron power, fractured personality, acts as a corruption vector.  

\medskip
\textit{These manifestations are temporary unless reinforced; they fade at the next Downtime.}
\end{tcolorbox}

\subsection*{Limits \& Interactions}
\begin{itemize}
\item \textbf{Stacking:} Cannot benefit from the same Rite twice.
\item \textbf{Visibility:} Songs are inherently noticeable. On Failure or Push, assume observers take note.
\item \textbf{Silence/Disruption:} Impose \emph{–1 to –3 dice} at Keeper’s discretion.
\end{itemize}

\subsection*{DV Guidance}
\label{talent:cantors-path-dv}
\begin{description}
\item[DV 2:] Personal augments, simple glamours.
\item[DV 3:] Zone effects, multi-target edges, time-bending moods.
\item[+1 DV:] Hostile crowds, loud environments, warded spaces.
\end{description}

\subsection*{Examples}
\begin{itemize}
\item \textbf{Perfect Note (Low):} Song resolves next round (DV 3). Push it to resolve immediately, mark +1 Fatigue, +1 Corruption, and trigger a Story Beat.
\item \textbf{Endless Revel (Low):} Creates revel zone on next turn. Push it to start now with the same costs.
\end{itemize}

