\chapter{Deck-Based Generators}
\label{chap:deck-generators}

\section{Introduction to Deck Generators}
\label{sec:deck-intro}

\index{Deck Generators} Deck generators in \textbf{Fate's Edge} transform random card draws into coherent narrative elements. Each deck has a distinct purpose and suit meanings so that randomness serves the story rather than derailing it. These generators provide structured inspiration for GMs while maintaining the game's narrative-first philosophy.

\section{Standard Deck Structure}
\label{sec:deck-structure}
\index{Decks!Structure}

\textbf{Fate's Edge} uses several card-based tools, each with specialized suit meanings:

\begin{description}
\item[\textbf{Travel Decks} (regional, 52-card)] \index{Decks!Travel} Used for journey content and location-based adventures.
\begin{itemize}
    \item \textbf{Spade} = Place/Location
    \item \textbf{Heart} = Actor/Faction
    \item \textbf{Club} = Pressure/Complication
    \item \textbf{Diamond} = Reward/Opportunity
\end{itemize}

\item[\textbf{Deck of Consequences} (scene drama)] \index{Decks!Consequences} Used for immediate complications and narrative twists during gameplay.
\begin{itemize}
    \item \textbf{Hearts} = Social/Emotional Fallout
    \item \textbf{Spades} = Harm/Escalation
    \item \textbf{Clubs} = Material Cost/Resource Drain
    \item \textbf{Diamonds} = Magical/Spiritual Disturbance
\end{itemize}
\end{description}

\noindent\textbf{Important:} Never mix suit meanings across decks. When rules reference ``Spade/Club/Diamond,'' they mean the \emph{Travel Deck}. When they say ``Hearts/Spades/Clubs/Diamonds,'' they mean the \emph{Deck of Consequences}.

\section{Rank Severity and Clock Size}
\label{sec:rank-severity}
\index{Decks!Rank Severity}
\index{Clocks!Size Determination}

Card rank sets the size/significance of the primary Clock:

\begin{itemize}
\item \textbf{2--5 (Minor):} 4-segment Clock \index{Clocks!4-segment}
\item \textbf{6--10 (Standard):} 6-segment Clock \index{Clocks!6-segment}
\item \textbf{J, Q, K (Major):} 8-segment Clock \index{Clocks!8-segment}
\item \textbf{Ace (Pivotal):} 10-segment Clock \index{Clocks!10-segment}
\end{itemize}

\textbf{Color Influence:}
\begin{itemize}
\item \textbf{Black suits} (\textcolor{black}{$\spadesuit$}, \textcolor{black}{$\clubsuit$}): Travel hazards, tangible threats, fatigue \index{Decks!Black Suits}
\item \textbf{Red suits} (\textcolor{red}{$\heartsuit$}, \textcolor{red}{$\diamondsuit$}): Social intrigue, reputational pressure, emotional complications \index{Decks!Red Suits}
\end{itemize}

\section{Draw Procedures}
\label{sec:draw-procedures}
\index{Decks!Draw Procedures}

\subsection{Quick Hook (2 cards)}
\label{subsec:quick-hook}
\index{Decks!Quick Hook}

Ideal for spontaneous scene generation or when players zag unexpectedly:
\begin{enumerate}
\item Draw one \textbf{Spade} (place) and one \textbf{Heart} (actor/faction).
\item Use the higher rank to set Clock size.
\item Combine elements into a simple, compelling scenario.
\end{enumerate}

\subsection{Full Seed (4 cards)}
\label{subsec:full-seed}
\index{Decks!Full Seed}

For full adventures or significant arcs:
\begin{enumerate}
\item Draw until one card of each suit appears:
\begin{itemize}
    \item \textbf{Spade} = Primary location \index{Decks!Spade}
    \item \textbf{Heart} = Main actor/faction \index{Decks!Heart}
    \item \textbf{Club} = Central complication \index{Decks!Club}
    \item \textbf{Diamond} = Key reward/opportunity \index{Decks!Diamond}
\end{itemize}
\item The highest rank sets the main Clock size.
\item If multiple face cards or Aces appear, create parallel Clocks for secondary threats or opportunities.
\end{enumerate}

\subsection{Act Builder}
\label{subsec:act-builder}
\index{Decks!Act Builder}

Structure sessions or multi-part adventures:
\begin{enumerate}
\item Draw three cards: setting (\textbf{Spade}), actor (\textbf{Heart}), complication (\textbf{Club}).
\item Treat \textbf{Diamond} cards drawn during play as foreshadowed opportunities or act payoffs.
\item Highest rank determines the session's primary challenge scope.
\end{enumerate}

\section{Using the Deck in Play}
\label{sec:deck-in-play}
\index{Decks!Usage}

\begin{enumerate}
\item Players roll; each die showing \textbf{1} generates \textbf{1 Complication Point (CP)}. \index{Complication Point (CP)}
\item The GM chooses one method for that roll:
\begin{enumerate}
    \item \textbf{Direct Spend:} Translate CP into immediate consequences or clock ticks.
    \item \textbf{Deck Draw:} Draw up to \textbf{min(CP, 3)} cards and synthesize a single twist guided by suit and highest rank.
\end{enumerate}
\item Interpret the cards to create a coherent complication that advances the narrative.
\end{enumerate}

\section{Combo Rules}
\label{sec:combo-rules}
\index{Decks!Combo Rules}

Special combinations add texture:
\begin{description}
\item[\textbf{Pair} (same rank)] Recurring motif with a twist. \index{Decks!Pairs}
\item[\textbf{Run} (3+ sequential ranks)] Momentum—reduce the main Clock by 1 segment. \index{Decks!Runs}
\item[\textbf{Flush} (3+ same suit)] Strongly theme the act toward that suit's axis. \index{Decks!Flushes}
\item[\textbf{Face + Ace}] Reveal a hidden patron or power behind the element. \index{Decks!Face + Ace}
\item[\textbf{All one color}] GM gains \textbf{+1 CP} to use in that scene. \index{Decks!Color Combos}
\end{description}

\section{Regional Generator Summary}
\label{sec:regional-summary}
\index{Decks!Regional Summary}

\begin{table}[htbp]
\centering
\begin{tabular}{p{3.3cm}p{5.3cm}p{5.0cm}}
\toprule
\textbf{Region} & \textbf{Theme} & \textbf{Special Mechanics} \\
\midrule
Acasia & Broken Marches & Curse motifs; every Ace adds a lingering omen \\
Aelaerem & Hearth \& Hollow & Red-thread motifs; Ace echoes quiet bells/watch-geese \\
Aeler & Crowns \& Under-Vaults & Stone/breath motifs; Ace keys click, bells answer \\
Aelinnel & Stone, Bough, Bright Things & Moonlight motifs; Ace adds a shortcut where none should be \\
Black Banners & Condotta \& Crowns & War \& winter motifs; Ace: weapons remember, ice holds the dead \\
Ecktoria & Marble \& Fire & Imperial forms; Ace carves precedent in marble \\
Kahfagia & Pilot's Mirror & Lantern-law jurisdiction shifts; Ace redefines lanes \\
Linn & Skerries \& Storm-Oaths & Sea omens; Ace horns on wind, white horses on swell \\
Mistlands & Bells, Salt, Breath & Breath/boundary motifs; Ace: bells answer across water \\
Silkstrand & City of Bridges \& Dyewater & Dye/bridge motifs; Ace adds a lingering omen \\
Theona & Three Greens, No Ninth & “No Ninth” custom; Ace adds a telling omission \\
Thepyrgos & City of a Thousand Stairs & Height/sound motifs; Ace echoes bells/wind/stair-steps \\
Ubral & Stone Between Spears & Upland motifs; Ace echoes horns/heather/stone \\
Valewood & Empire Under Leaves & Empire echoes (J/Q/K add relic-logic); Ace rearranges approach \\
Vhasia & Fractured Sun & Broken-sun motifs; Ace blots medal/scratches milestone \\
Vilikari & Laurels \& Longhouses & Two-laws motifs; Ace shows wolf/eagle side-by-side \\
Viterra & Last Kingdom & Legacy, parishes, and final-stand themes \\
Wilds & Roads, Ruins, Weather & Reskin palette for any biome \\
Zakov & Salt \& Serpent & Salt \& serpent omens; Ace: tides remember, reefs shift, deep listens \\
\bottomrule
\end{tabular}
\caption{Regional Generator Summary}
\label{tab:regional-summary}
\end{table}

\section{NPC Generation Deck}
\label{sec:npc-generation}
\index{NPCs!Generation Deck}

Every NPC should feel like a person with desires, convictions, and contradictions. This deck lets you assemble a complete profile quickly by drawing one element from each category.

\subsection{Generation Categories}
\label{subsec:npc-categories}

\begin{description}
\item[\textbf{Ambition}] What they seek to achieve or obtain. \index{NPCs!Ambition}
\item[\textbf{Belief}] The principle or philosophy guiding their worldview. \index{NPCs!Belief}
\item[\textbf{Attitude}] How they present themselves and interact day-to-day. \index{NPCs!Attitude}
\item[\textbf{Twist}] A contradiction or hidden facet that creates tension. \index{NPCs!Twist}
\end{description}

\subsection{Using the NPC Generator}
\label{subsec:npc-usage}

Select or draw one from each column and consider the frictions between public ambition, private belief, surface attitude, and the twist.

\begin{table}[htbp]
\centering
\begin{tabular}{p{3.3cm}p{3.3cm}p{3.3cm}p{3.3cm}}
\toprule
\textbf{Ambition} & \textbf{Belief} & \textbf{Attitude} & \textbf{Twist} \\
\midrule
Power & Might makes right & Arrogant & Secretly insecure \\
Wealth & Ends justify means & Charismatic & Betraying their allies \\
Revenge & Honor above all & Cold & Working for their enemy \\
Love & Truth is sacred & Friendly & Hiding a dark past \\
Knowledge & Loyalty is paramount & Paranoid & Actually an impostor \\
Survival & Family above all & Cruel & Deeply compassionate \\
Fame & Justice must prevail & Pious & Corrupted by power \\
Freedom & Fate can be changed & Optimistic & Hopelessly cynical \\
Protection & Tradition must be upheld & Pessimistic & Revolutionary at heart \\
Control & Change is necessary & Calculating & Acts on impulse \\
Recognition & The system works & Naive & Cynical manipulator \\
\bottomrule
\end{tabular}
\caption{NPC Generation Categories}
\label{tab:npc-categories}
\end{table}

\section{Practical Deck Usage Examples}
\label{sec:deck-examples}
\index{Decks!Examples}

\subsection{Example 1: Quick Scene Generation}
\label{subsec:example-quick}

The party detours through the Mistlands. The GM draws:
\begin{itemize}
\item \textbf{Spade (8)}: Ancient standing stones covered in moss
\item \textbf{Heart (Queen)}: A territorial spirit guardian
\end{itemize}
A 6-segment Clock \emph{Spirit's Wrath} begins: the guardian demands tribute for safe passage.

\subsection{Example 2: Consequences During Play}
\label{subsec:example-consequences}

Kael misses a stealth roll and generates \textbf{2 CP}. The GM draws:
\begin{itemize}
\item \textbf{Hearts (7)}: Social complication
\item \textbf{Clubs (3)}: Resource cost
\end{itemize}
Synthesis: \emph{A ceremonial urn shatters; cultists recognize your patron’s mark. Future dealings will demand extra tribute and materials.}

\subsection{Example 3: NPC Creation}
\label{subsec:example-npc}

Merchant in Valewood:
\begin{itemize}
\item \textbf{Ambition}: Wealth
\item \textbf{Belief}: Family above all
\item \textbf{Attitude}: Charismatic
\item \textbf{Twist}: Secretly compassionate
\end{itemize}
Result: \emph{A charming hard-bargainer who supports a large family and quietly donates to orphanages—even aiding struggling competitors.}

\section{GM Guidance for Deck Usage}
\label{sec:deck-guidance}
\index{Decks!GM Guidance}

\subsection{When to Use Which Deck}
\label{subsec:deck-selection}

\begin{itemize}
\item \textbf{Travel Decks}: journey planning, location adventures, regional exploration
\item \textbf{Deck of Consequences}: immediate twists during active scenes
\item \textbf{NPC Generator}: fast creation with built-in tension and hooks
\end{itemize}

\subsection{Interpreting Card Draws}
\label{subsec:card-interpretation}

\begin{itemize}
\item Prioritize narrative coherence over literalism.
\item Use suits as inspiration, not constraints.
\item Combine cards into layered complications rather than parallel noise.
\item Remember: players can mitigate, pivot, or overcome deck outcomes.
\end{itemize}

\subsection{Balancing Randomness and Narrative}
\label{subsec:balancing-randomness}

\begin{itemize}
\item Draw when you want surprise or need a nudge.
\item Ignore or modify draws that don’t serve the current story.
\item Treat combinations as creative prompts, not mandates.
\item The goal is to \emph{enhance} the narrative, not derail it.
\end{itemize}

\end{chapter}
