i\chapter{Appendices}
\label{chap:appendices}

%----------------------------------
% Visual setup for quick-ref tables
%----------------------------------
\definecolor{FeRow}{HTML}{F5F7FA}
\renewcommand{\arraystretch}{1.15}
\newcolumntype{Y}{>{\raggedright\arraybackslash}X}
\newcolumntype{C}{>{\centering\arraybackslash}p{1.8cm}}
\newcommand{\feTableStart}{\rowcolors{2}{FeRow}{white}\small}
\newcommand{\feTableEnd}{\rowcolors{2}{}{}}

\section{Quick Reference Sheets}
\label{sec:quick-reference}
\index{Appendices!Quick Reference}

\subsection{Core Mechanic Quick Reference}
\label{subsec:core-mechanic-ref}
\index{Appendices!Core Mechanic}

\begin{enumerate}
\item \textbf{Approach}: Player states intent and method (Attribute + Skill combination).\index{Core Mechanic!Approach}
\item \textbf{Execution}: Roll dice pool of d10s. Each 6+ is a success; each 1 generates 1 Story Beat.\index{Core Mechanic!Execution}
\item \textbf{Outcome}:
  \begin{itemize}
  \item \textbf{Basic} — Roll as-is; all 1s generate SB.
  \item \textbf{Detailed} — Re-roll one die showing 1.
  \item \textbf{Intricate} — Re-roll all 1s; add a positive flourish on success.
  \end{itemize}
\end{enumerate}

\subsection{Attribute and Skill Summary}
\label{subsec:attributes-skills-ref}
\index{Appendices!Attributes and Skills}

\textbf{Attributes:}
\begin{description}
\item[Body] Physical strength, endurance, athletic ability.\index{Attributes!Body}
\item[Wits] Perception, cleverness, quick thinking.\index{Attributes!Wits}
\item[Spirit] Willpower, intuition, spiritual resilience.\index{Attributes!Spirit}
\item[Presence] Charm, command, social influence.\index{Attributes!Presence}
\end{description}

\textbf{Skill Levels:}
\begin{description}
\item[0 Untrained] Rely on raw Attribute only.\index{Skills!Untrained}
\item[1 Familiar] Basic competence, occasional use.\index{Skills!Familiar}
\item[2 Skilled] Reliable training, regular practice.\index{Skills!Skilled}
\item[3 Expert] Professional mastery, notable ability.\index{Skills!Expert}
\item[4 Master] Renowned specialist, exceptional talent.\index{Skills!Master}
\item[5 Legendary] Near-mythic capability, extraordinary.\index{Skills!Legendary}
\end{description}

\subsection{Experience Point Costs}
\label{subsec:xp-costs-ref}
\index{Appendices!XP Costs}

\begin{center}
\feTableStart
\begin{tabularx}{\linewidth}{@{}l l l @{}}
\toprule
\textbf{Improvement} & \textbf{Cost} & \textbf{Downtime} \\
\midrule
Attribute increase & New rating $\times$ 3 XP & New rating days \\
Skill increase & New level $\times$ 2 XP & New level days \\
On-Screen Follower & Cap$^2$ XP & 1--3 days \\
Minor Asset & 4 XP & 1 day \\
Standard Asset & 8 XP & 1 week \\
Major Asset & 12 XP & 1 month \\
\bottomrule
\end{tabularx}
\feTableEnd
\end{center}

\subsection{Difficulty Value (DV) Reference}
\label{subsec:dv-reference}
\index{Appendices!DV Reference}

\begin{center}
\feTableStart
\begin{tabularx}{\linewidth}{@{}>{\centering\arraybackslash}p{1.2cm} l Y @{}}
\toprule
\textbf{DV} & \textbf{Difficulty} & \textbf{Typical Situations} \\
\midrule
2 & Routine   & Clear intent, modest stakes, controlled environment \\
3 & Pressured & Time pressure, mild resistance, partial information \\
4 & Hard      & Hostile conditions, active opposition, precise timing \\
5+ & Extreme  & Multiple constraints, high precision, dramatic failure risk \\
\bottomrule
\end{tabularx}
\feTableEnd
\end{center}

\section{Deck Usage Reference}
\label{sec:deck-reference}
\index{Appendices!Deck Reference}

\subsection{Deck Types and Meanings}
\label{subsec:deck-types}
\index{Appendices!Deck Types}

\begin{description}
\item[\textbf{Travel Decks} (regional, 52-card)] Used for journey content and location-based adventures.\index{Decks!Travel}
\begin{itemize}
\item Spade $=$ Place/Location
\item Heart $=$ Actor/Faction
\item Club $=$ Pressure/Complication
\item Diamond $=$ Leverage/Reward
\end{itemize}

\item[\textbf{Deck of Consequences} (scene drama)] Used for immediate complications and narrative twists.\index{Decks!Consequences}
\begin{itemize}
\item Hearts $=$ Social/Emotional fallout
\item Spades $=$ Harm/Escalation
\item Clubs $=$ Material cost/Resource drain
\item Diamonds $=$ Magical/Spiritual disturbance
\end{itemize}
\end{description}

\textbf{Important:} Never mix suit meanings across decks. Travel deck suits differ from Consequences deck suits.

\subsection{Deck Usage Procedure}
\label{subsec:deck-procedure}
\index{Appendices!Deck Procedure}

After a roll generating Story Beats:
\begin{enumerate}
\item \textbf{Direct Spend}: Translate SB into immediate consequences or clock ticks.
\item \textbf{Deck Draw}: Draw up to $\min(\text{SB},\,3)$ cards and synthesize a single twist.
\item Interpret cards based on suit meanings and highest rank.
\end{enumerate}

\subsection{Rank Severity Guide}
\label{subsec:rank-severity}
\index{Appendices!Rank Severity}

\begin{description}
\item[Ace--3] Minor inconvenience or flavor complication.\index{Decks!Ranks!Minor}
\item[4--6] Moderate setback with narrative impact.\index{Decks!Ranks!Moderate}
\item[7--9] Significant consequence altering the scene.\index{Decks!Ranks!Significant}
\item[10--King] Major fallout introducing new problems or lasting effects.\index{Decks!Ranks!Major}
\end{description}

\section{Magic System Quick Reference}
\label{sec:magic-reference}
\index{Appendices!Magic Reference}

\subsection{Magic Paths Comparison}
\label{subsec:magic-paths-ref}
\index{Appendices!Magic Paths}

\begin{center}
\feTableStart
\begin{tabularx}{\linewidth}{@{}l l l l @{}}
\toprule
\textbf{Path} & \textbf{Requirements} & \textbf{Key Feature} & \textbf{Risk Type} \\
\midrule
Caster (Freeform) & Caster's Gift (2 XP) & Flexible improvisation & Backlash \\
Runekeeper (Rites) & Thiasos + Codex (6 XP) & Structured Rites & Obligation \\
Invoker (Symbols) & Patron's Symbol (4 XP) & Ritual precision & Symbol compromise \\
\bottomrule
\end{tabularx}
\feTableEnd
\end{center}

\subsection{Casting Loop Summary}
\label{subsec:casting-loop-ref}
\index{Appendices!Casting Loop}

\begin{enumerate}
\item \textbf{Channel}: Wits + Arcana roll to gather Potential.\index{Magic!Channel}
\item \textbf{Weave}: Wits + Art roll to shape spell effect.\index{Magic!Weave}
\item \textbf{Backlash}: SB spent through thematic consequences.\index{Magic!Backlash}
\end{enumerate}

\subsection{Eight Elements of Magic}
\label{subsec:elements-ref}
\index{Appendices!Magic Elements}

\begin{description}
\item[Earth] Solidity, stability, foundation.\index{Magic!Elements!Earth}
\item[Fire] Energy, transformation, destruction.\index{Magic!Elements!Fire}
\item[Air] Movement, speed, freedom.\index{Magic!Elements!Air}
\item[Water] Fluidity, healing, adaptability.\index{Magic!Elements!Water}
\item[Fate] Destiny, inevitability, causality.\index{Magic!Elements!Fate}
\item[Life] Vitality, creation, growth.\index{Magic!Elements!Life}
\item[Luck] Chance, unpredictability, probability.\index{Magic!Elements!Luck}
\item[Death/Dreams] Endings, thresholds, subconscious.\index{Magic!Elements!Death}
\end{description}

\section{Combat and Conflict Reference}
\label{sec:combat-reference}
\index{Appendices!Combat Reference}

\subsection{Position States}
\label{subsec:position-ref}
\index{Appendices!Position States}

\begin{description}
\item[Controlled] Advantageous position, minor consequences.\index{Combat!Controlled}
\item[Risky] Standard situation, moderate consequences.\index{Combat!Risky}
\item[Desperate] Disadvantaged, severe consequences.\index{Combat!Desperate}
\end{description}

\subsection{Harm Levels and Effects}
\label{subsec:harm-ref}
\index{Appendices!Harm Reference}

\begin{center}
\feTableStart
\begin{tabularx}{\linewidth}{@{}l l l l @{}}
\toprule
\textbf{Harm Level} & \textbf{SB Generation} & \textbf{Penalty} & \textbf{Recovery} \\
\midrule
Minor & 1 SB on next 2 rolls & $-1$ die to related actions & Rest or basic care \\
Moderate & 1 SB on next roll & $-1$ die to most actions & Medical treatment \\
Severe & 2 SB on next roll & $-2$ dice to most actions & Extended care \\
Critical & 3 SB on next roll & Incapacitated & Major intervention \\
\bottomrule
\end{tabularx}
\feTableEnd
\end{center}

\subsection{Range Bands}
\label{subsec:range-bands-ref}
\index{Appendices!Range Bands}

\begin{description}
\item[Close] Arm's length, grappling distance.\index{Combat!Close Range}
\item[Near] Same room or immediate area.\index{Combat!Near Range}
\item[Far] Visible but not immediately reachable.\index{Combat!Far Range}
\item[Absent] Off-screen or out of current scene.\index{Combat!Absent Range}
\end{description}

\subsection{Movement Actions}
\label{subsec:movement-ref}
\index{Appendices!Movement}

\begin{itemize}
\item \textbf{1 Move}: Shift one range band (Close$\leftrightarrow$Near or Near$\leftrightarrow$Far).
\item \textbf{Dash Action}: Shift two bands in one action.
\item \textbf{Disengage}: Test to leave Close range when threatened.
\item \textbf{Sprint}: Rapid movement across the battlefield.
\end{itemize}

\section{Resource Management Reference}
\label{sec:resource-reference}
\index{Appendices!Resource Reference}

\subsection{Story Beat Economy}
\label{subsec:sb-economy-ref}
\index{Appendices!Story Beats}

\begin{center}
\feTableStart
\begin{tabularx}{\linewidth}{@{}>{\centering\arraybackslash}p{1.8cm} l Y @{}}
\toprule
\textbf{SB Cost} & \textbf{Effect Scale} & \textbf{Typical Effects} \\
\midrule
1 SB & Minor pressure & Noise, trace, time loss, +1 Supply segment \\
2 SB & Moderate setback & Alarm, lose position/cover, lesser foe appears \\
3 SB & Serious trouble & Reinforcements, key gear breaks, major complication \\
4+ SB & Major turn & Trap springs, authority arrives, scene shifts dramatically \\
\bottomrule
\end{tabularx}
\feTableEnd
\end{center}

\subsection{Boon Usage Guide}
\label{subsec:boon-usage-ref}
\index{Appendices!Boons}

\begin{center}
\feTableStart
\begin{tabularx}{\linewidth}{@{}l l Y @{}}
\toprule
\textbf{Boon Cost} & \textbf{Effect} & \textbf{Limitations} \\
\midrule
1 Boon & Re-roll one die            & Once per action \\
1 Boon & Activate on-screen Asset   & Plausibility test required \\
1 Boon & Improve Position by 1 step & One step maximum per action \\
2 Boons & Convert to 1 XP           & Once per session; max 2 XP \\
Variable & Power Rites/Abilities    & As specified \\
\bottomrule
\end{tabularx}
\feTableEnd
\end{center}

\textbf{Boon Limits:}
\begin{itemize}
\item Hold maximum of 5 Boons at any time.
\item Trim to 2 Boons at scene endings.
\item Maximum 2 Boons from failures per character per scene.
\item Conversion: 2 Boons $=$ 1 XP (max 2 XP per session).
\end{itemize}

\subsection{Supply Clock States}
\label{subsec:supply-ref}
\index{Appendices!Supply Clock}

\begin{description}
\item[\textbf{Full Supply} (0)] No penalties; well-equipped.\index{Supply!Full}
\item[\textbf{Low Supply} (2)] Minor narrative complications.\index{Supply!Low}
\item[\textbf{Dangerously Low} (3)] Each character gains 1 Fatigue.\index{Supply!Dangerously Low}
\item[\textbf{Out of Supply} (4)] Severe penalties; starvation risk.\index{Supply!Out of Supply}
\end{description}

\section{Travel and Exploration Reference}
\label{sec:travel-reference}
\index{Appendices!Travel Reference}

\subsection{Travel Clock Sizes}
\label{subsec:travel-clocks-ref}
\index{Appendices!Travel Clocks}

\begin{description}
\item[4 segments] Short, straightforward journeys.\index{Travel!Clocks!4-segment}
\item[6 segments] Standard travel legs.\index{Travel!Clocks!6-segment}
\item[8 segments] Extended or complex journeys.\index{Travel!Clocks!8-segment}
\item[10 segments] Epic or highly dangerous travel.\index{Travel!Clocks!10-segment}
\end{description}

\subsection{Card Draw Procedures}
\label{subsec:card-draw-ref}
\index{Appendices!Card Draw}

\textbf{Quick Hook (2 cards):}
\begin{itemize}
\item Draw one Spade (place) and one Heart (actor).
\item Use higher rank to set clock size.
\end{itemize}

\textbf{Full Seed (4 cards):}
\begin{itemize}
\item Draw until one card of each suit appears.
\item Spade $=$ location, Heart $=$ faction, Club $=$ pressure, Diamond $=$ leverage.
\item Highest rank sets main clock size.
\end{itemize}

\section{Character Advancement Guide}
\label{sec:advancement-reference}
\index{Appendices!Advancement}

\subsection{Reputation Tiers}
\label{subsec:reputation-tiers-ref}
\index{Appendices!Reputation Tiers}

\begin{description}
\item[Tier I -- Rookie (0--40 XP)] Local reputation; prestige locked.\index{Reputation Tiers!Rookie}
\item[Tier II -- Seasoned (41--90 XP)] Regional notice; prestige may unlock.\index{Reputation Tiers!Seasoned}
\item[Tier III -- Veteran (91--150 XP)] National influence; second follower suggested.\index{Reputation Tiers!Veteran}
\item[Tier IV -- Paragon (151--220 XP)] Movers and shakers; rivals emerge.\index{Reputation Tiers!Paragon}
\item[Tier V -- Mythic (221+ XP)] Legendary status; kingdoms respond.\index{Reputation Tiers!Mythic}
\end{description}

\subsection{Player Archetypes}
\label{subsec:archetypes-ref}
\index{Appendices!Player Archetypes}

\begin{description}
\item[Solo] 70--90\% self investment; minimal followers/assets.\index{Character Build!Solo}
\item[Mixed] 50--65\% self; balanced with followers/assets.\index{Character Build!Mixed Player}
\item[Mastermind] 25--40\% self; focuses on networks and followers.\index{Character Build!Mastermind}
\end{description}

\section{Gamemaster Guidance}
\label{sec:gm-reference}
\index{Appendices!GM Guidance}

\subsection{Session Preparation Checklist}
\label{subsec:session-prep-ref}
\index{Appendices!Session Preparation}

\begin{itemize}
\item Review previous session notes and unresolved threads.
\item Set initial SB budget (4 + character tiers).
\item Prepare key scenes and opposition.
\item Have consequence ideas ready for common actions.
\item Check ongoing clocks and faction status.
\item Prepare travel routes if journey expected.
\end{itemize}

\subsection{Adjudication Principles}
\label{subsec:adjudication-ref}
\index{Appendices!Adjudication}

\begin{itemize}
\item \textbf{Fiction First}: Mechanics serve the narrative, not replace it.
\item \textbf{Fail Forward}: Even failures should advance the story.
\item \textbf{Player Agency}: Offer choices rather than impose outcomes.
\item \textbf{Transparent Costs}: Clearly communicate risks and stakes.
\item \textbf{Collaborative Spirit}: Work with players to create compelling fiction.
\end{itemize}

\subsection{Pacing Tools}
\label{subsec:pacing-tools-ref}
\index{Appendices!Pacing}

\begin{itemize}
\item Use clocks to create urgency and track progress.
\item Vary scene intensity between high and low stakes.
\item Include downtime for character development.
\item Balance action, investigation, and social scenes.
\item Use travel sequences for world-building and random encounters.
\end{itemize}

\section{Common Rules Questions}
\label{sec:rules-questions}
\index{Appendices!Rules Questions}

\subsection{Core Mechanic Clarifications}
\label{subsec:core-clarifications}
\index{Appendices!Core Clarifications}

\textbf{Q: Can players re-roll 1s to remove Story Beats?}\\
A: No. Re-rolling 1s does not remove SB already generated. If re-rolled dice show 1 again, they generate additional SB.\index{Story Beats!Re-rolling}

\textbf{Q: When does a miss award a Boon?}\\
A: Only when all three conditions are met: procedure followed, stakes stated, and consequence lands immediately.\index{Boons!Award Conditions}

\textbf{Q: Can players assist each other on every action?}\\
A: Yes, but total assist dice are capped at +3 from all sources combined.\index{Mechanics!Assist Maximum}

\subsection{Magic System Questions}
\label{subsec:magic-questions}
\index{Appendices!Magic Questions}

\textbf{Q: Can a character use multiple magic paths?}\\
A: Yes, but each path has its own tracking (Backlash, Obligation, Symbol states). Specializing is more efficient.\index{Magic!Mixing Paths}

\textbf{Q: How does Crack the Seal work for Invokers?}\\
A: Convert a ritual to instant casting by setting the Symbol to \emph{Compromised} and marking +2/+3 Obligation.\index{Invokers!Crack the Seal}

\textbf{Q: What happens when a Patron's Obligation clock fills?}\\
A: The GM resolves the debt in-fiction through service demands, omens, or narrative consequences.\index{Magic!Obligation Resolution}

\subsection{Combat and Conflict Questions}
\label{subsec:combat-questions}
\index{Appendices!Combat Questions}

\textbf{Q: How does the Over-Stack rule work?}\\
A: If the party enters a scene with $\geq$ 3 structural advantages, either start one challenge at +1 difficulty \emph{or} bank +1 SB.\index{Combat!Over-Stack}

\textbf{Q: Can players spend Boons to improve Position?}\\
A: Yes. 1 Boon improves Position by 1 step for the current action.\index{Combat!Position Improvement}

\textbf{Q: How does harm recovery work?}\\
A: Minor clears with rest; moderate requires medical treatment (DV 2); severe needs extended care (DV 3); critical requires major intervention.\index{Combat!Harm Recovery}

\section{Regional Quick Reference}
\label{sec:regional-reference}
\index{Appendices!Regional Reference}

\subsection{Major Regions and Themes}
\label{subsec:regions-themes}
\index{Appendices!Regions}

\begin{description}
\item[Acasia] Broken marches, curses, lawless territory.\index{Acasia}
\item[Aeler] Underground vaults, dwarven culture, engineering.\index{Aeler}
\item[Ecktoria] Imperial remnants, bureaucracy, coinhouses.\index{Ecktoria}
\item[Kahfagia] Maritime trade, lantern-law, convoys.\index{Kahfagia}
\item[Mistlands] Bells, wards, supernatural boundaries.\index{Mistlands}
\item[Silkstrand] Trade hub, intrigue, Acasia's only major city.\index{Silkstrand}
\item[Vhasia] Fractured sun, political fragmentation.\index{Vhasia}
\item[Viterra] Last kingdom, river-based power.\index{Viterra}
\item[Valewood] Forest empire, natural magic.\index{Valewood}
\item[Ykrul] Steppe nomads, wolf standards.\index{Ykrul}
\item[Zakov] Salt and serpent, criminal syndicates.\index{Zakov}
\end{description}

\subsection{Key Geographical Features}
\label{subsec:geography-ref}
\index{Appendices!Geography}

\begin{itemize}
\item \textbf{Amaranthine Sea}: Western sea, major trade routes.\index{Geography!Amaranthine Sea}
\item \textbf{Dolmis Sea}: Inner sea, island networks.\index{Geography!Dolmis Sea}
\item \textbf{Astroegro Straits}: Crucial maritime chokepoint.\index{Geography!Astroegro Straits}
\item \textbf{Belworth River}: Major river system, boundary between regions.\index{Geography!Belworth River}
\item \textbf{Aelerian Mountains}: Extensive underground networks.\index{Geography!Aelerian Mountains}
\end{itemize}

\section{Campaign Management Tools}
\label{sec:campaign-tools}
\index{Appendices!Campaign Tools}

\subsection{Session Log Template}
\label{subsec:session-log-template}
\index{Appendices!Session Log}

\begin{center}
\feTableStart
\begin{tabularx}{\linewidth}{@{}p{4cm} Y @{}}
\toprule
\textbf{Session Element} & \textbf{Notes} \\
\midrule
Session Date & \\
Players Present & \\
Major Objectives & \\
Key Scenes & \\
Story Beats Generated & \\
Boons Awarded/Spent & \\
Clocks Advanced/Completed & \\
XP Awards & \\
Downtime Activities & \\
Next Session Hooks & \\
\bottomrule
\end{tabularx}
\feTableEnd
\end{center}

\subsection{Campaign Clock Examples}
\label{subsec:campaign-clocks}
\index{Appendices!Campaign Clocks}

\begin{description}
\item[\textbf{Faction Rivalry} (8 segments)] Tracks escalating conflict between major powers.\index{Clocks!Faction Rivalry}
\item[\textbf{Ancient Curse} (6 segments)] Progress of a regional supernatural affliction.\index{Clocks!Ancient Curse}
\item[\textbf{Imperial Collapse} (10 segments)] Decline of a major governing power.\index{Clocks!Imperial Collapse}
\item[\textbf{Magical Cataclysm} (8 segments)] Buildup to a reality-altering event.\index{Clocks!Magical Cataclysm}
\item[\textbf{Trade War} (6 segments)] Economic conflict affecting multiple regions.\index{Clocks!Trade War}
\end{description}

\subsection{Adventure Structure Template}
\label{subsec:adventure-template}
\index{Appendices!Adventure Template}

\textbf{Standard Three-Act Structure:}
\begin{enumerate}
\item \textbf{Introduction}: Establish situation; introduce key NPCs and locations.
\item \textbf{Development}: 2--3 challenges that advance the main objective.
\item \textbf{Climax}: Major confrontation or resolution point.
\item \textbf{Resolution}: Consequences and setup for future adventures.
\end{enumerate}

\textbf{Alternative Structures:}
\begin{itemize}
\item \textbf{Hex Crawl}: Exploration-focused with multiple points of interest.
\item \textbf{Mystery}: Investigation-driven with clue accumulation.
\item \textbf{Siege}: Defense-focused with resource management.
\item \textbf{Journey}: Travel-based with episodic encounters.
\end{itemize}

\section{Troubleshooting Common Issues}
\label{sec:troubleshooting-ref}
\index{Appendices!Troubleshooting}

\subsection{Player Engagement Issues}
\label{subsec:engagement-issues}
\index{Appendices!Engagement Issues}

\textbf{Issue: Players are passive or hesitant}
\begin{itemize}
\item \textbf{Solutions}: Use leading questions, offer clear options, create immediate stakes.
\item Provide obvious hooks and direct incentives for action.
\item Use NPCs to demonstrate active approaches.
\item Reward proactive play with narrative advantages.
\end{itemize}

\textbf{Issue: Rules discussions slow the game}
\begin{itemize}
\item \textbf{Solutions}: Make quick rulings; note for later review; keep momentum.
\item Establish “ruling now, researching later” policy.
\item Designate one player as rules reference to minimize lookups.
\item Use standardized procedures for common actions.
\end{itemize}

\subsection{Balance and Challenge Issues}
\label{subsec:balance-issues}
\index{Appendices!Balance Issues}

\textbf{Issue: Encounters are too easy or too hard}
\begin{itemize}
\item \textbf{Solutions}: Adjust opposition on the fly; use SB to modulate difficulty.
\item Remember that Position and Effect can be adjusted situationally.
\item Use environmental factors to change challenge levels.
\item Allow creative solutions to bypass straight combat.
\end{itemize}

\textbf{Issue: Magic feels overpowered or underpowered}
\begin{itemize}
\item \textbf{Solutions}: Ensure proper Backlash and Obligation application.
\item Remember that high-DV spells carry significant risks.
\item Use countermagic and magical opposition when appropriate.
\item Ensure non-magical characters have meaningful contributions.
\end{itemize}

\subsection{Tracking and Administration Issues}
\label{subsec:tracking-issues}
\index{Appendices!Tracking Issues}

\textbf{Issue: Too much bookkeeping slows play}
\begin{itemize}
\item \textbf{Solutions}: Simplify tracking to essential elements; delegate to players.
\item Use abstract ranges and conditions rather than precise measurements.
\item Focus on narrative consequences rather than numerical modifiers.
\item Use index cards or digital tools for complex tracking.
\end{itemize}

\textbf{Issue: Players forget abilities or resources}
\begin{itemize}
\item \textbf{Solutions}: Provide quick reference sheets; use visual aids.
\item Create character-specific reminder cards.
\item Use recap sessions to review capabilities.
\item Encourage players to maintain updated character sheets.
\end{itemize}

\section{Advanced Play Techniques}
\label{sec:advanced-techniques}
\index{Appendices!Advanced Techniques}

\subsection{Narrative-First Adjudication}
\label{subsec:narrative-adjudication}
\index{Appendices!Narrative Adjudication}

\begin{itemize}
\item Ask “What happens next?” rather than “What’s the rule?”
\item \textbf{Use the Fiction}: Let the narrative dictate mechanical outcomes.
\item \textbf{Embrace Improvisation}: Create rulings that serve the story.
\item \textbf{Collaborative World-Building}: Involve players in creating details.
\item \textbf{Consequence-Driven Play}: Ensure every action has meaningful results.
\end{itemize}

\subsection{Pacing and Rhythm Management}
\label{subsec:pacing-management}
\index{Appendices!Pacing Management}

\textbf{Scene Pacing:}
\begin{itemize}
\item Vary intensity between high-action and quiet moments.
\item Use clocks to create natural endpoints.
\item Alternate between player-driven and GM-driven scenes.
\item Include breather moments for character development.
\end{itemize}

\textbf{Campaign Rhythm:}
\begin{itemize}
\item Balance episodic adventures with ongoing arcs.
\item Use downtime effectively between major events.
\item Vary the scope of challenges (personal, local, regional, global).
\item Include both planned and emergent story elements.
\end{itemize}

\subsection{Player Spotlight Management}
\label{subsec:spotlight-management}
\index{Appendices!Spotlight Management}

\begin{itemize}
\item \textbf{Rotate Focus}: Ensure each character gets meaningful scenes.
\item \textbf{Personal Arcs}: Develop individual character stories.
\item \textbf{Group Dynamics}: Create situations that require teamwork.
\item \textbf{Specialization Respect}: Allow experts to shine in their domains.
\item \textbf{Shared Moments}: Include scenes that develop group bonds.
\end{itemize}

