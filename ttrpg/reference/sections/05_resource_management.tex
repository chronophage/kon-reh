\chapter{Resource Management}
\label{chap:resource-management}
\index{Resources}\index{Resource Management}

\section{Introduction to Resource Management}
\label{sec:resource-intro}
\index{Resources!Overview}

\indexterm{Resource Management} in \textbf{Fate's Edge} focuses on tracking the tangible and intangible assets that enable character actions while creating narrative tension. Rather than micromanaging every item, the system uses abstract clocks and conditions that trigger when dramatically appropriate. This keeps the focus on story consequences rather than bookkeeping. \index{Clocks}\index{Conditions}

\section{Supply Clock}
\label{sec:supply-clock}
\index{Supply}\index{Resources!Supply Clock}

The \indexterm{Supply Clock} is a shared condition for the entire party, representing food, water, ammunition, and basic gear. It tracks the group's overall readiness for extended endeavors. \index{Party resources}

\subsection{Supply States}
\label{subsec:supply-states}
\index{Supply!States}\index{Clocks!Supply}

\begin{description}
\item[\indexterm{Full Supply} (0 segments filled)] The party is well-equipped and prepared. No penalties or complications. \index{Supply!Full}
\item[\indexterm{Low Supply} (2 segments filled)] Minor narrative complications occur: bland food, damaged arrows, thinning waterskins, or worn gear. \index{Supply!Low}
\item[\indexterm{Dangerously Low} (3 segments filled)] Each character gains \textbf{Fatigue 1} due to exhaustion and deprivation. \index{Supply!Dangerously Low}\index{Fatigue}
\item[\indexterm{Out of Supply} (4 segments filled)] Severe penalties apply; characters face starvation, dehydration, and failing gear with significant mechanical consequences. \index{Supply!Out of Supply}
\end{description}

\subsection{Filling the Supply Clock}
\label{subsec:filling-supply}
\index{Supply!Filling}\index{Story Beats (SB)}

The Supply Clock advances under the following circumstances:
\begin{itemize}
\item Harsh travel conditions or lost pack animals (GM fiat). \index{GM fiat}
\item GM spends \textbf{2+ Story Beats (SB)} to represent resource depletion. \index{Story Beats (SB)!Spends}
\item The party chooses to travel light for speed or stealth advantages. \index{Travel!Travel light}
\item Failed foraging or resource-gathering attempts. \index{Survival!Foraging}
\end{itemize}

\subsection{Emptying the Supply Clock}
\label{subsec:emptying-supply}
\index{Supply!Emptying}\index{Downtime}

The Supply Clock can be reduced through:
\begin{itemize}
\item Reaching civilization or safe haven resets to \textbf{Full Supply}. \index{Supply!Civilization}
\item Successful foraging/hunting: group \emph{Survival} check (DV~2) clears 1 segment. \index{Survival!Foraging}\index{Difficulty Value (DV)}
\item Downtime spent in relative safety removes 1 segment. \index{Downtime!Supply}
\item Purchasing or trading for supplies in settlements. \index{Supply!Purchasing}
\end{itemize}

\section{Fatigue}
\label{sec:fatigue}
\index{Fatigue}\index{Conditions!Fatigue}

\indexterm{Fatigue} represents physical exhaustion, hunger, emotional strain, and spiritual depletion. It accumulates through extended effort, deprivation, or magical backlash. \index{Backlash}

\subsection{Fatigue Effects}
\label{subsec:fatigue-effects}
\index{Fatigue!Effects}

\begin{description}
\item[\indexterm{1 Fatigue}] Minor drain. On your next significant roll, re-roll \emph{one} success (player's choice). \index{Significant actions}
\item[\indexterm{2 Fatigue}] Worn down. On each significant roll, re-roll one success.
\item[\indexterm{3 Fatigue}] Failing fast. On each significant roll, re-roll \emph{two} successes.
\item[\indexterm{4 Fatigue}] Collapse/KO/spiritual break. You fall out of the scene until treated or rescued.
\end{description}

\subsection{Clearing Fatigue}
\label{subsec:clearing-fatigue}
\index{Fatigue!Clearing}\index{Rest}

\begin{itemize}
\item A night's rest with adequate Supply removes 1 level of Fatigue.
\item You cannot clear Fatigue if the party's Supply clock is \emph{Dangerously Low} or \emph{Out of Supply}. \index{Supply}
\item Magical healing or special abilities may provide additional Fatigue recovery. \index{Magic!Healing}
\item Extended downtime (3+ days) in safety clears all Fatigue. \index{Downtime!Fatigue Recovery}
\end{itemize}

\section{Gear Condition}
\label{sec:gear-condition}
\index{Gear}\index{Resources!Gear Condition}

Gear in \textbf{Fate's Edge} does not have hit points or detailed durability tracking. Instead, equipment suffers consequences only when drama demands it. \index{Narrative-first}

\subsection{Compromised Items}
\label{subsec:compromised-gear}
\index{Gear!Compromised}\index{Story Beats (SB)}

\begin{itemize}
\item Compromised status is introduced via SB spends or narrative consequence. \index{Story Beats (SB)!Complications}
\item A \textbf{Compromised} item gives $-1$ die on relevant rolls until repaired. \index{Penalties}
\item Multiple Compromised items affecting the \emph{same} action do not stack penalties.
\end{itemize}

\subsection{Breaking Point}
\label{subsec:gear-breaking}
\index{Gear!Breaking}

If a Compromised item suffers another significant setback, it breaks entirely and becomes unusable.

\subsection{Repair Options}
\label{subsec:gear-repair}
\index{Gear!Repair}\index{Craft}

\begin{description}
\item[\indexterm{Field Repair}] Temporary fix; requires \emph{Craft} or \emph{Survival} check (DV~2) to remove the penalty for one scene. \index{Survival}\index{Difficulty Value (DV)}
\item[\indexterm{Proper Repair}] Permanent restoration; requires proper tools, materials, and significant downtime. \index{Downtime}
\item[\indexterm{Magical Repair}] Certain spells or rituals can instantly restore gear, but may carry Obligation or Backlash costs. \index{Obligation}\index{Backlash}
\end{description}

\subsection*{Equipment Enchantments Quick Reference}

\textbf{XP Costs:}
\begin{itemize}
\item Minor Enchantments: 2-4 XP
\item Major Enchantments: 6-8 XP
\item Prestige Enchantments: 10+ XP
\end{itemize}

\textbf{Limitations:}
\begin{itemize}
\item Max enchantments = Spirit attribute
\item No stacking identical benefits
\item Maintenance affects functionality
\end{itemize}

\textbf{Creation Guidelines:}
\begin{itemize}
\item Price like Talents with similar mechanical impact
\item Consider campaign power level
\item Balance benefit against XP cost
\end{itemize}

\section{Asset and Follower Management}
\label{sec:asset-management}
\index{Assets}\index{Followers}\index{Resources!Assets and Followers}

\subsection{Followers (On-Screen Resources)}
\label{subsec:followers}
\index{Followers!On-Screen}

\begin{description}
\item[\indexterm{Cost}] Cap$^{2}$ XP to acquire. \index{Followers!Cost}\index{XP!Followers}
\item[\indexterm{Assist Dice}] When applicable, the follower adds help dice equal to $\min(\text{Cap},\ \text{helper's relevant Skill})$, capped at \textbf{+3} dice total from all sources. \emph{Exception:} \textit{Exceptional Coordination} Talent allows one follower to provide \textbf{+4} assist dice. \index{Assists}\index{Talents!Exceptional Coordination}
\item[\indexterm{Capability}] Ranges from 1--5 (5 is exceptional). \index{Followers!Capability}
\item[\indexterm{Upkeep}] Each Downtime period, pay XP equal to Cap \emph{or} spend a Scene tending the relationship. \index{Downtime!Followers}\index{XP!Upkeep}
\item[\indexterm{Risk}] If the GM spends 2+ SB on an action you take with assistance, they may endanger, injure, or separate the follower instead of you if fictionally appropriate. \index{Story Beats (SB)!Follower risk}
\item[\indexterm{Off-Screen Capability}] Once per downtime, a Cap~5 follower can solve one significant problem but generates 1 SB for the party; the GM must describe how their action creates consequences. \index{Followers!Off-Screen}
\end{description}

% !TEX root = resource_guide_main.tex
% Resource Guide: Upkeep (Expanded)

\section{Upkeep (Expanded)}\label{sec:upkeep-expanded}
\index{Upkeep}\index{Downtime}\index{Followers}\index{Assets}\index{Wary}\index{Seized}\index{Neglected}\index{Compromised}

Upkeep is the story of attention. Followers and assets thrive when seen and sour when ignored. This section expands the SRD rule with narrative intent, concrete examples, edge cases, and GM guidance.

\subsection{Design Intent}\label{subsec:upkeep-intent}\index{Design Philosophy}
\begin{itemize}
\item \textbf{Choice with Teeth.} Players choose \emph{time} or \emph{XP}. Either way, the fiction moves: quick delegation vs.\ hands-on scenes.
\item \textbf{Visible Drift.} Missing upkeep degrades things predictably: \textsc{Wary}→\textsc{Seized}, \textsc{Neglected}→\textsc{Compromised}.
\item \textbf{Story First.} Every payment should say something about the relationship or the tool.
\end{itemize}

\subsection{Asset Upkeep}\label{subsec:upkeep-core}
\textbf{Frequency.} Once per Downtime period.\par
\textbf{Option A — Efficient (Higher XP, Less Time).} Pay Upkeep XP $= \max\big(1, \tfrac{\text{XP Acquisition Cost}}{3}\big)$. Minimal effort in-fiction (a retainer handles it; you check in by letter).\par
\textbf{Option B — Intensive (Lower XP, More Time).} Pay 1 XP. Spend a \emph{dedicated Downtime action} personally tending the follower/asset.

\textbf{Failure.} If unpaid this Downtime:
\begin{itemize}
\item \emph{Follower:} Becomes \textbf{Wary} (or \textbf{Seized} if already Wary).
\item \emph{Asset:} Becomes \textbf{Neglected} (or \textbf{Compromised} if already Neglected).
\end{itemize}

\subsection{What the Conditions Look Like}\label{subsec:upkeep-conditions}
\begin{description}[leftmargin=1.6em]
\item[\textsc{Wary} (Follower).] They hesitate, ask for guarantees, or interpret orders narrowly. \emph{Mechanical nudge:} first social/test involving them is --1 position \emph{or} costs 1 extra Stress.
\item[\textsc{Seized} (Follower).] Someone else sets terms (rival, guild, family) or they withdraw until appeased. \emph{Nudge:} cannot assist this Downtime; to re-engage, pay 1 Boon \emph{or} clear via a narrative scene that addresses their grievance.
\item[\textsc{Neglected} (Asset).] It underperforms; parts go missing; paperwork piles. \emph{Nudge:} --1 effect on its next use \emph{or} an \emph{Alarmed Attention} clock +1/2.
\item[\textsc{Compromised} (Asset).] Faults propagate; someone has leverage; the tool is noisy or unsafe. \emph{Nudge:} first use this session automatically creates 1 \textbf{SB} (formerly CP) \emph{or} requires a costly fix scene.
\end{description}

\subsection{Examples by Type}\label{subsec:upkeep-examples}
\paragraph{Followers.}
\begin{tabularx}{\linewidth}{>{\bfseries}l X X}
\toprule
Type & Efficient Upkeep (pay XP, quick) & Intensive Upkeep (1 XP + action) \\
\midrule
Scribe & Send coin and a sealed brief with instructions. & Host a lesson; co-author a pamphlet that sharpens their craft. \\
Scout & Dispatch a courier with a new map and stipend. & Walk the route together; overhaul signals and caches. \\
Acolyte (Patron-bound) & Arrange a blessing via intermediaries. & Lead them in a minor rite; narrate the teaching moment. \\
Mercenary & Wire hazard pay with a short commendation. & Drill formation; settle a dispute; toast the unit. \\
Informant & Drop a dead-letter and payment. & Meet face-to-face; share protection protocol. \\
\bottomrule
\end{tabularx}

\paragraph{Assets.}
\begin{tabularx}{\linewidth}{>{\bfseries}l X X}
\toprule
Type & Efficient Upkeep (pay XP, quick) & Intensive Upkeep (1 XP + action) \\
\midrule
Workshop & Pay the foreman; authorize routine parts. & Personally recalibrate; craft a sample piece. \\
Safehouse & Hire a caretaker; basic supplies delivered. & Repair locks/windows; rewrite the cover story with neighbors. \\
Library & Fund copying; buy indices. & Catalogue a shelf; cross-ref a case; bind a damaged tome. \\
Boat & Pay moorage and minimal maintenance. & Scrape hull; patch sails; sail a proving run. \\
Sigil Network (Invoker) & Replace two worn plates. & Etch a new master glyph; clear 1 wear mid-scene next session. \\
\bottomrule
\end{tabularx}

\subsection{Edge Cases \& Rulings}\label{subsec:upkeep-edge}
\begin{itemize}
\item \textbf{Multiple Resources.} Each follower/asset checks upkeep separately. A single Intensive action can cover a \emph{cohesive group} if fiction supports it (e.g., one drill for a squad).
\item \textbf{Remote Care.} If you narrate convincing remote oversight (sending a trusted lieutenant), count as Efficient.
\item \textbf{Patron-Tinted Upkeep.} Tie upkeep scenes to a Patron's theme for small boons (advantage on the next related roll) without changing costs.
\item \textbf{Stacking Misses.} Missing upkeep across \emph{two} consecutive Downtime periods moves \textsc{Wary}→\textsc{Seized} or \textsc{Neglected}→\textsc{Compromised}; a third consecutive miss risks loss (follower quits; asset condemned) at GM discretion.
\end{itemize}

\subsection{Optional Modules}\label{subsec:upkeep-modules}\index{Ledger Credits}\index{Union Rules}\index{Patron Favor Swap}
\paragraph{Ledger Credits.} Track fractional prep: three Efficient payments can be banked to waive one Intensive action later.
\paragraph{Union Rules.} Some factions demand Intensive upkeep; Efficient counts as half (two Efficient = one Intensive).
\paragraph{Patron Favor Swap.} Spend a \textbf{Boon} to treat missed upkeep as paid for one follower/asset if the scene honors the Patron.

\subsection{GM Guidance}\label{subsec:upkeep-gm}
\begin{itemize}
\item \textbf{Making it a Scene.} Intensive upkeep should reveal something (new contact, flaw, rumor).
\item \textbf{Foreshadowing Degradation.} Before applying \textsc{Seized}/\textsc{Compromised}, show warning signs the players can act on.
\item \textbf{Naming the Toll.} If a \textbf{Patron's Largess} (wrath) applies, state the narrative reason and the extra toll (e.g., an added SB, a temporary lockout, or a vow demanded).
\end{itemize}

\begin{tcolorbox}[title={Quick Reference},colback=gray!5,colframe=black]
\textbf{Once per Downtime.} Pay Efficient (XP = $\max(1, \tfrac{\text{acq}}{3})$) or Intensive (1 XP + action). \textbf{Missed:} Follower $\to$ \textsc{Wary}→\textsc{Seized}; Asset $\to$ \textsc{Neglected}→\textsc{Compromised}. Tie scenes to your Patron for flavor, not discounts.
\end{tcolorbox}

\subsection{Follower Assist Rules}
\label{subsec:follower-assist}
\index{Followers!Assist}\index{Assists!Rules}

\begin{itemize}
    \item Assist dice come from the helper's capabilities, not the leader's. \index{Assists!Source}
    \item Total Assist on any roll (from any sources) is hard-capped at \textbf{+3}. \emph{Exception:} \textit{Exceptional Coordination} may allow \textbf{+4} from a single follower. \index{Assists!Maximum}
    \item Only one follower may assist a given action at a time. \index{Followers!Slot Limit}
    \item Followers cannot assist actions beyond their narrative scope or capabilities.
\end{itemize}

\subsection{Loyalty \& Bonds (Optional Rules)}
\label{subsec:follower-loyalty}
\index{Followers!Loyalty}\index{Bonds}

\begin{itemize}
    \item Track a simple Loyalty tag per follower: \emph{Wary / Steady / Devoted}. \index{Followers!Loyalty Tags}
    \item \emph{Devoted} followers can once per arc convert one GM Complication targeting them into a lesser setback. \index{Complications}
    \item \emph{Wary} followers cost +1 XP to maintain during Downtime. \index{XP!Upkeep}
    \item Loyalty can change based on how the PC treats the follower and shared experiences.
\end{itemize}

\subsection{Stress, Harm, \& Loss (GM Tools)}
\label{subsec:follower-stress}
\index{Followers!Stress}\index{GM tools}

\begin{description}
    \item[\indexterm{Pin}] The follower is separated/boxed out of the current action. \index{Followers!Pin}
    \item[\indexterm{Wound}] The follower is Injured: until treated off-screen, their effective Cap counts as 1 lower. \index{Followers!Wound}
    \item[\indexterm{Burn}] Mark the follower as \emph{Neglected} immediately. \index{Followers!Burn}\index{Assets!Neglected}
    \item[\indexterm{Seize}] Escalate to \emph{Compromised} status. \index{Followers!Seize}\index{Assets!Compromised}
    \item[\indexterm{PC Choice Lever}] The GM should offer the player a meaningful choice about follower risk. \index{Player choice}
\end{description}

\subsection{Off-Screen Assets}
\label{subsec:off-screen-assets}
\index{Assets!Off-Screen}

\begin{description}
\item[\indexterm{Minor Asset} (4 XP)] Safehouse, small charter, local business. \index{Assets!Minor}
\item[\indexterm{Standard Asset} (8 XP)] Noble title, guild section, spy ring, significant property. \index{Assets!Standard}
\item[\indexterm{Major Asset} (12 XP)] City license, regional network, major institution influence. \index{Assets!Major}
\item[\indexterm{Artifact Asset} (16+ XP)] Unique items or positions with campaign-level significance. \index{Artifacts}
\end{description}

\subsection{Asset Activations}
\label{subsec:asset-activations}
\index{Assets!Activations}\index{Boons}

\begin{itemize}
    \item \textbf{Off-Screen Activation:} At campaign start or during Downtime, activate an off-screen asset by spending \textbf{1 Boon} or \textbf{2 XP}. \index{Assets!Activation Cost}\index{XP!Assets}
    \item \textbf{Off-Screen Effects:} Use each Asset's listed off-screen effect once per session for free. \index{Assets!Off-Screen Effects}
    \item \textbf{On-Screen Activation:} To reshape the current scene, spend \textbf{1 Boon}. \index{Assets!On-Screen Effects}
    \item \textbf{Plausibility Test:} The Asset must have appropriate scope and reach for the intended effect. \index{Assets!Plausibility}
\end{itemize}

\subsection{Asset Condition Tracks}
\label{subsec:asset-conditions}
\index{Assets!Condition Tracks}

\begin{description}
\item[\indexterm{Maintained}] Full capability; no penalties. The asset is in good standing and fully functional. \index{Assets!Maintained}
\item[\indexterm{Neglected}] $-1$ die when used (assist or leverage). Narratively: slower response, sullen staff, short-staffed operations. \index{Assets!Neglected}
\item[\indexterm{Compromised}] Unavailable for use. Narratively: captured, burned, seized, defected, or otherwise incapacitated. \index{Assets!Compromised}
\end{description}

\subsection{Maintenance and Repair}
\label{subsec:asset-maintenance}
\index{Assets!Maintenance}\index{Downtime}

\begin{itemize}
\item After a story arc or 2--3 sessions of heavy use, the GM may flag the resource as \emph{at risk}. \index{Assets!Risk Flag}
\item To keep an asset \emph{Maintained}, the player must either:
  \begin{itemize}
  \item Spend Downtime repairing/servicing it, or \index{Assets!Downtime Repair}
  \item Pay XP immediately (representing replacement parts, masterwork upkeep). \index{Assets!XP Repair}
  \end{itemize}
\item If neglected, the asset becomes \emph{Neglected} ($-1$ die) and may eventually become permanently lost. \index{Assets!Breakage}
\item Superior and Artifact assets do not require normal upkeep; if \emph{Compromised} through complications, only narrative quests can repair them. \index{Artifacts}\index{Quests}
\end{itemize}

\section{Bond-Driven Resource Generation}
\label{sec:bond-resources}
\index{Bonds}\index{Boons}\index{Resources!Bond-Driven Generation}

When a player takes a significant action to aid an ally with whom they share a bond, and explicitly references that bond in an \emph{Intricate} description, they may mark that bond to gain \textbf{1 Boon} \emph{after} the action resolves. \index{Description Ladder!Intricate Action}

\subsection{Requirements}
\label{subsec:bond-requirements}
\index{Bonds!Requirements}

\textbf{Requirements for Bond-Driven Boon Generation:}
\begin{itemize}
    \item \textbf{Mutual Bond:} Player shares a defined bond with the ally they're aiding. \index{Bonds!Mutual Requirement}
    \item \textbf{Intricate Description:} Player describes how the bond motivates their action using rich, multi-sensory details. \index{Description Ladder!Intricate Action}
    \item \textbf{Significant Aid:} Meaningful assistance beyond basic dice bonuses. \index{Teamwork}\index{Assists}
    \item \textbf{Fiction First:} The bond genuinely drives the choice to help, not added retroactively. \index{Fiction-first}
\end{itemize}

\subsection{Examples}
\label{subsec:bond-examples}
\index{Bonds!Examples}

\textbf{Valid Examples:}
\begin{itemize}
    \item ``Remembering how they saved me from the falling rubble in Aeler, I throw myself in front of the crossbow bolt meant for them!''
    \item ``Thinking of our shared vow to protect the innocent, I use my last healing potion to stabilize them instead of saving it for myself.''
    \item ``Drawing on our years fighting side-by-side in the Border Wars, I rally the other mercenaries to keep fighting alongside them when morale fails.''
\end{itemize}

\subsection{Limitations}
\label{subsec:bond-limitations}
\index{Bonds!Limits}

\textbf{Restrictions on Bond-Driven Generation:}
\begin{itemize}
    \item Once per bond per session. \index{Bonds!Session Limit}
    \item Must involve meaningful sacrifice or risk. \index{Risk}
    \item GM approval required for what constitutes ``significant action.'' \index{GM discretion}
    \item Cannot be used for basic assistance rolls or minor favors. \index{Description Ladder!Basic Action}
    \item The Boon is awarded \emph{after} the action resolves, not before. \index{Boons!Timing}
\end{itemize}

\section{Over-Stack Rule}
\label{sec:over-stack}
\index{Combat!Over-Stack}\index{Advantages}

The Over-Stack rule prevents excessive accumulation of advantages from trivial sources while rewarding meaningful preparation.

\subsection{Structural Advantages}
\label{subsec:structural-advantages}
\index{Combat!Structural Advantages}

Structural advantages include:
\begin{itemize}
\item Active buffs or beneficial tags affecting the party. \index{Tags}
\item Favorable venue or environmental factors. \index{Environment}
\item Unused \emph{Follower Initiative} for the scene. \index{Followers!Initiative}
\item On-screen Asset activation providing immediate benefits. \index{Assets!On-Screen Effects}
\item Opponent disadvantaged by fiction (surprised, trapped, etc.). \index{Position!Surprise}
\item Ritual preparation that applies to the current situation. \index{Rituals!Preparation}
\end{itemize}

\subsection{Over-Stack Trigger}
\label{subsec:overstack-trigger}
\index{Combat!Over-Stack Trigger}\index{Story Beats (SB)}

\begin{itemize}
\item If the party enters a scene with $\geq 3$ structural advantages, apply Over-Stack once for that scene.
\item The GM chooses either:
  \begin{itemize}
  \item Start one named obstacle or challenge at \textbf{+1 DV}, or \index{Difficulty Value (DV)}
  \item The GM banks \textbf{+1 SB} for the first Deck Twist in the scene. \index{Deck of Consequences!Twists}
  \end{itemize}
\item This represents narrative pushback when characters have overwhelming advantages.
\item Over-Stack applies only once per scene, regardless of how many advantages accumulate.
\end{itemize}
