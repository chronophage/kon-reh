--------------
\pagestyle{fancy}
\fancyhf{}
\lhead{\textbf{Lord Vyr, the Crimson Regent}}
\rhead{Fate’s Edge – Villain Module}
\cfoot{\thepage}

% ---------------------------------------------------------
%   Colours & Commands
% ---------------------------------------------------------
\definecolor{myblue}{HTML}{003366}
\definecolor{mygray}{gray}{0.85}

\newcommand{\stat}[2]{%
  \noindent\textbf{#1}\hfill\textbf{#2}\par
}
\newcommand{\attribute}[2]{%
  \noindent\textbf{#1}\quad\textit{#2}\par
}
\newcommand{\skill}[3]{%
  \noindent\textbf{#1}\qquad\textit{#2}\quad\textbf{#3}\par
}
\newcommand{\ability}[2]{%
  \noindent\textbf{#1}\hfill\textit{#2}\par
}
\newcommand{\sectiontitle}[1]{%
  \vspace{6pt}
  \noindent\colorbox{myblue}{\parbox{\dimexpr\linewidth-2\fb%%%%%%%%%%%%%%%%%%%%%%%%%%%%%%%%%%%%%%%%%%%%%%%%%%%%%%%%%%%%
%  NPC / Villain Module – LaTeX Template
%  (Adapted for the Fate’s Edge SRD, but easily re‑used)
%
%  To compile:
%      pdflatex npc-villain.tex
%
%  Author:  Your Name
%  License: CC‑BY‑SA 4.0
%%%%%%%%%%%%%%%%%%%%%%%%%%%%%%%%%%%%%%%%%%%%%%%%%%%%%%%%%%%%

\documentclass[12pt,a4paper]{article}
\usepackage[margin=1in]{geometry}
\usepackage{array}
\usepackage{booktabs}
\usepackage{longtable}
\usepackage{enumitem}
\usepackage{xcolor}
\usepackage{hyperref}
\usepackage{graphicx}
\usepackage{calc}
\usepackage{lmodern}
\usepackage{tikz}
\usepackage{fancyhdr}
\usepackage{pgffor}

% ---------------------------------------------------------
%   Page layout
% -------------------------------------------oxsep}{%
    \centering\large\color{white}\textbf{#1}}}%
  \vspace{4pt}
}

% ---------------------------------------------------------
%   Document
% ---------------------------------------------------------
\begin{document}

\begin{center}
    {\LARGE \textbf{Lord Vyr, the Crimson Regent}}\\[4pt]
    {\large Human Noble, Tier III Villain (CR 5)}\\[2pt]
    \rule{\linewidth}{0.5pt}
\end{center}

% ---------------------------------------------------------
%   1. Overview
% ---------------------------------------------------------
\sectiontitle{1. Overview}
Lord Vyr is the iron‑fisted ruler of the border duchy of Ardentia.  He rose to power
through a mixture of political machination, dark patron‑binding rites, and
unparalleled martial skill.  Vyr now seeks the legendary **Emerald Lantern** to
fuel a forbidden ritual that will bind a spirit of the **Shadow Court** to his
will, granting him control over the entire northern frontier.

% ---------------------------------------------------------
%   2. Appearance & Personality
% ---------------------------------------------------------
\sectiontitle{2. Appearance \& Personality}
\begin{itemize}[leftmargin=*]
    \item \textbf{Appearance:} Tall, scar‑marked, always dressed in a blood‑red
          doublet trimmed with silver.  A black sigil of an eclipsed sun is
          embroidered on his cape – the mark of his patron, Ikasha.
    \item \textbf{Mannerisms:} Speaks in measured, measured tones;
          rarely raises his voice but his stare is intimidating.
    \item \textbf{Personality:} Calculating, ruthless, yet holds a twisted
          sense of honor.  He respects worthy opponents and will keep his
          word—unless it threatens his ultimate goal.
    \item \textbf{Motivation:} To secure the Emerald Lantern,
          complete the \emph{Shadow‑Crown Rite}, and become the
          unchallenged master of the north.
\end{itemize}

% ---------------------------------------------------------
%   3. Attributes & Skills (Fate’s Edge)
% ---------------------------------------------------------
\sectiontitle{3. Attributes \& Skills}
\noindent\textbf{Attributes (1–5)}\\[2pt]
\begin{tabular}{>{\bfseries}l>{\raggedright}p{6cm}}
Body    & 4 – Physically imposing, excellent endurance.\\
Wits    & 5 – Master tactician, quick thinker.\\
Spirit  & 4 – Strongly bound to his patron, high willpower.\\
Presence& 3 – Charismatic, but his aura is tinged with menace.\\
\end{tabular}

\vspace{6pt}
\noindent\textbf{Key Skills (0–5)}\\[2pt]
\begin{tabular}{>{\bfseries}l>{\raggedright}p{6cm}}
Melee       & 5 – Expert swordsman (longsword +2 dice).\\
Ranged      & 2 – Occasional crossbow use.\\
Command     & 4 – Leads troops, can rally allies.\\
Arcana      & 3 – Adept at binding rites.\\
Subterfuge  & 2 – Rarely hides his intentions.\\
\end{tabular}

% ---------------------------------------------------------
%   4. Notable Talents & Abilities
% ---------------------------------------------------------
\sectiontitle{4. Talents \& Abilities}
\begin{enumerate}[label=\Alph*. , leftmargin=*]
    \item \textbf{Crown‑Blade (Major, 8 XP)} \\
          \textit{When attacking a single target in Dominant Position, Vyr may
          add +2 dice and treat any 10 as a \emph{Legendary} success
          (triple Harm).}
    \item \textbf{Patron’s Gift – Shadow Veil (Minor, 2 XP)} \\
          \textit{Once per scene, Vyr may spend 1 Boon to gain \textbf{+1 die}
          on any Stealth or Shadow‑type roll.}
    \item \textbf{Obligation Mastery (Prestige, 12 XP)} \\
          \textit{Every time Vyr completes a Rite, the Obligation cost is
          reduced by 1 (minimum 1).}
    \item \textbf{Rite of the Eclipsed Sun (Ritual, DV = 5)} \\
          \textit{A high‑power rite that, when completed, binds a
          Shadow‑Court spirit to the caster for 1 scene.  Cost: +2 Obligation,
          automatic \emph{Backlash} (Major, Fire/Luck).}
\end{enumerate}

% ---------------------------------------------------------
%   5. Equipment
% ---------------------------------------------------------
\sectiontitle{5. Equipment}
\begin{itemize}[leftmargin=*]
    \item \textbf{Crimson Greatsword} (Melee Weapon, Weight \emph{Medium})\\
          +2 dice, \emph{[BANE]} tag (deals +1 Harm to armored foes).
    \item \textbf{Sigil of Ikasha} (Patron’s Symbol, Minor Asset)\\
          Allows casting of Ikasha‑aligned rites without extra Obligation.
    \item \textbf{Plate of the Red Dawn} (Heavy Armor)\\
          Grants \emph{[WARD]} against fire; counts as \emph{Dominant}
          Position when standing inside a structure.
    \item \textbf{Enchanted Dagger (Family Heirloom)}\\
          +1 die to melee, can be used for a quick \emph{Backstab}
          when attacking from \emph{Stealth}.
\end{itemize}

% ---------------------------------------------------------
%   6. Tactics
% ---------------------------------------------------------
\sectiontitle{6. Combat Tactics}
\begin{enumerate}[label=\arabic*., leftmargin=*]
    \item \textbf{Opening Move – Position Control}\\
          Vyr begins in \emph{Dominant} Position behind his plate,
          using \textbf{Patron’s Gift – Shadow Veil} to stay concealed
          while his minions (the \emph{Red Guard}) engage the party.
    \item \textbf{Crown‑Blade Assault}\\
          When an opponent reaches \emph{Controlled} or \emph{Desperate}
          Position, Vyr spends a Boon to gain the \textbf{Crown‑Blade}
          extra dice and attempts a \emph{Legendary} strike.
    \item \textbf{Rite Activation}\\
          If the fight stalls (more than 3 exchanges), Vyr may begin the
          \emph{Rite of the Eclipsed Sun}.  He spends 1 Boon to start the
          ritual (Channel phase).  While chanting, he uses \emph{Assist}
          from his guards to protect the ritual.  Failure on the \emph{Weave}
          roll causes \textbf{Backlash} (fire‑smoke, +1 Harm to Vyr).
    \item \textbf{Retreat or Finish}\\
          Should the party gain \emph{Dominant} Position and threaten his
          life, Vyr may expend a Boon to \emph{Escape} (Disengage with
          +1 Position) and flee to a hidden passage (pre‑placed on the map).
\end{enumerate}

% ---------------------------------------------------------
%   7. Narrative Hooks & Plot Seeds
% ---------------------------------------------------------
\sectiontitle{7. Narrative Hooks \& Plot Seeds}
\begin{enumerate}[label=\alph*., leftmargin=*]
    \item \textbf{The Lantern’s Light}\\
          Rumours spread that the \emph{Emerald Lantern} was stolen
          from the city vault.  The party is hired by the city’s magistrate
          to retrieve it before Vyr can complete his rite.
    \item \textbf{A Broken Pact}\\
          A former disciple of Vyr’s patron, a low‑level \emph{Cantor},
          seeks aid to undo Vyr’s binding.  He offers crucial information
          about the sigil’s weakness in exchange for protection.
    \item \textbf{The Red Guard’s Betrayal}\\
          Vyr’s loyal guard captain, Saren, has grown uneasy with the
          Shadow‑Court pact.  The party can approach Saren to turn
          a portion of the guard against Vyr—providing a dramatic
          midway‑scene twist.
    \item \textbf{Ritual Interruption}\\
          The rite requires a **five‑second Position** (Dominant) for
          completion.  If the party can force Vyr into \emph{Desperate}
          Position during the \emph{Weave} phase, the rite fizzles and
          his Obligation resets.
\end{enumerate}

% ---------------------------------------------------------
%   8. Advancement for the Villain
% ---------------------------------------------------------
\sectiontitle{8. Advancement \& Scaling}
\begin{itemize}[leftmargin=*]
    \item \textbf{Tier Increase:} Add +1 to Body and Wits, give \textbf{Berserker Rage}
          (Major, 8 XP) for Tier IV.
    \item \textbf{Additional Rites:} Grant a \textbf{Rite of Binding} (DV = 4) that
          can temporarily paralyze a single PC for one exchange.
    \item \textbf{Minion Upgrade:} Replace standard Red Guard with
          \textbf{Shadow‑bound Enforcers} (gain +1 die, \emph{[SHADOW]} tag).
\end{itemize}

% ---------------------------------------------------------
%   9. Quick Reference Card (optional)
% ---------------------------------------------------------
\newpage
\sectiontitle{9. Quick Reference Card}
\begin{center}
\begin{tikzpicture}[every node/.style={inner sep=4pt,outer sep=0pt}]
\node[draw,rounded corners,fill=mygray,minimum width=\linewidth,minimum height=0.6cm]
    (title){\large\bfseries Lord Vyr, the Crimson Regent};

\node[below=0.2cm of title,align=left]{
\begin{tabular}{ll}
\textbf{Tier:}& III (CR 5)\\
\textbf{Body:}& 4 \quad \textbf{Wits:}& 5\\
\textbf{Spirit:}& 4 \quad \textbf{Presence:}& 3\\
\textbf{Melee:}& 5 \quad \textbf{Command:}& 4\\
\textbf{Arcana:}& 3 \quad \textbf{Ranged:}& 2
\end{tabular}
};

\node[below=0.6cm of title,align=left]{
\begin{tabular}{p{0.45\linewidth}p{0.45\linewidth}}
\textbf{Talents}\\
\textbullet\ Crown‑Blade (Major) & \textbullet\ Patron’s Gift – Shadow Veil (Minor)\\
\textbullet\ Obligation Mastery (Prestige) &
\textbullet\ Rite of the Eclipsed Sun (Ritual)\\[6pt]
\textbf{Gear}\\
\textbullet\ Crimson Greatsword &
\textbullet\ Sigil of Ikasha (Asset)\\
\textbullet\ Plate of the Red Dawn &
\textbullet\ Enchanted Dagger (Heirloom)\\
\end{tabular}
};

\node[below=0.8cm of title,align=left]{
\textbf{Tactics}\\
\begin{itemize}[leftmargin=*,label=--,nosep]
    \item Open in Dominant Position, use Shadow Veil.
    \item Crown‑Blade for Legendary strikes.
    \item Begin the Eclipsed Sun rite when combat stalls.
    \item Retreat with a Boon if out‑matched.
\end{itemize}
};

\end{tikzpicture}
\end{center}

% ---------------------------------------------------------
%   End of Document
% ---------------------------------------------------------
\end{document}

