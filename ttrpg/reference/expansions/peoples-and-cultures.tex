\documentclass[11pt]{book}
\usepackage[margin=1in]{geometry}
\usepackage{fontenc}
\usepackage[T1]{fontenc}
\usepackage{lmodern}
\usepackage{enumitem}
\usepackage{titlesec}
\usepackage{sectsty}
\usepackage{xcolor}
\usepackage{hyperref}
\usepackage{graphicx}
\usepackage{array}
\usepackage{booktabs}
\usepackage{tabularx}
\usepackage{multicol}

\hypersetup{
    colorlinks=true,
    linkcolor=blue,
    filecolor=magenta,      
    urlcolor=cyan,
    pdftitle={Aeler Module for Fate's Edge},
    pdfauthor={AI Assistant}
}

\sectionfont{\raggedright\large\bfseries}
\subsectionfont{\raggedright\normalsize\bfseries}
\subsubsectionfont{\raggedright\normalsize\itshape}

\setlist{leftmargin=*}
\setlist[itemize]{itemsep=0pt}
\setlist[enumerate]{itemsep=0pt}

\newcommand{\talent}[1]{\textbf{#1}}
\newcommand{\stringtype}[1]{\textbf{#1}}
\newcommand{\complicationtype}[1]{\textbf{#1}}

\title{Fate's Edge: Peoples and Culturs}
\author{}
\date{}

\begin{document}

\maketitle

\chapter{The Aeler}

\section*{Core Concept}
The Aeler are masters of infrastructure, economics, and subtle influence. Their power lies not in flashy magic or brute force, but in their ability to make the world work - and to make others dependent on that work.

\section{Aeler — Stone, Breath, and the Ledger}

\subsection*{Mount-born engineers, keepers of underways, and masters of soft power}

\begin{quote}
``Hold what bears weight. Record what breath costs. Promise only what stone will keep.''
\end{quote}

\medskip\hrule\medskip

\subsection{I. Stone \& Breath}

The Aeler resemble their mountains: layered, load-bearing, slow to move, impossible to unmove once set. In the high Aelerian ranges, every life is counted in \emph{breaths, heat, light, and grain}. Underground, a lantern’s hue is a balance sheet; a bell-note means more than a shout. They call this discipline \textbf{deep accounting}: air tallies hung by vent-shafts, lamp-time written in chalk, calories measured in the language of ovens.

Above the tree line, stone yields little. Aeler communities rely on southern granaries to swell their numbers; in return, they build the works that make those granaries stand. To outsiders, Aeler influence feels like gravity: not a spear at the gate, but the gate itself, the levee above it, and the price of the mill-stone.

\medskip\hrule\medskip

\subsection{II. Grounds \& Holdings}

\subsubsection*{The Aelerian Range}
Alpine holds cling to ribs of granite. \emph{Vent-stacks} thrum in winter like organ pipes; ropeways hum with ore and kiln-brick. Pass-keeps mark every climb with \emph{keystone tablets}—names of those who swear to keep the road.

\subsubsection*{Underways}
A second country. Galleries where a cough wastes air; shafts where heat pools like a predator. The Aeler teach novices the \emph{Nine Measures}: light, draft, echo, dust, taste of iron, sweat-chill, lamp-shadow, bell-lag, and head-ache. Fail two, turn back. Fail three, seal the door.

\subsubsection*{The Mistlands (Protectorate)}
North of the range lies a pale country of low sun and breathing fog. Here, \textbf{Mist Wards}—slender towers with vane-bells and salt gutters—keep arable bands clear. Human farmsteads work barley, oats, and hardy roots under Aeler \emph{grain sureties}. The bargain: seed, levees, and warding for tithed surplus and winter labor on the towers. Some call it stewardship; others say the ledgers lean hard. Both are true in different winters.

\subsubsection*{Southern Commonweals}
South of the range, cities—Ecktoria, Acasia, Vhasia, Viterra, Thepyrgos, Ubral—wear Aeler work like hidden bones: bridges that never fall, floodgates that answer one key, ovens that feed thousands, mint dies that refuse to blur. Contracts are the quiet battlefield; concessions the spoils. A pledge of stone is better than a regiment that eats.

\medskip\hrule\medskip

\subsection{III. Threats from Below}

\textbf{Deep Drakes} are not simply big lizards. They are pressures that think: heat and hunger with a will, minds that push at the back of your teeth. The Aeler call the phenomenon \textbf{stone-press}. Symptoms begin as number-slips and end as people walking into shafts “to make the sums right.”

\begin{itemize}[leftmargin=*]
  \item \textbf{Signs:} condensation runs \emph{uphill}; lamp-blue goes \emph{flat}; bell-ring returns with a late second echo.
  \item \textbf{Defenses:} \emph{null-bells} tuned to kill harmonics; \emph{vein-salt} lines across thresholds; \emph{cold writs} (etched silver plates) worn over the heart; teams that count breaths in rotation and swap counters mid-phrase to catch mind-drift.
  \item \textbf{Doctrine:} never descend alone; never let a single mind do the ledgers; never meet a Drake’s gaze in still air.
\end{itemize}

\textit{In play (fronts):} \textbf{Stone-Press [6]}, \textbf{Miasma Spread [4]}, \textbf{Vent Failure [4]}. When \emph{Stone-Press} fills, escalate \emph{hallucinated ledgers} and \emph{mis-signed keystones}; treat social scenes as if the room itself is an opponent.

\medskip\hrule\medskip

\subsection{IV. Law, Ovens, and Soft Power}

Aeler law is \textbf{Tally-Law}: if it isn’t written, it isn’t owed; if it cannot bear weight, it isn’t promised. Their soft power rides four pillars:

\begin{itemize}[leftmargin=*]
  \item \textbf{Keystone Rights.} Control the piece that holds the whole. Aeler sign \emph{keystone clauses}: maintenance for access, repair for rate. Break the clause, and the bridge “politely rests.”
  \item \textbf{Grain Surety.} Winter ovens and storage domes. In exchange for tithe windows and public pricing courts, cities adopt \emph{Oven Charter Days}—no one starves while the charter bell rings.
  \item \textbf{Water \& Flood.} Sluice math is power. Aeler sluicewarden guilds lease keys, not walls; they can open a city for trade faster than any army can take it.
  \item \textbf{Mint \& Measure.} Calibrated weights, indelible dies. When a coin bears Aeler marks, courts sharpen.
\end{itemize}

\textit{In play (Strings):} \emph{keystone right}, \emph{grain surety}, \emph{sluice key}, \emph{mint die}, \emph{oven charter}. Cash a String to shift a negotiation’s DV by --1 or to force a \emph{public bowl} (fairness hearing) before a blade is drawn.

\medskip\hrule\medskip

\subsection{V. People \& Institutions}

\begin{itemize}[leftmargin=*]
  \item \textbf{Bell-Chains.} Signal corps that read echo and carry news down-lines faster than runners. Their captaincy exams are held in the dark.
  \item \textbf{Ledger-Kin.} Families who hold communal contracts and assign \emph{scrips of air, heat, and lamp}. Marriages are audited as carefully as bridges.
  \item \textbf{Vent Priors.} Engineers who treat airflow like liturgy. Their vestments are leather aprons burned in a pattern that maps the hold.
  \item \textbf{Oven-Wardens.} Bakers as quartermasters; saints of bread and ration-line. In famine, they outrank colonels.
\end{itemize}

\medskip\hrule\medskip

\subsection{VI. Aesthetics \& Speech}

\begin{itemize}[leftmargin=*]
  \item \textbf{Colors:} soot, iron, kiln-orange, glacial blue.
  \item \textbf{Materials:} rough-split stone outside; mirror-polished keystones within; leather scored with counting notches; vellum that smells of lye and smoke.
  \item \textbf{Proverbs:} 
  \begin{itemize}
    \item Stone keeps the promises you place on the right spot.
    \item Count in the light, breathe in the turn.
    \item A bridge is a treaty that learned to carry weight.
    \item Heat is a loan; pay it back with bread.
  \end{itemize}
  \item \textbf{Names:} given + craft + keep: \emph{Orra Vent-Prior of Third Stack}; \emph{Ghent Bell-Captain of Red Echo}. Silence is punctuation; bells are syntax.
\end{itemize}

\medskip\hrule\medskip

\subsection{VII. Relations \& Rivalries}

\begin{itemize}[leftmargin=*]
  \item \textbf{Ykrul.} Mutual respect where work meets weather. Rope meets ring; pass-oaths meet meadow concessions. Arguments end on the Board with a hand on the stone.
  \item \textbf{Linns.} Sea people who love a keystone you can sail; Aeler admire their hull-geometry, distrust their ledgers.
  \item \textbf{Lethai-al.} Tree-law and stone-law debate. Root and ridge agree more often than songs admit.
  \item \textbf{Deep Drakes.} The foe that teaches humility. Victories are written in sealed doors and names of the lost.
\end{itemize}

\medskip\hrule\medskip

\subsection{VIII. Strings \& Tools (at the table)}

\begin{itemize}[leftmargin=*]
  \item \textbf{Keystone Tablet.} Establishes or pauses a route; once/scene, convert a chase into a stand where \emph{Position +1} for defenders who prepared.
  \item \textbf{Null-Bell.} On ring, cancel one \emph{psychic push} or compel; costs \emph{lamp-time} (mark gear wear or Fatigue).
  \item \textbf{Oven Charter Seal.} Force a \emph{public bowl} in markets: one round of fair-price negotiation before violence can escalate.
  \item \textbf{Sluice Key.} DV --1 on any operation hinging on water, flood, or sanitation; if abused, create \emph{Public Outrage [4]}.
  \item \textbf{Air Scrip.} Negate a first underground suffocation/miasma consequence in a leg; on use, tick \emph{Vent Failure [1]}.
\end{itemize}

\medskip\hrule\medskip

\subsection{IX. Adventure Seeds}

\begin{enumerate}[leftmargin=*]
  \item \textbf{The Bell That Rang Twice.} A late echo signals stone-press in the east galleries. Audit the vent ledgers, find the mind-heat, and seal what thinks in the dark.
  \item \textbf{Bread Before Swords.} Southern grain is short; an Oven Charter is contested by syndics with sharp friends. Win the bowl, post the rate, keep the line fed.
  \item \textbf{Keystone Courtesy.} A bridge fails “politely” on the caravan road. Someone altered the tablet. Follow the chisels to a quiet war over a toll.
  \item \textbf{Mist Ward Winter.} Warding bells are silent and the fog eats fences. Repair the tower with farm help who distrust the ledger—and prove the sums keep them alive.
  \item \textbf{Drake’s Ledger.} A Deep Drake hoards \emph{numbers}: stolen tallies, miscounted breaths, lost names. Bring back the books and the people they keep.
\end{enumerate}

\medskip\hrule\medskip

\subsection{X. Portraying the Aeler}

Lead with craft and capacity. Show how a promise is a structure, not a sentence. Underground, make \emph{air and light} into currencies; overhead, let contracts move armies without banners. When the deep presses, slow your words, count your breaths, and let the bell decide who speaks next.

\medskip\hrule\medskip

\subsection{XI. Sumwrights \& Ledger-Compacts (Aelinnel Allies)}

Aeler holds commonly retain \textbf{Aelinnel Sumwrights} as neutral keepers of ledgers. Copper-stylus and split-book in hand, they maintain the \emph{two-ledger doctrine} (said/meant) and audit ovens, sluices, and mints. Sumwright contracts insist on:
\begin{itemize}[leftmargin=*]
  \item \textbf{Public Weights.} Scales, measures, and dies are tested at market-bell; falsifiers lose \emph{Display Rights} (see XII) for a season.
  \item \textbf{Even Witness.} A Sumwright must read accounts to both parties aloud; silence equals assent for that line only.
  \item \textbf{Split Custody.} One ledger remains in the hold; one travels with the Sumwright caravan; discrepancies ring the \emph{Red Bell}.
\end{itemize}
\textit{Strings:} \emph{sumwright retainer}, \emph{public weight right}, \emph{red-bell audit}. Cash to reduce a negotiation DV by --1 or force a fairness hearing before a blade is drawn.

\medskip\hrule\medskip

\subsection{XII. Commerce, Status, and Display}

Aeler society is \textbf{collectivist and mercantile}: wealth flows first to commonworks (ovens, wards, roads, keystones), then returns as \textbf{Display Rights} that signal status. The more you contribute, the more you may wear, carve, or light.
\begin{itemize}[leftmargin=*]
  \item \textbf{Display Charter.} Public marks (metal trim, lamp-halos, keystone etching) are licensed by the ledger-kin. Fraudulent display is a civic shame and fined in grain or labor.
  \item \textbf{Rings of Account.} Narrow bands on tools and belts show contract classes kept: water, bread, bridge, mint. Three rings grant first voice in oven courts.
  \item \textbf{Quiet Wealth.} Private hoard without public work is suspect; \emph{unworked shine} draws auditors and jokes in equal measure.
\end{itemize}
\textit{At the table:} presenting a valid \emph{Display Writ} improves \emph{Position +1} in civic negotiations once/scene; on a miss, start \emph{Audit Clock [4]}.

\medskip\hrule\medskip

\subsection{XIII. Orders \& Companies}

\begin{itemize}[leftmargin=*]
  \item \textbf{Iron Avengers.} Traditionalists who enforce blood-feuds inscribed on keystones. They keep \emph{Gray Lists} (names owing red silver) and are bound to answer when rung. Most cities tolerate them only under license.
  \item \textbf{Spirit Shield Warriors.} Ancestor-venerating protectors who wear \emph{mask-helms} etched with lineage prayers. They guard vent-stacks, memorial galleries, and children’s routes. When the \emph{Stone-Press} rises, they stand first.
  \item \textbf{True Masons.} Wanderers of Acasia, Vhasia, and Viterra who repair ancient Aeler work—bridges, sluices, ovens, wards. Paid in tithe-windows, charter clauses, and a bed in any house that stands because of their hands.
  \item \textbf{Edgewalkers.} Opportunists who scan borders and markets for the gap that pays. Part scout, part broker, part thief-catcher, they sell \emph{first knowledge} of breaks and bargains.
  \item \textbf{Reform Lodges.} Aeler who object to human exploitation concentrate in foothill freeholds and human cities. They press for \emph{grain sureties without hooks}, teach oven math to neighbors, and vote their ledgers in favor of mixed councils.
\end{itemize}

\textit{Strings \& Tools:} \emph{gray list token} (compels a parley before feud blows), \emph{ancestor mask} (once/scene resist fear/psychic pressure), \emph{mason’s oath} (DV --1 on repair/restoration scores), \emph{edgewalker marker} (create an opportunity clock on first contact).

\medskip\hrule\medskip

\subsection{XIV. Ethics \& Dissent}

Many Aeler reject exploitative arrangements with human protectorates. You find them teaching weights in river towns, serving as oven-wardens in mixed boroughs, and organizing \emph{Charter Kitchens} where the bell rings regardless of ledger. In Mistland winters, these dissenters argue for \emph{tithe holidays} and push to convert ward-labor into paid guild work. Their opponents answer with ledgers; their allies answer with bread.

\medskip\hrule\medskip

\subsection{XV. Play Hooks (Orders in Motion)}

\begin{enumerate}[leftmargin=*]
  \item \textbf{Red Bell, Gray List.} An Iron Avenger ring forces a feud claim in a city that banned duels. Can you translate blood into bread-price and retire the name from the stone?
  \item \textbf{Masks in the Miasma.} Spirit Shields request outside help to clear a vent gallery where Stone-Press twists numbers. Keep the count, keep your mind.
  \item \textbf{Bridge Without Purse.} True Masons will repair a flood-broken span if a mixed council signs a Display Charter that elevates commonworks over private shine.
  \item \textbf{Edge of Profit.} An Edgewalker offers first knowledge of a Syndicate price shock at the harbor. Take the lead, share the gain, or watch the sluice open for someone else.
  \item \textbf{Charter Kitchen.} Reform Lodges move to open a winter oven without tithe. Broker grain, tame politics, and post the bell before the ledgers close the door.
\end{enumerate}

\section*{Affinity (Aeler)}
\begin{itemize}
  \item \textbf{Stone \& Breath:} +1 die to Craft, Tinker, and Survival rolls related to infrastructure, construction, or resource management. In underground/dense urban environments, +1 Position on rolls to navigate, maintain, or sabotage systems.
  \item \textbf{Deep Accounting:} Once per scene, you can spend 1 Boon to ``audit'' a situation - ask one question about hidden resources, costs, or dependencies that the GM must answer truthfully.
\end{itemize}

\section*{Starting Talent}
\talent{Vent Prior's Training (3 XP - Minor Talent):}
\begin{itemize}
  \item Requirements: Craft 1+, Wits 2+
  \item Benefits: 
    \begin{itemize}
      \item +1 die to rolls involving air quality, ventilation, structural integrity, or underground navigation.
      \item Know the ``Nine Measures'' (light, draft, echo, etc.) - can detect environmental hazards or hidden passages with a successful Wits + Notice roll (DV 3).
      \item Once per scene, can ``read'' a structure like a ledger, gaining +1 die to rolls to understand its construction, weaknesses, or maintenance needs.
    \end{itemize}
\end{itemize}

\section*{Key Talents}

\subsection*{Mint \& Measure Path}
\talent{Calibrated Weight (4 XP - Minor Talent):}
\begin{itemize}
  \item +1 die to rolls involving trade, negotiation, or detecting counterfeits.
  \item Once per scene, can force a ``public bowl'' (fairness hearing) in a market or negotiation scene by spending 1 Boon. This shifts the DV of the negotiation by -1 and gives you +1 Position.
\end{itemize}

\talent{Keystone Clause (6 XP - Major Talent):}
\begin{itemize}
  \item Requirements: Calibrated Weight
  \item Benefits: 
    \begin{itemize}
      \item Can establish ``keystone rights'' over a piece of infrastructure, contract, or resource. This gives you a String (see below).
      \item Once per scene, can ``pause'' a route, contract, or service you control, converting a chase into a stand (Position +1 for defenders who prepared) or forcing a negotiation.
    \end{itemize}
\end{itemize}

\subsection*{Oven-Warden's Path}
\talent{Charter Bell (5 XP - Minor Talent):}
\begin{itemize}
  \item Requirements: Presence 2+, Command 1+
  \item Benefits:
    \begin{itemize}
      \item In a crisis involving survival (famine, shelter, etc.), can organize communities with a successful Presence + Command roll (DV 3). Success grants +1 Position to all allies for survival-related actions.
      \item Once per scene, can invoke an ``Oven Charter'' - force a public bowl in a crisis situation, ensuring fair distribution of resources or preventing violence.
    \end{itemize}
\end{itemize}

\subsection*{Vent-Captain's Path}
\talent{Bell-Chain Signal (4 XP - Minor Talent):}
\begin{itemize}
  \item Requirements: Wits 2+, Notice 1+
  \item Benefits:
    \begin{itemize}
      \item +1 die to rolls involving communication, signals, or detecting hidden information.
      \item Can establish a ``bell-chain'' communication network with allies. Once per scene, can send a message instantly to any ally within the network, regardless of distance (as long as there's a path).
    \end{itemize}
\end{itemize}

\section*{Strings (Aeler)}
Aeler characters accumulate ``Strings'' - markers of influence and control over infrastructure, contracts, and resources. These work like Bonds but are more concrete and economic.

\begin{itemize}
  \item \stringtype{Keystone Right:} Control a critical piece of infrastructure (bridge, sluice, ventilation shaft). Can ``pause'' it once/scene.
  \item \stringtype{Grain Surety:} Control a food source or storage. Can force fair distribution or gain leverage in negotiations.
  \item \stringtype{Sluice Key:} Control water flow or sanitation. Can shift DV by -1 on operations involving water/flood.
  \item \stringtype{Mint Die:} Control a source of currency or measurement. Can force a public bowl or detect counterfeits.
  \item \stringtype{Oven Charter:} Control a public resource (oven, shelter, etc.). Can invoke charter protections.
\end{itemize}

\section*{Patron Relationships}
Aeler typically bond with Patrons related to \textbf{Order, Infrastructure, Wealth, or Craft} (Sacred Geometry, Clockwork Monad, Mab, The Witness, Oath of Flame \& Light). Their approach to patronage is often contractual - they make deals and expect precise repayment.

\section*{Complications}
\begin{itemize}
  \item \complicationtype{Stone-Press Susceptibility:} When in deep underground environments or high-stress situations, make a Spirit test (DV 3) or suffer -1 die to rolls due to hallucinations or mental pressure.
  \item \complicationtype{Ledger Dependency:} You rely on precise accounting. If separated from your ledgers/records for more than a day, suffer -1 Position on rolls involving planning or resource management.
  \item \complicationtype{Surety Obligation:} You've guaranteed resources to a community. If you fail to deliver, mark 2 segments on your Obligation clock with your primary Patron (or gain a permanent Complication if you don't have one).
\end{itemize}

\section*{Sample Aeler Character}

\textbf{Orra Vent-Prior of Third Stack}
\begin{itemize}
  \item Body 2, Wits 3, Spirit 2, Presence 2
  \item Skills: Craft 2, Notice 2, Survival 1, Tinker 1
  \item Talents: Vent Prior's Training, Bell-Chain Signal
  \item Affinity: Stone \& Breath
  \item Strings: Keystone Right (on the main elevator shaft to the surface), Sluice Key (controls the water reclamation system)
  \item Complication: Stone-Press Susceptibility
\end{itemize}

\section*{Adventure Hooks}
\begin{enumerate}
  \item \textbf{The Bell That Rang Twice:} Investigate a malfunctioning signal system in an Aeler hold that's causing dangerous delays. Is it sabotage, a Deep Drake, or something worse?
  \item \textbf{Bread Before Swords:} A city's Oven Charter is being contested by a powerful syndicate. Can you ensure the winter ovens stay open and fair?
  \item \textbf{Keystone Courtesy:} A bridge has ``politely rested'' - someone altered the keystone tablet. Track down the culprit through the trail of chisel marks and contract disputes.
  \item \textbf{Mist Ward Winter:} The Mistland wards are failing and the fog is eating the farms. Repair the towers with locals who distrust Aeler ``ledgers'' - and prove the math keeps them alive.
  \item \textbf{Charter Kitchen:} A Reform Lodge wants to open a winter oven without tithe. Broker grain, tame politics, and post the bell before the ledgers close the door.
\end{enumerate}

\section*{Integration with Core Rules}

\subsection*{Deep Drakes as Fronts}
When encountering Deep Drakes, use the provided fronts:
\begin{itemize}
  \item \textbf{Stone-Press [6]}
  \item \textbf{Miasma Spread [4]}
  \item \textbf{Vent Failure [4]}
\end{itemize}
When Stone-Press fills, escalate with hallucinated ledgers and mis-signed keystones.

\subsection*{Tally-Law Mechanics}
Aeler law (Tally-Law) can be represented through:
\begin{itemize}
  \item Using Strings system for contractual obligations
  \item Applying -1 DV in negotiations when proper documentation exists
  \item Creating clocks for ``Audit [4]'' when contracts are violated
\end{itemize}

\subsection*{Display Rights}
Aeler status through Display Rights can be tracked as:
\begin{itemize}
  \item Minor Asset (4 XP): Public Display Charter - +1 Position in civic negotiations
  \item Major Asset (8 XP): Rings of Account - First voice in oven courts, DV -1 on resource management
\end{itemize}

\clearpage

\chapter{The Lethai}


\section*{Core Concept}
The Lethai are divided by an ancient curse that prevents them from fully accessing both physical and mental gifts. This division creates distinct elven cultures, each excelling in different aspects of existence while yearning for wholeness. They are masters of living law, environmental stewardship, and contextual knowledge.

\section*{Affinity (All Lethai)}
\textbf{Gift of the Body OR Gift of the Mind (Choose One):}
\begin{itemize}
  \item \textbf{Gift of the Body (Lethai-al/ar):} +1 die to physical actions (Body-based skills). Once per scene, can spend 1 Boon to enhance a physical ability beyond normal limits for one action.
  \item \textbf{Gift of the Mind (Lethai-thora):} +1 die to mental actions (Wits/Spirit-based skills). Once per scene, can spend 1 Boon to recall or deduce information that would normally be beyond immediate knowledge.
\end{itemize}

\textbf{The Curse of Division:} Lethai cannot take talents that enhance both physical and mental capabilities simultaneously. If a talent would grant bonuses to both, choose one benefit.

\section{Lethai-al — Root, River, and the Roof-Tree}

\subsection*{Woodwise lawkeepers, merchants of light and shade, and stewards of living borders}

\begin{quote}
``Name yourself once, name the river twice, and never name the forest as if it were yours.''
\end{quote}

\medskip\hrule\medskip

\subsection{I. Ground \& Memory}

The Lethai-al dwell where roof-trees braid the sky and rivers think aloud. Their memory is arboreal: rings and seasons, storms and healings, disputes recorded in coppice and replanting rather than tablets and seals. A Lethai-al oath is \textbf{root-law}: it binds living things—people, paths, waters—and is paid in years. To outsiders they appear quiet; to neighbors they are relentless auditors of footprint and flow.

\medskip\hrule\medskip

\subsection{II. Forest Commons (Places that Hold)}

\subsubsection*{Valewood}
Tall-canopy forest stitched with rain gardens and stone-steps. Trails shift by season; waystones carry moss-tallies (green for open, gray for mend, black for trespass). Songs mark crossings where words would bruise the undergrowth.

\subsubsection*{Roof-Tree Circles}
Villages built around elders: one monumental tree per circle, scaffolded with walkways and rain-catchers. Roof-Tree sap is consecrated—used to seal pacts and to dismiss them when harms outweigh promises.

\subsubsection*{River Courts}
Every significant stream keeps a court: a shingle bank, a willow-bench, a book of pebbles. Trade, ferry rights, and flood work are judged where the water can overhear.

\subsubsection*{Edge-Wards}
Where wood meets field or steppe, boundary groves are tended as treaties: thorn, ash, and fruit interplanted. The sharp feeds the soft—hedge law that stops hooves and nourishes neighbors.

\medskip\hrule\medskip

\subsection{III. Courtesy \& Everyday Law}

\textbf{Shade Etiquette} is the grammar of Lethai-al life. Observing it grants hospitality and hearing; ignoring it turns the forest against your plans.

\begin{itemize}[leftmargin=*]
  \item \textbf{Iron Covered.} Bare iron offends ward-lines; wrap it in leather or cloth. (Mechanic: on first entry, covered iron grants \emph{Position +1} in any parley or request to pass.)
  \item \textbf{Name Once.} Speak your name and intent at the edge; do not name the forest as property.
  \item \textbf{Step on Stone.} Where stones are laid, use them; every crushed sprout is a debt.
  \item \textbf{Water First.} Pour a first cup to the river or cistern before you drink.
  \item \textbf{Leave the Light.} Replace the shade you take: plant, mend, or pay. (Use \emph{Light-Due Strings} below.)
\end{itemize}

\textit{Clocks for breaches:} \textbf{Under-Root Grudge [4]}, \textbf{Stream-Clouding [4]}. When a clock fills, expect formal censure, raised tariffs, or wardens on the path.

\medskip\hrule\medskip

\subsection{IV. Courts \& Factions (Not One Voice)}

\begin{itemize}[leftmargin=*]
  \item \textbf{Wardens of the Roof-Tree.} Rangers and surveyors who measure canopy health, poacher pressure, and firebreaks. They carry leaf-badges and thorn-writs.
  \item \textbf{Merchant Foresters (Syndics of Light).} Manage sustainable fellings, resin trade, boat-timber quotas, and lantern-wood auctions. They argue that trade funds stewardship.
  \item \textbf{Songkeepers.} Priests of repair rites and remembrance festivals. They arbitrate forgiveness by counting regrowth, not coin.
  \item \textbf{Ferrymen of the Rain.} River pilots and dam-tenders; their votes swing courts in flood years.
\end{itemize}

\textit{Tension lines:} quota vs. canopy; export timbers vs. local craft; river straightening for mills vs. flood meadows for salmon.

\medskip\hrule\medskip

\subsection{V. Borders \& Neighbors}

\subsubsection*{Silence Furlong (with the Ykrul)}
A speechless border strip: no grazing, no felling, no names. Cross in silence, then speak once. Honored, it keeps peace warm; broken, gray-fletched messengers appear at dusk. Joint patrols mend hedges and lift stones; concessions are marked as seasonal strings on both boards.

\subsubsection*{Ridge Courtesy (with the Aeler)}
Root-law and stone-law trade proofs. Aeler build keystone stairs where Lethai-al choose the line; Lethai-al plant edge-wards that keep roads alive. Disputes end with bread in oven courts and sap at roof-trees.

\subsubsection*{Moots at the Mouth (with the Linns)}
Ship-timber quotas bind with replanting moots and harbor dues. When storm seasons bite, both sides sing the river clear.

\medskip\hrule\medskip

\subsection{VI. Economy of Light (Strings \& Ledgers)}

Light and shade are currencies. Every fell, ferry, and fire has a \textbf{light-due}—a balance to be repaid in replanting, canal-clearing, or trade at kinder rates.

\begin{itemize}[leftmargin=*]
  \item \textbf{Strings (examples):} \emph{light-due receipt}; \emph{shade-credit}; \emph{ferry right}; \emph{resin share}; \emph{canoe-lane priority}; \emph{seed tithe}.
  \item \textbf{Use:} Cash a light-due to gain \emph{DV --1} on any operation framed as repair, replanting, or flood work. Abuse it and start \emph{Under-Root Grudge [1]}.
  \item \textbf{Quotas as Clocks:} \emph{Boat-Timber Quota [6]}, \emph{Lanternwood Allotment [4]}. Fill the clock to unlock export; overfill and trigger \emph{Canopy Censure}.
\end{itemize}

\medskip\hrule\medskip

\subsection{VII. Threats \& Remedies}

\begin{itemize}[leftmargin=*]
  \item \textbf{Bark Blight.} A fungal rot spread by careless carts and damp blades. Remedy: ash-wash stations, tool quarantine, fire-lines by moon.
  \item \textbf{Poacher Pressure.} Syndicates take ripe resin and seedwood without dues. Remedy: undercover ferries; tariff traps at market; public shaming rites.
  \item \textbf{River Clouding.} Silt from mills and bad embankments chokes fish beds. Remedy: settle-ponds, scheduled sluices, mill-weirs redesigned with Aeler math.
  \item \textbf{Old Fire.} Lightning scars that wake in drought years. Remedy: fuel mosaics, controlled burns, ember patrols.
\end{itemize}

\textit{Fronts (example):} \textbf{Bark Blight [6]} → \textbf{Canopy Hunger [4]} → \textbf{Famine Tariff [4]}.

\medskip\hrule\medskip

\subsection{VIII. Tools \& Tokens (at the table)}

\begin{itemize}[leftmargin=*]
  \item \textbf{Leaf-Badge (Warden).} Once/scene, upgrade Position by +1 when enforcing forest law or rescue.
  \item \textbf{Seed-Tithe Seal.} Convert a sanction into a \emph{Repair Project [4]} instead of a fine or exile.
  \item \textbf{Ferry-Knot.} Skip the first \emph{River Hazard} consequence on a leg or chase; on use, pay a \emph{light-due}.
  \item \textbf{Moss-Tally.} Reveal one hidden quota clock in a market or lumberyard scene.
\end{itemize}

\medskip\hrule\medskip

\subsection{IX. Aesthetics \& Speech}

\begin{itemize}[leftmargin=*]
  \item \textbf{Materials:} bark-laminate armor, willow-bone frames, resin-glass panes, river-iron sheathed in leather.
  \item \textbf{Colors:} rain green, bark gray, amber, sky-through-leaf.
  \item \textbf{Proverbs:}
    \begin{itemize}
      \item ``Leave the forest no thinner than your shadow.''
      \item ``The river remembers who spoke over it and what they promised.''
      \item ``Cut where it wants to fall or do not cut at all.''
      \item ``If the hedge is hungry, the treaty is thin.''
    \end{itemize}
  \item \textbf{Names:} given + grove + season/deed: \emph{Talan of Reedfall, Who Unknotted the Flood}.
\end{itemize}

\medskip\hrule\medskip

\subsection{X. Integration Hooks}

\begin{itemize}[leftmargin=*]
  \item \textbf{Political Intrigue:} Forest syndics lobby city councils for lanternwood allotments; bribes look like “donations” to flood work.
  \item \textbf{Caravans:} Resin and seed routes require \emph{shade-credit}; escorts must mind iron covers and Silence Furlongs.
  \item \textbf{Wilderness:} Use \emph{Hinterlands} procedures with \emph{fuel mosaics} and \emph{blight quarantine} as special orders.
  \item \textbf{Violets \& Stone:} Dock courts host River Days; Lethai-al ferries grant \emph{canoe-lane priority} to crews who mend ladders and steps.
\end{itemize}

\medskip\hrule\medskip

\subsection{XI. Adventure Seeds}

\begin{enumerate}[leftmargin=*]
  \item \textbf{The Hedge That Ate a Road.} An edge-ward grew too well and swallowed a toll lane. Broker a cut and a replanting that keeps both law and trade.
  \item \textbf{Blight Under Boots.} Bark Blight rides a caravan’s wheel hubs. Build ash-wash stations without sparking a tariff war.
  \item \textbf{Silence Furlong, Twice Broken.} Two violations in one season risk a border feud. Walk the strip, plant the apology, and write a geometry both sides can live in.
  \item \textbf{River Court at Flood.} Mills want straighter banks; fishers want flood meadows; Aeler offer weirs. Choose a design, then stand in the water and defend it.
  \item \textbf{Lanternwood Ledger.} A stenographer vanished with the quota book before an export moot. Find them, or reconstruct the truth from moss tallies and ferry knots.
\end{enumerate}

\medskip\hrule\medskip

\subsection{XII. Portraying the Lethai-al}

Let \emph{policy} speak before mystique: quotas, ferries, hedges, and the cost of light. Make courtesy concrete (stones, coverings, cups) and consequences seasonal, not theatrical. Honor the forest as a \emph{party to the treaty}, not scenery. When in doubt, ask: \emph{What did this footstep cost, and who will we pay?}

\medskip\hrule\medskip

\subsection{XIII. Language \& Time (Context is King)}

Lethai speech is \textbf{context-saturated}. Syntax legible to grandparents will baffle grandchildren unless framed by the right place, season, and kinship. Meaning travels with \emph{context keys}—gesture, setting, shared story. A sentence lifted from its grove can die on the road.

\begin{itemize}[leftmargin=*]
  \item \textbf{Context Keys.} Place-name, season-mark, kinship-hand, and roof-tree sign. Absent any two, assume misreadings.
  \item \textbf{Old Texts.} Manuscripts older than two generations require \emph{gloss-trees} (marginal twig glyphs) or a \emph{songkeeper} to render.
  \item \textbf{In play.} When reading or pleading in older registers, increase DV by +1 unless the party holds a \emph{Context String} (gloss-tree, witness, song). Cashing a Context String reduces DV by --1 and creates a \emph{Shared Frame} for the scene (Position +1 on first exchange).
\end{itemize}

\medskip\hrule\medskip

\subsection{XIV. Peoples \& Lineages (Not One Grove)}

\paragraph{Lethai-al — \emph{The People of the Body}} 
The woodwise many. Found in most great forests, they are the elves most humans meet and the most willing to integrate in mixed cities. \textbf{Embodied and present}: heightened senses, quick strength, an ease with weather and work. Most \emph{half-elves} trace lineage here.

\begin{itemize}[leftmargin=*]
  \item \textbf{Everyday.} Ferry crews, hedge-wardens, resin syndics, ward dancers.
  \item \textbf{Gifts (examples).} \emph{Canopy Spring} (Position +1 on leaps/climbs once/scene); \emph{Scent of Rain} (clue on approaching weather/rot); \emph{Root-Balance} (resist shove/knockdown).
  \item \textbf{Strings.} ferry knot; resin share; roof-tree blessing.
\end{itemize}

\paragraph{Lethai-ar — \emph{The Oathbound (Dark Elves)}} 
Wood elves who swear to \textbf{Inae} (the Weaver, Angel of Spiders) or \textbf{Isoka} (the Serpent Queen, Angel of Snakes). Rare in the upper Amaranthine, growing along shaded trade arteries and cavernous underways. Their vows shape craft and law: webs of obligation, clean strikes, poison as \emph{medicine with teeth}.

\begin{itemize}[leftmargin=*]
  \item \textbf{Everyday.} Understory judges, silk engineers, antidote keepers, night pilots.
  \item \textbf{Gifts (examples).} \emph{Weaver’s Reading} (trace networks: DV --1 on uncovering plots/paths); \emph{Stillness} (be unnoticed in dim light until you move to act); \emph{Venin Lore} (convert Harm 1 from toxins to Fatigue once/scene).
  \item \textbf{Strings.} spider-vow token; serpent oath; silk-tithe.
  \item \textbf{Courtesy.} Oath-signs worn openly; breaking web-law costs \emph{Mask Rights} for a season.
\end{itemize}

\paragraph{Lethai-thora — \emph{The People of the Mind}} 
Urban circles, chiefly in \textbf{Thepyrgos}. Scholars, civil engineers, translators of other peoples’ law. \textbf{Long memories, sharp minds}; ironically, many are experts on human and Aeler culture. Their courts weigh arguments like bridges.

\begin{itemize}[leftmargin=*]
  \item \textbf{Everyday.} Archivists, canal designers, mint auditors, diplomatic tutors.
  \item \textbf{Gifts (examples).} \emph{Memory Canticle} (recall a text with line-accurate detail); \emph{Number Music} (DV --1 on design/repair/logistics projects framed aloud); \emph{Cold Reading} (Position +1 on first parley when you had time to observe).
  \item \textbf{Strings.} gloss-tree charter; canal seal; ledger witness.
\end{itemize}

\paragraph{Sundered Elves (Itinerant Thora)} 
Lethai-thora who reject the bench for the road. They carry portable gloss-trees and teach \emph{context literacy} in market towns. Neither fully grove nor fully court, they translate between.

\begin{itemize}[leftmargin=*]
  \item \textbf{Gifts (examples).} \emph{Bridge-Tongue} (treat cross-cultural etiquette DV as if in shared frame once/scene); \emph{Pocket Gloss} (create a one-use Context String on the fly).
\end{itemize}

\medskip\hrule\medskip

\subsection{XV. The Two Gifts (Body \& Mind) \& Mixed Blood}

Lethai tradition holds: \textbf{no one may bear both gifts} at once. A person chooses (or is chosen by) \emph{Body} (Lethai-al, Lethai-ar) \emph{or} \emph{Mind} (Lethai-thora). Training and rites reinforce the choice; law and courtesy expect it.

\begin{itemize}[leftmargin=*]
  \item \textbf{Rule of the Two Gifts.} A Lethai PC selects either \emph{Body} or \emph{Mind} at character creation; pick Gifts only from that list. Swapping lists requires a season-long project and comes with social costs (lose one String tied to the former gift).
  \item \textbf{Half-Elf Exception.} A half-elf whose heritage is \emph{quartered} (one quarter Lethai-al or -ar; one quarter Lethai-thora; remainder human or other) may take \emph{one} Body Gift and \emph{one} Mind Gift. This \textbf{Bridge-Born Clause} is rare and often controversial.
  \item \textbf{Most Half-Elves.} Most half-elves in mixed cities descend from Lethai-al lines; they default to Body Gifts unless the Bridge-Born Clause applies.
\end{itemize}

\textit{In play.} The Two Gifts are \emph{narrative permissions}: Body Gifts tend to grant Position shifts and physical resistances; Mind Gifts lean toward DV reduction in planning, repair, and rhetoric. Do not add new math—use core ladders.

\medskip\hrule\medskip

\subsection{XVI. Hooks \& Complications}

\begin{itemize}[leftmargin=*]
  \item \textbf{The Gloss That Wouldn’t Read.} A treaty text from three generations back refuses to make sense. Find the missing context keys or watch a border go hot.
  \item \textbf{Web-Law, River-Law.} Lethai-ar silk syndics claim a bridge-toll by Inae’s charter; Lethai-thora auditors counter with canal clauses. Bowl, then Board, then \emph{Shade Etiquette}.
  \item \textbf{Bridge-Born in Question.} A half-elf prodigy manifests both gifts; a circle moves to forbid the rite. Protect or persuade before the masks come down.
  \item \textbf{Mask Rights Forfeit.} A serpent oath-bearer allegedly broke web-law in human courts; can a seed-tithe seal commute sanction to a repair project?
  \item \textbf{The Sundered’s Lesson.} A Sundered gloss-carrier teaches a market to read Lethai context—someone powerful profits from the confusion and wants them gone.
\end{itemize}

\medskip\hrule\medskip

\subsection{XVII. Rivals \& Distance: Aeler and Ykrul}

\subsubsection*{With the Aeler (Stone-Law, Root-Law)}
The quarrel is old and mostly quiet: \textbf{stone-law} tallies weight and span; \textbf{root-law} tallies shade and years. Each doubts the other’s ledgers. Aeler keystone stairs bite deep; Lethai-al prefer lines that bend with slope and sap. They trade proofs—bridges that do not drown salmon, hedges that do not choke roads—and keep their councils separate when tempers rise. The Aeler call elves the "Aelaef"

\begin{itemize}[leftmargin=*]
  \item \textbf{Treaties in Practice.} \emph{Keystone-in-Bark} accords mark where a road enters a grove: the Aeler choose the tread, the Lethai choose the line. Breaking either creates fines paid as \emph{repair years}.
  \item \textbf{Isolation Habits.} Pass-keeps that only ring bells to one side; roof-tree circles that refuse iron even when wrapped. Delegations meet at \emph{split courts}—half oven, half river bank—and adjourn before dusk.
  \item \textbf{Strings.} keystone courtesy; shade-credit; river-weir writ; seed-tithe seal.
  \item \textbf{Clocks.} \textbf{Ridge–Root Stalemate [4]} (project stalls until a mixed design is signed); \textbf{Sluice vs. Salmon [4]} (silt, tariffs, tempers).
  \item \textbf{Levers at the Table.} If a scene honors both \emph{iron-covered} and \emph{step-on-stone}, begin in a \emph{Shared Frame} (Position +1 on the first exchange). If either party arrives with bare iron or cuts outside plan, advance \emph{Ridge–Root Stalemate} by +1.
\end{itemize}

\subsubsection*{With the Ykrul (Grass-Law, Shade-Law)}
Respect edged into distance. The \textbf{Silence Furlong} keeps hooves from roots and axes from borders: cross wordless, name yourself once, and keep to the stones. When honored, councils run warm; when broken twice in a season, gray-fletched messengers appear at dusk and rings draw tight.

\begin{itemize}[leftmargin=*]
  \item \textbf{Customs.} \emph{Bowl, Board, and Shade}—fairness, then geometry, then courtesy; a concession on the Ykrul board becomes a seasonal \emph{light-due} at the forest edge.
  \item \textbf{Isolation Habits.} Markets on the margin (hedge fairs) rather than deep exchange; joint patrols that mend hedges together but camp apart; songs traded by echo, not chorus.
  \item \textbf{Strings.} light-due receipt; ferry knot; windbreak right; border-song.
  \item \textbf{Clocks.} \textbf{Furlong Breach [4]} (speech, grazing, or felling in the strip); \textbf{Banner at the Hedge [3]} (Meadow banners crowd the edge; tempers tighten).
  \item \textbf{Levers at the Table.} When both sides keep the Furlong and the \emph{iron-covered} rule, reduce DV by --1 on any parley for passage or pasture. On a breach, increase DV by +1 and tick \emph{Furlong Breach}.
\end{itemize}

\paragraph{Notes for Portrayal}
Keep rivalry \emph{policy-first}: quotas, stairs, hedges, ferries, and the cost of light—rather than eternal enmity. Isolation is a choice with reasons: fewer misunderstandings, slower trade, less shared risk. Let dissenters exist on both sides (Aeler reform lodges; Lethai merchant foresters) who argue for \emph{mixed councils} where ovens, weirs, and hedges share one ledger.

\medskip\hrule\medskip

\subsection{XVIII. War by Shade — Deterrence \& Asymmetry}

The Lethai-al and Lethai-ar do not meet banners in open fields. They make \emph{routes expensive}, fragment columns, and end campaigns by ledger long before the first arrow. This doctrine—called \textbf{war by shade}—is why most peoples simply leave the forests alone.

\paragraph{Principles}
\begin{itemize}[leftmargin=*]
  \item \textbf{Fight the route, not the regiment.} Break ferries, confuse lanes, erase way-marks, and make supply walk twice.
  \item \textbf{One strike, five repairs.} Every cut obligates replanting or flood-work; war ends when the ledger of repair is signed.
  \item \textbf{Night, rain, understory.} Engage when wind and water cover sound; withdraw where canopy eats pursuit.
  \item \textbf{Courtesy as weapon.} An iron uncovered or a crushed sprout is grounds to raise tariffs, close ferries, or call wardens without drawing a blade.
\end{itemize}

\paragraph{Plays of the Shade}
\begin{itemize}[leftmargin=*]
  \item \textbf{Hedge War.} Edge-wards are cut and reknit to channel intruders into \emph{dead ground} pockets watched by wardens. (Invader \emph{Move/Scout} DV +1 while \emph{Hedge Mosaic} stands.)
  \item \textbf{River Denial.} Ferries vanish upstream; weirs open at dusk; mills idle to clear silt where pursuers need footing. (Start \emph{Supply Strangle [4]} on any hostile force dependent on crossings.)
  \item \textbf{Night Lanes.} Lethai-ar \emph{web-law} lays silk trip-lines and warning strings; Inae’s oath-bearers strike once and vanish; Isoka’s keepers use venin as medicine with teeth. (First ambush in dim light begins at \emph{Dominant} for defenders; on hit, intruders mark \emph{Fatigue} instead of direct Harm unless they stand and escalate.)
  \item \textbf{Canopy Runners.} Lethai-al move above sight-lines; arrows fall where footfalls never were. (Once/scene, defenders convert a \emph{pursuit} into a \emph{parley or retreat} with \emph{Position +1}.)
\end{itemize}

\paragraph{Deterrence, in Practice}
\begin{itemize}[leftmargin=*]
  \item \textbf{Fearsome Reputation.} Before any faction chooses a forest campaign, start \textbf{“Leave Them Alone” [2]}. On fill, leaders choose tariff, treaty, or route-around over invasion.
  \item \textbf{Intruder Exhaustion [6].} Ticks for lost hours, wet powder, spoiled grain, and wrong turns. On fill, morale folds without a decisive battle.
  \item \textbf{Arrow Ethics [3].} Songkeepers oversee. If intruders withdraw and sign repairs, arrows go quiet. If they burn, the mask rights come down and ambushes escalate.
\end{itemize}

\paragraph{Terrain Tags (at the table)}
\begin{itemize}[leftmargin=*]
  \item \textbf{Hedge Mosaic.} Invaders suffer DV +1 on Navigate/Scout; first defender action gains \emph{Position +1}.
  \item \textbf{Resin Smoke.} On ignition, obscure vision and sting eyes; convert first Ranged Harm 1 against defenders to \emph{Fatigue}.
  \item \textbf{Stone Steps.} If invaders \emph{step on stone} and keep iron covered, begin in a \emph{Shared Frame} (DV --1 to request safe passage). Breaches tick \emph{Under-Root Grudge}.
\end{itemize}

\paragraph{Why Most Leave Them Alone}
\begin{itemize}[leftmargin=*]
  \item Campaigns end in tariffs and repair years, not trophies—no glory to sell at court.
  \item Columns starve by inches while ledgers fatten with fines and ferry dues.
  \item Past nights have names (\emph{Two Bridges Gone, The River That Walked}); generals remember.
\end{itemize}

\paragraph{Hooks}
\begin{itemize}[leftmargin=*]
  \item \textbf{Five Cuts, One Bridge.} A hedge war strands a foreign regiment. Negotiate the repair ledger before someone lights the resin stores.
  \item \textbf{Web \& Weir.} Lethai-ar silk lines are blamed for a drowned patrol; River Denial saved three villages. Untangle oath from accident before \emph{Leave Them Alone} collapses to \emph{Banner at the Hedge}.
\end{itemize}

\medskip\hrule\medskip

\textbf{Context-Saturated Speech:} Lethai language evolves rapidly. Reading texts older than two generations requires successful Lore + Notice (DV 4-5) or a Context String. Speaking in older registers without proper context keys increases DV by +1.

\begin{center}
  {\LARGE Lethai-ar — The Vowed in Silk and Scale}
  
  \bigskip
  {\large ``Speak truth twice, whisper once; tie what you mean.''}
  \end{center}
  
  The Lethai-ar are not a separate bloodline so much as a vow. They are Lethai-al (the People of the Body, ``wood-elves'') who have sworn themselves to one of two ancient powers of the threshold: Inae, the Weaver (threads, mercy, binding context) or Isoka, the Serpent Queen (shedding, cunning, the price of cure). In some ages they are scarce, in others—like the present—their numbers grow wherever borders fray and oaths need teeth.
  
  They have shunned both the Gift of the Body and the Gift of the Mind, rejecting the paths of pure strength and pure knowledge. Instead, they seek the Blessing of Pattern from Inae and the Blessing of Change from Isoka—finding power in the spaces between certainty.
  
  They are the elves most other peoples prefer not to cross at night.
  
  \bigskip
  
  \textbf{Where They Live}
  
  \begin{itemize}
  \item \textbf{Edge-Forests \& Gloam Holds.} Rope-bridged canopy towns with silk wind-screens, scent-coded paths, and night gardens. On the valley floor: stone ``molt courts'' where the old is shed and sealed.
  \item \textbf{Under-Vales \& Rootways.} Knot-mapped tunnels, mirror-trick chambers, and pool-shrines where whispers travel farther than footsteps.
  \item \textbf{Port Enclaves.} Lantern-draped alleys near customs houses: contract brokers, poison-cure apothecaries, and ``mask-right'' parlors where parley is held under silk.
  \end{itemize}
  
  They're rare north of the Upper Amaranthine, but cells are appearing along caravan routes and siege lines—wherever a binding needs mercy or a lie needs a price.
  
  \bigskip
  
  \textbf{The Vows (Two Courts, One Edge)}
  
  \textbf{The Silk Courts of Inae (Weaver)}
  \begin{itemize}
  \item \textbf{Temper:} mercy with memory; the knot that mends rather than strangles.
  \item \textbf{Work:} draft and enforce multi-party compacts; rebuild frayed custom; hide the vulnerable in plain sight.
  \item \textbf{Signs:} three-strand cords, ledger-ribbons, masks with tear-slits.
  \item \textbf{Law:} Two ledgers, Said \& Meant, tied together with a visible clause.
  \item \textbf{Sin:} binding without consent; mending that erases the harmed.
  \end{itemize}
  
  \textbf{The Coil Courts of Isoka (Serpent)}
  \begin{itemize}
  \item \textbf{Temper:} cunning without needless cruelty; harm that carries its own cure.
  \item \textbf{Work:} expose weak seams; stage molts (identity exits) for those trapped; weaponize patience.
  \item \textbf{Signs:} shed-skin sashes, cup-and-vial pairs, scalpels wrapped in green thread.
  \item \textbf{Law:} every poison must be paired with a cure somewhere the oath-holder can reach.
  \item \textbf{Sin:} a wound with no offered remedy; a molt forced by shame.
  \end{itemize}
  
  Together: Inae binds the room to meaning; Isoka frees the person from traps. Lethai-ar make both moves feel like one motion.
  
  \bigskip
  
  \textbf{Language, Marks, and Memory}
  \begin{itemize}
  \item \textbf{Context-Locked Speech.} Elven is so contextual that two generations can't read each other cleanly. Lethai-ar solve this with marks (tattoo/ink/scar/silk beadwork) that carry context—names of witnesses, seasons, and prices—on the body.
  \item \textbf{Marks → Curses.} A mark without its living story curdles: a ``Hunter's Night'' becomes sleeplessness; a ``Mercy Knot'' becomes compulsion. The Lethai-ar curate and retire marks like hazardous heirlooms.
  \item \textbf{Artifact Burdens.} Many sacred tools (masks, cords, vials) store cultural knowledge. Lethai-ar treat them like ledgers: if the story breaks, so does the tool's kindness.
  \end{itemize}
  
  \bigskip
  
  \textbf{The Shunned Gifts}
  
  The Lethai-ar reject two fundamental elven paths, viewing them as forms of hubris:
  
  \textbf{The Gift of the Body}  
  The pursuit of physical perfection, martial prowess, and dominance through strength. The Lethai-ar see this as a trap—the body ages, weakens, and fails. Those who chase only physical mastery become brittle when their bodies betray them.
  
  \textbf{The Gift of the Mind}  
  The worship of pure knowledge, magical power, and intellectual supremacy. The Lethai-ar view this as dangerous isolation—the mind can become lost in abstractions, disconnected from the web of relationships that bind communities together.
  
  Instead, they seek the \textbf{Blessing of Pattern} from Inae and the \textbf{Blessing of Change} from Isoka—gifts that come not from individual perfection, but from understanding one's place in the greater weave of existence.
  
  \bigskip
  
  \textbf{Etiquette \& Rites (How to Survive Their Parlors)}
  \begin{itemize}
  \item \textbf{Mask-Right.} Serious parley is held under silk; roles and intentions are declared, then the mask goes on. Break mask-right and you will not be heard again that season.
  \item \textbf{Speak Twice, Whisper Once.} State the truth two ways; then the price as a whisper to the witness. If you can't name the price, you don't understand your own request.
  \item \textbf{Thread Before Blade.} Offer a binding solution first. If steel must speak, it speaks where a thread was refused.
  \item \textbf{Vial Courtesy.} In Isoka's houses, a dose sits beside its antidote. Taking one without the other marks you as untrustworthy.
  \end{itemize}
  
  \bigskip
  
  \textbf{War Without Battle (Doctrine)}
  
  The Lethai-ar are the textbook on asymmetric warfare:
  \begin{itemize}
  \item \textbf{Silence Corridors.} Silence furlongs across borders: no grazing, no felling, no names. Violations turn your scout signs against you.
  \item \textbf{Ambush as Audit.} They treat logistics like law—cut one keystone (a ferry, a ridge path) and make you pay repair years under witness.
  \item \textbf{Fear Economy.} Their reputation buys exits before arrows. Most armies learn to avoid ``black ribbon nights,'' when coils and webs both shift.
  \item \textbf{Night Doctrine.} Lamp placements create ``honor lanes'' where even pursuers must slow. They kill lights, not people—unless you force the price.
  \end{itemize}
  
  Most peoples leave them alone; those who don't tend to meet a contract they wish had been a skirmish.
  
  \bigskip
  
  \textbf{Neighbors \& Borders}
  \begin{itemize}
  \item \textbf{Ykrul:} Mutual distance, ritual respect. Border treaties often include a speechless strip and a joint ``Bowl \& Board'' at season's end. Ykrul price routes; Lethai-ar price behavior.
  \item \textbf{Aeler:} Underway adjacency breeds friction. Aeler bring breath-time and beams; Lethai-ar bring mask-right and mercy-knots. Both enforce oaths; they disagree about who writes the receipt.
  \item \textbf{Lethai-al / Lethai-thora:} The Lethai-ar recruit from the al; they contest with the thora over who defines ``context.'' Sundered thora wanderers sometimes find refuge under coil or web.
  \end{itemize}
  
  \bigskip
  
  \textbf{Economy \& Everyday}
  \begin{itemize}
  \item \textbf{Web-Law Brokerage.} Multi-party deals that hold even across language drift.
  \item \textbf{Poison-Cure Houses.} Venoms for cutting cruelty short; antidotes for those who choose return.
  \item \textbf{Night Markets.} Silk-screened stalls that sell contexts: a role, a mask, a witness-for-hire who'll remember what you meant.
  \end{itemize}
  
  \bigskip
  
  \textbf{What They Believe They're For}
  \begin{itemize}
  \item To bind harm so it cannot spread.
  \item To cut traps that call themselves tradition.
  \item To carry context so tomorrow's children won't inherit our mistakes as their curses.
  \end{itemize}
  
  They will tell you, if you ask in the right room: ``We are the price you pay to keep meaning.''
  
  \bigskip
  
  \textbf{Frictions \& Story Heat}
  \begin{itemize}
  \item \textbf{Marks Going Bad.} A famous mark has lost its last living witness; now it behaves like a curse. Do you retire it—or repair the story it needs?
  \item \textbf{Mask-Right vs. Under-Beam.} Aeler want a dispute under a single lamp (witness law). Lethai-ar demand silk (role law). Which threshold rules?
  \item \textbf{Snake's Mercy.} A tyrant's guard captain begs a molt (new life) after unspeakable work. The cure exists—who pays for it?
  \item \textbf{Thread in the Steppe.} A silence furlong through Ykrul pasture is broken; wolves and politics arrive together. Re-stitch the corridor or accept a border that now bites.
  \end{itemize}
  
  \bigskip
  
  \textbf{How to Portray Them (at the table)}
  \begin{itemize}
  \item \textbf{Embodied calm.} Even at war, they stand like dancers waiting for the beat.
  \item \textbf{Context first.} Every request has a price, every truth needs a witness, every blade wants a thread it can't cut.
  \item \textbf{Mercy with edge.} They save people by binding situations—and they do it with the precision of surgeons, not saints.
  \end{itemize}
  
  The Lethai-ar do not hide because they are weak. They hide because the world forgets context quickly, and they would rather mend it than burn it. By shunning the Gifts of Body and Mind, they seek a deeper wisdom—the Blessing of their chosen patrons, earned through submission to pattern and embrace of necessary change.
  
  The Lethai-ar are the elves you negotiate with if you're wise—and fight only if you enjoy losing to rules you didn't bother to learn.
    
\subsection{XIX. Marks — Burden, Blessing, and Lost Context}

Many Lethai artifacts carry \textbf{inherited knowing}. Leafblades etched with moss-tallies, resin-glass panels that hold songs, riverstones scored by ferrymen—these do not merely \emph{remember}; they \emph{obligate}. Among the Lethai—especially the woodwise Lethai-al—\textbf{past knowledge is a burden on the present}. What you inherit, you must carry carefully or set down with witness.

\paragraph{Passing-Knowledge Artifacts}
\begin{itemize}[leftmargin=*]
  \item \textbf{Leafblade (moss-tallied).} A knife whose edge carries hedge-law. \emph{Invoke:} gain a \emph{Context String} for border disputes; \emph{Cost:} tick \textbf{Burden Ledger [1]} (you owe repair work this season).
  \item \textbf{Resin-Glass Pane.} Sun-cured panels that hold a chorus. \emph{Invoke:} reduce DV by --1 on rites requiring the old wording; \emph{Cost:} \emph{Under-Root Grudge} +1 if used to excuse harm without replanting.
  \item \textbf{Riverstone Book.} Pebbles in corded order record ferry rights. \emph{Invoke:} create a temporary \emph{Ferry Knot String}; \emph{Cost:} pay \emph{light-due} before you drink.
\end{itemize}

\paragraph{Marks of the Body (Animist Rites)}
Lethai \textbf{Marks} (ink, scar, resin-inlay) amplify embodied gifts by binding person to pattern. Every Mark has \emph{Context Keys}—place, season, witness. \textbf{Out of context, a Mark drifts toward curse.}
\begin{itemize}[leftmargin=*]
  \item \textbf{Barkskin Mark.} Skin takes the grain of oak. \emph{Gift:} once/scene, downgrade a Physical Harm by one step (\emph{Position +1} when braced). \emph{Context Keys:} shaded work, sap rite, roof-tree witness. \emph{Curse (lost context):} brittleness—cold converts first Fatigue to Harm.
  \item \textbf{River-Ears.} Cartilage braided with silver reed. \emph{Gift:} sense current/voices across water (Clue +1 on rivers). \emph{Keys:} pour water first, ferryman’s knot. \emph{Curse:} flood-whispers—compelled toward lowest ground (SB: \emph{Downhill Pull}).
  \item \textbf{Spider’s Patience} (Inae oath). \emph{Gift:} hold still beyond breath; ambush begins at \emph{Dominant} in dim light. \emph{Keys:} web-tithe, oath-sign shown. \emph{Curse:} fixation—DV +1 to abandon a plan once set.
  \item \textbf{Serpent’s Breath} (Isoka oath). \emph{Gift:} once/scene convert Toxin Harm 1 to Fatigue; read heat like color. \emph{Keys:} antivenin rite, shed-skin offering. \emph{Curse:} cold-blooded—action slows in chill (Position --1 unless warmed).
  \item \textbf{Storm-Shadow.} Ink of ash and rain. \emph{Gift:} vanish in downpour; ranged attacks vs. you suffer --1 Effect in rain. \emph{Keys:} storm cup, gutter song. \emph{Curse:} thunder-call—loud noises pull attention (start \emph{Noticed [1]} on loud scenes).
  \item \textbf{Lanternbone.} Resin set in a healed break. \emph{Gift:} glow faintly to mark safe steps; \emph{once/scene} grant ally \emph{Position +1} on footing checks. \emph{Keys:} night watch, seed-tithe. \emph{Curse:} beacon—predators test the light (GM may spend SB to start \emph{Hungry Eyes [2]}).
\end{itemize}

\paragraph{Curses as Lost Context}
A Mark becomes a \textbf{Curse} when used without its Keys or against its ethic.
\begin{itemize}[leftmargin=*]
  \item \textbf{Context Drift [4].} Ticks when a Mark is invoked off-season, off-place, or without witness. On fill, the Mark expresses its \emph{Curse} until \emph{Context is Restored}.
  \item \textbf{Context Restored (project [4]).} Recover Keys: return to place, invite songkeeper, repay light-due. Aelinnel \emph{Sumwright gloss-trees} can substitute for witness once.
  \item \textbf{Burden Ledger [6].} Tracks obligation from inherited knowledge. On fill, pay with repair years, seed-tithes, or forfeit a String tied to the artifact.
\end{itemize}

\paragraph{At the Table (simple levers)}
\begin{itemize}[leftmargin=*]
  \item \textbf{Strings:} \emph{mark-right}, \emph{context key}, \emph{songkeeper witness}, \emph{gloss-tree charter}. Spend to reduce DV by --1 on rites/negotiations tied to that Mark or artifact.
  \item \textbf{Position/Effect:} Most Marks grant \emph{Position +1} or \emph{Effect +1} \emph{once/scene} when used with Keys; without Keys, instead tick \emph{Context Drift}.
  \item \textbf{Mitigation:} Aeler \emph{null-bells} can mute a Mark’s \emph{Curse} for a scene at the cost of \emph{lamp-time} or \emph{Public Outrage [1]} if used in a Lethai court.
\end{itemize}

\paragraph{Custodians \& Dissent}
\begin{itemize}[leftmargin=*]
  \item \textbf{Songkeepers} steward Keys and judge when burdens may be set down.
  \item \textbf{Lethai-ar Oathbinders} police oath-Marks; mask rights fall for abusers.
  \item \textbf{Lethai-thora Archivists} map Marks to contexts in \emph{gloss-trees}; Sundered Elves carry these into markets and mistakes.
\end{itemize}

\paragraph{Hooks}
\begin{itemize}[leftmargin=*]
  \item \textbf{The Bark That Broke.} A famed Barkskin bearer shattered in frost—find the missing roof-tree witness before winter claims more.
  \item \textbf{River-Ears in a Dry City.} A collector misuses a Riverstone Book; canals riot. Restore Keys or rewrite dues with a bowl at the fountain.
  \item \textbf{Spider’s Ledger.} Inae’s Mark is called curse after a web-judge froze a harvest lane. Untangle oath from context drift before masks come down.
\end{itemize}

\section{Valewood — Empire Under Leaves}

\subsection*{Living courts, phasing ruins, and the law of guest and host}

\begin{quote}
“Name yourself once, pour water first, and step on the stones. In the Valewood, truth grows wild and the paths remember your feet.”
\end{quote}

\medskip\hrule\medskip

\subsection{I. What the Valewood Is}
A vast, old-growth forest whose memory runs deeper than city annals. Phasing ruins (\emph{star-roads, living stone, breathing streets}) surface and sink with moon and season; oaths are measured in years of repair. Power moves through \textbf{hospitality}, \textbf{knowledge of ways}, and \textbf{careful speech}. The Lethai-al keep edge-wards and ferry rights; the Lethai-ar hold \emph{web-law} and shaded corridors; fae courts barter in \emph{truth told right now}; beast-kin patrol cairns none can map twice the same way.

\medskip\hrule\medskip

\subsection{II. Courtesy \& Entry (Shade Etiquette)}
Observe these to avoid waking the wood’s ledgers.
\begin{itemize}[leftmargin=*]
  \item \textbf{Iron Covered.} Wrap iron in leather or cloth. \emph{(Position +1 on first parley at a hedge or ferry.)}
  \item \textbf{Name Once.} Speak your name and intent at the edge; do not name the forest as property.
  \item \textbf{Step on Stone.} Use set stones and boardwalks; crushed sprouts become \emph{Light-Due}.
  \item \textbf{Water First.} Pour the first cup to the cistern or stream; then drink.
  \item \textbf{Leave the Light.} Replace what you take: plant, mend, or pay \emph{seed-tithe}.
\end{itemize}
\textit{Clocks for breaches:} \textbf{Under-Root Grudge [4]} (tariffs, closed ferries), \textbf{Stream-Clouding [4]} (mill shutdowns, fish wardens).

\medskip\hrule\medskip

\subsection{III. Routes \& Taboos (Using the Travel Seed)}
\paragraph{Shadow Corridor (Thin Shore).} The misted coastal lane north–south toward Zakov. Draw \emph{Spade} from Valewood, \emph{Heart} from Mistlands (or Valewood), \emph{Club} from Mistlands (wraiths, bell-line failures), \emph{Diamond} from either Mistlands (\emph{Ward-salt, Lantern Writ}) or Valewood (\emph{Way-Cord, Truce-Bough}) depending on which law you invoke.

\paragraph{Rule of 9s (Valewood \& Theona).} Whenever a 9 appears in a travel seed touching Valewood or the isle moots, add an \textbf{omission}: a missing step, an unsaid name, an unseen guest. If the 9 is a \emph{Diamond}, you may break the taboo once—someone will come to collect. Treat the omission as a one-scene \emph{Shared Complication} the table must name.

\medskip\hrule\medskip

\subsection{IV. Places That Hold (Spade Prompts)}
Use as scenes, sets, or advantages:
\begin{itemize}[leftmargin=*]
  \item \textbf{Star-Road Shard.} Pale flagstones hum when trod in order; a gate stirs if you match the cadence.
  \item \textbf{Rooted Amphitheatre.} Moss seats remember speeches not yet given; debate here shapes routes.
  \item \textbf{Glyphed Bridge.} Lies make glyphs glow; truth pays the toll.
  \item \textbf{Calendar Grove.} Solstice-aligned trunks; one tree is stone—time tallies are read here.
  \item \textbf{Unfound Arcade.} Arches appear at dusk, vanish at dawn; reality keeps hours.
  \item \textbf{Breathing City.} Vine-choked streets shift on the hour; the ruin is alive and listening.
\end{itemize}

\medskip\hrule\medskip

\subsection{V. Pressures of the Wood (Club Prompts)}
\begin{itemize}[leftmargin=*]
  \item \textbf{Ward-Trap.} Ancient mechanisms treat you like yesterday’s invader; tools and talk jam.
  \item \textbf{Oath-Magnet.} Words stick; a careless promise rewrites your plan (\emph{start a Task Clock you named aloud}).
  \item \textbf{Geas Catch.} A phrase turns into a job that names itself; the wood expects it done.
  \item \textbf{Name-Theft.} Edges of your true name are sampled; masks come down unless paid in courtesy.
  \item \textbf{Muster of Boughs.} Green banners rise—travel becomes trespass until a \emph{Truce-Bough} is raised.
  \item \textbf{Mirror Rain.} Leaf-drips replay choices you didn’t make; doubt taxes Position until resolved.
\end{itemize}

\medskip\hrule\medskip

\subsection{VI. Charms, Rights, \& Papers (Diamond Prompts)}
Treat these as \emph{Strings} that spend like documents or single-use boons:
\begin{itemize}[leftmargin=*]
  \item \textbf{Way-Cord.} Knot that points to the true next turn once. \emph{Spend:} DV --1 on Navigate/Scout; negates one \emph{Sweet Wind} lie.
  \item \textbf{Dew-Mirror.} Shows a thing as it is, not as sung, for one scene. \emph{Spend:} reveal a concealed intent or hazard; ignore one glamour.
  \item \textbf{Hazel Token.} Lawful crossing of a warded hedge without snag. \emph{Spend:} Position +1 to pass an edge-ward.
  \item \textbf{Honey-Right.} Eat and speak under bee-stone protection. \emph{Spend:} immunity to first social SB this parley.
  \item \textbf{Name-Bead.} Kept promise warms, broken one chills and glows. \emph{Spend:} convert one consequence into \emph{Repair Work [4]}.
  \item \textbf{Truce-Bough.} Cut, hunt, or camp in a named copse without offense. \emph{Spend:} suppress \emph{Muster of Boughs} for a scene.
  \item \textbf{City-Shard.} Wake one gate or stair in a phasing ruin. \emph{Spend:} create an exit where none is seen.
  \item \textbf{Oathsap.} Seal a pact that even echoes respect. \emph{Spend:} turn a verbal concession into a season-long right.
\end{itemize}

\medskip\hrule\medskip

\subsection{VII. People \& Practice}
\begin{itemize}[leftmargin=*]
  \item \textbf{Pathweavers.} Guides who knot safe routes into cord; their cords double as minutes of what was promised on the way.
  \item \textbf{Wardens of the Roof-Tree.} Canopy surveyors and hedge-keepers; they measure damage in \emph{repair years}.
  \item \textbf{Silk Engineers (Lethai-ar).} Keep night lanes and warning webs; \emph{mask rights} fall for oath abuse.
  \item \textbf{Songkeepers.} Render gloss and context for old speech; without them, even yesterday’s syntax can betray.
  \item \textbf{Beast-Kin.} Border guardians tied to cairns and seasons; they bargain in food, paths, and respect.
\end{itemize}

\medskip\hrule\medskip

\subsection{VIII. War by Shade (Deterrence in the Valewood)}
The wood teaches \emph{route warfare}: make supply walk twice; end campaigns by ledger. Ambushes begin at \emph{Dominant} under rain or understory; invaders who \emph{step on stone} and keep iron covered may parley for escorted withdrawal instead of bleed for it. Before anyone plans a forest campaign, start \textbf{“Leave Them Alone” [2]}; on fill, command chooses tariff, treaty, or route-around.

\medskip\hrule\medskip

\subsection{IX. Integration Hooks}
\begin{itemize}[leftmargin=*]
  \item \textbf{Political Intrigue.} \emph{Bowl, Board, and Shade}: fairness, then geometry, then courtesy. Convert a concession into a \emph{season-long right} at a ford, gate, or pasture.
  \item \textbf{Caravans.} Shadow Corridor runs require \emph{Way-Cord} or \emph{Lantern Writ}. Mistland law (ward-salt) vs. Valewood law (truce-bough) determines your final judgment.
  \item \textbf{Wilderness.} Treat edge-wards and fuel mosaics as special orders during camp phases; blight quarantine uses \emph{ash-wash stations}.
  \item \textbf{Aeler Roads.} \emph{Keystone-in-Bark} accords: Aeler choose tread, Lethai choose line. Breaking either creates fines paid as \emph{repair years}.
  \item \textbf{Ykrul Borders.} \emph{Silence Furlong}: cross wordless, then speak once. Keep it and reduce DV --1 on pasture passage; break it and tick \emph{Furlong Breach}.
\end{itemize}

\medskip\hrule\medskip

\subsection{X. Omens \& Seeds}
\begin{itemize}[leftmargin=*]
  \item \textbf{Sweet Wind.} A breeze that lies kindly; \emph{Wind-Veil} mutes it for a walk.
  \item \textbf{Missing Ninth.} Somewhere a ninth step or name is absent; pay in favors or get lost kindly.
  \item \textbf{Bell-Line Failure.} Mistland lantern law flickers; wraith indemnities come due.
\end{itemize}

\paragraph{Adventure Seeds}
\begin{enumerate}[leftmargin=*]
  \item \textbf{The Bowl at the Bridge.} A \emph{Glyphed Bridge} lights at every lie; arbitrate a toll war using \emph{Bowl, Board, and Shade}.
  \item \textbf{Way-Cord for a Price.} A pathweaver demands a \emph{Name-Bead} to knot your route; who pays when it chills?
  \item \textbf{Empire Wakes.} A \emph{Breathing City} aligns at dusk; open a \emph{City-Shard} gate or be counted as yesterday’s invader.
  \item \textbf{Silk and Sluice.} Lethai-ar web-law snared a patrol; river pilots claim the weir saved three villages. Balance oath and repair.
\end{enumerate}

\medskip\hrule\medskip

\subsection{XI. At the Table (Simple Levers)}
Keep to core ladders—use these as narrative permissions:
\begin{itemize}[leftmargin=*]
  \item \textbf{Context Strings} (\emph{gloss-tree, songkeeper witness}) reduce DV --1 when reading/pleading in older registers.
  \item \textbf{Light-Due \& Seed-Tithe} convert fines into \emph{Repair Projects [4]} instead of exile or coin.
  \item \textbf{Way-Cord / Truce-Bough / Dew-Mirror} spend for single-scene Position/DV shifts as listed; abusing them starts \emph{Under-Root Grudge}.
\end{itemize}

\textit{Portrayal note:} Let policy speak before mystique—ferries, hedges, quotas, and the cost of light—then add the wonder: paths that hum, rain that remembers, courts that sit where the river can overhear.

\section*{Starting Talents}

\subsection*{Lethai-al/ar Path: Embodied Presence}
\talent{Heightened Senses (3 XP - Minor Talent):}
\begin{itemize}
  \item Requirements: Wits 2+
  \item Benefits:
    \begin{itemize}
      \item +1 die to Notice and Survival rolls
      \item Can detect hidden creatures/objects with successful Wits + Notice (DV 3)
      \item In natural environments, +1 Position on stealth and tracking rolls
    \end{itemize}
\end{itemize}

\talent{Root-Balance (3 XP - Minor Talent):}
\begin{itemize}
  \item Requirements: Body 2+, Heightened Senses
  \item Benefits:
    \begin{itemize}
      \item +1 die to Athletics and resist shove/knockdown attempts
      \item Can move through natural terrain without penalty
      \item Once per scene, can stabilize on precarious footing with a successful Body + Athletics roll (DV 3)
    \end{itemize}
\end{itemize}

\subsection*{Lethai-thora Path: Mental Acuity}
\talent{Long Memory (3 XP - Minor Talent):}
\begin{itemize}
  \item Requirements: Spirit 2+
  \item Benefits:
    \begin{itemize}
      \item Perfect recall of events within the past week
      \item +1 die to Lore and Insight rolls involving historical or cultural knowledge
      \item Once per session, can recall a crucial detail from long-term memory (GM's discretion)
    \end{itemize}
\end{itemize}

\talent{Cold Reading (3 XP - Minor Talent):}
\begin{itemize}
  \item Requirements: Wits 2+, Long Memory
  \item Benefits:
    \begin{itemize}
      \item +1 die to Sway and Insight rolls involving observation and social analysis
      \item Can make intuitive leaps about people's motivations with successful Wits + Insight (DV 3)
      \item Once per scene, gain +1 Position on first parley when you had time to observe
    \end{itemize}
\end{itemize}

\section*{Key Talents}

\subsection*{Lethai-al Specific}
\talent{Canopy Spring (4 XP - Minor Talent):}
\begin{itemize}
  \item Requirements: Heightened Senses, Body 2+
  \item Benefits:
    \begin{itemize}
      \item +1 die to climbing, leaping, and arboreal movement
      \item Can perform feats of agility that would normally be impossible with a successful Body + Athletics roll (DV 4)
      \item Once per scene, gain +1 Position on movement actions in forested environments
    \end{itemize}
\end{itemize}

\talent{Scent of Rain (4 XP - Minor Talent):}
\begin{itemize}
  \item Requirements: Heightened Senses, Survival 1+
  \item Benefits:
    \begin{itemize}
      \item +1 die to weather prediction and detecting environmental hazards
      \item Can sense approaching storms, fires, or diseases with successful Wits + Survival (DV 3)
      \item In natural environments, can track by scent trails with +1 Effect
    \end{itemize}
\end{itemize}

\subsection*{Lethai-thora Specific}
\talent{Memory Canticle (4 XP - Minor Talent):}
\begin{itemize}
  \item Requirements: Long Memory, Lore 2+
  \item Benefits:
    \begin{itemize}
      \item Can recall texts with line-accurate detail
      \item +1 die to research and translation rolls
      \item Once per scene, can provide crucial historical context that advances investigation
    \end{itemize}
\end{itemize}

\talent{Number Music (4 XP - Minor Talent):}
\begin{itemize}
  \item Requirements: Long Memory, Wits 2+
  \item Benefits:
    \begin{itemize}
      \item +1 die to design, repair, and logistics projects
      \item Can perform complex calculations and engineering analysis with Wits + Craft (DV 3)
      \item When framing projects aloud, reduce DV by -1 on related rolls
    \end{itemize}
\end{itemize}

\subsection*{Lethai-ar Specific (Oathbound)}
\talent{Weaver's Reading (4 XP - Minor Talent):}
\begin{itemize}
  \item Requirements: Heightened Senses, Bond to Inae
  \item Benefits:
    \begin{itemize}
      \item +1 die to uncovering plots, paths, and hidden networks
      \item Can sense web-based communication and surveillance systems
      \item Once per scene, trace a network connection with successful Wits + Notice (DV 3)
    \end{itemize}
\end{itemize}

\talent{Venin Lore (4 XP - Minor Talent):}
\begin{itemize}
  \item Requirements: Bond to Isoka, Survival 1+
  \item Benefits:
    \begin{itemize}
      \item +1 die to identifying and treating poisons
      \item Can convert Harm 1 from toxins to Fatigue once per scene
      \item Can craft antidotes and venins with successful Lore + Survival (DV 4)
    \end{itemize}
\end{itemize}

\talent{Stillness (5 XP - Minor Talent):}
\begin{itemize}
  \item Requirements: Weaver's Reading OR Venin Lore, Stealth 2+
  \item Benefits:
    \begin{itemize}
      \item +1 die to stealth and hiding rolls in dim light
      \item Can remain unnoticed until you move to act
      \item Once per scene, become effectively invisible in appropriate cover for one exchange
    \end{itemize}
\end{itemize}

\subsection*{Sundered Elves Specific}
\talent{Bridge-Tongue (4 XP - Minor Talent):}
\begin{itemize}
  \item Requirements: Long Memory, Sway 2+
  \item Benefits:
    \begin{itemize}
      \item +1 die to cross-cultural communication and translation
      \item Can treat cross-cultural etiquette DV as if in shared frame once per scene
      \item Once per session, create a one-use Context String on the fly
    \end{itemize}
\end{itemize}

\subsection*{Half-Elf Exception}
\talent{Bridge-Born Clause (6 XP - Major Talent):}
\begin{itemize}
  \item Requirements: Half-elf heritage, both Lethai-al and Lethai-thora lineage
  \item Benefits:
    \begin{itemize}
      \item Can access both Gift of the Body and Gift of the Mind
      \item +1 die to rolls combining physical and mental skills
      \item Once per scene, can perform actions requiring both gifts simultaneously
      \item Immune to The Curse of Division
    \end{itemize}
\end{itemize}

\section*{Cultural Mechanics}

\subsection*{Shade Etiquette}
Lethai-al courtesy that grants hospitality and hearing:
\begin{itemize}
  \item \textbf{Iron Covered:} Bare iron offends ward-lines; wrap it in leather or cloth. (On first entry, covered iron grants Position +1 in parley)
  \item \textbf{Name Once:} Speak your name and intent at the edge; do not name the forest as property
  \item \textbf{Step on Stone:} Where stones are laid, use them; every crushed sprout is a debt
  \item \textbf{Water First:} Pour a first cup to the river or cistern before you drink
  \item \textbf{Leave the Light:} Replace the shade you take: plant, mend, or pay
\end{itemize}

\subsection*{Light-Due System}
Light and shade are currencies. Every fell, ferry, and fire has a light-due—balance to be repaid in replanting, canal-clearing, or trade at kinder rates.
\begin{itemize}
  \item Cash a light-due to gain DV -1 on operations framed as repair, replanting, or flood work
  \item Abuse it and start Under-Root Grudge [1]
\end{itemize}

\subsection*{Context Keys}
Lethai speech requires context keys:
\begin{itemize}
  \item Place-name, season-mark, kinship-hand, and roof-tree sign
  \item Absent any two, assume misreadings
  \item When reading or pleading in older registers, increase DV by +1 unless holding a Context String
  \item Cashing a Context String reduces DV by -1 and creates a Shared Frame (Position +1 on first exchange)
\end{itemize}

\section*{Strings (Lethai Cultural Influence)}
\begin{itemize}
  \item \stringtype{Light-Due Receipt:} Balance to be repaid in environmental work
  \item \stringtype{Shade-Credit:} Favor with forest communities
  \item \stringtype{Ferry Right:} Priority passage on waterways
  \item \stringtype{Resin Share:} Access to valuable forest products
  \item \stringtype{Canoe-Lane Priority:} Preferred river routes
  \item \stringtype{Seed Tithe:} Right to harvest forest reproduction
  \item \stringtype{Spider-Vow Token:} Oathbound commitment marker
  \item \stringtype{Serpent Oath:} Isoka-bound promise
  \item \stringtype{Silk-Tithe:} Payment for web-based services
  \item \stringtype{Gloss-Tree Charter:} Authority to interpret ancient texts
  \item \stringtype{Canal Seal:} Waterway management rights
  \item \stringtype{Ledger Witness:} Authority in scholarly disputes
  \item \stringtype{Keystone Courtesy:} Aeler-Lethai cooperation agreements
  \item \stringtype{River-Weir Writ:} Dam and water control permissions
  \item \stringtype{Windbreak Right:} Ykrul-Lethai border accommodations
  \item \stringtype{Border-Song:} Cross-cultural musical agreements
\end{itemize}

\section*{Patron Relationships}
\begin{itemize}
  \item \textbf{Lethai-al:} Often bond with nature-related Patrons (Old Man of the Black Forest, Carrion King, Nidhoggr)
  \item \textbf{Lethai-thora:} Prefer knowledge-related Patrons (The Witness, Sacred Geometry, Clockwork Monad)
  \item \textbf{Lethai-ar:} Bond with Inae (spiders) or Isoka (serpents) - dark, transformative Patrons
  \item \textbf{Half-elves:} Can bond with any Patron but often struggle with divided loyalties
\end{itemize}

\section*{Complications}
\complicationtype{The Curse of Division:}
\begin{itemize}
  \item When attempting to use talents from both paths, suffer -1 die to all rolls until scene ends
  \item Can only advance in one path per tier without the Bridge of Gifts talent
\end{itemize}

\complicationtype{Contextual Communication:}
\begin{itemize}
  \item Cannot effectively communicate with other Lethai without proper context keys
  \item Modern elven speech confuses outsiders (-1 die to social rolls with non-elves)
  \item Reading old texts requires special effort or assistance
\end{itemize}

\complicationtype{Cultural Tension:}
\begin{itemize}
  \item Lethai-al and Lethai-thora communities often distrust each other
  \item Half-elves face prejudice from both communities
  \item Lethai-ar are feared and often ostracized
\end{itemize}

\complicationtype{Environmental Responsibility:}
\begin{itemize}
  \item Every action in natural environments has consequences
  \item Breaking Shade Etiquette triggers social and environmental backlash
  \item Light-dues accumulate and must be repaid
\end{itemize}

\section*{Sample Characters}

\subsection*{Talan of Reedfall, Who Unknotted the Flood}
\begin{itemize}
  \item Body 3, Wits 3, Spirit 2, Presence 2
  \item Skills: Athletics 2, Survival 2, Notice 1, Craft 1
  \item Talents: Heightened Senses, Root-Balance, Canopy Spring, Scent of Rain
  \item Affinity: Gift of the Body
  \item Strings: Ferry Right, Resin Share, Shade-Credit
  \item Complication: Environmental Responsibility
\end{itemize}

\subsection*{Mira of the High Archive}
\begin{itemize}
  \item Body 2, Wits 4, Spirit 3, Presence 2
  \item Skills: Lore 3, Insight 2, Sway 1, Craft 1
  \item Talents: Long Memory, Cold Reading, Memory Canticle, Number Music
  \item Affinity: Gift of the Mind
  \item Strings: Gloss-Tree Charter, Ledger Witness, Canal Seal
  \item Complication: Cultural Tension
\end{itemize}

\subsection*{Sariel Half-Elf of the Crossroads}
\begin{itemize}
  \item Body 3, Wits 3, Spirit 2, Presence 3
  \item Skills: Melee 2, Lore 2, Survival 1, Sway 1
  \item Talents: Heightened Senses, Long Memory, Bridge-Born Clause, Bridge-Tongue
  \item Affinity: Both Gifts (through Bridge-Born Clause)
  \item Strings: Shade-Credit, Ferry Right, Gloss-Tree Charter
  \item Complication: Cultural Tension
\end{itemize}

\section*{Adventure Hooks}
\begin{enumerate}
  \item \textbf{The Hedge That Ate a Road:} An edge-ward grew too well and swallowed a toll lane. Broker a cut and a replanting that keeps both law and trade.
  \item \textbf{The Gloss That Wouldn't Read:} A treaty text from three generations back refuses to make sense. Find the missing context keys or watch a border go hot.
  \item \textbf{Web-Law, River-Law:} Lethai-ar silk syndics claim a bridge-toll by Inae's charter; Lethai-thora auditors counter with canal clauses. Bowl, then Board, then Shade Etiquette.
  \item \textbf{Bridge-Born in Question:} A half-elf prodigy manifests both gifts; a circle moves to forbid the rite. Protect or persuade before the masks come down.
  \item \textbf{Blight Under Boots:} Bark Blight rides a caravan's wheel hubs. Build ash-wash stations without sparking a tariff war.
\end{enumerate}

\section*{Integration with Core Rules}

\subsection*{The Curse Mechanic}
The Curse of Division works as a persistent condition:
\begin{itemize}
  \item When a Lethai attempts to use talents from both paths in the same scene, mark 1 segment on a Curse Clock [4]
  \item When the clock fills, suffer -1 die to all rolls until the next dawn
  \item The Bridge-Born Clause talent negates this effect
\end{itemize}

\subsection*{War by Shade}
Lethai defensive warfare principles:
\begin{itemize}
  \item \textbf{Hedge War:} Edge-wards channel intruders into dead ground pockets (Invader Move/Scout DV +1)
  \item \textbf{River Denial:} Ferries vanish, weirs open, mills idle (Start Supply Strangle [4] on hostile forces)
  \item \textbf{Night Lanes:} Web-law trip lines, single strikes, venom as medicine (First ambush in dim light begins at Dominant)
  \item \textbf{Canopy Runners:} Movement above sight-lines, arrows from impossible angles (Once/scene, convert pursuit to parley/retreat with Position +1)
\end{itemize}

\subsection*{Terrain Tags}
\begin{itemize}
  \item \textbf{Hedge Mosaic:} Invaders suffer DV +1 on Navigate/Scout; first defender action gains Position +1
  \item \textbf{Resin Smoke:} On ignition, obscure vision and sting eyes; convert first Ranged Harm 1 against defenders to Fatigue
  \item \textbf{Stone Steps:} If invaders step on stone and keep iron covered, begin in Shared Frame (DV -1 to request safe passage)
\end{itemize}
\clearpage

\begin{center}
  {\LARGE Ge'hai — Those Who Carry the Edge}
  
  \bigskip
  {\large Lethai elite cadres in body and in mind}
  \end{center}
  
  Among the Lethai, ge'hai is not a rank so much as a charge: to carry the edge where the people meet the world. The title is awarded in moot by three witnesses—one of body, one of mind, and one of place (a grove, a hall, a lamp under beam). From that day a ge'hai belongs to the work more than to themselves.
  
  Two great traditions shape them:
  \begin{itemize}
  \item \textbf{Lethai-al ge'hai (``People of the Body'')} — canopy commandos and keystone saboteurs, feared for asymmetric warfare and the calm speed with which they unmake an enemy's plan.
  \item \textbf{Lethai-thora ge'hai (``People of the Mind'')} — court tacticians and context-binders, infamous for turning rooms with etiquette, syntax, and the careful placement of witnesses.
  \end{itemize}
  
  Both hold the same maxim: \textbf{thread before blade}. Bind the harm; if binding fails, cut cleanly.
  
  \bigskip
  
  \section*{I. Lethai-al Ge'hai — Bark \& Sinew}
  
  \textbf{Doctrine:} Silence, keystone, vanish. Lethai-al ge'hai do not ``win battles''; they remove the pieces battles would need—ferries, fodder, bridges, nerve.
  
  \textbf{Training:} Night-sight drills, breath discipline, canopy movement, keystone reading (how a structure stands), mark-keeping (context tattoos that store orders and prices), and silence furlong protocol (no names, no smoke, no steel until witness).
  
  \textbf{Typical cadre (6–12):}
  \begin{itemize}
  \item \textbf{Path-Cutter} (route denial; hedge \& rope),
  \item \textbf{Keystone-Reader} (ferries, arches, dam lips),
  \item \textbf{Lantern-Thief} (light management; moon-mirrors),
  \item \textbf{Molt-Keeper} (field medic; poison/antidote pairing),
  \item \textbf{Whisper-Sergeant} (hand-sign command),
  \item \textbf{Sable-Bow} (harassment, lure, pursuit break).
  \end{itemize}
  
  \textbf{Field rites \& etiquette}
  \begin{itemize}
  \item \textbf{Mask-Right (field):} canvas veils with inked roles; parley can be held anywhere the masks are donned.
  \item \textbf{Vial Courtesy:} dose and cure travel together; to harm without an offered remedy is a ge'hai sin.
  \item \textbf{Speak Twice:} the order is given in common speech and then once in sign; if they diverge, the sign holds.
  \end{itemize}
  
  \textbf{Operations menu (what they actually do):}
  \begin{itemize}
  \item \textbf{Keystone Audit:} Mark the one stone, beam, or rope that bears a route. Remove or mend as needed; leave a repair year ledger if you break it.
  \item \textbf{Lamp Game:} Extinguish all but one light to create an honor lane. Pursuers must slow or lose face.
  \item \textbf{Gardened Ambush:} Shape a killing ground with vines, wire, and scent; fire once, then disappear into the furlong.
  \item \textbf{Ferry Night:} Hold a river crossing with bell-tokens and chalked prices; turn a rout into an orderly retreat.
  \end{itemize}
  
  \textbf{Strings / Diamonds they carry:}
  \begin{itemize}
  \item \textbf{Context Key} (treat a warded path as neutral once),
  \item \textbf{Mask-Right Seal} (declare parley under silk in the open),
  \item \textbf{Bell Token} (ring Bell Dawn to convert lethal fallout into payable cost),
  \item \textbf{Repair Beads} (pre-pledged seasons of mending).
  \end{itemize}
  
  \textbf{At the table (mechanical hooks):}
  \begin{itemize}
  \item In forest, under canopy, or along a marked furlong: Position +1 once per scene when acting unseen or disengaging.
  \item When a keystone is correctly identified in fiction, treat sabotage or shoring as DV −1.
  \item Spending a Bell Token in a retreat converts one lethal consequence to a named price or clock.
  \item Breaking Vial Courtesy or Mask-Right gives adversaries Position +1 against the cadre until amends are made.
  \end{itemize}
  
  \textbf{Three Lethai-al ge'hai ``builds'' (PC or cohort templates):}
  
  \textbf{1. Lantern-Thief — light control \& exits}
  \begin{itemize}
  \item \textbf{Edges:} Athletics, Stealth, Notice; Tags: Moon-Mirror, Smoke Net, Night Cord
  \item \textbf{Moves:} Steal the Lamp (shift a scene into dim—your terrain), Draw the Eye (create a false exit), Honor Lane (force pursuers to slow or suffer DV +1)
  \end{itemize}
  
  \textbf{2. Keystone-Reader — bridges, ferries, gates}
  \begin{itemize}
  \item \textbf{Edges:} Engineering/Lore (structures), Survival (water), Tinkering
  \item \textbf{Moves:} Name the Stone (spot the load point), Mend or Break (choose repair-year or collapse), Toll-Binder (chalk a temporary tariff that even foes hesitate to cross)
  \end{itemize}
  
  \textbf{3. Molt-Keeper — poison/antidote doctrine}
  \begin{itemize}
  \item \textbf{Edges:} Medicine, Alchemy, Insight
  \item \textbf{Moves:} Paired Vials (harm with an offered cure; secures parley), Clean Cut (field amputation as exit), Venom Truce (threat of escalation traded for withdrawal)
  \end{itemize}
  
  \newpage
  
  \section*{II. Lethai-thora Ge'hai — Vowel \& Veil}
  
  \textbf{Doctrine:} Name the room, bind the hour, tilt the future. They win by re-contexting: changing who counts as host/guest, who speaks under lamp, which ledger (Said/Meant) the court will honor, and what it would cost to ignore either.
  
  \textbf{Training:} Living grammar (syntax drift law), mask etiquette, ledger craft (recording Said \& Meant together), witness choreography (who stands under the lamp and when), memory palaces, and mark curation (retiring inherited marks before they turn to curses).
  
  \textbf{Court kit:} travel masks with role-ink, ribbon ledgers, name-beads for witness, bell-cord for single-lamp rites, antidote rings, and context knives (ritual blades that cut bonds after prices are named).
  
  \textbf{Operations menu:}
  \begin{itemize}
  \item \textbf{Two-Ledger Trial:} Force both letter and spirit on the table; the losing party must pay in repair years or reputation.
  \item \textbf{Mask Moot:} Declare roles, then seal parley under silk; shunts a lethal standoff into posture, concessions, and clocks.
  \item \textbf{Context Turn:} Move a meeting under a single lamp; what was rumor becomes witness. Truths spoken thereafter stick.
  \item \textbf{Molt Design:} Free a trapped official into a new name and role; avoid civil blood by changing the piece, not the board.
  \end{itemize}
  
  \textbf{Strings / Diamonds they carry:}
  \begin{itemize}
  \item \textbf{Clause-Bead} (insert a contingency into a live deal),
  \item \textbf{Name-Bead} (elevate witness under lamp; testimony counts),
  \item \textbf{Weaver's Writ} (bind a multi-party compact across language drift),
  \item \textbf{Shed-Skin Letter} (sanctioned exit from a coercive identity—price named, future favor owed).
  \end{itemize}
  
  \textbf{At the table (mechanical hooks):}
  \begin{itemize}
  \item When Mask-Right is established, treat first parley as Controlled→Dominant (Position +1).
  \item Presenting Said \& Meant together gives DV −1 to negotiate clauses; hiding one allows GM to spend SB for False Context complications.
  \item A declared under-beam witness turns a soft truth into a binding statement; refusing to stand under lamp shifts Effect −1 for your side until you do.
  \end{itemize}
  
  \textbf{Three Lethai-thora ge'hai ``builds'':}
  
  \textbf{1. Ribbon-Reader — contract surgeon}
  \begin{itemize}
  \item \textbf{Edges:} Diplomacy, Law/Customs, Insight
  \item \textbf{Moves:} Read the Stitch (spot hidden clause), Mercy-Knot (turn harm into a payable price), Seal of Seasons (make a concession last one season only)
  \end{itemize}
  
  \textbf{2. Mask-Herald — room control}
  \begin{itemize}
  \item \textbf{Edges:} Performance, Command, Etiquette
  \item \textbf{Moves:} Call the Roles (fix who is host/guest), Veil the Blade (threats become symbols; tempers cool), Seat the Witness (choose who stands under lamp)
  \end{itemize}
  
  \textbf{3. Shed-Singer — identity exits}
  \begin{itemize}
  \item \textbf{Edges:} Deception, Medicine/Lore (rites), Sway
  \item \textbf{Moves:} Design the Molt (legal/social escape plan with debt), Cure in the Cup (always carry a public antidote), Word-Cut (ritual severance after price named)
  \end{itemize}
  
  \newpage
  
  \section*{III. Friction, Rivalry, Integration}
  \begin{itemize}
  \item \textbf{With Ykrul:} Silence furlongs and Bowl \& Board share a border. Ge'hai respect Ykrul placement, but price behavior, not routes. Joint treaties tend to work; joint insults tend to last.
  \item \textbf{With Aeler:} Under-beam witness versus mask-right parley creates jurisdictional puzzles. Aeler Spirit Shields prefer lamps; Lethai-thora prefer silk. Ge'hai learn both.
  \item \textbf{With Lethai-ar:} Many ge'hai are also vowed (Weaver/Serpent). Coil and web add sharp edges: cure-paired poisons, mercy-knots, and the taboo against binding without consent.
  \end{itemize}
  
  \section*{IV. Story Heat (drop-in complications)}
  \begin{itemize}
  \item \textbf{Mark Gone Sour.} A cherished ge'hai mark has lost its last living context. It behaves like a curse (insomnia, compulsions). Repair the story or retire the mark.
  \item \textbf{Lamp vs. Veil.} A frontier court demands under-beam witness; the ge'hai require mask-right for safety. Can both thresholds be honored?
  \item \textbf{The Price of Cure.} A tyrant's enforcer begs a molt. The cure exists; naming its price may break a city.
  \item \textbf{Silence Broken.} A silence furlong is violated by foreign scouts; Lethai-al ge'hai must re-stitch the corridor before a war learns to speak there.
  \end{itemize}
  
  \section*{V. Playing Ge'hai (tone \& presence)}
  \begin{itemize}
  \item \textbf{Lethai-al:} move like a solved puzzle; every gesture pre-answers a chase. They speak with hands first. Their mercy is letting you leave by the exit they already cut.
  \item \textbf{Lethai-thora:} attend to breath, posture, and lamp distance. They do not ``win arguments''—they decide which truths exist in this room for this season.
  \end{itemize}
  
  Ge'hai do not seek glory. They seek rooms that stop killing people and roads that keep their promises. If you meet them at night, listen: the solution is already hanging from the next branch or folded in the next word; your only choice is whether you can live with its price.


  \section{Fanatic Variants: Inaea's Family-State \& Isoka's Permanent Revolution}

  \subsection*{Inaea Fanatics — The Family-State}
  \textit{“All under one roof. All under one line.”}
  
  \paragraph{Civic Doctrine.}
  Hospitality becomes law; \emph{Guest-Right} becomes jurisdiction. Neighborhoods are reorganized into \textbf{Households-of-Record} (block-roofs). Every door has a bell pattern; every conflict must occur \emph{under a line} with a registered witness. Travel requires \textbf{Line Permits} (who hosts you, who witnesses you).
  
  \paragraph{Institutions \& Tools.}
  Aunties of the Line (block wardens) • Bell-Wardens (signal police) • Mask-Mothers (context courts) • Household Books (who slept where) • Dew-Mirror Lattices (street-corner “mirrors” that log comings/goings).
  
  \paragraph{Everyday Feel.}
  Chimes everywhere; soft voices that demand your role; neighbors who “host” you for your own safety. People thank the state for keeping families intact while quietly living in fear of being declared \emph{Guestless}.
  
  \paragraph{Mechanics.}
  \begin{itemize}
    \item \textbf{Clocks:} Surveillance Grid [8], Household Compliance [6], Guestless List [4].
    \item \textbf{SB Menu (Keeper):} 
      1 SB — A bell mis-rings; position worsens for the unhosted. 
      2 SB — Auntie visit: papers, role, and host demanded. 
      3 SB — Safe-conduct revoked; \emph{under-the-line} arrest. 
      4 SB — District “under one roof” order: curfew lines drop.
    \item \textbf{Strings/Diamonds:} 
      \emph{Line-Pass} (bypass one checkpoint under escort), 
      \emph{Household of Record} (once per session, count as \emph{hosted}), 
      \emph{Witness Bell} (force parley, DV–1 if papers are clean).
  \end{itemize}
  
  \paragraph{How to Resist.}
  Variance in bells (Aveh shrines), two-ledgers (\emph{said/meant}) to expose coercive hospitality, Ykrul \emph{Bowl then Board} to reframe sanctuary as limited, seasonal rights instead of total jurisdiction.
  
  \paragraph{Adventure Seeds.}
  Smuggle a dissident \emph{off} the Household Books • Prove a “Guestless” family still has a host • Flip a Bell-Warden’s lattice to record \emph{the watchers}.
  
  \subsection*{Isoka Fanatics — The Permanent Revolution}
  \textit{“Decide by dawn or be decided.”}
  
  \paragraph{Civic Doctrine.}
  Endless purification by choice. The city is ruled by \textbf{Dawn Committees} that settle everything with \emph{Fang of Decision}. Indecision is treason; nuance is counter-revolution. Members \emph{Shed-Skin} (role swaps, recantations) in public rites; yesterday’s hero is today’s saboteur.
  
  \paragraph{Institutions \& Tools.}
  Dawn Tribunals • Purity Heralds (announce tests) • Choice-Boards (public schedules of decisions) • Shedding Courts (compelled confessions) • Venom Truce Zones (violence-pauses for show trials).
  
  \paragraph{Everyday Feel.}
  Posters promising clean tomorrows; midnight runners posting new dawn questions; friends practicing “the hard truth” to survive their turn on the steps. Relief when a choice lands; dread of the next one.
  
  \paragraph{Mechanics.}
  \begin{itemize}
    \item \textbf{Clocks:} Decision Hegemony [6], Purity Ledger [4] (per PC/faction), Cadre Schism [6].
    \item \textbf{SB Menu (Keeper):} 
      1 SB — Loyalty test now; speak a painful truth or lose Position. 
      2 SB — Cell split: an ally denounces a nuance. 
      3 SB — Purge wave: lose an Asset \emph{or} accept a branded geas. 
      4 SB — Summary Dawn: a tribunal seizes jurisdiction over the current scene.
    \item \textbf{Strings/Diamonds:} 
      \emph{Revolutionary Mandate} (override one standing order once, then tick Purity), 
      \emph{Truce Seal} (bind a dawn parley that \emph{must} end in a choice), 
      \emph{Scale Draught} (resist fear in a tribunal scene).
  \end{itemize}
  
  \paragraph{How to Resist.}
  Flood the witness stand (Gravel Seraphs, multiple contexts) • Convert “decision” into \emph{exits} via Kon’reh—split tolls, timebox rights • Force \emph{Bowl} (fairness) before \emph{Fang} (finality) in public framing.
  
  \paragraph{Adventure Seeds.}
  Save a moderate marked for “indecision” • Prove a forged dawn clause on a charter • Survive a Shedding Court by turning the hard truth on the tribunal itself.
  
  \subsection*{Running the Tone}
  \begin{itemize}
    \item \textbf{Inaea Fanatics = Creeping Intimacy as Control.} Bells, Aunties, hospitality audits, “for your safety.” Make every courtesy a checkpoint.
    \item \textbf{Isoka Fanatics = Purity as Motion.} Decisions as spectacle, slogans change weekly, cadres splinter mid-scene. Make every victory threaten to become yesterday’s heresy.
  \end{itemize}
  
  \subsection*{Cross-Pressure}
  When both wings contest a city: the Family-State demands \emph{roles under a line}; the Revolution demands \emph{choices at dawn}. PCs can weaponize the clash: force Inaea to sign seasonal exits (limits surveillance), force Isoka to accept Bowl-first hearings (limits purges).
  
  \subsection*{Table Dials}
  \begin{itemize}
    \item \textbf{Severity (0–3):} number of active clocks per district from each wing.
    \item \textbf{Visibility:} bells and posters obvious (street fear) vs. subtle (only papers and whispers).
    \item \textbf{Mercy Valve:} once/session any PC can spend 2 Boons to insert a humane clause into a line or a dawn decision.
  \end{itemize}
  
\clearpage
\chapter{The Threshold Folk: Aelinnel, Aelaerem and Kin}

section*{Design Goals}
\begin{itemize}[leftmargin=*]
\item \textbf{Liminal Focus:} Cultures that live in thresholds, borders, and in-between spaces
\item \textbf{Cultural Integration:} Non-human peoples as complex societies, not monsters  
\item \textbf{Scale \& Perspective:} Mechanics reflect size and worldview differences without new math
\item \textbf{Seamless Integration:} Works with districts, factions, and Patron systems
\item \textbf{Narrative Depth:} Hidden layers and mystery inside existing settings
\end{itemize}

\section*{Quickstart (2 minutes)}
\begin{enumerate}
\item Choose Play Mode: Small PCs • Mixed Company • Bigfolk in Small Realms
\item Pick a Threshold Culture: Aelaerem, Mazereth, Umbral Kin, Archive Keepers
\item Mark Tracks: Threshold Sense [6], Visibility [6] (opt. map Visibility ↔ Notice on the Nook)
\item Choose 2 Gifts and 1 Realm Bond
\item Start with Hidden Knowledge (1) and Surface Connection (1)
\item Make a Nook/Clan Sheet, map a Threshold (4–6 nodes), and start local clocks: Bigfolk Stir [6], Lantern Watch [4], Cat's Prowl [4]
\item Pick an opening Score: Borrowing Heist, Rescue \& Return, Underway Escort, Oath Moot, Threshold Intrusion, Knowledge Quest. On any 1, spend from Crumb \& Candle SB
\end{enumerate}

\section{Play Modes \& Scale}

\playmode{Scale Tags}
\textbf{Small:} mouse-to-hare sized folk; treat Infiltrate/Hide/Traverse (tight) as Position +1; treat Break/Force as Effect –1 vs big fixtures.

\textbf{Big:} human-scale and up; treat Reach as Effect +1 vs Small in open spaces; Position –1 in cramped venues (Beams/Pipes/Burrows) unless you Shrink the Scene.

\textbf{Harm Translation:} One scale larger → Effect +1; smaller → Effect –1 but may Call a Nook (convert into cover/Position once/scene instead of straight Harm).

\playmode{Mixed Company}
When Small and Big PCs act together, Assist may convert scale mismatch into Position +1 or DV –1 if they exploit venue tags (e.g., Small opens a vent; Big lifts a ledger).

\playmode{Shrink the Scene (optional)}
When Big PCs act in Small Realms, reframe threats as environmental (broom sweeps, lantern swings, cat shadows). Use Underway ladders with standard math.

\section{Aelaerem — The People of the Hearth}

\textbf{Homeland:} Amedell, south of the Aelinnel halls; neighbors to the Valewood and the Dales

The Aelaerem keep a country of gentle slopes and hedged lanes, cider barns and red-doored cottages. They are small in stature and large in memory: a people who bind promises with bread and lantern-light, who measure seasons by harvest masks and market bells. Hospitality is their public law, and the hidden law beneath it is older—hedge-counts, cup-marks, and the quiet attention of the Neighbors.

\subsection*{Hearth-Law \& Guest-Right}

A red door means bread, salt, and one safe night if you come honest; many Aelaerem homes keep a guest-loaf token specifically for travelers. Those who enter ``under bread and lantern'' find their next parley gentled—hospitality is a shield as much as a courtesy.

The Apple-Matron presides over such rites. In her circles, a feast can open barns and purses; to slight an invitation is to discover that every price has crept dear until proper amends are made. During high harvest she is as much magistrate as hostess, settling orchard feuds with a pour and a proverb.

\subsection*{Lantern-Law \& the Wardens}

Roads in Amedell are trimmed with little laws: stiles counted ``eight-and-one,'' village stones turned inward at dusk, door-nails blessed against trespass. Lantern-wardens keep the lanes bright and note which shadows are wrong, while Wold-Wardens speak hedge-law where crowns have no purchase. Offer oil or mend a lamp, and a lonely mile may find you under warded light.

Watch-geese are a common sight at mill and green—clever guardians whose alarms are taken as omen and ordinance both. Old millers swear their flocks know a stranger's tread before any hound.

\subsection*{The Quiet Powers (Neighbors)}

The Aelaerem speak of Neighbors who walk under hedges and barrows. Good manners keep them content: leave butter at the cup-marks, keep to the festival calendar, count the stiles aloud. Observance smooths the night; neglect draws the Hollow's attention, and then small things begin to go wrong—bells toll soft for no reason, red thread appears where you did not tie it, a door leads briefly somewhere it should not.

The Pale Shepherd is named in winter churches and lambing fields. Once, by clause and courtesy, a traveler may pass ``uncounted''—unseen by what tallies footfalls under the soil. Midwives and wardens trade his signs at the stile.

\subsection*{Seasons of Mask and Harvest}

Mummers keep stricter rules than any priest, and the Thresher-King's guard walks in red hoods when the fields demand order. Festivals turn the world sideways for a night: masks legitimate certain crossings after dark, private moots under the Oak settle quarrels, and an elder's blessing can make doors open that would not budge for coin.

Yet omens come with revels: scarecrows watch the lane, lanterns burn blue at the ford, chalk mazes fill with mist, and sometimes the Moot Oak bleeds sap the color of wine. Wise folk heed the bells, trade a secret for safe passage, and bury a mask at the crossroads when the dance goes wrong.

\subsection*{Trade, Craft, and Tokens}

Cider, perry, beeswax, and wool spin Amedell's economy. Orchard grafts from Mother's Orchard serve as living writs in rural parley; a mill-token buys rumor ground fine as flour; a shepherd's whistle can make dogs and door-bolts heed for a scene. Movers of unblessed pressings risk spoilage—favors salvage what coin cannot.

Aelaerem ``hearth magic'' is housekeeping writ large: red thread to bind promises, lantern-writ to hold the dark at bay, careful count and courtesy to keep the threshold sweet. It is less spell than system—the precise attention to seasons, doors, and debts.

\subsection*{Borders \& Tensions}

North, the Valewood's moods drift over the downs: moon-sap weather and dream-pollen make first night omens ring true. East and south, the Rivers and Dales host tight-lipped dalesfolk—stubborn parishers and millers—as often allies as rivals when tolls, water-rights, and mumming calendars misalign. When lantern-law frays or taboos are broken, figures step out of story: a Lantern Bailiff calling midnight moot, a Hollow Bride feeding on invitations, a scarecrow crowned by wrong knots.

\section{Aelinnel — The People of Sums}

\textbf{Homeland:} Gnomeholds in the hawthorn hills south of the Valewood; stone, bough, and bright things.

The Aelinnel keep to counted roads and measured courtesies. Their halls run like veins through granite and hawthorn; their bridges hum when tuned; their bargains arrive on two ledgers—what was said, and what was meant. To walk their country is to feel math underfoot: steps that are safer when even, doors that open to the right sequence, moonlight that prefers tidy logic.

\subsection*{The Hawthorn Halls}

South of the Valewood, the hills break into stone spurs and thorn-sheltered gullies. Paths are counted by antler-posts; tide-cut stairs descend to black sea-rifts; causeys of pale flags show themselves at dawn, at dusk—and whenever someone is counting aloud. Basalt organ cliffs breathe in primes; a moonwell stains coins green faster than names; dolmen stairs ring true if you tap the right interval.

Winged gnomes—fae-kin cousins with leaf-thin pinions—roost in hawthorn crowns and act as go-betweens to the courts. Gnomes as a whole resemble smaller, bright-eyed elves; the winged are simply closer to the other side of the hedge.

\subsection*{Law of Sums (Courtesy, Copper, Count)}

Aelinnel civility is exact: count or be counted. Speak your steps, breaths, or stitches and the land steadies—Position shifts safer for acts that exploit pattern or timing. Favor copper over iron before the courts; copper is polite, iron is an insult unless named or gilded. Recite a simple sequence when tension frays to cancel the first misstep in navigation or negotiation.

Courts of hawthorn require three clean courtesies: do not bring naked iron; speak in two ledgers; return what points the way (chalk, cord, antler). Keep them, and even thorns hold back; slight them, and arches close, time drifts, and conversations arrive folded and misaddressed.

\subsubsection*{Table Levers}
\begin{itemize}
\item \textbf{Counting Etiquette:} Once/scene, careful counting shifts Position +1 for patterned action.
\item \textbf{Copper over Iron:} Displaying copper/brass tools avoids the first fae-offense penalty in a scene.
\item \textbf{Two-Ledger Talk:} State what was said and what was meant to cancel the first social SB this scene.
\end{itemize}

\subsection*{People of Stone \& Bough}

You'll meet charcoal-burners who read omen by smoke hums; stone-singers whose low chords relax walls; forester-wardens who hammer copper nails where iron offends; goat-herds who measure danger in hoof-widths. Markets under living roofs sell truths wrapped for travel; weights and measures matter—producing a certified rod cancels the first jurisdiction or commerce snag.

Reputation echoes. Earn Hazel Favors by returning way-cords, restoring antler posts, paying tide-dues; spend to downgrade a glamour or geas once per leg. Masks and marks—Thorn-Courteous, Market-Square, Forester-Trusted—change which doors open and which clerks squint.

\subsection*{Courts, Hunts, and Gates}

Aelinnel bosses are etiquette engines and logic traps more than villains. The Lady of Thorns punishes breaches precisely and rewards perfect sequences; the Green Knight duels by paths and proofs; the Moonlit Ride offers one night's clemency if you name the right horn-count; the Green Gate demands exact change in truths.

When an Ace turns up in their stories, echo the motifs: antlers in shadow, petals that cut, a breathing tide, and shortcuts that insist on proper counting.

\subsection*{Tides, Ledgers, and Names}

Rivers and sea-caves carry their own arithmetic. Tide-reeves file plans before neap to earn a Tide Window; skip the ledger and the first Dolmis crossing suffers Wrong Tide. Barge seals that aren't maintained invite Wrong Hour. In green markets, you'll sometimes submit a said/meant receipt—try single-ledger haggling and the stalls mark you Exact Change Only, with memory or name demanded to square the sum.

\subsection*{Aelinnel Mood: Dark-Wonder}

Think ``math-bright Alice'' under a hawthorn sky: paths shorten for those who keep count and lengthen for the proud; petals fall like knives and settle into proofs; moonlight reveals hidden routes; antler-posts rearrange themselves when the land takes offense. Time miscounts and the sun arrives at the wrong hour with excellent logic.

\section{Mazereth — Deep Tunnelers (Aelinnel Subrace)}

\textbf{Stone-song, pressure shafts, and the etiquette of bearings \& bridges}

South of the Valewood the hills become ribbed stone and root-warrens, and the Aelinnel sub-culture called the \textbf{Mazereth} trace their lives along pressure lines and strata seams. Their halls feel like organs that breathe—pillars tuned to low chords, vents that whisper the time, chalk glyphs that count the load of a wall. But Mazereth are not only cave-keen: they tend weighhouses, keep ledgers for bridge trusts, and carry bearing-cords across market towns. They greet the hill with a palm on stone and a measure of weight; they greet the street with a counted price and a clean tally.

\subsection*{Stone-Law (Courtesy of Bearings)}
\begin{itemize}
\item \textbf{Count the Load.} Tap a support \emph{or} beam three times and listen; once/scene, a measured tap sequence grants \textbf{Position +1} to \emph{Traverse/Endure} in caves, bridges, or crowded structures.
\item \textbf{Copper Courtesy.} Copper is polite to stone and honest to labor. Presenting copper tools or a mason’s tally \emph{negates the first environment SB} tied to structure stress \emph{or} grants \textbf{DV --1} to parley with miners/masons/porters.
\item \textbf{Return the Chalk.} Anything that points the way (chalk, cord, placards) must be restored; doing so \textbf{cancels the first environment SB} this scene (dust, sway, signage loss).
\item \textbf{Two Bearings, One Path.} When asking directions, give both slope and seam (``down-two, east-by-vein'') \emph{or} price and span (``two-pence, one-arch''); this \textbf{cancels the first social SB} with wardens/foremen/brokers.
\end{itemize}

\subsection*{Heritage \& Build (as Aelinnel)}
Mazereth use \textbf{Aelinnel} baselines; apply the following package.

\subsubsection*{Traits}
\begin{itemize}
\item \textbf{Medium-Small}; \textbf{Strata Sense}: \emph{once/scene}, gain \textbf{+1 die} on a roll that reads load, route, or crowd-flow (Survey, Lore: Structures, Tactics: Maneuver) in stonework, bridges, or packed markets.
\item \textbf{Pressure Adaptation}: \emph{once/scene}, ignore the first \textbf{Desperate} caused by crush/squeeze \emph{or} stampede/crowd surge.
\end{itemize}

\subsubsection*{Gifts (choose 2)}
\begin{itemize}
\item \textbf{Deep Earthen Sense}: Vibration mapping; \textbf{Navigate DV --1} below ground or in heavy stoneworks.
\item \textbf{Tunnel Craft}: Declare a safe crawl, drain, or bypass \textbf{once/score}; GM sets a short clock to open it.
\item \textbf{Pressure Resistance}: Reduce crush/fall \textbf{Harm by one step} (to Fatigue if possible) \emph{once/scene}.
\item \textbf{Pattern Reading}: Beam/arch logic; \textbf{Disable/Repair DV --1} for structural tasks and braces.
\item \textbf{Bearing Factor}: In worksites/weighhouses, your clean ledger grants \textbf{DV --1} to negotiate labor, tolls, or right-of-way.
\item \textbf{Cord Etiquette}: Under a marked cord/line, you and guests gain \textbf{Position +1} on the opening exchange of parley or passage.
\end{itemize}

\subsubsection*{Complications (choose 1)}
\begin{itemize}
\item \textbf{Surface Sickness}: Open sky disorients—start scene at \textbf{Position --1} outdoors until you ground with a counted-breath ritual.
\item \textbf{Light Sensitivity}: First bright flare each scene triggers \textbf{Lantern Sway} (ranged penalties; GM tick or tag).
\item \textbf{Counting Compulsion}: Under stress, you stall to count supports; GM may \textbf{bank +1 SB} if you refuse, or you \textbf{mark 1 Fatigue} if you indulge.
\end{itemize}

\subsection*{Cultural Paths (pick one flavor; grants a situational edge)}
\begin{itemize}
\item \textbf{Tunnelwright}: When you \emph{prepare terrain}, your first brace/prop this scene adds a \textbf{Stability} tag (Position +1 vs collapse).
\item \textbf{Wayfactor}: At a bridge/ford/weighhouse, \textbf{DV --1} on the first toll/permit check each scene.
\item \textbf{Cord-Keeper}: While your bearing-cord is up, allies treating it as a guide gain \textbf{+1 die} to Traverse once/scene.
\end{itemize}

\subsection*{Places \& Omens}
\begin{itemize}
\item \textbf{Organ Cliffs:} basalt pipes that ``breathe'' in primes; wrong echoes mean a collapse front.
\item \textbf{Green Vein:} a copper seam used for oaths; promises sworn here bind extra tight.
\item \textbf{Sump of Coins:} a moonwell that stains copper; a safe ford when counted in fives.
\item \textbf{Foothill Weighhouse:} ledgers, cords, and posts; disputes settled by span and load before coin.
\item \textbf{Omens:} dust that falls upward; pebbles ticking in sequence; roots that beat like a slow heart.
\end{itemize}

\subsection*{Faces of Stone}
\begin{itemize}
\item \textbf{Bearing-Reeve Bronze:} audits props and pistons; will trade three safe spans for one good ledger.
\item \textbf{Chalk-Warden Ness:} scolds for stolen marks; swaps you a shortcut if you return a cord intact.
\item \textbf{Goat-Singer Ivrin:} her herd keeps time on cliff paths; she knows which ledges forgive missteps.
\item \textbf{Span-Broker Telma:} keeps the toll honest; buys you Position +1 in parley if your bearings are clean.
\end{itemize}

\subsection*{Strings \& Tokens}
seam-chalk, bearing cord, copper nail ring, pressure bead (changes hue under load), weighhouse scrip, mason’s tally.

\subsection*{Clocks \& Hazards}
\begin{itemize}
\item \textbf{Stone Attention [6]} (the hill ``notices'' you), \textbf{Squeeze Front [4]}, \textbf{Bad Air [4]}, \textbf{Vein Quarrel [4]}, \textbf{Rain-Swell [4]} (flooded cuts), \textbf{Market Panic [4]} (crowd crush).
\item \textbf{SB Menu (Underground/Works \& Markets):} Dust Plume (visibility), Lantern Sway (glare), Slick Calcite (lose Position), Root Snare (pursuit begins), Sour Air (Fatigue), Toll Dispute (parley Position --1), Span Creak (start Stability clock).
\end{itemize}

\subsection*{Play Hooks}
\begin{enumerate}
\item \textbf{Kiln in Thirds:} A charcoal draft hums off-pattern; fix the count before the Green Vein judges the debt.
\item \textbf{The Swallowing Span:} A climbing route ``shortens'' for liars—prove truth with two bearings.
\item \textbf{The Stolen Chalk:} A market steals way-marks for profit; restore cords or Stone Attention wakes a sleeping fault.
\item \textbf{Bridge Trust Audit:} The toll rose without span repair; expose the false ledger or brace the arch mid-parley.
\end{enumerate}

\subsection*{Using Mazereth at the Table}
Lean on \textbf{Strata Sense} for DV breaks in stonework \emph{and} tight urban spaces. Spend \textbf{bearing cords} as Strings for \textbf{Traverse +1 Position}, and treat \textbf{copper courtesy} as a one-shot ward against structure/environment SB \emph{or} a DV nudge with labor strata. Above ground, \textbf{Surface Sickness} is a soft tax the first minute of a scene—ground with counted breath to clear it. In banner play, a Mazereth Wayfactor smooths crossings; in dungeon play, a Tunnelwright quietly prevents the TPK.

\subsection*{Threshold Culture Stats}
\textbf{Realm:} underways, bridges, weighhouses, pressure shafts \\
\textbf{Traits:} Medium-Small; Strata Sense; Pressure Adaptation \\
\textbf{Gifts (choose 2):} Deep Earthen Sense; Tunnel Craft; Pressure Resistance; Pattern Reading; Bearing Factor; Cord Etiquette \\
\textbf{Complications (choose 1):} Surface Sickness; Light Sensitivity; Counting Compulsion \\
\textbf{Bond:} Earthbound \emph{or} Ledgerwise \\
\textbf{Cultural Tags:} Stone-Law; Copper Courtesy; Pressure Adaptation; Tunnel Craft; Weighhouse Savvy

\section{Umbral Kin — Ghost-Walkers of the Aelaerem}

\textbf{Twilight courtesies, reflection roads, and the price of borrowed shade}

The Umbral Kin are Aelaerem cousins who learned to live \emph{with} their shadows rather than flee them. They rent shade like others rent rooms, keep glazier ledgers alongside family bibles, and treat ward-lamps as municipal judges: name them, appease them, or slip between their gazes. In markets they stand where awnings overlap; on roads they walk the cool edge; at home they hang lamp hoods before a greeting.

\subsection*{Shadow-Law (Etiquette of Edges)}
\begin{itemize}
\item \textbf{Speak on the Edge.} Begin parley half-in, half-out of shadow (doorway, awning, lintel). \emph{Once/scene}, gain \textbf{Position +1} on \textbf{Sway/Diplomacy} for the opening exchange.
\item \textbf{No Naked Lights.} Unshaded flame is insult and provocation. A covered lantern (hood, shade, or hand-screen) \emph{negates the first ward-lamp penalty or flare} this scene.
\item \textbf{Mirror Oath.} Swear with faces side-by-side in a window or polished plate. \emph{Once/session}, a mirror-oath counts as a \textbf{Witness} for disputes, contracts, or safe-conduct.
\item \textbf{Name the Shade.} At dusk, name your shadow and touch heel to heel. \emph{This leg}, cancel the first \textbf{Memory Echo} complication that would expose you.
\end{itemize}

\subsection*{Heritage \& Build (as Aelaerem)}
Umbral Kin use \textbf{Aelaerem} baselines; apply this package.

\subsubsection*{Traits}
\begin{itemize}
\item \textbf{Small–Medium}; \textbf{Shadow Blend}: In dim/dappled light you count as having \emph{soft cover}; \emph{once/scene} gain \textbf{+1 die} to \textbf{Stealth/Subterfuge} when entering or leaving a shadowed zone.
\item \textbf{Reflection Reading}: \emph{Once/scene}, study glass or still water to ask 1: “Who’s watching?” “What was just done here?” “Where is the lamp’s blind side?”
\end{itemize}

\subsubsection*{Gifts (choose 2)}
\begin{itemize}
\item \textbf{Shadow Step}: \emph{Near-range} hop between contiguous shadows you can see. Costs \textbf{1 Fatigue} if crossing open light; otherwise free.
\item \textbf{Memory Echo}: Read emotional residue from a cool surface; on success gain \textbf{Clue +1} (lead clock +1) but GM may tick \textbf{Ward-Lamp Attention} if you linger.
\item \textbf{Light Bending}: \emph{Once/scene}, stage the light (hood, sheet, mirrored dish) to gain \textbf{Position +1} for a vanish/misdirect.
\item \textbf{Partial Phase}: Squeeze through a bar, grille, or door-gap by marking \textbf{1 Fatigue}; if you would take \textbf{Harm 1}, convert it to \textbf{1 Fatigue} instead.
\item \textbf{Lantern Cant}: You know the lamplighters’ marks; \textbf{DV --1} to navigate after curfew or to spoof a lamp’s patrol route.
\end{itemize}

\subsubsection*{Complications (choose 1)}
\begin{itemize}
\item \textbf{Bleach-Sick}: Harsh illumination (no shade) starts you at \textbf{Position --1} until you create or reach cover.
\item \textbf{Ward-Lamp Attention}: Named streetlamps \emph{remember} you; the GM may tick \textbf{Ward-Lamp Attention [6]} when you commit a notable act in view.
\item \textbf{Two Shadows}: At noon or under crossed lights you cast a double; the first \textbf{Subterfuge} each scene is \textbf{DV +1} unless you dim one source.
\end{itemize}

\subsection*{Guild Paths (pick one flavor; adds a situational edge)}
\begin{itemize}
\item \textbf{Night Courier}: When you \emph{carry under writ}, your first \textbf{Traverse/Stealth} this leg gains \textbf{DV --1}.
\item \textbf{Glazier of Quiet}: While you work a pane or mirror “set for silence,” allies in Near count as \textbf{Hidden} from casual watchers until they act.
\item \textbf{Parasol Marshal}: In zones you’ve flagged with shade markers, your side starts \textbf{Position +1} against ward patrols and informers.
\end{itemize}

\subsection*{Places \& Omens}
\begin{itemize}
\item \textbf{Reflection Lanes}: Arcades where one may “walk the pane” if step and breath match the double.
\item \textbf{Blue Hours}: Docks where time comes twice; safer to leave before names are called.
\item \textbf{Umbrelle Market}: Awnings upon awnings; secrets priced in shade.
\item \textbf{Omens}: Lamps that flare without wind; a second shadow arriving late; reflections that refuse to smile.
\end{itemize}

\subsection*{Faces of Twilight}
\begin{itemize}
\item \textbf{Parasol-Matron Sevi}: Rents legal shade; will cover a fugitive for one favor paid at dawn.
\item \textbf{Lamplighter Orr}: Turns ward-lamps to “rest” on a name; looks away if you speak his grandmother’s lullaby.
\item \textbf{Glazier Moth}: Sets “forgetful panes” that blunt Memory Echo for a week (and bill you twice).
\end{itemize}

\subsection*{Strings \& Tokens}
parasol writ; shade-marker ribbon; mirror-shard oath; lamp hood; lamplighter’s chalk.

\subsection*{Clocks \& Hazards}
\begin{itemize}
\item \textbf{Ward-Lamp Attention [6]} (streetlights learn you), \textbf{Glare Front [4]}, \textbf{Reflection Stalker [4]}, \textbf{Blue Hour Drift [4]}, \textbf{Curfew Sweep [4]}.
\item \textbf{SB Menu (Shadow)}: Lamp Flare (scene brightens); Second Shadow (a pursuer appears); Glass Whisper (old rumor resurfaces); Dawn Soon (time compresses); Hood Snag (lose cover).
\end{itemize}

\subsection*{Play Hooks}
\begin{enumerate}
\item \textbf{The Lamp That Watches}: A new ward-lamp memorizes faces; steal its name-plate before it “testifies.”
\item \textbf{Blue Hour Heist}: A vault exists twice at twilight; rob the reflection while your doubles distract the guards.
\item \textbf{Parasol Tax}: City watch bans uncovered candles; smuggling lamp hoods becomes a civic rebellion.
\item \textbf{The Second Shadow}: Someone wears your late shadow in the markets; retrieve it before your debts follow.
\end{enumerate}

\subsection*{Using Umbral Kin at the Table}
Treat covered light as a one-shot \textbf{DV --1} vs.\ ward systems; \textbf{mirror shards} can invoke \textbf{Witness} per \emph{Mirror Oath}. Trade \textbf{Memory Echo} for fast clues at the risk of \textbf{Ward-Lamp Attention} ticking. In bright scenes, expect an initial \textbf{Position --1} unless you fix the light; \textbf{Light Bending} or a \textbf{lamp hood} clears it. \textbf{Shadow Step} is Near-range and demands real, contiguous shade—alleys, awnings, cart undersides—keeping the talent street-grounded rather than supernatural flight.

\subsection*{Threshold Culture Stats}
\textbf{Realm:} shadows, reflections, twilight \\
\textbf{Traits:} Small–Medium; Shadow Blend; Reflection Reading \\
\textbf{Gifts (choose 2):} Shadow Step; Memory Echo; Light Bending; Partial Phase; Lantern Cant \\
\textbf{Complications (choose 1):} Bleach-Sick; Ward-Lamp Attention; Two Shadows \\
\textbf{Bond:} Shadowtouched \\
\textbf{Cultural Tags:} Shadow-Law; Mirror-Oaths; Light-Bending; Night Trade

\section{Archive Keepers — Knowledge-Bound Aelinnel}

\textbf{Index courts, pattern webs, and the peril of perfect memory}

The Archive Keepers are Aelinnel cousins who hoard what others forget. They curate hidden stacks, stitch ledgers to ledgers, and build “pattern webs” that let a fact walk from one book to another. Their manners are footnotes; their quarrels are citations; their pride is a shelf that never lies.

\subsection*{Index Law (Etiquette of Records)}
\begin{itemize}
\item \textbf{Cite or be Cited.} Offer a source when making a claim; \emph{once/scene}, a clean citation grants \textbf{DV --1} to \textbf{Plan/Research/Petition}.
\item \textbf{Return the Page.} Replace a card where it belongs to cancel the first \emph{Index Entropy} complication this scene.
\item \textbf{Two Copies, Two Locks.} A binding writ needs twin copies kept apart; declaring both voids the first \emph{forgery SB} this scene.
\item \textbf{Right of Errata.} Admit a small error to avoid a larger duel; convert a looming \emph{Sting/Inspection} into a lesser \textbf{Public Correction [4]}.
\end{itemize}

\subsection*{Traits, Gifts, Complications}
\paragraph{Traits}
\begin{itemize}
\item \textbf{Small; Stacks-Hardened.} Treat ladders, rolling stools, and shelf-rungs as \emph{normal ground}; ignore the first \emph{Climb} penalty in libraries and scriptoriums.
\item \textbf{Index Sense.} \emph{Once/scene}, ask a pattern question: “What’s missing?” “What contradicts itself?” or “Where would this belong?”
\item \textbf{Shelf-Strider.} You can “squeeze” through densely packed aisles; tight passages count as one size wider for you.
\end{itemize}

\paragraph{Gifts (pick 2)}
\begin{itemize}
\item \textbf{Knowledge Tap.} You keep a hidden index of topics; gain \textbf{DV --1} on \textbf{Lore/Investigation} when you can cite a plausible prior source.
\item \textbf{Memory Palace.} Store a scene perfectly; later “quote” it to gain \textbf{Clue +1} or \textbf{Position +1} to disprove a lie (once/scene).
\item \textbf{Information Bridge.} Join two records with cross-refs; allies pursuing that lead gain \textbf{Clue +1} (lead clock +1) this scene.
\item \textbf{Threshold Architecture.} Lay a subtle through-route with ribbons, tabs, and marks; \emph{once/session} your party gets \textbf{Traverse DV --1} through stacks/archives.
\end{itemize}

\paragraph{Complications (choose 1)}
\begin{itemize}
\item \textbf{Overload.} Too many inputs at once mark \textbf{1 Fatigue} and impose \textbf{--1 die} on your next \textbf{Insight/Notice}.
\item \textbf{Brittle Focus.} Blunt impacts shake you; the first \textbf{Harm 1 (blunt)} in a scene converts to \textbf{1 Fatigue}, then resolve Harm normally after.
\end{itemize}

\subsection*{Places \& Omens}
\begin{itemize}
\item \textbf{Stack Nine-Between:} a shelf that appears where two libraries agree; reachable by whispering the same title.
\item \textbf{Red Thread Court:} disputes settled by re-stitching a ledger; scissors are weapons here.
\item \textbf{Binder’s Bridge:} a paper-arc you can cross if your name is spelled correctly.
\item \textbf{Omens:} margins writing back; page numbers that skip a friend’s birthday; a shelf that turns its own ladder.
\end{itemize}

\subsection*{Faces of the Index}
\begin{itemize}
\item \textbf{Card-Clerk Lumo:} knows where the lost catalog sleeps; demands an errata signed in copper.
\item \textbf{Binder-Monk Pera:} can unmake a contract by re-sewing it; always asks for a memetic tithe.
\item \textbf{Ink-Archivist Thriss:} drinks spoiled ink to taste lies; offers detox at a price.
\end{itemize}

\subsection*{Strings \& Tokens}
errata slip; card-catalog tag; binder’s thread; reference stamp (acts as \emph{Witness} for documents once).

\subsection*{Clocks \& Hazards}
\begin{itemize}
\item \textbf{Index Entropy [6]} (systems decay), \textbf{Leak of Names [4]}, \textbf{Plagiarism Duel [4]}, \textbf{Redaction Front [6]}.
\item \textbf{SB Menu (Stacks):} Shelf Creep (path shifts), Loose Leaf (evidence scatters), Citation War (resource tax), Spilled Ink (Visibility rises).
\end{itemize}

\subsection*{Play Hooks}
\begin{enumerate}
\item \textbf{The Missing Reference.} A city writ cites a book that never existed; build an Information Bridge to the nearest true source before the court rules from nothing.
\item \textbf{Errata Night.} Archivists purge lies; protect a friend’s page from redaction—or convince the index it always belonged.
\item \textbf{The Folded Hall.} Two stacks overlap; map the “between” before a Leak of Names turns to amnesia.
\end{enumerate}

\subsection*{Using Archive Keepers at the Table}
Spend \textbf{errata slips} to downgrade \emph{Sting/Inspection} into \emph{Public Correction [4]}; deploy \textbf{Memory Palace} to capture a scene and later “quote” it for \textbf{Clue +1} or a decisive contradiction. \textbf{Threshold Architecture} grants the whole party a subtle shortcut \emph{once/session}. Track \textbf{Overload} as Fatigue when too many inputs land at once; let \textbf{Index Sense} steer you to the next honest place a fact might live.

\subsection*{Threshold Culture Stats}
\textbf{Realm:} information spaces, hidden stacks, pattern webs \\
\textbf{Traits:} Small; Stacks-Hardened; Index Sense; Shelf-Strider \\
\textbf{Gifts (choose 2):} Knowledge Tap; Memory Palace; Information Bridge; Threshold Architecture \\
\textbf{Complications (choose 1):} Overload; Brittle Focus \\
\textbf{Bond:} Knowledgebound \\
\textbf{Cultural Tags:} Index-Law; Memory Palace; Information Bridge; Stacks-Hardened

\section{Cross-Module Integration (all three)}
\begin{itemize}
\item \textbf{Violets \& Stone:} Mazereth stabilize vaults; Umbral run parasol markets and mirror-oaths; Archive Keepers arbitrate contracts with Red Thread Courts.
\item \textbf{Shadows \& Steel:} Shadow-Step smuggling; seam-chalk routes under districts; errata slips to soften crackdowns.
\item \textbf{Caravans/Wilderness:} Mazereth cut safe passes; Umbral guide night legs; Archive Keepers preserve manifests and ward-maps.
\item \textbf{Political Intrigue:} Copper-witness oaths, mirror-witness oaths, and dual-ledger writs recognized as auxiliary evidence.
\item \textbf{Psionics:} Information Bridges as noetic pathways; Deep Sense as tremor-sense; Umbral light-bending counters ward-storm glare.
\end{itemize}

Among the Aelinnel, sub-cultures are not splinters but lenses. The hill, the edge, the index—each teaches a way to keep to courtesy when the world tilts.

\subsection*{Omens \& Oddities (roll or pick)}
\begin{itemize}
\item \textbf{Spade:} Tide-rift steps with votive nails keeping count—safe when the nails agree.
\item \textbf{Heart:} A hedge-witch prices cures in unlesses; pay with a clause you can keep.
\item \textbf{Club:} Hawthorn arch closes behind you; your footfalls no longer match your steps.
\item \textbf{Diamond:} Hazel token allows one cut-free hedge crossing; the scratch you didn't get remembers you kindly.
\end{itemize}

\subsection*{Winged Kin (Faekin)}

Some Aelinnel are born to the boughs, pinions veined like leaves. They keep the hawthorn courtesies as breath: no naked iron beyond an arch, always return what points the way, speak debts in daylight so leaves can hear. They are the surest messengers between keep and court—and the quickest to warn when counting turns wrong.

\subsection*{Play Hooks}
\begin{enumerate}
\item \textbf{The Gate Wants Change:} The Green Gate opens at the wrong hour; pay a memory that fits the posted proof or roads rewire across your path.
\item \textbf{Kiln in Thirds:} A charcoal clan's draught hums off-pattern; smoke omens say a geas is miscounted. Set it straight before the court tithe arrives in flowers and warrants.
\item \textbf{Two Ledgers, One Lie:} A broker sells meanings at a green market; catch them in a one-ledger statement or produce a chilled oath-bead to void their trick.
\item \textbf{Antler-Posts Out of Order:} Posts have rearranged a stag road; foresters whisper that someone mocked the count. Restore the sequence before the Wild Hunt claims the short path.
\end{enumerate}

\subsection*{Using Aelinnel at the Table (quick rules)}
\begin{itemize}
\item \textbf{Count:} Once/scene, careful counting shifts Position +1 for patterned actions.
\item \textbf{Copper:} In fae-facing scenes, copper/brass tools negate the first offense; iron escalates.
\item \textbf{Said/Meant:} Declare both to cancel the first social SB; refuse, and the next bargain wants collateral (memory/name).
\item \textbf{Hazel Favors:} Earn by respectful upkeep of way-things; spend to downgrade a glamour/geas once per leg.
\item \textbf{Region SB (examples):} Hawthorn arch closes; petals cut like blades; moonlight reveals a hidden path.
\end{itemize}

In Aelinnel, numbers have manners. Keep the count, favor copper, return what points the way—and the hills will do the same for you.

\subsection*{Threshold Culture Stats}
\textbf{Realm:} liminal edges, doorways, thresholds \\
\textbf{Traits:} Small; +1 Position in tight spaces; +1 die to Hide/Notice minute details \\
\textbf{Gifts (choose 2):} Threshold Walking, Object Bond, Minute Craft, Liminal Sight \\
\textbf{Complications:} overlooked by surface powers; fragile in mass conflict \\
\textbf{Bond:} Liminbound \\
\textbf{Cultural Tags:} Law of Sums, Copper Courtesy, Two-Ledger Talk, Hazel Favors

\end{section}


\subsection*{Faces of the Hearth}
\begin{itemize}
\item \textbf{Apple-Matron:} Power sits where she pours; feast and precedent are one art.
\item \textbf{Thresher-King:} A title that moves but never leaves; his guard opens doors none other can.
\item \textbf{Lantern-Warden:} Trims lamps, reads shadows; escorts earned with oil and respect.
\item \textbf{Mummers' Captain:} Lawful masks and after-dark crossings on feast-days—anger them and gates close at dusk.
\item \textbf{Miller \& Watch-Geese:} Local alarm and local court in one flock and wheel.
\item \textbf{The Pale Shepherd:} Once, pass uncounted; always, pay the lane its due.
\end{itemize}

\subsection*{Play Hooks in Amedell}
\begin{itemize}
\item \textbf{The Guest-Loaf Forgers:} Counterfeit tokens sour trust from Amedell to the Way of Silk; track the mummers' license that legitimized the forgeries, or every night road turns hostile.
\item \textbf{The Moot Oak Bleeds:} A festival omen turns ugly; mediate before the Hollow answers the knives.
\item \textbf{Lanterns at the Ford:} Blue flames demand a toll ``more than coin''; learn what the Neighbors want this season.
\item \textbf{Apple-Matron's Summons:} Refuse her table and prices double; accept, and you inherit a feud with the dalesmen downstream.
\end{itemize}

\subsection*{Using the Aelaerem at the Table}

Treat bread, lamps, and counting as levers: produce guest-loaf and a lit lantern to soften a risky social exchange; observe courtesies to cancel the first strange complication in a scene; pause at a stile and listen to ask what the Hollow wants right now. Even in danger, a shepherd's whistle or mill-token can turn fiction your way—Amedell rewards those who mind the little laws.

In Amedell, the hearth is a treaty, the lane a ledger, and the night a neighbor. Keep the count, pour the cup, and the country will keep you.

\subsection*{Threshold Culture Stats}
\textbf{Realm:} liminal edges, doorways, thresholds \\
\textbf{Traits:} Small; +1 Position in tight spaces; +1 die to Hide/Notice minute details \\
\textbf{Gifts (choose 2):} Threshold Walking, Object Bond, Minute Craft, Liminal Sight \\
\textbf{Complications:} overlooked by surface powers; fragile in mass conflict \\
\textbf{Bond:} Liminbound \\
\textbf{Cultural Tags:} Hearth-Law, Lantern-Ward, Guest-Right, Festival Calendar

\end{section}

\culture{Mazereth — Deep Tunnelers}
\textbf{Realm:} underground networks, root warrens, pressure shafts  
\textbf{Traits:} Medium-Small; +1 die underground; Pressure Adaptation  
\gift{Gifts}{Deep Earthen Sense, Tunnel Craft, Pressure Resistance, Pattern Reading}  
\complication{Complications}{surface sickness; light sensitivity (SB: Lantern Sway)}

\culture{Umbral Kin — Shadow-Adjacent}
\textbf{Realm:} shadows, reflections, twilight  
\textbf{Traits:} Variable scale (partial phase); Shadow Blend, Reflection Reading  
\gift{Gifts}{Shadow Step, Memory Echo, Light Bending, Partial Phase}  
\complication{Complications}{instability in bright light; attention from ward-lamps}

\culture{Archive Keepers — Knowledge-Bound}
\textbf{Realm:} information spaces, hidden stacks, pattern webs  
\textbf{Traits:} Micro↔Small scale shift; Information Sense, Scale Shift  
\gift{Gifts}{Knowledge Tap, Memory Palace, Information Bridge, Threshold Architecture}  
\complication{Complications}{overload sensitivity; bodily fragility}

\section{Nook/Clan Sheet (Template)}

\begin{center}
\begin{tabularx}{\textwidth}{|X|}
\hline
\textbf{[NOOK / CLAN NAME]} \\
Culture: Aelaerem / Mazereth / Umbral Kin / Archive Keepers \\
Venue: Pantry • Cellar • Beams • Hedge • Roof • Shrine \\
Tags (2–3): Gnaw-Holes • Ferrier Line • Secret Hooks • Crumb-Bank • Ward Pins • Lantern Lookouts • Root-Road \\
Strings (2–3): door-charm bead • pantry tithe token • ferrier token • under-map • cat's truce knot • witness broom badge \\
Tracks: \\
- Repute [6] (standing among threshold folk) \\
- Notice [6] (ambient risk from Bigfolk/Predators) \\
- Threshold Sense [6] (opt., cultural) \\
- Visibility [6] (opt., public awareness) \\
Bank (tiny economy); Allies/Rivals; Nooks/Routes \\
Gifts (2): \_\_\_\_\_\_\_\_  Realm Bond (1): \_\_\_\_\_\_\_\_ \\
Hidden Knowledge (1): \_\_\_\_\_\_\_\_  Surface Connection (1): \_\_\_\_\_\_\_\_ \\
\\
\textbf{Nook Benefits} (choose 1): Ferrier Line • Ward Pins • Lantern Lookouts \\
\hline
\end{tabularx}
\end{center}

\section{Threshold Tracks (optional layer)}

\thresholdtrack{Threshold Sense}{6} — connection to liminal spaces \\
0–2: disconnected (Gifts weaken; Position –1 on realm moves) \\
3–4: steady access \\
5–6: deep current (Gifts enhanced) but visible to entities (Threshold Tension +1 when you flex)

\thresholdtrack{Visibility}{6} — how noticeable you are to surface powers \\
0–2: overlooked (Stealth +1 die in crowds) \\
3–4: normal \\
5–6: high profile (authorities begin Lantern Watch [4])

\thresholdtrack{Realm Stability}{6} — health of your home realm \\
0–2: unstable (realm sickness risk) \\
3–4: stable \\
5–6: overconnected (pulled toward realm; DV +1 to resist)

\thresholdtrack{Threshold Tension}{4} — pressure with guardians/veils. At full: crisis, closing, or entity attention

\textbf{Mapping} (low-overhead): If you prefer fewer dials, treat Visibility ↔ Notice, and let Realm Stability/Threshold Tension be represented by relevant Front clocks

\section{Gifts \& Abilities}

\giftcategory{Movement \& Scale}
Size Shift, Threshold Walking, Realm Step, Scale Mastery

\giftcategory{Perception \& Knowledge}
Liminal Sight, Micro-Sense, Pattern Reading, Memory Echo

\giftcategory{Craft \& Creation}
Minute Work, Object Bond, Realm Infusion, Threshold Architecture

\textbf{Gifts are narrative permissions; most grant Position, DV, or Effect shifts once per scene/leg as listed}

\section{Realm Bonds \& Complications}

\subsection*{Bonds (pick 1):}
Earthbound, Shadowtouched, Liminbound, Knowledgebound

\subsection*{Common Complications:}
Scale Shock, Realm Sickness, Oversight, Threshold Attention, Cycle Dependency

\section{Threshold Maps \& Venues}

\subsection*{Nodes (pick 4–6):}
Pantry Court • Beams \& Rafters • Gutter Run • Hedge Road • Cellar City • Shrine in the Mould • Clock-Room • Mill-Loft • Postern Stairs • Under-Dock Piles

\subsection*{Local Clocks:}
Bigfolk Stir [6], Lantern Watch [4], Cat's Prowl [4]

\section{Ladders \& Procedures}

\subsection*{Underway Travel Ladders}
Foot/Climb: DV 2 hooks • 3 beams • 4 pipes • 5 storm vent \\
Leaf-Ferry: DV 2 basin • 3 drain • 4 gutter run • 5 downspout

\subsection*{Venue Effects}
Beams \& Rafters: Infiltrate/Traverse Position +1 (Small); Big suffer Position –1 \\
Pantry Court: Market Position +1 with tithe token; on 1, Notice +1

\section{Hidden Knowledge \& Surface Connections}

\subsection*{Hidden Knowledge Types:}
\begin{itemize}[leftmargin=*]
\item Secret route map
\item Household weakness
\item Predator schedule
\item Realm boundary location
\item Ancient compact terms
\item Craft technique
\item Patron whisper
\end{itemize}

\subsection*{Surface Connections:}
\begin{itemize}[leftmargin=*]
\item Human patron
\item Guild contact
\item Merchant relationship
\item Authority blind spot
\item Family tie
\item Debt owed/owed to you
\end{itemize}

\section{Integration with Existing Modules}

\subsection*{Violets \& Stone:}
Hidden communities in district underways, secret services, threshold points in ancient buildings

\subsection*{Wilderness:}
Natural threshold spaces, small folk as guides, realm connections to natural features

\subsection*{Caravans:}
Specialized small-scale trade goods, hidden route knowledge, realm-based preservation

\subsection*{Political Intrigue:}
Secret advisors, hidden faction members, threshold-based communication networks

\section{Adventure Frameworks}

\subsection*{Score Types:}
\begin{itemize}[leftmargin=*]
\item Borrowing Heist
\item Rescue \& Return  
\item Underway Escort
\item Oath Moot
\item Threshold Intrusion
\item Knowledge Quest
\end{itemize}

\subsection*{New Score Types:}
\begin{itemize}[leftmargin=*]
\item \textbf{Realm Crisis:} Home realm becoming unstable
\item \textbf{Boundary War:} Conflict between threshold entities
\item \textbf{Predator Hunt:} Dangerous entity threatening communities
\item \textbf{Cycle Disruption:} Natural rhythms disturbed
\end{itemize}

\subsection*{Crumb \& Candle SB:}
Environmental complications, bigfolk attention, predator signs, realm instability

\section{Generators \& Tables}

\subsection*{Threshold Venues (d12):}
\begin{enumerate}
\item Pantry Court  
\item Beams \& Rafters  
\item Gutter Run  
\item Hedge Road  
\item Cellar City  
\item Shrine in Mould
\item Clock-Room  
\item Mill-Loft  
\item Postern Stairs  
\item Under-Dock Piles  
\item Root-Warren  
\item Shadow Gap
\end{enumerate}

\subsection*{Small Folk Complications (d10):}
\begin{enumerate}
\item Broom sweep  
\item Lantern sway  
\item Cat prowl  
\item Ferrier dispute  
\item Ward draft
\item Bigfolk stir  
\item Cycle shift  
\item Predator scent  
\item Realm tremor  
\item Compact breach
\end{enumerate}

\section{GM Toolkit}
% No extra packages required; compact, printable.

\section{Cultural Deep-Dives}

\paragraph*{Aelaerem Society:}
The Aelaerem organize into hearth-clans centered around craft specialties—coopers, weavers, brewers, and woodcarvers who maintain both trade secrets and seasonal rituals. Each clan keeps a memory-weaver who tracks debts, favors, and omens across generations through story-braids and token-craft. Information flows through guest-right networks and festival moots, where the Apple-Matron mediates disputes through feast-law and proverb-wrights. Threshold guardians like the Pale Shepherd are honored with offerings and careful observance; slight them and household luck turns sour. Clan tokens—bread-seals, lantern-writs, and red-thread knots—serve as both currency and covenant, creating an economy where reputation weighs more than coin.
\begin{tabular}{p{0.27\linewidth} p{0.68\linewidth}}
\textbf{Aelaerem — People of the Hearth} &
\textbf{Levers:} Bread \& Salt = Position +1 once/scene; Broom Witness = Oath [4–6]; Iron-Lace \& Red Thread = +1 Effect vs. compulsion.\\
& \textbf{Strings:} guest-loaf, broom badge, hedge measure. \quad
\textbf{Tensions:} Barrow Stirring [6], Gloam Choir [6].\\[0.5em]

\paragraph*{Mazereth Engineering:}
Mazereth architecture operates on pressure harmonics, where pillars, arches, and support beams are tuned to resonate at frequencies that strengthen stone rather than strain it. Master builders read stress-lines in rock like maps, placing copper nodes at compression points and iron dampeners at tension zones. Their deep history is preserved in geological strata—glyphs carved into cave walls that tell stories through the ages as erosion reveals new layers. Tunnel network politics revolve around bearing-rights and seam-access; clans feud over prime ventilation shafts and mineral veins, settling disputes through measured duels where contestants must navigate obstacle courses while maintaining perfect step-counts and load-balances.
\textbf{Mazereth — Deep Tunnelers} &
\textbf{Levers:} Deep Earthen Sense = Navigate DV –1 underground; Copper Nails avoid first “hill offense”; Count the Load = Position +1 on Traverse/Endure.\\
& \textbf{Strings:} seam-chalk, bearing cord, pressure bead. \quad
\textbf{Tensions:} Stone Attention [6], Bad Air [4].\\[0.5em]

\paragraph*{Umbral Kin Memory:}
Umbral Kin share memories through dream-reflection, where important experiences are "polished" until they gleam bright enough to be caught in glass surfaces and still water. Reflection communication allows instant long-distance contact—messages passed from mirror to mirror across a city's windows can traverse miles in moments. Shadow realm territorial disputes arise when one group's reflections begin overlapping another's; boundaries are negotiated through careful light-management and shadow-casting protocols. Memory Echo allows Umbral to read emotional residue from surfaces, making them excellent investigators but also vulnerable to psychic pollution from traumatic events embedded in walls and floors.
\textbf{Umbral Kin — Shadow-Adjacent} &
\textbf{Levers:} Covered Light avoids first ward-lamp penalty; Mirror Oath = Witness; Shadow Step (short hop) \& Light Bending = Position +1 to vanish.\\
& \textbf{Strings:} parasol writ, mirror shard, lamp hood. \quad
\textbf{Tensions:} Ward-Lamp Attention [6], Blue Hour Drift [4].\\[0.5em]

\paragrap*{Archive Keeper Knowledge:}
Archive Keepers store information in pattern-webs—three-dimensional mental constructs that connect related facts across multiple dimensions of meaning. Scale-shifting libraries exist in probability space, where books grow larger or smaller depending on the reader's need for detail; a general overview might require accessing a text the size of a grain of sand, while deep research could mean entering a tome the size of a cathedral. Knowledge debt systems track intellectual borrowing—when an Archive Keeper accesses another's memory palace or information bridge, they owe a tithe of new knowledge or a favor in return. The most dangerous debts are memetic—ideas that carry obligations or change the borrower's thinking patterns.
\textbf{Archive Keepers — Knowledge-Bound} &
\textbf{Levers:} Cite or be Cited = DV –1 (Plan/Research); Errata Slip downgrades Sting→Public Correction [4]; Threshold Architecture = Traverse DV –1 (once/session).\\
& \textbf{Strings:} errata slip, binder’s thread, catalog tag. \quad
\textbf{Tensions:} Index Entropy [6], Redaction [6].\\
\end{tabular}

\subsection*{Portrayal Notes:}
\begin{itemize}[leftmargin=*]
\item \textbf{Emphasize capability over diminutive stereotypes:} These cultures are not weak or helpless despite their size—they are experts in their environments with sophisticated knowledge systems. Show them as competent specialists who solve problems through precision, timing, and deep understanding of their realms rather than brute force.
\item \textbf{Show complex societies with internal diversity:} Each culture has factions, dissenters, reformers, and outliers. The Aelaerem have those who reject hospitality laws, Mazereth include those who favor iron over copper, Umbral Kin struggle with those who crave bright light, and Archive Keepers debate the ethics of memory manipulation. Not every member adheres strictly to cultural norms.
\item \textbf{Use hospitality and reciprocity as mechanical hooks:} Treat guest-right, copper courtesy, mirror-oaths, and citation protocols as more than flavor—they're tactical resources. Characters can spend cultural tokens for mechanical benefits, but violating these codes creates escalating complications that drive scenes forward.
\item \textbf{Balance hidden advantages with genuine vulnerabilities:} While these cultures have unique gifts and knowledge, they also face real limitations. Aelaerem are fragile in mass conflict, Mazereth suffer surface sickness, Umbral Kin are unstable in bright light, and Archive Keepers risk overload. Their strengths come with meaningful costs that create interesting choices.
\end{itemize}

\subsection*{Scene Framing:}
\begin{itemize}[leftmargin=*]
\item \textbf{Start in threshold spaces or realm boundaries:} Begin scenes at doorways, hedges, cave entrances, shadow edges, or between bookshelves. These liminal zones are where threshold cultures are strongest and where their unique abilities shine. Let the environment provide tactical advantages and narrative hooks.
\item \textbf{Use scale for tactical variety:} Design encounters that highlight size differences—small PCs can use tight spaces, hide in unexpected places, or exploit environmental hazards that don't affect larger characters. Big PCs can provide brute force solutions but may struggle with finesse requirements or environmental navigation.
\item \textbf{Frame conflicts around access and visibility:} Tension often comes from who can go where and who knows what. Information control, territorial disputes, and questions of who gets to cross thresholds create natural dramatic friction without resorting to simple combat encounters.
\item \textbf{Let cultural differences drive narrative tension:} Conflicts arise from different value systems—Aelaerem hospitality vs. Mazereth precision, Umbral Kin secrecy vs. Archive Keeper transparency, surface urgency vs. threshold patience. These philosophical differences create compelling roleplay opportunities.
\end{itemize}

\subsection*{Integration Hooks:}
\begin{itemize}[leftmargin=*]
\item \textbf{Hidden communities beneath existing venues:} Every major location in your campaign world has threshold spaces—the walls of a palace house Aelaerem guest-halls, the foundation stones contain Mazereth tunnels, the chandeliers cast Umbral reflection-lanes, and the library stacks hide Archive Keeper sanctums. These communities can provide aid, information, or complications.
\item \textbf{Small folk as information brokers:} These cultures excel at gathering and trading information through their networks. Aelaerem hearth-clans know everyone's business, Mazereth tunnels connect distant locations, Umbral Kin read emotional residue, and Archive Keepers maintain vast information webs. They make excellent contacts for investigation scenarios.
\item \textbf{Realm connections to major plot points:} Threshold spaces often intersect with significant locations—ancient vaults exist partially in threshold realms, important artifacts are hidden in scale-shifting spaces, political negotiations happen in neutral threshold zones, and supernatural threats emerge from realm boundaries. These cultures provide access to plot-critical locations.
\item \textbf{Scale-based tactical opportunities:} Create scenarios where size differences matter tactically—small PCs can sabotage large mechanisms, infiltrate secure areas, or deliver messages through impossible routes, while big PCs can provide protection, move heavy objects, or serve as mobile platforms for smaller allies.
\end{itemize}

\section{Example of Play}

\textbf{Setup:} An Aelaerem craftsperson named Thistle works with the surface merchant guild to investigate reports of a haunted vault beneath the old marketplace. Thistle brings Object Bond (lockpick expertise), Threshold Walking (navigation of liminal spaces), and a Liminbound bond (connection to threshold spaces). The guild provides resources and legal authority, but they're dependent on Thistle's unique abilities to access the vault's hidden sections.

\textbf{Scene:} The vault exists partially in threshold space, with some chambers accessible normally and others requiring passage through liminal boundaries. Thistle uses Threshold Walking to navigate sections that shift between dimensions, Object Bond to pick precision locks that respond to careful touch rather than force, and her Liminbound bond to sense stable passages versus areas that might collapse or shift unexpectedly. The guild representatives follow with lanterns and tools, but they're operating outside their expertise.

\textbf{Complications:} As Thistle delves deeper, her actions increase Threshold Tension, attracting the attention of a territorial threshold guardian who begins manifesting as locked doors that reappear after being opened and shadows that move against the light. Meanwhile, the guild masters above notice unusual activity and begin asking questions, raising Visibility and creating pressure to complete the job quickly. The vault's unstable nature threatens Realm Stability—walls occasionally shift, passages close temporarily, and the boundary between threshold and normal space becomes less distinct.

\textbf{Resolution:} Success reveals that the "haunting" was actually a Mazereth clan using the vault's threshold sections as a shortcut between their deep tunnels and surface access points. They were inadvertently disrupting the space through their passage. Thistle brokers a compromise—establishing proper guest-right protocols and payment in copper nails that strengthen the vault's structure. The solution opens new connections between surface and threshold communities but also increases Notice and Visibility, making Thistle a known quantity to both guild authorities and other threshold cultures who may seek her services in the future.

\clearpage
\chapter{Beastfolk, Dragonfolk, and Others}

\section{Overview}
\label{sec:nonhuman_overview}

In \emph{Fate's Edge}, Beastfolk, Dragonfolk, and Others are not common ``races'' but
rare, mythic outliers. They dwell primarily in the Valewood and the deep Wilds,
rarely appearing in human lands except as rumors, omens, or the subject of
contradictory stories.

These origins are intended to:
\begin{itemize}
  \item Emphasize \textbf{rarity and narrative weight}.
  \item Tie directly into core systems: Position \& Effect, Story Beats (SB), Boons,
        Fatigue, Obligation, and Patrons.
  \item Root nonhuman characters in specific places and stories rather than generic ancestry.
\end{itemize}

\subsection{Using Nonhuman Origins}
\label{subsec:nonhuman_usage}

\paragraph{GM Guidance.}
Nonhuman PCs should remain uncommon. As a default:
\begin{itemize}
  \item Limit to \textbf{0--1 nonhuman PC} per party unless the campaign is primarily
        set in the Valewood or Wilds.
  \item In human lands, nonhumans always shift Position \& Effect:
        \begin{itemize}
          \item Social actions often start from worse Position (risky or desperate),
                but can achieve higher Effect when leaning into fear, myth, or awe.
        \end{itemize}
  \item Nonhuman PCs should come with explicit ties: a Patron, a forest court, a dragon
        enclave, an old debt, or a prophecy.
\end{itemize}

\paragraph{The ``Other in Human Lands'' Clock.}
When a Beastfolk, Dragonfolk, or Other PC moves openly in human lands, the GM
creates an ``Otherness'' clock for that community (4--6 segments, as appropriate).
\begin{itemize}
  \item Each time the PC acts in a conspicuous way, mark 1 segment.
  \item When the clock fills, introduce a consequence: hunters, scholars, inquisitors,
        curious nobles, street rumors, or religious panic.
\end{itemize}
This clock replaces ambient prejudice with focused, story-relevant pressure.

% ============================================================
\section{Beastfolk}
\label{sec:beastfolk}

Where the road ends, the people of fur and claw remember no empire but the
forest. Beastfolk are Lethai rewritten by the Wild through old pacts, curses, or
bloodlines bound to animal spirits. They exist as reminders that civilization is
not the only intelligence on the map.

\subsection{Culture and Place}
\label{subsec:beastfolk_culture}

Beastfolk clans are loosely aligned with:
\begin{itemize}
  \item \textbf{Valewood Circles}: semi-nomadic bands tied to groves, standing stones, or
        ancient trails.
  \item \textbf{Wild Moots}: rare gatherings where multiple clans, and sometimes Others,
        negotiate seasonal rites, borders, and common threats.
\end{itemize}

The fallen Utar Empire is remembered as an infection that burned out. Human
kingdoms are regarded as lingering fever, not the natural order.

Common archetypes include:
\begin{itemize}
  \item \textbf{Wolf-blooded}: pack-minded scouts and guardians.
  \item \textbf{Stag-marked}: antlered wanderers, tied to sacrifice and omen.
  \item \textbf{Raven-tongued}: tricksters and memory-keepers.
  \item \textbf{Bear-hearted}: slow to anger, anchors of their communities, unstoppable when roused.
\end{itemize}

\subsection{Core Feature: Beast Instinct}
\label{subsec:beastfolk_instinct}

When you create a Beastfolk character, choose an \textbf{Instinct} that defines how
the Wild speaks through you:
\begin{itemize}
  \item \textbf{Hunt} (stalk, pursue, reveal weakness)
  \item \textbf{Guard} (protect, endure, stand firm)
  \item \textbf{Wander} (roam, explore, cross boundaries)
  \item \textbf{Trick} (misdirect, unsettle, reframe)
\end{itemize}

\paragraph{Instinct Move.}
Once per scene, when you clearly act in line with your Instinct in a risky moment:
\begin{itemize}
  \item Gain \textbf{+1 die} to the roll.
  \item On a hit, the GM gains \textbf{+1 SB} to spend later on complications tied to
        the Wild or your nature.
\end{itemize}

If you deliberately betray your Instinct to secure safety or comfort, the GM may:
\begin{itemize}
  \item Mark \textbf{+1 Fatigue}, or
  \item Mark \textbf{+1 segment} on your personal \emph{Alienation from the Wild} clock (4 segments).
\end{itemize}
When the clock fills, you owe the Wild---or a Patron of the Wild---a serious price
to reconcile who you have become.

\subsection{Beastfolk Talents}
\label{subsec:beastfolk_talents}

\subsubsection{Keen Scent of the Valewood (6 XP)}
\label{talent:keen_scent_valewood}

You can ``read'' a scene with your senses.

\begin{itemize}
  \item Once per scene in the Wilds, when you \textbf{Study} or \textbf{Take Stock}, you may ask
        \emph{one additional question} on a hit.
  \item The GM marks \textbf{+1 SB} tied to an unseen predator, rival clan, or old obligation.
\end{itemize}

\subsubsection{Run With the Pack (8 XP)}
\label{talent:run_with_the_pack}

You are never truly alone.

\begin{itemize}
  \item When you fight or act alongside an ally who trusts you, you both gain
        \textbf{+1 Effect} when you coordinate using pack tactics (flanking, driving prey,
        trading blows).
  \item On a miss, the GM may mark \textbf{+1 Harm} or \textbf{+1 Fatigue} on an ally instead of you;
        your instinctive coordination pulls them into danger.
\end{itemize}

% ============================================================
\section{Dragonfolk}
\label{sec:dragonfolk}

Where a dragon died screaming, someone always wakes up different. Dragonfolk are
not full dragons, but living fault-lines where draconic power has leaked into
blood, bone, and story.

Some claim descent from ancient wyrms of the mountains; others were changed by
proximity to hoards, old lairs, or dragon-Patrons. In human lands, most people
have never seen a Dragonfolk and would not believe in them outside of folktales.

Common archetypes include:
\begin{itemize}
  \item \textbf{Ashscale}: scarred by volcanic fire, eyes smoldering with embers.
  \item \textbf{Stormblood}: hair and scales crackling with static, voice like thunder.
  \item \textbf{Deepcoil}: serpentine, cold-eyed, hungry for secrets and leverage.
  \item \textbf{Shardsoul}: reflective, glassy scales, obsessed with memory and perfection.
\end{itemize}

\subsection{Core Feature: Draconic Echo}
\label{subsec:draconic_echo}

Choose one \textbf{Echo}, a thematic way that dragon-nature manifests in you:
\begin{itemize}
  \item \textbf{Fire} (fury, destruction, passion)
  \item \textbf{Storm} (change, disruption, omen)
  \item \textbf{Hoard} (possession, memory, obsession)
  \item \textbf{Dominion} (authority, command, pride)
\end{itemize}

\paragraph{Echo Move.}
Once per scene, when you invoke your Echo:
\begin{itemize}
  \item Describe how it shapes your presence (heat in the air, rumbling voice,
        unblinking gaze, the scent of rain, etc.).
  \item Gain \textbf{+1 die} to actions that rely on force of personality (such as
        \textbf{Command}, \textbf{Sway}, or similar moves).
\end{itemize}

On a hit, the GM gains \textbf{+1 SB}, earmarked for \emph{dragon-shaped consequences}:
old debts, dragon cults, relic awakenings, rival Dragonfolk, or attention from a
dragon-Patron.

If you refuse to draw on your Echo in a moment where it would clearly help, mark
\textbf{+1 Fatigue}; you are fighting your own nature.

\subsection{Dragonblooded Resilience}
\label{subsec:dragon_resilience}

Dragonfolk shrug off some threats but attract others.

\begin{itemize}
  \item You gain \textbf{+1 armor} against Fire, Lightning, or Fear effects when it is
        fictionally appropriate.
  \item Artifacts, wards, and Rites that bind or repel ``dragons and their kin''
        also affect you, even if they were not meant for your specific bloodline.
\end{itemize}

\subsection{Dragonfolk Talents}
\label{subsec:dragonfolk_talents}

\subsubsection{Hoard-Sense (6 XP)}
\label{talent:hoard_sense}

You can feel where stories and value pool.

\begin{itemize}
  \item When you \textbf{Survey} a location for treasure, secrets, or leverage, on a hit
        you may also ask:
        \begin{quote}
          ``What here would I be most tempted to claim and keep?''
        \end{quote}
  \item If you pursue that answer, gain \textbf{+1 die} on the relevant action.
  \item If you walk away from it, the GM may mark \textbf{+1 SB} for future temptation.
\end{itemize}

\subsubsection{Draconic Breath (12 XP, Tier II+)}
\label{talent:draconic_breath}

You have learned to vent a fraction of the dragon-fire (or storm, frost, venom)
in your blood.

\begin{itemize}
  \item Once per session, you may unleash a \textbf{Breath Attack} as a risky or desperate
        action.
  \item Gain \textbf{+2 Effect} against a group or large target when you do so.
  \item After using it, mark \textbf{+2 Fatigue}. If this would cause overflow, the GM
        instead marks \textbf{+1 SB} and you gain a new visible draconic mutation
        (eyes, scales, voice, or other change).
\end{itemize}

% ============================================================
\section{Others}
\label{sec:others}

The world remembers its own dreams, and sometimes they wake up in the shape of
people. The \textbf{Others} are those not wholly of the mortal world:
fae-touched, gloam-born, children of forgotten Rites, or walkers between dreams
and waking.

They stand at boundaries: between Valewood myth and human history, between one
Patron's claim and another's silence.

Common archetypes include:
\begin{itemize}
  \item \textbf{Gloam-touched}: eyes like dusk, never fully in one light or shadow.
  \item \textbf{Hollow-bound}: something else peers out through their eyes, patient and curious.
  \item \textbf{Green-vowed}: sworn to a forest, river, stone circle, or ancient tree.
  \item \textbf{Masks of Dawn}: can pass as human until the wrong light hits them.
\end{itemize}

\subsection{Core Feature: Liminal Nature}
\label{subsec:liminal_nature}

Choose one \textbf{Liminal Axis} that defines your between-ness:
\begin{itemize}
  \item \textbf{Dream / Waking}
  \item \textbf{Forest / City}
  \item \textbf{Life / Death}
  \item \textbf{Memory / Forgetting}
\end{itemize}

\paragraph{Liminal Move.}
When you act to \emph{bridge} or \emph{exploit} your Axis:
\begin{itemize}
  \item Gain \textbf{+1 die} to the roll.
  \item On a hit, the GM may mark \textbf{+1 SB} to introduce consequences that blur
        reality: time slippage, lost memories, omens, strange visitors, and similar effects.
\end{itemize}

\subsection{Bound Obligation}
\label{subsec:bound_obligation}

Others rarely exist without a binding.

Choose one:
\begin{itemize}
  \item Bound to a \textbf{Place} (a sacred grove, crossroads, riverbend).
  \item Bound to a \textbf{Story} (prophecy, curse, legend, song).
  \item Bound to a \textbf{Being} (Patron, ancient spirit, sleeping dragon, forest court).
\end{itemize}

Write one sentence:
\begin{quote}
  ``I am bound to \rule{2cm}{0.15mm} and must \rule{2cm}{0.15mm}.'' 
\end{quote}

Any time you defy this Bond in a major way:
\begin{itemize}
  \item The GM may mark \textbf{+2 SB}, or
  \item Immediately introduce a supernatural consequence tied to your Bond.
\end{itemize}

Any time you honor it at serious personal cost:
\begin{itemize}
  \item Clear \textbf{2 Fatigue}, or
  \item Mark \textbf{+1 Boon} for later use.
\end{itemize}

\subsection{Talents for Others}
\label{subsec:others_talents}

\subsubsection{Step Sideways (8 XP)}
\label{talent:step_sideways}

You move through liminal spaces as if they were doors.

\begin{itemize}
  \item Once per scene, when you move through a threshold (doorway, tree-line,
        fog bank, mirror, or similar boundary), you may \emph{reposition} yourself or
        an ally fictionally: appear behind foes, bypass mundane guards, or ``skip''
        a short distance.
  \item Roll the appropriate move. On a hit, you arrive where and how you intended.
  \item On a miss, you arrive altered, late, or not alone. The GM marks \textbf{+1 SB}
        tied to what followed you.
\end{itemize}

\subsubsection{Unsettling Truth (6 XP)}
\label{talent:unsettling_truth}

You can speak in the language of omens.

\begin{itemize}
  \item When you \textbf{Sway}, \textbf{Provoke}, or \textbf{Command} by revealing a hidden
        truth, omen, or uncanny insight about someone, gain \textbf{+1 Effect} on a hit.
  \item The target is left shaken or marked by dread or hope (GM's choice).
  \item On a miss, they reject you and label you a danger or blasphemy; mark
        \textbf{+1 Fatigue}.
\end{itemize}

% ============================================================
\section{Adventure Hooks}
\label{sec:nonhuman_hooks}

Use these peoples to complicate the Valewood and Wilds rather than simply
populate them.

\subsection{The Beastfolk Hunt}
\label{subsec:hook_beastfolk_hunt}

A Beastfolk circle believes a human village has broken an ancient trail-compact.
They are hunting a ``trespasser'' who is actually a human PC or ally. Can the
party negotiate new terms before blood is spilled?

\subsection{Dragon's Debt}
\label{subsec:hook_dragons_debt}

A Dragonfolk PC's Echo stirs an ancient dragon from dormancy. The dragon demands
repayment for a boon given to their bloodline long ago---or offers patronage
that could reshape the region.

\subsection{The Vanishing Grove}
\label{subsec:hook_vanishing_grove}

An Other bound to a sacred grove feels it thinning. Human loggers, Utar
relic-hunters, or a rising Patron's followers are siphoning its power. Saving it
may mean undermining a human ally or opposing a Patron the party relies on.

\subsection{The ``Monster'' in the City}
\label{subsec:hook_monster_city}

Rumors spread of a ``beast-headed killer'' in a human city. The culprit is a
terrified Beastfolk youth, trapped far from the Valewood and hunted by scholars
and zealots. Do the PCs smuggle them home, conceal them, or weaponize the panic?


\end{document}


