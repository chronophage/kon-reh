\documentclass[11pt]{article}
\usepackage[utf8]{inputenc}
\usepackage[T1]{fontenc}
\usepackage{lmodern}
\usepackage{geometry}
\usepackage{setspace}
\usepackage{titlesec}
\usepackage{hyperref}
\usepackage{graphicx}
\usepackage{fancyhdr}
\usepackage{enumitem}
\usepackage{multicol}
\usepackage{tikz}
\usepackage{mdframed}
\usepackage{sectsty}
\usepackage{needspace}
\usepackage{changepage}
\usepackage{array}
\usepackage{tabularx}
\usepackage{booktabs}
\usepackage{marginnote}

\geometry{margin=1in}
\setstretch{1.2}
\sectionfont{\large}
\subsectionfont{\normalsize}
\subsubsectionfont{\small}

\newmdenv[linecolor=black,linewidth=1pt,backgroundcolor=gray!5]{adventurebox}
\newmdenv[linecolor=gray,linewidth=0.5pt,backgroundcolor=gray!3]{mechanicbox}

\pagestyle{fancy}
\fancyhf{}
\fancyhead[L]{Linn: Mist \& Iron}
\fancyhead[R]{Fate's Edge Module}
\fancyfoot[C]{\thepage}

\title{Linn: Mist \& Iron \\ \large Raids, Rivers, and Oathbinding (Fate's Edge v0.1)}
\author{}
\date{}

\begin{document}

\maketitle

\begin{center}
\textbf{A culture-focused module for coastal raids, river thrusts, oath-law, and seasonal expeditions by the Linns---seafarers of the north who strike the Violet Steppe littoral, run the Yloka up into the Abderrian Sea and Mistlands, and range as far south as Theona in the Dolmis. Built to slot into Amaranthine Sea, Political Intrigue, Wilderness, and Caravans using core Fate's Edge procedures (Position/DV, SB, clocks, Strings, Favor/Leverage/Exposure).}
\end{center}

\section*{Design Goals}
\begin{itemize}
\item \textbf{Oath before oar.} Social bonds, boasts, and blood-silver shape play as much as blades.
\item \textbf{Raider--trader parity.} Raids, mercenary service, and escort trade are equally supported.
\item \textbf{Seamless integration.} Uses existing chase/ship rules, faction clocks, and currencies---no new dice math.
\item \textbf{Low bookkeeping.} Crew, ship, season wheel, and 2--3 visible clocks per scenario.
\end{itemize}


      \centering
      \caption{Linn: Mist \& Iron - The First Year (Quickstart)}
      \begin{longtable}{|>{\centering\arraybackslash}m{2.5cm}|>{\centering\arraybackslash}m{3cm}|>{\centering\arraybackslash}m{4cm}|>{\centering\arraybackslash}m{3.5cm}|}
      \hline
      \rowcolor{gray!20}
      \textbf{Season} & \textbf{Key Activities} & \textbf{Mechanical Focus} & \textbf{Sample Score} \\
      \hline
      \textbf{Winter Hearth/Thing} & Rest, repair, socialize, attend Thing moots & 
      \begin{itemize}
          \item Resolve Feuds with Blood-Silver
          \item Swear/fulfill Oaths
          \item Advance Repute through deeds
      \end{itemize} & 
      Thing Moot: Settle disputes, gain winter land, swear new oaths \\
      \hline
      \textbf{Spring Muster} & Prepare ships, gather crews, plan summer activities & 
      \begin{itemize}
          \item Ship repairs/upgrades
          \item Crew coordination rolls
          \item Plan expedition objectives
      \end{itemize} & 
      Scout Mission: Gather intelligence on threats/locations for summer \\
      \hline
      \textbf{Summer Raids} & Active expeditions, trading, warfare & 
      \begin{itemize}
          \item Coastal Raid (Plunder/Blood-Price clocks)
          \item River Strike (Pursuit/Current clocks)
          \item Combat encounters
          \item Weather/Ice hazards
      \end{itemize} & 
      Coastal Raid: Plunder a settlement while evading pursuit \\
      \hline
      \textbf{Autumn Trade/Settling} & Profit from summer deeds, settle accounts & 
      \begin{itemize}
          \item Escort/Trade (Market procedures)
          \item Saga Reputation gains
          \item Social maneuvering at Thing
          \item Plan for winter
      \end{itemize} & 
      Escort Mission: Protect trade goods while navigating hazards \\
      \hline
      \end{longtable}
      
      
      
      \centering
      \caption{Core Quickstart Rules Summary}
      \begin{longtable}{|>{\centering\arraybackslash}m{3cm}|>{\centering\arraybackslash}m{5cm}|>{\centering\arraybackslash}m{5cm}|}
      \hline
      \rowcolor{gray!20}
      \textbf{System} & \textbf{Quick Rule} & \textbf{Key Dice} \\
      \hline
      \textbf{Crew Tracks} & 
      Repute [6]: Fame/standing \\
      Feud [4]: Active quarrels \\
      Exposure [6]: Foreign attention &
      DV 2-5 \\
      Position: Dominant/Controlled/Desperate \\
      Re-roll 1 success/failure \\
      \hline
      \textbf{Ship Actions} & 
      Navigate (DV 2-5) \\
      Board \& Brace (DV 3-5) \\
      Chase (Sea/River) (DV 2-5) &
      Ship Tags modify DVs \\
      Weather/Ice Matrix [4-6] \\
      \hline
      \textbf{Oaths \& Saga} & 
      Oath Ledger: Favors owed/held \\
      Break Oath: Repute -1, Feud +1 \\
      Saga Reputation [6]: Local → Regional → Legendary &
      Saga Points for epic deeds \\
      Position/Social benefits by tier \\
      \hline
      \textbf{Complications} & 
      Mist \& Iron SB Menu (1-4 SB) \\
      Weather/Ice hazards \\
      Fen Corruption clock &
      Story Beats drive narrative \\
      Environmental clocks \\
      \hline
      \end{longtable}
      }

\section{Linn Crew/Clan Sheet (Template)}

\begin{center}
\begin{longtable}{|p{5cm}|p{5cm}|}
\hline
\textbf{[CLAN / CREW NAME]} & \\
\hline
Aett (home seat): \_\_\_\_\_\_\_\_\_\_\_\_ & Thing Affiliation: \_\_\_\_\_\_\_\_\_\_\_\_ \\
\hline
\end{longtable}
\end{center}

\textbf{Strings (2--3):} winter harbor $\bullet$ river-right $\bullet$ trade oath $\bullet$ feud settlement $\bullet$ warding chant $\bullet$ pilot's stone

\textbf{Tags (2--4):} Oathbound $\bullet$ Riverwise $\bullet$ Ice-Trained $\bullet$ Reaver-Known $\bullet$ Hospitable $\bullet$ Law-Strict $\bullet$ Skald-Loud $\bullet$ Wolf-Banner

\textbf{Tracks:}
\begin{itemize}
\item Repute [6] (standing among Linn and neighbors)
\item Feud [4] (active quarrel; on fill $\rightarrow$ blood-price owed or war)
\item Exposure [6] (to foreign powers; for non-Linn venues)
\end{itemize}

\textbf{Oath Ledger:} favors owed/held (name + what; acts like Favor/Leverage within Linn spheres)

\textbf{Notables:} steersman $\bullet$ war-leader $\bullet$ speaker $\bullet$ skald $\bullet$ shipwright

\textbf{Repute [6]:} Tick up for kept oaths, fair trade, valor; tick down for oath-breaking, cowardice, or sacrilege. High Repute grants Audience: Respectful at thing; low invites Audience: Skeptical.

\textbf{Feud [4]:} Name the counterparty. Ticks on insult, theft, or harm; clear by blood-silver, ordeal, or deed.

\section{Ship Sheet (Longhulls \& Rivercraft)}

\textbf{Pick a hull and 3--4 tags.}

\textbf{Longhull (raider):} shallow draft, fast oars, beachable.

\textbf{Broadshore (cargo):} higher freeboard, stout frames, slower oars.

\textbf{River-spear (cutter):} light, narrow, collapsible mast for weirs.

\textbf{Ship Tags (choose 3--4)}
\begin{itemize}
\item Shallow Draft --- River rapids Drive/Handle DV --1; cross bars, weirs.
\item Beaching Hull --- Land/sail from beaches; Disengage gains Position +1 at surf.
\item Ice-Ready --- In thaw/freeze scenes, first Condition +1 is ignored.
\item Mistwise --- In fog/gares, Navigate DV --1; Ambush at Sea starts one Position higher.
\item River-Runner --- Row Upstream DV --1; portage checks Position +1.
\item Wolf-Boarding Gear --- Board \& Brace DV --1; entangling hooks/ropes.
\item High Shields --- +1 die defending vs arrows/sling; on 1: Side-Slip penalty (windage) applies.
\item Skald-Drum --- Once/score gain Audience: Fierce aboard; can rally morale.
\item Pilots' Stones --- Hidden cairn-marks; once/score Navigate Position +1 in shoals.
\end{itemize}

\textbf{Roles (choose at table)}\\
Steersman (helm), War-Leader (boarding/assault), Speaker (parley), Skald (morale/saga), Shorehand (portage/repairs), Lookout (weather/ice).

\section{Season Wheel \& Theaters}

Spring Muster $\rightarrow$ Summer Raids $\rightarrow$ Autumn Trade/Settling $\rightarrow$ Winter Hearth/Thing.

At each transition:
\begin{itemize}
\item Advance Politics (Mandate/Crisis) for coastal powers.
\item Roll Weather/Ice [4--6] for the theater in play (coast, river, mistland).
\item Offer Oath Opportunities (escort, feud settlement, shrine warding).
\end{itemize}

\textbf{Theaters}
\begin{itemize}
\item Coast (Abderrian/Mistlands): cliffs, skerries, tide races, fog.
\item River (Yloka): bars, rapids, weirs, toll-fords, riverside towns.
\item Dolmis Reach (south): broader seas, warmer storms, foreign courts.
\end{itemize}

\section{Score Types \& Procedures}

Pick approach (Deceit $\bullet$ Speed $\bullet$ Shock $\bullet$ Parley) $\rightarrow$ set Position/DV from venue/tags $\rightarrow$ roll.

\subsection{Coastal Raid (Objective: Plunder / Message / Prisoner)}

\textbf{Clocks:} Alarm [4--6], Plunder [6--8], Ship Damage [4], Blood-Price [4].

\textbf{Entry:} surf landing, hidden cove, tide gate, harbor ruse.

\textbf{On 1s:} GM spends SB $\rightarrow$ alarm bells, hidden shoal, torch-chain across channel, watch-tower signal.

\textbf{Resolution:} When Plunder fills, choose coin or Strings (hostage pledge, toll writ, seasonal tithe). If Blood-Price fills, mark Feud +1 or pay blood-silver (Favor loss or obligation clock).

\subsection{River Strike (Objective: Tollhouse / Weir / Rival Barge)}

\textbf{Clocks:} Current [4], Pursuit [6], Sentries [4], Plunder/Terms [6].

\textbf{Position tweaks:} River-Runner/Shallow Draft help; crosswinds/hail hurt.

\textbf{On 1s:} weir gets raised, arrows from reed-blind, boom-chain snaps at wrong time.

\subsection{Escort/Trade (Objective: Profit / Standing)}

Use Market from Amaranthine/Caravans; add Linn perks:
\begin{itemize}
\item Thing Tokens (oath-markers) count as String once/session in Linn venues.
\item Skald-Drum can convert Audience: Warm $\rightarrow$ Favor (narrow) once/score with a saga.
\end{itemize}

\subsection{Thing Moot (Objective: Law / Settlement)}

\textbf{Venue:} ring of stones, winter hall.

\textbf{Moves:} Oath-Swear (commit under penalty), Wager Wyrd (ordeal by feat), Blood-Silver (compensation roll), Witness the Saga (Skald stakes a truth).

\textbf{Outcomes:} resolve Feud, write Oath to ledger (acts as durable String), assign winter land or river-right.

\section{Oaths, Blood-Silver, \& Repute}
\begin{itemize}
\item \textbf{Oath (currency):} Inside Linn spheres, treat Oath Ledger entries as Favor/Leverage equivalents; breaking one ticks Repute --1 and creates Feud +1.
\item \textbf{Blood-Silver (settlement):} Pay with coin, hostage-string, or deed. Roll Petition/Broker vs DV 2--4 (standing, witnesses, hurt). On hit, reduce Feud --2; on partial, --1; on miss, counter-oath demanded.
\item \textbf{Boasts \& Sagas:} A public boast creates Audience: Expectant; fulfill it to gain Repute +1, fail and mark Exposure or Feud.
\end{itemize}

\section{Sea \& River Procedures}

\subsection{DV Ladders}
\begin{itemize}
\item \textbf{Chase (Sea):} DV 2 reach $\bullet$ 3 skerries $\bullet$ 4 reef line $\bullet$ 5 storm eddies.
\item \textbf{Chase (River):} DV 2 open $\bullet$ 3 bars $\bullet$ 4 rapids $\bullet$ 5 weirs/locks.
\item \textbf{Board \& Brace:} DV 3--5 (tags: Wolf-Boarding Gear lowers DV).
\end{itemize}

\subsection{Weather/Ice Matrix [4--6]}

Advance on 1s or fiction:
\begin{itemize}
\item \textbf{Fog/Mist:} sight Position --1; Mistwise cancels. On 1, Pursuit +1 (lost bearings).
\item \textbf{Squall/Hail:} ranged actions --1 die; on 1, Condition +1 (sails/rig).
\item \textbf{Ice/Floe:} Navigate DV +1; on 1, choose Delay (Distance stalls) or Keel-Scuff (Condition +1).
\end{itemize}

\subsection{Portage \& Weirs}

Treat as Cross Hazard (Body+Tactics/Craft) DV 3--5; River-Runner/Shallow Draft grant Position +1. On 1, Oarline snaps or Axle-sled breaks.

\subsection{Mist \& Iron SB (GM menu)}
\begin{itemize}
\item Hidden Shoal: sudden ground; Ship Damage +1 unless Position was high.
\item Tower Fire: beacon lit; Pursuit +1 and Alarm +1.
\item Oarline Breaks: lose Position; repair or fight short-handed.
\item Witness at Cliff: an enemy skiff sees; Exposure +1 (foreign) or Feud +1 (Linn).
\item Saga Twisted: rumor flips an Audience tag against you.
\end{itemize}

\section{Linn Culture Tools (portrayal guidance)}
\begin{itemize}
\item Emphasize law and reciprocity (oaths, blood-silver, witness) over caricature.
\item Show plural livelihoods: fishers, traders, wardens, mercenaries, skalds---not only raiders.
\item Let women/elders hold seats and steer deals; avoid monolith tropes.
\item Lean into seasonality and thing assemblies as civic life.
\end{itemize}

\textbf{Etiquette Hooks (once/scene in Linn venues):} gift the host's hearth with salt/fish oil; name your mother's line; offer a verse---each can grant Position +1 in parley.

\section{Factions \& Fronts}
\begin{itemize}
\item \textbf{Linns Union (docks \& dues):} Strings---dock priority, barge pilots.
\item \textbf{Mistland Wardens:} Strings---fog bell chains, cliff beacons.
\item \textbf{Yloka Tollmen:} Strings---boom-chains, river seals.
\item \textbf{Dolmis Factors:} Strings---winter contracts, bonded warehouses.
\item \textbf{Shrine of Storm-Whale:} Strings---safe-run chants, tithe.
\end{itemize}

\textbf{Front Clocks (examples):}
\begin{itemize}
\item Reprisals Fleet [6--8] (coastal power organizes counterstrike).
\item Thing Schism [6] (oath controversies split halls).
\item Mistland Famine [6] (trade mission needed; gain Standing if solved).
\end{itemize}

\section{Generators}

\subsection{Coastal Targets (d66)}

11 tide mill $\bullet$ 12 beacon tower $\bullet$ 13 cliff monastery $\bullet$ 14 saltworks $\bullet$ 15 fishing fleet $\bullet$ 16 lord's boathouse $\bullet$ 21 skerry storehouse $\bullet$ 22 ropewalk $\bullet$ 23 ferry-chain $\bullet$ 24 dyeshed $\bullet$ 25 customs shed $\bullet$ 26 river gate $\bullet$ 31 amber beach $\bullet$ 32 quarry pier $\bullet$ 33 seal rookery $\bullet$ 34 barge yard $\bullet$ 35 sheep isle $\bullet$ 36 fortress quay $\bullet$ 41 smokehouse row $\bullet$ 42 eel-weirs $\bullet$ 43 pilot stone $\bullet$ 44 winter harbor $\bullet$ 45 tax sloop $\bullet$ 46 river lighthouse $\bullet$ 51 shrine cove $\bullet$ 52 smugglers' cut $\bullet$ 53 patrol launch $\bullet$ 54 tollhouse $\bullet$ 55 grain pier $\bullet$ 56 seawall breach $\bullet$ 61 ice slip $\bullet$ 62 wreckers' fires $\bullet$ 63 tide cave $\bullet$ 64 chain boom $\bullet$ 65 slate wharf $\bullet$ 66 mint barge.

\subsection{River Hazards (d12)}

1 bar on a bend $\bullet$ 2 sudden freshet $\bullet$ 3 weir rat-lines $\bullet$ 4 deadwood snag $\bullet$ 5 eel-pots $\bullet$ 6 ice pans $\bullet$ 7 hidden side-cut $\bullet$ 8 bluff echo $\bullet$ 9 toll chain half-raised $\bullet$ 10 reeds conceal archers $\bullet$ 11 rain-swollen ford $\bullet$ 12 sand-suck bank.

\subsection{Thing Cases (d12)}

1 insult in song $\bullet$ 2 stolen pilot stone $\bullet$ 3 broken oath on winter grain $\bullet$ 4 blood-price disputed $\bullet$ 5 marriage claim $\bullet$ 6 warding chant stolen $\bullet$ 7 hostage pledge lapsed $\bullet$ 8 salvage rights $\bullet$ 9 border cairn moved $\bullet$ 10 feud cooling terms $\bullet$ 11 saga witness contest $\bullet$ 12 mercenary pay withheld.

\subsection{Sagas \& Boons (d12)}

1 oar-song that steadies arms $\bullet$ 2 whale omen at dawn $\bullet$ 3 amber find $\bullet$ 4 pilot's ghost shows a cut $\bullet$ 5 storm-whale spares you $\bullet$ 6 omen of red sails $\bullet$ 7 skald's verse spreads $\bullet$ 8 winter hall adopts you $\bullet$ 9 river seal renewed $\bullet$ 10 cliff bell silent $\bullet$ 11 mist opens path $\bullet$ 12 oath-ring warms (truth told).

\section{Integration Notes}
\begin{itemize}
\item \textbf{Amaranthine Sea:} Use ship chase and blockade tools; Linn Skald-Drum converts Audience to Favor once/score in port riots or dock disputes.
\item \textbf{Caravans:} Swap staging at waystations with winter harbors; River-Runner aids barge convoys upriver.
\item \textbf{Wilderness:} Portage/overland jumps tie into outpost assets; fjord hunts use Hunt/Chase ladders.
\item \textbf{Political Intrigue:} Treat the Thing as a political venue; Repute sets default Position; Blood-Silver interacts with Favor/Exposure economies.
\item \textbf{Psionics:} Seers translate as omen-readers; allow Psychic Weather Sense to soften fog/ice penalties once/leg.
\item \textbf{Dragon's Lair:} Skerries and glacier valleys hide wyrm shrines; oaths may bind to ancient powers for perilous boons.
\end{itemize}

\section{Example of Play (short)}

\textbf{Setup:} Longhull with Shallow Draft, Mistwise, Wolf-Boarding Gear. Crew Repute 3/6, Feud 1/4 (with Yloka Tollmen). Score: River Strike against a boom-chain tollhouse (Plunder/Terms).

\textbf{Approach:} Parley-then-Shock. Speaker petitions for winter discount (DV 3). Partial $\rightarrow$ Position stays Controlled, GM spends 1 SB.

\textbf{Action:} War-Leader triggers Board \& Brace at the boom. Wolf-Boarding Gear drops DV to 3. Hit $\rightarrow$ Plunder +2; a 1 shows $\rightarrow$ GM spends Tower Fire (Alarm +1, Pursuit +1).

\textbf{Twist:} Fog rolls in. Mistwise cancels Position penalty. Steersman runs a Pilots' Stone line: Navigate gains Position +1; Pursuit --1.

\textbf{Close:} Plunder fills. Crew chooses a seasonal tithe String instead of coin. Blood-Price at 2/4; they pledge blood-silver at winter thing to avoid Feud tick.

\section{GM Reference (one page)}
\begin{itemize}
\item \textbf{Crew:} Repute [6] $\bullet$ Feud [4] $\bullet$ Exposure [6] $\bullet$ Oath Ledger.
\item \textbf{Ships:} choose hull + 3--4 tags. Roles: Steersman $\bullet$ War-Leader $\bullet$ Speaker $\bullet$ Skald $\bullet$ Shorehand $\bullet$ Lookout.
\item \textbf{Season Wheel:} Spring muster $\bullet$ Summer raids $\bullet$ Autumn trade/settle $\bullet$ Winter thing.
\item \textbf{Score Types:} Coastal Raid $\bullet$ River Strike $\bullet$ Escort/Trade $\bullet$ Thing Moot.
\item \textbf{Key Clocks:} Alarm $\bullet$ Plunder $\bullet$ Ship Damage $\bullet$ Blood-Price $\bullet$ Pursuit.
\item \textbf{Weather/Ice:} Fog/Mist $\bullet$ Squall/Hail $\bullet$ Ice/Floe.
\item \textbf{SB Menu:} Hidden Shoal $\bullet$ Tower Fire $\bullet$ Oarline Breaks $\bullet$ Witness at Cliff $\bullet$ Saga Twisted.
\item \textbf{Integration:} Portage $\leftrightarrow$ Wilderness $\bullet$ Dock riots $\leftrightarrow$ Amaranthine $\bullet$ Thing $\leftrightarrow$ Political $\bullet$ Omens $\leftrightarrow$ Psionics.
\end{itemize}

\section{Changelog}

\textbf{v0.1} --- First draft: crew/ship sheets, tags \& roles, season wheel, sea/river procedures, raid/river/thing scores, oath \& blood-silver economy, generators, integration, and an example.

\textbf{End of v0.1}
\clearpage

\textbf{Hunger in the Mist \\ \large A Linn Adventure for Fate's Edge (v0.1)}

\begin{center}
    \textbf{Hunger in the Mist \\ \large A Linn Adventure for Fate's Edge (v0.1)} \\
    A coastal‑river saga for Tier II–III parties, inspired by northern raiding epics and survival horror—reimagined for Fate's Edge without referencing any specific IP. Built to slot into Linn: Mist \& Iron, Wilderness, Caravans, Amaranthine Sea, Political Intrigue, and Psionics.
\end{center}

\section*{Pitch}

Winter's edge bites the Mistlands. Beacon towers fail. Villages whisper of Fog-Harriers—antler-helmed raiders who strike from fen and chalk caves, leaving carrion totems and vanished kin. The Linn hall of Stenskar seals its doors… then sends for help. The omens demand a mixed company: oarsmen and outsiders together, for the mist does not remember one tongue alone.

Play in 3–5 sessions: defend a hall; scout the black fen; unmask the Harriers; strike the warrens; end a brood-line at dawn.

\section*{Safety \& Tone}
\begin{itemize}
\item Horror and funerary imagery; keep consent tools handy.
\item Keep Linn culture plural: traders, fishers, skalds, wardens—not caricature raiders.
\item The Fog-Harriers are a syncretic cult of coastal and fen folk (not a ``lost people'' stereotype). Their practices are ritual, not racial.
\end{itemize}

\section*{What You Need}
\begin{itemize}
\item Fate's Edge core SRD (Position/DV, SB, clocks, Strings).
\item Linn: Mist \& Iron (crew/ship tags, thing, oaths, sea/river ladders).
\item Wilderness (camp, biomes: marsh/coast). Optional: Psionics (omens), Political (thing moot), Caravans/Sea if you bring escort/convoy play.
\end{itemize}

\section{Cast \& Hooks}

\textbf{PC hooks (choose or invent):}
\begin{itemize}
\item Outsider Envoy from a southern city bearing a winter contract.
\item Skald-Loud: your saga named a fen spirit; it answered.
\item Mistland Warden: your bell-chain went silent; your oath binds you to find why.
\item River Pilot: your pilot stones were moved; trade boats vanish.
\item Noetic Adept (Psionics): your dreams taste of chalk and marrow; omens point north.
\end{itemize}

\textbf{Notables}
\begin{itemize}
\item Jarl Arnhild Stenskar (Speaker at winter thing). Strings: winter harbor, feud settlement.
\item Hevr Skarn (War-Leader). String: oar-oath.
\item Friga of the Bell Chain (Warden). String: beacon rites.
\item The Fog-Harriers (cult). Strings: bog totems, whisper warrens, marrow resin.
\item The Brood-Matron (Harrier oracle). String: bone throne in the chalk deeps.
\end{itemize}

\section{Fronts \& Clocks}

\textbf{Primary front — Hunger in the Mist}
\begin{itemize}
\item Beacon Silence [6] — as it fills, coastal beacons fail; +1 Position to Harrier night strikes.
\item Missing Boats [6] — trade collapses; famine pressure rises.
\item Brood-Matron's Rite [8] — completes a mass anointing; Harriers fight at +1 Effect at night.
\end{itemize}

\textbf{Local clocks}
\begin{itemize}
\item Hall Panic [6] — rumor and grief; insert disadvantage tags until calmed.
\item Feud with Tollmen [4] — complicates river help; clear at thing or pay blood-silver.
\item Oath Debts [4] — favors owed by Stenskar; cash for aid or betrayal.
\end{itemize}

\section{Adventure Structure}

\subsection{Act I — The Hall at Dusk (Defense \& Oaths)}

\textbf{Scene A1: River Arrival}
\begin{itemize}
\item Enter by longhull or barge under fog. Mistwise helps; otherwise Position –1 to Navigate.
\item Encounter: wreckers' fires lure boats to a shoal (Hidden Shoal SB). Rescue or lose supplies.
\end{itemize}

\textbf{Scene A2: Stenskar Winter Hall}
\begin{itemize}
\item A funeral table, skalds quiet, children hidden. Hall Panic [6] starts at 2/6.
\item Thinglet: a small moot in the hall: testimonies about antler masks and bog-incense.
\item Oath: Arnhild swears guest-right and asks for a mixed company. Add Oar-Oath String: one boon per session on a roll taken in public defense of Stenskar.
\end{itemize}

\textbf{Scene A3: Night Probe}
\begin{itemize}
\item Harriers test the palisade with bone rattles and peat-smoke. Use Board \& Brace if defending the river gate; ranged volleys are hampered by fog.
\item On any 1, spend SB Witness at Cliff (enemy skiff sees you); tick Missing Boats +1.
\end{itemize}

\textbf{Outcomes:} stabilize Hall Panic, earn Beacon Rite String (Warden's trust), discover marrow resin (sweet narcotic used by Harriers).

\subsection{Act II — Black Fen Recon (Scouting \& Skirmish)}

\textbf{Travel:} Wilderness Marsh biome; set Distance [6], Danger [4], Weather [4: fog/rain].

\textbf{Intent:} Scout or Hide to reduce Sign.

\textbf{Key Sites}
\begin{itemize}
\item Bell Chain \#7: rope cut, bell stolen; recover clapper for Clue.
\item Totem Isle: antler totems, peat-fires burning cold; a captive left as bait.
\item Pilot Stone Cairn: moved downstream; reset to ease escape later (Position +1 on retreat).
\item Fen Bear's Den: optional hunt; pelt as leverage at thing.
\end{itemize}

\textbf{Encounters}
\begin{itemize}
\item Harrier Skirmishers (Near): reed-masks, bone darts (Harm 1, poison tag Drowse → Weariness +1).
\item Bog-Haze: Weather tick; on 1, Condition +1 unless Shelter tag.
\item Whisper Caves Mouth: chalk dust, singing vents; enter now or return prepared.
\end{itemize}

\textbf{Objectives}
\begin{itemize}
\item Track the Harriers to the warrens (accumulate 3 Clues: totem analysis, captive interrogation, resin tracing).
\item Seize totems/resin to learn their rite.
\item Optionally bargain at riverside Thing to clear Feud with Tollmen.
\end{itemize}

\subsection{Act III — The Whisper Warrens (Assault \& Choice)}

\textbf{Dungeon Frame:} single clock Warrens Depth [8]; three spokes: Bone Gallery, Steam Vents, Brood Halls. Advance on noise, heat, or delay. Harriers get Position +1 in darkness unless PCs carry warding light.

\textbf{Rooms \& Beats}
\begin{enumerate}
\item Bone Gallery: trophy totems; a riddle-song carved in chalk (Insight DV 3 decodes entry order; on 1, Alarm [4] +1).
\item Steam Vents: scalding fog bursts; Cross Hazard DV 3–4.
\item Brood Halls: sleeping pits; marrow resin harvest; captives. Mercy vs Zeal choice: rescue captives (slows assault, gain 1 Boon) or leave them (assault faster, Harriers fight with +1 Effect until Brood-Matron's Rite is reduced).
\end{enumerate}

\textbf{The Brood-Matron}
\begin{itemize}
\item Oracle-chieftain enthroned on jointed bone. Uses Drowse fumes, reed-pipes to coordinate, and antler staff to parry.
\item Moves: Frenzy the Pack (allies gain +1 Effect for one exchange), Fog Veil (Position –1 for intruders), Marrow Breath (Weariness +1 area; SB spend).
\item Strings: prophecy tokens (chalk sticks carved with tide glyphs) and a Bone Oath Ring that can settle Feud at a Thing if claimed.
\end{itemize}

\textbf{Finale Options}
\begin{itemize}
\item Strike the Matron: break the cult's coordination; Harriers fragment.
\item Collapse the Vents: explosives or psionic overpressure to seal the warrens (Weather +1; retreat clock).
\item Parley of Oaths: prove the Matron bent an ancient river oath; some Harriers stand down under witness.
\end{itemize}

\textbf{Aftermath:} if Brood-Matron's Rite was near full, pockets of Harriers fight on at night until Beacon Silence is restored.

\section{Running the Fog-Harriers}

They are human cultists. Their horror comes from ritual masks, peat-smoke tactics, and night discipline, not species. Frame them as a fen syncretic sect that fed on famine and fear. Some defect when confronted with oaths and proofs.

\textbf{Harrier Types (framework)}
\begin{itemize}
\item Skirmisher: reed mask, bone darts (Near), marsh knife (Close). Tags: Drowse, Reedslip (Hide/Disengage Position +1 in marsh). DV: 3 for ranged, 2 for melee.
\item Reaver: antler helm, hooked spear (Close), net (entangle). Tag: Hook \& Haul (on hit, reposition foe). DV: 2 for melee.
\item Whisperer: smoke-runner, pipe-signals, resin flasks. Move: Fog Veil (once/scene Position –1 to ranged against allies). DV: 2.
\item Brood-Matron: see above. DV: 4, Harm 2.
\end{itemize}

Use core opposition building: set DV by venue/cover (marsh, fog, night), hand out SB on 1s, and lean on Reedslip/Hook \& Haul to shape the fight.

\section{Oaths, Law, \& Thing}
\begin{itemize}
\item A Hall Thinglet can grant Witness for an oath; use it to settle Feud with Tollmen or bind local aid.
\item Blood‑Silver converts one Feud tick into truce; Boast publicly and fulfill it to gain Repute +1 at winter thing.
\item The Bone Oath Ring from the warrens is admissible proof at a Thing; returning it can grant Standing with Mistland Wardens.
\end{itemize}

\section{Wilderness \& Travel Procedures}
\begin{itemize}
\item Marsh Biome: Infiltrate Position +1; Drive Position –1. Hazards: sucking mud (Cross DV 3–5), miasma (Weariness +1 + Sickness [4]), hidden channel (Swim/Boat).
\item Weather Matrix (Fog/Rain/Ice): see Wilderness; fog adds Pursuit +1 on 1s unless Mistwise.
\end{itemize}

\textbf{Sample Leg:} Stenskar → Black Fen Mouth (Distance [4], Danger [4], Weather [4: fog]). Intent Hide. On strong Hide Sign, reduce Sign –2; on 1, Witness at Cliff SB.

\section{Sea \& River Hooks}
\begin{itemize}
\item River Strike: sabotage boom‑chains raising for Harrier skiffs (Pursuit/Current clocks).
\item Escort: run refugees or grain; Skald‑Drum can convert Audience: Warm → Favor at a desperate dock.
\item Ambush at Sea: Harrier cutters use kelp curtains; Mistwise counters.
\end{itemize}

\section{Psionics Hooks (optional)}
\begin{itemize}
\item Dream Weather: Noetic PC senses chalk singing → Clue to vent map; reduce Warrens Depth advance once.
\item Aegis of Will: disperse dart volleys (convert Harm 1 → Fatigue).
\item Foresight: warn of a night probe; set Hall Defense Position +1.
\end{itemize}

\section{Rewards \& Fallout}

\textbf{Strings:} Beacon Rites • Bone Oath Ring • Winter Harbor • Pilot's Stones Reset • Tollmen Truce.

\textbf{Boons:} Mist opens path (Position +1 on a fog scene once) • Skald's Verse spreads (Audience: Warm on arrival).

\textbf{Standing:} with Mistland Wardens, Dock unions, or Linn thing.

\textbf{Treasure:} marrow resin (contraband), antler crafts, river seals, captives ransomed.

\textbf{Campaign Ripples:}
\begin{itemize}
\item The marrow resin points to a southern buyer → city or sea arc.
\item The Bone Oath Ring ties to an ancient river oath → political arc.
\item The chalk vents hint at wyrm channels → dragon arc.
\end{itemize}

\section{Prep Aids}

Visible clocks to start: Beacon Silence 2/6 • Missing Boats 1/6 • Hall Panic 2/6.

Handouts: Oath tokens, totem sketches, bell‑chain map, pilot stones diagram.

Music: low frame drum; distant bells; water drip.

\section{One‑Page Reference}
\begin{itemize}
\item Acts: Hall Defense → Fen Recon → Warrens Assault/Parley.
\item Key Tags: Mistwise • Shallow Draft • River‑Runner • Reedslip • Drowse.
\item SB Menu (Mist \& Iron): Hidden Shoal • Tower Fire • Oarline Breaks • Witness at Cliff • Saga Twisted.
\item Wilderness SB: Unwelcome Smoke • Bad Footing • Foul Water.
\item Clocks: Beacon Silence • Missing Boats • Brood‑Matron's Rite • Hall Panic • Feud (Tollmen) • Oath Debts.
\item Outcomes: Matron struck • Vents collapsed • Oaths proven.
\item Advancement: Repute shift • Standing gained • Strings banked.
\end{itemize}

\section{Changelog}

\textbf{v0.1} — First pass: three‑act structure, fronts/clocks, scenes/venues, enemies, oaths \& thing, wilderness/river/psionic hooks, rewards, and a one‑page reference.

\textbf{End of v0.1}

\subsection{Temple of Light, Shrines, and the Devotion Spectrum}
\label{sec:temple-devotion}
\index{Temple of Light}\index{shrines}\index{devotion spectrum}

The Temple of Light and the old shrine cults do not exist as a simple on/off switch. Most Linn
live somewhere along a sliding scale of tradition and new faith. Rather than tracking separate
factions in detail, use the following light-weight overlays on existing mechanics (Position, SB,
Tags, and Conditions).

\subsubsection*{The Devotion Spectrum}
\index{devotion spectrum}

Each character (or crew, if you prefer) may mark a single \textbf{Devotion value} from \(-3\) to
\(+3\):

\begin{center}
\(-3\) \(\rightarrow\) \(-2\) \(\rightarrow\) \(-1\) \(\rightarrow\) \textbf{0} \(\rightarrow\) \(+1\) \(\rightarrow\) \(+2\) \(\rightarrow\) \(+3\)
\end{center}

\begin{itemize}
  \item \(-3\): \textbf{Pure Shrine} --- openly devoted to patrons, deeply suspicious of Temple.
  \item \(-1\) to \(+1\): \textbf{Mixed / Unsure} --- lives in both worlds, or keeps faith private.
  \item \(+3\): \textbf{Pure Temple} --- openly devout; treats shrines as questionable or obsolete.
\end{itemize}

\paragraph*{When It Moves}
\begin{itemize}
  \item Only move Devotion on \textbf{major story beats}: public conversions, dramatic repudiations
        of a faith, founding or closing a shrine/Temple, or major miracles.
  \item A quiet personal doubt is color; a public oath-breaking before a hall can shift the track.
\end{itemize}

\paragraph*{Mechanical Edge (Extremes Only)}
Devotion only has direct mechanical teeth at:
\begin{itemize}
  \item \(-2\) or \(-3\): \textbf{Shrine-Favored} --- gain \textbf{+1d} on patron rites or shrine
        diplomacy; suffer \(-1\) Position in clearly Temple-dominated venues.
  \item \textbf{0}: \textbf{Bridge-Walker} --- no innate bonus, but can qualify for \emph{Ambassador
        Status} (below).
  \item \(+2\) or \(+3\): \textbf{Temple-Favored} --- gain \textbf{+1d} on Temple rites or Temple
        court scenes; suffer \(-1\) Position in shrine strongholds.
\end{itemize}

\subsubsection*{Venue Tags \& Faith-Based Position}
\index{venue tags}\index{Position!faith-based}

Instead of detailed faction sheets, give each important location a single \textbf{faith tag}:

\begin{itemize}
  \item \textbf{Shrine Stronghold} --- hills, halls, or villages bound tightly to patrons.
  \item \textbf{Temple District} --- towns with active Temple of Light presence and law.
  \item \textbf{Mixed Quarter} --- places where both altars stand; everyday compromise.
  \item \textbf{Contested Ground} --- frontiers, halls, or thing-sites where both sides push.
\end{itemize}

Apply simple Position modifiers:

\begin{itemize}
  \item \textbf{Shrine-Friendly Venue}: Temple-leaning characters (\(+2/+3\)) suffer
        \(-1\) Position on overtly faith-based actions unless they show respect or humility.
  \item \textbf{Temple-Friendly Venue}: Shrine-leaning characters (\(-2/-3\)) suffer
        \(-1\) Position on faith actions unless they submit to Temple forms.
  \item \textbf{Mixed Quarter}: No automatic penalty; GM may assess \(-1\) Position for open
        grandstanding by either extreme.
  \item \textbf{Contested Ground}: Both sides suffer \(-1\) Position on faith actions \emph{unless}
        they can frame their approach as \textbf{bridge-building}. The GM may spend \textbf{1~SB}
        to create a brief opening where a sincere mediator can act at neutral Position.
\end{itemize}

\subsubsection*{Faith Tension as a Scene Tag}
\index{faith tension}

Treat religious tension like weather or terrain: a \textbf{scene-level tag} that sits on top of
everything else.

\begin{description}
  \item[Low Tension] Everyday life. No special modifiers.
  \item[Medium Tension] Rumors, side-eyes, low-level feuds. \emph{Faith-marked} actions (invoking
    shrines, Temple doctrine, public prayers) suffer \(-1\) Position.
  \item[High Tension] Open feuding, recent blasphemy, or fresh Temple decrees. Opposing-faith
    interactions take \(\mathrm{DV}+1\) and generate automatic \textbf{Exposure +1} on failure.
\end{description}

The GM can adjust Faith Tension up or down when:
\begin{itemize}
  \item A public slight, miracle, or atrocity occurs.
  \item A Saga Scene (\S\ref{sec:saga-system}) plays out around faith.
\end{itemize}

\subsubsection*{Synthesis Tokens: Blending Practices}
\index{Synthesis Token}\index{syncretism}

When characters successfully blend shrine and Temple practices in a scene \emph{and} the table
agrees it feels earned (shared prayers, joint rites, dual-oath funerals):

\begin{itemize}
  \item The crew gains a \textbf{Synthesis Token} (treat like a \emph{Boon} that can only be spent
        in faith contexts).
\end{itemize}

A Synthesis Token may be spent to:
\begin{itemize}
  \item Improve Position by one step for a faith-based action (e.g.\ Controlled$\to$Controlled).
  \item Gain \textbf{+1d} on diplomacy between Temple and shrine factions.
  \item Declare \textbf{temporary safe passage} through otherwise hostile religious territory for
        a single scene (guards stand aside, elders agree to host, etc.).
\end{itemize}

Unused Tokens do not stack endlessly; the crew may hold at most \textbf{two} Synthesis Tokens at a
time.

\subsubsection*{Dual Oaths \& Spiritual Conflict}
\index{dual oaths}\index{Conditions!Spiritual Conflict}

Some characters swear both to a patron shrine and the Temple of Light.

You do not need a separate clock; instead, when you roll for a clearly faith-marked action while
holding \textbf{both} shrine-oath and Temple vow:

\begin{itemize}
  \item If the roll shows one or more \textbf{1s}, you must \textbf{choose} which faith is
        offended by how you acted.
  \item The offended side gains a narrative opening: colder reactions, whispers, or calls to
        prove yourself later.
\end{itemize}

If you refuse to choose:

\begin{itemize}
  \item Mark the \textbf{Spiritual Conflict} Condition: \emph{``Pulled Between Altars''}
        (\(-1\) die to faith actions until resolved).
  \item Resolve it by taking a clear, public stand in a dramatic scene: favor one faith, attempt
        a risky synthesis, or refuse both and accept the fallout.
\end{itemize}

\subsubsection*{Ritual Adaptation \& Contamination}
\index{ritual adaptation}\index{Contamination token}

When you perform shrine rites in explicitly Temple spaces (or Temple rites in fiercely shrine-held
space), you are stepping into dangerous territory.

\paragraph*{GM SB Use (Ritual Adaptation)}
When characters attempt mixed rites:
\begin{itemize}
  \item \textbf{1 SB}: Minor offense (wrong tool, clumsy phrasing) --- social awkwardness,
        hushed whispers.
  \item \textbf{2 SB}: Moderate tension --- rumors, open criticism, minor feud Clocks tick.
  \item \textbf{3+ SB}: Major conflict --- accusations of blasphemy, forced choices, or legal
        challenges.
\end{itemize}

\paragraph*{Contamination Tokens}
Additionally, if the rite is clearly ``in the wrong place,'' the GM may mark a
\textbf{Contamination Token} on the character or venue:

\begin{itemize}
  \item While holding a Contamination Token, you suffer \(-1\) die to \emph{all} faith-based rolls
        linked to that space or rite.
  \item Clear a Token by:
        \begin{itemize}
          \item Spending part of downtime on cleansing rites, reparations, or pilgrimages; or
          \item Converting it into growth: treat 1 Token as \textbf{1 XP} toward a faith-aligned
                talent or Devotion shift (you learned from your mistake, in public).
        \end{itemize}
\end{itemize}

\subsubsection*{Faith Ambassadors \& Flexible Belief}
\index{Ambassador Status}\index{Belief Flexibility}

Some characters become known as bridge-builders, mediators, or ``those who can speak to both
altars''.

\paragraph*{Ambassador Status}
A character who has \emph{successfully} mediated at least two serious faith conflicts (GM
judgment) may be granted \textbf{Ambassador Status}:

\begin{itemize}
  \item Mark a single checkbox or tag: \emph{Temple–Shrine Ambassador}.
  \item Once per session, convert a \emph{faith-based Position penalty} (e.g.\ from venue or
        Devotion) into an equivalent \textbf{bonus} for a single roll (e.g.\ Controlled$\to$Controlled),
        by explicitly leaning on their reputation as a neutral voice.
  \item Refresh Ambassador Status when you play out another mediation scene that matters.
\end{itemize}

\paragraph*{Optional Talent: Belief Flexibility}
For characters who intentionally walk the line, you can allow a minor talent:

\begin{description}
  \item[\textbf{Belief Flexibility} (4 XP)] Once per session, ignore faith-alignment Position
    penalties (from venue tags or Devotion) for a single scene, representing a knack for ``reading
    the room'' and speaking both languages.
\end{description}

Limit: You cannot benefit from both shrine \emph{and} Temple-specific mechanical advantages in the
same scene (pick which side of your faith is active).

\subsubsection*{Using Faith in Play}
\index{faith!at the table}

\begin{itemize}
  \item \textbf{Keep It Light} --- These tools hang off mechanics you already use (Position, SB,
        Tags, Conditions). If you are ever tracking more than Devotion, one venue tag, and maybe a
        Token or two, simplify.
  \item \textbf{Play For Drama} --- Use Devotion shifts and Synthesis Tokens to spotlight big,
        meaningful choices, not background noise.
  \item \textbf{Tie Into Saga} --- Faith choices make excellent fuel for the Saga system
        (\S\ref{sec:saga-system}): conversions, heresy trials, shrine restorations, and Temple
        victories are all things skalds love to sing about.
\end{itemize}

\section{Saga System: Epic Reputation and Legendary Deeds}
\label{sec:saga-system}
\index{Saga Reputation}\index{reputation!saga}\index{legendary deeds}

Not every story told in smoke-filled halls is true, but the ones that \emph{stick} shape how the
world meets you. The Saga System tracks not only what your crew has done, but how those deeds are
remembered, twisted, and invoked by others.

Saga Reputation sits alongside personal \textbf{Repute}, \textbf{Audience} tags, and \textbf{Oaths}.
Where those mark who knows you locally, Saga Reputation marks when your story starts to move on
its own.

\subsection{Crew Saga Reputation Track}
\index{Saga Reputation!track}

Each crew maintains a shared \textbf{Saga Reputation} track \([6]\) that grows as they perform
legendary deeds. This track represents how their exploits are told in halls across the northern
lands.

\begin{itemize}
  \item \textbf{Saga 1--2: Local Tales} --- Known by name in nearby settlements; a few skalds are
        watching your story.
  \item \textbf{Saga 3--4: Regional Fame} --- Songs and verses carry your deeds along trade routes
        and fjords; strangers recognize you by deeds, if not by face.
  \item \textbf{Saga 5--6: Living Legend} --- Your name is invoked in boasts, oaths, and threats.
        Children play at being you. Enemies plan around you.
\end{itemize}

\paragraph*{Who Tracks It?}
Saga Reputation is \emph{crew-level}, not individual. Different characters may be more or less
central to specific stories, but the world tends to remember ``the crew'' or ``the Linn from the
Mistship'' as a whole.

\subsection{Earning Saga Points}
\index{Saga Reputation!earning}

You raise Saga Reputation by earning \textbf{Saga Points}. Mark 1--2 Saga Points when the crew
does something that people will \emph{actually talk about} later.

\paragraph*{Core Triggers}
\begin{itemize}
  \item \textbf{Epic Victories} --- Defeating foes significantly above your Tier or overturning a
        seemingly unwinnable situation (e.g.\ defending a hall against impossible odds).
  \item \textbf{Impossible Feats} --- Succeeding at DV~5+ Position tests where the table agrees
        ``this is the kind of thing skalds brag about''.
  \item \textbf{Honor-Keeping} --- Fulfilling major oaths under extreme circumstances, especially
        when keeping the oath costs you dearly.
  \item \textbf{Legendary Sacrifices} --- Choosing the harder path for the greater good:
        voluntary maiming, loss of wealth, exile, or similar stakes.
  \item \textbf{Story-Shaping Decisions} --- A choice that permanently alters a hall, village,
        route, or region (toppling a jarl, ending a feud, founding a Temple or shrine).
\end{itemize}

\paragraph*{GM Guidance}
\begin{itemize}
  \item \emph{Ask:} ``Would someone in three winters still be telling this story at a thing?'' If
        yes, award Saga Points.
  0 \item \emph{Scale:} 1 Saga Point for strong deeds; 2 for true legends that also change fronts
        or clocks on the campaign map.
\end{itemize}

Each time the Saga track fills \([6/6]\), clear it and advance the crew’s \textbf{Saga Tier} by 1
(step their Saga Reputation band up to the next level). Optionally record a short \emph{Saga
Line} in the crew sheet: a one-sentence summary skalds repeat.

\subsection{Saga Benefits: Position \& Social Edge}
\index{Position!Saga benefits}

As Saga Reputation increases, the crew gains powerful Position and advantage benefits that reflect
how the world receives them.

\paragraph*{Saga 1--2: Local Heroes}
\begin{description}
  \item[\textbf{Local Hero}] Once per scene, when \emph{defending your home, kin, or named allies},
    one PC may improve Position by one step (Desperate$\to$Controlled or Controlled$\to$Controlled) on a
    single roll.
\end{description}

\paragraph*{Saga 3--4: Regional Champions}
\begin{description}
  \item[\textbf{Regional Champion}] Once per session, a PC may declare a specific past deed
    (\emph{``We held the bridge at Stormfjord for three days''}) that is now widely known. With GM
    approval, they gain \textbf{+1d} to a single roll where that story logically applies (rallying
    defenders, intimidating raiders, convincing a jarl to trust them, etc.).
\end{description}

\paragraph*{Saga 5--6: Living Legends}
\begin{description}
  \item[\textbf{Legendary Warrior}] Once per session, when the crew acts in a hall or region where
    their legend is well-known, one PC may treat any \emph{Controlled} Position as \emph{Dominant}
    for a single action, reflecting that the world \emph{expects} them to triumph.
\end{description}

\paragraph*{Optional: Individual Saga Marks}
Individual PCs can also record \textbf{Saga Marks} for personally famous deeds (see
\ref{sec:legendary-deeds}), but these are color until the GM calls on them for Position shifts,
Audience tags, or Strings.

\subsection{Saga Complications: Fame as a Knife}
\index{Saga Reputation!complications}

Saga Reputation is a double-edged axe. Fame brings attention, expectations, and enemies.

\begin{itemize}
  \item \textbf{Rival Challenges} --- At higher Saga levels, the GM is encouraged to introduce
        rival crews, warriors, or Temple champions who seek glory by besting you. For each Saga
        Tier above 2, once per arc start a \textbf{Rival Challenge [4]} clock.
  \item \textbf{Expectation Burden} --- Audience tags tilt toward
        \emph{``expects-impossible-feats''}. Failures in public scenes may tick additional fallout
        clocks as people feel \emph{betrayed}, not merely disappointed.
  \item \textbf{Target Status} --- Enemies who come specifically hunting the famous crew gain
        \textbf{+1d} to \emph{Assess, Track, or Ambush} rolls against them; their scouts already
        know your tactics and tales.
  \item \textbf{Saga Debt} --- The GM may introduce \textbf{Saga Debt} complications: a hall that
        sheltered you now calls in a favor, a child you inspired has gone missing hunting glory,
        or someone lied using your name.
\end{itemize}

As a rule of thumb, for each Saga Tier above 1, allow the GM to add one additional twist or
String per session based on fame alone.

\subsection{Saga Scenes: Fame Arrives Before You Do}
\index{Saga scenes}

When the crew’s Saga Reputation is \textbf{3+}, once per session a player may declare a
\textbf{Saga Scene}.

\begin{itemize}
  \item Set the scene in a location where your fame plausibly precedes you: a coastal hall, a
        trade-town, a Temple court that follows skald verse.
  \item The active PC begins with \textbf{Position +1} on their first significant social
        interaction (e.g.\ Controlled instead of Controlled).
  \item All allies in the scene gain \textbf{+1d once} when they explicitly invoke the crew’s
        legend (quoting a verse, showing a trophy, calling on a past deed).
  \item The scene must showcase the crew’s legendary nature in some way: public boast, challenge,
        judgment, or sacrifice. If it resolves quietly, the GM may rule that no new Saga Points
        are earned from it.
\end{itemize}

Saga Scenes are a spotlight tool: use them to turn ordinary negotiations or meetings into
memorable chapters in the crew’s story.

\subsection{Legendary Deeds \& Personal Boons}
\label{sec:legendary-deeds}
\index{legendary deeds}\index{boons!saga}

Certain deeds are so monumental that they grant permanent, personal benefits in addition to Saga
Points. These should be \emph{rare}, negotiated moments, not routine rewards.

\begin{description}
  \item[\textbf{Slayer of Beasts}] Defeat a creature of myth or a unique monster that has haunted
    a hall, route, or region for generations. Gain \textbf{+1d} on all future rolls directly
    confronting similar creatures or their kin.
  \item[\textbf{Oath-Keeper Supreme}] Fulfill an ``impossible'' oath that multiple NPCs declared
    unachievable (e.g.\ ``end this blood-feud in a single winter'', ``bring both shrine and Temple
    to the same altar''). Once per session, convert one \emph{single failed} roll related to oath,
    duty, or honor into a success at \emph{limited} Effect (mark 1~SB as cost).
  \item[\textbf{Shield of the Innocent}] Defend the defenseless against overwhelming odds in a
    public, widely-witnessed stand. Gain \textbf{+1 Armor} (non-stacking) when you explicitly
    protect others, even without a physical shield.
  \item[\textbf{Master of Elements}] Survive an impossible natural force (shipwreck, avalanche,
    volcanic ash, storm-ice) in a way that becomes a hall-story. Gain permanent \textbf{resistance}
    to one chosen environmental hazard (e.g.\ cold, storm, fire, sea).
\end{description}

\paragraph*{Saga Scars}
Optionally, pair each Legendary Deed with a \textbf{Saga Scar}: a visible mark, habit, or quirk
that signals the deed and may impose small costs (difficulty sleeping in calm weather, reflexive
shielding behavior, faint frostbite, etc.). These are narrative prompts the GM can tag for SB
spends.

\subsection{Saga Decline \& Fallen Heroes}
\index{Saga Reputation!decline}\index{Fallen Hero}

Saga Reputation is not guaranteed. It can decline when your story curdles.

\paragraph*{Loss Triggers}
\begin{itemize}
  \item \textbf{Oath-Breaking} --- Publicly failing to keep a major oath, especially one bound
        before many witnesses or under shrine/Temple.
  \item \textbf{Cowardice} --- Refusing to act when honor or duty clearly demands it, and others
        see this choice.
  \item \textbf{Betrayal} --- Turning against guests, sworn allies, or those under your protection.
  \item \textbf{Hubris} --- Actions that show dangerous overconfidence leading to tragic fallout
        (e.g.\ provoking needless war, squandering hard-won peace).
\end{itemize}

When one of these occurs in a public way, the GM may:
\begin{itemize}
  \item Reduce Saga Reputation by 1 step (e.g.\ from 4 to 3).
  \item Strip one Saga Benefit (e.g.\ you no longer count as \emph{Regional Champion}).
  \item Introduce a \textbf{Fallen Hero} Complication: former allies become rivals, children who
        idolized you turn bitter, or a skald begins composing a mocking counter-saga.
\end{itemize}

\paragraph*{Redemption Arcs}
Falling does not have to be the end:
\begin{itemize}
  \item Create a \textbf{Redemption Saga [6]} clock. When it fills through self-sacrifice,
        amends, or impossible tasks, you may regain lost Saga steps or transform your legend into
        a new shape: \emph{the Penitent}, \emph{the Oath-Restorer}, etc.
\end{itemize}

\subsection{Integration with Temple of Light \& Shrines}
\index{Temple of Light!Saga}\index{shrines!Saga}

Religious choice is part of your legend. Whether you stand with shrines, the Temple of Light, or
both, the songs will say something about it.

\paragraph*{Temple-Forward Sagas}
\begin{itemize}
  \item \textbf{Temple Conversion} --- When the crew openly converts to the Temple of Light and
        champions its causes, they may earn Saga Points for:
        \begin{itemize}
          \item Founding a new Temple or saving an existing one from ruin.
          \item Ending a Faith Feud through Temple law and mercy.
          \item Converting a powerful hall, jarl, or famed shrine-keeper.
        \end{itemize}
  \item Mechanically, treat these as \emph{Story-Shaping Decisions}. However, shrine-devout
        Audiences may assign negative tags (\emph{Oath-Turners}, \emph{Temple-Pawns}) and reduce
        certain Saga Benefits when dealing with traditionalists.
\end{itemize}

\paragraph*{Shrine-Devout Sagas}
\begin{itemize}
  \item Crews who visibly champion shrines against Temple expansion may gain extra Saga Points for:
        \begin{itemize}
          \item Restoring a desecrated shrine to power.
          \item Driving out abusive Temple agents without collapsing law.
          \item Crafting syncretic rites that keep old powers honored.
        \end{itemize}
  \item Temple Audiences may see them as \emph{heretics} or \emph{obstinate pagans}, increasing
        DV or lowering starting Position in Temple courts.
\end{itemize}

\paragraph*{Syncretic Sagas}
\begin{itemize}
  \item \textbf{Syncretic Path} --- Some crews walk between: swearing shrine-oaths while bearing
        Temple blessings. Their Saga tends to be \emph{contested}.
  \item Mechanically, they may earn Saga Points for brokering peace between faiths or inventing
        new shared rites, but:
        \begin{itemize}
          \item Start certain faith scenes with \textbf{mixed Audience tags} (\emph{Curious},
                \emph{Suspicious}, \emph{Conflicted}).
          \item The GM may tick \textbf{Dual Oath Conflict} or \textbf{Faith Feud} clocks whenever
                a syncretic act goes badly.
        \end{itemize}
\end{itemize}

\paragraph*{Heretic Legends}
\begin{description}
  \item[\textbf{Heretic Status}] If the crew openly defies Temple doctrine \emph{and} shrine
    agreements (e.g.\ consorting with banned powers, profaning both altars), their Saga might
    shift into a darker mode:
    \begin{itemize}
      \item Temple and shrine both gain license to hunt them; enemies gain \textbf{+1d} to find
            religious allies.
      \item At the same time, desperate halls, outcasts, and hidden cults might invoke them as
            \emph{necessary monsters} or \emph{storm-bringers}, granting unusual aid.
      \item Treat them as \emph{Legendary Adversaries} in some stories, \emph{secret saints} in
            others; Position swings can be dramatic.
    \end{itemize}
\end{description}

\subsection{Using Saga at the Table}
\index{Saga system!table use}

\begin{itemize}
  \item \textbf{For Players:} Think of Saga as the camera pulling back. When you chase Saga
        Points, you are choosing which scenes become the ones future characters will hear about.
  \item \textbf{For GMs:} Use Saga Reputation to choose who has heard of the crew, who cares, and
        how strongly. Let it steer front creation, rival crews, and which NPCs show up asking for
        help or vengeance.
  \item \textbf{For the Campaign:} Over time, the Saga System can track the arc from unknown
        raiders to hall-builders, faith-champions, outcast heretics, or founders of a new order.
        Decide together which saga you are writing.
\end{itemize}

\section{Glossary of Nordic Terms}

\subsection{Social and Political Terms}

\begin{description}
    \item[Aett] A clan or family grouping, often tracing descent from a common ancestor. The basic social unit in Linn society.
    
    \item[Blood-silver] Compensation paid to settle disputes and avoid feuds. A monetary or material payment that satisfies honor without violence.
    
    \item[Boast] A public declaration of intended action, creating social obligation. Failure to fulfill a boast damages reputation and may start feuds.
    
    \item[Feud] An ongoing conflict between families or crews, tracked mechanically. Allows for escalation and resolution through various means.
    
    \item[Lawspeaker] Traditional role at Things who memorizes and recites the law. Could be incorporated as a specialized crew role.
    
    \item[Repute] Social standing and honor within Linn society. Affects social interactions and access to opportunities.
    
    \item[Saga] Epic tales of heroic deeds that enhance reputation. Also represents the mechanical system for tracking legendary status.
    
    \item[Skald] A poet-singer who composes and performs verses about heroic deeds. Important for maintaining and spreading reputation.
    
    \item[String] A favor, connection, or obligation that can be called upon. Represents social capital and relationships.
    
    \item[Thing] Political assembly where laws are made, disputes settled, and oaths sworn. Winter gatherings for community decision-making.
    
    \item[Thinglet] Smaller, informal assembly for local matters, often held within a hall.
    
    \item[Wyrd] Fate or doom, often settled through ordeal or trial by combat when legal resolution fails.
\end{description}

\subsection{Maritime and Travel Terms}

\begin{description}
    \item[Beaching Hull] Ship design feature allowing vessels to land directly on shore for easy access and escape.
    
    \item[Broadshore] Cargo vessel type with higher freeboard and stouter construction, slower but more seaworthy than raiding ships.
    
    \item[Gare] Local term for fog or mist conditions, particularly those affecting navigation and visibility.
    
    \item[Longhull] Sleek raiding vessel with shallow draft and fast oars, designed for speed and beach landings.
    
    \item[Mistwise] Specialized knowledge of navigating in foggy conditions, a valuable crew skill.
    
    \item[Pilot Stone] Cairn or marker used for navigation in familiar waters, often moved to confuse rivals.
    
    \item[River-spear] Light, narrow vessel designed for inland waterways, often with collapsible mast.
    
    \item[Shallow Draft] Ship design allowing navigation in waters too shallow for deeper vessels.
    
    \item[Skerry] Small rocky island or reef, often hazardous to navigation but useful for ambushes.
\end{description}

\subsection{Cultural and Religious Terms}

\begin{description}
    \item[Hospitable] Cultural value emphasizing the sacred duty to provide guest-right and proper hospitality.
    
    \item[Muster] Spring gathering of crews and preparation for summer activities.
    
    \item[Oathbound] Character committed to keeping vows and promises, regardless of cost.
    
    \item[Riverwise] Specialized knowledge of river navigation, hazards, and local customs.
    
    \item[Shrine Warding] Ritual protection of settlements or territories through sacred sites and ceremonies.
    
    \item[Wolf-Banner] Warrior tradition or crew identity associated with fierce combat and boarding actions.
\end{description}

\subsection{Seasonal Terms}

\begin{description}
    \item[Hearth Time] Winter period of indoor activities, maintenance, and community bonding.
    
    \item[Muster] Spring assembly for preparation and crew organization.
    
    \item[Raids] Summer period of seaborne expeditions and military activities.
    
    \item[Settling] Autumn period for resolving disputes, trade negotiations, and harvest activities.
    
    \item[Thing Season] Winter period when assemblies are held due to weather and availability.
\end{description}

\subsection{Combat and Conflict Terms}

\begin{description}
    \item[Board \& Brace] Naval combat action involving boarding enemy vessels and fighting hand-to-hand.
    
    \item[Wolf-Boarding Gear] Specialized equipment for ship-to-ship combat, including hooks, ropes, and boarding planks.
\end{description}

\end{document}