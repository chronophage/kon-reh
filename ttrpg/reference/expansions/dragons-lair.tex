\documentclass[11pt,letterpaper]{article}
\usepackage[utf8]{inputenc}
\usepackage{geometry}
\usepackage{array}
\usepackage{booktabs}
\usepackage{multicol}
\usepackage{enumitem}
\usepackage{setspace}
\usepackage{titlesec}
\usepackage{titling}

\geometry{margin=1in}
\setlength{\parskip}{0.8em}
\setlength{\parindent}{0em}

\titleformat{\section}{\large\bfseries\uppercase}{\thesection}{1em}{}
\titleformat{\subsection}{\bfseries}{\thesubsection}{1em}{}

\newcommand{\sectionheader}[1]{\section*{#1}}
\newcommand{\subsectionheader}[1]{\subsection*{#1}}

% =========================================================
% TITLE PAGE
% =========================================================

\title{\textbf{\Huge Dragon's Lair}\\[0.75em]
\Large A Fate's Edge Supplement}
\author{}
\date{}

\begin{document}
\maketitle
\begin{center}
\emph{Where ancient wings blot out the sun, and the earth remembers every claw.\\
Dragons are not beasts. They are epochs with teeth.}
\end{center}

\vspace{2em}

\sectionheader{Introduction}

Dragons in \textbf{Fate's Edge} are more than apex predators or piles of gold atop a nest. They are:
\begin{itemize}[leftmargin=*]
\item elder sovereignties wrapped in scale and fire,
\item tyrants and sages older than cities,
\item walking catastrophes whose slumber shapes continents.
\end{itemize}

A dragon is a \emph{force of narrative gravity}. When one wakes, kingdoms bend.  
When one dies, the world changes.

\subsectionheader{What This Supplement Adds}
\begin{itemize}[leftmargin=*]
\item \textbf{A complete Dragon Lair Generator} — rooms, traps, hazards, ancient puzzles, and living geography.
\item \textbf{Hoard Generator} — treasure worth dying for, and curses worth fleeing from.
\item \textbf{Named Elder Dragons} — unique sovereigns with motives, voices, and agendas.
\item \textbf{Knightly Orders \& Gaesea} — mortal traditions forged under draconic shadow.
\item \textbf{Minor Wyrms, Drakes, \& Cults} — lesser terrors and those who worship them.
\item \textbf{Runekeeper Patron Rules} — how mortals gain draconic power without losing themselves.
\item \textbf{Adventure Seeds \& Campaign Arcs} — lairs that swallow parties whole.
\end{itemize}

\subsectionheader{Tone \& Playstyle}

\textbf{Dragon's Lair} is written to support:
\begin{itemize}[leftmargin=*]
\item deadly delves,
\item moral and political dilemmas,
\item mythic bargains,
\item and the uneasy awe of meeting something older than hope.
\end{itemize}

This is not a bestiary update. It is a \textbf{mythic escalation}.  
A dragon is a rival kingdom, a natural disaster, and a thinking god.

\subsectionheader{Using This Book}

Each section stands alone, but together they let you:
\begin{itemize}[leftmargin=*]
\item build lairs on the fly,
\item create new dragons with rules and personality,
\item tie entire campaigns to a single ancient sovereign.
\end{itemize}

No two dragons are the same.  
Every lair breathes.  
Every hoard has a history.

When you enter a dragon’s domain, remember:

\begin{center}
\textbf{It is not the dragon’s home.}\\
\textbf{It is the world that belongs to the dragon.}
\end{center}

\newpage



\section*{DRAGONS: GENERATORS (Lairs, Hoards, and Terrestrial Allegiances)}

\Note{Use these tables fast: pick or roll. Results stack. Where a result says ``Ask a Player,'' treat it as a prompt that grants a Clue if they answer honestly. All \Tag{TAGS} are Fate's Edge effects you can apply as scene traits, treasures, or obstacles.}

% =========================================================
% 1) LAIR GENERATOR
% =========================================================
\section{Lair Generator}

\subsection{Step 1: Domain (Roll \dN{12})}
\begin{dtable}{\dN{12}}
1  & Volcanic Caldera; magma lenses; ash snows \Tag{HEAT} \Tag{IGNITE} \\
2  & Storm Fortress upon Anvil-Clouds; chained towers \Tag{WIND} \Tag{LIGHTNING} \\
3  & Desert Canyon of Petrified Titans; fossil courts \Tag{STONE} \Tag{ECHO} \\
4  & River Gorge; mirrored pools, bone dams \Tag{WATER} \Tag{REFLECTION} \\
5  & Black Pines; trees bow when the dragon dreams \Tag{WOOD} \Tag{DREAD} \\
6  & Sunken Metropolis; air-bubble halls \Tag{PRESSURE} \Tag{DROWN} \\
7  & Glacier Cathedral; blue caverns sing \Tag{COLD} \Tag{RESOUND} \\
8  & Subterranean Geode; crystal weather \Tag{SHARD} \Tag{GLARE} \\
9  & Sky Needles; knife peaks, updraft mazes \Tag{HEIGHT} \Tag{LETHAL FALL} \\
10 & Salt Flats; mirage gates, glass currents \Tag{MIRAGE} \Tag{CORRODE} \\
11 & Ruined City Crown; throne of broken oaths \Tag{OATH} \Tag{WARD} \\
12 & Dream-Scar; overlap of waking and myth \Tag{ONEIRIC} \Tag{UNREAL} \\
\end{dtable}

\subsection{Step 2: Territory Signs (Roll \dN{8}, pick 2)}
\begin{dtable}{\dN{8}}
1 & Tracks flow uphill; rivers reverse at dusk. \\
2 & Herds kneel at noon facing the lair. \\
3 & Lightning strikes only marked stones. \\
4 & Ash carries whispers that name trespassers. \\
5 & Trees lean; boughs form an archway road. \\
6 & Fossils turn to watch; mouths shape words. \\
7 & Coins in pockets sweat; metal tastes of blood. \\
8 & Breath fog forms runes warning a tithe is due. \\
\end{dtable}

\subsection{Step 3: Lair Heart (Roll \dN{6})}
\begin{dtable}{\dN{6}}
1 & Hoard Chamber arranged as a \emph{memory theatre}; narrative paths \Tag{SCRY}. \\
2 & Throne of Element; terrain obeys the dragon’s breath \Tag{AREA} \Tag{FORCE}. \\
3 & Court of Oaths; binding circles, herald-gargoyles \Tag{BIND} \Tag{WARD}. \\
4 & Egg Crypt; slumbering brood, choral heartbeat \Tag{AWE}. \\
5 & Observatory of Scars; sky windows show bygone wars \Tag{DIVINE}. \\
6 & Ossuary Forge; bones smelted into relics \Tag{RELIC} \Tag{BLASPHEMY}. \\
\end{dtable}

\subsection{Step 4: Approaches \& Hazards (Roll \dN{10}, pick 2–3)}
\begin{dtable}{\dN{10}}
1 & Knife-ledge path; crosswinds every beat (\Tag{TEST: Mobility}). \\
2 & Ashfall; smothers light; torches fail \Tag{SUFFOCATE}. \\
3 & Flood-tunnel; pressure doors; shifting currents. \\
4 & Fossil courts demand a plea; fail → turned aside. \\
5 & Glass rain; movement causes cuts (\Tag{Harm 1}, mitigable). \\
6 & Null updraft; flight negated; climbing ropes mandatory. \\
7 & Siren vents; infrasound lures gear from hands. \\
8 & Mirror-pool maze; reflections lie about exits. \\
9 & Avalanche memory; any loud noise triggers rockfall. \\
10 & Sky-leeches; drain Stamina (\Tag{Fatigue +1} on linger). \\
\end{dtable}

\subsection{Step 5: Lair Moves (use 1–3 as soft/hard moves)}
\begin{itemize}
\item \textbf{Reposition the World}: stairs invert; Near becomes Far; archways swallow pursuit.
\item \textbf{Demand the Tithe}: the place itself asks a price (blood, vow, name, memory).
\item \textbf{Answer With Weather}: the dragon emotes via storm/quake/flood for a beat.
\item \textbf{Arouse the Hoard}: objects animate to defend a \emph{story} (not a room).
\item \textbf{Close the Eye}: vision narrows; ranged actions become Desperate until you change position.
\end{itemize}

\subsection{Step 6: Denizens (Roll \dN{8}, 1–2 results)}
\begin{dtable}{\dN{8}}
1 & Drake-Knights (Tier II) patrolling. \\
2 & Hoard-Wyrds (Tier II–III) bonded to artifacts. \\
3 & Sky Serpents (Tier III) nesting. \\
4 & Fossil Jurors (Tier II) animate to judge intruders. \\
5 & Ember Imps (Tier I) trade heat for secrets. \\
6 & River Wights (Tier II) bound by drowned oaths. \\
7 & Glass Wyrmlings (Tier II) mirror and shatter. \\
8 & Herald Gargoyles (Tier II) enforce etiquette. \\
\end{dtable}

\subsection{Quick Build Procedure (2 minutes)}
\begin{enumerate}
\item Roll Domain, pick two Territory Signs.
\item Choose one Lair Heart and 2–3 Approaches/Hazards.
\item Pick 1–2 Denizens and 2 Lair Moves.
\item State a \textbf{Tithe}: ``What must mortals surrender to pass?''
\item Name a \textbf{Violation}: the one act that wakes the dragon immediately.
\end{enumerate}

% =========================================================
% 2) HOARD / TREASURE GENERATOR
% =========================================================
\section{Hoard Generator}

\Note{A dragon’s hoard is a \emph{biography}. Build by strata. Use \dN{12} within each stratum; pick 1–2 per stratum for a compact hoard, or roll all for a mythic hoard. Apply \Tag{TAGS} as treasure properties or scene traits.}

\subsection{Stratum A: Coin \& Commodity (\dN{12})}
\begin{dtable}{\dN{12}}
1 & Scales of electrum stamped with extinct dynasties. \\
2 & River-pearls that hum in rain \Tag{WATER}. \\
3 & Salt-bricks worth a city’s winter. \\
4 & Demon-minted coppers—warm to touch \Tag{BLIGHT}. \\
5 & Glass coins: spend with a lie, shatter on truth. \\
6 & Star-iron beads (forge-grade) \Tag{RELIC}. \\
7 & Amber lumps with insects that whisper routes. \\
8 & Scrip of a fallen bank; redeemable with the right story. \\
9 & Temple tithes bound with blue twine (oath-sealed). \\
10 & Trade bars etched with river-depth marks (pilot’s tools). \\
11 & Coral crowns; brittle but potent at sea \Tag{COMMAND (sailors)}. \\
12 & Black spice; anesthetic or poison by dose. \\
\end{dtable}

\subsection{Stratum B: Art \& Relic (\dN{12})}
\begin{dtable}{\dN{12}}
1 & Masks of the Seven Winds (sing for \Tag{WIND} once/scene). \\
2 & Tapestry that updates current wars nightly. \\
3 & Bone flutes that call extinct birds. \\
4 & Chalice that refuses poisoned liquid. \\
5 & Crown of Oaths: swearing while wearing binds magically \Tag{BIND}. \\
6 & Twin mirrors: speak into one, echo from the other at midnight. \\
7 & Zodiac astrolabe; points to eclipses \Tag{DIVINE}. \\
8 & Ember harp; strings ignite when lies are told nearby. \\
9 & Saint’s gauntlet; unburnt by any fire. \\
10 & Library sigil-stamp: seals are obeyed by lesser courts. \\
11 & Fossil codex; pages of shale can be read once each. \\
12 & Lantern with bottled dawn (3 uses) \Tag{LIGHT}. \\
\end{dtable}

\subsection{Stratum C: Names, Oaths, and Intangibles (\dN{12})}
\begin{dtable}{\dN{12}}
1 & A king’s true name in a lead vial. \\
2 & A bridge’s right-of-way (you collect the toll). \\
3 & Seven unspent apologies; each removes one curse. \\
4 & A ship’s luck bound in twine; cut to claim it. \\
5 & The deed to a moonlit crossroads. \\
6 & A treaty’s missing clause; add it to rewrite borders. \\
7 & The last lullaby of a nation; sing to calm armies. \\
8 & A festival’s first toast; begin it to unite feuds. \\
9 & Weather’s favor over one valley for a year. \\
10 & The memory of a siege ladder; place it to open a gate. \\
11 & A duel’s outcome, never fought; declare to bind fate. \\
12 & The debt of a cathedral to its mason’s bloodline. \\
\end{dtable}

\subsection{Stratum D: Cursed Complications (\dN{10})}
\begin{dtable}{\dN{10}}
1 & Taking coins awakens a Hoard-Wyrd. \\
2 & The art objects are witnesses; they may testify in court. \\
3 & Removing any crown asserts a claim; rivals appear. \\
4 & Intangibles are tracked by Herald Gargoyles. \\
5 & A rival dragon has a lien on half the hoard. \\
6 & Curse of Counting: must tally treasure nightly or suffer \Tag{Fatigue +1}. \\
7 & Hoard Scent: predators pursue the bearer. \\
8 & Echo of Theft: the original owners dream of you. \\
9 & Oath-Magnet: you attract sworn duels. \\
10 & Tide-Tithe: waters reclaim 10\% during each full moon. \\
\end{dtable}

\subsection{Quick Hoard Build}
\begin{enumerate}
\item Pick 1–2 from A, 1–2 from B, 1 from C, 1 from D.
\item Assign 1–2 \Tag{TAGS} as properties (e.g., \Tag{WARD}, \Tag{COMMAND}, \Tag{LIGHT}).
\item State a \textbf{Hoard Law}: who may touch what (breaking it triggers a Lair Move).
\end{enumerate}

% =========================================================
% 3) TERRESTRIAL ALLEGIANCES (MORTAL PATRONS)
% =========================================================
\section{Terrestrial Allegiances Generator}

\Note{Mortal patrons use the same cadence as mystic patrons but grounded: \textbf{Boons}, \textbf{Obligations}, \textbf{Claims}, \textbf{Fallout}. Build one with the tables below or roll to discover who steps into the dragon’s shadow.}

\subsection{Step 1: Who Are They? (\dN{12})}
\begin{dtable}{\dN{12}}
1 & River Guildmaster (controls ferries and floodgates). \\
2 & Air-Navy Commodore (skyships, storm-anchors). \\
3 & Fossil Court Magistrate (petrified law). \\
4 & Ember Syndicate Kapitan (ash trade, hot metals). \\
5 & Abbot of the Seven Vows (oath economy). \\
6 & Royal Cartographer (maps that make roads true). \\
7 & Archivist of Banned Names (permits and erasures). \\
8 & Duchess of the Green March (beast levies). \\
9 & Master of the Glassworks (mirrors, lenses, spies). \\
10 & Harbor Warden (customs, tides, quarantine). \\
11 & Warden of the Sky-Needles (peak fortresses). \\
12 & Speaker for the Displaced (refugee armadas). \\
\end{dtable}

\subsection{Step 2: What Do They Want? (\dN{10})}
\begin{dtable}{\dN{10}}
1 & Bind or divert the dragon, not kill it. \\
2 & Monopolize a tithe (they profit forever). \\
3 & Recover a stolen oath/name from the hoard. \\
4 & Weaponize dragon weather against a rival. \\
5 & Legalize a new rite or outlaw an old one. \\
6 & Seat on the Fossil Court. \\
7 & Move a city; redraw a river. \\
8 & Elevate their house via draconic heraldry. \\
9 & Provoke rival dragon to war (then broker peace). \\
10 & Break the ancient pact (free their people). \\
\end{dtable}

\subsection{Step 3: Boons (choose 1–2)}
\begin{itemize}
\item \textbf{Seal of Passage} \Tag{WARD}: ignore one lair hazard once/scene for bearers.
\item \textbf{Leveraged Favor} \Tag{COMMAND}: call 1 squad of specialists when in their domain.
\item \textbf{Licensed Rite}: learn 1 \emph{legal} minor rite without heat.
\item \textbf{Logistics Surge}: +1 Supply per delve while allied; revoke on betrayal.
\item \textbf{Heraldry of Safe-Conduct}: neutral status among denizens who honor law.
\end{itemize}

\subsection{Step 4: Obligations (pick 1, \dN{6} for flavor)}
\begin{dtable}{\dN{6}}
1 & Pay a tithe (coin, captured relic, or name) each session. \\
2 & Render one unpleasant task without question. \\
3 & Uphold their edict in the field; report violations. \\
4 & Carry their mark openly; accept legal consequences. \\
5 & Share first pick of hoard items that match their aim. \\
6 & Do not slay the dragon without their assent. \\
\end{dtable}

\subsection{Step 5: Claims \& Fallout}
\begin{itemize}
\item \textbf{Claim (they assert)}: jurisdiction over a route, rite, or rumor.
\item \textbf{Fallout on Betrayal}: lose Boons; gain \Tag{OUTLAW} tag; a bounty or legal curse.
\item \textbf{Escalation Track} (4): \emph{Notice} $\rightarrow$ \emph{Audit} $\rightarrow$ \emph{Sanction} $\rightarrow$ \emph{Seizure}. Tick on refusal or deception.
\end{itemize}

\subsection{Quick Allegiance Build}
\begin{enumerate}
\item Roll \textbf{Who} and \textbf{Want}.
\item Pick Boons that further that Want.
\item Set 1 Obligation and the Escalation Track position.
\item Name the \textbf{Shared Enemy} (often a rival patron or a dragon cult).
\end{enumerate}

% =========================================================
% GM QUICK START CARD
% =========================================================
\section{GM Quick Start (Index Card)}
\begin{itemize}
\item \textbf{Lair}: Domain \& Heart; state Tithe and Violation.
\item \textbf{Two Lair Moves}: Reposition the World; Demand the Tithe.
\item \textbf{Denizens}: 1 patrol, 1 guardian.
\item \textbf{Hoard}: 1 Coin, 1 Art, 1 Intangible, 1 Complication; set Hoard Law.
\item \textbf{Terrestrial Patron}: Who + Want; 1 Boon, 1 Obligation; Escalation = Notice.
\end{itemize}

% =========================================================
% NAMED ELDER DRAGONS
% =========================================================
\section{Named Elder Dragons}

\Note{Use these as legendary fixtures: a dragon is a setting choice, not a single encounter. Their lairs and hoards reflect their personality, memories, and the debts of nations.}

% ---------------------------------------------------------
\subsection{Vyrmja the Winter Coil (The Linnic Wyrm)}
\begin{itemize}
\item \textbf{Domain}: Glacier Cathedral (singing ice, hollow spires)
\item \textbf{Disposition}: Patient, judicial, slow to speak and slower to forgive
\item \textbf{Lair Moves}: 
  \begin{itemize}
  \item \emph{Seal the Throat}: Ice doors grow shut; PCs must negotiate or cut through echoing frost.
  \item \emph{Witness of Snow}: Falling frost reveals every footstep, lie, and hidden name.
  \end{itemize}
\item \textbf{Legend}: Vyrmja remembers every betrayal carved into her scales—each scar is a story.
\item \textbf{Hoards}: 
  \begin{itemize}
  \item Frost-bloom sapphires that sing regrets.
  \item Treaties frozen into slabs of river-ice (still legally binding).
  \item The first lost name of a king; speak it to end a bloodline.
  \end{itemize}
\item \textbf{Boons}: 
  \begin{itemize}
  \item \textbf{Winter Oath} \Tag{WARD}: as long as you keep your sworn word, winter spirits ignore you.
  \item \textbf{Glacial Memory}: once/session, Vyrmja recounts a forgotten truth from centuries past.
  \end{itemize}
\item \textbf{Price}: You owe her the truth of your lineage—if unknown, she will take blood for proof.
\item \textbf{Why She Wakes}: Broken pacts, defiled graves, fire-rites in frozen lands.
\end{itemize}

% ---------------------------------------------------------
\subsection{Azghal of the Red Vault (The Aelaerem Doomfire)}
\begin{itemize}
\item \textbf{Domain}: Ruined City Crown—throne of broken oaths and molten gates
\item \textbf{Disposition}: Eloquent tyrant; thinks in conquests, currencies, and catastrophes
\item \textbf{Lair Moves}: 
  \begin{itemize}
  \item \emph{Tithe by Ash}: Any spoken falsehood ignites as choking smoke.
  \item \emph{Molten Ledger}: Gold flows uphill to write debts in liquid script along the walls.
  \end{itemize}
\item \textbf{Legend}: Built his hoard from \emph{payments of surrender}. Entire nations paid tribute in hopes he’d sleep forever.
\item \textbf{Hoards}:
  \begin{itemize}
  \item Crowns bent into collars.
  \item Vaults of conquered treaties, each with a missing clause.
  \item Relics he thrones on: each burned clean of its former bearer.
  \end{itemize}
\item \textbf{Boons}:
  \begin{itemize}
  \item \textbf{Doomfire Brand} \Tag{IGNITE} \Tag{COMMAND}: speak a command word; flames mark a liar’s tongue.
  \item \textbf{Cindersworn Heraldry}: soldiers step aside rather than challenge your passage.
  \end{itemize}
\item \textbf{Price}: Must bring him a symbol of pride taken from another—humiliation fuels his hoard.
\item \textbf{Why He Wakes}: Armies mass, vaults open, kings forget fear.
\end{itemize}

% ---------------------------------------------------------
\subsection{Tyrgoth the Thunder-Eater (Ykrul Trial Dragon)}
\begin{itemize}
\item \textbf{Domain}: Sky Needles—knife peaks, broken storm-altars
\item \textbf{Disposition}: Pure instinct and honor; words mean nothing, deeds everything
\item \textbf{Lair Moves}:
  \begin{itemize}
  \item \emph{Sky-Split}: lightning carves paths between foes; melee becomes a gauntlet.
  \item \emph{Bone Bellows}: roar turns stone to splinters; must keep footing or fall.
  \end{itemize}
\item \textbf{Legend}: The Ykrul tell of champions who challenge Tyrgoth for three breaths: if they survive, they earn glory. If not—he remembers their courage forever.
\item \textbf{Hoards}:
  \begin{itemize}
  \item Weapons that shattered against his scales (kept as trophies).
  \item Shards of peak-altars struck by lightning.
  \item The ashes of fallen challengers held in iron urns.
  \end{itemize}
\item \textbf{Boons}:
  \begin{itemize}
  \item \textbf{Storm-Mark}: gain \Tag{WIND} on leaps and \Tag{AREA} on shouts once/session.
  \item \textbf{Honor of the Dead}: you may request a name from the urns—use it as Inspiration once.
  \end{itemize}
\item \textbf{Price}: One blow, honestly struck—on dragon-scale. Harm him, or be found wanting.
\item \textbf{Why He Wakes}: Cowards rule, mountains bow, or a false boast reaches his ears.
\end{itemize}

% ---------------------------------------------------------
\subsection{Sir Cadmorrant The Gilded (Chivalric Golden Drake)}
\begin{itemize}
\item \textbf{Domain}: Valley of Heralds—ruined chapels, broken tilting-grounds, gilded bones
\item \textbf{Disposition}: Regal, prideful, obsessed with honor and reputation
\item \textbf{Lair Moves}:
  \begin{itemize}
  \item \emph{Court of Challenge}: intruders must name titles and lineage or face his wrath.
  \item \emph{Knight’s Charge}: illusions of undead lancers ride beside him.
  \end{itemize}
\item \textbf{Legend}: Cadmorrant once demanded knights swear oaths of defense to his valley. When they failed, he took their banners and bones for his heraldic host.
\item \textbf{Hoards}:
  \begin{itemize}
  \item Suits of armor filled with golden dust—empty but vigilant.
  \item Banners soaked in sunlight; shine too bright for false heraldry.
  \item Scepters bearing the marks of noble houses long erased.
  \end{itemize}
\item \textbf{Boons}:
  \begin{itemize}
  \item \textbf{Sun-Banner}: reveal a banner to frighten lesser foes; \Tag{FEAR} until the end of scene.
  \item \textbf{Knight’s Courtesy}: parley before claws—Cadmorrant grants terms.
  \end{itemize}
\item \textbf{Price}: Speak a lineage he has never heard—true or false—and bear its consequences.
\item \textbf{Why He Wakes}: Dishonor spreads, heraldry is forged, or a fallen knight seeks vengeance.
\end{itemize}

% =========================================================
% KNIGHTLY ORDERS, GEASA, AND MINOR DRAGONS
% =========================================================
\section{Knightly Orders, Geasa, and Minor Dragons}

\Note{Knights are political weather vanes as much as warriors. A geis (plural: geasa) is the sharp edge of honor; it binds meaning into the world. Minor dragons are the living barometer of frontier fate.}

% ---------------------------------------------------------
\subsection{Knightly Orders}

\subsubsection{Order of the Sun-Banner}
\textit{Heraldry}: Sun on scarlet field; pennons stitched with gold thread.\\
\textit{Ethos}: Radiant mercy, public vows, honor-by-daylight.

\paragraph{Tenets}
\begin{itemize}[leftmargin=*]
\item Never draw first blood in shadow.
\item Keep faith with the poor and the pledged.
\item Names are to be spoken clearly; titles weighed.
\end{itemize}

\paragraph{Moves}
\begin{itemize}[leftmargin=*]
\item \textbf{Proclaim Terms}: Before violence, declare terms; if accepted, gain \textit{Dominant} Position for the first exchange.
\item \textbf{Sun-Blessed Stand}: When you shield another, mark 1 Fatigue to give them +1d and \Tag{WARD} for this beat.
\end{itemize}

\paragraph{Order Rite (DV 3)} \Tag{LIGHT} \Tag{WARD}
Unfurl a luminous banner; lesser foes must hesitate or test \textit{Resolve} at --1d to act against you for a scene. \textit{Backlash}: Your name spreads; a rival order learns your route.

\paragraph{Obligation (4-segment)}
\begin{itemize}[leftmargin=*]
\item \textbf{Dues of Mercy}: Rescue a declared innocent within three days of plea.
\item \textbf{Violation}: If you refuse a plea in daylight, tick Obligation twice and lose access to the Rite until you atone.
\end{itemize}

\paragraph{Boons \& Favors}
\begin{itemize}[leftmargin=*]
\item Sanctuary at templar hostels; \Tag{SUPPLY} 1 once per delve.
\item \textbf{Sun-Banner} (relic): Reveal to impose \Tag{FEAR} on Tier I mobs for one beat.
\end{itemize}

\paragraph{Rivalries}
Despises oath-courts that bargain in secret. Touchy peace with Sir Cadmorrant’s valley heralds.

\bigskip

\subsubsection{Thorn-Guard Compact}
\textit{Heraldry}: Black bramble on iron-grey.\\
\textit{Ethos}: Keep borders. Hold lines. Speak little.

\paragraph{Tenets}
\begin{itemize}[leftmargin=*]
\item No trespass without toll or task.
\item The hedge remembers every cut.
\item Winter keeps what winter is owed.
\end{itemize}

\paragraph{Moves}
\begin{itemize}[leftmargin=*]
\item \textbf{Set the Hedge}: Mark 1 Supply to lace thorns; enemies entering a Zone act at \textit{Risky} and suffer \textit{Harm 1 (Bleed)} on a miss.
\item \textbf{Countermarch}: When the enemy surges, step back one Zone with formation intact; cancel their positional gain.
\end{itemize}

\paragraph{Order Rite (DV 4)} \Tag{BIND} \Tag{EARTH}
Raise a living briar wall; \Tag{WALL} across a chokepoint for a scene. \textit{Backlash}: The hedge wants payment—sacrifice blood or a keepsake (lose a tag) or suffer \textit{--1 die} until rest.

\paragraph{Obligation (6-segment)}
\textbf{Border Dues}: Patrol a named boundary each new moon. Missing a circuit ticks twice and invites a supernatural crossing.

\paragraph{Boons}
Custom \textit{thornmail} (Armor: counts as 2 vs. \textit{Grapple/Pull}); hedge-wives share \textit{old-road} maps.

\bigskip

\subsubsection{Order of the Glass Vow}
\textit{Heraldry}: Clear chalice on midnight blue.\\
\textit{Ethos}: Transparent oaths, public records, duels by witness.

\paragraph{Tenets}
\begin{itemize}[leftmargin=*]
\item No sealed bargains.
\item A liar stands alone.
\item Mercy is written first, then wrath.
\end{itemize}

\paragraph{Moves}
\begin{itemize}[leftmargin=*]
\item \textbf{Read the Cut}: When speech and steel mingle, ask: ``What part of this is pretense?'' GM answers plainly; gain +1d to act against the pretense this beat.
\item \textbf{Chalice Interdict}: Hold the chalice aloft; violence pauses for one exchange unless a party breaks faith (they act \textit{Desperate} if they do).
\end{itemize}

\paragraph{Order Rite (DV 3)} \Tag{TRUTH} \Tag{BIND}
Inscribe the terms of a dispute into sand or glass; those present cannot knowingly contradict the text without suffering \textit{Harm 1 (Burn)} each time. \textit{Backlash}: The text clings—any later edits cost 1 Stress per change.

\paragraph{Obligation (5-segment)}
\textbf{Record of Account}: File a public writ after any duel or negotiation you mediate. Failure breeds rumor: --1d to sway officials for a session.

\paragraph{Boons}
Access to public archives; \textit{Glass Signet} (once/session: certify a truth; NPCs in earshot must treat it as if witnessed).

\bigskip

\subsubsection{Ashen Spur Brotherhood}
\textit{Heraldry}: Silver spur over coal-black wing.\\
\textit{Ethos}: Ride fast, end threats, no graves for tyrants.

\paragraph{Tenets}
\begin{itemize}[leftmargin=*]
\item Strike first against monsters that stalk roads.
\item No ransom for slavers or marauders.
\item A swift horse is a citizen.
\end{itemize}

\paragraph{Moves}
\begin{itemize}[leftmargin=*]
\item \textbf{Break the Line}: On a mounted charge from Far to Near, gain +2d and \Tag{AREA} for the first clash; then mark 1 Fatigue.
\item \textbf{Spur Smoke}: Kick up dust/ash: create \textit{cover} for one exchange; ranged foes act at --1d.
\end{itemize}

\paragraph{Order Rite (DV 2)} \Tag{WIND} \Tag{SPEED}
Whistle the Spur Cant; mounts and riders move one Zone without provoking. \textit{Backlash}: Your trail is obvious; next ambush against you gains +1d.

\paragraph{Obligation (4-segment)}
\textbf{Road Tithe}: Answer posted horn-calls within a week. Missed calls convert to bounties against your name.

\paragraph{Boons}
Relay shelters; \textit{Greywind Spurs} (ignore the first \textit{Entangle/Slow} each delve).

% ---------------------------------------------------------
\subsection{Geasa \& Draconic Bonds}

\subsubsection{Swearing a Geis (Player-Facing Rule)}
\begin{itemize}[leftmargin=*]
\item \textbf{Propose Terms}: A concrete vow with a trigger and a limit (scope, place, time).
\item \textbf{Seal It}: Choose a seal (blood, name, relic, witness). Mark 1 Obligation (Geis) per magnitude of the boon.
\item \textbf{Boon}: Gain a situational tag (e.g., \Tag{WARD}, \Tag{FEAR}, \Tag{WIND}) or +1d in the vowed context.
\item \textbf{Breakage}: If you knowingly break the geis, suffer \textit{Backlash} and tick all Obligation segments at once.
\end{itemize}

\paragraph{Magnitude Table}
\begin{tabular}{@{}lll@{}}
\toprule
\textbf{Scope} & \textbf{Boon} & \textbf{Obligation}\\
\midrule
Duel/Scene & +1d or 1 situational Tag & 2-segment \\
Quest/Arc & +1d and 1 Tag; or 2 Tags & 4-segment \\
Season/War & 2 Tags and special clause & 6-segment \\
\bottomrule
\end{tabular}

\paragraph{Backlash on Break}
Roll 1d6:
\begin{itemize}[leftmargin=*]
\item 1: \textit{Harm 2 (Ban)} vs. your gift (cannot use granted Tag this session).
\item 2: \textit{Truth Leech}: you lose a memory tied to the vow (GM chooses).
\item 3: \textit{Name Tarnish}: --1d with any witness or herald.
\item 4: \textit{Omen Debt}: a rival gains +1d once against you.
\item 5: \textit{Geis Echo}: the geis attempts to re-bind with a harsher clause.
\item 6: \textit{Dragon’s Interest}: a local drake or elder hears your broken word.
\end{itemize}

\subsubsection{Draconic Bonds (NPC or Elder-Facing)}
\begin{itemize}[leftmargin=*]
\item \textbf{Hoard Clause}: Payment must change a story (humiliate a tyrant, bury a feud, return a relic).
\item \textbf{Scale Clause}: Accept a scar or brand; while marked, you gain a relevant Tag, but dragons notice you.
\item \textbf{Word Clause}: Speak your lineage or craft a new one. If found false, immediate Backlash and enmity.
\end{itemize}

\paragraph{Sample Geasa (d6)}
\begin{enumerate}[leftmargin=*]
\item Guard a boundary until first snow melts (\Tag{WARD} at gates).
\item Speak no lies under open sky (+1d to \textit{Parley}; cannot \textit{Deceive} outside).
\item Draw first blood only after terms declared (+1d on formal duels).
\item Offer quarter once each battle (gain \Tag{FEAR} vs. oath-breakers).
\item Shed blood to save a stranger by sundown (+1d to \textit{Rescue/Heal}).
\item Carry no gold by hand (\Tag{LUCK} on travel checks).
\end{enumerate}

% ---------------------------------------------------------
\subsection{Minor Dragons (Tier I–III)}

\Note{Use these freely. They aren’t puzzles so much as omens. Clever play, bribes, and geasa matter more than raw harm.}

\monsterbox{
\textbf{Emberdrake (Tier I)}\\
\textit{Harm 1 (Fire/Nip)} • Armor 1 • Fast • \Tag{IGNITE}\\
\textbf{Moves}: Scorch the ground; steal a bright trinket; flare in panic to blind for a beat.\\
\textbf{Pressure}: In tight spaces, actions with \textit{cloth/leather} act at \textit{Risky}.\\
\textbf{Weakness}: Milk, ash, or sweet-smoke pacifies.\\
\textbf{Trophy}: \textit{Cinder Pearl} (once: add \Tag{IGNITE} to a small action).
}

\monsterbox{
\textbf{Mirewing Basilisk-Drake (Tier II)}\\
\textit{Harm 2 (Venom/Glare)} • Armor 1 • Loping • \Tag{BIND}\\
\textbf{Moves}: Fix a victim with a muddy gaze (rooted); lash tail to topple; foul water.\\
\textbf{Pressure}: Any stumble becomes \textit{Desperate} in swampy Zones.\\
\textbf{Weakness}: Polished mirrors; sudden bright clatter.\\
\textbf{Trophy}: \textit{Gleam Reed} (counter one \textit{Root/Slow} effect).
}

\monsterbox{
\textbf{Barrow Serpent (Tier II)}\\
\textit{Harm 2 (Grave-Chill)} • Armor 2 in darkness • Burrow • \Tag{FEAR}\\
\textbf{Moves}: Suck warmth from air; crumble graves into pits; coil to pin a shieldwall.\\
\textbf{Pressure}: Torches sputter; without steady light, --1d.\\
\textbf{Weakness}: Funeral bells or true names of the buried.\\
\textbf{Trophy}: \textit{Cold Scale} (ignore one chill/terror effect).
}

\monsterbox{
\textbf{Sky-Needle Wyvern (Tier II--III)}\\
\textit{Harm 2--3 (Pierce/Wind)} • Armor 1 • Fly • \Tag{WIND}\\
\textbf{Moves}: Snatch and drop; turn arrows with wing sheer; scream to scatter.\\
\textbf{Pressure}: Open ground is \textit{Desperate} unless you secure cover or ropes.\\
\textbf{Weakness}: Nets, whistling cords, anchored lines.\\
\textbf{Trophy}: \textit{Pinion Hook} (climb/descend a cliff at \textit{Dominant} once).
}

\monsterbox{
\textbf{Lampwyrm Archivist (Tier I--II)}\\
\textit{Harm 1 (Gum/Ink)} • Armor 0 • Glide • \Tag{TRUTH}\\
\textbf{Moves}: Illuminate lies (speaker acts at --1d); hoard scribbles; hum lullabies.\\
\textbf{Pressure}: Any deception in scene ticks a \textit{Suspicion} clock [4].\\
\textbf{Weakness}: Candles snuffed respectfully; gifted annotation.\\
\textbf{Trophy}: \textit{Proof-Mote} (once: ask ``What here is misfiled?'').
}

\monsterbox{
\textbf{Frostling Wyrm (Tier III)}\\
\textit{Harm 3 (Rime/Crush)} • Armor 2 • Slow • \Tag{WARD} \Tag{COLD}\\
\textbf{Moves}: Breathe hoarfrost (\Tag{AREA}); seal doors with ice-ribbing; hibernate-ambush.\\
\textbf{Pressure}: Cold checks tax Fatigue; gear becomes \textit{Brittle}.\\
\textbf{Weakness}: Resonant bells; shared fire and stories (reduce hostility).\\
\textbf{Trophy}: \textit{Rime Crown} (once: gain \Tag{COLD} and \Tag{WARD} for a scene; then mark 1 Fatigue).
}

% ---------------------------------------------------------
\subsection{Knights \& Dragons: Shared Tables}

\subsubsection{Honor or Hunger? (d6 prompt)}
\begin{enumerate}[leftmargin=*]
\item A geis is offered if you return a banner.
\item The border hedge demands blood for passage.
\item A drake steals a treaty—who signed it and why?
\item A knight’s record contradicts the living memory of a wyrm.
\item A hoard piece sings a rival’s name at dawn.
\item A spur-call horn echoes from two directions at once.
\end{enumerate}

\subsubsection{Quick Geis Seeds (d6)}
\begin{enumerate}[leftmargin=*]
\item Eat only saltless bread until the pact is done (\Tag{LUCK} on travel).
\item Speak first to the lowest-born in any hall (+1d to \textit{Gather Rumors}).
\item Do not cross water after dusk (\Tag{WARD} vs. \textit{Night-Terrors}).
\item Wear no helm while bearing news (NPCs treat you as messenger).
\item Take no payment for slaying beasts that cannot parley (\Tag{MERCY}).
\item Let enemies choose ground; you choose terms (+1d to \textit{Set Terms}).
\end{enumerate}

% =========================================================
% DRAGON CULTS
% =========================================================
\section{Dragon Cults, Schisms, and Infiltration}

\Note{Not every dragon claims a cult. Some inherit them. Some tolerate them. Some devour them. Most cults collapse long before a wyrm ever notices. When one survives, entire provinces tilt.}

\subsection{Cult Anatomy}

A Dragon Cult is built from three pillars:
\begin{enumerate}[leftmargin=*]
\item \textbf{Myth Engine} (what story powers it)
\item \textbf{Scarce Rite} (what only initiates can do)
\item \textbf{Territorial Claim} (where its influence holds)
\end{enumerate}

At the table, a cult is treated like a slow-moving faction with a \textit{Public Face}, \textit{Inner Knives}, and a \textit{Hidden Egg}.

\begin{tabular}{@{}lll@{}}
\toprule
\textbf{Layer} & \textbf{What They Want} & \textbf{What They Do} \\
\midrule
Public Face & Pilgrimage, blessings, harvest rites & Festivals, petitions, tithes \\
Inner Knives & Power, silence, rival removal & Kidnapping, extortion, sabotage \\
Hidden Egg & Transform the world for the dragon & Summonings, wyrm-binding, prophecy \\
\bottomrule
\end{tabular}

\subsection{Cult Clocks}

\begin{itemize}[leftmargin=*]
\item \textbf{Infiltration [6]} — Converts militia, merchants, clergy, children.
\item \textbf{Manifestation [8]} — Tries to physically call, awaken, or anchor a dragon.
\item \textbf{Rupture [10]} — The cult’s success changes law, season, or geography.
\end{itemize}

Ticks when:
\begin{itemize}[leftmargin=*]
\item PCs ignore rumors or disappearances.
\item Blood rites occur at solstice or equinox.
\item A hoard-piece changes hands.
\end{itemize}

Reduce ticks by:
\begin{itemize}[leftmargin=*]
\item Breaking oaths \textit{publicly}.
\item Shaming the cult in open ritual.
\item Returning a stolen “scale-word” (artifact, relic).
\end{itemize}

\subsection{Cells \& Schisms (d6)}

\begin{enumerate}[leftmargin=*]
\item \textbf{Ash-Feather Choir} — Believe their dragon is dead, but fragments of its soul live in hymns. Singers awaken pieces in listeners.
\item \textbf{Molten Ascetics} — Burn mundane identity; take new names in heat and ash. Wear iron masks. Suffer no lies.
\item \textbf{Hoard-Tenders of Quiet Coin} — Thieves’ guild turned “curators.” Every stolen jewel becomes “a verse of praise.”
\item \textbf{The Charm-Broken} — Former drake victims; trauma canonized. They hijack caravans to “spare others” and accidentally summon the dragon’s attention.
\item \textbf{Veil of the First Scale} — Alchemists chasing transmutation via drakestone dust. Brew “scale-tonics” that sometimes work, sometimes rot bone.
\item \textbf{Candle-Keepers of the Hollow} — Believe light is sin against the dragon’s perfect night. Snuff lamps and open crypts.
\end{enumerate}

\subsection{Cult Rites}

\paragraph{Low Rite: Ember-Send (DV 2)} \Tag{IGNITE} \Tag{MESSENGER}\\
Burn a sealed scrap; the smoke carries a whispered message to a known cultist within a day’s travel. Backlash: the ember remembers—leave a traceable scent of ash for one scene.

\paragraph{Standard Rite: Scaled Veil (DV 4)} \Tag{WARD} \Tag{FEAR}\\
Those under the veil resemble “favored kin.” NPCs feel a prickle of dread. Mortal foes at Near must test Resolve (Risky) to strike first. Backlash: the veil cracks; one participant gains serpentine eyes for a night, acting at --1d vs. bright light.

\paragraph{Greater Rite: Hoard-Calling (DV 5)} \Tag{BIND} \Tag{STONE} \Tag{AREA}\\
Stones tremble; coins shiver; metal leaps toward a chosen point, forming a nest-mound. Can trap foes (\Tag{BIND}), or prepare a resting ground for a minor drake. Backlash: metal becomes brittle afterward; any gear used this scene risks loss of a tag.

\paragraph{Cataclysm Rite: Scale-Dawn (DV 6+)} \Tag{IGNITE} \Tag{WIND} \Tag{FEAR} \Tag{AREA}\\
Sky reddens; a spectral draconic shape blots the sun. Panic in a town or keep; all social rolls begin at \textit{Desperate}. If the Manifestation Clock is full, a real dragon takes notice. Backlash: caster coughs embers, marking 2 Fatigue and 1 Corruption.

\subsection{What They Really Want (d6 Truths)}

\begin{enumerate}[leftmargin=*]
\item Not power — \textit{certainty}. A world without contradictions.
\item To resurrect a dragon that does not wish to return.
\item To bargain: share their souls to become one being.
\item To build a “perfect” kingdom the dragon once dreamed.
\item To stop a prophecy about the dragon’s final death.
\item To feed the dragon with \textit{memory}, not flesh.
\end{enumerate}

\subsection{Cult Leaders (d6)}

\begin{enumerate}[leftmargin=*]
\item Exiled monk with a shard of dragon tooth lodged in his ribs.
\item Merchant-matriarch who believes the wyrm saved her ancestors.
\item Knight stripped of banner, now “chosen herald.”
\item Street preacher who speaks in dragon-tongue while asleep.
\item Scholar who mapped ley-lines into a hoard-shaped sigil.
\item War-orphan raised by a drake’s distant dream.
\end{enumerate}

\subsection{Infiltration Scene Hooks}

\begin{itemize}[leftmargin=*]
\item A town’s taxes are being paid “in gems only.”
\item A child draws a dragon crest they’ve never seen.
\item A caravan guard wakes breathing smoke.
\item Bells ring backward in a border keep.
\item Sheep bleed gold-dust when shorn.
\item A knight’s blade curls like warm wax.
\end{itemize}

\subsection{Cult Collapse (When PCs Win)}

\begin{itemize}[leftmargin=*]
\item The hoard scatters: treasures seek new owners.
\item The dragon’s “dream” breaks—storms or tremors cease.
\item A grieving remnant becomes \textit{fanatically good} or \textit{murderously bitter}.
\item Lost eggs surface: orphans, relics, half-made creations.
\item Power vacuum: bandits, lords, elders move to claim land.
\end{itemize}

\subsection{Cult Ascendance (When PCs Lose)}

\begin{itemize}[leftmargin=*]
\item Local laws change: tithes paid in metal, not grain.
\item Heralds declare amnesty for “scaled blood.”
\item A drake circles the keep at dawn, unseen but heard.
\item Crops fail, then regrow into gold-tinged weeds.
\item The dragon speaks through dreams to hundreds at once.
\end{itemize}

% =========================================================
% DRACONIC HOARD GENERATOR
% =========================================================
\section{Draconic Hoards \& Treasure That Changes Futures}

\Note{A dragon’s hoard is a ledger of grudges, victories, insults, and impossible promises. No two hoards are alike. Each object has \textbf{weight}, \textbf{memory}, and a \textit{price to use}.}

\subsection{Hoard Procedure (Fast Table)}

Roll 3d6:
\begin{enumerate}[leftmargin=*]
\item \textbf{Form} (what the hoard looks like)
\item \textbf{Heart-Piece} (what defines it)
\item \textbf{Volatile Treasure} (dangerous item)
\end{enumerate}

Optional: roll a second Heart-Piece for elder wyrms.

\subsection{Form of the Hoard (d6)}

\begin{enumerate}[leftmargin=*]
\item A chamber of molten gold: coins flow like syrup, heat warps steel.
\item Catacomb of trophies: banners, crowns, bones, swords in stone.
\item Flooded vault of gems: water refracts hypnotic patterns.
\item Dust-plain of ash and broken weapons: no metal holds a shine.
\item A serene temple of coins stacked perfectly into pillars.
\item A living garden of ore-veins and crystal blooms that change color with breath.
\end{enumerate}

\subsection{Heart-Piece (d8)}

\begin{enumerate}[leftmargin=*]
\item \textbf{The Crown of Three Kingdoms} — Wearer gains +1d to Command, but must never kneel or they take Harm 2 (Shame).
\item \textbf{A Mirror Full of Stars} — Shows a possible future; first viewer gains +1 Boon next session; second viewer suffers 1 Corruption.
\item \textbf{Seven Iron Coils} — Chains that bind spirits, ghosts, or oaths. Using them marks +1 Obligation (Dragon).
\item \textbf{Heart-Forge Ember} — Heat of a dying world; can reforge any blade; next time it draws blood, GM may tick a Doom clock.
\item \textbf{A Name in a Lantern} — Speak the name to summon its owner (dead or alive). Each use burns a memory of the speaker’s past.
\item \textbf{Tear of the First Storm} — Once: create a violent storm. Afterwards, the user dreams of the dragon every night.
\item \textbf{Ledger of Broken Promises} — Read it to reveal a tyrant’s secret. Writing in it creates a new secret for the GM to use later.
\item \textbf{Seed of Stone} — Bury it: grows a fortress in a day. It will “remember” the builder’s fears and shape itself accordingly.
\end{enumerate}

\subsection{Volatile Treasure (d8)}

\begin{enumerate}[leftmargin=*]
\item \textbf{Gold-Eater Idol} — Animates at night, swallowing coins unless fed a vow.
\item \textbf{Candle of Reverse Shadows} — Reveals invisible things; extinguishing it summons one of them.
\item \textbf{Grave-Silver Thimble} — Sews any wound shut; steals one breath per stitch.
\item \textbf{Drake-Bone Horn} — Call for help; a drake answers, furious you have its relative’s bone.
\item \textbf{Memory-Coin} — Flip it: recall a forgotten truth; or lose one.
\item \textbf{Obsidian Chalice} — Liquids become poison or panacea; GM chooses secretly.
\item \textbf{Mirror-Shard Key} — Opens a “door” into reflected spaces; someone follows you back.
\item \textbf{Scales of Debt} — Weigh a soul’s worth; the dragon learns the result instantly.
\end{enumerate}

\subsection{Value (What It Buys)}

Treasure from a wyrm’s hoard does not buy grain—it buys \textbf{exceptions}. Each PC chooses:

\begin{itemize}[leftmargin=*]
\item \textbf{One-time Favor} from a lord, guild, or knightly order.
\item \textbf{A Seal of Transit}, ignoring borders or tolls.
\item \textbf{A Year of Silence}: no one may legally question you.
\item \textbf{A Cleared Name}: erase a crime or accusation.
\item \textbf{A Writ of Passage} into sacred or forbidden ground.
\end{itemize}

Each time treasure is spent this way, tick a clock:

\begin{center}
\textbf{Dragon’s Attention [6]}
\end{center}

At 6 ticks, the wyrm knows where its treasure went.

\subsection{Hoard Curses (d6)}

\begin{enumerate}[leftmargin=*]
\item The taker dreams of wings every night and speaks in smoke.
\item Metal gleams unnaturally; thieves follow you.
\item A rival cult believes \textit{you} are their prophesied herald.
\item Birds fall silent when you approach.
\item Children stare, animals kneel, elders weep.
\item Fire bends toward you like a hungry pet.
\end{enumerate}

\subsection{Hoard Guardians (d6)}

\begin{enumerate}[leftmargin=*]
\item Wyrm-bound knights (oath-ink on tongue)
\item Coin-golems that assemble from treasure
\item Paper-wyrmlings (burn like phosphorus)
\item Shadow-bats that steal names
\item Thief-priest pretending to help you escape
\item Dragon’s echo (astral projection)
\end{enumerate}

\subsection{Turning Treasure Into Power}

\paragraph{Forge-Boons}
Your smith reforges a relic → gain a unique item tag (GM chooses one):
\begin{itemize}[leftmargin=*]
\item \textbf{Hoard-Eater} (ignores armor)
\item \textbf{Moon-Reflective} (blocks illusions)
\item \textbf{Sky-Drawn} (returns to hand)
\item \textbf{Fate-Marked} (+1 Boon when spilling noble blood)
\end{itemize}

\paragraph{Ritual Sale}
Selling treasure to a cult increases Manifestation +1. Selling to a kingdom starts a war.

\paragraph{Draconic Favor}
Return a Heart-Piece as tribute:
\begin{itemize}[leftmargin=*]
\item Ask one question the dragon \textit{must} answer
\item Gain one safe night in its territory
\item Mark +1 Renown in noble courts
\end{itemize}

\paragraph{Draconic Wrath}
Attempt to destroy a Heart-Piece:
\begin{itemize}[leftmargin=*]
\item Dragon awakens or sends herald
\item Weather changes dramatically
\item Wards falter across a province
\end{itemize}

% =========================================================
% PATRON: THE ANCIENT WYRMS (DRACONIC RUNES)
% =========================================================
\subsection*{The Ancient Wyrms (Primordial Sovereignty)}
\label{subsec:patron-wyrms}
\index{Patrons!Ancient Wyrms}
\index{Runes!Draconic}

\textbf{Symbol:} A spiral of seven scales encircling an empty eye.  
\textbf{Epithets:} The Crowned Flames, First Sovereigns, The Hoard Unending, The Living Citadels.  
\textbf{Nature:} Any wyrm of sufficient age and will may forge a pact. Each bond is unique, sealed in molten speech and unbreakable oath. Dragons do not choose lightly: to grant runes is to acknowledge a mortal as \emph{kin by fire}.

\medskip

\textbf{Doctrine:}
\begin{itemize}[leftmargin=*]
\item Power is owed only to power; tribute is proof of worth.
\item What is taken must be paid for—in gold, in oath, or in blood.
\item Knowledge is treasure; hoards guard secrets, not merely gold.
\item A promise given in a dragon’s presence is a chain in the soul.
\end{itemize}

\textbf{Pact Price (Obligation):}
\begin{itemize}[leftmargin=*]
\item Protect a treasure (object, name, land, or secret).
\item Enforce a vow sworn by others.
\item Extend the dragon’s influence: fear, tribute, renown.
\item Once per season: deliver a worthy offering.
\end{itemize}
Refusing a demand marks +2 Obligation. Betrayal awakens the wyrm.

\subsection*{Draconic Rites}

\textbf{Low Rites:}
\begin{itemize}[leftmargin=*]
\item \textbf{Spark of the First Flame}—Ignite a melee weapon with searing heat (Harm 2) for a scene.
\item \textbf{Scaled Skin}—Target gains Armor 1 vs. mundane attacks.
\item \textbf{Voice of Embers}—Speak with commanding resonance; +1d to Coercion, Threats, or Demands.
\item \textbf{Hoard-Sense}—Sense precious metals or magical relics within Near range.
\end{itemize}

\textbf{Standard Rites:}
\begin{itemize}[leftmargin=*]
\item \textbf{Dragon’s Gaze}—Force a creature to obey a single command (Short and simple); if resisted, they suffer Harm 1 (Fear).
\item \textbf{Crown of Scales}—Your skin becomes iron-hard; gain Armor 2 for a scene and immunity to fire.
\item \textbf{Molten Breath}—Exhale a cone of flame: Harm 2 (Area), Set objects ablaze, produce smoke cover.
\item \textbf{Wyrm-Ward}—Raise a shimmering wall of heat or stone that blocks passage until shattered or dispelled.
\end{itemize}

\textbf{High Rites (Tier IV+):}
\begin{itemize}[leftmargin=*]
\item \textbf{Fire Unending}—Sustain a blazing inferno that burns water, stone, and armor; ignore cover. Scene-long if Concentrating.
\item \textbf{Name-Binding Coil}—A creature who speaks their name aloud is bound to fulfill a sworn term. Breaking it causes immediate Harm 3 and a Doom clock begins.
\item \textbf{Wings of the Crowned Flame}—Grow spectral wings; fly freely; your voice carries for miles.
\item \textbf{Dragonheart Ascendancy}—For one scene, act as a minor wyrm: +1 Tier, immune to fear, fire, and mundane weapons. At scene’s end: 2 Fatigue and +2 Obligation.
\end{itemize}

\subsection*{Runekeeper Notes}
\begin{itemize}[leftmargin=*]
\item Every dragon grants different Rune “accents”—a frost wyrm’s breath freezes; a desert wyrm’s voice scorches the soul.
\item Using Rites against a dragon is not betrayal, but \emph{challenge}; most elders approve.
\item A Runekeeper who dies with honor may have their name placed in the Hoard, becoming a draconic ancestor-spirit.
\end{itemize}

\subsection*{Boons \& Complications}
\begin{itemize}[leftmargin=*]
\item Spend 1 Boon to speak any mortal language with draconic authority.
\item Gain +1 Corruption when hoarding wealth without purpose.
\item While wearing metal, you leave faint scorch-marks where you step.
\item Children stare. Horses bow. Priests tremble.
\end{itemize}

\medskip
\noindent
\textbf{Tone:} This Patron lets any ancient dragon act as a cosmic power without naming or binding the GM to a specific entity. Each wyrm interprets the pact differently—some demand gold, others secrets, others worship. The Runekeeper becomes a herald, tax-collector, prophet, or enforcer depending on culture and dragon.


\sectionheader{Adventure Seeds \& Campaign Arcs}

Dragons are not random encounters. They are story engines.
Use these hooks to build entire arcs around a single sovereign.

\subsectionheader{The Sleeping Crown}
A dragon older than the kingdom lies beneath the capital’s foundation.
For centuries, priests have maintained the wards that keep it dreaming.
Now the wards fail, one by one.
\begin{itemize}[leftmargin=*]
\item \textbf{Early Signs:} Stone sweating, iron bending, nightmares in the noble courts.
\item \textbf{Middlegame:} Streets buckle; ancient districts sink; the dragon murmurs in its sleep.
\item \textbf{Climax:} Do the PCs restore the wards… or awaken a god-king who remembers betrayal?
\item \textbf{Twist:} The dragon does not want vengeance. It wants a throne rebuilt.
\end{itemize}

\subsectionheader{Ash Above the Orchard}
A small village prospers beyond reason: lush crops, perfect health, uncanny fortune.
Their patron? A young scarlet drake nesting in the orchard caves.
\begin{itemize}[leftmargin=*]
\item \textbf{Complication:} A wounded elder dragon comes to reclaim a runaway “child.”
\item \textbf{Choices:} Protect the village, negotiate the return, or help the drake escape into legend.
\item \textbf{Price:} The drake’s love for mortals might doom them.
\end{itemize}

\subsectionheader{The Knight With No Shadow}
A legendary knight commands armies without speaking.  
His foes break like dust in the wind.  
He never casts a shadow—because his shadow is a dragon bound in human form.
\begin{itemize}[leftmargin=*]
\item \textbf{Goal:} Free the dragon, or break the knight’s pact.
\item \textbf{Clue:} Where the knight passes, mirrors crack.
\item \textbf{Twist:} The dragon does not want freedom—it wants the knight’s body.
\end{itemize}

\subsectionheader{The Broken Sky}
Stars vanish one by one. Astronomers panic.  
A celestial wyrm circles the heavens, devouring forgotten constellations.
\begin{itemize}[leftmargin=*]
\item \textbf{Play:} Skyborne chase, ancient observatories, riddles in star-tongue.
\item \textbf{Threat:} When the last constellation falls, prophecy ends—fate becomes chaos.
\item \textbf{Hope:} Restore a constellation with a sacrifice of memory.
\end{itemize}

\subsectionheader{A Tax of Wings}
A kingdom pays tribute in grain, steel, and prisoners.  
Every spring, the sky darkens—black-scaled tithe-collectors return.
\begin{itemize}[leftmargin=*]
\item \textbf{Complication:} The tyrant-dragon is gone; its brood comes anyway.
\item \textbf{Underlying truth:} The kingdom was never forced. Its nobles offered the pact willingly.
\item \textbf{PC role:} Unravel a generations-long lie, end a monstrous tradition, or seize the pact.
\end{itemize}

\subsectionheader{The Dragon’s Bride}
A cursed noble bloodline produces a marriage-bonded “tribute” every century.
The wedding gift: immense prosperity.
The wedding cost: the bride never returns.
\begin{itemize}[leftmargin=*]
\item \textbf{Conflict:} The newest bride refuses to go—and bears dragon-touched powers.
\item \textbf{Twist:} The dragon is not cruel—it is protecting her from a worse fate.
\end{itemize}

\newpage

\sectionheader{Dragon-Forged Artifacts}

A dragon’s breath changes steel.  
A dragon’s blood writes spells.  
A dragon’s will makes history.

These relics are found in hoards, as knightly heirlooms, or as cursed treasures.

\subsectionheader{Irisil, the Moon-Thread Bow}
\begin{itemize}[leftmargin=*]
\item \textbf{Tag:} [PRECISION] [SILENCE] [MOON]
\item \textbf{Power:} Shoot a line of silver thread through shadow; arrow passes through walls of darkness and illusion.
\item \textbf{Cost:} If drawn under a blood moon, the bow whispers truths the wielder cannot forget.
\end{itemize}

\subsectionheader{Remorse, the Silver-Edge Sword}
\begin{itemize}[leftmargin=*]
\item \textbf{Tag:} [BLEED] [CLEAVE] [COMMAND]
\item \textbf{Power:} A leader struck by Remorse rolls social actions with +1d while blood spills—they become decisive, ruthless, brilliant.
\item \textbf{Curse:} When the blade is sheathed, every decision made becomes unbearable guilt.
\end{itemize}

\subsectionheader{Wyrm-Heart Lantern}
\begin{itemize}[leftmargin=*]
\item \textbf{Tag:} [LIGHT] [TRUE SIGHT]
\item \textbf{Effect:} Reveals invisible, astral, or shapeshifted creatures.  
Burns cold; does not harm flesh.
\item \textbf{Price:} It remembers every lie told within its light.  
A dragon may ask for those truths later.
\end{itemize}

\subsectionheader{The Crown of Molten Brass}
Forged from the molten scale of a volcano wyrm.
\begin{itemize}[leftmargin=*]
\item \textbf{Tag:} [FIRE] [RULE] [AURA]
\item \textbf{Effect:} The wearer cannot be burned; fire bends away.  
Mortals instinctively obey commands delivered with heat or flame.
\item \textbf{Curse:} Every night, the crown dreams of conquest.  
If ignored, it imposes +1 Obligation.
\end{itemize}

\subsectionheader{Grimwing Mantle}
A cloak made from a young storm-wyrm’s feathers.
\begin{itemize}[leftmargin=*]
\item \textbf{Tag:} [FLIGHT] [SILENCE] [STORM]
\item \textbf{Effect:} Glide across city rooftops; never leave a footprint.  
Grant Dominant Position when ambushing from above.
\item \textbf{Curse:} Thunder follows your arrival. Eventually, someone notices.
\end{itemize}

\subsectionheader{The Heart-Shard Chalice}
\begin{itemize}[leftmargin=*]
\item \textbf{Tag:} [HEAL] [SOUL] [BARGAIN]
\item \textbf{Effect:} Drinking from it restores one Harm and removes a Curse.
\item \textbf{Price:} A dragon now knows your name.  
The next time you call for help, it answers—at a cost.
\end{itemize}

\subsectionheader{Ash-coil Arbalest}
A dragonbone siege–crossbow that fires iron like lightning.
\begin{itemize}[leftmargin=*]
\item \textbf{Tag:} [AREA] [PIERCE] [SHOCK]
\item \textbf{Effect:} Harm 3 to a Near zone; armor counts as 1 lower.
\item \textbf{Catastrophe:} On a 1, the arbalest explodes—dragonbone remembers how to scream.
\end{itemize}

\newpage

\sectionheader{Relics of Draconic Memory}

These are stranger, older, and rarely safe.

\subsectionheader{The Egg That Will Not Hatch}
A cold, glass-smooth egg. Inside, an idea sleeps.
\begin{itemize}[leftmargin=*]
\item \textbf{Effect:} Whisper a question; the egg answers in dreams.
\item \textbf{Truth:} It will only hatch when someone dies willingly for it.
\item \textbf{Consequence:} Whatever emerges remembers the one who fed it.
\end{itemize}

\subsectionheader{The Scale of Ancestors}
A black scale as large as a shield.
\begin{itemize}[leftmargin=*]
\item \textbf{Effect:} Once per campaign: negate a dragon’s attack entirely.
\item \textbf{Cost:} The next time you face a wyrm, you act at Desperate Position.  
Dragons smell betrayal.
\end{itemize}

\subsectionheader{The Library of Ashen Wings}
A portable archive of scorched pages bound in silver wire.
\begin{itemize}[leftmargin=*]
\item \textbf{Effect:} Learn any ancient secret, prophecy, or ward.
\item \textbf{Ruin:} One page burns itself away for every truth gained.  
When the last page burns, something escapes.
\end{itemize}

\end{document}