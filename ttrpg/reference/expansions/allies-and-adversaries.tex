
\begin{document}
\begin{center}
{\LARGE \textbf{Allies \& Adversaries}}\\[3pt]
{\large A Fate's Edge module for companions, cohorts, and organizations}\\[2pt]
\textit{Version 0.1 (Playtest)}
\end{center}
\vspace{0.25em}
\hrule\vspace{0.75em}

\section*{Design Goals}
\begin{itemize}
  \item \textbf{Companions with teeth:} Followers act on-screen and off-screen with clear risks, clocks, and costs.
  \item \textbf{Social fabric as engine:} Loyalty, morale, bonds, and rivalries generate scenes and consequences.
  \item \textbf{Organizations as levers:} Cohorts and institutions move through clocks, strings, and orders without spreadsheets.
  \item \textbf{Plug into core:} Uses Position/DV, SB, clocks, Favor/Leverage/Exposure, Reputation, and Strings. No new dice math.
\end{itemize}
\emph{Principle:} Add procedures and tracks that tell you when to roll, what to tick, and how to escalate.

\section*{Quickstart (2 minutes)}
\begin{enumerate}
  \item Make 1--3 \textbf{Follower Sheets}; mark \textbf{Loyalty [6]} and \textbf{Morale [6]} on each.
  \item Attach \textbf{Strings} the follower controls (permits, keys, oaths). These matter more than stats.
  \item Pick an \textbf{Organization} (if any) and fill \textbf{Cohesion [6--8]}, \textbf{Exposure [6]}, and \textbf{Bandwidth} (orders/phase) = 1--3.
  \item At scene start, choose for each follower: \textbf{Assist}, \textbf{Cover}, or \textbf{Delegate}.
  \item Between sessions, run \textbf{Orders \& Entanglements} for organizations.
\end{enumerate}

\tableofcontents
\newpage

\section{Follower Sheets \& Templates}
\subsection{Follower Sheet (Template)}
\begin{verbatim}
[NAME & ROLE]
Tier: Green / Trained / Veteran / Elite
Playbook: Combat Ally / Specialist / Magical Assistant
Capabilities (2–3): short phrases like "shield wall", "ledger-eye", "silent step"
Edges (1–2): what they do exceptionally well
Liabilities (1–2): what gets them in trouble
Strings (1–3): permits, keys, rites, access, local reputation
Costs: wages / shares / favor / oath / taboo
Tracks:
- Loyalty [6]: long-term commitment to the PCs' cause
- Morale [6]: short-term willingness to risk; wavers in hard scenes
- Harm [3]: minor / serious / broken (or corrupted, if mystical)
Tags: (2–3 personal or social tags that matter on-screen)
Notes: relationships, debts, secret, ambition
\end{verbatim}

\paragraph{Tier (quality) guidance}
\begin{itemize}
  \item \textbf{Green:} When acting alone, start one step worse Position.
  \item \textbf{Trained:} Baseline Position.
  \item \textbf{Veteran:} Treat DV --1 when acting on their Capabilities.
  \item \textbf{Elite:} Position +1 when acting on their Edges.
\end{itemize}
\emph{No new dice.} Followers don't have separate pools; their tier nudges Position/DV and their tracks/tags determine fallout.

\subsection{Templates}
\subsubsection*{Combat Ally (guard, scout, duelist)}
Capabilities: formation fighting; hold a doorway; skirmish in alleys\\
Edges: fearless under missile fire; shield others\\
Liabilities: pride; feud; easily baited\\
Strings: watch rotation; barracks armory key\\
Costs: shares; honor code taboo

\subsubsection*{Specialist (scribe, pilot, barrister, spy, apothecary)}
Capabilities: crack ledgers; navigate fog; writs \& filings; ghost a meeting; treat wounds\\
Edges: unobtrusive; encyclopedic gossip\\
Liabilities: temper; debts; guild loyalty first\\
Strings: archive desk; ferry passbooks; guild seals\\
Costs: wages; bribe budget; guild tithes

\subsubsection*{Magical Assistant (rite-acolyte, hedge-mage, psion acolyte)}
Capabilities: ward a room; amplify rites; anchor a vision; soothe a crowd\\
Edges: ritual precision; calm presence\\
Liabilities: taboo; frail; omen-prone\\
Strings: sanctuary access; oath-tablet registry; reliquary key\\
Costs: offerings; rest days; ritual purity

\subsection{Onboarding a Follower}
Name two non-overlapping Capabilities and one Liability; the table adds one Tag. Tie them to the map with one \textbf{String} that matters this arc. Mark \textbf{Loyalty 2/6}, \textbf{Morale 3/6}, Harm none.

\section{Followers On-Scene}
At the start of a scene, declare the role for each present follower.

\subsection{Assist}
They bolster a PC's action on their Capability.
\begin{itemize}
  \item \textbf{Effect:} Improve \textbf{Position +1} \emph{or} treat \textbf{DV --1} (once/scene per follower).
  \item \textbf{SB Hook:} On any 1 in the roll, the GM may tick follower \textbf{Morale +1} (shaken) or \textbf{Harm (minor)} instead of a PC-facing complication.
\end{itemize}

\subsection{Cover}
They absorb heat or run interference.
\begin{itemize}
  \item \textbf{Effect:} Once this scene, reduce a GM \textbf{Social SB} spend by 1 or cancel a single \textbf{tag flip} against the party in their venue.
  \item \textbf{Cost:} Tick follower \textbf{Exposure +1} (if tracked) or \textbf{Morale +1} (took the blame).
\end{itemize}

\subsection{Delegate}
They take an independent scene action (legwork, duel, petition, infiltration). Resolve with the \textbf{PC's roll} that commands, enables, or coordinates (e.g., Presence+Command, Wits+Tactics, Wits+Investigation).
\begin{itemize}
  \item \textbf{Effect:} On a hit, fill the target Situation Clock by effect or create a useful tag; on a partial, do it but suffer \textbf{Harm (minor)} or \textbf{Morale +1}; on a miss, GM banks SB and may apply a temporary Liability tag (e.g., \emph{Compromised}).
  \item \textbf{Tier Nudge:} Apply the Tier bonus/penalty when setting Position/DV.
\end{itemize}
\emph{Alone vs with PCs:} Acting alone in hostile ground starts one Position worse; with a PC on-screen, use the higher Position among the pair.

\subsection{Harm \& Recovery}
\begin{itemize}
  \item \textbf{Harm [3]:} minor (--) / serious (can't Assist/Delegate) / broken (removed until healed). Treat mystic overloads similarly.
  \item \textbf{Heal:} \emph{Shield} or \emph{Biofeedback} scenes may reduce Harm; otherwise Significant Time + coin/favor reduces one level.
\end{itemize}

\section{Loyalty \& Morale}
\subsection{Loyalty [6] (long-term)}
Tick up when promises are kept, credit/coin shared, public protection given, or their Ambition advances. Tick down for oath-breaking, humiliation, skipped payment, or betrayal of values.\\
At 0: they leave or flip neutral (rescue resets to 1). At 6: trigger a \textbf{Loyalty Event}---they offer an oath or ask a Price; choose for a lasting Tag (e.g., \emph{Oath-Bound}) or a new String.

\subsection{Morale [6] (short-term)}
Mark Morale for fear, hunger, injury, or public shame.
\begin{itemize}
  \item 4+: bold; may Assist/Delegate freely.
  \item 2--3: hesitant; Delegate starts one Position worse.
  \item 0--1: balk; require \emph{Reassure} (Presence+Sway) or \emph{Command} at DV 3--4 to act; on miss, they withdraw.
  \item \textbf{Reset:} Rest, food, respect scene, or public praise clears 1--2 Morale.
\end{itemize}

\subsection{Paying the Costs}
Each follower lists Costs (wages, shares, favor, taboo). When skipped two sessions in a row, tick Loyalty --1 and start \textbf{Debt [4]}. Clearing the clock cancels the penalty.

\section{Follower Advancement (Long-Term)}
At arc end or every 2--3 sessions, if the follower was central, choose one:
\begin{itemize}
  \item \textbf{Capability +1:} add a new Capability or sharpen an Edge; gain that Edge's Tier nudge for one more context.
  \item \textbf{String Gained:} seize/cut a String via play; write it on their sheet.
  \item \textbf{Harm Tolerance:} add \emph{Armor (1)} tag once per session when acting on their Capabilities.
  \item \textbf{Bond Deepens:} start \textbf{Bond [4]} with a PC; acting for that PC's ideal grants Position +1 once/scene.
  \item \textbf{Reputation:} convert three favors from one institution into \textbf{Standing (Tag)} tied to the follower.
\end{itemize}
If a follower becomes plot-defining, promote them to a \textbf{Lieutenant}: they can carry an \textbf{Off-Screen Order} without PC oversight.

\section{Organizations (Guilds, Armies, Syndicates)}
\subsection{Organization Sheet (Template)}
\begin{verbatim}
[ORG NAME]
Tier: street / guild / city / crown / synod
Aim (this season): what this org wants now
Leaders & Lieutenants: names + one-line intent
Cohorts (2–4): "dockside crew", "auditors", "pickets" …
Strings (3): permits, routes, seals, charters
Tracks:
- Cohesion [6–8]: unity/discipline; fill → schism/strike/coup
- Exposure [6]: heat/visibility; fill → audit/raid/purge
- Resources [6]: operational slack; empty → shortages & concessions
Bandwidth: 1–3 orders per cycle
Entanglements: debts, rivals, promises
\end{verbatim}

\subsection{Using Cohorts}
A cohort is a follower group. When a cohort acts on-screen, apply Tier nudges like a Veteran/Elite follower. Wounds/fear usually tick \textbf{Cohesion +1} rather than individual Harm.

\subsection{Orders \& Entanglements (Between Sessions)}
Once per session or when Significant Time passes:
\begin{enumerate}
  \item \textbf{Allocate Bandwidth:} choose up to Bandwidth orders: Audit, Guard, Smuggle, Petition, Strike-bust/Strike, Patrol, Build, Research, Evangelize, Bribe, Expose.
  \item \textbf{Resolve Each Order:} one roll by the directing PC (or a Lieutenant). Set Position/DV from fiction; apply Org Tier as DV --1 within its wheelhouse.
  \item \textbf{On Hit:} fill a related Project Clock (4--8) or create a lasting Tag. On 1s, GM banks SB and may tick Exposure +1 or Cohesion +1.
  \item \textbf{Upkeep:} if Resources drops to 0, choose: accept Crisis +1, lose Bandwidth --1, or start \textbf{Shortages [6]}.
  \item \textbf{Entanglement Roll:} for each banked SB this cycle, trigger a minor entanglement (rival, debt call-in, inspector visit). Convert into clocks, not binary losses.
\end{enumerate}

\section{Relationship Dynamics}
\subsection{Bonds \& Frictions}
Start a \textbf{Bond [4]} between a PC and a follower who has shared scenes. When the PC publicly protects/praises the follower, fill 1; when they shame or endanger them, clear 1. When full, add a permanent tag (\emph{Devoted}, \emph{Trusted Confidant}); once per scene, acting in line with that bond grants Position +1.\\
\textbf{Frictions [4]} capture triangles, rival mentors, or ideology splits. On fill, start a \textbf{Follower Conflict} or add the tag \emph{Soured}.

\subsection{Romance \& Family (optional)}
Treat as Bonds with consent. A full romance Bond may convert to a \textbf{String} (access to a house/court) or a \textbf{cost} (taboos, obligations).

\subsection{Reputation \& Standing}
Followers can hold \textbf{Standing} with institutions. In that venue, Standing acts like a once/scene Position +1 or DV --1, and counts toward party reputation checks.

\section{Follower Conflicts}
\subsection{When Allies Disagree}
Frame the splitting question. Start a \textbf{Dilemma Clock [4--6]}. Each side argues via Petition/Broker/Expose; use Assist/Cover from their allies. On 1s, spend Social SB to escalate: \emph{Ultimatum}, \emph{Walk-out}, \emph{Leverage Leak}, \emph{Public Scene}, \emph{Split the Crew}.\\
\textbf{Resolution:} on fill choose one---\textbf{Concession} (one side wins; other Loyalty --1), \textbf{Compromise} (create a new String you must honor), or \textbf{Break} (follower departs; start \textbf{Recruit/Repair [6]}).

\subsection{Competing Agendas (Long-Term)}
Give key followers a private \textbf{Agenda [4]} tied to a String or ideal. Tick when off-screen orders favor them; clear when you publicly choose against it. On fill, they demand a scene to cash it in---or flip a String against you.

\section{Generators \& Tables}
\subsection{Follower Seeds (d66)}
11--12: penitent enforcer; 13--14: debt-sold pilot; 15--16: oath-keep scribe; 21--22: relic courier; 23--24: disgraced auditor; 25--26: crowd-reader mummer; 31--32: hedge-rite acolyte; 33--34: picket captain; 35--36: ash-lantern warden; 41--42: smugglers' quartermaster; 43--44: bridge-lord's nephew; 45--46: psion novice; 51--52: syndicate fixer; 53--54: canal pilot; 55--56: inquisitor's clerk; 61--62: ex-legion drill; 63--64: dye house chemist; 65--66: caravan law clerk.

\subsection{Costs \& Ambitions}
\textbf{Costs:} wages; shares; favor; bribes; offerings; rest days; ritual purity; family duty.\\
\textbf{Ambitions:} vindication; mastery; homecoming; reform; riches; reputation; faith; revenge; discovery.

\subsection{Entanglements (roll 1--2 between sessions)}
Rival poaches a cohort; Auditor audits your Strings; a family claim interrupts a mission; Debtor calls; old oath resurfaces; festival duel challenge; informant flips; patrol biased; riot brewing in their home ward.

\section{Example of Play (short)}
\textbf{Scene:} The party needs a sealed ledger from the Archivolt. Their \emph{Specialist} (ledger-eye) \textbf{Delegates} to pull it during a playhouse feast. Position: \textbf{Dominant} (venue tags + String); Tier: \textbf{Veteran} (DV --1). Roll \emph{Wits+Investigation} to coordinate. Partial: ledger secured but \textbf{Morale +1}. A 1 shows $\to$ GM spends Social SB to start \textbf{Rumor [4]} about bribery.\\
Between sessions, the party's \textbf{Guild} runs two \textbf{Orders}: \emph{Audit} (to validate the ledger) and \emph{Bribe} (to cushion exposure). One hit, one partial: a \textbf{Project [6]} advances; \textbf{Exposure +1} ticks on the Guild. Next arc, the Specialist's \textbf{Agenda [4]} (protect guild loyalties) fills and they demand edits before release.

\section*{GM Reference (one page)}
\begin{itemize}
  \item \textbf{Follower roles:} Assist (Pos +1 or DV --1) \textbullet{} Cover (cancel 1 Social SB or tag flip once/scene; cost: Morale/Exposure) \textbullet{} Delegate (independent action via PC roll; apply Tier).
  \item \textbf{Tracks:} Loyalty [6] (0 leave/flip; 6 event) \textbullet{} Morale [6] (0--1 balk; 2--3 hesitant; 4+ bold) \textbullet{} Harm [3].
  \item \textbf{Org tracks:} Cohesion [6--8] \textbullet{} Exposure [6] \textbullet{} Resources [6] \textbullet{} Bandwidth 1--3.
  \item \textbf{Orders:} Audit \textbullet{} Guard \textbullet{} Smuggle \textbullet{} Petition \textbullet{} Strike-bust / Strike \textbullet{} Patrol \textbullet{} Build \textbullet{} Research \textbullet{} Evangelize \textbullet{} Bribe \textbullet{} Expose.
  \item \textbf{SB spends (social):} Ultimatum \textbullet{} Walk-out \textbullet{} Leverage Leak \textbullet{} Public Scene \textbullet{} Split the Crew \textbullet{} Permit Pulled.
\end{itemize}

\vfill
\section*{Changelog}
\textbf{v0.1} --- First pass: follower sheets/tiers, on-scene roles, loyalty \& morale, advancement, organization sheets \& orders phase, bonds/frictions, conflicts, generators, and GM reference.

\begin{center}\small This module adds procedures only; defer to the core SRD for roll math and basic adjudication.\end{center}

\clearpage

\begin{center}
  {\LARGE \textbf{Assets \& Worldly Patrons}}\\[3pt]
  {\large A Fate's Edge module for holdings, leases, and mortal patronage}\\[2pt]
  \textit{Version 0.2 (Playtest Revised)}
  \end{center}
  \vspace{0.25em}
  \hrule\vspace{0.75em}
  
  \section*{Design Goals}
  \begin{itemize}
    \item \textbf{Make holdings matter:} Assets have tags, strings, upkeep, and project clocks that change scenes.
    \item \textbf{Patrons with teeth:} Patronage brings Position, seals, and coin---and obligations, audits, and scandals.
    \item \textbf{Use the core:} Position/DV, SB, clocks, Favor/Leverage/Exposure, Reputation, Strings. No new dice math.
    \item \textbf{Low bookkeeping:} Track names, tags, and a few clocks; orders and events move the world.
    \item \textbf{Campaign continuity:} Assets and patrons persist and evolve across story arcs.
  \end{itemize}
  
  \section*{Quickstart (2 minutes)}
  \begin{enumerate}
    \item Pick/design an \textbf{Asset}; mark \textbf{Integrity [6]} and \textbf{Resources [6]}.
    \item Name a \textbf{Patron} (optional); set \textbf{Patron Tier} and \textbf{Obligations}.
    \item Attach 1--2 \textbf{Strings} (permits, seals, routes, rites).
    \item Choose 1--2 \textbf{Project Clocks [4--8]} to improve or expand the asset.
    \item Each session: issue \textbf{Asset Orders}, confront \textbf{Events \& Audits}, and tick \textbf{Upkeep}.
  \end{enumerate}
  
  \tableofcontents
  \newpage
  
  \section{Asset Sheet (Template \& Tags)}
  \subsection{Asset Sheet (Template)}
  \begin{framed}
  \noindent\textbf{[ASSET NAME]}\\
  \textbf{Type:} safehouse / barge company / workshop / archive / shrine / foundry line / office / permit / theatre / farm / mine / caravan yard / canal lock lease\\
  \textbf{Locale:} district \& city (matters for Position \& dials)\\
  \textbf{Strings (1--3):} permits, seals, routes, rites, keys\\
  \textbf{Tags (2--4):} see tag lists below\\
  \textbf{Tracks:}\\
  \hspace{10pt}- Integrity [6]: condition/standing; fill $\rightarrow$ shutdown, collapse, or seizure\\
  \hspace{10pt}- Resources [6]: cashflow, inventory, staff slack; empty $\rightarrow$ shortages/concessions\\
  \hspace{10pt}- Heat [6] (optional): locals' attention; fill $\rightarrow$ protest, inquiry, rough visit\\
  \textbf{Projects (0--3):} named [4--8] clocks to upgrade, expand, or pivot\\
  \textbf{Upkeep:} coin/favor/rites required each cycle (choose 1--2)\\
  \textbf{Legacy Notes:} campaign events, reputation effects, historical significance\\
  \textbf{Notes:} staff names, neighbors, rivals, liens
  \end{framed}
  
  \subsection{Asset Tags (pick 2--4)}
  \begin{itemize}
    \item \textbf{Fortified} (doors, shutters, hidden room) --- \emph{Shield/Petition} here starts \textbf{Dominant}.
    \item \textbf{Sanctified} (rites, hospitality) --- breaking rites here ticks \textbf{Exposure +1} (offender).
    \item \textbf{Licensed} (writ, seal) --- \emph{Broker/Petition} \textbf{DV --1} in scope of license.
    \item \textbf{Secret} (concealed use) --- first \emph{Expose} against you here starts \textbf{Desperate} for the attacker.
    \item \textbf{Crowd-Facing} (shop, theater) --- \emph{Audience tags} created here persist an extra scene.
    \item \textbf{Hazardous} (kilns, reagents) --- on 1s, GM may start \textbf{Accident [4]} instead of other SB spend.
    \item \textbf{Mobile} (barge/caravan) --- can act in adjacent districts without penalty.
    \item \textbf{Prestige} (old name, art) --- \emph{Petition} here \textbf{DV --1} with patricians; \emph{Blackmail} \textbf{+1 SB} against you if scandal hits.
    \item \textbf{Ward-Woven} (sigils, bells) --- \emph{Infiltrate} here starts \textbf{Desperate} unless key is held.
    \item \textbf{Unionized} --- \emph{Strike} is a valid Event; \emph{Broker} with unions here starts \textbf{Dominant}.
    \item \textbf{Shadowed} (underworld ties) --- \emph{Smuggle} \textbf{DV --1}; \emph{Expose} \textbf{Position --1} against you.
    \item \textbf{Water-Right} --- acts as a \textbf{String} for bridges/canals toll negotiation.
    \item \textbf{Archive} --- \emph{Research/Expose} \textbf{DV --1} with proof assembled here.
  \end{itemize}
  
  \subsection{Integrity \& Resources}
  \begin{itemize}
    \item \textbf{Integrity [6]} ticks from sabotage, audits, disasters, or neglect. At fill: choose \textbf{Shutdown}, \textbf{Seizure}, or a \textbf{Catastrophe} scene.
    \item \textbf{Resources [6]} drop from upkeep, shocks, strikes; refill via Orders, scenes, or Projects. If Resources = 0, future Upkeep requires \textbf{Favor} or a concession clock.
  \end{itemize}
  
  \section{Asset Play}
  \subsection{Acting Through an Asset (On-Scene)}
  Stage a scene at or with an asset; apply its Tags and Strings to Position/DV like any venue. A named staffer may act as a \emph{Follower} or as a cohort (tick \textbf{Integrity} instead of Harm on failures).
  
  \subsection{Upkeep \& Yield (Each Cycle)}
  \begin{itemize}
    \item \textbf{Upkeep:} Pay 1--2 of coin/favor/rite. If skipped: tick \textbf{Resources --1} and start \textbf{Creditor [4]} or \textbf{Inspection [4]}.
    \item \textbf{Yield:} If \textbf{Resources $\geq$ 3} and no active \textbf{Accident/Inspection}, gain one: coin, \textbf{Favor (narrow)}, \textbf{Clue}, or \textbf{Leverage (1)} themed to the asset.
  \end{itemize}
  
  \subsection{Asset Orders (Between Sessions)}
  Choose up to \textbf{2 orders} per session per asset (1 if Resources $\leq$ 2):
  \begin{itemize}
    \item \textbf{Operate:} generate Yield with risk (on 1s, tick \textbf{Heat +1} or \textbf{Integrity +1}).
    \item \textbf{Improve:} advance a \textbf{Project} [4--8].
    \item \textbf{Secure:} reduce \textbf{Heat --1} or add a \textbf{Security} tag for one scene.
    \item \textbf{Expand:} start a new Project to add a String (route, permit, office).
    \item \textbf{Audit:} convert one unspent Favor at this venue into \textbf{Standing (Tag)} after proof.
    \item \textbf{Exploit:} trade 1 \textbf{Integrity} for immediate coin + Favor; mark \textbf{Exposure +1} to someone.
  \end{itemize}
  \paragraph{Resolution} The directing PC rolls once per order (appropriate action). Tier/Tags adjust Position/DV. On hits, apply the order effect; on 1s, GM banks SB and hits Heat/Integrity/Exposure.
  
  \section{Campaign Continuity}
  \subsection{Asset Legacy System}
  Assets evolve and leave marks across campaigns through their \textbf{Legacy Notes}:
  \begin{itemize}
    \item \textbf{Reputation Effects:} Notorious (feared), Respected (trusted), Infamous (recognized), Legendary (historical significance)
    \item \textbf{Historical Events:} Siege Survived, Scandal Weathered, Expansion Completed, Crisis Averted
    \item \textbf{Relationship Changes:} Patron Gained/Lost, Rival Created/Resolved, Community Bond/Feud
    \item \textbf{Physical Evolution:} Renovated, Damaged, Expanded, Relocated
  \end{itemize}
  
  \textbf{Legacy Benefits:} Assets with positive legacy gain +1 Resources or +1 Integrity at start of new campaigns.\\
  \textbf{Legacy Burdens:} Assets with negative legacy start with Heat +2 or Sanction +1 with relevant patrons.
  
  \subsection{Patron Relationship Continuity}
  Worldly Patron relationships persist with modified standing:
  \begin{itemize}
    \item \textbf{Active Patrons:} Reduce Tier by 1 but retain 1 Standing tag
    \item \textbf{Dormant Patrons:} Convert Tier to Favor Ledger entries; can be reactivated
    \item \textbf{Former Patrons:} Become Rivals [4] or Contacts [4] based on ending relationship
    \item \textbf{Legacy Patrons:} Historical figures whose mandates still influence current politics
  \end{itemize}
  
  \subsection{Campaign Transition Events}
  At campaign end/beginning, roll for transition events:
  \begin{itemize}
    \item \textbf{Economic Shift (d6):} 1-2 Resources +1, 3-4 No change, 5-6 Resources -1
    \item \textbf{Political Change (d6):} 1-2 Gain new patron opportunity, 3-4 Status quo, 5-6 Lose a patron string
    \item \textbf{Physical Event (d6):} 1-2 Asset improves, 3-4 No change, 5-6 Asset takes Integrity -1
  \end{itemize}
  
  \section{Worldly Patrons}
  \subsection{Patron Sheet (Template)}
  \begin{framed}
  \noindent\textbf{[PATRON NAME]}\\
  \textbf{Type:} noble / guild / office / temple / factor / satrap / councilor\\
  \textbf{Stance toward PCs:} Allied / Wary / Hostile\\
  \textbf{Strings (3):} writs, permits, routes, rites, audiences\\
  \textbf{Boons:} what they can grant (seals, escorts, stipends, protection)\\
  \textbf{Obligations:} tithes, appearances, tasks, ideological lines\\
  \textbf{Tracks:}\\
  \hspace{10pt}- Favor Ledger (narrow favors owed or granted)\\
  \hspace{10pt}- Sanction [4]: warning $\rightarrow$ censure $\rightarrow$ seizure/revocation $\rightarrow$ hunt/prosecution\\
  \hspace{10pt}- Patron Exposure [6]: public risk to them from association with you\\
  \textbf{Patron Tier:} 0 Contact / 1 Sponsor / 2 Patron / 3 Protector\\
  \textbf{Mandate/Crisis Effects:} how their public wins/losses alter your Position/Exposure in their venues\\
  \textbf{Legacy Status:} Historical role, ongoing influence, campaign connections
  \end{framed}
  
  \subsection{Patron Tiers \& Benefits}
  \begin{itemize}
    \item \textbf{Tier 0 --- Contact:} 1 \textbf{Audience}/session; one minor seal once.
    \item \textbf{Tier 1 --- Sponsor:} \textbf{+1 Position} once/scene in their venues; 1 \textbf{Stipend}/session (coin or permit access).
    \item \textbf{Tier 2 --- Patron:} cancel \textbf{one Social SB} per session in public; \textbf{DV --1} on \emph{Petition} to their offices; claim \textbf{Escort} once.
    \item \textbf{Tier 3 --- Protector:} start public scenes \textbf{Dominant} in their venues; \textbf{Endorsement} creates a 4-clock \emph{Bandwagon} on targets.
  \end{itemize}
  \paragraph{Advance a Tier} Hold \textbf{Standing} with them or convert \textbf{3 favors} into Standing and complete a \textbf{Patron Task [4--6]} on-screen.
  
  \subsection{Obligations \& Sanctions}
  \begin{itemize}
    \item \textbf{Obligations:} tithe; keep scandal quiet; show at rites; take contracts; avoid rivals; uphold a doctrine. Skipping two cycles: tick \textbf{Sanction +1}.
    \item \textbf{Sanction [4]:} \emph{Admonish} (lose once/scene Position boost) $\to$ \emph{Censure} (no stipends; add \textbf{Audience: Skeptical}) $\to$ \emph{Seizure/Revocation} (lose a String or asset tag) $\to$ \emph{Hunt/Prosecution} (start \textbf{Warrant [6]}).
    \item \textbf{Patron Exposure [6]:} Your public failures can tick this; at fill they \emph{distance} (Tier --1) or \emph{flip} to Hostile.
  \end{itemize}
  
  \subsection{Bargaining Procedure (Negotiation Scene)}
  \textbf{Frame:} what boon you want and what obligations you accept.\\
  \textbf{Set:} venue tags; your Reputation/Standing apply; use \emph{Petition/Broker/Expose}.\\
  \textbf{On hit:} gain the boon; write the Obligation and tick Favor Ledger $\pm$ as appropriate.\\
  \textbf{On 1s:} GM may add a secret clause, leak leverage, or start \textbf{Rival Patron [4]}.
  
  \subsection{Multiple Patrons}
  You may keep two active patrons without penalty. A third creates \textbf{Split Loyalty [4]}; on fill, one patron issues an ultimatum: choose, public denunciation, or give up a String.
  
  \section{Events, Audits, \& Market Shocks}
  Roll or draw 1--2 between sessions per active asset/patronage.
  \begin{itemize}
    \item \textbf{Inspection:} start \textbf{Inspection [4]}; on fill, tick \textbf{Integrity +1} or \textbf{Sanction +1}.
    \item \textbf{Accident:} \textbf{Accident [4]} threatens staff; on fill, Integrity +1 and \textbf{Audience: Fearful}.
    \item \textbf{Shortages:} \textbf{Resources --1} and start \textbf{Short Rations [4]}.
    \item \textbf{Tax/Lease Hike:} choose coin cost or \textbf{Sanction +1}.
    \item \textbf{Rival Claim:} assert prior right; start \textbf{Litigation [6]} or \textbf{Duel of Proof [4]}.
    \item \textbf{Rumor Run:} if \emph{Crowd-Facing}, add \textbf{Skeptical}; if \emph{Prestige}, add \textbf{Fascinated} (cuts both ways).
    \item \textbf{Favor Called:} a patron demands service; refuse $\to$ \textbf{Sanction +1}.
  \end{itemize}
  
  \section{Blueprints (Projects \& Upgrades)}
  Pick a \textbf{Project [4--8]} to install a blueprint; on fill, add the effect and tag.
  \begin{itemize}
    \item \textbf{Safehouse (4):} add \emph{Fortified} + \emph{Secret}; gain \textbf{Leverage (1)} once/session from stashed goods.
    \item \textbf{Workshop (6):} add \emph{Hazardous}; once/session, turn coin $\to$ \textbf{Clue/Prototype} tag.
    \item \textbf{Archive Annex (6):} add \emph{Archive}; \emph{Research/Expose} DV --1 here; bank \textbf{1 Clue} each cycle if Upkeep is paid.
    \item \textbf{Shrine-Nave (6):} add \emph{Sanctified}; once/session \emph{Host Rite} starts \textbf{Dominant}.
    \item \textbf{Barge Fleet (8):} add \emph{Mobile} + \emph{Water-Right}; once/session move a scene to adjacent district at same Position.
    \item \textbf{Guard Contract (6):} add \emph{Licensed}; once/session \emph{Call the Watch} auto-succeeds at minor level.
    \item \textbf{Union Hall (6):} add \emph{Unionized}; once/session flip a \textbf{Strike} counter one step toward peace (if dues paid).
    \item \textbf{Playhouse Front (4):} add \emph{Crowd-Facing}; create Audience tags more easily; once/session convert \textbf{Audience: Warm} to \textbf{Favor (narrow)}.
  \end{itemize}
  
  \section{Regional Kits (Examples)}
  \subsection*{Mid Ahkaz --- Violet Steppe/Meadows}
  \textbf{Assets:} Caravan Yard (Mobile, Licensed); Dye Vault (Hazardous, Secret).\\
  \textbf{Patrons:} Coin-Weigh Tribunal (Sponsor$\to$Patron), Oasis Clans (Protector with water rites).\\
  \textbf{Events:} forged Water Share deeds; desert guides strike; curfew at Steppe Gate.
  
  \subsection*{Ecktoria --- Marble \& Fire}
  \textbf{Assets:} Aqueduct Valve Lease (Licensed, Ward-Woven); Foundry Line (Hazardous, Unionized).\\
  \textbf{Patrons:} Imperial Exarchate (audit stipends), Legions Remnant (escort writs).\\
  \textbf{Events:} water theft panic; audit sweep; relic procession crowds disrupt supply.
  
  \subsection*{Silkstrand --- City of Bridges}
  \textbf{Assets:} Tollhouse on Archivolt (Licensed, Prestige); Playhouse (Crowd-Facing, Secret).\\
  \textbf{Patrons:} Bridge-Lords (tolls), Playhouse Guild (licenses).\\
  \textbf{Events:} fog bell failure; satire lawsuit; Night Keys leak.
  
  \subsection*{Thepyrgos --- Synod \& Collegium}
  \textbf{Assets:} Harbor Pilot Office (Water-Right, Licensed); Scriptorium Desk (Archive, Prestige).\\
  \textbf{Patrons:} Archons' Synod (writs), Collegium (oath registry).\\
  \textbf{Events:} censure threat; procession scandal; tithe curse.
  
  \subsection*{Zakov --- Iron River, Ash Lanterns}
  \textbf{Assets:} Ash-Lantern Line (Licensed, Hazardous); Smugglers' Cut Lease (Shadowed, Mobile).\\
  \textbf{Patrons:} Ironmasters' Collegium (ore contracts), Lantern Wardens (curfews).\\
  \textbf{Events:} ash curfew; lantern line failure; picket riots.
  
  \section{Example of Play (short)}
  \textbf{Setup:} PCs lease a \emph{Tollhouse} (Licensed, Prestige) in Silkstrand. Integrity 4/6, Resources 3/6. Patron: \emph{Bridge-Lords} (Tier 1 Sponsor; Obligations: dues, public decorum).\\
  \textbf{Scene --- Petition:} They seek a \emph{Tariff Exemption} for a festival barge. Position \textbf{Dominant} (Prestige + patron venue). DV 3. Strong hit: gain the boon; Favor Ledger +1; GM banks SB from a rolled 1 $\to$ \textbf{Rumor Run} starts.\\
  \textbf{Orders:} \emph{Operate} (yield coin; a 1 ticks Heat +1). \emph{Improve} (Archive Annex [6] +1 segment).\\
  \textbf{Event:} \textbf{Inspection [4]} begins. PCs \emph{Secure} next cycle; on a hit reduce Heat --1. A satirical Playhouse offers help for a cut---risking \textbf{Patron Exposure} if mocked.\\
  \textbf{Dilemma:} A rival \textbf{Patron} (Playhouse Guild) offers Tier 2 if they host a scandalous masque; \textbf{Split Loyalty [4]} begins. On fill, Bridge-Lords demand: cancel the masque or surrender \emph{Night Keys (String)}.\\
  \textbf{Campaign Transition:} At arc's end, roll Economic Shift (d6: 3 = No change), Political Change (d6: 1 = Gain new patron opportunity), Physical Event (d6: 6 = Asset takes Integrity -1). Tollhouse becomes "Notorious" legacy with "Scandal Weathered" event.
  
  \section*{GM Reference (one page)}
  \begin{itemize}
    \item \textbf{Asset Tracks:} Integrity [6] (fill = shutdown/seizure), Resources [6] (0 = shortages), Heat [6] (fill = rough visit/audit).
    \item \textbf{Orders:} Operate \textbullet{} Improve \textbullet{} Secure \textbullet{} Expand \textbullet{} Audit \textbullet{} Exploit.
    \item \textbf{Upkeep/Yield:} pay costs $\to$ choose coin/Favor/Clue/Leverage if safe.
    \item \textbf{Patron Tiers:} 0 Contact \textbullet{} 1 Sponsor (+Pos once/scene; 1 stipend) \textbullet{} 2 Patron (cancel 1 Social SB; DV --1 to Petition) \textbullet{} 3 Protector (start Dominant; Bandwagon clock).
    \item \textbf{Sanctions:} Admonish $\to$ Censure $\to$ Seizure/Revocation $\to$ Hunt/Prosecution.
    \item \textbf{Events:} Inspection \textbullet{} Accident \textbullet{} Shortages \textbullet{} Tax/Lease Hike \textbullet{} Rival Claim \textbullet{} Rumor Run \textbullet{} Favor Called.
    \item \textbf{Continuity:} Legacy effects modify starting conditions; patrons retain Standing; transition events reshape holdings.
  \end{itemize}
  
  \newpage

\section{Villains: Narrative Engines of Change}

\subsection{Introduction: Villains as Narrative Engines}

In Fate's Edge, villains are not merely obstacles to be overcome—they are narrative forces that shape the world and drive the story. The best villains in this system are not defined by their hit points or damage output, but by their impact on the world and the choices they force the players to make.

This chapter provides GMs with tools for creating villains that are:
\begin{itemize}
  \item \textbf{Narrative Engines:} Forces that drive the plot forward
  \item \textbf{Systemic Challenges:} Problems that require more than combat to resolve
  \item \textbf{Living Worlds:} Villains who react to player actions
  \item \textbf{Thematic Forces:} Embodiments of the campaign's core themes
\end{itemize}

\noindent\textit{Design Philosophy:} Villains should never feel like static obstacles but dynamic elements that evolve alongside the campaign.

\subsection{The Villain Spectrum: From Obstacle to Catalyst}

Fate's Edge villains exist on a spectrum of narrative impact:

\begin{center}
\begin{tabular}{|p{0.15\textwidth}|p{0.25\textwidth}|p{0.2\textwidth}|p{0.2\textwidth}|}
\hline
\textbf{Type} & \textbf{Description} & \textbf{Example} & \textbf{Best For} \\ \hline
Obstacle & Minor barrier to overcome & Bandit leader & Early Tier I sessions \\ \hline
Adversary & Direct opponent with clear goals & Rival explorer & Mid-tier, character-focused stories \\ \hline
Rival & Equal threat who may become ally & Noble house leader & Political drama \\ \hline
Antagonist & Central to the plot, drives the story & Usurper king & Mid-tier campaigns \\ \hline
Catalyst & Doesn't need to be "beaten"—creates lasting change & The Shattered Crown & High-tier campaigns \\ \hline
\end{tabular}
\end{center}

\subsection{The Villain Construction Toolkit}

\subsubsection{Core Villain Blueprint}

Every villain in Fate's Edge should have:

\begin{itemize}
  \item \textbf{The Drive (Heart):} What do they want and why? (Not just "world domination" but their personal, emotional reason)
  \item \textbf{The Structure (Spades):} How do they operate? Their organization, resources, and methods
  \item \textbf{The Complication (Clubs):} What makes them dangerous beyond raw power?
  \item \textbf{The Reward (Diamonds):} What the players gain by defeating them (not just loot but narrative currency)
\end{itemize}

\textbf{Example: Lord Silas (The Merchant Prince)}
\begin{itemize}
  \item \textbf{Drive:} To prove his family's merchant house is the only one worthy of ruling the trade routes
  \item \textbf{Structure:} A vast commercial empire, with networks in every port city
  \item \textbf{Complication:} His power is diffused; attacking him means angering neutral merchants
  \item \textbf{Reward:} Control of the trade routes, or exposure of his corruption
\end{itemize}

\subsubsection{Villain Clocks}

Rather than hit points, villains are tracked through narrative clocks:

\begin{itemize}
  \item \textbf{Influence Clock (6-8 segments):} How much control they have over the world
  \item \textbf{Momentum Clock (6 segments):} How close they are to achieving their goal
  \item \textbf{Moral Clock (4-6 segments):} How far they've strayed from their original ideals
  \item \textbf{Alliance Clock (4 segments):} Who still stands with them
\end{itemize}

\textbf{Example: The Usurper King}
\begin{itemize}
  \item Influence: 4/8 (controls capital, but not the countryside)
  \item Momentum: 3/6 (gathering forces for invasion)
  \item Moral: 2/4 (killed one noble, now justifying more)
  \item Alliance: 3/4 (most nobles oppose him, but he has the army)
\end{itemize}

\subsubsection{The Villain's Action Economy}

Villains don't just wait to be fought—they act on their own timeline:

\begin{itemize}
  \item \textbf{Strategic Actions:} The villain's major moves (1 per scene)
  \item \textbf{Tactical Actions:} Reactions to player actions (1 per clock segment filled)
  \item \textbf{Environmental Actions:} How the villain shapes the world around them
\end{itemize}

\subsection{Villain Archetypes for Fate's Edge}

\subsubsection{The Systemic Villain}

\textbf{Core Concept:} Not a person but a system, institution, or force that must be reformed or dismantled.

\textbf{Examples:}
\begin{itemize}
  \item The Guild of Silent Traders (corrupt merchant guild)
  \item The Clockwork System (oppressive bureaucracy)
  \item The War Machine (endless conflict)
\end{itemize}

\textbf{Mechanics:}
\begin{itemize}
  \item Multiple clocks representing different facets
  \item Defeat requires changing the system, not killing a leader
  \item Player actions affect multiple clocks simultaneously
\end{itemize}

\textbf{Example: The Fractured Guild}
\begin{itemize}
  \item \textbf{Structure:} Divided into warring merchant houses
  \item \textbf{Complication:} Any action strengthens one house while weakening another
  \item \textbf{Defeat Path:} Reunite the guild or replace it with something new
\end{itemize}

\subsubsection{The Mirror Villain}

\textbf{Core Concept:} A dark reflection of the PCs, showing what they could become.

\textbf{Examples:}
\begin{itemize}
  \item A former ally who took a different path
  \item A future version of one of the players
  \item Someone who shares the PCs' goals but disagrees on methods
\end{itemize}

\textbf{Mechanics:}
\begin{itemize}
  \item Shared clocks with the PCs (Moral Clocks, Relationship Clocks)
  \item Defeat is not always destruction—may require redemption or integration
  \item Player choices directly impact the villain's path
\end{itemize}

\textbf{Example: The Shadow Chancellor}
\begin{itemize}
  \item \textbf{Drive:} To reform the government by any means necessary
  \item \textbf{Mirror Trait:} Willing to sacrifice principles for the greater good
  \item \textbf{Defeat Path:} Can be recruited to the cause or must be stopped
\end{itemize}

\subsubsection{The Unbeatable Villain}

\textbf{Core Concept:} A threat the players cannot defeat through force, only outmaneuver.

\textbf{Examples:}
\begin{itemize}
  \item A plague that can't be cured
  \item A prophecy that seems inevitable
  \item A force of nature
\end{itemize}

\textbf{Mechanics:}
\begin{itemize}
  \item No "defeat" clock—only alternative paths
  \item Player success is measured by reducing consequences
  \item Often requires redefining what "victory" means
\end{itemize}

\textbf{Example: The Drowning City}
\begin{itemize}
  \item \textbf{Drive:} Natural disaster (rising sea levels)
  \item \textbf{Complication:} The city must be saved or abandoned
  \item \textbf{Defeat Path:} There is no defeat—only adaptation or loss
\end{itemize}

\subsection{High-Tier Villain Design}

\subsubsection{The Mythic Antagonist}

\textbf{Core Concept:} A threat that reshapes the world and requires multi-stage resolution.

\textbf{Framework:}
\begin{itemize}
  \item Has their own \textbf{Legacy Project Clock} (8-12 segments)
  \item Their goals are world-shaping
  \item Defeat requires multiple steps across different domains
  \item Leaving a legacy: Defeating them creates a new problem or opportunity
\end{itemize}

\textbf{Example: The World-Weaver}
\begin{itemize}
  \item \textbf{Legacy Project:} Rebuild the world according to their vision
  \item \textbf{Clocks:}
    \begin{itemize}
      \item World Reformation [8]
      \item Follower Loyalty [6]
      \item Moral Cost [10]
    \end{itemize}
  \item \textbf{Defeat Path:} Stop the reformation, reverse the changes, or become the new World-Weaver
\end{itemize}

\subsubsection{The Villain as Player}

In high-tier play, let players take control of a villain for a session:

\begin{itemize}
  \item \textbf{The Villain's Turn:} Alternate between player and GM control of the villain
  \item \textbf{Shared Vision:} Players and GM collaborate on the villain's goals
  \item \textbf{Consequences:} Actions taken by players as the villain affect the real campaign
\end{itemize}

\textbf{Example: The Usurper's Perspective}
\begin{itemize}
  \item Players take turns controlling the Usurper
  \item Each session, one player makes the villain's strategic moves
  \item Their choices directly impact the next session's conflict
\end{itemize}

\subsection{Sample Villain: The Shattered Crown}

\subsubsection{Blueprints}

\begin{itemize}
  \item \textbf{Drive (Heart):} To reunite the fractured kingdoms, no matter the cost
  \item \textbf{Structure (Spades):} A splintered royal court with rival factions
  \item \textbf{Complication (Clubs):} The more he unites, the more he loses himself
  \item \textbf{Reward (Diamonds):} The crown's secrets or his redemption
\end{itemize}

\subsubsection{Clocks}

\begin{itemize}
  \item \textbf{Crown Unity [8]:} How many of the fractured kingdoms he has united
  \item \textbf{Personal Cost [6]:} How much of himself he has sacrificed
  \item \textbf{Resistance [4]:} How many factions still oppose him
\end{itemize}

\subsubsection{Villain Actions}

\textbf{Strategic Actions:}
\begin{itemize}
  \item Unite two city-states (Advance Crown Unity by 1)
  \item Suppress a rebellion (Reduce Resistance by 1, but increase Personal Cost by 1)
  \item Search for the Crown's Secret (Advance Clock, risk Personal Cost)
\end{itemize}

\textbf{Tactical Actions:}
\begin{itemize}
  \item When players intervene, the Shattered Crown may:
    \begin{itemize}
      \item Offer a truce (if Personal Cost is high)
      \item Demand loyalty (risking Resistance)
      \item Withdraw and regroup (strategic retreat)
    \end{itemize}
\end{itemize}

\textbf{Environmental Actions:}
\begin{itemize}
  \item The land itself responds to his actions:
    \begin{itemize}
      \item Unified regions become prosperous
      \item Resisted regions fall into disrepair
    \end{itemize}
\end{itemize}

\subsubsection{Player Options}

\begin{itemize}
  \item \textbf{Unite with the Shattered Crown:} Accept his vision but risk losing themselves
  \item \textbf{Defeat him:} Restore the old order but lose the opportunity for change
  \item \textbf{Transform him:} Find a way to help him heal the crown without sacrificing himself
\end{itemize}

\subsection{GM Implementation Tools}

\subsubsection{Villain Generation Deck}

Using the standard 52-card deck, draw one card per category:

\begin{itemize}
  \item \textbf{Spades (Structure):} 2-5=Minor threat, 6-10=Significant, J-Q-K=Major, A=Pivotal
  \item \textbf{Hearts (Drive):} 2-5=Personal, 6-10=Ideological, J-Q-K=World-shaping
  \item \textbf{Clubs (Complication):} 2-5=Logistical, 6-10=Systemic, J-Q-K=Existential
  \item \textbf{Diamonds (Reward):} 2-5=Material, 6-10=Strategic, J-Q-K=Narrative
\end{itemize}

\subsubsection{Villain Relationship Matrix}

Track how the villain interacts with the world:

\begin{center}
\begin{tabular}{|p{0.15\textwidth}|p{0.2\textwidth}|p{0.15\textwidth}|p{0.2\textwidth}|}
\hline
\textbf{Faction} & \textbf{Relationship} & \textbf{Influence} & \textbf{Current Action} \\ \hline
City Guard & Hostile & 2 & Patrolling borders \\ \hline
Merchant Guild & Tense & 3 & Demanding tribute \\ \hline
Religious Order & Neutral & 1 & Observing \\ \hline
Player Characters & Adversarial & 4 & Preparing defense \\ \hline
\end{tabular}
\end{center}

\subsubsection{Villain Timeline}

Create a timeline of the villain's actions:
\begin{itemize}
  \item What they did before the players arrived
  \item What they're doing now
  \item What they plan to do next
  \item When their clock will fill without intervention
\end{itemize}

\subsection{Philosophy: Why This Approach Works}

\begin{itemize}
  \item \textbf{Narrative First:} Villains become story engines rather than combat encounters
  \item \textbf{Player Agency:} Players engage with villains on their own terms
  \item \textbf{Meaningful Consequences:} Defeating a villain changes the world in tangible ways
  \item \textbf{Scalable:} Works for Tier I to Tier VI play
  \item \textbf{Thematic Resonance:} Villains embody the campaign's core themes
\end{itemize}

The best villains in Fate's Edge are not the most dangerous foes, but those who force players to make hard choices that change who they are. They are the shadows that define the light, the counterpoint that gives the story its meaning.

By using this framework, GMs can create villains that feel like an organic part of the world, whose defeat (or redemption) matters because it changes the narrative in meaningful ways. In a game where "every action matters," villains should be the forces that make those actions truly count.

\newpage
\section{Non-Player Characters: The Living World}

\subsection{Introduction: NPCs as Narrative Threads}

In Fate's Edge, non-player characters are not background props but living threads in the world's tapestry—each with their own stories, desires, and relationships that weave through the campaign. The best NPCs don't just hand out quests or stand in combat—they breathe life into the world and give meaning to player actions.

This chapter provides GMs with tools for creating NPCs that:
\begin{itemize}
  \item \textbf{Emerge from the world:} Rather than being pre-fabricated, NPCs should grow organically from the setting
  \item \textbf{React to player actions:} NPCs who remember, adapt, and respond meaningfully
  \item \textbf{Drive narrative:} Characters who create new story opportunities, not just obstacles
  \item \textbf{Feel real:} Individuals with depth, not cardboard cutouts
\end{itemize}

\noindent\textit{Design Philosophy:} NPCs should feel like people first, game pieces second. The best NPCs are those the players remember long after the mechanics have been forgotten.

\subsection{The NPC Spectrum: From Background to Catalyst}

Fate's Edge NPCs exist on a spectrum of narrative importance:

\begin{center}
\begin{tabular}{|p{0.15\textwidth}|p{0.25\textwidth}|p{0.2\textwidth}|p{0.2\textwidth}|}
\hline
\textbf{Type} & \textbf{Description} & \textbf{Example} & \textbf{Best For} \\ \hline
Cameo & Brief appearance with no lasting impact & Barkeep, messenger & Scene color \\ \hline
Background & Minor role with potential impact & Shopkeeper, guard & World building \\ \hline
Faction Member & Represents a larger group & Guild enforcer, soldier & Political dynamics \\ \hline
Contact & Regular ally with specific skills & Blacksmith, information broker & Player support \\ \hline
Rival & Opponent with competing goals & Merchant, noble, rival explorer & Character conflicts \\ \hline
Foil & Reflects PC traits in a different way & Former mentor, mirror character & Character development \\ \hline
Catalyst & Drives plot through actions & Prophecy bearer, revolutionary & Major story arcs \\ \hline
\end{tabular}
\end{center}

\subsection{The NPC Construction Framework}

\subsubsection{Core NPC Blueprint}

Every meaningful NPC in Fate's Edge should have:

\begin{itemize}
  \item \textbf{The Spark (Heart):} What makes them interesting or memorable beyond their role
  \item \textbf{The Role (Spades):} Their place in the world and connections to power structures
  \item \textbf{The Need (Clubs):} What they want and why they need it now
  \item \textbf{The Connection (Diamonds):} How they relate to the players and campaign themes
\end{itemize}

\textbf{Example: Alis, The Wounded Scholar}
\begin{itemize}
  \item \textbf{Spark:} A brilliant historian whose eyes were burned out by forbidden knowledge
  \item \textbf{Role:} Keeper of the Forbidden Archives, respected but feared
  \item \textbf{Need:} To share her knowledge before her mind fades completely
  \item \textbf{Connection:} Her knowledge holds the key to the city's founding secret
\end{itemize}

\subsubsection{NPC Tracks}

Rather than stat blocks, track key narrative elements:

\begin{itemize}
  \item \textbf{Favor Clock (4-6 segments):} How well-liked the NPC is by the players
  \item \textbf{Loyalty Clock (6 segments):} Their commitment to the party or cause
  \item \textbf{Knowledge Clock (4-8 segments):} What they know and are willing to share
  \item \textbf{Influence Clock (4-6 segments):} Their power and connections in the world
\end{itemize}

\textbf{Example: Captain Rael}
\begin{itemize}
  \item Favor: 3/6 (helped during crisis)
  \item Loyalty: 2/6 (has competing loyalties)
  \item Knowledge: 4/8 (knows about the sea routes)
  \item Influence: 3/4 (respected among sailors)
\end{itemize}

\subsubsection{The NPC's Action Economy}

NPCs don't just wait to be interacted with—they act on their own timeline:

\begin{itemize}
  \item \textbf{Strategic Actions:} Major moves that affect the world (1 per scene)
  \item \textbf{Tactical Actions:} Reactions to player actions (1 per clock segment filled)
  \item \textbf{Personal Journey:} How the NPC is changing based on events
\end{itemize}

\subsection{NPC Archetypes for Fate's Edge}

\subsubsection{The Dynamic Ally}

\textbf{Core Concept:} Not just a sidekick, but an active participant who evolves with the story.

\textbf{Examples:}
\begin{itemize}
  \item Former enemy turned ally
  \item Recruit from the ranks of the common people
  \item Noble with a change of heart
\end{itemize}

\textbf{Mechanics:}
\begin{itemize}
  \item Has their own clocks: Loyalty, Skill, and Morale
  \item Gains experience through exposure to the party's journey
  \item Can gain \textbf{Follower} status with the party through Loyalty Clock advancement
  \item Can choose to leave if Loyalty Clock drops too low
\end{itemize}

\textbf{Example: Kaelen the Outcast}
\begin{itemize}
  \item \textbf{Role:} Scavenger from the edge of the city
  \item \textbf{Loyalty Clock:} 3/6 (helped the party, but fears retribution)
  \item \textbf{Knowledge Clock:} 4/6 (knows the undercity)
  \item \textbf{Growth Path:} At 6/6 Loyalty, gains Follower status; at 0/6, leaves the party
\end{itemize}

\subsubsection{The Shifting Rival}

\textbf{Core Concept:} An opponent whose relationship with the party is dynamic and can change.

\textbf{Examples:}
\begin{itemize}
  \item Noble house rival
  \item Competing explorer
  \item Disgraced former ally
\end{itemize}

\textbf{Mechanics:}
\begin{itemize}
  \item Tracks: Rivalry Clock (6 segments), Trust Clock (6 segments)
  \item Rivalry Clock: Advances when interests conflict; fills at 6
  \item Trust Clock: Advances when the rival trusts the party; fills at 6
  \item When Rivalry Clock fills, the rival must choose: compete, cooperate, or withdraw
  \item When Trust Clock fills, the rival becomes an ally
\end{itemize}

\textbf{Example: Lord Dain, the Rival Noble}
\begin{itemize}
  \item \textbf{Rivalry:} 4/6 (competing for royal favor)
  \item \textbf{Trust:} 2/6 (saw party help a commoner)
  \item \textbf{Tipping Point:} One more conflict will force a choice
\end{itemize}

\subsubsection{The Hidden Patron}

\textbf{Core Concept:} A mysterious figure who guides the party without revealing their true agenda.

\textbf{Examples:}
\begin{itemize}
  \item A masked benefactor
  \item A religious order representative
  \item An old friend from the past
\end{itemize}

\textbf{Mechanics:}
\begin{itemize}
  \item \textbf{Patron Clock:} Tracks the patron's trust in the party (6 segments)
  \item \textbf{Secret Clock:} Tracks how much the party knows about the patron (4 segments)
  \item \textbf{Favor Bank:} Spent to activate hidden benefits
\end{itemize}

\textbf{Example: The Veiled Benefactor}
\begin{itemize}
  \item \textbf{Patron Clock:} 3/6 (has provided safe passage and resources)
  \item \textbf{Secret Clock:} 2/4 (party suspects they're a former royal)
  \item \textbf{Favor Bank:} 2 (can be spent to avoid danger or gain information)
\end{itemize}

\subsection{High-Tier NPC Design}

\subsubsection{The Legacy Figure}

\textbf{Core Concept:} A character who leaves a lasting mark on the world, even when not present.

\textbf{Framework:}
\begin{itemize}
  \item \textbf{Legacy Clock (8-12 segments):} Tracks the character's influence after they leave the story
  \item \textbf{Echoes:} Minor NPCs who carry on their work or philosophy
  \item \textbf{Legacy Events:} Specific effects that occur at key milestones
  \item \textbf{Fading Clock:} Tracks how long their influence lasts
\end{itemize}

\textbf{Example: Master Elara, the Archivist}
\begin{itemize}
  \item \textbf{Legacy Clock:} 7/10 (her knowledge is slowly being lost)
  \item \textbf{Echoes:} Three students who follow different interpretations of her teachings
  \item \textbf{Legacy Events:}
    \begin{itemize}
      \item At 10/10: Her final work is rediscovered
      \item At 5/10: A schism forms among her students
      \item At 0/10: Her knowledge is nearly lost
    \end{itemize}
\end{itemize}

\subsubsection{The World-Builder}

In high-tier play, some NPCs shape the very fabric of the world.

\textbf{Mechanics:}
\begin{itemize}
  \item \textbf{Influence Sphere:} The area or domain they control
  \item \textbf{World Clock:} Tracks how the world changes under their influence
  \item \textbf{Faction Network:} Their connections to other NPCs and organizations
\end{itemize}

\textbf{Example: The First Speaker}
\begin{itemize}
  \item \textbf{Influence Sphere:} The Council of Voices (a governing body)
  \item \textbf{World Clock:} 4/8 (has reshaped trade laws but faces opposition)
  \item \textbf{Faction Network:} 3 major factions, 2 minor ones, and several contacts
\end{itemize}

\subsection{Sample NPC: Elara, The Wounded Scholar}

\subsubsection{Blueprints}

\begin{itemize}
  \item \textbf{Spark (Heart):} A brilliant historian whose eyes were burned out by forbidden knowledge
  \item \textbf{Role (Spades):} Keeper of the Forbidden Archives, respected but feared
  \item \textbf{Need (Clubs):} To share her knowledge before her mind fades completely
  \item \textbf{Connection (Diamonds):} Her knowledge holds the key to the city's founding secret
\end{itemize}

\subsubsection{NPC Tracks}

\begin{itemize}
  \item \textbf{Favor Clock [6]:} 4/6 (helped the party, but they must earn her trust)
  \item \textbf{Loyalty Clock [6]:} 3/6 (willing to guide them through the archives)
  \item \textbf{Knowledge Clock [8]:} 5/8 (can share city history and secrets)
  \item \textbf{Influence Clock [4]:} 2/4 (respected by scholars, distrusted by authorities)
\end{itemize}

\subsubsection{NPC Actions}

\textbf{Strategic Actions:}
\begin{itemize}
  \item Share a fragment of dangerous knowledge (Loyalty Clock +1)
  \item Warn of impending danger (Favor Clock +1)
  \item Withhold information (Loyalty Clock -1)
\end{itemize}

\textbf{Tactical Actions:}
\begin{itemize}
  \item When players make a mistake: "I could help you, but at what cost to your understanding?"
  \item When players help her: "The knowledge you seek is not free—it will change you"
  \item When danger approaches: "We must move quickly before they come for the archives"
\end{itemize}

\textbf{Personal Journey:}
\begin{itemize}
  \item Beginning: Reclusive and distrustful
  \item Midpoint: Opening up to share her knowledge
  \item Climax: Making a choice between safety and truth
\end{itemize}

\subsubsection{Player Options}

\begin{itemize}
  \item \textbf{Earn her trust:} Through patience and respect for knowledge
  \item \textbf{Bargain for information:} Offer to protect the archives
  \item \textbf{Rescue her:} When her knowledge makes her a target
\end{itemize}

\subsection{GM Implementation Tools}

\subsubsection{Relationship Matrix}

Track NPC connections to the world and the party:

\begin{center}
\begin{tabular}{|p{0.15\textwidth}|p{0.2\textwidth}|p{0.15\textwidth}|p{0.2\textwidth}|}
\hline
\textbf{Faction} & \textbf{Relationship} & \textbf{Influence} & \textbf{Current Action} \\ \hline
City Council & Distrustful & 3 & Monitoring her closely \\ \hline
Scholar's Guild & Respected & 4 & Protecting her from the Council \\ \hline
Religious Order & Hostile & 2 & Seeking her books \\ \hline
Player Characters & Cautious & 4 & Seeking knowledge \\ \hline
\end{tabular}
\end{center}

\subsubsection{NPC Evolution Clock}

Track how the NPC changes through the campaign:

\begin{itemize}
  \item \textbf{Level 1:} Background Character (no special mechanics)
  \item \textbf{Level 2:} Notable Character (1 clock)
  \item \textbf{Level 3:} Impactful Character (2 clocks, can evolve)
  \item \textbf{Level 4:} Central Figure (3+ clocks, can change world)
  \item \textbf{Level 5:} World-Shaper (leaves lasting legacy)
\end{itemize}

\subsubsection{The NPC Timeline}

Create a timeline of the NPC's journey:

\begin{itemize}
  \item What they were doing before meeting the players
  \item What they are doing now
  \item What they plan to do next
  \item How they will react to major events
\end{itemize}

\subsection{Philosophy: Why This Approach Works}

\begin{itemize}
  \item \textbf{Narrative First:} NPCs become story engines rather than statistics
  \item \textbf{Player Agency:} Players influence NPC development in meaningful ways
  \item \textbf{Meaningful Relationships:} Player choices create real connections
  \item \textbf{Scalable:} Works for Tier I to Tier VI play
  \item \textbf{Thematic Resonance:} NPCs embody the campaign's core themes
\end{itemize}

The best NPCs in Fate's Edge are not mere obstacles but characters who matter to the story. They should feel like real people who breathe life into the world and respond to the players' actions.

By using this framework, GMs can create NPCs that feel like part of a living, breathing world—one where every interaction has weight and consequence. In a game where "every action matters," NPCs should be the people who make those actions matter most.

\end{document}

