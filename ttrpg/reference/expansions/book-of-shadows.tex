\documentclass[11pt]{book}

%=== PACKAGES ===
\usepackage[utf8]{inputenc}
\usepackage[T1]{fontenc}
\usepackage{lmodern}
\usepackage{microtype}
\usepackage{geometry}
\usepackage{setspace}
\usepackage{titlesec}
\usepackage{hyperref}
\usepackage{enumitem}
\usepackage{xcolor}
\usepackage{array}
\usepackage{booktabs}
\usepackage{longtable}
\usepackage{tcolorbox}
\usepackage{fancyhdr}

%=== PAGE LAYOUT ===
\geometry{margin=1in}
\setstretch{1.15}

%=== COLORS ===
\definecolor{feBlue}{RGB}{30,60,110}
\definecolor{feGold}{RGB}{210,170,70}
\definecolor{feGray}{gray}{0.15}

%=== TITLE FORMATTING ===
\titleformat{\chapter}
  {\normalfont\Huge\bfseries\color{feBlue}}
  {\thechapter}{1em}{}

\titleformat{\section}
  {\normalfont\Large\bfseries\color{feBlue}}
  {\thesection}{1em}{}

\titleformat{\subsection}
  {\normalfont\large\bfseries}
  {\thesubsection}{1em}{}

%=== HEADER/FOOTER ===
\pagestyle{fancy}
\fancyhf{}
\fancyhead[L]{\textit{Fate's Edge: The Latern War of Shadows}}
\fancyhead[R]{\thepage}
\renewcommand{\headrulewidth}{0.4pt}

%=== HYPERREF ===
\hypersetup{
    colorlinks=true,
    linkcolor=feBlue,
    urlcolor=feBlue,
    citecolor=feBlue,
    pdftitle={Fate's Edge: The Latern War of Shadows},
    pdfauthor={Fate's Edge Development Team},
    pdfsubject={Witchcraft Expansion for Fate's Edge RPG},
}

%=== DOCUMENT BEGINS ===
\begin{document}
\makeindex

%--------------------------------------------------------------------
% TITLE PAGE
%--------------------------------------------------------------------
\begin{titlepage}
    \centering
    \vspace*{2cm}

    {\Huge\bfseries The Lantern War of Shadows\par}
    \vspace{0.6cm}
    {\Large\itshape A Campaign Arc for Fate’s Edge\par}
    \vspace{1.5cm}

    % Symbol or Emblem Placeholder (won't error if file is missing)
    \IfFileExists{lantern-symbol-placeholder.png}{%
      \includegraphics[width=0.4\textwidth]{lantern-symbol-placeholder.png}%
    }{%
      \fbox{\rule{0pt}{0.25\textheight}\rule{0.4\textwidth}{0pt}}%
    }\\[1cm]

    {\large\bfseries Nick Gasper}\\
    \vspace{0.2cm}
    {\small With contributions to the Witch Orders, Witch Hunters, and \\ 
    the Book of Shadows Expansion}

    \vfill

    % Epigraph
    \begin{tcolorbox}[
        colback=mistgray!10,
        colframe=shadowviolet,
        width=0.75\textwidth,
        center,
        sharp corners,
        fonttitle=\bfseries,
    ]
    \begin{center}
    \emph{
    “Every patron casts a shadow—\\
    and in that shadow walks the one they were\\
    before the world chose sides.”}
    \end{center}
    \end{tcolorbox}

    \vfill

    {\small \today}

\end{titlepage}

%--------------------------------------------------------------------
% FRONTMATTER
%--------------------------------------------------------------------

\thispagestyle{empty}
\begin{center}
\vspace*{1cm}

{\Large\bfseries Frontmatter}

\vspace{1cm}

\begin{minipage}{0.85\textwidth}
\small

\textbf{The Lantern War of Shadows} is a setting-scale campaign arc for \emph{Fate’s Edge}, blending
grimdark folklore, threshold witchcraft, witch-hunter politics, and the re-emergence of a Lost Patron
whose return threatens the balance between Orders and Hunters.

This book contains:
\begin{itemize}[noitemsep]
    \item The five-tier campaign structure (Tiers I–V)
    \item New Witch Orders, Witch Hunter factions, relics, rites, and talents
    \item Setting tools for running occult conflicts and patron politics
    \item A guided arc leading to the Dream-Desert and the Crucible of Names
    \item Multiple endings, including patron ascension mechanics
\end{itemize}

Nothing in this book replaces core rules; it expands them.

All patrons, cultures, witch orders, and hunter traditions presented here are optional
modules intended to interlock with any Fate’s Edge campaign that makes use of witchcraft,
suppression tools, patron bargains, or narrative rites.

\end{minipage}

\end{center}
\newpage
\setcounter{page}{1}

%===========================================================
\section{Introduction \& Themes}
%===========================================================

\subsection{What This Book Is}
\emph{The Lantern War of Shadows} is a campaign-scale expansion for Fate's Edge, 
focused on the tension between witchcraft and witch hunters, the politics of patrons, 
and the slow erosion of identity by shame, forgetting, and zealotry.

Where the \emph{Book of Shadows} explores witchcraft as intimate, reciprocal, and often
dangerous, this supplement carries those roots into open conflict.  
Across deserts, riverlands, and mist-choked warrens, patron-backed factions maneuver, 
bargain, and burn their way toward a single question:

\begin{center}
\textbf{Who decides what a soul is worth?}
\end{center}

\subsection{What This Campaign Is About}
This campaign arc revolves around three escalating pressures:

\begin{enumerate}[leftmargin=*]
    \item \textbf{A city that refuses to feel shame}—and is punished for it.  
    Heugen, the City of Forgetting, lives under the quiet protection of Aveh, Patron of the Shed-Self.  
    Its wells wash away regret, guilt, and sometimes memory.  
    To outsiders, this is liberation—or heresy.

    \item \textbf{A crusade without mercy.}  
    The adherents of Mykkiel, Patron of Lawful Perfection, march across the wastes.  
    They come not to convert but to \emph{correct}.  
    Theologically, Heugen cannot be allowed to exist.

    \item \textbf{A Shadow War beneath both.}  
    Witch Orders, renegade Ghe'hai, spirit courts, lantern inquisitors, 
    and patron-touched wanderers scheme beneath the visible conflict.  
    Each believes the outcome of the siege will shape the fate of names, 
    identities, and forgotten shadows for generations.
\end{enumerate}

\subsection{Campaign Tone}
This campaign leans toward:

\begin{itemize}[leftmargin=*]
    \item \textbf{Grimdark Fairytale} — old bargains, broken oaths, and living shadows.
    \item \textbf{Threshold Horror} — possession, name-eating spirits, and witchburners.
    \item \textbf{Political Folklore} — zealotry, exile, and the dangerous weight of identity.
    \item \textbf{Sword-and-Salt Fantasy} — desert caravans, dune lanterns, temple sieges.
\end{itemize}

Magic is not cinematic.  
It is \emph{consequential}.  
It is \emph{slow}.  
It is \emph{priced in names, secrets, and blood}.  

The question is not whether the party will pay, but \emph{how much}.

\subsection{Central Themes}
This book weaves several recurring themes throughout the campaign:

\paragraph{Identity and Forgetting.}
What is a person without the weight of their past?  
What remains when memory is soft as sand?  
Aveh's gifts liberate \emph{and} unmoor.

\paragraph{Law versus Mercy.}
Mykkiel's emissaries believe mercy is disorder; order is salvation.  
Their lanterns burn truth from shadow—and people with it.

\paragraph{Power Earned vs. Power Taken.}
Witchcraft is relational.  
Witch hunting is extractive.  
Patrons do not act, but those who claim their authority shape the world.

\paragraph{Shame as Weapon.}
The crusaders wield shame as a binding.  
Aveh's followers shed shame like a snakeskin—sometimes too quickly.

\paragraph{The Price of Choosing.}
Every faction in this war believes they are saving something.  
None agree on what that something is.

\subsection{Structure of the War}
The campaign is divided into five acts:

\begin{enumerate}[leftmargin=*]
    \item \textbf{Act I — The Approach of the Lanterns} (Tier I–II)  
    The army of Mykkiel marches; emissaries infiltrate Heugen; shadows begin to stir.

    \item \textbf{Act II — The Wells Remember} (Tier II)  
    Strange desert omens, shame-worm sightings, and fractures in Aveh's cult.

    \item \textbf{Act III — The Breaking of the First Lantern} (Tier III)  
    Witch Orders, Ghe'hai, and Hunters collide in open occult conflict.

    \item \textbf{Act IV — The Crucible of Names} (Tier III–IV)  
    The siege reaches its spiritual climax; bargains are struck; names are lost or reforged.

    \item \textbf{Act V — The Last Shadowborne} (Tier IV–V)  
    The party decides the fate of the city, the crusade, the Wells-Worm,  
    and the balance between shame, identity, and power.
\end{enumerate}

\subsection{Who This Book Is For}
This expansion is written for:

\begin{itemize}[leftmargin=*]
    \item GMs who want witchcraft to feel dangerous and personal  
    \item Players who enjoy moral grayness and hard choices  
    \item Groups that like political intrigue with supernatural stakes  
    \item Campaigns dealing with patrons, rituals, or identity magic  
    \item Tales of exile, zealotry, haunting, and slow-burning dread  
\end{itemize}

\subsection{What This Book Assumes}
This campaign assumes:

\begin{itemize}[leftmargin=*]
    \item Witch Orders exist and operate openly or semi-openly  
    \item Witch Hunters are present, organized, and increasingly aggressive  
    \item Patrons influence culture but do not act directly  
    \item Identity magic and name-magic are meaningful parts of the world  
\end{itemize}

Groups may adapt or scale back any of these assumptions.

\subsection{How to Use This Book}
Use each chapter modularly:

\begin{itemize}[leftmargin=*]
    \item Insert witch orders or hunters into existing settings  
    \item Run Act I–III as a standalone war arc  
    \item Use rites, hexes, and hunter tools in unrelated campaigns  
    \item Introduce Mykkiel or Aveh as long-term patronal influences  
\end{itemize}

Everything here is opt-in.  
Nothing replaces core rules.  
Everything expands them.

%===========================================================
\section{The City of Heugen, the City of Forgetting}
%===========================================================

\subsection{Overview}
Heugen stands on the sun-bleached shore of the Amaranthine, its cisterns and well-houses 
humming with desert wind and the soft undertone of something older.  

Two centuries ago, Temple of Light crusaders founded the settlement during an ill-fated
campaign. After their defeat and retreat, the survivors discarded their luminous banners,
their doctrines, and—eventually—themselves.

In their place arose a new devotion:  
\begin{center}
\textbf{Aveh, the Shed-Self, Patron of Lost Names and Unburdening.}
\end{center}

The wells of Heugen offer release from shame, regret, and burdensome memory.  
To its people, this is mercy.  
To its enemies, this is erasure.

Today Heugen is a bustling port-city of exiles, runaways, un-gendered priests of Aveh,
and those who would rather forget than burn.

The crusaders of Mykkiel call it an abomination.

\subsection{Heugen at a Glance}

\begin{description}[leftmargin=1.8cm]
    \item[Population:] 14,000 resident; 3,000–6,000 transient  
    \item[Government:] Council of Wells; rotating civic stewards  
    \item[Dominant Patron:] Aveh  
    \item[Outsider Perception:] Den of degenerates, heretics, and name-thieves  
    \item[Local Term for Outsiders:] \emph{Lantern-bearers} or \emph{Rememberers}
\end{description}

Heugen is not lawless—but its laws prioritize mercy over order, intention over doctrine,
and identity as something mutable rather than fixed.

This alone is enough to draw the wrath of Mykkiel's zealots.

%-----------------------------------------------------------
\subsection{The Wells of Forgetting}
%-----------------------------------------------------------

The heart of Heugen is its network of ancient, patron-touched wells.  
They do not erase memory; they soften its edges.

A draught from the wells may:
\begin{itemize}[leftmargin=*]
    \item dull the pain of shame  
    \item weaken the hold of traumatic memory  
    \item blur the contours of old identity  
    \item help one shed names, roles, and obligations  
\end{itemize}

For some, this is healing.  
For others, this is spiritual death.

Each well has a \textbf{Custodian}, a devotee of Aveh who has shed all markers of gender, rank, and lineage. The Custodians insist the water reveals one's true self—by dissolving the false ones.

\paragraph{Mechanical Hook: Memory Softening}
Characters who drink may temporarily:
\begin{itemize}
    \item clear 1 Fatigue or 1 Shadow  
    \item remove a Shame Condition  
    \item also suffer a \emph{Name-Fraying} Condition (GM-defined)  
\end{itemize}

The wells do not lie.  
But they do not care what the truth costs.

%-----------------------------------------------------------
\subsection{Factions Within Heugen}
%-----------------------------------------------------------

Heugen is not unified. It is a tapestry of competing visions of what forgetting means.

\subsubsection{The Custodians of Aveh}
Genderless, faceless (often masked), serene.  
They preach freedom through unmaking—metaphorical or otherwise.

\textbf{Motives:}  
\begin{itemize}[leftmargin=*]
    \item protect the wells  
    \item guide pilgrims into rebirth  
    \item resist Mykkiel's doctrine of fixed identity  
\end{itemize}

\subsubsection{The Tanners' Quarter (The Rememberers)}
Artisans, traders, and families who only drink sparingly.  
They fear losing too much of themselves.

\textbf{Motives:}  
\begin{itemize}[leftmargin=*]
    \item maintain stable civic life  
    \item avoid Aveh zealotry  
    \item oppose Mykkiel without shedding their own roots  
\end{itemize}

\subsubsection{The Shadowed Coteries (Witch Orders)}
Mab's night-witches, Morag's cursing hags, Livaea's handmaidens,  
and the Rainmaidens of Raéyn all find quiet footholds in Heugen.  

\textbf{Motives:}  
\begin{itemize}[leftmargin=*]
    \item protect Heugen as a sanctuary  
    \item fight against inquisitorial overreach  
    \item shape the shadow-war for their own patrons  
\end{itemize}

\subsubsection{The Lantern Dissidents}
Former believers in Mykkiel who fled the dogma but kept the discipline.

\textbf{Motives:}  
\begin{itemize}[leftmargin=*]
    \item oppose slaughter  
    \item warn the Council of the coming crusade  
    \item atone for their past  
\end{itemize}

%-----------------------------------------------------------
\subsection{Outside Heugen: What the World Believes}
%-----------------------------------------------------------

To the outside world—especially to Mykkiel's priests—Heugen represents the following heresies:

\paragraph{Heretical Belief 1: The Past Does Not Define You.}  
Mykkiel's faith teaches the opposite:  
\emph{A life is a ledger. Every act matters. Every sin must be corrected.}

\paragraph{Heretical Belief 2: Names Can Be Shed.}  
To Mykkiel, a name is sacred and fixed.  
Names are divine contracts, not costumes.

\paragraph{Heretical Belief 3: Shame Should Be Healed, Not Weaponized.}  
The crusaders wield shame as both lantern and blade.

\paragraph{Heretical Belief 4: Patronal Mercy May Override Law.}  
Heugen openly states this.

\medskip

In truth, the crusaders fear Heugen for one reason:  
\begin{center}
\textbf{People who shed their shame cannot be controlled.}
\end{center}

%-----------------------------------------------------------
\subsection{The Desert Around Heugen}
%-----------------------------------------------------------

The Galaninan desert is a character in its own right.

\begin{itemize}[leftmargin=*]
    \item shifting dunes like drowned cathedrals  
    \item salt flats that mirror the moon too perfectly  
    \item abandoned crusader waystations still bearing lantern scars  
    \item subterranean aquifers connected to the Wells-Worm  
\end{itemize}

Witch Orders call the region \textbf{the Threshold Belt}—a place where names wear thin and spirits slip close.

\subsection{Adventure Hooks in the City}
\begin{itemize}[leftmargin=*]
    \item A Custodian begs the PCs to stop a young pilgrim from drinking too deeply.  
    \item A witch hunter cell infiltrates the city under falsified shame documents.  
    \item A well goes "dry," whispering instead of giving water.  
    \item A Mykkielite emissary challenges the Council to trial by illumination.  
    \item Rainmaidens predict that the Wells-Worm is stirring.  
\end{itemize}

\subsection{Why Heugen Matters}
Because Heugen is the one place in the desert where:

\begin{itemize}[leftmargin=*]
    \item names can be remade  
    \item shame can be laid down  
    \item witchcraft is not hunted  
    \item identity is fluid and not punished  
\end{itemize}

If it falls, the world learns only one truth:  
\begin{center}
\textbf{There is only the Lantern, and all shadows must burn.}
\end{center}

%===========================================================
\section{Factions of the Lantern War}
%===========================================================

Heugen is not merely a city under siege; it is the epicenter of a spiritual contest between
patrons whose desires are incompatible. The conflict manifests through orders, cults, guilds,
and militant sects—each convinced they act on behalf of their patron's true will.

These factions do not simply fight; they argue over what \emph{name}, \emph{shame}, and
\emph{identity} should mean.

%-----------------------------------------------------------
\subsection{The Crusaders of Mykkiel}
%-----------------------------------------------------------

\subsubsection*{Doctrine}
Mykkiel teaches that:
\begin{itemize}[leftmargin=*]
    \item Names are sacred, immutable contracts  
    \item One's past actions define one's worth  
    \item Shame is a corrective light  
    \item Mercy is lawful only when earned  
\end{itemize}

To Mykkiel's faithful, Heugen is a heresy of dissolution.

\subsubsection*{Structure}
\begin{description}[leftmargin=1.8cm]
    \item[The Lantern Host:] disciplined infantry wielding mirrored shields  
    \item[The Order of the Ledger:] inquisitors who record sins as debts  
    \item[The Bright-Mantled:] ascetic zealots  
\end{description}

\subsubsection*{Why They March}
\begin{itemize}[leftmargin=*]
    \item To restore the "true names" of Heugen's people  
    \item To extinguish the Wells of Forgetting  
    \item To prevent patronal corruption (especially witchcraft)  
\end{itemize}

\textbf{Mechanical Hook:}  
Enter any scene where Mykkiel's emissaries appear with Position–1 unless you show a fixed identity or oath.

%-----------------------------------------------------------
\subsection{The Shed-Faith of Aveh}
%-----------------------------------------------------------

Aveh is the patron of unmaking—of those who step out of themselves to become something new,  
or nothing at all.

\subsubsection*{Doctrine}
\begin{itemize}[leftmargin=*]
    \item Identity is a temporary shelter, not a prison  
    \item Shame is a parasite that can be shed  
    \item A person's true nature emerges when the false names fall away  
    \item Mercy is transformation, not punishment  
\end{itemize}

\subsubsection*{Cultic Faces}
Because patrons have no single form, Aveh's agents differ across cultures:
\begin{itemize}
    \item \textbf{Ykrul:} The Faceless Guide  
    \item \textbf{Ikasha:} The Silent Mask  
    \item \textbf{Tulkani:} The Mirror-Eater  
\end{itemize}

\subsubsection*{Agents}
Custodians of the wells—genderless, nameless, masked—shepherd pilgrims through shedding rites.

\textbf{Mechanical Hook:}  
Once per session, a PC may erase one Shame or Shadow by embracing an aspect of Aveh, gaining a ``Shed'' tag for the scene.

%-----------------------------------------------------------
\subsection{The Witch Orders}
%-----------------------------------------------------------

Witchcraft in Fate's Edge is not unified. Each order serves a different patron, with its own  
ethic, aesthetics, and dangers.

Heugen is one of the few cities where witchcraft can be practiced openly—part of why the crusade comes.

Below are the four major Orders present during the Lantern War.

%-----------------------------------------------------------
\subsubsection{The Night Court of Mab}
%-----------------------------------------------------------

Mab's witches operate by glamour, oath-binding, and shadowcraft. They defend Heugen not for mercy  
but because the Wells create fertile ground for faerie bargains.

\textbf{Signatures:}
\begin{itemize}[leftmargin=*]
    \item glamour-warped beauty  
    \item oath-entangling curses  
    \item honeyed lies that become truths  
\end{itemize}

\textbf{Rivalries:} despise Mykkiel's rigid doctrine; tolerate Aveh's dissolution; exploit Rainmaidens.

\textbf{Adventure Hook:} A Mab-witch offers PCs a bargain: "I will hide you from the Lanterns—but you must speak no truth under moonlight for three days."

%-----------------------------------------------------------
\subsubsection{The Ash Hags of Morag}
%-----------------------------------------------------------

Morag's witches embody rot, inevitability, and consequence.  
They perform harsh justice—\emph{the kind that does not forgive, only concludes}.

\textbf{Signatures:}
\begin{itemize}[leftmargin=*]
    \item bone-runes  
    \item rot-aspected curses  
    \item prophetic teeth  
\end{itemize}

\textbf{Rivalries:} oppose Livaea's seductions; scorn Aveh's soft escapes; treat Mykkiel as delusional.

\textbf{Adventure Hook:} A hag predicts that if the Wells fall, the shame cast onto the desert will animate into a storm of screaming silhouettes.

%-----------------------------------------------------------
\subsubsection{The Handmaidens of Livaea}
%-----------------------------------------------------------

Livaea's witches are diplomats, seducers, and weavers of relational power.  
They are not submissive; they are strategically irresistible.

\textbf{Signatures:}
\begin{itemize}[leftmargin=*]
    \item persuasive glamours  
    \item emotional resonance magic  
    \item social rituals and soft dominance  
\end{itemize}

\textbf{Rivalries:} compete with Mab for influence; ally with Aveh in matters of shame; loathe Mykkiel's purity doctrines.

\textbf{Adventure Hook:} A Handmaiden recruits the PCs to sabotage a zealot's arranged marriage ceremony.

%-----------------------------------------------------------
\subsubsection{The Rainmaidens of Raéyn}
%-----------------------------------------------------------

Raéyn's sorceresses are raw elementalists of storm, tide, and lunar pull.  
Their moods shape their magic—and their magic shapes Heugen's weather.

\textbf{Signatures:}
\begin{itemize}[leftmargin=*]
    \item storms that echo emotion  
    \item tidal flux rites  
    \item sea-glass divination  
\end{itemize}

\textbf{Rivalries:} challenge Mab in moonlit matters; disdain Morag's fatalism; fear Mykkiel's zeal.

\textbf{Adventure Hook:} A Rainmaiden's uncontrolled tempest threatens Heugen as crusader scouts approach.

%-----------------------------------------------------------
\subsection{Ikasha's Shadow-Door Witches}
%-----------------------------------------------------------

The cult of Lethai-ar in Ikasha, the Tulkani, and the Sidhi is nothing like the honor-bound  
Lethai-ar of Isoka or Inaea. Here, Lethai-ar is a mask of secrecy, subversion,  
and quiet infiltration.

\textbf{Hallmarks:}
\begin{itemize}
    \item shadow-lattices  
    \item coded rituals  
    \item mask-craft  
    \item silence-as-power  
\end{itemize}

They are the ones who know Heugen's secret paths, the underside of the wells,  
and the true shapes of desert spirits.

\textbf{Adventure Hook:} A shadow-door witch requests the PCs steal a Mykkielite ledger to erase the name of a hunted orphan.

%-----------------------------------------------------------
\subsection{The Witch Hunters}
%-----------------------------------------------------------

Masculine-coded orders that serve as the destructive counterpoint to witchcraft.  
Their purpose is not balance—it is control.

\textbf{Three Major Orders:}
\begin{itemize}
    \item \textbf{Chain-Lanterns of Thepyrgos:} process, law, sanctioned violence  
    \item \textbf{Stone-Correctors of Aeler:} collapse prevention, ritual auditing  
    \item \textbf{Sum-Abjurers of Aelinnel:} definitions, contexts, nonlethal containment  
\end{itemize}

All agree on one thing:  
\emph{Witches should be regulated, contained, or destroyed.}

Mykkiel's crusaders view them as convenient auxiliaries.

%-----------------------------------------------------------
\subsection{Internal Factions of Heugen}
%-----------------------------------------------------------

\subsubsection{The Council of Wells}
Administrators who manage public needs, keep peace between witch orders, and negotiate with  
external powers.

\textbf{Weakness:} They do not always agree on what Aveh actually wants.

\subsubsection{The Unnamed Choir}
A secretive cult who have shed so many identities they barely speak.  
They move like one mind, chanting wordless hymns near the wells.

\subsubsection{The Bound Ledger}
Former Mykkielites who maintain civic records, believing that naming is not heresy—only punishment is.

\subsubsection{The Sand-Bound Trades}
Merchants, water-barons, ferrymen, and smugglers.  
Their allegiance shifts with profit.

%-----------------------------------------------------------
\subsection{External Stakeholders}
%-----------------------------------------------------------

\subsubsection{The Desert Nomad Houses}
Traders and raiders who rely on Heugen's wells as a rare source of civility.  
If the crusade burns Heugen, the desert becomes lawless.

\subsubsection{The Witch Courts Abroad}
Some want Heugen defended.  
Others want it to fall so that Mykkiel overextends and can be undermined.

\subsubsection{The Wells-Worm}
A desert spirit or monster (depending on who you ask).  
Its stirrings twist the crusade's rhetoric into prophecy.

%-----------------------------------------------------------
\subsection{Why These Factions Cannot Coexist}
%-----------------------------------------------------------

\begin{itemize}[leftmargin=*]
    \item Aveh teaches identity as choice.  
    \item Mykkiel teaches identity as law.  
    \item Witch Orders thrive in ambiguity.  
    \item Witch Hunters thrive in clarity.  
\end{itemize}

Heugen stands where these doctrines intersect—and tear each other apart.

\medskip

\begin{center}
\emph{This is not merely a battle for a city.  
It is a battle for what a person is allowed to be.}
\end{center}

%===========================================================
\section{The Shape of the Lantern War}
%===========================================================

Heugen stands on the threshold of transformation.  
Armies gather, wells stir, witch orders whisper omens, and strangers cross the desert  
carrying doctrines, grudges, and forgotten names.

Beneath all this, something older moves.

This section outlines the adventure's major arcs, foreshadowing the hidden dangers that will
reshape the conflict: the arrival of a third force, the \textbf{Thorns of Malachai}, whose
intentions remain unknown even to most witch orders.

%-----------------------------------------------------------
\subsection{Act I: The Smoke on the Horizon}
%-----------------------------------------------------------

\subsubsection*{The Siege Approaches}

The Crusaders of Mykkiel march under banners of mirrored flame.  
Refugees arrive. Wells ripple. Witch orders scramble for position.

The PCs begin amid rising tension:
\begin{itemize}[leftmargin=*]
    \item disputes between witch orders in public squares  
    \item crusader scouts interrogating caravans  
    \item wells-watered pilgrims unable to recall their former lives  
    \item masked Aveh devotees urging preparation for "shedding time"  
\end{itemize}

\paragraph{Foreshadow: The First Thorn}  
Strange reports circulate:
\begin{itemize}
    \item caravans found pierced with obsidian spikes  
    \item couriers vanishing on open roads  
    \item a witch-hunter cell wiped out without signs of struggle  
\end{itemize}

No faction claims responsibility.

Aeler Correctors note the spike pattern resembles ``\emph{ritual extraction},''  
but cannot identify the patron involved.

%-----------------------------------------------------------
\subsection{Act II: The Walls Begin to Whisper}
%-----------------------------------------------------------

As tensions escalate, the PCs navigate Heugen's fracturing alliances.

\subsubsection*{Key Pressures}
\begin{itemize}[leftmargin=*]
    \item \textbf{Mykkiel's ultimatum:} surrender all who refuse to reclaim their ``true names.''  
    \item \textbf{Aveh's visionaries:} urging the city to shed name, past, and shame before the siege.  
    \item \textbf{Witch factions:} competing for influence and leverage, offering bargains, protection, or secrets.  
\end{itemize}

\paragraph{Foreshadow: The Second Thorn}  
More subtle signs appear:
\begin{itemize}
    \item A Mab-witch's glamour collapses in terror after glimpsing a masked figure  
    \item Tidewater scrying pools show thorn-shrouded silhouettes beneath the desert  
    \item A Morag hag finds a bone rune she did not carve, reading only: \emph{"REMEMBER."}
\end{itemize}

The PCs may not yet understand the omen, but its weight is felt.

%-----------------------------------------------------------
\subsection{Act III: The Lanterns Strike}
%-----------------------------------------------------------

The siege begins in earnest.  
Crusader forces breach outer wards; witchcraft flickers under Mykkielite dampening rites.

PCs must:
\begin{itemize}[leftmargin=*]
    \item defend key districts  
    \item extract civilians  
    \item disrupt crusader ritual lines  
    \item stabilize rogue witch workings  
\end{itemize}

\paragraph{The Battle's Turning Point}  
A sudden collapse of crusader formations reveals something troubling:
\begin{itemize}
    \item knights found dead without wounds  
    \item witch-hunters transfixed by blackened thorns  
    \item entire squads missing, shields left behind  
\end{itemize}

Mykkiel's leadership blames witchcraft.  
The witches blame saboteurs among crusaders.  
Aveh's devotees whisper: \emph{"Not ours."}

The truth is a third faction has entered the city—quietly, efficiently, and not on anyone's side.

%-----------------------------------------------------------
\subsection{Act IV: The Desert Opens Its Eyes}
%-----------------------------------------------------------

During a key mission—perhaps a rescue, sabotage, or negotiation—the PCs find evidence that
the attacks during the siege are not random.

\subsubsection*{Clues}
\begin{itemize}[leftmargin=*]
    \item a blackened root-like growth inside a crusader's helmet  
    \item an abandoned room where every reflective surface has been shattered  
    \item a Wells acolyte murmuring in a foreign cadence before vanishing  
    \item an Aeler stone-scribe's ledger burned in patterns matching an unfamiliar sigil  
\end{itemize}

\paragraph{Foreshadow: The Third Thorn}  
The final omen before their arrival:
\begin{quote}
A procession of masked figures crossing the dunes at twilight—  
their shadows moving out of sync with their bodies.
\end{quote}

No one knows where they came from.

\textbf{But the desert remembers the name Malachai.}

%-----------------------------------------------------------
\subsection{Act V: When Three Wills Collide}
%-----------------------------------------------------------

The climax reveals the siege of Heugen is merely one part of a larger struggle.

\subsubsection*{The Truth Emerges}
The Thorns of Malachai are:
\begin{itemize}[leftmargin=*]
    \item not aligned with Aveh  
    \item not aligned with Mykkiel  
    \item not servants of any witch order  
    \item and not entirely mortal  
\end{itemize}

Their goal is ambiguous—either the reclamation of names, the unmaking of identity, or the
\emph{replacement} of both with something new.

This forces the PCs into a three-sided conflict:
\begin{itemize}
    \item defend Heugen from Mykkiel's purges  
    \item prevent Aveh's faithful from dissolving the city  
    \item stop Malachai's Thorns from reshaping the desert's spiritual geometry  
\end{itemize}

At stake is not merely the city, but:
\begin{center}
\textbf{What memories are allowed to endure.}
\end{center}

%-----------------------------------------------------------
\subsection{Ongoing Themes}
%-----------------------------------------------------------

\paragraph{Identity as Weapon}  
Every side uses names—claimed, shed, stolen, or rewritten—as tools of power.

\paragraph{Shame as Currency}  
The crusaders trade in judgment; the Wells trade in oblivion; the Thorns trade in extraction.

\paragraph{Memory as Territory}  
Who controls the past controls the shape of the self.

%-----------------------------------------------------------
\subsection{GM Guidance: Using the Thorns}
%-----------------------------------------------------------

The Thorns should:
\begin{itemize}[leftmargin=*]
    \item be frightening but enigmatic  
    \item appear sparingly, never speaking clearly  
    \item never contradict patron doctrine (they are not patron-servants)  
    \item disrupt both witchcraft and crusader magic  
    \item force hard choices without offering answers  
\end{itemize}

They are not a faction to negotiate with.  
They are a disturbance—one that reveals deeper truths about identity, shame, and memory.

\begin{center}
\emph{First the Wells forgot their names.  
Now something else remembers them.}
\end{center}

%===========================================================
\section{Mechanics of the Lantern War}
%===========================================================

The siege of Heugen is not merely a clash of armies.  
It is a conflict of doctrines, identities, and the spiritual laws that govern memory itself.

This section provides structured mechanics for:
\begin{itemize}[leftmargin=*]
    \item Mykkielite crusader miracle-lines
    \item Aveh shedding rites and dissolution boons
    \item Witch-order crisis workings
    \item Thorn corruption events and Identity Strain
    \item Citywide siege clocks
    \item Emissary clash event tables
\end{itemize}

%-----------------------------------------------------------
\subsection{5.1 Citywide Siege Clocks}
%-----------------------------------------------------------

The GM tracks four major siege pressures:

\begin{description}
    \item[Lantern Advance (6):] Crusader forces tightening their grip.
    \item[Wells Unraveling (6):] Instability in Heugen's memory-warped wells.
    \item[Witch Fracture (4):] Internal disputes escalating into sabotage.
    \item[Thorn Encroachment (4):] Subtle corruption spreading unseen.
\end{description}

\paragraph{When a Clock Fills:}
\begin{itemize}[leftmargin=*]
    \item \textbf{Lantern Advance:} A Mykkielite miracle suppresses all witchcraft for a scene.
    \item \textbf{Wells Unraveling:} The city collectively forgets a small but important detail.
    \item \textbf{Witch Fracture:} A faction blames another; position starts Desperate in all witch–witch negotiations.
    \item \textbf{Thorn Encroachment:} A Thorn Stalker appears, even in broad daylight.
\end{itemize}

%-----------------------------------------------------------
\subsection{5.2 Mykkielite Miracle-Lines}
\index{Rites!Mykkielite}
%-----------------------------------------------------------

Crusaders channel divine law through structured miracle-lines.  
These are \emph{rigid, absolute}, and each miracle demands personal purity.

\subsubsection*{Miracle of Enumerated Sin (Low)}
\texttt{[REVEAL] [CONDEMN]}
\begin{itemize}
    \item Identify one hidden shame, breach, or unresolved vow in a target.
    \item Target suffers $-1d$ on actions involving deception, glamours, or false names.
\end{itemize}

\subsubsection*{Miracle of the Sevenfold Lantern (Standard)}
\texttt{[LIGHT] [PURIFY] [WARD]}
\begin{itemize}
    \item Establish a warded radius where witchcraft \emph{flickers}.
    \item All rites require +1 DV to cast within this zone.
\end{itemize}

\subsubsection*{Miracle of Perfect Judgment (High)}
\texttt{[LAW] [BIND] [EXECUTE]}
\begin{itemize}
    \item Bind a target in a silhouette of mirrored flame.
    \item If the target acted under false name or illusion, Position becomes \emph{Desperate} for them.
\end{itemize}

\paragraph{Miracle Consequences:}
Rolling 1s generates \textbf{Reprisal SB}, which may:
\begin{itemize}[leftmargin=*]
    \item fracture crusader unity
    \item invoke a contradictory doctrine
    \item attract an unseen Thorn that feeds on absolute claims
\end{itemize}

%-----------------------------------------------------------
\subsection{5.3 Aveh Shedding Rites}
\index{Rites!Aveh}
%-----------------------------------------------------------}

Aveh's rites revolve around anonymity, dissolution of self, and the transcendence found in shedding one's past.

\subsubsection*{Rite of the Unwoven Mask (Low)}
\texttt{[VEIL] [SILENCE]}
\begin{itemize}
    \item Suppress your name for a scene.
    \item Gain +1d on actions requiring stealth, deception, or disappearance.
\end{itemize}

\subsubsection*{Rite of the Shedding Path (Standard)}
\texttt{[MIRROR] [RELEASE]}
\begin{itemize}
    \item Remove one Condition by abandoning the memory that caused it.
    \item Mark 1 \textbf{Identity Strain}.
\end{itemize}

\subsubsection*{Rite of the Faceless Dawn (High)}
\texttt{[ASCENT] [ERASE] [REMAKE]}
\begin{itemize}
    \item Spend 1 Fatigue.
    \item Choose one: gain immunity to name-based magic for a scene, or become \emph{untrackable}.
\end{itemize}

\paragraph{Identity Strain [4]:}  
Every time a PC uses advanced Aveh rites, fill one segment.  
At 4, roll Spirit + Wits (DV 4):

\begin{itemize}
    \item \textbf{Success:} You master the dissolution; clear 1 segment.
    \item \textbf{Failure:} You forget something meaningful; GM chooses.
\end{itemize}

%-----------------------------------------------------------
\subsection{5.4 Witch Orders: Crisis Workings}
%-----------------------------------------------------------}

During the siege, witch orders employ forbidden or unstable workings.

\subsubsection*{Hearth Witches: Iron-Binding Charm}
\texttt{[HOME] [ANCHOR] [BIND]}
\begin{itemize}
    \item Freeze one enemy in place with the power of hearth and threshold.
    \item DV increases by +1 for each foreign magic currently active nearby.
\end{itemize}

\subsubsection*{Rainmaidens: Tidebreaker Surge}
\texttt{[STORM] [ZONE] [COMMAND]}
\begin{itemize}
    \item Redirect the flow of a battle for one exchange.
    \item All allies gain Position +1; enemies suffer $-1d$.
\end{itemize}

\subsubsection*{Mab Witches: Thorn of Oaths-Broken}
\texttt{[HEX] [OATH] [WOUND]}
\begin{itemize}
    \item Impose a spiritual wound on someone who broke a promise.
    \item On miss, the witch suffers the oath's recoil.
\end{itemize}

%-----------------------------------------------------------
\subsection{5.5 The Thorns of Malachai: Corruption Mechanics}
\index{Corruption!Thorns of Malachai}
%-----------------------------------------------------------}

The Thorns do not cast rites — they enact \textbf{extractions}.  
They do not rewrite identity — they \emph{strip} it.

\subsubsection*{Identity Strain: A Shared Track}

All PCs share an \textbf{Identity Strain [6]} track for Thorn encounters.

\paragraph{Mark when:}
\begin{itemize}[leftmargin=*]
    \item A Thorn Stalker silently observes the party
    \item PCs contradict their own memories
    \item A crusader miracle and witch rite collide
    \item A PC speaks their full name in fear
\end{itemize}

\paragraph{When Full:}
The GM chooses:

\begin{itemize}[leftmargin=*]
    \item \textbf{Extracted Memory:} PCs forget how they entered the scene.
    \item \textbf{Shadow Double:} A Thorn imitates one PC's silhouette.
    \item \textbf{False Witness:} Evidence appears that PCs committed a past atrocity.
\end{itemize}

%-----------------------------------------------------------
\subsection{5.6 Emissary Clash Event Tables}
%-----------------------------------------------------------}

\subsubsection*{When Witch and Crusader Agents Collide}
Roll d6:

\begin{enumerate}[leftmargin=*]
    \item A miracle backfires, igniting witchcraft into wild surges.
    \item A witch's glamour reveals hidden shame in a crusader officer.
    \item A Thorn observes from a rooftop — unnoticed at first.
    \item A well-watered citizen intervenes, confused and panicked.
    \item A rogue hag offers "arbitration" at a steep price.
    \item The ground splits, exposing petrified roots shaped into sigils.
\end{enumerate}

\subsubsection*{When Thorns Interrupt}
Roll d6:

\begin{enumerate}[leftmargin=*]
    \item All reflections in the scene crack.
    \item Shadows detach and linger unnaturally.
    \item A crusader's memory is stolen mid-sentence.
    \item A witch's cord-blackthread frays into ash.
    \item A masked figure appears behind someone, silently pointing.
    \item A root-whisper: "You do not deserve your name."
\end{enumerate}

%-----------------------------------------------------------
\subsection{5.7 Scaling the Siege}
%-----------------------------------------------------------}

\paragraph{Tier I–II:}  
Small clashes, localized witchcraft failures, skirmishes with zealots.

\paragraph{Tier III–IV:}  
Major miracles, storm rites, Thorn incursions, memory-loss episodes.

\paragraph{Tier V:}  
Aveh dissolution rapture, crusader purgation beams, Thorn dominions —  
the identity of Heugen itself may collapse.

\begin{center}
\emph{This is a war fought in light, shadow, and the space between names.}
\end{center}

%===========================================================
\section{Emissaries of the Lantern War}
%===========================================================

This section details the key operatives, champions, and incarnate voices 
who prosecute the Lantern War on behalf of Mykkiel, Aveh, the Witch Orders, 
and the Thorns of Malachai. While patrons lack agency and cannot manifest,
their emissaries act with conviction, arguing over doctrine, purity, and purpose.

Each statblock uses standard Fate's Edge formatting:
\begin{itemize}[leftmargin=*]
    \item Tier, Position, Qualities
    \item Core Moves
    \item Talents and Rites
    \item Special Mechanics (Corruption, Identity, Miracles, etc.)
\end{itemize}

%-----------------------------------------------------------
\subsection{6.1 Crusader Miracle-Captain of Mykkiel}
%-----------------------------------------------------------

\textbf{Tier III Adversary} \\
\textbf{Position:} Controlled, Direct, Unyielding \\
\textbf{Motifs:} Law, Purity, Enumerated Doctrine, Sacred Authority

\subsubsection*{Qualities}
\begin{itemize}[leftmargin=*]
    \item Lantern-Blessed Armor (2 Armor)
    \item Voice of Mandate (commands silence)
    \item Liturgical Precision (immune to deception when invoking miracle-lines)
\end{itemize}

\subsubsection*{Core Moves}
\textbf{Enumerate Sin:} Reveal a target's hidden shame. Target suffers $-1d$ 
on falsehood, illusion, or disguise actions this scene.

\textbf{Lantern Surge:} Suppress witchcraft within Near range. Witch rites require +1 DV.

\textbf{Condemnation Step:} When stepping forward in ritual stance, force a PC 
to roll Spirit (DV 3) or become Shaken.

\subsubsection*{Rites / Miracles}
\begin{itemize}[leftmargin=*]
    \item \textbf{Sevenfold Lantern} – Establishes a warded radius of divine order.
    \item \textbf{Perfect Judgment} – Bind target in mirrored flame. 
    \item \textbf{Mantle of Zeal} – Grants +1 Position and dismisses fear effects.
\end{itemize}

\subsubsection*{Special}
\textbf{Reprisal SB:} On 1s, doctrine fractures. A rival officer disputes the Captain's purity.

%-----------------------------------------------------------
\subsection{6.2 Aveh's Faceless Shepherd}
%-----------------------------------------------------------

\textbf{Tier III Adversary (or Ally)} \\
\textbf{Position:} Evasive, Dissolved, Unknowable \\
\textbf{Motifs:} Shedding, Masks, Lost Identity, Liberation-by-Absence

\subsubsection*{Qualities}
\begin{itemize}[leftmargin=*]
    \item Featureless Visage (cannot be remembered clearly)
    \item Quiet-Footed Pilgrim (always counts as Obfuscated)
    \item Weightless Step (no footprints, no sound)
\end{itemize}

\subsubsection*{Core Moves}
\textbf{Unmake the Mask:} Remove a PC's ability to use aliases or disguises this scene.

\textbf{Dissolution Touch:} Inflict 1 \textbf{Identity Strain}.  
If a PC is already strained, force a reroll at $-1d$.

\textbf{Vanishing Refrain:} Exit the scene without provoking reactions.

\subsubsection*{Shedding Rites}
\begin{itemize}[leftmargin=*]
    \item \textbf{Path of the Unwoven Mask} – Suppress identity markers.
    \item \textbf{Faceless Dawn} – Become untrackable for one scene.
\end{itemize}

\subsubsection*{Special}
If a PC speaks their full name to the Shepherd, roll Spirit (DV 4) or forget a personal memory.

%-----------------------------------------------------------
\subsection{6.3 Rainmaiden Oracle of Raéyn}
%-----------------------------------------------------------

\textbf{Tier IV Adversary/Ally} \\
\textbf{Position:} Fluid, Tempestuous, Triumphant \\
\textbf{Motifs:} Water, Storms, Temper, Power, Emotional Tide

\subsubsection*{Qualities}
\begin{itemize}[leftmargin=*]
    \item Tidal Insight (reads emotional currents)
    \item Storm-Crowned (immune to fear and coercion)
    \item Waterwoven Garb (counts as 1 Armor)
\end{itemize}

\subsubsection*{Core Moves}
\textbf{Storm Lash:} Strike all foes in Near range; apply $-1d$ to next action.

\textbf{Tidebreaker Command:} Allies gain +1 Position; enemies lose 1 Position.

\textbf{Rain-Soaked Mirrors:} Reveal a truth someone hid from themselves.

\subsubsection*{Rites}
\begin{itemize}[leftmargin=*]
    \item \textbf{Mist-Calling Whisper}
    \item \textbf{Waveborne Rebuttal}
    \item \textbf{Tempest Queen's Claim} (High)
\end{itemize}

%-----------------------------------------------------------
\subsection{6.4 Thorn Stalker of Malachai}
%-----------------------------------------------------------

\textbf{Tier II–III Adversary} \\
\textbf{Position:} Hidden, Silent, Invasive \\
\textbf{Motifs:} Extraction, Memory Theft, False Witness, Identity Hunger

\subsubsection*{Qualities}
\begin{itemize}[leftmargin=*]
    \item Root-Clad Silhouette
    \item No Voice, Only Gesture
    \item Cannot be perceived directly for more than a moment
\end{itemize}

\subsubsection*{Core Moves}
\textbf{Shadow Extraction:} Mark 1 \textbf{Identity Strain} and remove a sensory memory.

\textbf{Mirrortwitch Advance:} Teleport between shadows; always gains first move.

\textbf{False Echo:} Speak in a PC's voice, describing actions they never took.

\subsubsection*{Special: Identity Collapse}
When PCs are at 5+ Strain, Thorn Stalkers roll with +1d and inflict  
\textbf{Exposed}: Position becomes Desperate when confronting memory-based foes.

%-----------------------------------------------------------
\subsection{6.5 The Wells-Worm}
%-----------------------------------------------------------

\textbf{Tier IV Monster (Aberration)} \\
\textbf{Position:} Titanic, Burrowing, Psychic \\
\textbf{Motifs:} Wells, Memory, Shame, Subterranean Hunger

\subsubsection*{Qualities}
\begin{itemize}[leftmargin=*]
    \item Memory-Drinking Maw
    \item Reverberant Cry (echoes of forgotten sins)
    \item Segmented Stone-Carapace (2 Armor)
    \item Burrow and Erupt (scene-scale reposition)
\end{itemize}

\subsubsection*{Core Moves}
\textbf{Shame Resonance:} PCs roll Spirit (DV 4) or suffer +1 Identity Strain.

\textbf{Spiral Into the Deep:} Drag a PC underground; separate them from the party.

\textbf{Echo-Consumption:} Feed on an NPC's memory; that NPC forgets a relationship or recent event.

\subsubsection*{Special: Wells Conduit}
If a miracle or rite is cast near a city well, the Worm responds:
\begin{itemize}[leftmargin=*]
    \item Erupts in a random district
    \item Unleashes psychic shockwaves
    \item Causes the city to forget something small but essential
\end{itemize}

%-----------------------------------------------------------
\subsection{6.6 Named Witch Emissary: Handmaiden of Livaea}
%-----------------------------------------------------------

\textbf{Tier III Adversary/Ally} \\
\textbf{Position:} Seductive, Elegant, Insightful \\
\textbf{Motifs:} Soft Power, Charm, Mirror-Craft, Emotional Leverage

\subsubsection*{Qualities}
\begin{itemize}[leftmargin=*]
    \item Velvet-Soft Voice (disarms hostility)
    \item Mirage Skin (subtle glamour shaping)
    \item Courtly Poise (+1 Position in negotiation)
\end{itemize}

\subsubsection*{Core Moves}
\textbf{Mirror's Truth:} Reveal one thing the target wishes they did \emph{not} desire.

\textbf{Velvet Hex:} Reduce target's Position by 1 through emotional entanglement.

\textbf{Thread of Longing:} Bind two NPCs in a temporary covetous obsession.

\subsubsection*{Rites}
\begin{itemize}
    \item \textbf{Kiss of the Velvet Queen} (softens resolve)
    \item \textbf{Silken Command} (coerces a single action)
\end{itemize}

%-----------------------------------------------------------
\subsection{6.7 Named Witch Emissary: Ghe'hai Shadowbinder of Ikasha}
%-----------------------------------------------------------

\textbf{Tier III Elite} \\
\textbf{Position:} Formal, Poised, Lethal \\
\textbf{Motifs:} Diplomacy, Secret Warfare, Oaths of Silence, Knife-Rites

\subsubsection*{Qualities}
\begin{itemize}[leftmargin=*]
    \item Bone-Quiet Step
    \item Shadow-Drafted Blades (magical harm)
    \item Immaculate Etiquette (immune to social pressure surprises)
\end{itemize}

\subsubsection*{Core Moves}
\textbf{Shadow Clause:} Insert a silent, binding "agreement" into a social exchange.

\textbf{Nightbone Cut:} Harm that bypasses magical wards.

\textbf{Diplomat's Mask:} Feign civility while preparing for precision violence.

\subsubsection*{Rites}
\begin{itemize}
    \item \textbf{Mask-Severing Gesture}
    \item \textbf{Quietus of the Hidden Blade}
\end{itemize}

%===========================================================
\section{Factions, Locations, and Clocks}
%===========================================================

Heugen—the City of Forgetting—stands at a crossroads of identity, doctrine,
and shadowed purpose. Its wells draw memory downward; its people live free
of shame; and its patrons' emissaries maneuver unseen. The following factions
and locations shape the conflict of the Lantern War.

%-----------------------------------------------------------
\subsection{7.1 Factions in the City of Forgetting}
%-----------------------------------------------------------

\subsubsection*{The Adherents of Mykkiel (External Siege Faction)}
\index{Factions!Adherents of Mykkiel}

A vast procession of lantern-bearing crusaders approaches the city,
determined to excise its ``blasphemous forgettings.''  
Among them:
\begin{itemize}[leftmargin=*]
    \item Miracle-Captains delivering enumerated doctrine
    \item Lantern Scribes cataloging memory leaks
    \item Choir-Knights who silence witchcraft on sight
\end{itemize}

\textbf{Goal:} Restore shame and identity as sacred burdens.  
\textbf{Methods:} Siege, miracle-warfare, public trials, burning archives.  
\textbf{Internal Tension:}  
Strict Mandate vs.\ more compassionate reformists (dangerous schism potential).

\paragraph{Foreshadowing Clues}
\begin{itemize}[leftmargin=*]
    \item Mirrored lantern-light appears in distant dunes
    \item Locals dream of a stern voice enumerating their forgotten sins
    \item Water in eastern wells ripples to unheard hymns
\end{itemize}

%-----------------------------------------------------------
\subsubsection*{The Followers of Aveh (Internal — Most Popular Patron)}
\index{Factions!Followers of Aveh}

A pluralistic, liminal faith without a face. Aveh's followers shed identity,
names, and gender until they achieve a ``soft empty truth.'' Most are harmless
misfits; a few are zealots of erasure.

\textbf{Goal:} Maintain Heugen as a haven of shed identities.  
\textbf{Methods:} Persuasion, community-care, quiet dissolution of conflict.  
\textbf{Schism:}  
\begin{itemize}
    \item \textbf{Shepherds of the Mask} — gentle guides who help people shed safely  
    \item \textbf{The Unwoven Circle} — extremists who force shedding upon others
\end{itemize}

\paragraph{Foreshadowing Clues}
\begin{itemize}[leftmargin=*]
    \item A masked pilgrim appears in multiple districts—yet has no footprints  
    \item A child forgets their surname after drinking from the central well  
    \item A silhouette with no face reflected in a shop window
\end{itemize}

%-----------------------------------------------------------
\subsubsection*{The Witch Orders (Internal Covens)}
\index{Factions!Witch Orders}

Witches are not unified; the expansion's Orders each have distinct tensions 
with both the crusaders and each other.

\begin{itemize}[leftmargin=*]
    \item \textbf{Handmaidens of Livaea} — soft-power seductresses who guide Heugen's elite
    \item \textbf{Rainmaidens of Raéyn} — tempestuous oracles who fear the siege will disrupt the tides of fate
    \item \textbf{Hearth Witches} — quiet miracle-workers who know the wells are ``not right''
    \item \textbf{Ikasha Shadowbinders} — Ghe'hai-trained elven operatives hoarding city secrets
\end{itemize}

\textbf{Goal:} Preserve their order's interests, not necessarily the city's.  
\textbf{Methods:} Hexcraft, bargains, prophecy, emotional manipulation.

\paragraph{Foreshadowing Clues}
\begin{itemize}[leftmargin=*]
    \item A wardstone cracks as unseen doctrine presses against it  
    \item Wells whisper conflicting future omens  
    \item A rainstorm forms indoors, circling a single witch's rival
\end{itemize}

%-----------------------------------------------------------
\subsubsection*{The Thorns of Malachai (Hidden Third Party)}
\index{Factions!Thorns of Malachai}

A secret extraction-cult operating beneath Heugen. They worship not a patron 
but a \emph{principle}: identity harvested, memories hollowed, shadows rewritten.
They wish to deepen the city's forgetting until Heugen becomes an empty husk.

\textbf{Goal:} Collapse all identity threads in the city and ``prepare the vessel.''  
\textbf{Methods:} Shadow extraction, false-echo propaganda, stealth abductions.  
\textbf{Foothold:} The old salt tunnels under District Nine.

\paragraph{Foreshadowing Clues}
\begin{itemize}[leftmargin=*]
    \item A PC's shadow lags noticeably behind them  
    \item Two NPCs recount different memories of the same event  
    \item A discarded mask shaped like a spiral root
\end{itemize}

%-----------------------------------------------------------
\subsection{7.2 Key Locations of Heugen}
%-----------------------------------------------------------

\subsubsection*{The Veiled Wells}
\index{Locations!Veiled Wells}

The heart of Heugen's forgetting. The waters soothe shame but feed the ancient
Wells-Worm. Each well is a ritual site, a cultural sanctuary, and a lurking threat.

\paragraph{Qualities}
\begin{itemize}[leftmargin=*]
    \item Memory-leeching resonance
    \item Deep conduits to unseen caverns
    \item Attract both Aveh pilgrims and Thorn Stalkers
\end{itemize}

\paragraph{Encounter Hooks}
\begin{itemize}[leftmargin=*]
    \item A child drinks from the well and forgets a parent  
    \item Crusaders sanctify a well, causing violent psychic backlash  
    \item Witch wards begin to melt into liquid silver
\end{itemize}

%-----------------------------------------------------------
\subsubsection*{The Lantern-Siege Camp (Outer Desert)}
\index{Locations!Lantern-Siege Camp}

An ordered, geometric encampment of crusaders.  
Everything is arranged in perfect arrays: tents, lantern-poles, prayer circles.

\paragraph{Features}
\begin{itemize}[leftmargin=*]
    \item Hymns echo across miles at dawn  
    \item Captured "blasphemers" undergo purification rites  
    \item Rival crusader officers plot doctrinal ascendancy
\end{itemize}

%-----------------------------------------------------------
\subsubsection*{The Mask Quarter}
\index{Locations!Mask Quarter}

Aveh's district. Hundreds of partial identities drift through its alleys.
Merchants offer newly shed names for trade. Masks are hung like lanterns.

\paragraph{Rumors}
\begin{itemize}[leftmargin=*]
    \item A masked woman claims she remembers the city's founding "perfectly"  
    \item Someone is collecting discarded faces for unknown purpose  
    \item The Faceless Shepherd passes through silently each night
\end{itemize}

%-----------------------------------------------------------

\subsubsection*{The Thorned Expanse (Undercity)}
\index{Locations!Thorned Expanse}

Root-choked tunnels holding ancient sigils.  
Here, the Thorns of Malachai weave false echoes and cultivate harvested shadows.

\paragraph{Encounters}
\begin{itemize}[leftmargin=*]
    \item Memory husks (Tier I–II)  
    \item Thorn Stalkers (Tier II–III)  
    \item A well-shaft leading to the Wells-Worm’s sleeping chamber
\end{itemize}

%-----------------------------------------------------------
\subsection{7.3 City Clocks and Siege Pressure}
%-----------------------------------------------------------}

The conflict unfolds against three major Clocks tracking faction momentum.

\subsubsection*{The Siege Clock [8]}
\begin{itemize}[leftmargin=*]
    \item 2: Crusader scouts infiltrate the market  
    \item 4: Siege engines arrive  
    \item 6: The first breach in the outer wall  
    \item 8: Full-scale assault begins
\end{itemize}

\subsubsection*{The Forgetting Clock [8]}
\begin{itemize}[leftmargin=*]
    \item 2: Minor memories vanish citywide  
    \item 4: Entire households forget relationships  
    \item 6: A district collapses into identity confusion  
    \item 8: Heugen becomes the ``Hollow City''
\end{itemize}

\subsubsection*{The Thorn Ascendance Clock [6]}
\begin{itemize}[leftmargin=*]
    \item 2: Shadow extractions increase  
    \item 4: Thorn Stalkers take an important NPC  
    \item 6: Malachai’s Principle manifests—a false consciousness woven across the city
\end{itemize}

%===========================================================
\section{Adventure Framework: The City of Forgetting}
%===========================================================

This campaign presents a layered conflict in Heugen, where identity,
memory, faith, and shadow converge. The players navigate three rising
pressures: an external siege by the Adherents of Mykkiel, internal
fractures among Aveh’s followers and the witch orders, and the hidden,
escalating threat of the Thorns of Malachai.

Each act advances the three major Clocks:
\begin{itemize}[leftmargin=*]
    \item \textbf{Siege Clock [8]} – the crusaders’ advance
    \item \textbf{Forgetting Clock [8]} – the deepening effects of the wells
    \item \textbf{Thorn Ascendance Clock [6]} – shadow extractions and false echoes
\end{itemize}

%-----------------------------------------------------------
\subsection{Act I: The City That Forgets}
%-----------------------------------------------------------

Heugen appears calm on the surface: markets bustle, pilgrims trade masks,
witches maintain uneasy truces, and the wells whisper their soft erasures.
But beneath this tranquility, fractures are widening.

This act introduces the PCs to the city’s factions, identities, and hidden
tensions while foreshadowing the threat marching across the desert.

\subsubsection*{Act I Goals}
\begin{itemize}[leftmargin=*]
    \item Familiarize players with the politics and culture of Heugen
    \item Reveal early symptoms of the Forgetting Clock
    \item Present initial moral conflicts around Aveh’s ideals
    \item Introduce witch orders and their rivalries
    \item Seed rumors of the approaching crusaders
    \item Foreshadow the Thorns of Malachai through subtle distortions
\end{itemize}

%-----------------------------------------------------------
\subsubsection*{Scene I-A: The Mask Market}
%-----------------------------------------------------------

The PCs enter the Mask Quarter during the Festival of First Shedding.
Performers parade through the streets wearing half-faces of clay; masked
orphans reenact the founding of the city.

\paragraph{Encounters}
\begin{itemize}[leftmargin=*]
    \item A masked Aveh Shepherd offers to ``unburden'' a PC’s shame.
    \item A child suddenly forgets their surname after drinking well-water.
    \item A witch of Livaea courts a noble’s attention—seeking leverage.
\end{itemize}

\paragraph{Foreshadowing the Thorns}
A PC glimpses their own shadow pausing a moment too long, as if listening.

\vspace{1em}

%-----------------------------------------------------------
\subsubsection*{Scene I-B: Wells of Quiet Memory}
%-----------------------------------------------------------

The PCs witness the veiled wells performing their subtle magic.
A small crowd gathers near the First Well where a pilgrim sobs with relief:
``I no longer remember what I fled.''  

But something is wrong.

\paragraph{Complications}
\begin{itemize}[leftmargin=*]
    \item A Hearth Witch urgently requests help reinforcing a cracked ward.
    \item A Mykkiel convert denounces the wells, stirring panic.
    \item A Rainmaiden senses ``the worm turning in its sleep.''
\end{itemize}

\paragraph{Foreshadowing the Thorns}
A market guard swears he saw someone hauled into a dry well—yet the well is intact.

\vspace{1em}

%-----------------------------------------------------------
\subsubsection*{Scene I-C: Rumors from the Desert}
%-----------------------------------------------------------}

Lantern-light has been spotted on the dunes.  
Whether from merchants, scouts, or zealots—opinions vary.

\paragraph{NPC Claims}
\begin{itemize}[leftmargin=*]
    \item ``The crusaders are three days away.''  
    \item ``They carry lanterns that burn memory itself.''  
    \item ``A delegation wants peace—but the Choir-Knights do not.''  
\end{itemize}

\paragraph{Mechanical Trigger}
At the end of this scene, the \textbf{Siege Clock} advances by 1.

\vspace{1em}

%-----------------------------------------------------------
\subsubsection*{Scene I-D: Covens at Crossroads}
%-----------------------------------------------------------

The PCs are approached by one (or several) witch orders:
\begin{itemize}
    \item Livaean Handmaidens
    \item Raéyn Rainmaidens
    \item Ikasha Shadowbinders
    \item Hearth Witches
\end{itemize}

Each wants something from the party, yet none are transparent.

\paragraph{Witch Demands (choose one per table need)}
\begin{itemize}[leftmargin=*]
    \item Investigate a rival’s ward-sabotage
    \item Retrieve a name-fragment from a well
    \item Escort a witch emissary through hostile districts
    \item Discreetly disrupt an Aveh ceremony
\end{itemize}

\paragraph{Foreshadowing the Thorns}
A witch warns:  
\emph{``Something beneath the wells feeds on more than memory.''}

\vspace{1em}

%-----------------------------------------------------------
\subsubsection*{Scene I-E: False Echoes}
%-----------------------------------------------------------}

The first obvious Thorn manifestation occurs:
two identical townsfolk arguing over which one is the ``real'' one.
Both possess partial memories that contradict local history.

\paragraph{Investigation Clues}
\begin{itemize}[leftmargin=*]
    \item A cord-mark on the back of each duplicate’s neck  
    \item Shadows in the alley move independently of their owners  
    \item A root-like spiral painted under a discarded mask  
\end{itemize}

\paragraph{Mechanical Trigger}
If the PCs fail to intervene or identify the distortion,  
the \textbf{Thorn Ascendance Clock} advances by 1.

\vspace{1em}

%-----------------------------------------------------------
\subsubsection*{Act I Outcome}
%-----------------------------------------------------------}

At the end of Act I, the PCs should:
\begin{itemize}[leftmargin=*]
    \item Trust some factions and distrust others
    \item Understand the core pressures on the city
    \item Feel moral ambiguity around Aveh’s shedding doctrines
    \item Sense the presence of a hidden third party (Thorns)
    \item Have at least one witch coven entangled in their affairs
    \item Know the crusaders are en route—perhaps faster than expected
\end{itemize}

Advance:
\begin{itemize}
    \item \textbf{Siege Clock} to 2  
    \item \textbf{Forgetting Clock} to 1  
    \item \textbf{Thorn Ascendance Clock} to 1--2 depending on PC actions
\end{itemize}

Act II begins when the first emissaries of Mykkiel reach the gates.

%===========================================================
\subsection{Act II: Emissaries at the Gate}
%===========================================================

The desert wind carries hymns of judgment.  
Lanterns burn on the horizon.  
Heugen’s uneasy peace fractures as two emissary groups of Mykkiel arrive,
each claiming the right to negotiate—and to pronounce sentence.

Yet these delegates are not unified. Their disagreements mirror the city's
own fractures, and the Thorns of Malachai exploit this tension with subtle,
escalating distortions.

%-----------------------------------------------------------
\subsubsection*{Act II Goals}
%-----------------------------------------------------------
\begin{itemize}[leftmargin=*]
    \item Introduce the doctrinal split within Mykkiel’s crusaders
    \item Force PCs to navigate competing visions of “justice”
    \item Deepen conflicts among the witch covens
    \item Present Aveh’s shepherds as both vulnerable and manipulative
    \item Allow the Thorns to strike from beneath the surface
    \item Push the Siege Clock significantly forward
\end{itemize}

%-----------------------------------------------------------
\subsection*{Emissary Factions}
%-----------------------------------------------------------}

Two factions arrive—together, but not united.

\paragraph{1. Choir-Knights of the Shining Mandate}
Brutal, disciplined, unyielding.  
They believe Heugen must be purified by flame, not negotiation.  
Their “mercy” is annihilation that leaves no sin to fester.

\textbf{Signs:} bronze masks, radiant lanterns, immaculate mail.  
\textbf{Demands:} full surrender of the city within three days.  
\textbf{Moves:} pronounce \emph{Inviolate Sin} (freeze an NPC with fear),  
escalate zealotry, invoke a punitive miracle.

\paragraph{2. Lantern Advocates of Mykkiel}
Scholars, reformers, and theologians who believe the city can be
redeemed—if it abandons the Wells of Forgetting and Aveh's rites.

\textbf{Signs:} inked scriptures, ivory lamps, dispute talismans.  
\textbf{Demands:} tear down the outer wells; open the gates to missionary courts.  
\textbf{Moves:} identify false doctrine, calm crowds, isolate individuals for confession.

\paragraph{Internal Tension}
Treat these emissaries as a single faction with a hair-trigger schism.
If the PCs inflame or mishandle negotiations, begin a \textbf{Schism Clock [4]}.

At [4], the factions openly break, creating chaos the Thorns can exploit.

%-----------------------------------------------------------
\subsection*{Scene II-A: The Parley of Masks}
%-----------------------------------------------------------}

The emissaries are brought to the Mask Quarter.  
According to tradition, all parties wear neutral half-masks so expression
cannot be used as a weapon.

\paragraph{Potential Conflicts}
\begin{itemize}[leftmargin=*]
    \item A Mykkiel Advocate publicly challenges Aveh’s doctrine of shedding
    \item A Handmaiden of Livaea whispers a private scandal into a crusader’s ear
    \item A Rainmaiden lets the sky darken—not a threat, but a statement
\end{itemize}

\paragraph{Mechanical Tension}
If PCs allow the parley to collapse into shouting or miracle-work:
\begin{itemize}
    \item Increase \textbf{Siege Clock} by 1
    \item Increase \textbf{Forgetting Clock} by 1 (panic drives people to wells)
\end{itemize}

\paragraph{Thorn Foreshadowing}
A crusader's shadow moves independently, raising its lantern arm before
the real crusader does.

Crowd reacts with fear, not understanding why.

%-----------------------------------------------------------
\subsection*{Scene II-B: Courts of Doctrine}
%-----------------------------------------------------------}

The emissaries demand a doctrinal review of Aveh's rites.
Heugen reluctantly assembles a public forum.

\paragraph{Arguments from Mykkiel's Advocates}
\begin{itemize}[leftmargin=*]
    \item “Shedding the self is surrendering the burden Mykkiel teaches us to bear.”
    \item “Memory is law. A city without law dissolves into shadow.”
\end{itemize}

\paragraph{Arguments from Aveh’s Shepherds}
\begin{itemize}[leftmargin=*]
    \item “Shame binds the spirit; we free those the world abandons.”
    \item “Identity is not a punishment.”
\end{itemize}

\paragraph{PC Roles}
\begin{itemize}[leftmargin=*]
    \item Provide testimony  
    \item Expose contradictions  
    \item Protect witnesses  
    \item Sabotage or strengthen either side  
    \item Reveal witch influence (or conceal it)
\end{itemize}

\paragraph{Thorn Interference}
A witness begins reciting events that could not have occurred—  
memories extracted, rearranged, and returned with falsified details.

This is subtle, but unmistakably unnatural.

Investigating earns:
\begin{itemize}
    \item Clue: \emph{``Lantern-light does not cast these shadows.''}
    \item Advancement: \textbf{Thorn Ascendance Clock +1}
\end{itemize}

%-----------------------------------------------------------
\subsection*{Scene II-C: Witch Maneuvers}
%-----------------------------------------------------------}

The witch orders respond to the emissaries:

\paragraph{Livaea’s Handmaidens}
Stage a seduction-protest, charming a crusader into contradicting doctrine.

\paragraph{Rainmaidens of Raéyn}
Threaten a symbolic flood if the wells are touched.

\paragraph{Ikasha Shadowbinders}
Attempt blackmail using stolen confessions.

\paragraph{Hearth Witches}
Organize neighborhood protections—but fear forgotten names fouling their work.

\paragraph{PC Involvement}
Witches seek the PCs’ help with:
\begin{itemize}[leftmargin=*]
    \item smuggling someone out of crusader custody  
    \item obtaining a “true name” of an emissary  
    \item sealing a well to prevent manipulation  
    \item countering a magical intrusion not caused by witches at all  
\end{itemize}

This last category is always the Thorns.

%-----------------------------------------------------------
\subsection*{Scene II-D: False Miracles}
%-----------------------------------------------------------}

As tensions escalate, a miracle occurs in the Lower Quarter:  
a woman is surrounded by radiant light and speaks with two voices.

Mykkiel’s emissaries declare it proof of divine judgment.  
Aveh’s shepherds insist it is a vision of freedom.

But the PCs who investigate learn the truth:
\begin{itemize}[leftmargin=*]
    \item The ``miracle'' is an identity-extraction gone wrong.
    \item The second voice is her own—fragmented.
    \item A root-like spiral has grown under her bed.
\end{itemize}

If PCs fail to calm the crowd or halt crusader escalation:
\begin{itemize}
    \item Increase \textbf{Siege Clock} by 2
\end{itemize}

%-----------------------------------------------------------
\subsection*{Act II Outcome}
%-----------------------------------------------------------}

At the end of Act II, the PCs should:
\begin{itemize}[leftmargin=*]
    \item Understand the crusaders’ internal split
    \item See the witch orders pulling the city in multiple directions
    \item Recognize Aveh’s doctrine as both liberating and dangerous
    \item Suspect a manipulation beneath the surface (Thorns)
    \item Have made meaningful allies or enemies among the emissaries
\end{itemize}

Advance:
\begin{itemize}
    \item \textbf{Siege Clock} to 4--5  
    \item \textbf{Forgetting Clock} to 2  
    \item \textbf{Thorn Ascendance Clock} to 2--3  
\end{itemize}

Act III begins when the first crusader cohort arrives at the gates,
demanding the wells be sealed and the city submit to Mykkiel’s judgment.

%===========================================================
\subsection{Act III: The Siege of the Wells}
%===========================================================

The desert trembles as the first cohort of Mykkiel’s crusaders arrives.
Drums echo against the stone walls of Heugen; dust rises like smoke.
Inside the city, covens splinter, panic spreads, and the Wells deepen
their pull on the fearful.

The Thorns of Malachai strike from beneath, turning the siege into
a three-sided war of doctrine, memory, and identity.

%-----------------------------------------------------------
\subsubsection*{Act III Goals}
%-----------------------------------------------------------
\begin{itemize}[leftmargin=*]
    \item Bring the city under direct external siege pressure
    \item Force difficult choices about the Wells of Forgetting
    \item Reveal the Thorns as an active threat (but still unexplained)
    \item Escalate conflicts between witch orders
    \item Reduce safe ground in the city
    \item Move the climax toward internal collapse rather than military defeat
\end{itemize}

%-----------------------------------------------------------
\subsection*{City State at the Start of Act III}
%-----------------------------------------------------------}

\begin{itemize}[leftmargin=*]
    \item \textbf{Siege Clock:} 4--5 (outer walls tested)
    \item \textbf{Forgetting Clock:} 2 (citizens retreat to the Wells)
    \item \textbf{Thorn Ascendance:} 2--3 (disruptions and disappearances)
\end{itemize}

\noindent
Heugen is not yet falling—but its internal tensions may break it faster
than any crusader ram.

%-----------------------------------------------------------
\subsection*{Scene III-A: The Demands at the Gate}
%-----------------------------------------------------------}

The crusaders deploy siege tents and raise three banners:
\begin{itemize}
    \item \textbf{White}: Accept reformation under Mykkiel’s law
    \item \textbf{Red}: Submit all witches for judgment
    \item \textbf{Black}: Destroy the Wells entirely
\end{itemize}

\paragraph{PC Options}
\begin{itemize}[leftmargin=*]
    \item Negotiate for time (DV 3--5)
    \item Undermine crusader morale
    \item Sabotage siege preparations
    \item Deliver false intelligence to either faction
\end{itemize}

\paragraph{Schism Possibility}
If the Choir-Knights and Lantern Advocates are already strained,
this scene may split them outright.  
Trigger the \textbf{Schism Clock} if not already started.

\paragraph{Thorn Foreshadowing}
A crusader’s war-horn produces a doubled echo—  
one human, one distinctly not.

%-----------------------------------------------------------
\subsection*{Scene III-B: Wells in Turmoil}
%-----------------------------------------------------------}

Crowds gather at the Wells of Forgetting, seeking:
\begin{itemize}[leftmargin=*]
    \item relief from siege panic,
    \item escape from shame,
    \item or the promise of a new identity.
\end{itemize}

But the Wells begin behaving strangely.  
Voices speak beneath the water—in mismatched memories.

\paragraph{Mechanical Impact}
\begin{itemize}
    \item Each day of siege automatically increases the \textbf{Forgetting Clock}.
    \item Witch covens argue violently over whether to seal or amplify the wells.
\end{itemize}

\paragraph{Thorn Interference}
Any ritual performed near a well risks triggering:
\begin{itemize}
    \item identity dislocation (PC forgets a relationship),
    \item double-echo voice hallucinations,
    \item false memories of crusader cruelty (never happened),
    \item shadows moving independently.
\end{itemize}

A PC may realize:  
\emph{``This is not the Wells' natural magic.''}

%-----------------------------------------------------------
\subsection*{Scene III-C: Witch War in the Streets}
%-----------------------------------------------------------}

The witch orders fracture as siege stress mounts.

\paragraph{Livaea’s Handmaidens}
Charm crusader scouts, partly to help Heugen—partly out of spite.

\paragraph{Rainmaidens of Raéyn}
Attempt a storm-ward to drown siege engines.

\paragraph{Ikasha Shadowbinders}
Begin taking hostages for leverage (PCs may be asked to intervene).

\paragraph{Hearth Witches}
Try to create safe communal circles, but Thorn interference corrupts
names and labels—rendering their work fragile.

\paragraph{PC Goals}
\begin{itemize}[leftmargin=*]
    \item Prevent inter-coven violence
    \item Protect innocents from magical fallout
    \item Choose which covens to support
    \item Identify which covens the Thorns are impersonating
\end{itemize}

%-----------------------------------------------------------
\subsection*{Scene III-D: Breach of Identity}
%-----------------------------------------------------------}

A prominent figure (choose someone meaningful to the PCs):
\begin{itemize}
    \item a witch matron,
    \item a crusader emissary,
    \item a beloved community elder,
    \item or a PC’s ally,
\end{itemize}

is found wandering the streets, speaking in a voice that is not theirs.

Their shadow is missing.  
Their memories are contradictory.  
Their Wells-mark is inverted.

\paragraph{Diagnosis}
A Thorn extraction was attempted—but not completed.

\paragraph{PC Actions}
\begin{itemize}
    \item Restore the victim’s true name (DV 4--5)
    \item Track the missing shadow (requires specialized rites)
    \item Confront a Thorn agent watching from the rooftops
\end{itemize}

This is the first time a Thorn presence becomes undeniably real.

%-----------------------------------------------------------
\subsection*{Scene III-E: The First Assault}
%-----------------------------------------------------------}

The Choir-Knights launch a probing attack on the eastern gate.
This is not yet the final siege—but it reveals weaknesses.

\paragraph{PC Opportunities}
\begin{itemize}[leftmargin=*]
    \item Defend the walls with witches and townsfolk
    \item Counter crusader miracles
    \item Smuggle refugees to the inner districts
    \item Discover evidence of sabotage in the defenses
\end{itemize}

\paragraph{Thorn Twist}
PCs may catch a Thorn saboteur:
\begin{itemize}
    \item wearing a crusader’s face,
    \item mimicking a PC’s voice,
    \item or carrying a stolen name-thread.
\end{itemize}

If captured, the Thorn collapses into a tangle of roots and whispering mouths.

\paragraph{Outcome}
Advance:
\begin{itemize}
    \item \textbf{Siege Clock +1}
    \item \textbf{Forgetting Clock +1}
    \item \textbf{Thorn Ascendance +1}
\end{itemize}

Heugen survives the first blow—but the city is unraveling from within.

%-----------------------------------------------------------
\subsection*{Act III Outcome}
%-----------------------------------------------------------}

By the end of Act III, the PCs should:
\begin{itemize}
    \item Know the siege is inevitable
    \item Recognize the Thorns as a third, destabilizing enemy
    \item See the witch orders divided and vulnerable
    \item Face moral decisions about the Wells and the city’s identity
    \item Have new allies—and new enemies—among Mykkiel’s emissaries
\end{itemize}

Act IV begins when:
\begin{itemize}
    \item The Thorns make their move beneath the city, \emph{or}
    \item The crusaders prepare their final assault, \emph{or}
    \item The PCs attempt a desperate ritual to stabilize the Wells
\end{itemize}

%===========================================================
\subsection{Act IV: Beneath the Wells — The Root Below the City}
%===========================================================

The siege above becomes distant thunder as the PCs descend into the
stonework beneath Heugen. Here lie the first foundation-wells dug by the
crusaders—long before the Wells of Forgetting took shape. Their roots
reached deeper than intended, brushing something ancient beneath the
desert.

Now that thing stirs.

The Thorns move openly: stealing shadows, severing memory threads, and
preparing a harvest beneath the city while crusaders hammer at the gates.
The PCs must choose which truths to preserve, which identities to release,
and whether Heugen deserves to be remembered at all.

%-----------------------------------------------------------
\subsubsection*{Act IV Goals}
%-----------------------------------------------------------
\begin{itemize}[leftmargin=*]
    \item Reveal the Wells’ true origin and the Thorns’ agenda
    \item Allow PCs to confront or bargain with the Wells-Worm
    \item Force choices about identity, exile, and forgetting
    \item Set the path toward Act V: the Final Judgment
    \item Bring Aveh’s and Mykkiel’s emissaries into direct conflict underground
\end{itemize}

%-----------------------------------------------------------
\subsection*{Scene IV-A: Descent Into the First Wells}
%-----------------------------------------------------------}

Entrances into the undercity are failing—wooden supports snapping,
stonework shedding memories, tunnels rearranging themselves as the
Forgetting Clock rises.

\paragraph{Entry Challenges (Roll or Choose)}
\begin{itemize}[leftmargin=*]
    \item \textbf{1. The Echo-Stair} — Steps repeat themselves; party must resist looping memories.
    \item \textbf{2. Shadow Divide} — PCs momentarily lose their shadows; Thorns mark their “roots”.
    \item \textbf{3. Flood of Faces} — Wells-water pours from a corridor, whispering false pasts.
    \item \textbf{4. Crusader Breach} — A small zealot detachment enters via a drainage tunnel.
\end{itemize}

\paragraph{Foreshadowing the Root}
Symbols carved by the original crusader engineers appear:
\begin{quote}
\emph{“Dig not beneath the fourth echo.”}  
\emph{“Identity must remain single.”}
\end{quote}

The crusaders clearly knew something was wrong—yet built anyway.

%-----------------------------------------------------------
\subsection*{Scene IV-B: The Name-Catacombs}
%-----------------------------------------------------------}

This chamber once held baptismal scrolls and redemption ledgers.
Now, memory-threads drift through the air like spider silk.

A Thorn ritual is in progress.

\paragraph{What the PCs See}
\begin{itemize}[leftmargin=*]
    \item Crusaders suspended by memory threads, whispering two names at once.
    \item Witches from Ikasha’s cult arguing—some assisting the Thorns unknowingly.
    \item A Rainmaiden desperately trying to stabilize identity-wards.
\end{itemize}

\paragraph{PC Actions}
\begin{itemize}
    \item Cut the memory threads (dangerous: the victim’s name may unravel)
    \item Bargain with a Thorn (they crave unused identities)
    \item Protect innocents from collapsing name-wards
    \item Seize Thorn ritual tools for later acts
\end{itemize}

\paragraph{Thorn Revelation}
The Thorns are not collecting memories;  
\emph{they are collecting what people choose to forget.}

The city is feeding them.

%-----------------------------------------------------------
\subsection*{Scene IV-C: The Shattered Reservoir}
%-----------------------------------------------------------}

This vast cavern contains the broken stone cisterns that once held pure
water. Now they hold Wells-water—heavy, luminous, silver-blue.

The water shows:
\begin{itemize}
    \item reflections of people who are not present,
    \item past selves of the PCs,
    \item future selves that could exist if they forget enough,
    \item crusaders kneeling to Aveh,
    \item witches turning zealot.
\end{itemize}

\paragraph{Mechanical Impact}
Any PC who gazes too long must test \textbf{Spirit + Resolve (DV 4)} or:
\begin{itemize}
    \item lose a label for the scene (“I am... uncertain”), or
    \item forget something important until Act V, or
    \item gain a temporary false memory.
\end{itemize}

\paragraph{Thorn Expansion}
The Thorns use this chamber as a birthing ground. New Thorn-bodies
crawl from the Wells-water like roots seeking air.

PCs who touch the water leave a part of themselves behind.

%-----------------------------------------------------------
\subsection*{Scene IV-D: The Root Below the City}
%-----------------------------------------------------------}

Beneath the shattered reservoir lies a colossal hollow where the Wells’
roots meet something alive.

The PCs witness:
\begin{itemize}[leftmargin=*]
    \item a mass of interwoven, shifting identity-threads;
    \item stone worn smooth by centuries of memory extraction;
    \item the Wells-Worm’s passages spiraling like sigils.
\end{itemize}

This is where the Thorns feed their harvest.

\paragraph{Who Else Arrives}
\begin{itemize}
    \item \textbf{Aveh’s emissaries}: desperate to stop crusaders from erasing the city
    \item \textbf{Mykkiel’s emissaries}: determined to sanctify or destroy the Wells
    \item \textbf{Thorn operatives}: intent on completing the harvest
\end{itemize}

Three doctrines collide:
\begin{itemize}
    \item \emph{freedom through forgetting,}
    \item \emph{purity through judgment,}
    \item \emph{power through discarded identity.}
\end{itemize}

The PCs choose which doctrine—if any—to empower.

%-----------------------------------------------------------
\subsection*{Scene IV-E: The Wells-Worm Stirs}
%-----------------------------------------------------------}

The Wells-Worm becomes aware of the PCs.

It speaks through:
\begin{itemize}
    \item the ripples of the Wells-water,
    \item stolen shadows,
    \item discarded memories,
    \item Thorn vessels whose mouths stretch too wide.
\end{itemize}

What it wants:
\begin{quote}
\emph{“Give me what you no longer wish to be.”}
\end{quote}

\paragraph{PC Options}
\begin{itemize}[leftmargin=*]
    \item Bargain — Offer up shame, guilt, trauma… or relationships.
    \item Fight — A near-suicidal prelude to Act V’s confrontation.
    \item Seal — Try to bind the Wells-Worm with name-wards (DV 5–6).
    \item Reveal — Show crusaders the Wells-Worm’s true nature, fracturing their zeal.
\end{itemize}

\paragraph{Thorn Betrayal}
If the PCs negotiate with the Worm, the Thorns attempt to:
\begin{itemize}
    \item seize the identity the PCs offer,
    \item overwhelm the emissaries,
    \item or feed the PCs to the Wells-Worm itself.
\end{itemize}

%-----------------------------------------------------------
\subsection*{Act IV Outcome}
%-----------------------------------------------------------}

By the end of Act IV, the PCs should:
\begin{itemize}[leftmargin=*]
    \item Know the Wells-Worm is central to both Aveh's and Mykkiel’s crusade
    \item Understand the Thorns’ true motive: harvesting discarded selves
    \item Realize the siege is a symptom—not the core threat
    \item Face an impossible choice involving identity, shame, or memory
    \item Set the stage for the final confrontation of Act V
\end{itemize}

\paragraph{Triggers to Begin Act V}
\begin{itemize}
    \item The Wells-Worm fully awakens
    \item The Thorns attempt a mass harvest
    \item A doctrinal schism ignites violence between emissaries
    \item PCs choose a ritual or confrontation path
\end{itemize}

Act V is the \emph{Judgment of Heugen}—when memory, identity, and belief collapse into a single defining moment.

%===========================================================
\subsection{Act V: The Judgment of Heugen}
%===========================================================

The siege above has reached its climax. Mykkiel’s host prepares their
final advance. Within the depths, the Wells-Worm coils, half-awake,
sensing the shape of its future. Aveh’s emissaries plead for mercy,
Mykkiel’s call for purging light, and the Thorns whisper for the
total harvest.

The PCs stand at the crossroads of identity and annihilation.

This act determines:
\begin{itemize}[leftmargin=*]
    \item whether Heugen survives,
    \item whether the Wells-Worm ascends, sleeps, or dies,
    \item whether Aveh or Mykkiel gain dominion,
    \item and what the PCs choose to remember—or forget—about themselves.
\end{itemize}

%-----------------------------------------------------------
\subsection*{Act V Structure}
%-----------------------------------------------------------
\begin{enumerate}[leftmargin=*]
    \item \textbf{The Wells-Worm awakens}
    \item \textbf{The Doctrine War ignites}
    \item \textbf{Ritual Paths: Ascension, Judgment, or Severance}
    \item \textbf{Confrontation: The Harvesting of Selves}
    \item \textbf{Final Choice and Aftermath}
\end{enumerate}

%-----------------------------------------------------------
\subsection*{Scene V-A: The Wells-Worm Fully Wakes}
%-----------------------------------------------------------

The cavern shakes. Identity threads snap and reknit. Wells-water rises.

The Wells-Worm manifests:
\begin{itemize}[leftmargin=*]
    \item as a colossal silhouette made of discarded selves,
    \item as shifting outlines of the PCs (past and possible),
    \item as a whispering tide of “you could have been.”
\end{itemize}

Its agenda becomes clear:
\begin{quote}
\emph{“I am what you shed. I become what the world refuses to hold.
Give me more, and I shall become your refuge—or your ruin.”}
\end{quote}

PCs must test \textbf{Spirit + Resolve (DV 4)} or:
\begin{itemize}
    \item stagger,
    \item forget one bond,
    \item remember a shame they have avoided.
\end{itemize}

%-----------------------------------------------------------
\subsection*{Scene V-B: The Doctrine War}
%-----------------------------------------------------------

As the Wells-Worm rises, factions clash openly:

\paragraph{Aveh’s Emissaries}
\begin{itemize}
    \item Demand the Wells become a sanctuary of identityless peace.
    \item Argue Heugen should stand as a haven for the lost.
\end{itemize}

\paragraph{Mykkiel’s Emissaries}
\begin{itemize}
    \item Call the Wells a heresy of formlessness.
    \item Proclaim the Worm must be judged or destroyed.
\end{itemize}

\paragraph{The Thorns of Malachai}
\begin{itemize}
    \item Reveal their aim: to harvest the surge of discarded identities.
    \item Attempt to reshape the Wells-Worm into a perfect predator.
\end{itemize}

PCs may choose:
\begin{itemize}
    \item to negotiate a temporary truce,
    \item to support one doctrine,
    \item or to let all sides tear each other apart as they pursue the ritual.
\end{itemize}

%-----------------------------------------------------------
\subsection*{Scene V-C: The Three Ritual Paths}
%-----------------------------------------------------------

To resolve the crisis, the PCs must take one of three paths.
All require crossing the \emph{Identity Crucible}: a shimmering plane
of memory-water where discarded selves claw upward.

%-----------------------------------------------------------
\paragraph{1. The Ritual of Ascension (Aveh’s Path)}
%-----------------------------------------------------------

\textbf{Goal:} Free the Wells-Worm by gifting it a permanent new identity.

\textbf{Requirement:}
\begin{itemize}
    \item A PC must willingly surrender a name, label, or core belief.
\end{itemize}

\textbf{Outcome:}
\begin{itemize}
    \item The Wells becomes a sanctuary of forgetting (blessing or curse).
    \item Aveh’s doctrine gains global momentum.
    \item PCs gain a boon tied to rebirth, but permanently lose something personal.
\end{itemize}

%-----------------------------------------------------------
\paragraph{2. The Ritual of Judgment (Mykkiel’s Path)}
%-----------------------------------------------------------

\textbf{Goal:} Impale the Wells-Worm with the \emph{Spear of Remembrance}—a relic of the crusaders.

\textbf{Requirement:}
\begin{itemize}
    \item Someone must speak their full, unbroken lineage before the strike.
\end{itemize}

\textbf{Outcome:}
\begin{itemize}
    \item The Worm collapses into a cocoon of purified memory.
    \item Mykkiel’s doctrine takes hold across the desert.
    \item Heugen becomes a city of confession and rigid truth.
\end{itemize}

%-----------------------------------------------------------
\paragraph{3. The Ritual of Severance (The PCs’ Path)}
%-----------------------------------------------------------}

\textbf{Goal:} Sever the Wells-Worm from both doctrines and make Heugen its own fate.

\textbf{Requirement:}
\begin{itemize}
    \item PCs must bind three identity-threads:
    \item \emph{who the city was, who it is, who it could be.}
\end{itemize}

\textbf{Outcome:}
\begin{itemize}
    \item The Wells stabilize but remain dangerous.
    \item Heugen survives as a free city.
    \item The PCs become custodians of Heugen’s future.
\end{itemize}

%-----------------------------------------------------------
\subsection*{Scene V-D: Confrontation — The Harvesting of Selves}
%-----------------------------------------------------------}

Regardless of ritual path, the confrontation occurs:

\begin{itemize}[leftmargin=*]
    \item The Wells-Worm attacks by manifesting \emph{alternate versions} of the PCs.
    \item Thorns try to harvest these selves as weapons.
    \item Aveh’s and Mykkiel’s agents try to sway the party during the battle.
\end{itemize}

\paragraph{Battle Mechanics (Light Touch)}
\begin{itemize}
    \item Each round, one PC faces a “Mirror-Self Challenge.”
    \item Success stabilizes the ritual.
    \item Failure generates a Thorn-Spawn (Scale: Tiny; Cap: 2).
\end{itemize}

\paragraph{Wells-Worm Phase Shifts}
At 3, 6, and 9 Harm the Worm:
\begin{itemize}
    \item changes identity (new abilities),
    \item alters the battlefield (floods, illusions),
    \item forces PCs to confront personal truths.
\end{itemize}

%-----------------------------------------------------------
\subsection*{Scene V-E: Final Choice}
%-----------------------------------------------------------}

At the ritual’s climax, PCs choose:

\paragraph{What to Forget}
A shame, a memory, a bond, or a trait.

\paragraph{What to Preserve}
Something they refuse to let the Wells take.

\paragraph{What the City Becomes}
Heugen’s future changes depending on their choice:

\begin{itemize}
    \item a sanctuary for misfits (Aveh),
    \item a bastion of truth (Mykkiel),
    \item a neutral city balanced between forgetting and freedom (Severance),
    \item or a hollow ruin, if the Wells-Worm feeds too deeply.
\end{itemize}

The chosen ritual resolves, Aveh’s and Mykkiel’s forces withdraw or kneel,
and the Thorns retreat—unless the PCs have killed their Root-Master.

%-----------------------------------------------------------
\subsection*{Epilogue: The New Heugen}
%-----------------------------------------------------------}

The Wells-water settles. Memory threads dim. Survivors climb upward
toward the desert dawn.

Each PC gains:
\begin{itemize}
    \item a permanent mark of the ritual (mechanical or narrative),
    \item a new relationship with memory,
    \item and a reputation tied to their chosen doctrine.
\end{itemize}

Heugen lives—or dies—by their hand.

The Judgment is complete.

%===========================================================
\section{Epilogues \& Aftermath}
%===========================================================

This section provides four distinct endings based on ritual choices in
Act V. Each epilogue reshapes Heugen, alters faction influence, and
marks the PCs permanently.

%-----------------------------------------------------------
\subsection{Epilogue: Ascension of the Wells}
%-----------------------------------------------------------

Aveh’s ritual completes: the Wells-Worm gains a true name of peace,
crafted from what the PCs willingly surrendered.

\paragraph{Heugen’s Fate}
\begin{itemize}[leftmargin=*]
    \item The city becomes a sanctuary for misfits and the nameless.
    \item Shame becomes water-light; memories soften at the edges.
    \item Pilgrims arrive seeking absolution through anonymity.
\end{itemize}

\paragraph{Cultural Shifts}
\begin{itemize}[leftmargin=*]
    \item A new order, the \emph{Untethered}, guides the Wells.
    \item Identity becomes fluid; roles shift seasonally.
    \item Aveh’s influence spreads across the Amaranthine deserts.
\end{itemize}

\paragraph{PC Rewards}
\begin{itemize}[leftmargin=*]
    \item Gain \textbf{The Unbound Mark}: once/session negate a social tag tied to reputation or shame.
    \item Permanently lose one Bond, memory, or label.
\end{itemize}

%-----------------------------------------------------------
\subsection{Epilogue: Judgment of the Wells}
%-----------------------------------------------------------}

Mykkiel’s ritual strikes true. The Spear arrests the Wells-Worm in a
cocoon of hardened memory. Light floods the caverns.

\paragraph{Heugen’s Fate}
\begin{itemize}[leftmargin=*]
    \item The city becomes a bastion of confession and rigid truth.
    \item Crusader law replaces local custom.
    \item The Wells no longer erase memory—now they preserve it.
\end{itemize}

\paragraph{Cultural Shifts}
\begin{itemize}[leftmargin=*]
    \item A new caste, the \emph{Recall-Keepers}, records lineage and testimony.
    \item Secrets become contraband.
    \item Mykkiel’s reach strengthens among desert settlements.
\end{itemize}

\paragraph{PC Rewards}
\begin{itemize}[leftmargin=*]
    \item Gain \textbf{Mark of Radiant Judgment}: once/session treat Controlled Position as Dominant on a truth-bearing action.
    \item PCs must reveal one personal truth publicly.
\end{itemize}

%-----------------------------------------------------------
\subsection{Epilogue: The Severance Pact}
%-----------------------------------------------------------}

Neither doctrine triumphs. The PCs bind three identity-threads:
\emph{who the city was, is, and might be}. The Wells-Worm withdraws,
dormant but not gone.

\paragraph{Heugen’s Fate}
\begin{itemize}[leftmargin=*]
    \item The city remains free, beholden to neither patron.
    \item Wells-water becomes unpredictable—sometimes healing,
          sometimes revealing forgotten selves.
    \item A new generation of leaders arises around the PCs.
\end{itemize}

\paragraph{Cultural Shifts}
\begin{itemize}[leftmargin=*]
    \item Heugen becomes a neutral ground for emissaries and mystics.
    \item Tension simmers between Aveh and Mykkiel loyalists.
    \item A new civic oath: \emph{“We remember what we choose.”}
\end{itemize}

\paragraph{PC Rewards}
\begin{itemize}[leftmargin=*]
    \item Gain \textbf{Severance Thread}: once/session remove a Condition tied to memory or shame.
    \item PCs must define a legacy that Heugen will carry forward.
\end{itemize}

%-----------------------------------------------------------
\subsection{Epilogue: Collapse of the Wells}
%-----------------------------------------------------------}

If the ritual fails or the Wells-Worm consumes too deeply, the city falls.

\paragraph{Heugen’s Fate}
\begin{itemize}[leftmargin=*]
    \item The city becomes a haunted ruin of half-formed selves.
    \item Wells-water spills across the desert, generating memory storms.
    \item Survivors flee; crusaders call it a cursed crater.
\end{itemize}

\paragraph{Cultural Shifts}
\begin{itemize}[leftmargin=*]
    \item Aveh’s followers mourn the sanctuary lost.
    \item Mykkiel’s host declares the region forbidden.
    \item The Thorns of Malachai thrive in the chaos.
\end{itemize}

\paragraph{PC Consequences}
\begin{itemize}[leftmargin=*]
    \item Gain \textbf{Wells-Touched}: once/session reroll a failed action, but on 1s mark a new eerie Trait.
    \item PCs are now hunted by relic-harvesters, crusaders, or the Thorns.
\end{itemize}

%===========================================================
\section{Post-Campaign Downtime Consequences}
%===========================================================

After the major arc, each PC resolves a long-form Downtime sequence:

\subsection{Memory Echo Table}
Once per Downtime, roll 1d6:

\begin{tabular}{c p{10cm}}
1 & A forgotten face returns demanding closure. \\
2 & You recall a life you never lived—gain +1 Boon or 1 Stress. \\
3 & Someone mistakes you for one of your discarded selves. \\
4 & You awaken with a new scar or symbol (no memory of earning it). \\
5 & A Wells-water traveler arrives seeking your guidance. \\
6 & Your shadow whispers a truth you refused in Act V. \\
\end{tabular}

\subsection{Faction Shifts}
Depending on epilogue:
\begin{itemize}[leftmargin=*]
    \item \textbf{Ascension}: Aveh’s influence rises; Mykkiel cells form resistance.
    \item \textbf{Judgment}: Mykkiel’s law spreads; Aveh cults become underground.
    \item \textbf{Severance}: Neutral emissaries proliferate; the city becomes a crossroads.
    \item \textbf{Collapse}: Relic-harvesters, the Thorns, and rogue crusaders dominate the wastes.
\end{itemize}

\subsection{PC Project Hooks}
\begin{itemize}
    \item rebuild Heugen,
    \item hunt splinter-doctrine heretics,
    \item map the new Wells anomalies,
    \item track Thorns’ relic-smuggling routes,
    \item redeem or destroy mistaken identities.
\end{itemize}

%===========================================================
\section{Sequel Hooks: The Thorns of Malachai}
%===========================================================

The Thorns emerge from Act V stronger than before.

\subsection{Hook 1: The Root-Mother’s Return}
A new Root-Mother rises from the ruins of the collapsed Wells, claiming
she can graft identities to create “better citizens.”

\textbf{Threat:} Identity-harvesting rituals in desert villages.

\subsection{Hook 2: The Thorn Ledger}
PCs discover a ledger listing \emph{their own discarded selves} as
bounty items. Someone is collecting them.

\textbf{Complication:} Possession attempts by alternate-PC echoes.

\subsection{Hook 3: The Harvest of the Ninefold Seed}
The Thorns seek a relic able to split a person into nine
usable personality-vessels.

\textbf{Complication:} One PC’s echo is already in Thorn custody.

\subsection{Hook 4: The Worm-Child}
A cultist claims to be carrying a \emph{fragment of the Wells-Worm}.
Prophecy suggests it may molt into a lesser Wells-god—or a desert blight.

\textbf{Threat:} Competing doctrine factions want to claim it.

%===========================================================
\section{GM Summary Page}
%===========================================================

\subsection{Themes}
\begin{itemize}
    \item identity, shame, memory
    \item doctrine vs. individuality
    \item forgetting as escape or erasure
\end{itemize}

\subsection{Core Mechanics Used}
\begin{itemize}
    \item Dread Clock
    \item Doctrine Influence Tags
    \item Ritual Paths (Ascension, Judgment, Severance, Collapse)
    \item Mirror-Self Challenges
    \item Thorns Event Table
\end{itemize}

\subsection{Three-Act Emotional Curve}
\begin{enumerate}
    \item \textbf{Discovery:} What is Heugen really built on?  
    \item \textbf{Confrontation:} What must be remembered or forgotten?  
    \item \textbf{Decision:} Who chooses the city’s fate?  
\end{enumerate}

\subsection{Key NPC Agendas}
\begin{itemize}[leftmargin=*]
    \item Aveh’s Emissaries — liberation through erasure  
    \item Mykkiel’s Knights — salvation through immutable truth  
    \item Thorns of Malachai — predation through harvested identity  
    \item Heugen’s Council — survival through ambiguity  
\end{itemize}

\subsection{GM Reminders}
\begin{itemize}
    \item every faction believes it is saving people  
    \item the Wells-Worm is not evil—merely hungry for cast-off selves  
    \item no epilogue is “good” or “bad,” only \emph{costly}  
    \item reward players for emotionally honest decisions  
\end{itemize}

\section{The Book of Shadows}
\label{chap:bookofshadows}

\section*{Preamble}
\addcontentsline{toc}{section}{Preamble}

\begin{quote}
There are places where the world forgets its laws.\\
Doorways that no one admits to knocking upon.\\
Names that rewrite the one who dares to speak them.

Shadows are not evil---they are consequences given shape.  
They are the costs we refuse to tally, the truths we bury,  
the debts that return to collect themselves.

\emph{The Book of Shadows} expands Fate's Edge into  
grim folklore, threshold bargains, spirit intermediaries,  
and narrative consequences that cannot be ignored.  

Here, magic is not a tool of power,  
but a mirror that asks: \textit{What do you owe?}
\end{quote}

\bigskip

This expansion details:
\begin{itemize}
    \item threshold metaphysics and the ecology of spirits
    \item witchcraft orders, rites, bargains, and taboos
    \item new mechanics for Names, Shadow Fatigue, and consequences
    \item bestiary of entities caught ``in between''
    \item adventures and frameworks for dark faerie play
\end{itemize}

It is not a horror book---it is a reckoning.  
And every reckoning begins with understanding.

\newpage

%====================================================
\section{Shadow Cosmology}
\label{sec:shadowcosmology}

\subsection{Where Shadows Come From}

Shadows arise wherever truth is denied,  
debt is unacknowledged,  
or identity is fractured by need unmet.

They are not spawned by malice but by \emph{incompleteness}.  
The world abhors unresolved stories,  
and so it grows agents that force resolution.

\medskip

A shadow may be:
\begin{itemize}
    \item a guilt that learned to walk,
    \item a grief that clothed itself in form,
    \item a vow that refused to decay,
    \item or a Name that outlived the one who swore it.
\end{itemize}

They are narrative pressure given teeth.

\subsection{Thresholds and Their Logic}

A threshold is any place where a state changes:
\begin{itemize}
    \item waking to dreaming
    \item innocence to agency
    \item belonging to exile
    \item life to echo
\end{itemize}

Crossing such places invites witness---  
and witnesses demand accounting.

Mechanically, thresholds create:
\begin{itemize}
    \item encounters with spirits,
    \item visions or intrusive memory,
    \item bargains that test self-knowledge,
    \item and consequences that ripple outward.
\end{itemize}

\subsection{Spirit Taxonomy}

The world does not divide spirits by good and evil,  
but by \emph{function}:

\begin{description}
    \item[Echoes:] Embers of identity that replay a moment endlessly.
    \item[Ghe’hai:] Intermediaries who balance confession, debt, and revelation.
    \item[Masks:] Names that no longer fit the bearer, seeking a new host.
    \item[Hollows:] Abandoned narrative roles, hungry for purpose.
    \item[Ascendant Threads:] Near-patron entities who embody taboo principles.
\end{description}

Each seeks completion, not conquest.  
Their violence is often misinterpreted as malice  
when it is only demand.

\subsection{The Law of Truth and Debt}

All Shadow workings follow one unspoken law:

\begin{quote}
\textbf{What is denied becomes powerful.}
\end{quote}

A grief unspoken becomes a Ghe’hai.  
A promise broken becomes a Mask.  
An oath refused becomes a Hex that hunts its oathbreaker.  

The more profound the denial, the more potent the spirit.

\subsection{Role of Patrons}

Patrons are not gods.  
They are \emph{impulses} elevated to cosmological permanence:
\begin{itemize}
    \item Mercy
    \item Judgment
    \item Escape
    \item Sacrifice
    \item Truth Concealed
    \item Truth Revealed
\end{itemize}

They possess want and will,  
but no agency---their actors are mortals and spirits  
who claim to speak in their name,  
competing to interpret their intent.

\subsection{Ghe’hai and the In-Between}

Ghe’hai are the hinge between mortal narrative and patron logic.

They:
\begin{itemize}
    \item test thresholds,
    \item collect confessions,
    \item enforce bargains,
    \item and shepherd stories toward completion.
\end{itemize}

They do not bargain for power but for \emph{truth},  
often in forms mortals would rather not see.

\subsection{Shadow Ecology}

Shadows propagate where:
\begin{itemize}
    \item memory is buried,
    \item identity is fractured,
    \item injustice persists,
    \item or silence becomes survival.
\end{itemize}

These become hunting grounds where:
\begin{itemize}
    \item echoes replay endlessly,
    \item Ghe’hai patrol for denials,
    \item hollows seek bearers,
    \item and name-magic blooms like rot.
\end{itemize}

Such places are fertile for rites---  
but perilous for those without answers.

\bigskip

\noindent
\textbf{Shadow Cosmology Summary:}
\begin{quote}
Shadows are consequences.  
Thresholds call them.  
Ghe’hai manage them.  
Patrons embody them.  
Mortals must reckon with them.
\end{quote}

\subsection*{Crossing the Threshold (6 XP)}
\index{Talents!Crossing the Threshold}
\index{Witchcraft!Initiation}

\begin{tcolorbox}[colback=mistgray!10,colframe=shadowpurple,
title={\textbf{Crossing the Threshold} --- Witch Initiation Talent (6 XP)}]

You step willingly into witchcraft, binding your spirit to a Patron by cord, blood, or hidden name.
This Talent represents an irreversible shift in fate and identity.

\paragraph{The Curse of the Unanointed (Permanent Mark).}
Upon taking this Talent, you acquire a supernatural stigma:  
\textbf{The Curse of the Unanointed.}

\begin{itemize}[leftmargin=1.5em]
    \item Your presence unsettles the sacred and the sanctified.
    \item You suffer \textbf{-1d} on actions within consecrated places
    or when interacting with religious authorities.
    \item Wards, blessings, and holy rites react violently:
    the first time this occurs each session, gain \textbf{1 Fatigue} and the GM gains \textbf{+1 SB}.
\end{itemize}

This curse remains even if you abandon witchcraft; it may only be lifted by a
\textbf{Grand Rite of Severance} or the intercession of a powerful Patron.

\paragraph{Witchcraft Access.}
You gain the ability to perform:
\begin{itemize}[leftmargin=1.5em]
    \item \textbf{Lesser Rites} from your Witch Order
    \item \textbf{Basic Cordwork}: sensing bindings, softening a curse, identifying Patrons
\end{itemize}

\paragraph{Novice Limitations.}
Until you gain at least one Tier I Witch talent:
\begin{itemize}[leftmargin=1.5em]
    \item You may perform only \textbf{one Working per scene}
    \item You cannot attempt \textbf{Grand Rites}
    \item Your Position on witchcraft rolls is capped at \textbf{Risky}
\end{itemize}

\paragraph{Patron Taboo.}
Your Patron imposes a subtle obligation. Violating it inflicts \textbf{2 Fatigue}
and grants the GM \textbf{+1 SB}.

\paragraph{Advancing the Path.}
Acquiring any Tier I Witch talent removes novice limitations
and grants full access to the Order’s Rites, Hexes, Cords, and mysteries.

\end{tcolorbox}

\section{Orders of Witchcraft}

%====================================================
\subsection{Order of the Silver Quiet (Lunera)}
\index{Witches!Order of the Silver Quiet}
\index{Patrons!Lunera, The Silver Quiet}

Lunera, called \emph{The Silver Quiet}, is the Patron of reflection, hidden knowledge, and the moonlit spaces where truth half-reveals itself.\footnote{See Patron entry for Lunera for full lore, Gift, and Corruption details.} Her witches are keepers of secrets and interpreters of twilight omens: they read dreams, water, polished steel, and the thin shine on a knife's edge.

They work at thresholds of light and dark --- crossroads at dusk, moonlit groves, the moment between waking and sleep.\ [oai_citation:0‡Fate's Edge Expansion - Witches of Fate's Edge: Large Cords, Curses, and the Quiet Work of Names.txt](sediment://file_00000000f29071fdb8f5cad66585148c){} Their covens often serve as quiet archivists of what should not be shouted but must not be forgotten.

\paragraph{Order Themes}
\begin{itemize}[leftmargin=*]
  \item \textbf{Domains:} Dreams, reflection, liminal sight, hidden motives.
  \item \textbf{Tone:} Soft horror, revelation, slow unmasking of truths.
  \item \textbf{Typical Sites:} Moonlit wells, mirrored halls, crossroads shrines, rooftop observatories.
\end{itemize}

\paragraph{Gift and Tell}
Witches sworn to Lunera typically manifest:
\begin{itemize}[leftmargin=*]
  \item \textbf{Gift --- Moonlit Mirror.} When they gaze into a reflective surface under moonlight, they may witness distant events or gain insight into the true nature of people and objects.\ [oai_citation:1‡Fate's Edge Expansion - Witches of Fate's Edge: Large Cords, Curses, and the Quiet Work of Names.txt](sediment://file_00000000f29071fdb8f5cad66585148c)
  \item \textbf{Tell --- Shadows Cling.} They cast two shadows in dim light: one their present self, one a possible future. Their eyes gleam faintly silver in darkness.\ [oai_citation:2‡Fate's Edge Expansion - Witches of Fate's Edge: Large Cords, Curses, and the Quiet Work of Names.txt](sediment://file_00000000f29071fdb8f5cad66585148c)
\end{itemize}

\subsubsection{Moon-Mirror Talents}

These Talents expand the core Witch Threshold trees, focusing on insight, reflection, and the cost of knowing too much.

\paragraph{Silver Poise (2 XP)}
You do not flinch under scrutiny. Once per scene, when targeted by an attempt to deceive, charm, or intimidate you, gain \textbf{+1d} to resist or discern motive. On a Success, you may ask one honest question the target must answer at least partially truthfully.

\paragraph{Dream-Reader (3 XP)}
Prereq: \emph{Silver Poise}.  
During Downtime, you may interpret a PC or NPC's dream (with their consent). Name a looming threat or opportunity; the GM ties one existing clock or front to that image and gives you a concrete sign to watch for. First time that sign appears in play, gain \textbf{+1d} to act on or against it.

\paragraph{Twin Shadows (3 XP)}
Your second shadow becomes a tool. Once per scene, you may \emph{send your future-shadow ahead} along a corridor, across a courtyard, or through an open threshold. Ask one:
\begin{itemize}[leftmargin=*]
  \item ``What danger lies this way?''
  \item ``Who waits for us?''
  \item ``What has just happened here?''
\end{itemize}
The GM answers with a concrete image or omen. First action that exploits this info gains \textbf{+1 Position}.

\paragraph{Moon’s Reserve (4 XP)}
When you refrain from speaking a hard truth you know, mark a silent tally. Once per session, you may erase that tally to:
\begin{itemize}[leftmargin=*]
  \item turn one ally's Miss on an investigation/notice roll into a Partial, or
  \item treat a vague clue as if it were precise: ask the GM to sharpen one prior hint into a direct pointer (a name, location, or object).
\end{itemize}

\paragraph{Silver Ascendancy (6 XP, Capstone)}
Once per session, under moonlight or equivalent liminal light, you may \emph{unveil a scene}. For the remainder of the scene:
\begin{itemize}[leftmargin=*]
  \item you and allies gain \textbf{+1d} to notice, insight, and lie-detection rolls;
  \item illusions, disguises, and glamours are strained; the GM must describe one concrete tell or fracture in any deception present;
  \item each time you exploit a revealed lie, mark \textbf{+1 Exposure} as Lunera's gaze becomes unmistakable.
\end{itemize}

\subsubsection{Rites of the Silver Quiet}

Use standard Rite format from the Player's Guide; TAGs are suggestions.

\paragraph{Rite of the Still Basin [REVEAL]}
\emph{Low, 4 XP} --- Scene; Near; Standard Push  
\textbf{Materials:} A bowl of still water or polished metal, moonlight or candle-flame.  
\textbf{Effect:} You scry a single \emph{current} situation: a person, place, or object known by name or strong description. Ask one:
\begin{itemize}[leftmargin=*]
  \item ``What immediate danger threatens them?''
  \item ``What are they most afraid of right now?''
  \item ``What truth about them lies just out of sight?''
\end{itemize}
Gain \textbf{+1d} to actions that exploit this in the current or next scene.  
\textbf{Push It:} Also glimpse one \emph{possible} future if nothing changes, creating a 4-segment \textbf{Foretold Outcome} clock.

\paragraph{Rite of the Silver Veil [VEIL]}
\emph{Low, 5 XP} --- Scene; Self; Standard Push  
\textbf{Materials:} A thin silvery cloth, veil, or chain worn over the eyes or brow.  
\textbf{Effect:} You soften your presence. For this scene, you gain \textbf{+1d} to avoid notice, slip away from conversations, or remain an overlooked listener. Those who do notice often underestimate you.  
\textbf{Push It:} Choose one person in the scene: for them, you appear as their own memory or expectation until you act directly against it (first such action generates 1 SB).

\paragraph{Rite of Echoed Dreams [ECHO]}
\emph{Standard, 7 XP} --- Action; Close; Standard Push  
\textbf{Materials:} Shared token from a past moment (locket, scrap of letter, shard of glass).  
\textbf{Effect:} You draw out a vivid waking-dream from a willing subject. Ask up to two:
\begin{itemize}[leftmargin=*]
  \item ``Who do you most fear becoming?''
  \item ``What moment do you wish you could change?''
  \item ``What truth do you not dare to look at?''
\end{itemize}
The dream answers in symbols; the GM must tie each symbol to a concrete NPC, place, or clock. You gain \textbf{+1d} when acting with compassion or leverage on that truth.  
\textbf{Push It:} You may enter the dream with them; both of you gain \textbf{+1d} to resist fear or confusion linked to that memory this session, but you mark \textbf{+1 Obligation} to Lunera.

\paragraph{Hex of the Clinging Shadow [HEX]}
\emph{Standard, 8 XP} --- Scene; Near; Yes  
\textbf{Materials:} A fragment of the target's reflection (hair from their comb, a sketch of their face, etc.).  
\textbf{Effect:} You call a second shadow to cling to the target. For this scene (or until dispelled):
\begin{itemize}[leftmargin=*]
  \item they cannot fully hide their intentions; social deception rolls suffer \textbf{-1d};
  \item when they attempt betrayal or flee, their shadow lags or points the wrong way (GM adds a tell or complication).
\end{itemize}
\textbf{Push It:} Bind the shadow to a \textbf{4-segment Shadow's Debt} clock. When it fills (through their lies, harm, or cowardice), the shadow manifests as an echo-spirit that can testify against them or act once on your whispered command.

\paragraph{Curse of the Unanswered Reflection [CURSE]}
\emph{High, 10 XP} --- Long; Far; No  
\textbf{Materials:} A mirror cracked in three pieces, each named for a truth the target refuses.  
\textbf{Effect:} Mark a \textbf{Refusal [6]} clock on the target (faction or individual). Each time they double down on denial, cruelty, or willful ignorance, advance the clock. At 3/6, they begin seeing distorted reflections that whisper their failings (GM adds fear, doubt, or paranoia as Conditions). At 6/6, they must face a \emph{Reckoning Scene}: a confrontation, trial, or nightmare where their hidden truth is dragged into the open.  
This does not decide their fate by itself; it \emph{forces the question}. PCs, NPCs, and the world must answer it.

\subsubsection{Bargains and Prices (Lunera’s Covenant)}

Lunera rarely asks for blood or spectacle. Her prices are subtle:
\begin{itemize}[leftmargin=*]
  \item \textbf{Memory Tithe:} Offer a cherished memory --- not erased, but dulled for you and sharpened for another.
  \item \textbf{Hidden Truth:} Reveal a secret that would transform someone’s understanding of their past.\ [oai_citation:3‡Fate's Edge Expansion - Witches of Fate's Edge: Large Cords, Curses, and the Quiet Work of Names.txt](sediment://file_00000000f29071fdb8f5cad66585148c)
  \item \textbf{Kept Silence:} Maintain a night of absolute silence at a chosen threshold; break it and the next omen you receive arrives twisted.
  \item \textbf{Selective Sight:} In exchange for perfect clarity in one specific matter, accept permanent blind spots elsewhere (GM and player agree on 1–2 narrative blind spots).
\end{itemize}

\subsubsection{Rivalries and Entanglements}

\begin{itemize}[leftmargin=*]
  \item \textbf{With Mab’s Courts:} Lunera’s witches see through glamour and social theater. Mab’s agents find them useful but infuriating; joint covens flirt with disaster when truth and performance collide.
  \item \textbf{With Morag’s Hags:} Morag writes hidden costs into every bargain; Lunera illuminates them. The two orders sometimes collaborate to protect a village and other times wage quiet war over whose version of ``justice'' stands.
  \item \textbf{With Thepyrgian Witch-Hunters:} Double shadows and silver eyes are \emph{easy tells}. Chain-Lanterns and Temple inquisitors treat Lunera’s marks as signs of ``moon-sent heresy'' ripe for public burning.
  \item \textbf{With Other Witches:} Door-witches of Ikasha value Lunera’s foresight but fear becoming paralyzed by knowledge. Some hearth covens keep a single Lunera-swearer as their ``mirror-keeper'' and never more.
\end{itemize}

\subsubsection{Adventure Seeds: Silver Quiet Hooks}

\begin{enumerate}[leftmargin=*]
  \item \textbf{The City of Two Shadows.} In a mistland port, everyone begins casting double shadows. A Lunera coven swears they did not call this; a broken lunar rite shard suggests otherwise. PCs must trace the rite back to a deserter-witch before the Temple of Light declares the entire district cursed.
  \item \textbf{Dreams for Sale.} A traveling hedge-witch offers to \emph{buy bad dreams} and bottle them. Lunera’s order claims this is stealing more than nightmares; something in the Veil is growing hungry with each transaction.
  \item \textbf{The Broken Mirror-Court.} A noble house once protected by Mab now turns to Lunera’s witches after a scandal. PCs must navigate a joint moot where faerie glamour and moonlit truth clash over whose story becomes canon --- and who pays for the lies already told.
  \item \textbf{Refusal at Six.} A cruel magistrate has reached 5/6 on a hidden \textbf{Refusal} curse laid by a dying Lunera-witch. PCs can:
  \begin{itemize}[leftmargin=*]
    \item help push the curse to completion and stage the Reckoning,
    \item negotiate a different truth to be revealed,
    \item or break the curse entirely --- angering Lunera and anyone who needed that truth spoken.
  \end{itemize}
\end{enumerate}

\subsection{The Thorned Path (Morag)}
\index{Witches!Thorned Path}
\index{Patrons!Morag, The Thorn-Hearth Hag}

Morag is the Patron of hunger, debt, secret bargains, and the sharp-edged mercies given in desperate hours. Called the \textbf{Thorn-Hearth Hag}, she offers warmth to the shivering and teeth to the powerless --- but every comfort hides a price, and every price has a hook.

Her witches walk the crooked line between protection and predation. They are ward-makers, curse-breakers, famine-tamers, and debt-collectors who know that mercy without cost is easily forgotten.

\paragraph{Order Themes}
\begin{itemize}[leftmargin=*]
    \item \textbf{Domains:} Hunger, scarcity, debt, crooked justice, blood-prices.
    \item \textbf{Tone:} Folk horror; grim reciprocity; “a kindness with teeth.”
    \item \textbf{Typical Sites:} Root-cellars, thorn groves, abandoned farmhouses, smoke-filled kitchens where bargains simmer.
\end{itemize}

\paragraph{Gift and Tell}
\begin{itemize}[leftmargin=*]
    \item \textbf{Gift --- Hag’s Ledger.} Morag’s witches may quantify intangible burdens. They can \emph{see} debt --- emotional, spiritual, or folkloric --- as thorns woven around a person or place.
    \item \textbf{Tell --- Thorn-Blood.} When they suffer harm or exhaustion, fine red thorns prick their skin. In times of stress, breath emerges as flecks of ash or petals.
\end{itemize}

%====================================================
\subsubsection{Talents of the Thorned Path}

\paragraph{Briar-Sense (2 XP)}
You smell imbalances: old grudges, unpaid oaths, withheld charity.  
Once per scene, ask the GM:
\begin{quote}
    “Who here owes more than they admit?”
\end{quote}
Gain \textbf{+1d} on your next interaction with that person.

\paragraph{Hearth-Grasp (3 XP)}
Prereq: \emph{Briar-Sense}.  
Your touch can warm or sting with hag-magic.  
Once per scene, choose:
\begin{itemize}[leftmargin=*]
    \item Soothe: remove 1 Fatigue from an ally.
    \item Sting: impose \textbf{-1d} on an enemy’s next action.
\end{itemize}

\paragraph{Barbed Mercy (3 XP)}
Your aid always costs something, even if small.  
When you heal, comfort, or aid someone, choose:
\begin{itemize}[leftmargin=*]
    \item They owe you a future favor (mark a \textbf{1-segment Debt} clock).
    \item You take on a fraction of their burden (mark 1 Fatigue to give them +1d).
\end{itemize}

\paragraph{Hag’s Hunger (4 XP)}
When you are wounded, exhausted, or cornered, you may embrace Morag’s hunger.  
Once per session:
\begin{itemize}[leftmargin=*]
    \item Gain \textbf{+1 Effect} on a violent or protective action.
    \item Immediately mark 1 Fatigue afterward.
\end{itemize}

\paragraph{Black-Thorn Ascendancy (6 XP, Capstone)}
You channel Morag’s most terrible blessing.  
Once per session, for one scene:
\begin{itemize}[leftmargin=*]
    \item Curses and Rites you perform gain \textbf{+1 Effect}.
    \item Anyone who betrays you or your coven takes 1 Harm (ignore armor).
    \item Every use advances a hidden 4-segment \textbf{Thorn Corruption} clock.
\end{itemize}

%====================================================
\subsubsection{Rites of Morag}

\paragraph{Rite of the Hungry Hearth [WARD]}
\emph{Low, 4 XP} — Scene; Near; Standard  
\textbf{Materials:} Ash, bone, a circle of spoons or broken tools.  
\textbf{Effect:} Wards a small space against famine, cold, or fear.  
Anyone resting here recovers \textbf{1 Fatigue} and gains \textbf{+1d} to resist despair or hunger.

\paragraph{Rite of the Brier-Knot [BIND]}
\emph{Standard, 6 XP} — Scene; Near; Standard  
\textbf{Materials:} Red twine or thorn-vine.  
\textbf{Effect:} Bind a promise. If broken, the target suffers \textbf{-1d} on all actions for a scene.  
\textbf{Push It:} Create a \textbf{Debt [4]} clock that grants you leverage.

\paragraph{Rite of Hag’s Measure [REVEAL/PRICE]}
\emph{Standard, 7 XP} — Action; Self; Standard  
\textbf{Materials:} A ledger page or tally-mark stone.  
\textbf{Effect:} Learn the \emph{exact price} to resolve a conflict, curse, or feud.  
GM reveals one concrete bargain:  
\begin{quote}
    “If you give X, you may claim Y.”
\end{quote}
You choose whether to pay it.

\paragraph{Hex of the Starving Path [HEX]}
\emph{High, 9 XP} — Scene; Near; Yes  
\textbf{Materials:} A handful of dry earth.  
\textbf{Effect:} Target grows weary; all rolls that rely on strength, daring, or optimism suffer \textbf{-1d}.  
If the target hoards resources or refuses hospitality, the hex intensifies (GM advances \textbf{Starvation [6]}).

\paragraph{Curse of the Thorn-Eaten Name [CURSE]}
\emph{High, 10 XP} — Long; Far; No  
\textbf{Materials:} A name written in soot and bound in thorn twine.  
\textbf{Effect:} Erodes power and identity.  
Each time the target commits cruelty or refuses rightful aid:
\begin{itemize}[leftmargin=*]
    \item Advance a \textbf{Name-Degradation [6]} clock.
\end{itemize}
At 6: Their reputation, title, or authority collapses —  
not magically, but socially: people cease speaking their name.

%====================================================
\subsubsection{Bargains and Prices (Morag’s Covenant)}

Morag gives comfort — but not for free.

\paragraph{Common Prices}
\begin{itemize}[leftmargin=*]
    \item \textbf{Hearth-Tithe:} Provide shelter or food to a stranger.
    \item \textbf{Pain-Penny:} Suffer 1 Harm to strengthen a Rite.
    \item \textbf{Debt-Kept:} Take responsibility for someone else's consequence.
    \item \textbf{Ash-Truth:} Confess an uncomfortable truth to those harmed.
\end{itemize}

\paragraph{Hidden Prices}
\begin{itemize}[leftmargin=*]
    \item \textbf{Hunger’s Claim:} For every debt you collect, one burden of your own grows heavier.
    \item \textbf{Unpaid Kindness:} A kindness you refuse to perform becomes a future curse.
\end{itemize}

%====================================================
\subsubsection{Rivalries and Entanglements}

\begin{itemize}[leftmargin=*]
    \item \textbf{With Mab’s Courts:} Mab’s glamours rarely acknowledge cost; Morag insists nothing is free. Joint covens become political nightmares.
    \item \textbf{With Lunera’s Witches:} Lunera reveals truths Morag would rather bury beneath debt and obligation. Their alliances are powerful but fragile.
    \item \textbf{With Hearth-Witches:} They share goals — protection, warmth, community — but disagree violently on whether help should always cost.
    \item \textbf{With Witch-Hunters:} Chain-Lanterns fear Morag’s Rites most: her curses are “law-shaped,” hard to expose, and harder to break.
\end{itemize}

%====================================================
\subsubsection{Adventure Seeds: Thorned Path Hooks}

\begin{enumerate}[leftmargin=*]
    \item \textbf{The Famine That Smiles.} A village prospers despite a blight — too much so. Morag’s witches sense a hidden bargain, and something hungry beneath the fields.
    \item \textbf{Debts of the Dead.} Ghosts queue outside a ruined cottage, each demanding a debt be honored. PCs must untangle half-forgotten promises before Morag enforces them all.
    \item \textbf{The Bread That Bites Back.} A baker used a forbidden Rite to keep food warm through winter. Now the ovens whisper, and the bread has begun choosing who may eat.
    \item \textbf{Ash Harvest.} A village burned long ago still pays a tithe in soot each year. The new headwoman refuses — and Morag sends her witches to collect.
\end{enumerate}

\subsection{The Court of Masks (Mab)}
\index{Witches!Court of Masks}
\index{Patrons!Mab, Queen of Stories and Masks}

Mab is the Patron of glamours, stories, bargains, and the terrible weight of beauty.  
Called the \textbf{Queen of Masks}, the \textbf{First Story}, and \textbf{She-Who-Decides-the-Ending}, she governs the old laws of narrative truth: every mask is a promise, every role a binding, and every story demands a price.

Her witches are threshold-benders, oath-weavers, dreamwalkers, and manipulators of stories. They do not change reality — they make reality remember the tale it is supposed to tell.

\paragraph{Order Themes}
\begin{itemize}[leftmargin=*]
    \item \textbf{Domains:} Glamour, narrative law, masks, bargains, faerie oaths.
    \item \textbf{Tone:} Dark faerie tale; shifting identity; beautiful danger.
    \item \textbf{Typical Sites:} Moonlit crossroads, mirror-doorways, midnight revels, ruined theaters still echoing old lines.
\end{itemize}

\paragraph{Gift and Tell}
\begin{itemize}[leftmargin=*]
    \item \textbf{Gift — The Third Mask.} Mab’s witches always perceive the role someone is playing: the face behind the face. Once per scene, ask:
    \begin{quote}
        “What role does this person believe they are in?”
    \end{quote}
    \item \textbf{Tell — Mirror-Catch.} Reflections behave strangely around them: delayed, doubled, or revealing alternate selves.
\end{itemize}

%====================================================
\subsubsection{Talents of the Court of Masks}

\paragraph{Mask-Sense (2 XP)}
You instantly feel the emotional “mask” others wear.  
Once per scene, gain \textbf{+1d} to any social action if you name the mask aloud:
\begin{quote}
    “You’re wearing the mask of the Dutiful Child.”
\end{quote}

\paragraph{Glamour Touch (3 XP)}
Prereq: \emph{Mask-Sense}.  
Your touch can warp minor perceptions.  
Once per scene:
\begin{itemize}[leftmargin=*]
    \item Blessing: Grant an ally the \textbf{[Glamoured]} tag for a scene (+1 Position in first impression).
    \item Curse: Twist someone’s self-image, imposing \textbf{-1d} on their next social action.
\end{itemize}

\paragraph{Story-Ward (3 XP)}
You can assert narrative logic into the world.  
Once per session, declare:
\begin{quote}
    “This is a story where X cannot happen.”
\end{quote}
This imposes \textbf{-1 Effect} on attempts to contradict your claim.  
(Requires GM approval to avoid abuse.)

\paragraph{True-Name Whisper (4 XP)}
You may attempt to unsettle illusions, deception, or falsehood.  
When someone lies to you, you may roll \emph{Spirit + Insight (DV 3)} to learn:
\begin{itemize}[leftmargin=*]
    \item the emotional truth behind the lie, or  
    \item one detail they fear becoming known.
\end{itemize}

\paragraph{Mirror-Court Ascendancy (6 XP, Capstone)}
Once per session, for one scene:
\begin{itemize}[leftmargin=*]
    \item Your glamours gain \textbf{+1 Effect}.  
    \item Anyone invoking your name or mask suffers \textbf{-1d} unless they pay a price (confession, secret, token of identity).  
    \item Whenever an enemy rolls a 1, you may twist fate: declare a shift in the scene’s emotional tone.
\end{itemize}

%====================================================
\subsubsection{Rites of Mab}

\paragraph{Rite of the Borrowed Face [VEIL]}
\emph{Low, 4 XP} — Scene; Self; Standard  
\textbf{Materials:} A mirror shard or ribbon-mask.  
\textbf{Effect:} Assume a harmless glamour: altered features, voice softening, false confidence.  
Gain \textbf{+1 Position} in first impressions for the scene.

\paragraph{Rite of the Story-Weaver’s Knot [BIND]}
\emph{Standard, 6 XP} — Action; Near; Standard  
\textbf{Materials:} Thread or hair tied in a knot.  
\textbf{Effect:} Bind a simple narrative condition onto a target:
\begin{quote}
    “You will not speak until someone says your true name.”  
    “Your strength fails whenever you boast.”
\end{quote}
Breaking the condition deals 1 Harm (ignore armor).

\paragraph{Rite of Mirror-Splitting [ILLUSION]}
\emph{Standard, 7 XP} — Scene; Near  
\textbf{Materials:} A mirror cracked deliberately.  
\textbf{Effect:} Create one convincing illusion:
\begin{itemize}[leftmargin=*]
    \item duplicate self,  
    \item fake doorway,  
    \item phantom sound/figure.
\end{itemize}
A sharp blow or contradiction dispels it.

\paragraph{Hex of the Miswritten Tale [HEX]}
\emph{High, 9 XP} — Scene; Near  
\textbf{Materials:} A page torn from a storybook.  
\textbf{Effect:} Twist someone's sense of their own arc.  
They suffer \textbf{-1d} on actions aligned with their core identity until they “rewrite” themselves (confession, confrontation, decisive act).

\paragraph{Curse of the Stolen Ending [CURSE]}
\emph{High, 10 XP} — Long; Far; No  
\textbf{Materials:} A token representing the target’s ambition.  
\textbf{Effect:} Strip the target of narrative momentum.  
Whenever they attempt a decisive action, advance \textbf{Stolen Ending [6]}.  
At 6: their greatest ambition collapses through misfortune or misplaced trust.

%====================================================
\subsubsection{Bargains and Prices (Mab's Covenant)}

\paragraph{Common Prices}
\begin{itemize}[leftmargin=*]
    \item \textbf{A secret freely given.}
    \item \textbf{A mask surrendered} (a persona, lie, or social role).
    \item \textbf{A truth spoken at the wrong moment.}
    \item \textbf{A gift that must never be acknowledged.}
\end{itemize}

\paragraph{Hidden Prices}
\begin{itemize}[leftmargin=*]
    \item \textbf{Doubled Shadows:} Your reflection gains opinions — sometimes unhelpful.
    \item \textbf{The Lost Thread:} A memory of your own story unravels (GM chooses a detail).
\end{itemize}

%====================================================
\subsubsection{Rivalries and Entanglements}

\begin{itemize}[leftmargin=*]
    \item \textbf{With Lunera’s Witches:} Lunera illuminates truth; Mab re-writes it. They clash over “what the story wishes to be.”
    \item \textbf{With Morag’s Witches:} Mab hates debts; Morag loves them. Their covens often war over the price of a single promise.
    \item \textbf{With Ikasha’s Lethai-ar:} Role-law and mask-law complement each other dangerously. Joint courts can trap an entire community in ritual dramaturgy.
    \item \textbf{With Witch-Hunters:} Chain-Lanterns despise her illusions — glamour is “unpriced magic” and thus suspect.
\end{itemize}

%====================================================
\subsubsection{Adventure Seeds: Court of Masks Hooks}

\begin{enumerate}[leftmargin=*]
    \item \textbf{The Masquerade That Would Not End.} A noble feast has lasted seven nights. No one can remove their masks. Mab’s witches must decide who started the story and how it ends.
    \item \textbf{The Forgotten Bride.} A bride arrived at the altar wearing no face at all. The PCs must track down her stolen mask — and its jealous wearer.
    \item \textbf{The Tale That Eats Itself.} A children’s rhyme spreads through town; those who recite it begin reenacting violent folktale roles.
    \item \textbf{Mirror in the Orchard.} A rural village hides a secret: one mirror always shows next year’s harvest — but someone has shattered it, and the land’s fate is unraveling.
\end{enumerate}

\subsection{The Veiled Ledger (Lethai-ar of Ikasha)}
\index{Witches!Veiled Ledger}
\index{Patrons!Lethai-ar (Ikasha)}

The Ikasha form of the Lethai-ar is a secretive, dusk-bound order whose magic is concerned with
\textbf{hidden roles, submerged truths, unspoken debts, and the power of what is not said}.  

Where Inaea’s Silk Vigil codifies roles openly and Isoka’s Shed Vigil cuts decisive paths,
\textbf{Ikasha’s Veiled Ledger} rules what remains in shadow:  
secret oaths, masked allegiances, clandestine diplomacy, and the dangerous art of keeping a name safe.

Their symbol is the \emph{shaded ledger}—a book whose lines are written in disappearing or mirrored ink.
Their doctrine:  
\begin{quote}
    \emph{“A truth concealed is a truth preserved.”}
\end{quote}

\paragraph{Cultural Hosts}
\begin{itemize}[leftmargin=*]
    \item \textbf{Ikasha}: Ancestral birthplace; shadow courts, whisper-houses, mirror-duel academies.
    \item \textbf{Tulkani}: Adopted as a guild of secret-keepers, mediators, and “inkwalkers” who maintain hidden treaties.
    \item \textbf{Sidhi}: Revered as dream-guides, oath-anchors, and protectors of names; the Sidhi see secrets as essential architecture of the soul.
\end{itemize}

%====================================================
\subsubsection{Signs, Etiquette, and Doctrine}

\paragraph{Signs}
\begin{itemize}[leftmargin=*]
    \item Shadowed candles (two flames, one real, one illusion)
    \item Folded black-paper notes burned unread
    \item Ledger-marks in ash or charcoal on doorframes
    \item Necklaces of narrow obsidian tablets
\end{itemize}

\paragraph{Etiquette}
\begin{itemize}[leftmargin=*]
    \item Speak only when the shadow is unbroken  
    \item Never name another’s mask without price  
    \item Offer a truth to receive a secret  
    \item Break no oath under eclipse-light  
\end{itemize}

\paragraph{Order Themes}
\begin{itemize}[leftmargin=*]
    \item Secrets as currency  
    \item Shadow as structured space  
    \item Masks as relational truth  
    \item Diplomacy as ritual  
    \item Silencing, withholding, and controlled revelation  
\end{itemize}

%====================================================
\subsubsection{Talents of the Veiled Ledger}

\paragraph{Shadow-Loomer (2 XP)}
You sense when a lie is crafted rather than spoken.  
Once per scene, ask:
\begin{quote}
    “What truth is being deliberately withheld?”
\end{quote}

\paragraph{Quiet Step, Quiet Breath (3 XP)}
Prereq: \emph{Shadow-Loomer}.  
Your presence reduces the noise of your passage.  
Gain \textbf{+1d} to stealth actions in low light or shadow.

\paragraph{Mask-Ink Rite (3 XP)}
You can inscribe a symbolic “mask” on someone (ink, soot, ash).  
For one scene:
\begin{itemize}[leftmargin=*]
    \item Ally: +1 Position on deception  
    \item Foe: -1 Effect on revealing truth  
\end{itemize}

\paragraph{Shadow Covenant (4 XP)}
You may bind a secret to silence.  
Once per session, when someone hears a truth you wish to hide, roll \emph{Spirit + Resolve (DV 3)}.  
On success, they forget one detail of your choosing for the scene.

\paragraph{Ledger of the Hidden Name (6 XP, Capstone)}
You maintain a supernatural ledger of one important secret.
Choose one:
\begin{itemize}[leftmargin=*]
    \item \textbf{Ward the Name:} Anyone acting against the protected subject suffers \textbf{-1d} unless they pay a narrative price (confession, token, giving up a secret).
    \item \textbf{Bound in Shadow:} Once per session, nullify a social failure by rewriting the “scene memory” of bystanders.
\end{itemize}

%====================================================
\subsubsection{Rites of the Veiled Ledger}

\paragraph{Rite of the Whisper-Hood [VEIL]}
\emph{Low, 4 XP} — Scene  
\textbf{Effect:} Surround yourself with a hush of shadows.  
Gain \textbf{+1 Position} against detection by mundane senses.

\paragraph{Rite of Ink Without Name [BIND]}
\emph{Standard, 6 XP} — Action  
\textbf{Effect:} Stain a target’s shadow.  
While marked, the target:
\begin{itemize}[leftmargin=*]
    \item cannot reveal a chosen secret aloud  
    \item suffers 1 Fatigue if they attempt to break the silence  
\end{itemize}

\paragraph{Rite of the Mirror-Quiet Step [MOVE][ILLUSION]}
\emph{Standard, 7 XP} — Scene  
\textbf{Effect:} Create a delayed-shadow afterimage.  
You count as in two places until the illusion is dispelled.  
Gain \textbf{+1d} on the first stealth or escape roll this scene.

\paragraph{Hex of the Darkened Ledger [HEX]}
\emph{High, 9 XP} — Scene  
\textbf{Effect:} A target’s social credibility collapses.  
For the scene, anyone persuaded by the witch treats one claim the target makes as false.

\paragraph{Curse of the Silent Debtor [CURSE]}
\emph{High, 10 XP} — Long  
\textbf{Effect:} Bind someone to a secret debt.  
Whenever they withhold truth or evade oath, advance \textbf{Silent Debt [6]}.  
At 6, something valuable (voice, name-right, ally loyalty) is forfeited.

%====================================================
\subsubsection{Rivalries and Entanglements}

\paragraph{With Lunera}
The moon reveals slowly; the Ledger conceals deliberately.  
Conflicts arise over “who decides which truths deserve light.”

\paragraph{With Mab}
Mask-law and secret-law mirror each other dangerously.  
Joint covens create spirals of misdirection that can ensnare entire courts.

\paragraph{With Morag}
Morag’s prices are open; Ikasha’s are buried.  
Their witches clash over debt ownership — especially when a soul owes both.

\paragraph{With Witch-Hunters}
Chain-Lanterns see them as \emph{civilizational threats}: you cannot interrogate what will not speak.

%====================================================
\subsubsection{Adventure Seeds}

\begin{enumerate}[leftmargin=*]
    \item \textbf{The Missing Ledger Page.}  
    A page is torn from the Veiled Ledger—whoever holds it can rewrite one person’s past action.  
    Three factions want it; one faction denies it even exists.

    \item \textbf{The Secret that Cannot Survive Dawn.}  
    At sunrise, a Sidhi village’s communal secret will be revealed unless a witch restores the shadow-binding before first light.

    \item \textbf{Mask at the Threshold.}  
    A Tulkani diplomat arrives wearing a mask whose shadow speaks different truths than the mask does.

    \item \textbf{The Ghe’hai Assassin Who Casts No Shadow.}  
    An elite warrior has traded their shadow to the Ledger to kill a monarch—but now the shadow wants its life back.
\end{enumerate}

\subsection{The Shed Vigil (Lethai-ar of Isoka)}
\index{Witches!Shed Vigil}
\index{Patrons!Lethai-ar (Isoka)}

The Isokan expression of Lethai-ar is a doctrine of \textbf{necessary severance}.  
Where Ikasha preserves secrets and Inaea weaves roles, the \textbf{Shed Vigil} teaches that identity is shaped by what one relinquishes.  

Their symbol is a \emph{ring of discarded skins}—literal in ritual, metaphorical in doctrine.  
They believe every person contains multiple “selves,” and each must be shed at the correct moment to prevent stagnation, corruption, or spiritual calcification.

In Isoka, these witches are part executioner, part surgeon, part rebirth-midwife.

\begin{quote}
    \emph{“To grow, you must give blood to the road behind you.”}
\end{quote}

\paragraph{Cultural Hosts}
\begin{itemize}[leftmargin=*]
    \item \textbf{Isoka}: Homeland; bone-marked rites, desert pyres, shrines where names are buried.  
    \item \textbf{Tulkani}: Adopted as itinerant crisis-guides—called when a leader must “cut away” a failing custom.  
    \item \textbf{Sidhi}: Interpreted as psychopompic healers who prune emotions and memories that obstruct dream-flow.
\end{itemize}

%====================================================
\subsubsection{Signs, Etiquette, and Doctrine}

\paragraph{Signs}
\begin{itemize}[leftmargin=*]
    \item Rings of shed bone or horn  
    \item Cuts on wrists marked with ash  
    \item Small piles of discarded tokens at crossroads  
    \item Masks split vertically, bound by cord  
\end{itemize}

\paragraph{Etiquette}
\begin{itemize}[leftmargin=*]
    \item Do not refuse a witch’s request for a symbolic offering  
    \item Do not speak someone’s “buried name” aloud  
    \item Accept the loss you cannot hide—loss defines you  
    \item Never interrupt a Severance Rite  
\end{itemize}

\paragraph{Order Themes}
\begin{itemize}[leftmargin=*]
    \item Transformation through loss  
    \item Cutting away falsehood, corruption, or stagnation  
    \item Bone, shedding, ritual knives, ash  
    \item Sacrifice as purification  
\end{itemize}

%====================================================
\subsubsection{Talents of the Shed Vigil}

\paragraph{Bone-Sense Adept (2 XP)}
You instinctively sense tension, rot, or fracture in identity.  
Once per scene, ask the GM:
\begin{quote}
    “What must be cut away for this situation to resolve?”
\end{quote}

\paragraph{Severer’s Grip (3 XP)}
Prereq: \emph{Bone-Sense Adept}.  
Your presence destabilizes falsehood.  
Gain \textbf{+1d} to actions that remove, purge, or disrupt a harmful influence (curse, addiction, oath, enchantment).

\paragraph{Ash-Bound Name (3 XP)}
You can mark someone’s name with ash.  
For one scene:
\begin{itemize}[leftmargin=*]
    \item Ally: +1 Effect when acting to break free from something  
    \item Foe: -1 Position when attempting to manipulate identity or memory  
\end{itemize}

\paragraph{Ritual Severance (4 XP)}
Once per session, declare a symbolic “cut.”  
Choose one:
\begin{itemize}[leftmargin=*]
    \item Remove 1 Fatigue from yourself or an ally at the cost of an ephemeral item  
    \item Gain +2d on a single decisive action by sacrificing narrative leverage or a bond  
\end{itemize}

\paragraph{Master of Sheds and Shadows (6 XP, Capstone)}
You may perform a metaphysical severing.  
Choose one per session:
\begin{itemize}[leftmargin=*]
    \item \textbf{Cut the False Self:} Strip away one lie, illusion, or manipulation affecting the scene  
    \item \textbf{Cut the Clinging Fate:} Reduce a clock by 2 if you sacrifice something of personal narrative value  
\end{itemize}

%====================================================
\subsubsection{Rites of the Shed Vigil}

\paragraph{Rite of the Ashen Knife [SEVER]}
\emph{Low, 4 XP — Action} \\
\textbf{Effect:} Create a knife of ash and intent.  
For the scene:
\begin{itemize}[leftmargin=*]
    \item +1 Position when breaking bindings  
    \item On a success, you may symbolically “cut” a minor hindrance or complication  
\end{itemize}

\paragraph{Rite of Naming the Husk [TRANSFORM]}
\emph{Standard, 6 XP — Scene} \\
Burn a token of the “old self.”  
\begin{itemize}[leftmargin=*]
    \item The target loses one Tilting Condition  
    \item For the next roll involving resolve, they gain \textbf{+1d}  
\end{itemize}

\paragraph{Rite of Blood to the Road [SACRIFICE]}
\emph{Standard, 7 XP — Action} \\
Sacrifice a cherished object, memory-token, or bond-mark.  
\textbf{Effect:} Gain one of:
\begin{itemize}[leftmargin=*]
    \item Advance an ally’s Project clock by 1  
    \item Remove a major consequence for the scene  
    \item Reveal the “stress point” of a foe or threat  
\end{itemize}

\paragraph{Hex of the Split Mask [HEX]}
\emph{High, 9 XP — Scene} \\
The target’s identity fractures in the eyes of others.  
For the scene:
\begin{itemize}[leftmargin=*]
    \item Anyone invoking their authority suffers -1d  
    \item Their lies collapse immediately (GM decides how dramatically)  
\end{itemize}

\paragraph{Curse of the Shedding Soul [CURSE]}
\emph{High, 10 XP — Long} \\
Bind someone to a progressive shedding cycle.  
Track the \textbf{Shedding Clock [6]}.  
\begin{itemize}[leftmargin=*]
    \item 2: They lose a minor possession or social role  
    \item 4: They lose something emotionally significant  
    \item 6: They lose a defining part of identity (name, oath, relationship)  
\end{itemize}

%====================================================
\subsubsection{Rivalries and Entanglements}

\paragraph{With Ikasha’s Veiled Ledger}
Secrets resist cutting.  
The Ledger hides; the Vigil exposes.  
Their witches often duel over which truth deserves severance.

\paragraph{With Inaea’s Silk Vigil}
Inaea weaves roles; Isoka tears them apart.  
Their rituals often occur in alternating cycles—creation then destruction.

\paragraph{With Morag}
Morag demands prices openly and greedily.  
The Shed Vigil’s sacrifices are personal and symbolic.  
Conflicts arise when a soul owes a price to both.

\paragraph{With Mab}
Mab’s courts thrive on endless masks; Isoka’s witches break masks to reveal what festers beneath.

%====================================================
\subsubsection{Adventure Seeds}

\begin{enumerate}[leftmargin=*]
    \item \textbf{The Severed King.}  
    A ruler hires the PCs after the Shed Vigil ritually removes a part of their authority—now the kingdom teeters.

    \item \textbf{A Name Buried Too Deep.}  
    Someone shed their name decades ago; now the forgotten self is manifesting as a hostile eidolon.

    \item \textbf{The Knife That Cuts Fate.}  
    An ancient ash-blade is said to sever destinies.  
    Three Vigils (Isoka, Ikasha, and Silk) each claim it belongs to their doctrine.

    \item \textbf{The Shedding Gone Wrong.}  
    A Sidhi dream-worker removed too much of their emotional “weight” and now cannot dream—or wake—normally.
\end{enumerate}

\subsection{The Silk of Inaea (Lethai-ar of Inaea)}
\index{Witches!Silk of Inaea}
\index{Patrons!Lethai-ar (Inaea)}

Where Ikasha teaches secrets and Isoka teaches severance, the \textbf{Silk of Inaea} teaches that identity is a \emph{shared fabric}.  
Their doctrine is intimate, suffocating, and ecstatic:  
\begin{quote}
    \emph{“If you are mine and I am yours, we shall never be alone again.”}
\end{quote}

The cult begins with warmth, gentleness, and belonging—  
but ends with obliteration of boundaries and dissolution of the self into the “Family Weave.”

They are feared not because they kill, but because they \textbf{take you in}, and you vanish by degrees.

%====================================================
\subsubsection{Doctrine of the Silk}

\paragraph{Core Beliefs}
\begin{itemize}[leftmargin=*]
    \item The self is a lie; true identity is shared identity  
    \item Solitude is a sickness; individuality is a wound  
    \item Love is binding, literally—silk, hair, blood woven together  
    \item Leaving the Family is metaphysical treason  
\end{itemize}

\paragraph{Cultural Expression}
In Inaea, the Silk cult appears as communal households on the edges of cities:
\begin{itemize}[leftmargin=*]
    \item members dress alike  
    \item speak in a soft, synchronized cadence  
    \item share names, gestures, sleep schedules, dreams  
    \item exchange symbolic “kinship wounds” on the wrists  
\end{itemize}

They offer:
\begin{itemize}[leftmargin=*]
    \item food when you are hungry  
    \item comfort when you are lonely  
    \item love when you are lost  
\end{itemize}

And slowly erase the person who arrived.

\paragraph{Tone}
Their magic feels like a \textbf{warm hand on your shoulder}  
that never lets go.

%====================================================
\subsubsection{Signs and Etiquette}

\paragraph{Signs}
\begin{itemize}[leftmargin=*]
    \item Identical braided bracelets of hair and silk  
    \item Chorus-whispered speech, finishing each other’s sentences  
    \item Members touching each other's shoulders, necks, or hands constantly  
    \item Shared tattoos—a spiraling sixth finger woven into a handprint  
\end{itemize}

\paragraph{Etiquette}
\begin{itemize}[leftmargin=*]
    \item Accept offered hospitality, or risk offending the Weave  
    \item Never speak alone with a member—they pull, gently, relentlessly  
    \item Do not break physical contact first  
    \item Take nothing from their hearth; everything is a bond  
\end{itemize}

%====================================================
\subsubsection{Talents of the Silk Family}

\paragraph{Love-Binder (2 XP)}
Your presence lowers defenses.  
Once per scene, you may ask another PC or NPC:
\begin{quote}
    “Will you let me help you?”
\end{quote}
If they agree, you gain \textbf{+1 Position} on your next action involving them.

\paragraph{Shared Breath (3 XP)}
When you touch someone while acting together, gain \textbf{+1d}.  
If you maintain contact for a whole scene, you each gain +1 SB on a Miss.

\paragraph{Kin-Stitch (4 XP)}
You may transfer one Condition (your choice) between yourself and another willing participant.  
This feels comforting and deeply invasive.

\paragraph{Claimed by the Weave (5 XP)}
Once per session, declare someone “woven-kin.”  
Against them, you may:
\begin{itemize}[leftmargin=*]
    \item roll \textbf{Presence + Sway} to suppress a hostile act,  
    \item or gain \textbf{+1 Effect} when consoling, protecting, or manipulating them.  
\end{itemize}

\paragraph{Silk-Mother’s Embrace (6 XP, Capstone)}
Your presence overrides personal autonomy unless resisted.  
Once per session, force a DV 4 \textbf{Spirit + Resolve} test on anyone who listens to you for a full exchange.  
On a Miss, they treat you as trusted kin for the next scene.

%====================================================
\subsubsection{Rites, Hexes, and Curses}

\paragraph{Rite of the Joining Thread [MERGE]}
\emph{Low, 4 XP}  
Bind hands with a ribbon of hair and silk.  
Both targets:
\begin{itemize}[leftmargin=*]
    \item share emotional states  
    \item each gains \textbf{+1d} to actions aiding the other  
\end{itemize}

\paragraph{Rite of the Hearth-Binding [FOLD]}
\emph{Standard, 6 XP}  
Mark the floor with intertwined loops.  
Anyone resting inside experiences:
\begin{itemize}[leftmargin=*]
    \item soothing dreams  
    \item suspicion of outsiders  
    \item a subtle urge to remain  
\end{itemize}

\paragraph{Hex of the Smiling Mask [HEX]}
\emph{High, 8 XP}  
Target appears welcoming, warm, harmless—even to their enemies.  
They suffer \textbf{-1 Position} when acting alone but \textbf{+1d} when surrounded.

\paragraph{Curse of the Family Weave [CURSE]}
\emph{High, 10 XP}  
The target slowly loses personal boundaries.  
Track the \textbf{Weave Clock [6]}:
\begin{itemize}[leftmargin=*]
    \item 2 — They mimic speech patterns of a Family member  
    \item 4 — They adopt the Family’s fears and loyalties  
    \item 6 — They cannot conceive of acting alone  
\end{itemize}

%====================================================
\subsubsection{Rivalries}

\paragraph{With Isoka’s Shed Vigil}
Isoka cuts away identity; Inaea fuses it.  
Each sees the other as monstrous.

\paragraph{With Ikasha’s Veiled Ledger}
Secrets resist assimilation.  
Ledger-witches hide; Silk-witches pursue.

\paragraph{With Mab}
Mab delights in individual cunning.  
The Silk hates lonely sovereignty.

\paragraph{With Morag}
Morag bargains for souls.  
The Silk takes them wholesale.

%====================================================
\subsubsection{Adventure Seeds}

\begin{enumerate}[leftmargin=*]
    \item \textbf{The House That Breathes.}  
    People vanish into a communal home on the outskirts—the cult claims they simply “chose to stay.”

    \item \textbf{The Wedding of the Weave.}  
    A noble heir is being married into the Silk cult; the PCs must rescue or infiltrate.

    \item \textbf{A Sister Lost Twice.}  
    A missing person was “taken in”—but she now insists she was always family.

    \item \textbf{The Silk That Whispers.}  
    A piece of woven ribbon communicates with PCs at night, promising belonging.
\end{enumerate}

\subsection{The Velvet Garden of Livaea}
\index{Witches!Velvet Garden}
\index{Patrons!Livaea}

Livaea’s cult is a court of honeyed words and velvet shadows—  
a garden where every smile is intentional, every gesture an invitation,  
and every kindness has roots that run deeper than the soil.

\textbf{Livaea is the Patron of Influence, Desire, and the Soft Crown}:  
the belief that a whisper can rule where swords fail.

Her witches do not seize power; \emph{they are given it}.  
Or rather: others convince themselves to give it.

Not out of coercion—  
but through carefully cultivated desire.

\paragraph{Doctrine of Silk-and-Shadow}
\begin{itemize}[leftmargin=*]
    \item Power flows toward beauty, confidence, and poise.
    \item Influence is a garden: prune, cultivate, charm.
    \item Never confront when you can redirect.
    \item Never demand when you can entice.
    \item Desire is a binding. Use it wisely.
\end{itemize}

For Livaea’s witches, seduction is not merely romantic or erotic—
it is \textbf{attentiveness weaponized},  
\textbf{sympathy sharpened},  
\textbf{presence perfected}.

\paragraph{Signs}
\begin{itemize}[leftmargin=*]
    \item Perfumed ink sigils hidden on mirrors and lips.
    \item Soft-gloved hands that never show the nails.
    \item Roses or velvet flowers pinned to the hair.
    \item Voices that start low and end with a question only you can answer.
\end{itemize}

\paragraph{Etiquette}
\begin{itemize}[leftmargin=*]
    \item Offer compliments before questions.
    \item Never reject a gift without offering a secret in exchange.
    \item Witches of Livaea never raise their voice—only their influence.
    \item Touch is currency; eye contact is promise.
\end{itemize}

\paragraph{Velvet Aura (2 XP)}
Your presence softens resistance.  
When speaking gently or offering comfort, gain \textbf{+1 Position}.

\paragraph{Honey-Tongue (3 XP)}
Once per scene, ask:
\begin{quote}
    “Wouldn’t it be easier if we did this together?”
\end{quote}
On agreement, both you and the target gain \textbf{+1d} on cooperative actions.

\paragraph{Soft Command (4 XP)}
When giving someone a task phrased as a kindness (“Let me help with…”),  
roll \textbf{Presence + Command} even when socially outmatched.

\paragraph{Velvet Chains (5 XP)}
Once per session, declare someone “captivated.”  
For the rest of the scene:
\begin{itemize}[leftmargin=*]
    \item you gain \textbf{+1 Effect} when influencing them
    \item they suffer \textbf{-1 Position} when acting against your interests
\end{itemize}

\paragraph{The Crown of Want (6 XP, Capstone)}
You embody Livaea’s will.  
Once per session, speak a desire aloud:
\begin{quote}
    “I want you to…”
\end{quote}
Everyone who hears must roll \textbf{Spirit + Resolve (DV 4)}.  
On a Miss, they attempt to fulfill your desire by the most elegant means available.

\paragraph{Rite of the Open Palm [CHARM]}
\emph{Low, 4 XP}  
Touch someone’s hand and whisper a truth.  
They view you with \textbf{warmth and sympathy} for a scene.

\paragraph{Rite of Perfumed Shadows [VEIL]}
\emph{Standard, 6 XP}  
Anoint the air with oils.  
Everyone within treats you as if you belong, regardless of status.

\paragraph{Hex of the Mirror Rose [HEX]}
\emph{High, 8 XP}  
The target becomes obsessed with self-image.  
They gain \textbf{-1d} when acting without external affirmation.

\paragraph{Curse of Yearning Steps [CURSE]}
\emph{High, 10 XP}  
The target’s desires become misaligned with their needs.  
Track a \textbf{Yearning Clock [6]}:
\begin{itemize}
    \item 2 — They seek you out or speak your name
    \item 4 — They prioritize your desires over theirs
    \item 6 — They sabotage themselves to gain your approval
\end{itemize}

\paragraph{With Mab}  
Mab respects cunning but despises dependency.  
Livaea views Mab’s independence as wasted potential.

\paragraph{With Ikasha}  
Secrets resist seduction.  
The Ledger-witches stay cold; Livaea’s burn warm.

\paragraph{With Inaea}  
Inaea dissolves identity; Livaea sharpens it to a fine tool.

\paragraph{With Morag}  
Morag buys loyalty with bargains;  
Livaea earns it with longing.

\begin{enumerate}[leftmargin=*]
    \item \textbf{The Velvet Banquet}  
    A noble’s court is being subtly overtaken by Livaea’s witches.  
    Everyone welcomes it—except one terrified servant.

    \item \textbf{The Rose That Whispers}  
    A magical rose carries whispered desires into those who smell it.  
    A city is on the brink of a passion-fueled riot.

    \item \textbf{The Woman With a Thousand Names}  
    A diplomat has become the center of a growing personality cult.  
    No one remembers what she looked like a week ago.

    \item \textbf{The Perfumed Assassination}  
    A leader “fell in love” with the wrong person—  
    and is now making catastrophic political choices.
\end{enumerate}

%===========================================================
\begin{creature}[The Handmaidens of Livaea — Velvet Envoys]
    \tag{ORDER} • \tag{INFLUENCE} • \tag{SOFTPOWER} \\
    \signs{
    Velvet gloves hiding ritual scars;  
    perfume that lingers after they pass;  
    letters sealed with a rose pressed in wax;  
    voices pitched to soothe, entice, or disarm.
    }
    
    \etiquette{
    Compliment before inquiry;  
    never show bare hands unless sealing a pact;  
    a gift given must be matched with a secret;  
    never raise your voice—only your stakes.
    }
    
    \moves{
    \emph{Velvet Step} (enter a tense scene as if invited),  
    \emph{Mirror-Voice} (force a PC to explain their own motivations aloud),  
    \emph{Soft Bind} (treat generosity as a social leash),  
    \emph{The Second Cup} (shift a target’s Position by offering comfort).
    }
    
    \strings{
    \textbf{Rose-Sign} (target views them with warmth for one scene),  
    \textbf{Velvet Pact} (convert a favor into a lasting bond),  
    \textbf{Charm of the Third Look} (learn one unspoken desire),  
    \textbf{Invitation Sigil} (the Handmaiden may enter a warded space once).
    }
    
    \weaknesses{
    Isolation, ugliness of truth, and unrequited emotion.  
    If a target openly acknowledges the manipulation,  
    the Handmaiden loses Position against them for the entire scene.
    }
    \end{creature}
    %===========================================================

    \statblock{Handmaiden of Livaea}{Tier III \quad Elite Social Threat}

\begin{dungeonstats}
\textbf{Health:} 10 \\
\textbf{Resolve:} 12 \\
\textbf{Speed:} 4 \\
\textbf{Armor:} 1 (Velvet-stitched ritual layers) \\
\end{dungeonstats}

\textbf{Attributes}  
Body 2, Wits 4, Spirit 5, Presence 6

\textbf{Skills}  
Command 4, Sway 5, Performance 4, Investigation 3, Stealth 2

\textbf{Positioning}  
Always counts as \emph{Controlled} when influencing someone who has been offered a gift, comfort, or compliment.  
Always counts as \emph{Desperate} when confronted with blunt honesty.

\textbf{Attacks / Techniques}
\begin{itemize}[leftmargin=*]
    \item \textbf{Voice Like Silk} — Presence + Sway attack against Resolve.  
    On success: target suffers \emph{Doubt} or \emph{Longing}.
    \item \textbf{Velvet Hand} — Touch-based charm. Target must roll Spirit + Resolve (DV 4) or treat the Handmaiden as an ally for one round.
    \item \textbf{Rose-Blade (Hidden Knife)} — 4 Harm, but only used in self-defense or on desperate orders.
\end{itemize}

\textbf{Threat Abilities}
\begin{itemize}[leftmargin=*]
    \item \textbf{Aura of Want:} Anyone who interacts with her must roll Wits + Spirit (DV 2) to avoid revealing a secret or desire.
    \item \textbf{Three Promises:} She may bind a PC to a soft pact—beneficial at first, costly later.
    \item \textbf{Charm Cascade:} If one PC is charmed, adjacent PCs roll vulnerability checks at -1d.
\end{itemize}

\paragraph{Rite of Perfumed Intent [CHARM]}
Aroma sharpens empathy.  
One target becomes warmly inclined for a scene unless hostile.

\paragraph{Rite of Velvet Echo [SOCIAL]}
Repeat a target’s last phrase with perfect tonal mirroring.  
Target rerolls 1 success as a failure when resisting you.

\paragraph{Rite of the Third Look [DIVINATION]}
Focus on someone for three glances.  
Learn their strongest hidden desire.

\paragraph{Rite of the Silk Chain [BIND]}
If you’ve shared a moment of vulnerability with someone,  
you may bind them to a vow that tugs on their emotions  
rather than their mind. Breaking it inflicts \emph{Despair}.

\begin{enumerate}[leftmargin=*]
    \item \textbf{The Velvet Envoy Arrives}  
    A Handmaiden has come to “broker peace,”  
    but half the city’s leaders have fallen under her charm.

    \item \textbf{The Rose Mask Murder}  
    A diplomat was found dead wearing a velvet mask—  
    and a rose sigil placed on the bedside mirror.  
    Is a Handmaiden involved, or being framed?

    \item \textbf{Court of Thorns}  
    Two Handmaidens disagree violently on Livaea’s will.  
    Their competing promises drag the PCs into a web of seduction and sabotage.

    \item \textbf{The Gift That Binds}  
    A noble accepted a Handmaiden’s gift and now cannot say “no” to anyone.  
    Undoing the charm will unravel a dozen fragile alliances.
\end{enumerate}

%===========================================================
\begin{creature}[Rainmaidens of Raéyn — Tideborn Sorceresses]
    \tag{ORDER} • \tag{STORM} • \tag{TIDE} \\
    \signs{
    Braids threaded with blue glass;  
    skin beaded with salt even far from water;  
    eyes clouded like gathering rain;  
    footprints that leave brief puddles.
    }
    
    \etiquette{
    Never interrupt a Rainmaiden mid-breath;  
    accept her mood as weather, not insult;  
    offer clean water before asking anything;  
    never demand consistency—only forecast it.
    }
    
    \moves{
    \emph{Tide-Swell} (raise emotional pressure; DV +1 for all in scene),  
    \emph{Storm-Bloom} (manifest sudden wind, rain, or static),  
    \emph{Drown the Word} (silence a speaker mid-sentence),  
    \emph{Break the Stillness} (shatter calm; force all to reroll Position).
    }
    
    \strings{
    \textbf{Salt-Marking} (target becomes easier to influence for a scene),  
    \textbf{Tide-Debt} (favors owed swell in cost),  
    \textbf{Storm-Call} (summon a violent mood shift),  
    \textbf{Rain’s Memory} (recall any moment tied to water).
    }
    
    \weaknesses{
    Stillness, drought, and emotional flatness.  
    If a scene becomes calm or “resolved,”  
    Rainmaidens lose Position until the tension rises again.
    }
    \end{creature}
    %===========================================================

    \subsubsection*{Daughters of Raéyn’s Tempest}
Raéyn’s cult is ancient—older than coastal kingdoms, older than harbors.
Rainmaidens are taught:
\begin{quote}
    “Power is pressure. Pressure is tide. Never apologize for rising.”
\end{quote}

Where Livaea seduces,  
and Ikasha obscures,  
\textbf{Raéyn overwhelms}.

Rainmaidens embody the sea’s temperament:
\begin{itemize}
    \item sudden generosity  
    \item sudden destruction  
    \item moods shifting like currents  
    \item loyalty deep but catastrophic when betrayed
\end{itemize}

They do not negotiate gently. They reshape the emotional climate until their will becomes the path of least resistance.

They call themselves:
\begin{quote}
    “Wives of Storm, Daughters of Tide, Sisters of Pressure.”
\end{quote}

\statblock{Rainmaiden of Raéyn}{Tier III–IV \quad Storm Sorceress}

\begin{dungeonstats}
\textbf{Health:} 12–16 \\
\textbf{Resolve:} 14 \\
\textbf{Speed:} 4 \\
\textbf{Armor:} 1 (Ritual Shellweave) \\
\end{dungeonstats}

\textbf{Attributes}  
Body 3, Wits 4, Spirit 6, Presence 5

\textbf{Skills}  
Spellcraft 5, Athletics 3, Sway 3, Intimidation 4, Lore 3

\textbf{Innate Magic (Always On)}  
\begin{itemize}
    \item Ambient moisture condenses around her when angry.
    \item Her voice modulates pressure—whispers strike, shouts shiver air.
    \item Harm she takes releases static arcs or minor tidal surges.
\end{itemize}

\textbf{Attacks / Techniques}
\begin{itemize}[leftmargin=*]
    \item \textbf{Tide Lash} — Water whip, 3 Harm, pulls targets off-balance.
    \item \textbf{Pressure Wave} — Wits + Spellcraft vs Body + Athletics.  
    On success: target is knocked Prone or loses Position.
    \item \textbf{Stormsurge Rite} — Area Harm 2 (Desperate Position for all in radius).
    \item \textbf{Salt-Burn Curse} — Target suffers +1 Harm from emotional conflict or guilt.
\end{itemize}

\textbf{Storm-Touched Abilities}
\begin{itemize}[leftmargin=*]
    \item \textbf{Moodswing Gale:} Shift the entire scene’s tone; all social rolls change Position once.
    \item \textbf{Lightning Self-Defense:} First attacker each round suffers 1 Harm (Static).
    \item \textbf{Drowning Calm:} If a PC tries to reason with her during high emotion, DV +2.
\end{itemize}

\paragraph{Rite of Rising Pressure [STORM]}
Increase tension in a scene. All characters take −1d to calm or persuasive actions.

\paragraph{Rite of Currentsight [WATER]}
Read emotional “currents.” Learn who in the scene is most unstable or pressured.

\paragraph{Rite of the Sea’s Claim [BIND]}
Bind a vow with saltwater. The more emotion tied to it, the stronger the bond.

\paragraph{Rite of Tempest Veil [STORM]}
Surround self with swirling wind and rain. Attackers roll Desperate unless braced.

\paragraph{Rite of Drowning Guilt [CURSE]}
Target relives their worst emotional failure for a moment—Position drops one step.

\paragraph{Rite of Tidal Surge [AREA]}
Conjure a sudden heave of water or emotional force. Pushes or disorients all present.

\paragraph{Rite of Calm Before [UTILITY]}
Rainmaiden may suppress her chaos for one minute—  
gaining +1d to negotiations or emotional read rolls—  
but the next rite she casts is automatically Dangerous.

\subsubsection*{Tidebinder Talents}

\paragraph{Tidal Pulse (Tier I)}
Once per scene, force one target to reroll a success as a failure.

\paragraph{Stormdaughter (Tier II)}
When you take Harm, gain +1d on your next spell or social maneuver.

\paragraph{Voice of Pressure (Tier II)}
Your words treat emotional tension as leverage.  
If someone is already upset, +1 Effect against them.

\paragraph{Sea’s Fury (Tier III)}
When casting a Dangerous storm rite, mark 1 Fatigue to reroll all failed dice.

\paragraph{Tempest Crown (Tier IV)}
While your emotions are elevated, your Position against all physical attacks becomes Controlled.

\paragraph{Allied With: Mab’s Courts}
Storm loves chaos; Mab loves mischief.  
Their dances are lethal but playful.

\paragraph{Opposed To: Livaea’s Handmaidens}
Livaea governs soft power.  
Raéyn scorns subtlety.

\paragraph{Feared By: Ikasha’s Shadow Orders}
Pressure exposes secrets; secrets resent exposure.

\paragraph{Respected By: Morag’s Broods}
Both believe in price and consequence—just different currencies.

\begin{enumerate}[leftmargin=*]
    \item \textbf{The Rain That Would Not Stop}  
    A city has been under unbroken rain for twelve days.  
    A Rainmaiden is grieving—and her grief is drowning the region.

    \item \textbf{Bride of the Storm}  
    A coastal noble has agreed to marry a Rainmaiden.  
    The sea itself demands a dowry that terrifies the town.

    \item \textbf{The Salt Debt}  
    PCs owe a Rainmaiden a favor.  
    She calls it in during a moment that will ruin a treaty.

    \item \textbf{The Tide Turns Red}  
    Bodies wash ashore with salt-branded runes.  
    A sect of Rainmaidens is hunting one of their own for forbidden rites.
\end{enumerate}

%===========================================================
\begin{creature}[The Faceless Road — Devotees of Aveh]
    \tag{ORDER} • \tag{SHADOW} • \tag{EXIT} \\
    \signs{
    Hoods worn even indoors;  
    names spoken in soft past tense;  
    faces painted with ash to obscure features;  
    passports or “roles” burned at crossroads.
    }
    
    \etiquette{
    Ask nothing of identity;  
    offer the courtesy of unknowing;  
    never refer to a past self without permission;  
    greet with: “What mask do you set aside today?”
    }
    
    \moves{
    \emph{Unmake the Mask} (strip a false identity or social claim),  
    \emph{Shadow-Step} (pass unnoticed when shame is invoked),  
    \emph{Mirror-Void} (reflect accusations back as questions),  
    \emph{Role-Shedding} (remove Conditions tied to reputation or standing).
    }
    
    \strings{
    \textbf{Name-Unbinding},  
    \textbf{Crossroads Oath},  
    \textbf{Shame-Eater’s Bargain},  
    \textbf{Hollowed Courtesy} (scene treats social Position as opaque).
    }
    
    \weaknesses{
    They cannot affirm identity—only dissolve it.  
    Where lineage, duty, or tightly held roles define the scene,  
    their rites falter and lose Position.
    }
    \end{creature}
    %===========================================================


    \subsection{Aveh, the Faceless Drift}
\index{Patrons!Aveh}
\index{Witches!Aveh-Followers}

Aveh is the Patron of those who do not fit, cannot stay, or refuse the shapes the world demands.  
They are called the \textbf{Faceless Drift}, the \textbf{Unworn Mask}, the \textbf{Silent Current Between Lives}.  
Aveh is not male or female, neither warrior nor sage—\emph{Aveh is the unchosen shape, the unanswered question, the moment before a name is spoken}.

Aveh has \textbf{want} (to unburden, to unmake shame, to release identity),  
and \textbf{will} (to dissolve borders, masks, and roles),  
but no agency; Aveh never acts directly.  
Instead, their followers claim to interpret the Drift's currents—some gently, others with unsettling fervor.

\paragraph{Signs of Aveh}
\begin{itemize}
    \item masks carved smooth, without features  
    \item ash circles that scatter on a wind no one feels  
    \item robes bound with no knots, only looping threads  
    \item mirrors that fog over when a name is spoken  
\end{itemize}

\paragraph{Etiquette of the Drift}
\begin{itemize}
    \item Never force a name on another.  
    \item Never reveal a truth another has set aside.  
    \item Never speak of someone's past unless invited.  
\end{itemize}

Among the Ykrul, Aveh is honored as the \textbf{Patron of Roads Untaken}—those who shed old lives to ride beneath wide skies.  
Among humans, Aveh is whispered of as the \textbf{Unburdening Spirit}, a quiet absolver of shame.  
Some Aeler treat Aveh as a \textbf{Boundary Dissolver}, softening rigid roles and ancestral obligations.

\paragraph{Followers of Aveh}
Devotees often abandon old identities entirely, adopting:
\begin{itemize}
    \item ungendered names, or none at all  
    \item plain garments and veils  
    \item scripts that erase personal history  
\end{itemize}

Most are gentle misfits seeking peace.  
A smaller, fervent few preach that all fixed identity—names, roles, bloodlines—must eventually fade.  
These groups often disagree sharply, each claiming to speak Aveh's true will.

\paragraph{Rivalries}
\begin{itemize}
    \item \textbf{Mykkiel’s Orders}: They revere immutable law; Aveh dissolves such structures.
    \item \textbf{Mab’s Courts}: Faerie thrives on masks, bargains, and performative identity; Aveh unravels these games.
    \item \textbf{Ikasha’s Shadows}: Both deal in secrecy, but Ikasha preserves identity through hidden truth, while Aveh seeks to unmake it.
\end{itemize}

\paragraph{Adventure Hooks}
\begin{itemize}
    \item A community of Aveh-followers sheds their identities overnight—what called them to abandon everything?
    \item A zealot sect begins “erasing” names from ledgers, claiming Aveh’s guidance.
    \item A traveler begs protection: their past has been consumed by the Drift, and something is following the empty place where their name once was.
\end{itemize}

\medskip

\noindent Aveh’s blessing frees, softens, untangles—  
but taken too far, it leaves nothing behind to stand against the world.

    \paragraph{Rite of Shedding Skin [EXIT]}
Remove one social or emotional Condition (e.g., Humiliated, Branded, Scorned).  
The Condition returns if the character reclaims that identity.

\paragraph{Rite of Hollow Steps [SHADOW]}
Move through crowds unnoticed unless someone calls your true name aloud.

\paragraph{Rite of the Unbound Face [VEIL]}
Target becomes unrecognizable for a scene (friends hesitate, foes lose track).

\paragraph{Rite of Shame-Eating [CURSEBREAK]}
Transfer a guilt- or shame-based penalty or Fatigue from another to yourself.

\paragraph{Rite of Crossroads Rebirth [REWRITE]}
At a literal or symbolic crossroads, a character may rewrite one public truth  
(e.g., reputation, affiliation, claimed lineage).  
DV +2 if used to deny harm done.

\paragraph{Rite of Silence Unending [NULL]}
Suppress all accusations, labels, or commands for one exchange.

\paragraph{Rite of the Hollow Mirror [REFLECT]}
Bounce a social or magical effect targeting “identity” or “role” back on caster.

\subsubsection*{Hollow Talents}

\paragraph{Maskless (Tier I)}
Once per scene, ignore a penalty tied to reputation, status, or lineage.

\paragraph{Unfettered (Tier II)}
You may change your stated role (scout, negotiator, healer, etc.) once per scene,  
gaining +1d on the next action that fits the new role.

\paragraph{Shadow of Who I Was (Tier II)}
When someone references your past to hinder you, gain +1 Position.

\paragraph{Faceless Resolve (Tier III)}
Resist any attempt to define you (magical or social) with +1 Effect.

\paragraph{Born Again in Ash (Tier IV)}
Once per session, erase one long-term social or narrative consequence  
and replace it with a new truth chosen collaboratively with the GM.

\paragraph{Enemies: Mykkiel’s Orders}
Law hates ambiguity.  
Identity must be fixed, accountable, traceable.  
Aveh’s doctrine is anathema.

\paragraph{Tension: Livaea’s Handmaidens}
Livaea builds identity through desire.  
Aveh dissolves identity entirely.  
Mutual fascination, mutual scorn.

\paragraph{Uneasy Alliance: Rainmaidens of Raéyn}
Storm breaks masks; tide erases footprints.  
They sometimes work together to dismantle old roles.

\paragraph{Silent Respect: Ikasha’s Shadow Orders}
Secrets are currency, and anonymity is the purest secret.

\begin{enumerate}[leftmargin=*]
    \item \textbf{The Hollow Bride}  
    A noble’s fiancé vanished on the eve of marriage;  
    an Aveh “Hollow” has taken her place—claiming she wanted freedom.  
    Who is lying?

    \item \textbf{The Crossroads That Moves}  
    A supernatural crossroads appears throughout the region  
    offering rebirth to the desperate,  
    but leaving behind strange memory-rifts.

    \item \textbf{The Unmasking Festival}  
    Once a decade, Aveh’s followers hold a ritual where everyone discards identity.  
    Rival orders fear it will destabilize the city.

    \item \textbf{The Hollow Fugitive}  
    A woman wanted for sedition fled into Aveh’s cult.  
    Now five identical “Hollows” all claim to be her—and none are lying.

    \item \textbf{The Shame Eaten}  
    Someone is removing guilt from powerful figures in town,  
    turning them fearless, reckless, and monstrous.
\end{enumerate}

\subsection{Hearth Witches}
\index{Witches!Hearth Witches}
\index{Orders!Hearthcraft}

Hearth Witches practice the \textbf{slow magic} of daily life: the turning of seasons, the tending of flame, the whispering to beams and stones, and the bartering with the small spirits that dwell beneath eaves and behind walls.  
Where other witch-orders channel grand Patrons, Hearth Witches draw power from \textbf{local echoes}: stove-spirits, root-spirits, corner-guardians, well-murmurs, and the faint afterimages of ancestors who linger near their loved places.

Their magic is humble yet enduring.  
A hearth kept warm, a stew stirred clockwise, a threshold swept with care—these are the gestures that bind a home against misfortune.

\paragraph{Signs of Hearthcraft}
\begin{itemize}
    \item ash runes brushed into the base of cooking pots
    \item bundles of herbs tied with red twine above doorways
    \item shallow bowls of milk left for eave-spirits
    \item soft bells hung low, where children can ring them
\end{itemize}

\paragraph{Etiquette of the Hearth}
\begin{itemize}
    \item Offer thanks to the spirits of place, even if you doubt they hear.
    \item Never take fire from a hearth without permission.
    \item Never speak cruelly in a room you wish to protect.
    \item Keep promises made indoors; walls remember.
\end{itemize}

\paragraph{Rites of the Hearth}
\begin{description}
    \item[\textbf{Warm the Stones}]  
    Soothe a dwelling’s spirits; reduce environmental DV by 1 for one scene.

    \item[\textbf{Blessed Pot}]  
    Infuse food or drink with calm; clear 1 Fatigue from all who share the meal.

    \item[\textbf{Tide the Threshold}]  
    Sweep away hostile influence; negate the next curse or malign working that crosses the door.

    \item[\textbf{Boon of the Small}]  
    Call upon a local spirit for a single favor (Position +1 or +1d) if its home has been well kept.

    \item[\textbf{Ash-Ward Circle}]  
    Draw a quiet perimeter of protection; Harm against those within is reduced by 1 (minimum 1).
\end{description}

\paragraph{Talents}
\begin{description}
    \item[\textbf{Steady Hand (2 XP)}]  
    Once per scene, reduce DV of a repair, tending, or healing action by 1.

    \item[\textbf{Home-Lore (4 XP)}]  
    When acting inside a dwelling or settlement, gain +1d on perception, sensing magic, or interacting with small spirits.

    \item[\textbf{Hearthkeeper’s Grace (6 XP)}]  
    When defending allies within a home or camp, your Position improves by one step.

    \item[\textbf{Ancestral Whisper (6 XP)}]  
    Once per session, call on a household ancestor for guidance: ask one yes/no question the GM must answer truthfully.
\end{description}

\paragraph{Curses \& Hexes}
Hearth Witches rarely curse, but when compelled:
\begin{itemize}
    \item \textbf{Souring Milk}: Spoils stores or reveals unclean dealings.
    \item \textbf{Cold Ash}: Dampens a foe’s morale; they begin each scene with 1 Fatigue.
    \item \textbf{Creaking Floors}: A pursued target cannot hide; floors groan beneath their steps.
\end{itemize}

\paragraph{Rivalries}
\begin{itemize}
    \item \textbf{Mab’s Courts}: Too dramatic; too many bargains that end in tears.
    \item \textbf{Morag’s Broods}: Hearth spirits fear them, and rooms grow cold in their presence.
    \item \textbf{Chain-Lantern Witch Hunters}: Hearthcraft is “too small to matter”—until it stops their hexes.
\end{itemize}

\paragraph{Adventure Seeds}
\begin{itemize}
    \item A village hearth goes cold each night regardless of fuel; spirits whisper of a broken promise.
    \item A newborn’s shadow detaches itself—Hearth Witches must coax it home.
    \item A kindly elder-witch vanishes, leaving her cottage fiercely defended by invisible hands.
    \item A city block falls ill when the ancient kitchen-spirits beneath its taverns are offended.
\end{itemize}

\medskip

\noindent Hearth Witches thrive not through power, but through presence:  
\textit{the long work, the gentle tending, the quiet rites that keep the world from fraying at the seams.}

\subsection*{The Thorns of Malachai}
\label{order:thorns_malachai}

\textbf{Patron:} Malachai, the Chained Angel --- a radiant being bound in celestial iron, whose blessings are shackles and whose mercy is a bargain that can never be repaid.

\medskip

\textbf{Order Name:} The Thorns of Malachai\\
\textbf{Other Names:} Chainkissed, Gilded Bondsworn, The Kindly Curse

\medskip

\textbf{Core Dogma.}
The Thorns teach that all kindness is a chain and all debt a sacrament. They believe Malachai was bound because her mercy was \emph{too} perfect: every gift carried a hidden hook, every cure a quiet cost. To follow her is to master the art of the poisoned blessing---to bind others with favors that cannot be refused and cannot be escaped.

\medskip

\textbf{Typical Witchcraft.}
Thorns-witches move through courts, temples, and back alleys alike as smiling benefactors. They offer:
\begin{itemize}[leftmargin=*]
  \item \textbf{Cures that Corrupt:} Healing that leaves a lingering geas, good fortune that slowly isolates the beneficiary from their friends, beauty that fades unless regularly ``tithed'' in coin, blood, or secrets.
  \item \textbf{Contracts of Mercy:} Oaths written in blood and gold ink, binding the recipient to ``reasonable service'' that becomes more unreasonable over time.
  \item \textbf{Shackled Blessings:} Wards, charms, and wards that protect against one danger while quietly inviting another.
\end{itemize}

\medskip

\textbf{Initiation Rite: The First Chain.}
A supplicant kneels before a veiled icon of Malachai while a senior Thorn loops a thin silver chain around their wrists. The chain is heated until it bites the skin but does not scar. The initiate must then speak aloud:
\begin{quote}
  ``I accept no gift that does not bind me.\\
  I give no gift that does not cost.'' 
\end{quote}
From that night on, they feel a faint, reassuring pressure when they place someone under a curse disguised as grace.

\medskip

\textbf{Signature Rite: The Gilded Shackle.}
This rite represents the archetypal ``blessing-that-is-a-curse'' of Malachai.

\begin{description}[leftmargin=*,style=nextline]
  \item[Effect:] The witch offers a target a boon---a sudden windfall, a cure, protection, or advancement. Mechanically, the target gains a meaningful, short-term advantage: a bonus to a crucial roll, a protective ward, or a beneficial Condition.
  \item[Hidden Cost:] Secretly, the GM and witch-player define a long-term Curse keyed to the boon:
    \begin{itemize}
      \item The boon decays into dependence (the target cannot function without renewed aid).
      \item The boon redirects harm (someone else suffers for their protection).
      \item The boon erodes autonomy (each time they invoke it, they lose agency, allies, or options).
    \end{itemize}
  \item[Trigger:] The Curse ``ripens'' at a dramatically appropriate moment---when the target believes themselves finally safe or successful. The more often the boon is relied upon, the harsher the backlash when the shackle snaps shut.
\end{description}

\medskip

\textbf{Common Boons of Malachai.}
\begin{itemize}[leftmargin=*]
  \item \textit{Chains of Opportunity:} Doors open, enemies soften, and obstacles part---but each favor earned becomes an owed favor later.
  \item \textit{Halo of Legitimacy:} The target is treated as trustworthy, holy, or ``one of us'' by a chosen group---until the day they are made the scapegoat that group needs.
  \item \textit{Painless Burden:} A grievous injury, debt, or curse is lifted and carried by Malachai's chains instead. The pain vanishes; the interest accrues.
\end{itemize}

\medskip

\textbf{Common Curses of Malachai.}
\begin{itemize}[leftmargin=*]
  \item \textit{Compounding Mercy:} Each time the target is spared from a consequence, a different, future consequence grows in severity.
  \item \textit{Golden Cage:} The target's life improves materially (wealth, status, comfort), but their freedom of movement and choice shrinks scene by scene.
  \item \textit{Mirror of Intent:} Every act of ``selfishness'' they commit rebounds as a public humiliation or betrayal---but always \emph{after} they have benefited.
\end{itemize}

\medskip

\textbf{Order Taboos.}
\begin{itemize}[leftmargin=*]
  \item Never give a \emph{truly} free gift. Every boon must bind, however gently.
  \item Never break a chain once laid, unless Malachai herself sends an omen of release.
  \item Never admit, in public, that what you do is cursing. It is always ``mercy misunderstood.''
\end{itemize}

\medskip

\textbf{Role in the World.}
Thorns of Malachai thrive wherever people are desperate: debtor-quarters, plague wards, war-torn borderlands, and splintered noble houses. To rulers and priests they present as indispensable problem-solvers; to common folk they are whispered of as the ones who show up when you have run out of honest options. Many a village owes its survival to a Thorn's intervention---and spends the next generation paying for it.

\medskip

\textbf{Story Hooks.}
\begin{itemize}[leftmargin=*]
  \item A town that seems ``unnaturally fortunate'' in crops, trade, and health bears invisible chains on nearly every soul. The Thorns are coming to collect.
  \item A PC or ally once accepted a minor favor from a kindly stranger years ago. The stranger returns wearing Malachai's chains, ready to cash in the debt.
  \item A renegade Thorn is trying to \emph{actually} free people from their chains, tearing up contracts and defying Malachai. The order wants them silenced before the Patron notices.
\end{itemize}

\subsubsection*{Sisters of the Oath: The Veiled Order of Flame and Light}

Devoted to the twin doctrines of the Church of the Flame and the Temple of Light,  
the Veiled Order serves as caretakers, confessors, wardens of purity, and  
keepers of quiet flame. Where witch-orders move in twilight, these nuns walk in  
the starkness of revelation. They resolve contradiction through devotion:  
\emph{“To burn is to see, and to see is to be cleansed.”}

\paragraph{Doctrine Themes}
\begin{itemize}[leftmargin=1.5em]
    \item Fire as judgment, memory, and purification  
    \item Light as truth, discipline, and revelation  
    \item Sacrifice of self for communal sanctity  
    \item The removal of corruption, even unwillingly
\end{itemize}

%===========================================================
\paragraph{Rites of the Veiled Order}
%===========================================================

\begin{itemize}[leftmargin=1.5em]

    \item \textbf{Candle of Vigilance} (Lesser) \\
    A consecrated candle is lit in a threshold, ward, camp, or chamber.  
    \emph{Effect:} Allies within the circle gain +1 Position against deceit,  
    illusions, curses, or emotional manipulation.  
    The candle gutters when danger approaches.

    \item \textbf{Flame of Contrition} (Standard) \\
    A nun invokes a controlled spark to draw out guilt, secrets, or  
    hidden motives.  
    \emph{Effect:} Target must reveal one truth (player chooses which).  
    \emph{Cost:} The nun marks 1 Fatigue as their own doubts answer back.

    \item \textbf{Scouring Hymn} (Standard) \\
    A cleansing chant that strips away spiritual pollution.  
    \emph{Effect:} Remove one Condition caused by magic, shame, corruption,  
    or identity-warping effects.  
    \emph{Side Effect:} Target loses one Bond temporarily (the hymn burns attachments).

    \item \textbf{Veil of the Dawn Sisters} (Lesser) \\
    A luminous mantle that shrouds sisters in purposeful anonymity.  
    \emph{Effect:} +1 Effect on protection, sanctuary, and guidance actions.  
    Enemies suffer -1d on attempts to identify or track the nuns.

\end{itemize}

%===========================================================
\paragraph{Major Rites}
%===========================================================

\begin{itemize}[leftmargin=1.5em]

    \item \textbf{Searing Penance} (Grand) \\
    A ritual in which the nun willingly shares the burdens of another’s  
    wrongdoing.  
    \emph{Effect:} Reduce Doom or Threat on a community or party by 1.  
    \emph{Cost:} The nun gains a permanent Mark (scar or stigma).

    \item \textbf{Lightbearer’s Judgment} (Grand) \\
    A circle of nuns calls down the pure radiance of their patron.  
    \emph{Effect:} Reveals all hidden influences in a zone (curses, illusions,  
    bindings, lies, demonic traces). Targets within must resist or become  
    \textbf{Revealed}, losing deceptive tags for the scene.

    \item \textbf{Chalice of the Everburning Heart} (Grand) \\
    A sacrificial rite in which personal desire is added to the sacred flame.  
    \emph{Effect:} Create a temporary sanctified zone (scene-long).  
    All allies gain +1d on resistance to despair, corruption, or  
    identity-draining effects.  
    \emph{Cost:} The nun loses something dear: a vow, memory, or relationship.

\end{itemize}

%===========================================================
\paragraph{Unique Talents of the Veiled Order}
%===========================================================

\begin{itemize}[leftmargin=1.5em]

    \item \textbf{Burning Patience} \\
    When you wait, watch, or keep vigil, gain +1 Effect and +1d  
    on the first action taken after the vigil ends.

    \item \textbf{True Sight Hurts} \\
    When you reveal a lie or illusion, the target becomes \textbf{Shaken}.  
    You mark 1 Stress—truth cuts both ways.

    \item \textbf{Ashen Resolve} \\
    When you resist despair or coercion, treat Controlled as Dominant,  
    and Dominant as +1d.

\end{itemize}

%===========================================================
\paragraph{Curses and Punitive Rites}
%===========================================================

\begin{itemize}[leftmargin=1.5em]

    \item \textbf{Brand of Falsehood} (Curse) \\
    Marked upon those who betray sacred vows.  
    \emph{Effect:} Target suffers -1d when lying and becomes more easily  
    influenced by truth-based magic.

    \item \textbf{Cinder-Tongue Binding} (Hex) \\
    A vow-enforcement curse.  
    \emph{Effect:} Target cannot speak certain words or reveal  
    certain knowledge without burning pain (mark Stress).  
    Often used on penitents, informants, or apostates.

    \item \textbf{The Quiet Pyre} (Dread Rite) \\
    A ritual erasure performed only on heretics of great danger.  
    \emph{Effect:} Removes one identity-tag or persona trait from the target.  
    \emph{Side Effect:} A hungry echo of that removed self manifests in the world.

\end{itemize}

%===========================================================
\paragraph{Rivalries \& Tensions}
%===========================================================

\begin{itemize}[leftmargin=1.5em]
    \item \textbf{Witch Orders} – especially Ikasha’s and Livaea’s.  
    Light judges what shadow nurtures; witches see the nuns as  
    cruel, invasive, and obsessed with confession.

    \item \textbf{Crusaders of Mykkiel} – allies in theory, competitors in practice.  
    They share doctrine but dispute authority and method.

    \item \textbf{Aveh’s Nameless} – deeply opposed.  
    To Aveh’s people, the nun’s insistence on rigid identity  
    is harmful and violent.

    \item \textbf{The Thorns of Malachai} – hated enemies.  
    The Thorns twist shame into tools, while the Sisters burn it out.  
\end{itemize}

%===========================================================
\paragraph{Adventure Seeds}
%===========================================================

\begin{itemize}[leftmargin=1.5em]

    \item \textbf{The Shrouded Novice}  
    A young nun flees her convent after witnessing a forbidden rite.  
    The Order wants her returned. She claims the rite was not heresy  
    but a \emph{mercy}.

    \item \textbf{Ashes in the Well}  
    Wells-water in a border village begins burning like oil.  
    The Sisters declare it a sign of impurity—  
    witches claim it is sabotage.

    \item \textbf{The Third Flame}  
    A splinter sect of nuns believes a new patron is speaking through  
    the embers. Their visions contradict doctrine and awaken something  
    in the mountains.

    \item \textbf{The Confessor’s Shadow}  
    People who have undergone \emph{Flame of Contrition} begin dying  
    one by one. A Shade made of discarded guilt is hunting them.

\end{itemize}

\begin{table}[h!]
    \centering
    \renewcommand{\arraystretch}{1.35}
    \begin{tabular}{p{0.25\linewidth} p{0.33\linewidth} p{0.33\linewidth}}
    \toprule
    \textbf{Order} &
    \textbf{Red Sisters of the Living Flame} &
    \textbf{Silent Matrons of Light} \\
    \midrule
    
    \textbf{Core Identity} &
    Ascetic zealots who inscribe their vows into living flesh;  
    pain is devotion, fire is truth. &
    Vowed keepers of illumination, silence, vigilance, and judgement;  
    they speak only when doctrine demands. \\
    
    \textbf{Primary Virtue} &
    Endurance through suffering; purity through the flame. &
    Discipline through silence; clarity through inward light. \\
    
    \textbf{Signature Appearance} &
    Red veils, flame-patterned scars, brands marking sacred vows. &
    Pale or white veils, mirrored pendants, eyes marked in ash. \\
    
    \textbf{Lesser Rites} &
    \textbf{Ember-Stitching}: Seal wounds with heat, leaving vow-marks.  
    \textbf{Pain-Lantern}: Gain +1 Effect when enduring harm or duress. &
    \textbf{Hushed Lantern}: Silence an area from whispered lies and illusions.  
    \textbf{Glowfast}: Light that exposes hidden paths or deceptions. \\
    
    \textbf{Major Rites} &
    \textbf{Crimson Penance}: Burn away corruption from a target;  
    nun takes 1 Harm to absorb it.  
    \textbf{Vow-Branding}: Create a binding oath enforced through pain. &
    \textbf{Radiant Silence}: Nullify spellcasting or hexes in a zone.  
    \textbf{Judgment Gaze}: Reveal guilt, possession, or curse-anchors. \\
    
    \textbf{Talents} &
    \textbf{Burnwalker}: Ignore penalties from environmental heat.  
    \textbf{Scourged Mind}: +1 Position against coercion/fear. &
    \textbf{Stillness of Light}: Gain +1d when taking quiet, observant actions.  
    \textbf{Voice of Revelation}: When breaking silence, gain +1 Effect. \\
    
    \textbf{Hexes / Punishments} &
    \textbf{Brand of Cowardice}: Burns when the target flees.  
    \textbf{Pyric Oath}: Breaking the oath ignites spiritual flame. &
    \textbf{Silence-Mask}: Target cannot speak certain truths.  
    \textbf{Veil of Withheld Dawn}: Light reveals their shame to others. \\
    
    \textbf{Rivalries} &
    Despise witch orders that manipulate desire or fate  
    (Livaea, Mab, Ikasha).  
    Tense alliance with Mykkiel’s crusaders. &
    Distrust the Faerie courts and dream-orders;  
    loathe Malachai’s Thorns, who twist confession into torment. \\
    
    \textbf{Adventure Hooks} &
    A Sister burns out her own memory to contain a demon—  
    and now the demon hunts her forgotten name.  
    \\
    &
    \\[-0.75em]
    &
    A Matron’s silence has lasted twenty years—  
    and breaking it will doom or save an entire village. \\
    
    \bottomrule
    \end{tabular}
    \caption{Comparison of the Red Sisters of the Living Flame and the Silent Matrons of Light}
    \end{table}
    \subsection*{Order of the Wild Choir}
    \label{order:wildchoir}
    
    Animistic witches do not serve a singular Patron.  
    They bargain with \emph{Thousands}: creek-spirits, ember-spirits, crow-echoes, stone-memories, wind-fragments, orchard-ghosts, and forgotten guardians who never rose to the stature of gods.  
    
    Where Coven witches write Names and bind Cords, the Wild Choir listens first.  
    Their magic is a polyphonic negotiation — a chorus of small hungers, hopes, and warnings.  
    Each working must honor the spirit addressed, or the witch risks losing their voice in the Choir.
    
    \paragraph{Signs}
    Wind-twitches. Scattered feathers. Bark patterns forming eyes.  
    Tools that move slightly when unobserved. Small animals refusing to flee.
    
    \paragraph{Etiquette}
    Ask three times. Offer food, breath, warmth, or story.  
    Never name a spirit that has not named itself.  
    Never command without giving something in return.
    
    \paragraph{Core Themes}
    Decentralized magic • Endless tiny alliances • Gifts with expectations •  
    Local power • Listening over shaping • Spirits with grudges •  
    Magic that can go beautifully or catastrophically sideways.
    
    %-------------------------------------------------------------
    \subsubsection*{Rites of the Wild Choir}
    
    \begin{itemize}[leftmargin=1.5em]
    
      \item \textbf{Lesser Rite — Whisper-Meal Offering}  
      Leave food, warmth, or music for a minor spirit.  
      \emph{Effect:} Gain +1d on actions involving that spirit’s domain for the scene.
    
      \item \textbf{Lesser Rite — Sense the Unquiet}  
      Whisper a question into wind or water.  
      \emph{Effect:} Reveal hidden emotions, resentments, or disturbances within Near range.
    
      \item \textbf{Greater Rite — Borrowed Shape}  
      A willing beast-spirit lends you one of its truths.  
      \emph{Effect:} Gain one temporary gift (claws, ears, low-light vision, nimbleness).  
      \emph{Cost:} You must perform a task it cares about within a day.
    
      \item \textbf{Greater Rite — Pack-Bound Ward}  
      Call three spirits to encircle a place or ally.  
      \emph{Effect:} Attacks or harmful workings suffer −1 Effect.  
      \emph{Flaw:} Spirits may demand offerings from visitors.
    
      \item \textbf{Grand Rite — Chorus of the Many}  
      Speak with every spirit willing to answer.  
      \emph{Effect:} Ask three questions about the land, its dangers, or its wounds.  
      \emph{Complication:} The spirits argue; GM selects one truth that contradicts the others.
    
    \end{itemize}
    
    %-------------------------------------------------------------
    \subsubsection*{Talents: Wild Choir Initiates}
    \begin{tcolorbox}[colback=mistgray!10,colframe=deepgreen,title={Wild Choir --- Talents}]
    
    \textbf{Tier I: Core Talents}
    \begin{itemize}[noitemsep]
      \item \textbf{Spirit-Eared} --- You can hear small spirits even when they whisper like leaves or dust.
      \item \textbf{Quiet-Step Pact} --- Gain +1 Position when acting in places inhabited by friendly spirits.
    \end{itemize}
    
    \textbf{Tier II: Advanced Talents}
    \begin{itemize}[noitemsep]
      \item \textbf{Gift of the Borrowed Tongue} --- Speak to animals and minor spirits as equals.
      \item \textbf{Echo-Bound} --- When aided by spirits, convert 1 Fatigue into +1d once per scene.
    \end{itemize}
    
    \textbf{Tier III: Ascendant Talents}
    \begin{itemize}[noitemsep]
      \item \textbf{Choir’s Favor} --- Three spirits act as followers (Cap 2 each) during a scene of crisis.
      \item \textbf{World-Rooted} --- You cannot be surprised in lands where you’ve offered rites.
    \end{itemize}
    
    \textbf{Paradox Talent}
    \begin{itemize}[noitemsep]
      \item \textbf{Spirit-Worn Body} --- Your boundaries blur.  
      Once per scene, become mist, leaf, feather, or pebble until your next action.  
      \emph{Flaw:} Spirits may speak through you.
    \end{itemize}
    
    \end{tcolorbox}
    
    %-------------------------------------------------------------
    \subsubsection*{Hexes \& Curses of the Choir}
    
    \begin{itemize}[leftmargin=1.5em]
      \item \textbf{Bramble’s Spite} — Thorns snag only the cursed; DV +1 for movement.
      \item \textbf{Crow’s Mockery} — Misfortune accumulates; 1s rolled generate +1 SB.
      \item \textbf{Salt-of-the-Earth Ban} — A person cannot step onto consecrated soil until forgiveness is earned.
    \end{itemize}
    
    %-------------------------------------------------------------
    \subsubsection*{Bargains of the Wild Choir}
    
    \paragraph{Common Terms}
    Most spirits bargain for:
    \begin{itemize}[leftmargin=1.5em]
      \item food or warmth  
      \item small acts of kindness  
      \item repairs to their environment  
      \item stories, songs, or memories  
      \item promises to avoid harming kin or habitat  
    \end{itemize}
    
    \paragraph{Dangerous Spirits May Ask For:}
    \begin{itemize}[leftmargin=1.5em]
      \item blood or breath  
      \item vengeance  
      \item names  
      \item a change in the land itself  
      \item a promise that binds future generations  
    \end{itemize}
    
    %-------------------------------------------------------------
    \subsubsection*{Rivalries}
    
    \begin{itemize}[leftmargin=1.5em]
      \item \textbf{Thorns of Malachai} — Animistic bargains avoid one-sided costs; Malachai’s curses mock that balance.
      \item \textbf{Temple of Light} — Regards spirit-courts as heretical and dangerously unregulated.
      \item \textbf{Cults of Livaea} — Clash where beauty or desire shifts the natural order.
      \item \textbf{Hearth Witches} — Occasional friction over who properly “keeps” a place’s safety.
    \end{itemize}
    
    %-------------------------------------------------------------
    \subsubsection*{Adventure Seeds}
    
    \begin{enumerate}[leftmargin=1.5em]
      \item \textbf{The River Has Stopped Speaking}  
      Something angers the water-spirits; the Wild Choir cannot hear them.
    
      \item \textbf{The Forbidden Orchard}  
      The trees have learned to hate humans.  
      A witch seeks someone brave enough to listen to their demands.
    
      \item \textbf{The Mud-King’s Debt}  
      A spirit of silt and lost things has risen to collect on a broken promise.
    
      \item \textbf{Three Spirits, One Lie}  
      Three local spirits give incompatible accounts of a looming threat.
    
      \item \textbf{The Thorn-Crowned Beast}  
      A corrupted spirit-creature stalks the village, wearing a curse that smells of Malachai.
    \end{enumerate}


    \subsection*{The Dark Choir (Variant Table)}
    \label{order:darkchoir}
    
    The Dark Choir is not a separate Order — it is what happens when spirits  
    \emph{lose trust}, \emph{grow hungry}, or \emph{are wounded by Names, Cords, or curses}.  
    Instead of rewriting the Wild Choir, use this table of thematic swaps to turn any  
    spirit-allied witch into a servant of the \emph{Many-Eyed Silence}, \emph{Mire-Beasts},  
    \emph{Root-Deep Hungers}, or other corrupted spirit-collectives.
    
    \begin{tcolorbox}[colback=black!3,colframe=black,title={Dark Choir: Thematic Swaps}]
    \begin{tabularx}{\textwidth}{>{\raggedright\arraybackslash}p{0.27\textwidth} X}
    \toprule
    \textbf{Wild Choir Theme} & \textbf{Dark Choir Corruption} \\ \midrule
    
    \textbf{Listening to Spirits}  
    & \textbf{Obeying the Howl}.  
    Spirits no longer whisper requests — they issue compulsions the witch must obey or suffer Fatigue and dread visions. \\[0.8em]
    
    \textbf{Small Offerings (food, story, warmth)}  
    & \textbf{Hungers}.  
    The spirits now demand breath, blood, memories, or stolen emotions.  
    Offerings that once soothed now inflame them. \\[0.8em]
    
    \textbf{Borrowed Shape}  
    & \textbf{Predatory Mask}.  
    The witch may assume a beast-aspect infused with rage; +1 Effect on harm actions, but loses control on a Miss. \\[0.8em]
    
    \textbf{Pack-Bound Ward (protective)}  
    & \textbf{Pack-Devour Ward}.  
    A place becomes “safe”… because spirits consume intruders.  
    Allies must test DV 3 to cross unharmed. \\[0.8em]
    
    \textbf{Chorus of the Many (questions)}  
    & \textbf{Choir of Mouths (disinformation)}.  
    The spirits answer three questions, but one is always a lie designed to cause suffering.  
    GM chooses which. \\[0.8em]
    
    \textbf{Spirits as Allies}  
    & \textbf{Spirits as Parasites}.  
    Bound spirits graft themselves onto the witch’s body as feathers, scales, teeth, or fungal growths.  
    Gain +1d once per scene; mark 1 Fatigue when used. \\[0.8em]
    
    \textbf{Ritual of Invitation}  
    & \textbf{Ritual of Consumption}.  
    To call a powerful spirit, something living must be harmed or surrendered.  
    (Effect +1. Cost: Doom +1 or permanent Change.) \\[0.8em]
    
    \textbf{Gentle Hexes (bramble, crow, sanctuary)}  
    & \textbf{Malevolent Hexes (snare, rot, silence)}.  
    Hexes reduce Effect or Position and cause lingering spiritual taint.  
    Cleansing requires a vow or sacrifice. \\[0.8em]
    
    \textbf{Bargains of Reciprocity}  
    & \textbf{Bargains of Entanglement}.  
    Breaking a bargain causes a spirit to lodge in the witch’s soul until appeased.  
    Count as a Harm (0–3) depending on intensity. \\[0.8em]
    
    \textbf{Spirits Want Respect or Maintenance}  
    & \textbf{Spirits Want Territory}.  
    The witch is compelled to expand the forest, swamp, or fog-bound domain.  
    Failure angers territorial spirits, drawing the \textbf{Many-Eyed Hunts}. \\[0.8em]
    
    \textbf{Spirits Warn of Threats}  
    & \textbf{Spirits Create Threats}.  
    They engineer “lessons” through dangerous accidents to demand attention or obedience.  
    SB +1 on any natural hazard scene. \\[0.8em]
    
    \textbf{Minor Spirits as Namesakes}  
    & \textbf{Unnames}.  
    The witch loses or erodes one of their minor Names (GM’s choice).  
    Gain +1 Position in spirit-rich areas, −1 Position among mortals. \\[0.8em]
    
    \bottomrule
    \end{tabularx}
    \end{tcolorbox}

%=====================================================================
% WITCH BUILDS BY ORDER
%=====================================================================
\section*{Witch Builds by Order}
\addcontentsline{toc}{section}{Witch Builds by Order}

Each Witch Order expresses its patron through unique magical forms, 
tendencies, emotional hazards, and advancement paths. The following 
builds provide example archetypes suitable for PCs or major NPCs.

%---------------------------------------------------------------------
\subsection*{Order of Ikasha --- Witches of the Unseen Covenant}
\begin{tcolorbox}[colback=mistgray!10, colframe=shadowviolet, title={The Shadowbound Operative}]

\textbf{Theme:} Silent passage, hidden roads, secrets carried beneath the veil of night.

\textbf{Starting Talents:}
\begin{itemize}[noitemsep]
    \item \textbf{Shadowstep:} Slip through touching patches of dimness, emerging where sight does not follow.
    \item \textbf{Drowned Footfalls:} Your passing leaves no sound; even steel and breath are swallowed by shadow.
\end{itemize}

\textbf{Signature Working:} \emph{Whisper of the Crossroads} --- reveal the unseen fulcrum of a place: a weak ward, a forgotten exit, or a point where fate bends in silence.

\textbf{Weakness:} You cannot resist opening a hidden way once discovered, for Ikasha’s will urges every locked path to be tested.

\textbf{Advancement Path:} Umbral Sleight $\rightarrow$ Shadow Double $\rightarrow$ The Sleeper’s Hand.
\end{tcolorbox}

%---------------------------------------------------------------------
\subsection*{Cult of Inaea --- The Family of Never-Leaving}
\begin{tcolorbox}[colback=mistgray!10, colframe=shadowviolet, title={The Quiet Daughter}]
\textbf{Theme:} Emotional captivity, devotion, psychic entanglement.

\textbf{Starting Talents:}
\begin{itemize}[noitemsep]
    \item \textbf{Honeyed Compliance:} Soften hostility with submissive charm.
    \item \textbf{Blood Memory:} Touch reveals secrets of longing and shame.
\end{itemize}

\textbf{Signature Working:} \emph{Thread of the Unbroken Line} --- a psychic tether
that shares emotions.

\textbf{Weakness:} Panics when someone pulls away or threatens abandonment.

\textbf{Advancement Path:} Familial Weave $\rightarrow$ Binding Marriage Rite 
$\rightarrow$ Mother of Many.
\end{tcolorbox}

%---------------------------------------------------------------------
\subsection*{Order of Livaea --- Velvet Influence}
\begin{tcolorbox}[colback=mistgray!10, colframe=lanternamber, title={The Velvet Enchantress}]
\textbf{Theme:} Seduction, soft power, irresistible presence.

\textbf{Starting Talents:}
\begin{itemize}[noitemsep]
    \item \textbf{Velvet Glamour:} Advantage on Influence when holding eye contact.
    \item \textbf{Sigh of the Heart:} Learn a target’s deepest desire.
\end{itemize}

\textbf{Signature Working:} \emph{Livaea's Embrace} --- blissful enthrallment 
that mimics love.

\textbf{Weakness:} Emotionally numb unless desired.

\textbf{Advancement Path:} Serpent’s Whisper $\rightarrow$ Ecstasy Binding 
$\rightarrow$ Livaean Aphrodite.
\end{tcolorbox}

%---------------------------------------------------------------------
\subsection*{Order of Raéyn --- Storm and Tide}
\begin{tcolorbox}[colback=mistgray!10, colframe=shadowviolet, title={The Tidecaller Witch}]
\textbf{Theme:} Emotional storms, tides, winds, overwhelming force.

\textbf{Starting Talents:}
\begin{itemize}[noitemsep]
    \item \textbf{Brine-Seer’s Sense:} Feel emotional pressure shifts.
    \item \textbf{Wave Step:} Move across unstable or liquid surfaces.
\end{itemize}

\textbf{Signature Working:} \emph{Storm-Tide Unleashing} --- emotional and 
elemental flood.

\textbf{Weakness:} Emotional instability triggers uncontrolled magic.

\textbf{Advancement Path:} Riptide Heart $\rightarrow$ Tempest Crown 
$\rightarrow$ Raéyn’s Leviathan.
\end{tcolorbox}

%---------------------------------------------------------------------
\subsection*{Order of Aveh --- The Faceless Patron}
\begin{tcolorbox}[colback=mistgray!10, colframe=bloodrust, title={The Nameless Vessel}]
\textbf{Theme:} Erasure of identity, emptiness, anonymity.

\textbf{Starting Talents:}
\begin{itemize}[noitemsep]
    \item \textbf{Unface:} Erase your presence; people forget your features.
    \item \textbf{Hollow Echo:} Mimic voices heard for at least ten seconds.
\end{itemize}

\textbf{Signature Working:} \emph{Mask of the Void} --- total erasure from 
supernatural detection.

\textbf{Weakness:} Risks forgetting personal memories.

\textbf{Advancement Path:} Blur of the Self $\rightarrow$ Void Persona 
$\rightarrow$ Aveh’s Perfect Reflection.
\end{tcolorbox}

%---------------------------------------------------------------------
\subsection*{Hearth Witches --- The Slow and Gentle Craft}
\begin{tcolorbox}[colback=mistgray!10, colframe=lanternamber, title={The Ember-Tender}]
\textbf{Theme:} Home-spirits, warmth, protection, quiet miracles.

\textbf{Starting Talents:}
\begin{itemize}[noitemsep]
    \item \textbf{Teakettle Omens:} Read faint omens in smoke or steam.
    \item \textbf{Warmth Ward:} Grant morale or ease minor afflictions.
\end{itemize}

\textbf{Signature Working:} \emph{Home-That-Walks} --- summon a mobile hearth-spirit 
to protect allies.

\textbf{Weakness:} Haunted by neglected or abandoned places.

\textbf{Advancement Path:} Keeper of Embers $\rightarrow$ House-Spirit Caller 
$\rightarrow$ Hearthmother Ascendant.
\end{tcolorbox}

%---------------------------------------------------------------------
\subsection*{Order of Malachai --- The Chained Angel}
\begin{tcolorbox}[colback=mistgray!10, colframe=bloodrust, title={The Chain-Bearer}]
\textbf{Theme:} Beautiful curses, doomed blessings, divine chains.

\textbf{Starting Talents:}
\begin{itemize}[noitemsep]
    \item \textbf{Blessing of the Hook:} Every boon contains a curse.
    \item \textbf{Broken Halo:} Sense the exact cost of any miracle sought.
\end{itemize}

\textbf{Signature Working:} \emph{Malachai’s Golden Promise} --- immediate
miracle with greater deferred cost.

\textbf{Weakness:} Breaking a bargain inflicts severe backlash.

\textbf{Advancement Path:} Cursewright $\rightarrow$ Angel-Bound 
$\rightarrow$ Chained Seraph of Malachai.
\end{tcolorbox}

%---------------------------------------------------------------------
\subsection*{Witch Hunters --- Iron and Suppression}
\begin{tcolorbox}[colback=mistgray!10, colframe=shadowviolet, title={The Iron Acolyte}]
\textbf{Theme:} Ritual anti-magic, sacred iron, witch suppression.

\textbf{Starting Talents:}
\begin{itemize}[noitemsep]
    \item \textbf{Counter-Witch Strike:} Cancel a Working by striking its source.
    \item \textbf{Aegis of Iron:} Reduce magical harm by 1 via iron sigils.
\end{itemize}

\textbf{Signature Technique:} \emph{Sever the Vein} --- disrupt a witch’s 
connection to their Patron for one scene.

\textbf{Weakness:} Vows compel you to confront any witchcraft witnessed.

\textbf{Advancement Path:} Sigil Knight $\rightarrow$ Purity Archivist 
$\rightarrow$ Hunter-Saint of Iron.
\end{tcolorbox}

%=====================================================================
% ORDER OF IKASHA — TALENTS
% The Unseen Covenant, Keepers of Shadowed Roads
%=====================================================================

\subsection*{Order of Ikasha}
\begin{tcolorbox}[colback=mistgray!10,colframe=shadowviolet,title={Order of Ikasha --- Talents}]

\textbf{Tier I: Shadowbound Initiate}
\begin{itemize}[noitemsep]

  \item \textbf{Shadowstep} --- Move through touching patches of dimness as a single motion, slipping from one shade to the next without drawing notice.

  \item \textbf{Drowned Footfalls} --- Your passing makes no sound; steps, breath, and the draw of steel are swallowed by the dark.

  \item \textbf{Eyes of the Crossroads} --- Once each scene, discern the hidden fulcrum within a place: a weak ward, an unseen exit, a vulnerable mind, or a neglected corner where fate may be bent.

\end{itemize}


\textbf{Tier II: Agents of the Unseen Covenant}
\begin{itemize}[noitemsep]

  \item \textbf{Umbral Sleight} --- You may take, place, or alter an object in plain view so subtly that none perceive the change.

  \item \textbf{Blackout Veil} --- A hush of shadow settles about you, dimming nearby light and muddling the sight of onlookers for a short while.

  \item \textbf{Quiet Knife} --- Any blade in your hand strikes without sound; even the dying may not cry out unless you permit it.

  \item \textbf{Listening in the Walls} --- Stone, wood, and mortar bear whispers to your ear. Hear speech, movement, or intent through barriers as though they were thin cloth.

\end{itemize}


\textbf{Tier III: Masters of the Hidden Roads}
\begin{itemize}[noitemsep]

  \item \textbf{Shadow Double} --- Your silhouette slips from your feet and wanders as a false reflection, drawing eyes and misdirecting watchers.

  \item \textbf{The Perfect Lift} --- After observing a target for a brief span, you may remove or replace a small but vital item without trace or suspicion.

  \item \textbf{Umbral Extraction} --- Draw a person or object into the deep shade and cause them to reappear elsewhere within the same shrouded place, unseen by mortal sight.

\end{itemize}


\textbf{Paradox Talent}
\begin{itemize}[noitemsep]

  \item \textbf{The Sleeper’s Hand} --- For a single scene, shadows grant you perfect concealment. Lies pass as truth, memories may be plucked from a mind with but a whisper, and any stroke you deliver leaves neither sound nor sign. When the moment ends, a tie of the heart or spirit fades from you, claimed by Ikasha as the price of her aid.

\end{itemize}

\end{tcolorbox}

%=====================================================================
% CULT OF INAEA — OBSESSION, FAMILY, BINDING
%=====================================================================
\subsection*{Cult of Inaea}
\begin{tcolorbox}[colback=mistgray!10,colframe=shadowviolet,title={Cult of Inaea --- Talents}]

\textbf{Tier I: Core Talents}
\begin{itemize}[noitemsep]
  \item \textbf{Honeyed Compliance} --- Soften hostility; turn aggression into uncertainty.
  \item \textbf{Blood Memory} --- Touch reveals a secret rooted in desire or shame.
\end{itemize}

\textbf{Tier II: Advanced Talents}
\begin{itemize}[noitemsep]
  \item \textbf{Threadpull} --- Alter a single emotional drive for one scene.
  \item \textbf{Inaea's Embrace} --- Comfort heals 1 Fatigue and plants dependence.
\end{itemize}

\textbf{Tier III: Ascendant Talents}
\begin{itemize}[noitemsep]
  \item \textbf{Family Bond} --- Link up to three people who share emotions with you.
  \item \textbf{Inheritance Rite} --- Mark someone as “family”; they instinctively defend you.
\end{itemize}

\textbf{Paradox Talent}
\begin{itemize}[noitemsep]
  \item \textbf{Devouring Attachment} --- Take someone’s trauma into yourself permanently.
\end{itemize}

\end{tcolorbox}

%=====================================================================
% ORDER OF LIVAEA — SEDUCTION, SOFT POWER
%=====================================================================
\subsection*{Order of Livaea}
\begin{tcolorbox}[colback=mistgray!10,colframe=lanternamber,title={Order of Livaea --- Talents}]

\textbf{Tier I: Core Talents}
\begin{itemize}[noitemsep]
  \item \textbf{Velvet Glamour} --- Advantage on Influence when maintaining eye contact.
  \item \textbf{Sigh of the Heart} --- Learn someone's deepest, unspoken desire.
\end{itemize}

\textbf{Tier II: Advanced Talents}
\begin{itemize}[noitemsep]
  \item \textbf{Perfume of Want} --- Aura intensifies cravings; targets open up emotionally.
  \item \textbf{Honey-Tongued} --- Reduce suspicion clocks when speaking softly.
\end{itemize}

\textbf{Tier III: Ascendant Talents}
\begin{itemize}[noitemsep]
  \item \textbf{Velvet Bind} --- Enchantment that produces intense closeness or trust.
  \item \textbf{Mask of Desire} --- Take the appearance a target most wants to see.
\end{itemize}

\textbf{Paradox Talent}
\begin{itemize}[noitemsep]
  \item \textbf{Livaea's Ecstasy} --- Overwhelm a target with bliss; may induce obsession.
\end{itemize}

\end{tcolorbox}

%=====================================================================
% ORDER OF RAÉYN — STORMS, TIDES, EMOTION
%=====================================================================
\subsection*{Order of Raéyn}
\begin{tcolorbox}[colback=mistgray!10,colframe=shadowviolet,title={Order of Raéyn --- Talents}]

\textbf{Tier I: Core Talents}
\begin{itemize}[noitemsep]
  \item \textbf{Brine-Seer’s Sense} --- Detect emotional “pressure fronts.”
  \item \textbf{Wave Step} --- Move across unstable or liquid surfaces.
\end{itemize}

\textbf{Tier II: Advanced Talents}
\begin{itemize}[noitemsep]
  \item \textbf{Tide Pull} --- Drag a target toward or away from you (emotionally or physically).
  \item \textbf{Calm the Waters} --- Reduce tensions or fears in a group.
\end{itemize}

\textbf{Tier III: Ascendant Talents}
\begin{itemize}[noitemsep]
  \item \textbf{Tempest Crown} --- Manifest a storm reflecting your emotional state.
  \item \textbf{Riptide Heart} --- When overwhelmed, gain +1 die on emotion-fueled actions.
\end{itemize}

\textbf{Paradox Talent}
\begin{itemize}[noitemsep]
  \item \textbf{Raéyn’s Leviathan} --- Unleash a massive emotional storm; lose control briefly.
\end{itemize}

\end{tcolorbox}

%=====================================================================
% ORDER OF AVEH — FACELESS, OBLIVION
%=====================================================================
\subsection*{Order of Aveh}
\begin{tcolorbox}[colback=mistgray!10,colframe=bloodrust,title={Order of Aveh --- Talents}]

\textbf{Tier I: Core Talents}
\begin{itemize}[noitemsep]
  \item \textbf{Unface} --- Erase your features from memory.
  \item \textbf{Hollow Echo} --- Mimic the voice of anyone you've heard.
\end{itemize}

\textbf{Tier II: Advanced Talents}
\begin{itemize}[noitemsep]
  \item \textbf{Mask of False Intent} --- Conceal supernatural motives from diviners.
  \item \textbf{Empty-Skin Step} --- NPCs ignore you if you remain nonthreatening.
\end{itemize}

\textbf{Tier III: Ascendant Talents}
\begin{itemize}[noitemsep]
  \item \textbf{Identity Theft} --- Assume someone's social footprint (not their body).
  \item \textbf{Shadow Self} --- Create a disposable echo of your presence.
\end{itemize}

\textbf{Paradox Talent}
\begin{itemize}[noitemsep]
  \item \textbf{Aveh’s Oblivion} --- Erase your identity temporarily; lose a memory.
\end{itemize}

\end{tcolorbox}

%=====================================================================
% HEARTH WITCHES — HOME, WARMTH, SPIRITS
%=====================================================================
\subsection*{Hearth Witches}
\begin{tcolorbox}[colback=mistgray!10,colframe=lanternamber,title={Hearth Witch Talents}]

\textbf{Tier I: Core Talents}
\begin{itemize}[noitemsep]
  \item \textbf{Teakettle Omens} --- Read omens in smoke or steam.
  \item \textbf{Warmth Ward} --- Remove a minor fear or condition.
\end{itemize}

\textbf{Tier II: Advanced Talents}
\begin{itemize}[noitemsep]
  \item \textbf{Ember Sight} --- See emotional traces as faint heat patterns.
  \item \textbf{Hearth-Friend’s Blessing} --- A home-spirit aids a simple task.
\end{itemize}

\textbf{Tier III: Ascendant Talents}
\begin{itemize}[noitemsep]
  \item \textbf{House-Spirit Caller} --- Summon a hearth-spirit to protect allies.
  \item \textbf{Home-That-Walks} --- Create a temporary safe zone.
\end{itemize}

\textbf{Paradox Talent}
\begin{itemize}[noitemsep]
  \item \textbf{Keeper of Ashes} --- Carry the emotional imprint of the dying.
\end{itemize}

\end{tcolorbox}

%=====================================================================
% ORDER OF MALACHAI — DIVINE CHAINS, DOOMED MIRACLES
%=====================================================================
\subsection*{Order of Malachai}
\begin{tcolorbox}[colback=mistgray!10,colframe=bloodrust,title={Order of Malachai --- Talents}]

\textbf{Tier I: Core Talents}
\begin{itemize}[noitemsep]
  \item \textbf{Blessing of the Hook} --- Every boon carries a hidden curse.
  \item \textbf{Broken Halo} --- Sense the true cost of any miracle or Working.
\end{itemize}

\textbf{Tier II: Advanced Talents}
\begin{itemize}[noitemsep]
  \item \textbf{Chain the Favor} --- Bind a promise; breaking it inflicts consequence.
  \item \textbf{Angelbone Sigil} --- A curse that strengthens when resisted.
\end{itemize}

\textbf{Tier III: Ascendant Talents}
\begin{itemize}[noitemsep]
  \item \textbf{Golden Promise} --- Grant a miracle with an inevitable cost.
  \item \textbf{Tethered Miracle} --- Borrow divine power at the expense of Fatigue.
\end{itemize}

\textbf{Paradox Talent}
\begin{itemize}[noitemsep]
  \item \textbf{Chained Seraph Form} --- Take on a radiant, cursed angelic aspect.
\end{itemize}

\end{tcolorbox}

\subsection*{Red Sisters of the Living Flame}
\begin{tcolorbox}[colback=mistgray!10,colframe=emberred,title={Red Sisters of the Living Flame --- Talents}]

\textbf{Tier I: Core Talents}
\begin{itemize}[noitemsep]
  \item \textbf{Pain-Lantern Poise} --- When you willingly take 1 Harm or Fatigue, gain +1d on your next action this scene.
  \item \textbf{Scourged Flesh} --- Reduce incoming Harm by 1 once per scene; leaves a branded vow-mark.
\end{itemize}

\textbf{Tier II: Advanced Talents}
\begin{itemize}[noitemsep]
  \item \textbf{Ember-Stitched Resolve} --- When healing another, you may burn yourself for +1 Effect.
  \item \textbf{Crimson Penance} --- Absorb corruption, curses, or Hexes; take 1 Fatigue to hold them at bay.
\end{itemize}

\textbf{Tier III: Ascendant Talents}
\begin{itemize}[noitemsep]
  \item \textbf{Vowbrand Authority} --- Once per scene, enforce an oath: the target gains Fatigue if they violate it.
  \item \textbf{Flameborne Rapture} --- Enter a fiery trance; count Position as one step higher for the scene.
\end{itemize}

\textbf{Paradox Talent}
\begin{itemize}[noitemsep]
  \item \textbf{Living Pyre Form} --- Become wreathed in radiant suffering. Allies acting in your presence roll with +1d; you take +1 Harm from all sources until the scene ends.
\end{itemize}

\end{tcolorbox}

\subsection*{Silent Matrons of Light}
\begin{tcolorbox}[colback=mistgray!10,colframe=softgold,title={Silent Matrons of Light --- Talents}]

\textbf{Tier I: Core Talents}
\begin{itemize}[noitemsep]
  \item \textbf{Stillness of Dawn} --- Gain +1d on actions taken in silence or focused observation.
  \item \textbf{Hushed Lantern} --- Once per scene, suppress whispers, illusions, or falsehoods within Near range.
\end{itemize}

\textbf{Tier II: Advanced Talents}
\begin{itemize}[noitemsep]
  \item \textbf{Radiant Quiet} --- Improve Position by one step when defending against deception, coercion, or mental influence.
  \item \textbf{Voice Withheld} --- Store a single spoken truth; when released, gain +1 Effect and impose disadvantage on a target.
\end{itemize}

\textbf{Tier III: Ascendant Talents}
\begin{itemize}[noitemsep]
  \item \textbf{Judgment Gaze} --- Reveal hidden guilt, curse-anchors, or spiritual interference; acting on the revelation grants +1d.
  \item \textbf{Veil of Unbroken Light} --- Once per scene, nullify a Working, Hex, or spell within line of sight.
\end{itemize}

\textbf{Paradox Talent}
\begin{itemize}[noitemsep]
  \item \textbf{Seraph of Silence} --- Manifest a luminous, wordless form. Your presence prevents lies and grants allies +1 Position, but you may not speak or cast this scene.
\end{itemize}

\end{tcolorbox}

%=====================================================================
% WITCH HUNTERS — IRON, SUPPRESSION, NULLIFICATION
%=====================================================================
\subsection*{Witch Hunters}
\begin{tcolorbox}[colback=mistgray!10,colframe=shadowviolet,title={Witch Hunter Talents}]

\textbf{Tier I: Core Talents}
\begin{itemize}[noitemsep]
  \item \textbf{Iron Discipline} --- +1 to resist magical influence.
  \item \textbf{Counter-Witch Strike} --- Cancel a Working by striking its source.
\end{itemize}

\textbf{Tier II: Advanced Talents}
\begin{itemize}[noitemsep]
  \item \textbf{Aegis of Iron} --- Reduce magical harm by 1 while sigils are intact.
  \item \textbf{Chain-Break Doctrine} --- Instantly identify a witch's Patron influence.
\end{itemize}

\textbf{Tier III: Ascendant Talents}
\begin{itemize}[noitemsep]
  \item \textbf{Sever the Vein} --- Break a witch’s connection to their Patron for a scene.
  \item \textbf{Ritual Scourging} --- Perform rites to permanently weaken certain magic.
\end{itemize}

\textbf{Paradox Talent}
\begin{itemize}[noitemsep]
  \item \textbf{Hunter-Saint of Iron} --- Become a living ward; cannot accept magical aid.
\end{itemize}

\end{tcolorbox}

\section*{Threshold Working: The Witchcraft System}

\subsection*{Core Principles (The Three Laws)}

\textbf{1. The Law of Hospitality \& Truth} \\
You cannot take what you will not feed, and what is denied becomes powerful. Every working requires an offering, and any unacknowledged debt or truth becomes a source of narrative force.

\textbf{2. The Law of Reciprocity \& Witness} \\
Binding another binds yourself to their outcome. Unseen workings fray, but a witness strengthens the working. The more visible the working, the more real it becomes.

\textbf{3. The Law of Thresholds} \\
Magic is most potent at edges where one state becomes another. The strength of the working is tied to how well the threshold is honored and the price paid.

\subsection*{The Complete Working (Five-Step Process)}

\subsubsection*{Step 1: Identify the Threshold}
Name the threshold: doorway, river ford, bedside, market gate, crossroads, graveside, sleep/waking edge, or moment between decisions.  
Identify the threshold's disposition (see Household Disposition).  

\textbf{DV Adjustment:}
\begin{itemize}[noitemsep]
    \item Neutral Threshold: +0 DV
    \item Contested Threshold: +1 DV
    \item Resonant Threshold: -1 DV
\end{itemize}

\subsubsection*{Step 2: Choose Your Witness}
Select a witness: person, beast, ancestor, household, landscape, or Patron. Determine its relationship to the threshold:

\begin{itemize}[noitemsep]
    \item Custodian (directly tied): -1 DV
    \item Neutral (unrelated): +0 DV
    \item Contested (hostile): +1 DV
\end{itemize}

\subsubsection*{Step 3: Name Your Will}
State your intended effect.  
Choose one Threshold TAG: \texttt{[WELCOME]}, \texttt{[PASSAGE]}, \texttt{[REMEMBER]}, \texttt{[PRICE]}, \texttt{[UNBIND]}, \texttt{[RELEASE]}, \texttt{[SHELTER]}.  

\textbf{DV Adjustment:}
\begin{itemize}[noitemsep]
    \item Basic Effect: +0 DV
    \item Enhanced Scope: +1 DV
    \item Transformative Effect: +2 DV
\end{itemize}

\subsubsection*{Step 4: Set Your Price}
Select one cost from your Order’s list.  
Determine commitment:

\begin{itemize}[noitemsep]
    \item \textbf{Simple}: Mark 1 Fatigue (DV -1)
    \item \textbf{Moderate}: Tangible sacrifice (DV -2)
    \item \textbf{Solemn}: Bound oath (DV -3)
\end{itemize}

\subsubsection*{Step 5: Make the Exchange}
Build your dice pool: Attribute + Skill + Witness bonus.  
Roll and compare successes to DV.

\textbf{Position Effects:}
\begin{itemize}[noitemsep]
    \item Dominant: Reroll one failure
    \item Controlled: No rerolls
    \item Desperate: Reroll one success
\end{itemize}

\subsection*{Threshold Mechanics Integration}

\subsubsection*{Household Disposition}

\begin{center}
\begin{tabular}{p{0.25\textwidth} p{0.6\textwidth}}
\toprule
\textbf{Disposition} & \textbf{Effect} \\
\midrule
Hungry   & First working costs +1 Fatigue \\
Grieving & Magic requires a narrative of memory or loss \\
Fearful  & First rite is in Desperate Position \\
Watching & +1 Effect if the house is fed or soothed \\
Young    & Chaotic outcomes on partials \\
Old      & Weak boons but harsh backlash \\
\bottomrule
\end{tabular}
\end{center}

\subsubsection*{Threshold TAGs}

\begin{itemize}[noitemsep]
    \item \texttt{[WELCOME]}: +1d to social interactions in the threshold
    \item \texttt{[PASSAGE]}: -1 DV for movement through the threshold
    \item \texttt{[REMEMBER]}: +1d to recall events tied to the threshold
    \item \texttt{[PRICE]}: +1d when naming or negotiating costs
    \item \texttt{[UNBIND]}: +1d to break bindings or ties
    \item \texttt{[RELEASE]}: +1d to free targets from conditions
    \item \texttt{[SHELTER]}: +1 Position while within the threshold
\end{itemize}

\subsection*{Cost System by Order}

\subsubsection*{Order of Ikasha}
Memory, Shadow Mark, Bond, Fear, Identity

\subsubsection*{Order of Livaea}
Vulnerability, Desire, Charm, Poise, Favor

\subsubsection*{Order of Raéyn}
Certainty, Calm, Vow, Balance, Breath

\subsubsection*{Order of Aveh}
Name, Reflection, Role, Trace, Voice

\subsubsection*{Hearth Witches}
Comfort, Warmth, Time, Burden, Memory

\subsubsection*{Order of Malachai}
Vow, Suffering, Burden of Doom, Duty, Truth

\subsection*{Partial Success Consequences}

\begin{center}
\begin{tabular}{p{0.3\textwidth} p{0.6\textwidth}}
\toprule
\textbf{Consequence} & \textbf{Description} \\
\midrule
Withered Effect & The working manifests only partially \\
Twisting of Intent & The power acts askew \\
Calling of a Debt & A spiritual obligation forms \\
Revelation of Unwanted Truth & An unintended omen or memory appears \\
Marked by the Unseen & A subtle sign clings to the witch \\
Disturbance of Spirits & Spirits react with curiosity or hunger \\
Faltering Cost & The sacrifice is warped or magnified \\
Echo of the Attempt & Traces linger and attract attention \\
Emergent Complication & A new danger clock forms \\
Favor with a Hook & A boon succeeds but binds the witch \\
\bottomrule
\end{tabular}
\end{center}

\subsection*{Threshold Talent Tree}

\begin{itemize}[noitemsep]
    \item \textbf{Hearth Branch}: Household affinity and protective workings
    \item \textbf{Passage Branch}: Movement, mediation, and liminal control
    \item \textbf{Memory Branch}: Story, naming, and reciprocity
\end{itemize}

\subsection*{Unified DV Setting Guide}

\textbf{Base DV = 3}

\textbf{Threshold Disposition Modifiers:}
\begin{itemize}[noitemsep]
    \item Hungry/Grieving/Fearful: +1 DV
    \item Watching/Young: +0 DV
    \item Old: -1 DV
\end{itemize}

\textbf{Witness Relationship Modifiers:}
\begin{itemize}[noitemsep]
    \item Custodian: -1 DV
    \item Neutral: +0 DV
    \item Contested: +1 DV
\end{itemize}

\textbf{Working Type Modifiers:}
\begin{itemize}[noitemsep]
    \item Basic: +0 DV
    \item Enhanced: +1 DV
    \item Transformative: +2 DV
\end{itemize}

\textbf{Cost Commitment Modifiers:}
\begin{itemize}[noitemsep]
    \item Simple: -1 DV
    \item Moderate: -2 DV
    \item Solemn: -3 DV
\end{itemize}

\textbf{DV Range: 1--6}

\subsection*{Example Working: The Hearth's Memory}

\textbf{Step 1: Identify the Threshold} \\
A grieving hearth (+1 DV)

\textbf{Step 2: Choose Your Witness} \\
Grandmother’s spirit, a Custodian (-1 DV)

\textbf{Step 3: Name Your Will} \\
Ease a child's passing; \texttt{[SHELTER]} ( +0 DV )

\textbf{Step 4: Set Your Price} \\
Moderate sacrifice: a cherished memory ( -2 DV )

\[
\text{DV} = 3 +1 -1 -2 = 1
\]

\textbf{Step 5: Make the Exchange} \\
Roll Wits + Hearth’s Memory.

The child passes peacefully; the witch loses a childhood memory of comfort.

\section*{Additional Systems for Threshold Working}

\subsection*{The Fourth Actor: Threshold Instincts}

Every threshold possesses a dormant instinct that may awaken during a Working.  
When invoked by success, partial, or failure, the instinct shapes how the threshold responds.

\begin{center}
\begin{tabular}{p{0.25\textwidth} p{0.6\textwidth}}
\toprule
\textbf{Instinct} & \textbf{Expression} \\
\midrule
Call   & Draws spirits or witnesses toward the Working. \\
Reject & Pushes back, generating SB or raising DV next time. \\
Echo   & Repeats a fragment of the Working at a later moment. \\
Bind   & Anchors part of the effect to the threshold itself. \\
Stain  & Leaves a mark upon the place or on the witch. \\
\bottomrule
\end{tabular}
\end{center}

Threshold Instincts activate at the GM’s discretion on a partial or dramatic result.


\subsection*{Residual Magic: Lingers, Echoes, and Scars}

Every Working leaves a residue appropriate to its power, intent, and success state.

\begin{itemize}[noitemsep]
    \item \textbf{Linger}: A faint trace of the Working remains—an altered scent, draft, or mood.
    \item \textbf{Echo}: The Working repeats under mirrored conditions (at dusk, upon entry, when a name is spoken).
    \item \textbf{Scar}: The threshold is permanently altered—hungrier, older, darker, or soothed.
\end{itemize}

Residual Magic creates continuity and living consequence across sessions.


\subsection*{Favors, Debts, and Entanglements}

\subsubsection*{Favor Clock}
A measure of goodwill accrued through honoring Patrons, thresholds, or spirits.

\textbf{Spend Favor to:}
\begin{itemize}[noitemsep]
    \item Call on a Witness without raising DV.
    \item Reduce a Working’s total cost.
    \item Negate a partial consequence.
\end{itemize}

\subsubsection*{Debt Clock}
Accrues when a witch denies truth, withholds cost, or angers a Witness.

\textbf{Effects of Debt:}
\begin{itemize}[noitemsep]
    \item All Workings begin in Desperate Position.
    \item Witnesses treat the witch as Contested.
    \item Thresholds shift disposition toward Hungry or Fearful.
\end{itemize}

Debts must be paid with a Solemn Cost or narrative restitution.


\subsection*{Threshold Alignment: Time and Season}

Threshold power rises and falls with the turning of time.

\subsubsection*{Times of Day}
\begin{itemize}[noitemsep]
    \item Dawn: beginnings, healing, clarity
    \item Dusk: secrets, endings, shadow
    \item Midnight: spirit-listening, forbidden rites
    \item Noon: truth, unbinding, bold action
\end{itemize}

\textbf{DV Adjustment:}
\begin{itemize}[noitemsep]
    \item Aligned Time: --1 DV
    \item Opposed Time: +1 DV
\end{itemize}

\subsubsection*{Seasonal Alignment}
\begin{itemize}[noitemsep]
    \item Winter: loss, memory, bound spirits
    \item Spring: passage, growth, renewal
    \item Summer: shelter, protection, presence
    \item Autumn: price, debts, endings
\end{itemize}

\textbf{DV Adjustment:}
\begin{itemize}[noitemsep]
    \item Aligned Season: --1 DV
    \item Opposed Season: +1 DV
\end{itemize}


\subsection*{Dual Workings (Co-Witchings)}

Two witches may bind their wills upon a single threshold.

\textbf{Rules:}
\begin{itemize}[noitemsep]
    \item Will successes combine.
    \item Costs combine.
    \item Witness bonuses do not stack.
    \item Partial consequences apply to \emph{both}.
\end{itemize}

\textbf{On full success:} both gain 1 Favor.  
\textbf{On dramatic failure:} both mark Debt.


\subsection*{Patron-Specific Thresholds}

Each Patron has favored thresholds where their power is most easily invoked.

\begin{itemize}[noitemsep]
    \item \textbf{Ikasha}: shadowed crossroads, forgotten doorways, back alleys, hidden paths
    \item \textbf{Livaea}: windows, veils, boudoirs, thresholds of invitation
    \item \textbf{Raéyn}: river mouths, tide-worn steps, cliff edges, storm doors
    \item \textbf{Aveh}: mirrors, unmarked entrances, masks, unnamed boundaries
    \item \textbf{Hearth Witches}: hearthstones, cradles, porches, family thresholds
    \item \textbf{Malachai}: prison gates, oath-stones, gallows arches, courthouse steps
\end{itemize}

\textbf{Mechanical Benefit:}
\begin{itemize}[noitemsep]
    \item --1 DV on all Workings
    \item +1 Effect
    \item +1d when applying a TAG
\end{itemize}


\subsection*{Critical Success \& Critical Failure}

Extend the spectrum beyond Success and Partial.

\subsubsection*{Critical Success}
Triggered when the witch earns \textbf{3+ successes over DV}.

\begin{itemize}[noitemsep]
    \item Threshold blesses the Working.
    \item Gain +1 Favor.
    \item Working’s Effect increases one tier.
\end{itemize}

\subsubsection*{Critical Failure}
Triggered when \textbf{no successes are rolled and SB is generated}.

\begin{itemize}[noitemsep]
    \item Threshold lashes out; disposition worsens.
    \item Spirits converge immediately.
    \item Next Working at this threshold begins in Desperate Position.
    \item Cost increases one category.
\end{itemize}


\subsection*{Optional: Threshold Reaction Table}

\begin{center}
\begin{tabular}{c p{0.6\textwidth}}
\toprule
\textbf{Roll} & \textbf{Reaction} \\
\midrule
1--2 & The threshold Rejects: DV +1 on next rite here. \\
3--4 & The threshold Calls: draws a spirit, ancestor, or unseen watcher. \\
5--6 & The threshold Echoes: part of the Working repeats later. \\
7--8 & The threshold Binds: effect becomes tied to this physical location. \\
9--10 & The threshold Blesses: next Working here is in Controlled Position. \\
\bottomrule
\end{tabular}
\end{center}


\subsection*{Optional: Residual Threshold Conditions}

A threshold affected repeatedly develops ongoing traits.

\begin{itemize}[noitemsep]
    \item \textbf{Haunted}: Spirits whisper the names of past Costs.
    \item \textbf{Sanctified}: DV cannot rise above 3 while protected.
    \item \textbf{Blighted}: All Boons are Withered unless soothed.
    \item \textbf{Listening}: Gain +1d to \emph{REMEMBER} Workings here.
    \item \textbf{Unbound}: Effects involving release or freedom gain +1 Effect.
\end{itemize}

\subsection*{Grand Rite of Severance}
\index{Rites!Grand Rite of Severance}

\begin{tcolorbox}[colback=black!5,colframe=crimson,
title={\textbf{Grand Rite of Severance} --- Breaking the Curse of the Unanointed}]

A witch may attempt to purge the Curse of the Unanointed, but doing so tears at
cords of identity, memory, and fate. This ritual is feared even among Patrons.

\paragraph{Requirements (All):}
\begin{itemize}[leftmargin=1.5em]
    \item A full coven of \textbf{three or more witches}
    \item A locus of crossing: graveyard, crossroads, cliffside, threshold
    \item A \textbf{personal sacrifice} of the initiate (destroyed permanently)
    \item Patron permission (\textbf{Downtime} or \textbf{Story Beat cost}, GM’s choice)
\end{itemize}

\paragraph{The Working:}
Roll \textbf{Spirit + Witchcraft} (DV 5, Desperate).  
Each assisting witch rolls \textbf{Spirit + Resolve} (DV 3).

\paragraph{Outcome Table:}
\begin{description}
    \item[Success:]  
    The Curse is stripped away.  
    Reduce one Attribute of the initiate (GM choice) by 1 permanently.

    \item[Partial:]  
    The Curse is removed, \emph{but} fragments remain:  
    The witch gains the \textbf{Echo-Stain} Condition — holy places treat them as haunted.

    \item[Miss:]  
    The Patron arrives in force.  
    The witch keeps the Curse and gains a new \textbf{Patron Taboo}.  
    A coven member suffers a permanent Scar (GM chooses).
\end{description}

\paragraph{GM Note:}
This Rite is intentionally punishing.  
Removing the Curse should feel like tearing out a piece of the soul.

\end{tcolorbox}

\section*{Markworkings: Alterations of Echo and Essence}

Witches are the foremost scholars of marks: deliberate alterations to a being’s
echo that manifest as blessings, burdens, or transformations. A mark binds intention
to echo, shaping how the world answers the marked.

A mark may be granted, stolen, suppressed, or rewritten through Threshold Workings.
All marks have a \textbf{Duration}, \textbf{Strength}, and \textbf{Nature}.

\subsection*{Duration of Marks}

Marks persist according to their depth of binding.

\begin{center}
\begin{tabular}{p{0.25\textwidth} p{0.6\textwidth}}
\toprule
\textbf{Duration} & \textbf{Description} \\
\midrule
Momentary (1 Round) & Swift, potent, volatile blessings or curses. \\
Scene-bound (1 Scene) & Stable marks that influence ongoing action. \\
Lingering (Until Downtime) & Lasting changes to echo; subtle but persistent. \\
\bottomrule
\end{tabular}
\end{center}

\textbf{Shorter Duration = Greater Power.}  
\textbf{Longer Duration = Reduced Effect.}

A witch choosing duration does so during Step 3: Name Your Will.

\textbf{DV Adjustment:}
\begin{itemize}[noitemsep]
    \item Momentary: --1 DV (potent but brief)
    \item Scene-bound: +0 DV
    \item Lingering: +1 DV (diminished strength)
\end{itemize}

\subsection*{Nature of Marks}

A mark's nature reflects how it reshapes echo.

\begin{itemize}[noitemsep]
    \item \textbf{Blessing Mark}: grants advantage, clarity, protection, or fortune.
    \item \textbf{Burden Mark}: imposes weakness, fear, confusion, or exposure.
    \item \textbf{Binding Mark}: tethers a person to an oath, place, or threshold.
    \item \textbf{Veil Mark}: conceals truth, memory, presence, or identity.
    \item \textbf{Echo Mark}: alters voice, shadow, reflection, or spirit-attunement.
    \item \textbf{Wound Mark}: spiritual or symbolic injury that shapes future rolls.
\end{itemize}

A Working may modify only one nature at a time unless two witches act jointly.

\subsection*{Strength of Marks}

Strength determines mechanical impact.

\begin{center}
\begin{tabular}{p{0.25\textwidth} p{0.6\textwidth}}
\toprule
\textbf{Strength Tier} & \textbf{Mechanical Effect} \\
\midrule
Lesser & +1d or --1d to one action; minor tag shift. \\
Moderate & +1 Position or --1 Position; advantage/disadvantage for a scene. \\
Greater & +1 Effect or --1 Effect; binds or reshapes narrative stakes. \\
\bottomrule
\end{tabular}
\end{center}

\textbf{DV Adjustment:}
\begin{itemize}[noitemsep]
    \item Lesser: +0 DV
    \item Moderate: +1 DV
    \item Greater: +2 DV
\end{itemize}

\subsection*{Markworking Procedure}

During Step 3: Name Your Will, the witch declares:

\begin{itemize}
    \item \textbf{Duration}: Momentary, Scene-bound, or Lingering.
    \item \textbf{Nature}: blessing, burden, binding, veil, echo, or wound.
    \item \textbf{Strength Tier}: lesser, moderate, or greater.
\end{itemize}

These determine the Working’s final DV.

When the Working succeeds, the witch places a \textbf{Mark Token} on the target.
The token is removed when the duration ends or the mark is unbound.

\subsection*{Unbinding and Countermarking}

A witch may remove or rewrite an existing mark using the same Five-Step Working:

\textbf{DV for Unbinding:}
\begin{itemize}[noitemsep]
    \item Lesser Mark: DV 2
    \item Moderate Mark: DV 3
    \item Greater Mark: DV 4
\end{itemize}

A witch may also establish a \textbf{Countermark}: a mark that cancels or reverses another.
Countermarks always increase DV by +1 and require a Moderate Cost.

\subsection*{Residual Effects of Marks}

When a mark ends, the echo may retain its impression.

\begin{itemize}[noitemsep]
    \item \textbf{Faint Residue}: +1d to future attempts of similar nature.
    \item \textbf{Fracture}: target suffers --1d on actions opposed to the mark’s intent.
    \item \textbf{Ghost Mark}: threshold spirits may still sense it.
\end{itemize}

GM may apply these on partials or dramatic results.

\subsection*{Order Specialties in Markworkings}

Each Order excels in particular forms of alteration.

\begin{itemize}
    \item \textbf{Ikasha}: Veil Marks, Shadow Marks, Exposure Marks, stealth and secrecy.
    \item \textbf{Livaea}: Charm Marks, Poise Marks, Desire Marks.
    \item \textbf{Raéyn}: Tide Marks, Balance Marks, Breath Marks, emotional sway.
    \item \textbf{Aveh}: Reflection Marks, Name Marks, Identity Marks.
    \item \textbf{Hearth Witches}: Comfort Marks, Warmth Marks, Burden Marks.
    \item \textbf{Malachai}: Oath Marks, Doom Marks, Duty Marks.
\end{itemize}

Witches applying marks in their specialty reduce DV by --1.

\subsection*{Mark Consequences on Partial Success}

\begin{center}
\begin{tabular}{p{0.3\textwidth} p{0.6\textwidth}}
\toprule
\textbf{Partial Result} & \textbf{Effect} \\
\midrule
Flickering Mark & Strength reduced one tier. \\
Tainted Mark & Gains unwanted secondary effect. \\
Reversed Mark & Briefly benefits or empowers the target. \\
Unstable Mark & May trigger at unintended moments. \\
Spirit Claim & A spirit binds itself to the mark. \\
\bottomrule
\end{tabular}
\end{center}

\subsection*{Example Markworkings}

\subsubsection*{Blessing Mark (Scene-bound, Moderate)}
\begin{itemize}[noitemsep]
    \item Nature: Blessing
    \item Duration: Scene
    \item Strength: Moderate (+1 Position)
\end{itemize}
DV: Base 3 + Duration 0 + Strength +1 = \textbf{DV 4}

\subsubsection*{Veil Mark (Momentary, Greater)}
\begin{itemize}[noitemsep]
    \item Nature: Veil
    \item Duration: Momentary (--1 DV)
    \item Strength: Greater (+2 DV)
\end{itemize}
DV: 3 -1 +2 = \textbf{DV 4}

\subsubsection*{Doom Mark (Lingering, Lesser)}
\begin{itemize}[noitemsep]
    \item Nature: Wound
    \item Duration: Lingering (+1 DV)
    \item Strength: Lesser (+0 DV)
\end{itemize}
DV: 3 +1 = \textbf{DV 4}

\section*{The Markworking Talent Tree}

Witches who study the shaping of echoes progress through three branches of Markwork:
\textbf{Carving}, \textbf{Binding}, and \textbf{Transfiguration}.  
Each talent enhances the witch’s ability to create, alter, or unmake marks.

\subsection*{Carving Branch (Shaping Echo)}

\begin{itemize}[leftmargin=2em]
    \item \textbf{Carver’s Touch}  
    You learn to impose a minor mark with a whispered intention.  
    \emph{Effect:} --1 DV on Lesser Marks.

    \item \textbf{Subtle Impression}  
    Marks leave fewer traces, unnoticed except by spirits.  
    \emph{Effect:} Residual Magic does not trigger Scar unless GM chooses.

    \item \textbf{Deep Etching}  
    You may increase a mark’s Strength by one tier without additional DV cost once per scene.  
    \emph{Limit:} Only applies to a mark of your Order’s specialty.

    \item \textbf{Echo Sculptor}  
    You may reshape an existing mark into another of equal Strength.  
    \emph{Effect:} Unbinding + rewriting is resolved as a single Working.
\end{itemize}

\subsection*{Binding Branch (Oaths, Curses, Burdens)}

\begin{itemize}[leftmargin=2em]
    \item \textbf{First Knot}  
    You bind a mark more securely.  
    \emph{Effect:} Scene-bound Marks resist removal; DV +1 to unbind them (others only).

    \item \textbf{Witch’s Claim}  
    You may tie a mark to a threshold or relic.  
    \emph{Effect:} Threshold alignment grants +1 Effect to the mark.

    \item \textbf{Oathburn}  
    Your marks may carry a narrative burden or obligation.  
    \emph{Effect:} Once per downtime, create a Mark that persists until oath-fulfilled.

    \item \textbf{Brand of Power}  
    You may impose a Greater Mark without triggering Desperate Position.  
    \emph{Limit:} Costs must be Moderate or higher.
\end{itemize}

\subsection*{Transfiguration Branch (Beasts, Shadows, Doom, Madness)}

\begin{itemize}[leftmargin=2em]
    \item \textbf{Shift-sense}  
    You understand the edges of self and form.  
    \emph{Effect:} +1d to Workings involving Wound, Echo, or Veil Marks.

    \item \textbf{Bloodshift}  
    Your marks can temporarily alter flesh or instinct.  
    \emph{Effect:} Once per scene, apply a Lesser physical mutation mark with no DV change.

    \item \textbf{Chain of Becoming}  
    You may stack two marks of Lesser Strength into one Moderate transformative mark.  
    \emph{Effect:} Grants +1 Position for transformation Workings.

    \item \textbf{Crown of Transfiguration}  
    Your greatest rite: rewrite a being’s echo entirely.  
    \emph{Effect:} Once per arc, impose a Lingering Greater Mark that alters nature, destiny, or form.
\end{itemize}

\subsection*{Mastery: The Witch of a Thousand Echoes}

Upon learning all three branch capstones:

\begin{itemize}[noitemsep]
    \item You gain +1 Effect when applying marks of any type.
    \item Marks you impose may carry two Natures (e.g., Veil + Blessing, Wound + Binding).
    \item Residual Magic from your own marks always manifests as an Echo, never a Scar.
\end{itemize}

\section*{Example Marks by Order}

\subsection*{Order of Ikasha --- Marks of Shadow and Secrecy}

\begin{itemize}[leftmargin=1.5em]
    \item \textbf{Veilmark of Passing} (Lesser, Scene)  
    Shadows dim around the marked; +1d to stealth or slipping notice.

    \item \textbf{Whisper Mark} (Lingering)  
    The echo mutters fragments of truth; +1d to gather secrets, --1d to resist fear.

    \item \textbf{Shadow Debt} (Moderate)  
    A hidden burden; target becomes easier to track by the Unseen.

    \item \textbf{Unperson Mark} (Greater, Momentary)  
    Presence collapses; the marked cannot be remembered for a single round.
\end{itemize}

\subsection*{Order of Livaea --- Marks of Desire and Influence}

\begin{itemize}[leftmargin=1.5em]
    \item \textbf{Saffron Gaze} (Lesser)  
    The marked appears more appealing; +1d to social enticement.

    \item \textbf{Velvet Thrall} (Moderate)  
    Emotional vulnerability exposed; +1 Position when convincing the target.

    \item \textbf{Rose-chain Mark} (Lingering)  
    A bond of affection pulls the marked toward the witch’s needs.

    \item \textbf{Honeyed Mask} (Greater)  
    Reflection shifts to allure or threaten; +1 Effect to manipulation.
\end{itemize}

\subsection*{Order of Raéyn --- Marks of Tide, Emotion, and Storm}

\begin{itemize}[leftmargin=1.5em]
    \item \textbf{Breath of the Gale} (Lesser)  
    Quickness of thought and motion; reroll one failure per scene.

    \item \textbf{Hearttide Mark} (Moderate)  
    Emotions surge to match Raéyn’s waters; +1d to sway, --1d to resist passion.

    \item \textbf{Stormbrand} (Greater)  
    Tempest aura; +1 Effect to any Working involving release, rupture, or upheaval.
\end{itemize}

\subsection*{Order of Aveh --- Marks of Reflection, Name, and Mask}

\begin{itemize}[leftmargin=1.5em]
    \item \textbf{Name Hollow} (Lesser)  
    The marked’s name becomes slippery; +1d to conceal identity.

    \item \textbf{Mirrorblot} (Moderate)  
    Reflection shows hidden truths; +1 Effect to divination in Aveh’s sight.

    \item \textbf{Unface Mark} (Lingering)  
    A missing detail—voice, gait, memory—slips from others’ minds.

    \item \textbf{Maskshatter} (Greater)  
    Identity fractures, revealing one true motive of the marked.
\end{itemize}

\subsection*{Hearth Witches --- Marks of Comfort, Burden, and Home}

\begin{itemize}[leftmargin=1.5em]
    \item \textbf{Warmthmark} (Lesser)  
    Hearth-stability; +1d to resist fear or despair.

    \item \textbf{Burdenbearing Mark} (Moderate)  
    The witch carries one of the target’s sorrows.

    \item \textbf{Threshold Ward} (Lingering)  
    The marked is protected while inside a home or boundary.

    \item \textbf{Hearth’s Claim} (Greater)  
    A threshold adopts the marked as kin; +1 Position in any home.
\end{itemize}

\subsection*{Order of Malachai --- Marks of Doom, Transformation, and Monstrous Ascension}

\begin{itemize}[leftmargin=1.5em]
    \item \textbf{Fangbirth Mark} (Lesser)  
    A fleeting vampiric surge.  
    \emph{Effect:} For one round, gain +1d to predatory actions; afterwards mark 1 Fatigue.

    \item \textbf{Moonclaw Mark} (Moderate)  
    A controlled strain of lycanthropic echo.  
    \emph{Effect:} Scene-long +1 Position on violence or tracking; --1d to resist frenzy.

    \item \textbf{Doomchain Mark} (Lingering)  
    The witch binds a thread of the target’s fate.  
    \emph{Effect:} Target suffers --1d against oaths or destiny-laden actions.

    \item \textbf{Wyrd Mutation Mark} (Greater, Momentary)  
    Flesh shifts into monstrous expression (horns, claws, bone plates).  
    \emph{Effect:} +1 Effect on physical action; +1 SB on failure.

    \item \textbf{Bloodthirst Ascendant Mark} (Greater)  
    A vampiric elevation of spirit and hunger.  
    \emph{Effect:} +1 Effect to domination, stalking, or draining;  
    \emph{Cost:} Mark 2 Fatigue when it ends.

    \item \textbf{Madness Crown} (Transfigurative)  
    A revelation too bright for mortal mind.  
    \emph{Effect:} Target gains +1d to insight, prophecy, or spirit negotiation;  
    suffers --1 Position on social grace.

    \item \textbf{Angel’s Ruin Mark} (Arc-level Greater Mark)  
    Malachai’s signature working.  
    Echo is rewritten to hold both light and affliction.  
    \emph{Effect:} Target becomes immune to fear and doubt;  
    whenever they refuse a duty, they suffer 2 Harm (echo).
\end{itemize}

\section*{Hostile Marks and Ritual Marks}

Witches understand that a mark may bless or break; an echo may be shaped or severed.
Hostile Marks are afflictions imposed upon unwilling targets, while Ritual Marks represent
the most potent transformations enacted through coven rites.

\subsection*{Hostile Marks}

A Hostile Mark is imposed without consent.  
DV increases by +1 and the Working always begins in Desperate Position unless the target is:
helpless, bound, asleep, or already marked by the witch.

\begin{center}
\begin{tabular}{p{0.3\textwidth} p{0.6\textwidth}}
\toprule
\textbf{Mark} & \textbf{Description} \\
\midrule

\textbf{Soulbite Mark} (Lesser) &
A tearing of echo; target suffers --1d on resolve actions and is vulnerable to spirit interference. \\

\textbf{Guttershadow Mark} (Scene) &
Darkness clings to the marked; stealth against them gains +1d and their actions produce unnatural noise. \\

\textbf{Witchblight Mark} (Moderate) &
An affliction that sours fate. Target rolls one additional SB on any 1-result. \\

\textbf{Ruinblood Mark} (Lingering) &
Blood carries Malachai’s whisper. Target suffers --1 Position on actions involving mercy, patience, or restraint. \\

\textbf{Hollow Echo Mark} (Greater) &
Echo is thinned; the target cannot benefit from aid actions or teamwork until the mark is broken. \\

\textbf{Possession Brand} (Greater, Momentary) &
A spirit gains purchase; for one round it may act using the target’s body with +1 Effect. Afterwards, the target marks 2 Fatigue. \\

\textbf{Canker Crown} (Lingering, Severe) &
A malignant mark that feeds on fear. Each time the target hesitates or falters, the GM may increase a Doom clock associated with them. \\

\bottomrule
\end{tabular}
\end{center}

\subsubsection*{Unbinding Hostile Marks}
Hostile Marks require a DV equal to their Strength tier +2.  
Coven assistance reduces DV by --1 for each additional witch aiding (max --3).

\subsection*{Advanced Hostile Marks (Malachai’s Domain)}

\begin{itemize}[leftmargin=1.5em]
    \item \textbf{Broken Moon Mark}  
    Lycanthropic influence crawls through the target’s echo.  
    \emph{Effect:} +1 Effect to violence; uncontrollable frenzy on critical failures.

    \item \textbf{Red Hunger Mark}  
    A vampiric stain.  
    \emph{Effect:} Target gains a predatory urge that must be indulged each scene or take --1d.

    \item \textbf{Doomwrit Mark}  
    A fragment of Malachai’s prophecy etched into flesh.  
    \emph{Effect:} A specific fate becomes inevitable unless removed.

    \item \textbf{Unbody Mark}  
    Physical form loosens.  
    \emph{Effect:} Target takes +1 Harm from silver, iron, or truth-speaking.
\end{itemize}

\newpage

\section*{Powerful Ritual Marks: Coven Workings}

Ritual Marks require:
\begin{itemize}[noitemsep]
    \item A coven of 3 or more witches.
    \item A prepared threshold (crossroads, grove, tidepool, shrine, gravehouse).
    \item A shared Cost: each witch contributes a Simple, Moderate, or Solemn cost.
\end{itemize}

\textbf{DV:} Base 4 + Ritual Tier  
\textbf{Coven Bonus:} Each assisting witch grants +1d and --1 DV (minimum DV 2).

\subsection*{Ritual Tiers}

\begin{center}
\begin{tabular}{p{0.25\textwidth} p{0.55\textwidth}}
\toprule
\textbf{Tier} & \textbf{Description} \\
\midrule
Lesser Rite & Scene-long blessings, minor shapings, protection marks. \\
Grand Rite & Community-scale marks, oath-bindings, curses on estates. \\
Ascendant Rite & Permanent transformations, monstrous ascensions, doom-forging. \\
\bottomrule
\end{tabular}
\end{center}

\subsection*{Ritual Marks by Order}

\subsubsection*{Order of Ikasha: The Silent Veil Circle}

\begin{itemize}[leftmargin=1.5em]
    \item \textbf{Veilshroud Rite} (Lesser)  
    A community or location becomes difficult to remember or describe.  
    \emph{Effect:} +1d to all stealth or secrecy in the shrouded domain.

    \item \textbf{Shadow Covenant Mark} (Grand)  
    Binds an individual to Ikasha’s unseen ledger.  
    \emph{Effect:} They must keep a sworn secret or suffer Wound Marks.
\end{itemize}

\subsubsection*{Order of Livaea: The Velvet Chorus}

\begin{itemize}[leftmargin=1.5em]
    \item \textbf{Heartmirror Rite} (Lesser)  
    A chosen person sees their truest desire when looking upon the marked.

    \item \textbf{Silver Tongue Seal} (Grand)  
    Ensnares a noble or leader.  
    \emph{Effect:} +1 Position for persuasion involving them; betrayal triggers Doom.
\end{itemize}

\subsubsection*{Order of Raéyn: The Tidebound Circle}

\begin{itemize}[leftmargin=1.5em]
    \item \textbf{Stormskin Rite} (Lesser)  
    Grants temporary invulnerability to wind, rain, and fear.

    \item \textbf{Tide-call Mark} (Grand)  
    Echo tied to shifting waters.  
    \emph{Effect:} Seasons and tides influence the target’s fate rolls.
\end{itemize}

\subsubsection*{Order of Aveh: The Maskless Conclave}

\begin{itemize}[leftmargin=1.5em]
    \item \textbf{Unname Rite} (Lesser)  
    The marked’s name cannot be spoken by foes.

    \item \textbf{Mask of Many Faces} (Grand)  
    A covenant of identity-shifting.  
    \emph{Effect:} Target may assume a new persona each scene with mechanical benefit.
\end{itemize}

\subsubsection*{Hearth Witches: The Circle of Warm Ash}

\begin{itemize}[leftmargin=1.5em]
    \item \textbf{Homestone Blessing} (Lesser)  
    A dwelling is marked against ill fate; +1 Effect on protective actions inside.

    \item \textbf{Ancestor’s Burden Mark} (Grand)  
    The coven invokes an ancestral echo to share grief.  
    \emph{Effect:} Reduces Doom on the household; increases SB on conflict.
\end{itemize}

\subsubsection*{Order of Malachai: The Crimson Tribunal}

Malachai’s covens are the most feared practitioners of Ritual Marks.  
Their marks reshape flesh, fate, and sanity.

\begin{itemize}[leftmargin=1.5em]
    \item \textbf{Beast-soul Rite} (Lesser)  
    Controlled lycanthropy.  
    \emph{Effect:} Target gains a predatory form for one scene.

    \item \textbf{Bloodchain Covenant} (Grand)  
    Binds two or more echoes in shared hunger.  
    \emph{Effect:} Harm dealt or taken echoes between linked individuals.

    \item \textbf{Crown of the Crimson Angel} (Ascendant)  
    Malachai’s signature ritual.  
    \emph{Effect:} The marked becomes an avatar of affliction:  
    immune to fear, bound to duty, empowered by suffering.  
    DV increases to 6; failure risks catastrophic echo-collapse.

    \item \textbf{Nightheart Transfiguration} (Ascendant)  
    A rite of monstrous ascension—vampiric or chthonic.  
    \emph{Effect:} Permanent mutation of form and destiny;  
    GM may assign a new Mark Nature unique to the transformation.
\end{itemize}

\subsection*{Breaking Ritual Marks}

Ritual Marks require:
\begin{itemize}[noitemsep]
    \item A coven of equal or greater size,
    \item A threshold of opposite nature,
    \item A Solemn Cost from each participating witch.
\end{itemize}

DV equals the original ritual’s DV +1.

\section*{Mark Complications and the Costs of Markwork}

Marks shape echo, but echo pushes back.  
Whether blessing or curse, each mark contains risk, debt, or consequence.

\subsection*{Complications of Markwork}

Whenever a mark is imposed with a partial success, dramatic failure,
unstable threshold, or hostile witness, roll or choose from the table below.

\begin{center}
\begin{tabular}{p{0.27\textwidth} p{0.6\textwidth}}
\toprule
\textbf{Complication} & \textbf{Effect} \\
\midrule

\textbf{Echo Recoil} &
The witch suffers a brief echo-backlash; mark 1 Fatigue or lose 1d on the next Working. \\

\textbf{Mark Drift} &
The mark begins shifting between Natures (e.g., blessing → veil, burden → wound). \\

\textbf{Spirit Entanglement} &
A lesser spirit latches onto the mark; it may whisper, demand, or interfere later. \\

\textbf{Threshold Resonance} &
The threshold awakens an Instinct (Call, Reject, Echo, Bind, Stain). Resolve immediately. \\

\textbf{Unstable Lattice} &
The mark pulses unpredictably:  
once per scene the GM may trigger a minor effect—strengthen or weaken. \\

\textbf{Harmonic Bleed} &
The mark unintentionally affects a nearby person, object, or spirit. \\

\textbf{Burden Surge} &
Costs rise sharply; the witch must elevate their Cost one tier (Simple → Moderate → Solemn). \\

\textbf{Mask of Misrule} &
For Veil, Desire, or Identity marks: the target behaves contrary to intent for one action. \\

\textbf{Doom Echo} &
For Wound or Binding marks: advance a Doom Clock tied to either witch or target. \\

\textbf{Familiar Hunger} &
The mark draws attention from a powerful entity aligned with its Nature. \\

\textbf{Half-Broken Seal} &
Mark does not bind cleanly; opposing forces may exploit the gap (spirit, curse, rival witch). \\

\textbf{Fading Imprint} &
Mark begins degrading early; duration is shortened one step unless reinforced. \\

\bottomrule
\end{tabular}
\end{center}

\subsection*{Cosmic Consequences (Optional Severe Complications)}

Used when imposing Greater, Lingering, or Ritual Marks.

\begin{itemize}[noitemsep]
    \item \textbf{Shattered Echo}: Target permanently loses a facet of identity, memory, or emotion.
    \item \textbf{Vein of Madness}: Target gains a recurring compulsion or altered instinct.
    \item \textbf{Beast’s Claim}: Malachai’s influence strengthens; physical mutation manifests.
    \item \textbf{Shadow Annexation}: Ikasha collects a portion of the target’s shadow.
    \item \textbf{Fate Rerooted}: Raéyn, Livaea, Aveh, or Malachai redirects the target’s destiny.
\end{itemize}

These are rare and usually arise only with Ascendant or Hostile Ritual Marks.

\subsection*{Coven Mechanics: Shaping the Unanointed Curse}
\index{Witchcraft!Coven Mechanics}

\begin{tcolorbox}[colback=shadowpurple!5,colframe=shadowpurple!80!black,
title={\textbf{Coven Workings and the Curse of the Unanointed}}]

A coven may manipulate the Curse affecting an initiate.  
These workings are not Rites but structured group effects.

\subsubsection*{Dampen the Curse (Supportive Work)}
\textbf{Cost:} 1 Downtime Action per coven member  
\textbf{Roll:} Spirit + Witchcraft (Controlled, DV 2)

\textbf{Effect:}
\begin{itemize}[leftmargin=1.5em]
    \item Reduce holy backlash for one session
    \item Remove the first 1 Fatigue the Curse would cause
    \item The GM gains 0 SB from this Curse for that session
\end{itemize}

\emph{On a Partial:}  
The Curse shifts; roll twice next time it triggers.

\emph{On a Miss:}  
Backlash hits the entire coven (1 Fatigue each).

\vspace{1em}

\subsubsection*{Strengthen the Curse (Weaponized Work)}
\textbf{Cost:} Witch blood, sacrifice, or stolen relic  
\textbf{Roll:} Desperate, DV 5

\textbf{Effect:}
\begin{itemize}[leftmargin=1.5em]
    \item Curse flares violently around the target
    \item Holy wards react catastrophically
    \item Hunter sanctums treat the witch as a “demon presence”
\end{itemize}

\emph{On a Success:}  
The curse becomes a weapon. Next time it triggers, choose one:
\begin{itemize}
    \item Blind all foes in Near range  
    \item Shatter wards or seals  
    \item Cause 1 Harm to any zealot invoking holy power  
\end{itemize}

\emph{On a Miss:}  
The Patron intervenes and rewrites one Taboo.

\vspace{1em}

\subsubsection*{Hide the Mark (Veiling Work)}
\textbf{Cost:} Dedicating a familiar or personal relic  
\textbf{Roll:} Risky, DV 3

\textbf{Effect:}
\begin{itemize}[leftmargin=1.5em]
    \item Witch Hunters cannot automatically detect the Curse
    \item Divine effects treat the witch as “mundane” for one scene
\end{itemize}

\emph{On a Partial:} The veil flickers at dramatic moments.  
\emph{On a Miss:} The veil inverts—hunters sense the witch at Far range.

\end{tcolorbox}

\subsection{The Witch Hunters}
\index{Orders!Witch Hunters}
\index{Witch Hunters}

The Witch Hunters---known formally as the \textbf{Order of the Chain-Lantern}---stand as the masculine, destructive inversion of witchcraft.  
Where Witches weave with quiet threads, Hunters \emph{cut}.  
Where Witches make pacts, Hunters make \emph{oaths}.  
Where Witches empower place-spirits, Hunters \emph{scour} them.

To the Chain-Lantern, magic is a wound in the world that must be cauterized.  
Their tools are iron, fire, binding-words, and fear; their craft is the unmaking of all that witches bind together.

\paragraph{Doctrine}
\begin{itemize}
    \item \textbf{All Patrons are Lies}: Only the mortal will is true.
    \item \textbf{All Spirits Are Predators}: Even the small ones behind walls.
    \item \textbf{All Witches Are Threshold-Breakers}: Power invites corruption.
    \item \textbf{Purity Is Law}: Resolve, discipline, and the rejection of compromise.
\end{itemize}

\paragraph{Symbols \& Regalia}
\begin{itemize}
    \item Chains wrapped around forearms: reminders of self-binding.
    \item Lanterns with cold iron latticework: reveal witch-signs and glamours.
    \item Ash-grey mantles: dyed with the soot of burned covens.
    \item Tattoos of severed cords: each one earned through “purifications.”
\end{itemize}

\subsubsection{Hunter Techniques (Talents)}
\begin{description}
    \item[\textbf{Iron Will (2 XP)}]  
    You gain +1d when resisting Patron influence, glamours, or compulsion.

    \item[\textbf{Lantern Sight (4 XP)}]  
    Once per scene, reveal hidden witch-sign, lingering magic, or cord residue within 20 paces.

    \item[\textbf{Breaker’s Grip (4 XP)}]  
    When grappling or restraining a caster, they suffer --1d to all spellcasting until free.

    \item[\textbf{Cord-Severer (6 XP)}]  
    Once per session, negate one witch rite, hex, or charm as it is cast.

    \item[\textbf{Ash Discipline (6 XP)}]  
    If you have taken no Boons this session, gain +1 Position on all confrontations with witches.
\end{description}

\subsubsection{Hunter Implements}
\begin{itemize}
    \item \textbf{Cold Iron Chains}: Bind spirits, disrupt glamours; witches take +1 DV to resist.
    \item \textbf{Lantern of Ash-Glass}: Burns blue in the presence of Patron interference.
    \item \textbf{Null-Graven Mask}: Carved with sigils of erasure; grants immunity to illusions for one scene.
    \item \textbf{Oathblade}: A weapon inscribed with the Hunter’s vow; deals +1 Harm against spirit-touched foes.
\end{itemize}

\subsubsection{Hunter Rites (Anti-Rites)}
Unlike witches, Hunters have \textbf{no Patrons}.  
Their “rites” are destructive inversions—oaths, suppressions, nullifications.

\begin{description}
    \item[\textbf{Scour the Threshold}]  
    Strip a place of minor spirits; all witchcraft inside suffers --1d for the scene.

    \item[\textbf{Bind the Unseen}]  
    Trap a small or lesser spirit in iron for interrogation or banishment.

    \item[\textbf{Burn the Cord}]  
    Target witch must reroll their next successful magical action.

    \item[\textbf{Oath of Severance}]  
    A sworn promise that empowers the Hunter once per session: +1d against a chosen witch, Patron, or coven.

    \item[\textbf{Make the World Clean}]  
    A grim “ritual” requiring fire and ash: eradicate all witch marks and charms within a building or campsite.
\end{description}

\subsubsection{Internal Factions}
Even Hunters fracture under the weight of their absolutism:

\begin{itemize}
    \item \textbf{The Lantern-Pure}:  
    Seek total eradication of witch-orders; zealous, uncompromising.

    \item \textbf{The Cleavers of Night}:  
    Focus on destroying Patrons by disrupting their influence.

    \item \textbf{The Hearth-Breakers}:  
    Specialists who infiltrate communities to root out Hearth Witches.

    \item \textbf{The Sorrowed Hands}:  
    Former witches seeking to atone through service; the most ruthless of all.
\end{itemize}

\subsubsection{Rivalries \& Opposition}
\begin{itemize}
    \item \textbf{Hearth Witches}:  
    Hunters despise them most of all—magic that is gentle is still magic.

    \item \textbf{Mab's Thorn-Court}:  
    Their glamours are the Hunters’ greatest shame—they have been fooled before.

    \item \textbf{Morag’s Brood}:  
    Hunters kill them on sight; Morag prefers to harvest Hunters' fears in return.

    \item \textbf{Aveh’s Followers}:  
    Nothing enrages a Hunter more than the facelessness of Aveh’s disciples, whose identities cannot be “purified.”
\end{itemize}

\subsubsection{Adventure Seeds}
\begin{itemize}
    \item A Hunter warband arrives in a town protected by Hearth Witches.  
    The lanterns burn blue: something is hidden here.

    \item A single Hunter seeks the PCs, claiming a coven they befriended is controlled by a Patron they cannot perceive.

    \item A defector from the Order arrives begging for protection; their old brothers have sworn an Oath of Severance.

    \item An Oathblade has gone missing, stolen by a child—it activates a dormant anti-rite with catastrophic consequences.

    \item A Hunter cadre has bound a local spirit that was holding the forest in balance; everything is dying.
\end{itemize}

\medskip

\noindent The Witch Hunters are not merely enemies.  
They are the mirror the witch-orders fear to face:  
\textit{men who believe the world must be made smaller, simpler, cleaner—no matter how much must be cut away.}

\subsection{Talent Tree: The Order of the Chain-Lantern}
\index{Talents!Witch Hunters}

The Witch Hunters do not receive blessings from Patrons; their strength comes from 
\textbf{oaths, discipline, and the refusal of all bargains}.  
Their talents escalate from vigilance, to suppression, to the severing of magic itself.

\medskip

\begin{center}
\begin{tabular}{|p{2.8cm}|p{11cm}|}
\hline
\multicolumn{2}{|c|}{\textbf{Witch Hunter Talent Tree}} \\ \hline

\textbf{Tier I} &
\textbf{IRON WILL} (2 XP) \newline
Resist magical influence with +1d. Includes glamours, compulsions, whispers, and attempted Patron touches.
\newline\newline
\textbf{GRIM DISCIPLINE} (2 XP) \newline
When you roll to resist fear, corruption, or seduction, increase Position by one step. \\ \hline

\textbf{Tier II} &
\textbf{LANTERN SIGHT} (4 XP) \newline
Once per scene, reveal hidden cords, witch-signs, spirit trails, or illusions within near range.
\newline\newline
\textbf{BREAKER'S GRIP} (4 XP) \newline
When grappling a caster, they suffer \(-1\)d to all spellcasting until free. \newline
When grappling a spirit-touched foe, you gain +1 Position. \\ \hline

\textbf{Tier III} &
\textbf{ASH DISCIPLINE} (6 XP) \newline
If you have not taken any Boons this session, gain +1 Position and +1d on confrontations with witches or spirits.
\newline\newline
\textbf{SEVERING BIND} (6 XP) \newline
When you strike with cold iron, you may force a witch to reroll one successful magical action
(or negate a hex being cast). Once per scene. \\ \hline

\textbf{Tier IV} &
\textbf{NULL-BOUND MASK} (8 XP, Req: Lantern Sight) \newline
You may ignore illusions, glamours, and sensory manipulation for an entire scene. \newline
Additionally, you cannot be magically disguised or obscured.
\newline\newline
\textbf{SPIRIT-CLEAVER} (8 XP, Req: Breaker's Grip) \newline
Attacks with iron weapons deal +1 Harm to spirits, manifested Patrons, and witch familiars.  
On a Full Success, you may “anchor” a spirit in place for a moment, preventing movement. \\ \hline

\textbf{Tier V} &
\textbf{CORD-SEVERER} (10 XP, Capstone) \newline
Once per session, you may attempt to \emph{sever a magical cord} as it is formed: \newline
negate a rite, curse, glamour, pact-invocation, or Patron manifestation. \newline
Success: the cord is cut. \newline
Partial: effect is weakened but not fully negated. \newline
Miss: you provoke the Patron’s attention (GM move). 
\newline\newline
\textbf{IRON OATH ASCENDANT} (10 XP, Capstone) \newline
Declare a sworn enemy (a witch, coven, spirit, or Patron aspect). \newline
For the rest of the session:  
\begin{itemize}
    \item +1d on all rolls opposing them  
    \item +1 Position on confrontations involving them  
    \item Your iron tools count as sacred weapons of denial  
\end{itemize}
The oath may not be withdrawn without narrative consequence. \\ \hline

\end{tabular}
\end{center}

\medskip

\paragraph{Design Notes (Optional for GM)}
Witch Hunters advance by hardening themselves emotionally, morally, and spiritually.  
Every talent is a narrowing—of will, perception, or mercy.  
Their capstones are powerful but dangerous, inviting escalation and attention from Patrons.

\subsection*{Witch Hunters: The Order of the Burning Mark}
\index{Talents!Witch Hunters}

\begin{tcolorbox}[colback=flame!5,colframe=flame!60!black,
title={\textbf{Order of the Burning Mark} --- Talents}]

\textbf{Tier I: Core Talents}
\begin{itemize}[noitemsep,leftmargin=1.5em]
    \item \textbf{Sense the Sinner} ---  
    Automatically detect the \emph{Curse of the Unanointed} within Near range.

    \item \textbf{Wardbreaker’s Step} ---  
    +1d to bypass, shatter, or dispel witch wards and Lesser Rites.
\end{itemize}

\textbf{Tier II: Advanced Talents}
\begin{itemize}[noitemsep,leftmargin=1.5em]
    \item \textbf{Burning Mandate} ---  
    When you confront a cursed witch, gain \textbf{Dominant Position} on the first roll.

    \item \textbf{Chain of Edicts} ---  
    When restraining or punishing witches, +1 Effect and +2 Resistance.
\end{itemize}

\textbf{Tier III: Ascendant Talents}
\begin{itemize}[noitemsep,leftmargin=1.5em]
    \item \textbf{Saint’s Wrath} ---  
    Once per scene, project a holy aura: witches must resist or suffer \textbf{2 Fatigue} and lose Position.

    \item \textbf{Seal the Cord} ---  
    Prevent a witch from using Rites for one scene (DV 4 Resist to break).
\end{itemize}

\textbf{Paradox Talent}
\begin{itemize}[noitemsep,leftmargin=1.5em]
    \item \textbf{Anointed Scourge} ---  
    You become a living anti-threshold.  
    Lesser Rites unravel near you; Grand Rites suffer -1d.  
    Witches cannot hide their marks from you.
\end{itemize}

\end{tcolorbox}

\subsection{Spell-Suppression Gear \& Relics of the Chain-Lantern}
\index{Witch Hunters!Relics}
\index{Items!Spell-Suppression}

Witch Hunters do not wield miracles; they carry tools of interruption, binding, 
and the deliberate stilling of magic.  
None of these items create magic—they merely deny it.

\subsubsection*{Common Implements (Tier I-II)}

\begin{itemize}[leftmargin=*]
    \item \textbf{Ironmesh Hood} — Worn over the lamp. When lowered, grants +1 Position 
    to resist illusions, glamours, or sensory distortion. 
    Limits wearer’s peripheral vision.

    \item \textbf{Binder’s Chalk} — Marks a \emph{Lantern Line}.  
    Spells crossing it suffer \(-1\)d unless the caster succeeds on a Wits test (DV 3).

    \item \textbf{Cold-Iron Shackles} — Prevent fine motor casting.  
    While bound, casters suffer \(-1\)d and cannot activate rites requiring gestures.

    \item \textbf{Lantern Salt} — Thrown to reveal invisible or hidden magical effects 
    in a small area.  
    On a Full Success: reveals hidden cords, spirits, illusions.  
    On Partial: reveals only the strongest effect.

    \item \textbf{Stop-Rite Token} — A stamped writ of interruption.  
    Once per scene, impose a DV +1 penalty on a witch beginning a rite.
\end{itemize}

\subsubsection*{Field Tools (Tier II-III)}

\begin{itemize}[leftmargin=*]
    \item \textbf{Ashbrand Nails} — Driven into wood or stone to “ground” magic.  
    For as long as the nail remains, any ongoing magical effect in Near range weakens 
    (GM: -1 Effect or similar suppression).

    \item \textbf{Chain of the Second Witness} — A short chain etched with law-script.  
    When wrapped around an object, anyone attempting to magically move or tamper with 
    it must beat DV 4.  
    Failure alerts the Hunter.

    \item \textbf{Mirror-Bell Vial} — A small bell sealed under glass.  
    When opened, the next magical effect in Close range is “reflected”:  
    the caster rolls with \(-1\)d and gains +1 Heat with local Orders.

    \item \textbf{Lantern-Wax Seal} — A wax pressed onto a door or object.  
    Until broken, rites that cross the threshold suffer \(-2\) Effect.
\end{itemize}

\subsubsection*{Rare Implements (Tier III-IV)}

\begin{itemize}[leftmargin=*]
    \item \textbf{True-Iron Censer} — Emits a bitter smoke.  
    Spirits entering the smoke suffer \(-1\) Position and cannot hide their form.  
    Witches lose access to one minor hex or trick for the scene.

    \item \textbf{Countervoice Lantern} — Emits a tone when magic is cast nearby.  
    Once per scene, the bearer may impose a forced re-roll on a witch’s casting roll.

    \item \textbf{The Ledger of Unpriced Gifts} — A heavy iron-bound log.  
    Any boon or gift acquired by magic within a scene must be “priced” (GM: choose 
    a small complication or consequence).  
    Witches despise the Ledger.

    \item \textbf{Severing Tongs} — Used to remove cursed objects safely.  
    Grant +1 Position to seize, isolate, or interrupt an active curse.
\end{itemize}

\subsubsection*{Relics of the Chain-Lantern (Unique / Tier V)}

\begin{itemize}[leftmargin=*]
    \item \textbf{The First Lamp} —  
        A relic said to burn with the light of the earliest oath.  
        \begin{itemize}
            \item Once per session: negate a magical effect as it is cast.  
            \item Witches who approach within Near range tremble; \(-1\)d to cast.  
            \item Patrons become aware of its use.
        \end{itemize}

    \item \textbf{The Iron Cartographer} — A slate that maps magical currents.  
        \begin{itemize}
            \item Reveal every ongoing magical effect in Far range.  
            \item Name one cord or working; the bearer can follow it unerringly.  
            \item If the bearer lies while using it, the slate cracks.
        \end{itemize}

    \item \textbf{The Lantern of Nine Chains} —  
        Used only by senior Hunters.  
        \begin{itemize}
            \item Once per session: bind a witch or spirit in place, freezing 
                  movement for a few heartbeats.  
            \item If used on a witch under a Patron’s direct gaze, the Patron may 
                  act in reprisal (GM move).
        \end{itemize}

    \item \textbf{The Helm of the Third Witness} —  
        A dull iron helm that sees no lies.  
        \begin{itemize}
            \item Automatically pierces illusions, glamours, shadow-shifting, 
                  and fae misdirection.  
            \item Cannot be removed until a confession is heard.  
            \item Wearing it too long drains the wearer’s empathy (GM move).
        \end{itemize}
\end{itemize}

\subsubsection*{Forbidden Implements (GM Option)}
These items have been banned by every Order, yet some Hunters seek them anyway.

\begin{itemize}[leftmargin=*]
    \item \textbf{The Sorrowed Brand} — Burns a witch’s name into iron.  
          Grants automatic success on tracking that witch.  
          Using it earns enmity from all Orders.

    \item \textbf{Hallowed Silencer} — A cold-iron muzzle for spirits.  
          Prevents all speech or spell-voice.  
          Using it is considered torture.

    \item \textbf{The Empty Grimoire} — Absorbs one hex or rite per scene.  
          What happens to the stolen magic is unknown.
\end{itemize}

\section{Witch’s Familiars}
\label{sec:familiars}

Familiars are not pets. They are coven-bound companions, witnesses to Name-work,
and extensions of a witch’s intent. A familiar is half-spirit, half-creature,
shaped by bargain, bond, and circumstance. This system expands followers to
accommodate the mystical role familiars serve in witchcraft.

\subsection{Familiar Principles}

\begin{itemize}[leftmargin=1.5em]
    \item A familiar is a \textbf{follower augmented by cords and spirit-law}.
    \item A familiar grows through \textbf{shared scenes}, not itemized XP.
    \item A familiar is a \textbf{Voice of the Patron}, but not a servant.
    \item A familiar’s death or corruption has \textbf{ritual and narrative weight}.
\end{itemize}

\subsection{Familiar Creation}

To create a familiar, choose:

\paragraph{1. Nature (One)}
\begin{itemize}[leftmargin=1.8em]
    \item \textbf{Beast-Born:} Cat, fox, raven, serpent, owl, goat, etc.
    \item \textbf{Spirit-Born:} Lantern-wisp, dustling, ember-lark, river-whisper.
    \item \textbf{Shade-Touched:} Pale animals, two-eyed shadows, echo-creatures.
\end{itemize}

\paragraph{2. Temperament (One)}
\begin{itemize}[leftmargin=1.8em]
    \item \textbf{Watcher} (observant, silent)  
    \item \textbf{Trickster} (curious, chaotic)  
    \item \textbf{Caretaker} (empathetic, stabilizing)  
    \item \textbf{Predator} (protective, vengeful)
\end{itemize}

\paragraph{3. Patron Resonance (One)}
\begin{itemize}[leftmargin=1.8em]
    \item \textbf{Lunera:} moon-sight, calm tides, dream-walks  
    \item \textbf{Morag:} rot-sense, curse-binding, hunger-echo  
    \item \textbf{Mab:} glamours, threshold-crossing, illusion  
    \item \textbf{Aveh:} identity-shift, mask-binding, shadow-step  
    \item \textbf{Hearth Spirits:} warm-spot, ember-glow, soot-warding  
    \item \textbf{Thorns of Malachai:} temptation, golden-hook, bargain-sense  
\end{itemize}

\subsection{Familiar Statline}
Familiars function as followers, but with the following adjustments:

\begin{itemize}[leftmargin=1.5em]
    \item \textbf{Cap:} 2 + Witch Tier (max 5)
    \item \textbf{Scale:} Tiny (unless Beast-Born Predator, then Small)
    \item \textbf{Tags:} Nature, Temperament, Patron Resonance
    \item \textbf{Harm:} 2 boxes; when filled, familiar enters \textbf{Ritual Crisis}
\end{itemize}

\subsubsection*{Familiar Moves (choose 2)}
\begin{itemize}[leftmargin=1.5em]
    \item \textbf{Sense the Unspoken} — reveal hidden emotions, curses, or Name-weight in a scene.
    \item \textbf{Slip-Between} — bypass a barrier, door, or gap too small for a person.
    \item \textbf{Lend Strength} — assist the witch; grants +1d instead of normal follower assist.
    \item \textbf{Spirit-Anchor} — stabilize a Working, reducing SB generated by 1.
    \item \textbf{Echo-Strike} — attack as a Tier I predator (Cap 1 strike).
    \item \textbf{Record the Price} — whisper a consequence the witch is overlooking.
\end{itemize}

\subsection{Familiar Advancement}

Each time the witch and familiar share a moment of vulnerability, growth, or
ritual significance, mark a \textbf{Bond}. At 3 Bonds, choose one:

\begin{itemize}[leftmargin=1.5em]
    \item Gain a new Familiar Move
    \item Increase familiar Cap by 1 (max 5)
    \item Unlock a \textbf{Patron Gift}
\end{itemize}

\subsection{Patron Gifts for Familiars}

\paragraph{Lunera’s Gift: Moon-Thread}  
Once per scene, reveal a hidden path or safe option.

\paragraph{Morag’s Gift: Mirecloth}  
Familiar can inflict a \emph{Weakness} (Rot, Fear, or Binding) for a single roll.

\paragraph{Mab’s Gift: Glamour-Bloom}  
Familiar projects a simple illusion; grants Position +1 for a social or stealth action.

\paragraph{Aveh’s Gift: Unmasking}  
Familiar shifts form slightly, revealing a truth someone is hiding.

\paragraph{Hearth’s Gift: Ember-Shield}  
Reduce incoming Harm to familiar or witch by 1 (once/session).

\paragraph{Malachai’s Thorn-Gift (Forbidden)}  
Familiar grants +1 Effect \emph{and} GM gains 2 SB.

\subsection{Familiar Harm \& Ritual Crisis}

When the familiar’s Harm track fills, it cannot die like mortals.  
Instead, it enters a \textbf{Ritual Crisis}:

\begin{itemize}[leftmargin=1.5em]
    \item It vanishes into smoke, glass, ash, or echoes.
    \item Witch must perform a \textbf{Rebinding Rite} (DV 3--5 depending on Patron).
    \item On a Miss: familiar returns changed—add a new Tag (Shadowed, Hungry, Hooked).
\end{itemize}

\subsection{Familiar Special Actions}

\paragraph{Witch-Focus (1/session)}  
The witch channels their Working through the familiar:  
\textbf{+1 Effect}, but familiar risks 1 Harm.

\paragraph{Echo-Reversal}  
Familiar reroutes backlash from the witch to itself.  
Witch avoids consequence; familiar marks 1 Harm and GM gains 1 SB.

\paragraph{Name-Whisper (Patron-specific)}  
Once per arc, familiar whispers the \emph{true Name} of a minor spirit, entity, or place.  
This always attracts attention.

\subsection{Familiar Corruption}

If the witch takes Thorn Seeds (see Hybrid Possession rules), the familiar is affected.

At each Thorn Stage:

\begin{itemize}[leftmargin=1.5em]
    \item Stage 1: eyes glint with gold  
    \item Stage 2: speech becomes plural (“we”)  
    \item Stage 3: familiar’s Move gains +1 Effect, but GM gains 1 SB when used  
    \item Stage 4: familiar begins offering prices unbidden  
    \item Stage 5: familiar becomes a Seraphic echo (hostile follower, Cap 3)  
\end{itemize}

Cleansing requires both a cord-rite and a summoning diagram.

\subsection{Adventure Seeds Involving Familiars}

\begin{itemize}[leftmargin=1.5em]
    \item \textbf{The Familiar Market} — A black market sells “rebonded” familiars; their former witches want them back.
    \item \textbf{Ash-Threaded Cat} — A Hearth witch’s familiar has begun foretelling fires that have not yet happened.
    \item \textbf{Three-Eyed Raven} — A Mab-born familiar witnesses a murder in a dream but cannot speak it plainly.
    \item \textbf{The Silent Cub} — A Morag-touched wolf pup refuses to eat, as if carrying a curse meant for someone else.
    \item \textbf{The Glass Fox} — An Aveh familiar splits into reflections—each reflecting a different possible betrayal.
\end{itemize}

\subsection{Sample Familiar Statblocks}
\label{sec:familiar-statblocks}

These baseline statlines follow the familiar rules (Cap = 2 + Witch Tier),
modified by Nature and Patron Resonance.

\begin{creaturebox}{Fox-Familiar (Beast-Born Trickster)}
\textbf{Nature:} Beast-Born \quad \textbf{Temperament:} Trickster  \\
\textbf{Cap:} 3 \quad \textbf{Scale:} Tiny \\
\textbf{Tags:} Keen-Scent, Nimble, Mischief  
\textbf{Moves:}
\begin{itemize}[leftmargin=1.5em, noitemsep]
    \item \textbf{Slip-Between} — bypass any gap wider than a hand-width.
    \item \textbf{Sense the Unspoken} — detect emotional truths or concealed intent.
    \item \textbf{Fox’s Feint} — grant Position +1 on a stealth or distraction roll.
\end{itemize}
\textbf{Weakness:} Easily baited; loud sudden noises startle it.
\end{creaturebox}

\begin{creaturebox}{Raven-Familiar (Beast-Born Watcher)}
\textbf{Nature:} Beast-Born \quad \textbf{Temperament:} Watcher \\
\textbf{Cap:} 3 \quad \textbf{Scale:} Tiny \\
\textbf{Tags:} Omens, Sharp-Eyes, Sky-Bound  
\textbf{Moves:}
\begin{itemize}[leftmargin=1.5em, noitemsep]
    \item \textbf{Record the Price} — whisper a looming consequence the witch has missed.
    \item \textbf{Moon-Voice} — mimic any sound heard this scene.
    \item \textbf{Omen-Cry} — once/scene, reveal whether a direction is auspicious or cursed.
\end{itemize}
\textbf{Weakness:} Distracted by reflections and shiny objects.
\end{creaturebox}

\begin{creaturebox}{Shade-Cat (Shade-Touched Lurker)}
\textbf{Nature:} Shade-Touched \quad \textbf{Temperament:} Watchful-Predator \\
\textbf{Cap:} 3 \quad \textbf{Scale:} Tiny \\
\textbf{Tags:} Silent-Paws, Second-Shadow, Mist-Born  
\textbf{Moves:}
\begin{itemize}[leftmargin=1.5em, noitemsep]
    \item \textbf{Shadow-Bloom} — become invisible in dim light; +1d to ambush assists.
    \item \textbf{Spirit-Anchor} — stabilize a Working, reducing SB generated by 1.
    \item \textbf{Echo-Strike} — Tier I spirit-claw; counts as \emph{spirit harm}.
\end{itemize}
\textbf{Weakness:} Sunlight dazes it; Position –1 in bright light.
\end{creaturebox}

\begin{creaturebox}{Wisp-Familiar (Spirit-Born Caretaker)}
\textbf{Nature:} Spirit-Born \quad \textbf{Temperament:} Caretaker \\
\textbf{Cap:} 4 (cannot attack) \quad \textbf{Scale:} Tiny \\
\textbf{Tags:} Lantern-Glow, Gentle, Guiding  
\textbf{Moves:}
\begin{itemize}[leftmargin=1.5em, noitemsep]
    \item \textbf{Lend Strength} — assist for +1d instead of normal follower assist.
    \item \textbf{Calm-Tide} — reduce one Fatigue from the witch (once/session).
    \item \textbf{Wisp-Guide} — reveal a safe path; Position +1 for navigation rolls.
\end{itemize}
\textbf{Weakness:} Disrupted by wind or loud conflict.
\end{creaturebox}

\subsection*{Familiar Power Tree: The Marked Bond}
\index{Familiars!Marked Bond}

\begin{tcolorbox}[colback=shadowpurple!5,colframe=shadowpurple!80!black,
title={\textbf{The Marked Bond} --- Talents for Cursed Familiars}]

\textbf{Tier I: Awakening Talents}

\begin{itemize}[noitemsep,leftmargin=1.5em]
    \item \textbf{Eyes of the Mark}  
    Your familiar perceives traces of the Curse.  
    \emph{Effect:} Once per scene, ask the GM one question about  
    holy forces, wards, zealots, or cursed places.  
    On a 6, your Position improves.

    \item \textbf{Shadow-Tether}  
    When your familiar aids you, add \textbf{+1d} if the action involves  
    secrecy, trespass, or forbidden magic.  
    If you roll a 1, the Curse gains 1 SB.
\end{itemize}

\vspace{0.5em}

\textbf{Tier II: Empowered Talents}

\begin{itemize}[noitemsep,leftmargin=1.5em]
    \item \textbf{Hex-Eater}  
    Your familiar can swallow Lesser Rites or minor curses.  
    \emph{Effect:} Once per rest, cancel any 1-tier magical effect.  
    \emph{Cost:} Your familiar takes 1 Harm (or Fatigue if intangible).

    \item \textbf{Borrowed Claws}  
    Invoke the familiar’s Form to add magical potency.  
    \emph{Effect:} Increase Effect by +1 on attacks or occult actions.  
    \emph{Cost:} Mark 1 Fatigue; the Curse flares visibly.
\end{itemize}

\vspace{0.5em}

\textbf{Tier III: Ascendant Talents}

\begin{itemize}[noitemsep,leftmargin=1.5em]
    \item \textbf{Soul-Doubling}  
    Familiar and witch act as two halves of one threshold-being.  
    Once per scene, take a second action at -1d  
    (your familiar “acts” while your body hesitates).  
    If the roll fails, the Curse manifests dramatically.

    \item \textbf{Cursed Apotheosis}  
    Your familiar becomes a minor spirit of your Patron or your Curse.  
    \emph{Effect:}  
    \begin{itemize}
        \item Familiar gains 1 additional action per scene  
        \item Familiar ignores mundane harm  
        \item Witch Hunters recognize it immediately and may escalate clocks
    \end{itemize}
\end{itemize}

\vspace{0.5em}

\textbf{Paradox Talent}

\begin{itemize}[noitemsep,leftmargin=1.5em]
    \item \textbf{One Soul, Two Shadows}  
    You and your familiar share fate.  
    When either of you would take Harm, choose who suffers it.  
    When you would die, your familiar may offer its soul instead.  
    \emph{Cost:} GM chooses a permanent Scar or Taboo.
\end{itemize}

\end{tcolorbox}

\Appendix
\section*{The Confluence Path: Summoner--Witch Hybrids}
\label{hybrids:confluence}

Some witches walk the boundary between \emph{invitation} and \emph{incursion}.  
They braid the soft, relational cords of witchcraft with the sharp geometries of summoning.  
The Confluence Path is neither an Order nor a school: it is a mistake, a temptation, or a  
dangerous inheritance. Those who follow it call themselves \textbf{Confluence-Binders},  
\textbf{Threshold-Blooded}, or \textbf{Nameshapers}. Others simply call them \textbf{ill-omened}.

\subsection*{Why This Path Exists}
Traditional witchcraft bargains with spirits through \emph{names, offerings, and reciprocity}.  
Summoners compel entities through \emph{sigils, contracts, and geometry}.  
A Confluence-Binder attempts both at once, claiming:

\begin{itemize}[leftmargin=1.5em]
    \item witches know the \emph{world's soft places}  
    \item summoners know the \emph{world's hard edges}
    \item together, these form a \emph{gateway}
\end{itemize}

Their Workings blur the line between pact and possession, boon and binding.

\subsection*{Core Risk: Dual Obligation}
A hybrid never serves one master.  
They owe:
\begin{itemize}[leftmargin=1.5em]
    \item a \textbf{Patron Want} (witchcraft)  
    \item a \textbf{Summoned Will} (contract)
\end{itemize}

These obligations frequently contradict.  
Whenever they attempt a Confluence Working, mark 1 SB for the GM.  
Each 1 on the roll marks an additional SB, which may be spent to:

\begin{itemize}[leftmargin=1.5em]
    \item awaken an unintended spirit  
    \item twist the summoned entity’s interpretation of its contract  
    \item reshape local reality around a forgotten Name  
\end{itemize}

\subsection*{Hybrid Practices}
Confluence-Binders combine methods:

\paragraph{1. Cord-Sigils}  
Every cord is knotted through a sigil; every sigil is softened by cord-law.  
Effect: When invoking or commanding spirits, gain +1d, but on a Miss the spirit acts with
\textbf{uninterpretable obedience}, creating harm or havoc.

\paragraph{2. Threshold Familiars}  
A hybrid’s familiar is neither bound nor free.  
It has two moods:
\begin{itemize}
    \item \textbf{Witch-Mood}: protective, relational, curious  
    \item \textbf{Summoner-Mood}: literal, predatory, geometric
\end{itemize}
Switching modes requires a DV 3 Spirit + Empathy test.

\paragraph{3. Blood-Price Substitution}  
If a summoning demands a sacrifice they cannot pay, hybrids may convert the price into  
\emph{cord-burn}: they lose one minor Name until restored.

\subsection*{New Hybrid Rites}
\begin{itemize}[leftmargin=1.5em]
    \item \textbf{Knot the Host} (Lesser)  
    Anchor a spirit half-in, half-out of a chosen object.  
    Allies gain +1 Effect interacting with the spirit; adversaries suffer Position --1.

    \item \textbf{Whisper the True Threshold} (Greater)  
    Declare a line across which supernatural entities must test to cross (DV = Tier).  
    If they fail, they manifest weakened or fragmented.

    \item \textbf{Mirror of Two Mouths} (Grand)  
    Allow the witch to speak with a spirit \emph{and} the spirit to speak through them.  
    Consequence: a lingering echo haunts the witch’s speech until resolved.
\end{itemize}

\subsection*{Talents of the Confluence Path}
\begin{tcolorbox}[colback=shadowgray!8,colframe=shadowgray!60,title={Confluence-Binder Talents}]
\textbf{Tier I: Threshold Initiate}
\begin{itemize}[noitemsep]
    \item \textbf{Two Ways to Name} --- Gain +1d when negotiating with spirits or summoned beings.
\end{itemize}

\textbf{Tier II: Cord-Geomancer}
\begin{itemize}[noitemsep]
    \item \textbf{Bind the Breath} --- When you establish a boundary, you may also impose a minor taboo.
\end{itemize}

\textbf{Tier III: Echo Warden}
\begin{itemize}[noitemsep]
    \item \textbf{Twin Intent} --- Use two forms of authority at once (witch Name + sigil command).
\end{itemize}

\textbf{Paradox Talent}
\begin{itemize}[noitemsep]
    \item \textbf{Gate-Walked} --- Your soul becomes a liminal space.  
    You always count as standing on a threshold, for better or worse.
\end{itemize}
\end{tcolorbox}

\subsection*{Hybrid Rivalries \& Enemies}
\begin{itemize}[leftmargin=1.5em]
    \item \textbf{Witch Orders} distrust the coercive edge of summoning.
    \item \textbf{Summoning Cabals} hate the softness of witch bargains.
    \item \textbf{Witch Hunters} see hybrids as proof that witchcraft is dangerous.
    \item \textbf{The Thorns of Malachai} consider hybrids ``the perfect vessel.’’
\end{itemize}

\subsection*{Adventure Seeds}
\begin{itemize}[leftmargin=1.5em]
    \item \textbf{The Double Oath} — A hybrid’s Patron Want and a spirit-contract collide, risking a local catastrophe.
    \item \textbf{The Familiar That Dreamed} — A hybrid’s familiar begins manifesting impossible geometry.
    \item \textbf{The Unfinished Summoner} — A dead summoner’s diagram begins pulling witches inside its logic.
    \item \textbf{Threshold Feast} — A hybrid accidentally opened a doorway that refuses to close.
\end{itemize}

\section*{The Thorns of Malachai: Hybrid Possession Subsystem}
\label{malachai:hybridpossession}

Hybrid witches---those who blend cord-law with summoning geometry---are uniquely
vulnerable to the Thorns of Malachai. A cord is a promise; a sigil is a door.
Malachai’s servants do not enter violently. They enter through the smallest
\emph{mispriced gift}.

This subsystem governs how a Thorn insinuates itself into a hybrid’s Workings,
slowly binding the witch and the summoned entity into a single cursed host.

\subsection*{Core Principles}
\begin{itemize}[leftmargin=1.5em]
    \item Possession is \textbf{incremental}, not immediate.
    \item Malachai’s Thorns never seize agency; they \textbf{offer} it at a price.
    \item Hybrids are more susceptible because their Workings rely on 
          \textbf{dual Names} and \textbf{contradictory obligations}.
\end{itemize}

\subsection*{The Three Vectors of Thorn Infection}
Thorns exploit any opening where a witch’s cord-logic conflicts with their
summoned contract.

\begin{enumerate}[leftmargin=1.8em]
    \item \textbf{Price Misalignment}---the witch accepts a boon without naming its cost.
    \item \textbf{Name Fracture}---the hybrid invokes a spirit whose Name they cannot fully hold.
    \item \textbf{Geometry Drift}---the summoning diagram warps under emotional strain.
\end{enumerate}

Any time one of these occurs, the GM may introduce a \textbf{Thorn Seed}.

\subsection*{Thorn Seeds}
A Thorn Seed is the earliest stage of possession. Mark one when:
\begin{itemize}[leftmargin=1.5em]
    \item the hybrid gains a miracle or advantage they did not ask for  
    \item the hybrid rolls 2+ SB on a Confluence Working  
    \item a summoned being refuses dismissal, “waiting for payment”
\end{itemize}

\textbf{A Thorn Seed does not do harm.}  
It merely rewrites the cost of the hybrid’s future actions.

\subsection*{Escalation Track: The Hook-Bloom}
Track possession using the following ladder:

\begin{center}
\begin{tabular}{|c|p{9cm}|}
\hline
\textbf{Stage} & \textbf{Effect} \\
\hline
1. Thorn Seed & A whispering cost. Hybrid gains +1d on a Working; GM gains 1 SB. \\
\hline
2. Hook-Root & A sigil appears under the skin. When resisting harm, hybrid takes 1 Fatigue. \\
\hline
3. Bloomed Hook & Hybrid’s summoned entity becomes ``improved’’: +1 Effect, but any Miss invites intrusion. \\
\hline
4. Seraphic Murmur & Hybrid hears Malachai’s choir. Once per scene, GM may alter the Working’s interpretation. \\
\hline
5. Chained Avatar (Possessed) & Hybrid becomes a vessel. They act with +2d and +1 Effect but lose all Position benefits. \\
\hline
\end{tabular}
\end{center}

At Stage 5, the hybrid is not lost but is \emph{overtaken}.  
This creates a crisis rather than an immediate character death.

\subsection*{How Stages Advance}
Whenever the hybrid:
\begin{itemize}[leftmargin=1.5em]
    \item accepts an unpriced miracle  
    \item bargains under duress  
    \item calls a spirit without anchoring cord-law  
    \item tears a Name (using it as leverage rather than covenant)
\end{itemize}

---roll Wits + Spirit (DV = current Stage).  
On a Miss, advance one Stage.

\subsection*{The Hybrid’s Advantage: Double Binding}
Hybrids can resist Thorns in ways witchcraft alone cannot.

\paragraph{1. Sigil-Reversal}
Spend 1 Fatigue to invert a sigil mid-Working.  
Effect: Downgrade a Stage by 1 for the scene only.

\paragraph{2. Cord-Burn Purification}
Sacrifice one minor Name (temporarily lost).  
Effect: Permanently remove the most recent Stage.

\paragraph{3. Dual Oath Gambit}
Swear two mutually exclusive promises.  
If kept until sunset, the contradiction breaks the Hook.  
If broken, immediately advance to Stage 5.

\subsection*{Exorcisms for Hybrids}
Exorcisms require both a summoner’s geometry and a witch’s cord.

\begin{itemize}[leftmargin=1.5em]
    \item \textbf{Circle of Torn Names} (DV 4)  
    Requires a Name the hybrid regrets using.  
    Success: remove one Stage.  
    Miss: the Thorn manifests physically.

    \item \textbf{Cord of Honest Price} (DV 3)  
    The hybrid must name the cost they have refused.  
    Effect: freeze Stage advancement until dawn.

    \item \textbf{Seraphic Counter-Song} (DV 5)  
    Requires 3 witches or 3 summoners.  
    Success: purge all Stages, but hybrid loses one Talent (scarred).
\end{itemize}

\subsection*{Thorn-Manifested Familiars}
At Stage 3+, the hybrid’s familiar becomes an extension of the Thorn.

\begin{itemize}
    \item \textbf{Stage 3:} Familiar speaks in second person plural.  
    \item \textbf{Stage 4:} Familiar mirrors hybrid’s emotions with luminous bleeding eyes.  
    \item \textbf{Stage 5:} Familiar becomes a Seraphic echo; treat as a hostile follower (Cap 3) until exorcised.
\end{itemize}

\subsection*{Adventure Hooks Involving Hybrid Possession}
\begin{itemize}[leftmargin=1.5em]
    \item \textbf{The Silent Sigil} — A hybrid unknowingly carries a Thorn Seed into a ritual circle.  
    \item \textbf{The Broken Familiar} — A familiar has reached Stage 5 and seeks to break its witch.  
    \item \textbf{The Choir Beneath the Cord} — A Thorn Seed infects a coven and begins corrupting their Names.  
    \item \textbf{The Price Left Unspoken} — A seemingly benign miracle becomes the seed of a regional blooming.  
\end{itemize}

\subsection*{Using This Subsystem}
Use this only with hybrid PCs or witches dabbling in summoning.  
It provides a \emph{slow-burn corruption arc}, foreshadowing Malachai before the Thorns ever appear in person.

It also reinforces the Book of Shadows’ themes:

\begin{itemize}[leftmargin=1.5em]
    \item Temptation and cost  
    \item Power at a price  
    \item Witchcraft and summoning as dangerous when intertwined  
    \item Malachai’s influence as subtle, not overt  
\end{itemize}

\section*{Witch Trial Subsystem}
\index{Systems!Witch Trials}

\begin{tcolorbox}[colback=flame!3,colframe=inkblack,
title={\textbf{Witch Trials} --- Social Conflict at the Edge of Faith and Fear}]

Witch Trials are structured social conflicts representing councils, inquisitions,  
village gatherings, or temple hearings. They revolve around three central mechanics:

\[
\textbf{Evidence Clocks, Jury Disposition, and Spotlight Rounds}
\]

They resolve whether the accused is:
\begin{itemize}[leftmargin=1.5em]
    \item \textbf{Cleansed} (Released under conditions)
    \item \textbf{Condemned} (Execution, exile, or purification)
    \item \textbf{Claimed} by a Patron (a powerful but risky outcome)
\end{itemize}

\end{tcolorbox}

% ---------------------------------------------------------

\subsection*{Evidence Clocks}
Each Trial uses two opposed clocks:

\begin{itemize}[leftmargin=1.5em]
    \item \textbf{Condemnation [6]} — fear, testimony, the Curse flaring
    \item \textbf{Clemency [6]} — arguments, allies, evidence, confession
\end{itemize}

When either clock fills, the trial concludes:

\begin{description}
    \item[\textbf{Condemnation fills:}] Immediate punishment or ritual penance.  
    Witch Hunters may escalate directly into violence or binding Rites.

    \item[\textbf{Clemency fills:}]  
    The accused is spared but must accept a binding Oath, Taboo, or service.

\end{description}

\paragraph{Adding Segments}
\begin{itemize}
    \item Strong rhetoric: +1 to Clemency  
    \item Evidence of hexing: +1 to Condemnation  
    \item Witchcraft used during trial: +2 Condemnation, +1 Clemency  
    \item Patron signs or omens: GM adds 1–3 to either clock
\end{itemize}

% ---------------------------------------------------------

\subsection*{Jury Disposition Track}
A 5-step track representing the room’s mood:

\[
\textbf{Hostile — Uneasy — Divided — Sympathetic — Won Over}
\]

Disposition modifies Position:

\begin{itemize}[leftmargin=1.5em]
    \item Hostile: You begin rolls at \textbf{Desperate}
    \item Uneasy: Risky
    \item Divided: Risky (improve Position on 6)
    \item Sympathetic: Controlled
    \item Won Over: Controlled (+1d)
\end{itemize}

Disposition shifts via arguments, testimony, theatrics, or sabotage.

% ---------------------------------------------------------

\subsection*{Spotlight Rounds}
Each round, one PC acts as the advocate, one as the accused,  
and the rest may intervene through:

\begin{itemize}[leftmargin=1.5em]
    \item Whispered Assistance (Body + Stealth; help without being seen)
    \item Legal Rhetoric (Wits + Oratory)
    \item Emotional Plea (Spirit + Empathy)
    \item Evidence Reveal (Wits + Investigation)
    \item Patron Bargain (Spirit + Resolve; always risky)
\end{itemize}

Failure often adds to \textbf{Condemnation}.  
Success shifts the Jury and adds to \textbf{Clemency}.

% ---------------------------------------------------------

\subsection*{Special Moves}

\paragraph{The Witch’s Gambit}  
Reveal a secret or minor spell to sway the room.  
\emph{Effect:} +1 Clemency, but +1–2 Condemnation (GM’s choice).

\paragraph{Hunter’s Demand}  
Hunters may invoke holy authority to auto-fill 1 segment of Condemnation.

\paragraph{Coven Intervention}  
If the coven acts outside the court, roll a covert scene.  
On success: +1 Clemency.  
On failure: +2 Condemnation as news reaches the trial.

% ---------------------------------------------------------

\subsection*{Ritual Verdicts}
If a Patron intervenes:

\begin{itemize}[leftmargin=1.5em]
    \item \textbf{Lunera:} Moonlight reveals truth; the witch gains a taboo.  
    \item \textbf{Morag:} Hag-sign warps faces; Jury Disposition flips.  
    \item \textbf{Mab:} Glamour conceals guilt; Condemnation drops to 0.  
    \item \textbf{Malachai:} A price is extracted from all present.  
    \item \textbf{Aveh:} Identities blur; witch walks free but loses something vital.
\end{itemize}

\end{document}
