\section{Reviewer's Guide: Understanding Fate's Edge}

\textit{This will be removed in Production}

\subsection{Reframing the Evaluation}

Fate's Edge is not a traditional tabletop RPG to be evaluated by comprehensive rule mastery. Instead, it is a \textbf{narrative partnership} designed to evolve with your table's storytelling sophistication. Approach this review through the lens of \textbf{progressive storytelling enhancement} rather than static mechanical complexity.

\subsection{Key Evaluation Framework}

\subsubsection{Start Simple, Discover Naturally}
\begin{itemize}
    \item \textbf{Session 1}: Core mechanic + 1-2 regional generators = complete playable experience
    \item \textbf{Session 5}: Comfortable integration of multiple subsystems
    \item \textbf{Session 20}: Intuitive mastery where mechanics become invisible
\end{itemize}

\subsubsection{The Partnership Model}
Rather than "learning a game," players develop a "storytelling partnership" where:
\begin{itemize}
    \item Rules serve narrative consequences, not simulation accuracy
    \item Complexity emerges from interaction, not individual component sophistication  
    \item Prep burden shifts from authoring content to facilitating discovery
\end{itemize}

\subsection{Common Misunderstandings to Address}

\subsubsection{Content Volume vs. Discovery Process}
\textbf{Misconception}: "Too much content to master"
\textbf{Reality}: Content is designed for gradual discovery through play

\textbf{Reviewer Guidance}: 
\begin{quote}
"Imagine a library where you only need to know about the books relevant to today's story. The rest exist to be discovered when narrative ambition demands them."
\end{quote}

\subsubsection{Prep-Light vs. Prep-Free}
\textbf{Misconception}: "No prep means unstructured play"
\textbf{Reality}: Prep is embedded in the constraint lattice

\textbf{Reviewer Guidance}:
\begin{quote}
"The system replaces authored prep with guided storytelling. Every card draw is improvisation with guardrails."
\end{quote}

\subsubsection{Mechanical Simplicity vs. Narrative Sophistication}
\textbf{Misconception}: "Simple mechanics = shallow stories"
\textbf{Reality}: Sophistication emerges from constraint interaction

\textbf{Reviewer Guidance}:
\begin{quote}
"Like jazz with a vast repertoire, the same simple rules that handle Session 1 magic effortlessly scale to manage Session 50 world-changing consequences."
\end{quote}

\subsection{Evaluation Methodology}

\subsubsection{Progressive Complexity Assessment}
\begin{enumerate}
    \item \textbf{Tier I Play} (30-40 XP): Evaluate session-to-session narrative flow with minimal subsystems
    \item \textbf{Tier III Play} (90-150 XP): Assess how character growth naturally expands storytelling scope  
    \item \textbf{Tier V Play} (220+ XP): Examine how simple rules handle complex, multi-threaded narratives
\end{enumerate}

\subsubsection{Mastery Curve Analysis}
\textbf{Key Metric}: Mastery effort should decrease as narrative complexity increases

\textbf{Success Indicators}:
\begin{itemize}
    \item Session 1: Clear, engaging storytelling with minimal reference
    \item Session 10: Comfortable integration of multiple subsystems
    \item Session 30: Mechanics become invisible; focus entirely on narrative
\end{itemize}

\subsection{What to Look For}

\subsubsection{Narrative-First Design}
\begin{itemize}
    \item Every mechanical element should serve story consequences
    \item Player agency should manifest through narrative choices, not optimization
    \item GM authority should prioritize "what happens" over "what are the rules"
\end{itemize}

\subsubsection{Constraint-Based Creativity}
\begin{itemize}
    \item Random elements should create coherent, not chaotic, storytelling
    \item Regional generators should maintain thematic consistency while enabling variety
    \item Mechanical constraints should inspire creativity, not limit options
\end{itemize}

\subsubsection{Scalable Sophistication}
\begin{itemize}
    \item Same core mechanics should handle intimate scenes and epic narratives
    \item Character progression should expand narrative possibilities, not mechanical options
    \item World complexity should emerge from player actions, not pre-authored content
\end{itemize}

\subsection{Red Flags vs. Design Features}

\subsubsection{Actual Red Flags}
\begin{itemize}
    \item Mechanics that require constant reference during play
    \item Subsystems that don't integrate naturally with core philosophy
    \item Progression that adds mechanical burden rather than narrative freedom
\end{itemize}

\subsubsection{Misinterpreted Design Features}
\begin{itemize}
    \item \textbf{High content volume} = Creative resource library, not mastery burden
    \item \textbf{Multiple magic paths} = Narrative vocabulary expansion, not mechanical complexity
    \item \textbf{Regional generators} = Inspiration engines, not reference requirements
\end{itemize}

\subsection{Recommended Review Approach}

\subsubsection{Session 1-3 Evaluation}
Focus on:
\begin{itemize}
    \item How quickly engaging stories emerge
    \item Clarity of core mechanical principles
    \item Integration of generated content into coherent narratives
\end{itemize}

\subsubsection{Session 10+ Evaluation}  
Assess:
\begin{itemize}
    \item How naturally complexity emerges from player choices
    \item Integration of multiple subsystems without mechanical overhead
    \item Evolution from "using rules" to "telling stories"
\end{itemize}

\subsubsection{Long-term Assessment}
Consider:
\begin{itemize}
    \item Whether the system grows with narrative ambition
    \item If mastery effort genuinely decreases over time
    \item How world complexity emerges organically from play
\end{itemize}

\subsection{The Core Truth}

Fate's Edge succeeds not by being simple, but by being \textbf{designed for simplicity to emerge naturally from complex storytelling}. Evaluate it not as a comprehensive system to master, but as a storytelling partnership that grows in sophistication at exactly the same rate your table develops narrative ambition.

\begin{quote}
\textbf{This isn't a game you learn - it's a storytelling evolution you grow into.}
\end{quote}
