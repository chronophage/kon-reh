\documentclass[11pt]{article}

% --- Core packages & order ---
% math text & symbols
\usepackage{amsmath,amssymb}
% define \texorpdfstring (load late is safest)
\usepackage[hidelinks]{hyperref}
\usepackage[T1]{fontenc}
\usepackage{lmodern}
\usepackage[margin=1in]{geometry}
\usepackage{microtype}
\usepackage{xcolor}
\usepackage{graphicx}          % for \resizebox
\graphicspath{{figures/}}
\usepackage{booktabs}
\usepackage{tabularx,makecell,array}
\usepackage{multicol}
\usepackage{enumitem}
\usepackage{parskip}
\usepackage{ellipsis} % improves spacing around \dots/\ldots automatically
\usepackage{tabularx}
\usepackage{tikz}              % load TikZ first...
\usetikzlibrary{arrows.meta,calc,positioning}  % ...then libraries

\usepackage{subcaption}
\usepackage{needspace}
\usepackage{float}
\usepackage{placeins}

% --- Color palette ---
\definecolor{royal}{RGB}{12,64,145}
\definecolor{sanctum}{RGB}{0,120,80}
\definecolor{ink}{RGB}{30,30,30}
\definecolor{ccol}{RGB}{55,110,180}
\definecolor{scol}{RGB}{120,60,160}
\definecolor{rocol}{RGB}{220,130,40}
\definecolor{rccol}{RGB}{190,40,50}
\definecolor{capcol}{RGB}{60,140,80}

% --- tcolorbox (load once, after colors) ---
\usepackage[most]{tcolorbox}

% --- Small UI helpers (now safe because TikZ is loaded) ---
\newcommand{\CrownIcon}{%
  \tikz[baseline=-0.5ex,scale=0.14]{
    \fill[orange!70!yellow] (0,0) -- (0.65,0.9) -- (1.3,0) -- (1.95,0.9) -- (2.6,0) -- cycle;
    \fill[orange!80!brown]  (0,-0.36) rectangle (2.6,0);
  }%
}

% --- Optional badge & variant box (uses tcolorbox) ---
% Pill that pops on a dark title bar
\newtcbox{\optbadgeOnDark}{on line, arc=4pt, boxrule=0pt,
  colback=orange!70!yellow, colframe=white,
  left=4pt, right=4pt, top=1pt, bottom=1pt, boxsep=1.5pt,
  tcbox raise base, fontupper=\scriptsize\bfseries\color{black}}
  

\newtcolorbox{rulevariant}[1][]{%
  enhanced, breakable,
  colback=royal!3, colframe=royal!70!black, boxrule=0.8pt,
  rounded corners, left=7pt, right=7pt, top=7pt, bottom=7pt,
  coltitle=white,
  attach boxed title to top left={yshift=-2mm, xshift=3mm},
  boxed title style={colback=royal!85!black, colframe=royal!85!black},
  fonttitle=\bfseries,
  title={\CrownIcon\;Crown Buyback\;\optbadgeOnDark{OPTIONAL}},
  #1}

% --- Table/layout tweaks ---
\setlength{\columnsep}{0.75em}
\setcounter{tocdepth}{2}
\setcounter{secnumdepth}{2}
\newcolumntype{Y}{>{\raggedright\arraybackslash}X}
\setlength{\tabcolsep}{6pt}
\renewcommand{\arraystretch}{1.15}
\setlength{\parindent}{0pt}

% --- Status badges (TEXT mode; do not put them in \( ... \)) ---
\newcommand{\CC}[1]{\textcolor{blue!60!black}{\scriptsize\ttfamily[CF:#1]}}
\newcommand{\SC}[1]{\textcolor{red!60!black}{\scriptsize\ttfamily[S:#1]}}
\newcommand{\RoC}{\textcolor{teal!60!black}{\scriptsize\ttfamily[Rooted]}}
\newcommand{\RC}{\textcolor{purple!70!black}{\scriptsize\ttfamily[RC]}}
\newcommand{\CapC}[1]{\textcolor{green!40!black}{\scriptsize\ttfamily[G:#1]}}

% --- Thought helper ---
\newcommand{\think}[1]{\emph{\footnotesize #1}}

% =========================
% Board config + TikZ styles
% =========================
\newcommand{\BoardN}{8}               % board size (NxN)
\newcommand{\SanctumA}{1/4}           % left side-apex Sanctum
\newcommand{\SanctumB}{8/5}           % right side-apex Sanctum

\tikzset{
  sq/.style={draw, line width=0.3pt},
  cross/.style={fill=blue!22},      % center Cross (2x2)
  apexA/.style={fill=green!18},     % top apex square
  apexB/.style={fill=green!26},     % bottom apex square
  sanctumA/.style={fill=red!28},    % left Sanctum
  sanctumB/.style={fill=red!34},    % right Sanctum
  zoc/.style={fill=black!18},
  moveArrow/.style={-Latex, line width=0.8pt},
  piece/.style={circle, draw, fill=white, line width=0.8pt, minimum size=7.5pt, inner sep=0pt}
}

% Grid + shading (top-left origin: physical y = BoardN - y)
\newcommand{\DrawGrid}{%
  \foreach \x in {1,...,\BoardN}{%
    \foreach \y in {1,...,\BoardN}{%
      \pgfmathtruncatemacro{\yphys}{\BoardN-\y}%
      \pgfmathtruncatemacro{\shade}{mod(\x+\y,2)==0 ? 3 : 0}%
      \fill[black!\shade] ({\x-1},{\yphys}) rectangle ++(1,1);
      \draw[sq] ({\x-1},{\yphys}) rectangle ++(1,1);
    }%
  }%
}
\newcommand{\ShadeSquares}[2]{% #1=style, #2="x/y,x/y,..."
  \foreach \x/\y in {#2}{%
    \pgfmathtruncatemacro{\yphys}{\BoardN-\y}%
    \path[#1] ({\x-1},{\yphys}) rectangle ++(1,1);
  }%
}
\newcommand{\ShadeCross}[1]{%
  \pgfmathtruncatemacro{\L}{\BoardN/2}%
  \pgfmathtruncatemacro{\H}{\L+1}%
  \foreach \x/\y in {\L/\L,\L/\H,\H/\L,\H/\H}{%
    \pgfmathtruncatemacro{\yphys}{\BoardN-\y}%
    \path[#1] ({\x-1},{\yphys}) rectangle ++(1,1);
  }%
}
\newcommand{\RedrawGridLines}{%
  \foreach \x in {1,...,\BoardN}{%
    \foreach \y in {1,...,\BoardN}{%
      \pgfmathtruncatemacro{\yy}{\BoardN-\y}%
      \draw[sq] ({\x-1},{\yy}) rectangle ++(1,1);
    }%
  }%
}

% Cardinal labels used in your minis (adjust as you prefer)
\newcommand{\LabelN}{OR}
\newcommand{\LabelE}{OL}
\newcommand{\LabelS}{HR}
\newcommand{\LabelW}{HL}

% Piece placers (safe now that TikZ is loaded)
\newcommand{\PlaceA}[3]{\pgfmathtruncatemacro{\yphys}{\BoardN-#3}\node[piece,fill=white,text=black] at ({#2-0.5},{\yphys+0.5}) {\scriptsize\bfseries #1};}
\newcommand{\PlaceB}[3]{\pgfmathtruncatemacro{\yphys}{\BoardN-#3}\node[piece,fill=black,text=white] at ({#2-0.5},{\yphys+0.5}) {\scriptsize\bfseries #1};}

% ==== Directional move notation (robust) ====
% Usage: \On[OL]{2}, \On[OR]{3}, \Hm[HL]{1}, \Hm[HR]{2}
\makeatletter
\newcommand{\KR@OnPretty}[1]{%
  \def\tmp{#1}\def\OL{OL}\def\OR{OR}%
  \if\relax\detokenize{#1}\relax\else
    \ifx\tmp\OL L\else\ifx\tmp\OR R\else #1\fi\fi
  \fi
}
\newcommand{\KR@HmPretty}[1]{%
  \def\tmp{#1}\def\HL{HL}\def\HR{HR}%
  \if\relax\detokenize{#1}\relax\else
    \ifx\tmp\HL L\else\ifx\tmp\HR R\else #1\fi\fi
  \fi
}
\newcommand{\KR@MoveCore}[3]{%
  \mbox{\textsc{#1}\if\relax\detokenize{#2}\relax\else$_{\mathrm{#2}}$\fi\,\textbf{#3}}%
}
% Force (re)define even if a 1-arg legacy \On/\Hm exists
\DeclareRobustCommand{\On}[2][]{\KR@MoveCore{on}{\KR@OnPretty{#1}}{#2}}
\DeclareRobustCommand{\Hm}[2][]{\KR@MoveCore{hm}{\KR@HmPretty{#1}}{#2}}
\makeatother

% === Faction blurb block (name + one-line desc + optional tagline) ===
% Usage:
%   \factionblurb{Ykrul (Kon'reh)}{Control-first pragmatists...}{“Count exits, not victims.” — Kargath}
%   \factionblurb{Ecktorian}{Engineers of symmetry...}{}   % <- no tagline line
\newcommand{\factionblurb}[3]{%
  \par\medskip
  \Needspace{3\baselineskip}% keep label+desc+tagline together when possible
  \noindent\textbf{#1. }#2\par
  \smallskip
  \if\relax\detokenize{#3}\relax\else\emph{#3}\par\fi
}

% ==== Simple Coach's Notes (no tables) ====
\newtcolorbox{coachbox}[1]{%
  enhanced, breakable,
  colback=royal!3, colframe=royal!70!black, boxrule=0.8pt,
  rounded corners, left=7pt, right=7pt, top=7pt, bottom=7pt,
  coltitle=white,
  attach boxed title to top left={yshift=-2mm, xshift=3mm},
  boxed title style={colback=royal!85!black, colframe=royal!85!black},
  fonttitle=\bfseries,
  title={#1}, before skip=6pt, after skip=6pt, width=\linewidth,
}

% Coach’s Notes environment: uses description list for Cue → Note
\newenvironment{coachnotes}[1]{%
  \begin{coachbox}{#1}%
    \footnotesize
    \begin{description}[leftmargin=2.8cm,labelsep=0.6em,font=\scshape,
                        itemsep=0.25em,parsep=0pt,topsep=0.2em]
}{%
    \end{description}%
  \end{coachbox}%
}

% Convenience macro for rows
\newcommand{\CNRow}[2]{\item[#1] #2}

% --- helpers (safe: only define if missing) ---
\providecommand{\playdesc}[1]{\par\smallskip\noindent\small\textbf{Play.} #1\par}
\newcommand{\flavline}[2]{\noindent\textbf{#1.} \textit{#2}\par}

% --- Engine-log friendly wrappers (reuse \On and \Hm) ---
% LANE: pass L or R (OL/OR/HL/HR still OK via your pretty-mapper)
\makeatletter

% convenience tags
\newcommand{\CFIn}[1]{\CC{in #1/3}}
\newcommand{\CFOut}{\CC{out}}

% allow \RC both bare and with a count: \RC  or  \RC[3/5]
\renewcommand{\RC}[1][]{%
  \textcolor{purple!70!black}{\scriptsize\ttfamily[RC%
  \if\relax\detokenize{#1}\relax\else~#1\fi]}}

% verb dispatchers -> your existing \On / \Hm
\expandafter\def\csname mvverb@on\endcsname#1#2{\On[#1]{#2}}
\expandafter\def\csname mvverb@hm\endcsname#1#2{\Hm[#1]{#2}}

% specials (no lane/step)
\expandafter\def\csname mvverb@H\endcsname#1#2{{\SC{H}}}      % Hop-capture tag
\expandafter\def\csname mvverb@D\endcsname#1#2{{\SC{D}}}      % Displacement tag
\expandafter\def\csname mvverb@seed\endcsname#1#2{{\textsc{seed}}}

% directional move: \mv[Side]{Piece}{verb}{Lane}{Steps}{tail}
\DeclareRobustCommand{\mv}[6][]{%
  \if\relax\detokenize{#1}\relax\else\textbf{#1:}\ \fi
  \textbf{#2}\ %
  \csname mvverb@#3\endcsname{#4}{#5}%
  \if\relax\detokenize{#6}\relax\else\ #6\fi
}

% special-only (no lane/steps): \mvs[Side]{Piece}{Special}{tail}
\DeclareRobustCommand{\mvs}[4][]{%
  \if\relax\detokenize{#1}\relax\else\textbf{#1:}\ \fi
  \textbf{#2}\ %
  \csname mvverb@#3\endcsname{}{ }%
  \if\relax\detokenize{#4}\relax\else\ #4\fi
}
\makeatother
\newcolumntype{L}{>{\raggedright\arraybackslash\hspace{0pt}}X}


% Optional scale arg with default 1.2
% --- KON'REH Crest (TikZ) ----------------------------------------
% Usage: \KonrehCrest[1.0]   % optional scale (default 1.0)
% Place on title page or anywhere as a vector icon.


\begin{document}\color{ink}

%==============================
% Title Page
%==============================
\begin{titlepage}
  \thispagestyle{empty}
  \begingroup
  \centering
  \vspace*{1.5cm}

  {\color{royal}\fontsize{36}{40}\selectfont\bfseries KON'REH}\par
  \vspace{6pt}
  {\Large\bfseries The Ninth Rim \par
  {\large Core Rules \& Lore}\par

  \vspace{14pt}
  \rule{0.62\linewidth}{0.6pt}\par
  \vspace{6pt}
  {\normalsize \textit{Another Expansion to a Game of Apex, Sanctum, and Reforge}}\par
  \vspace{6pt}
  \rule{0.62\linewidth}{0.6pt}\par

  \vspace{18pt}
  {\Large \textit{by} Nicholas A.\ Gasper}\par
  {\normalsize Setting \& Lore by Nicholas A.\ Gasper}\par

  % Optional crest/emblem (uncomment and provide asset)
  % \vspace{12pt}
% Crest (PNG). Compiles even if the file is missing.
\vspace{10pt}
% Crest (PNG) with fallbacks in fig/
\IfFileExists{ninth-crest.png}{%
  \includegraphics[
    width=0.28\linewidth,
    keepaspectratio,
    trim=8 8 8 8,clip
  ]{fig/concordance-crest.png}\par\vspace{8pt}%
}

  \vfill

  % Optional epigraph — comment out if you prefer a blank footer
  {\small\itshape
    \CrownIcon\quad “Play.” \hfill — Taraksa Ghez
  }\par

  \vspace{8pt}
  {\footnotesize
    Concordance Edition \textbullet\ \today
    % \quad|\quad Draft v0.9
  }\par

  \vspace*{1cm}
  \endgroup
\end{titlepage}

\pagenumbering{roman}
\tableofcontents
\clearpage

%==============================
% Copyright / Legal Page
%==============================
\clearpage
\thispagestyle{empty}
\vspace*{2cm}

{\small
\noindent \textbf{KON'REH} and associated setting terms including but not limited to:
\textit{Canray}, \textit{K’thra}, \textit{Kanry}, \textit{Twin Apex Seed}, \textit{Reforge},
the names of cultures (e.g., Ykrul, Ecktorian, Vhasian, Viterran, Aeler, Vilikari, Thepyrgosi (Thepyric), Ubral, Silkstrand),
proper nouns, places, characters, flavor quotes, worldbuilding lore, diagrams, iconography,
and the specific textual expression of rules, examples, and notation in this book are
© \the\year\ \textit{Nicholas A. Gasper}. All rights reserved.\\[8pt]

\noindent \textbf{Mechanics Disclaimer.}
The underlying game mechanics, procedures of play, and functional systems described herein are
not claimed as proprietary subject matter. No copyright is asserted in the \emph{ideas} of movement rates,
zones of control, countdowns, or other rules mechanics \emph{as mechanics}; copyright subsists in the
\emph{expression} of those ideas in this book (text, arrangement, examples, graphics, naming, and lore).\\[8pt]

\noindent \textbf{Trademarks.}
KON'REH and other marks herein may be trademarks or registered trademarks of their respective owners.
Use of the marks does not imply endorsement.\\[8pt]

\noindent \textbf{Fan Content Policy (Non-Commercial).}
You may reference these rules in reviews, tutorials, and fan aids, and you may create non-commercial
scenarios and player aids that include brief excerpts, provided you (i) credit
\textit{“Kon’reh © \the\year\ Nicholas A. Gasper”}, (ii) do not reproduce large portions of this book verbatim,
and (iii) do not imply official status. For commercial use, please contact the publisher.\\[8pt]

\noindent \textbf{All Rights Reserved.}
Except as permitted above or by applicable law, no portion of this publication may be reproduced,
stored in a retrieval system, or transmitted in any form or by any means without prior written permission
of the publisher.\\[8pt]

\noindent \textbf{Credits.}
Design \& Development: Nicholas A. Gasper \\
Editing: \textit{PLACEHOLDER} \\
Playtesting: \textit{PLACEHOLDER} \\[8pt]

\noindent \textbf{Publisher.}
PLACEHOLDER \\
\textit{ISBN:} (TBD)
}


\section*{Preface — A Note from Markus Fenwood}
\label{sec:preface-markus}
\phantomsection
\addcontentsline{toc}{section}{Preface — A Note from Markus Fenwood}

My name is \textbf{Markus Fenwood}, and I am not the scholar my father was. I came to \textit{Thepyrgos} to learn good ink and straight numbers—ledger hands, not liturgy. Father pressed a travel copy of the \emph{Concordance} into my hands as we parted and said only, ``Our fourth legacy is yours now.'' He smiled at the joke, and I tried to smile back. It was one of the last complete sentences he managed.

If this letter feels halting, it is because I am halting. I am keeping faith with his pencil--scrawl more than with my own certainty. \textbf{Aqyl}, son of Aqyl, has been patient beyond reason, correcting my cadence and teaching me that a clear gloss is an ethical act. ``You are not required to be clever,'' Aqyl told me in my first week in the city. ``You are required to be honest.'' I have tried.

The work that follows begins as a student's humble collation: toll stamps, city maps, fragments copied on rainy nights in the reading stalls.

But the edges begin to speak to the middle\ldots{}

I found a thin slip \textit{buried beneath a pastedown} in a cracked ledger—a binder's repair on a \textit{House of Wells} account book. Someone had tucked a narrow vellum between cover and board: a column of sums in the hand of \textbf{the Cartwright}, neat as ever, but with oddities in the margins—little circles every fifth line, a note beside a rounded total: ``ford held,'' and three triangles drawn where there should have been tallies. It was nothing you could take to a magistrate. It was enough to take to Aqyl.

``Strange,'' Aqyl said, tapping once at the triangles. ``Strange does not mean false.'' We began to read more attentively. The Cartwright's later writings, thought to be merely practical advisories on roadwork and tariffs, \textit{flicker} when set beside the city's maps and the way we count aloud over a game. The numbers behave like prices that want to be \textit{prayers}. I do not know what to do with that sentence except set it down and let you test it at the table.

Father's last season at home was a quiet one. He spoke less, but his pencil moved more. He marked my copy of the \emph{Concordance} with small private arrows and once wrote, faintly, ``Ask Aqyl about the ford.'' When I asked what he meant, he pretended not to hear and instead asked after my rent. I think he wanted me to be \textit{less brave and more careful} than he was. I hope this book honors that wish.

\medskip
\noindent\textit{For my father, His Grace, Braedon Fenwood III, who taught me to count aloud and to write what I can prove; for Aqyl, who showed me the difference; and for the players, who will decide which of these rumors hold.}

\begin{flushright}
\textit{—Thepyrgos, late flood season}
\end{flushright}


Here’s a clean, publication-ready Introduction you can drop in after the preface. It keeps the tone formal, explains scope and use, and avoids spoilers.

\section*{Introduction}
\label{sec:introduction}
\phantomsection
\addcontentsline{toc}{section}{Introduction}

\subsection*{Purpose and Scope}
\textit{Kon'reh: The Ninth Rim} is a dossier and scenario volume intended to be read as narrative and used at the table without altering tournament play. It presents historical fragments, ledgers, and field notes in facsimile, followed by seven optional, modular play overlays (\emph{Rites}) that can be toggled on or off per session.

\subsection*{How to Use This Book}
Read \emph{Part I} for context and atmosphere; use \emph{Part II} to run one-shots or link all seven scenarios into a short campaign. Each Rite is self-contained, clearly labeled, and defaults to \textsc{off}. Turning a Rite \textsc{on} changes the \emph{feel} of play (information, pacing, geography) without changing the core rules of the game.

\subsection*{Structure of the Volume}
\begin{itemize}
  \item \textbf{Part I — The Dossier (Lore in Facsimile):} curated documents, marginalia, and maps that frame the investigation.
  \item \textbf{Part II — Rites \& Scenarios:} seven independent modules (\emph{Salt Stitch}, \emph{Witness at the Ford}, \emph{Veil of Names}, \emph{Candle Count}, \emph{Copper \& Salt}, \emph{Ash-Fenn Rite}, \emph{The Ninth}) with setup, procedure, examples, and a short “Why It’s Safe” box.
  \item \textbf{Part III — Puzzles \& Scholar’s Path:} hint ladders and a sealed appendix for readers who wish to decode the embedded ciphers.
  \item \textbf{Part IV — Kon’metry (Theory Sidebars):} brief, non-binding notes on measurements, survey drift, cadence-to-digit maps, and related curiosities.
  \item \textbf{Appendices:} compatibility matrix, printable props, credits, and index.
\end{itemize}

\subsection*{Rites: Modularity and Safety}
A \emph{Rite} is an \emph{overlay} that can be applied to any casual session. It never rewrites movement, capture, costs, or timers of the core rules. Each module includes:
\begin{enumerate}
  \item a concise rules box, 
  \item a “Why It’s Safe” reassurance, 
  \item an example of play, and 
  \item (when relevant) a campaign \emph{clue beat}.
\end{enumerate}
For linked play, the \emph{Compatibility Matrix} in the appendices indicates which Rites pair smoothly.

\subsection*{On Puzzles and the Sealed Appendix}
Some pages contain enciphered details based on counting cadence and document layout. These are optional. Groups may ignore puzzles entirely, follow the \emph{Scholar’s Path} hinting ladders, or open the sealed appendix for full solutions. No puzzle is required to use any Rite; solving them enriches the narrative and, in the final scenario, grants a one-turn \emph{Rite-Break} (suspending one active Rite for the solver’s current turn).

\subsection*{What You Need to Play}
The core game components you already own. For this volume’s atmosphere and optional tracking you may substitute common items for:
\begin{itemize}
  \item a perimeter ribbon or markers (for \emph{Salt Stitch}),
  \item five beads or counters (for \emph{Candle Count}),
  \item a few copper-colored tokens (for \emph{Copper \& Salt}),
  \item printed handouts (maps, slips, seals) from the templates in the appendices.
\end{itemize}
All props are provided as printables; a deluxe set may include tactile versions.

\subsection*{Editorial Notes and Conventions}
Voice labels (e.g., \emph{Markus}, \emph{Aqyl}, \emph{Clerk}) appear in margins to indicate provenance. Dates follow city-record style. Facsimiles are lightly normalized for legibility; conjectural restorations are bracketed. Diagram callouts and iconography are standardized for quick reference.

\subsection*{Continuity}
This volume preserves the game’s tournament core. Scenario language is intentionally conservative and has been reviewed for rules safety. Historical attributions reflect best-available sources; where disputes exist, they are marked as such rather than harmonized.

\medskip
\noindent\textit{Read what you like, use what you will, and leave the rest in the archives.}

\addcontentsline{toc}{section}{Part I: }

\subsection*{Fragment L--1 (648 AR) --- On the Lineage of Names and the Measure of Days}
\label{frag:l1}
\phantomsection
\addcontentsline{toc}{subsection}{Fragment L-1 (648 AR) --- On the Lineage of Names and the Measure of Days}

\noindent\textit{Copied from a student booklet in a small slanted hand; iron-gall ink, edges smoke-kissed. A later owner has ruled a faint line for marginal glosses.}

\medskip
\noindent\textbf{Lexicus of Thepyrgos, Junior Reader}\\
\textit{Year 648 after the River, third month of Planting}

Herewith I set down, that I may not forget, certain observations gathered in the reading-stalls and in the counting-rooms. This morning I was shown a parcel of leaves attributed to one \textit{Aqyl of Thepyrgos}, surnamed in the index as \textit{the elder}. They concern cadence, clarity, and the ethics of a gloss. Their sentences are of such right plainness that I blushed to see my own drafts beside them.

Being of an inquisitive temper, and not wishing to labor under a confusion of persons, I inquired of my mentor, \textit{Master Aqyl}, whether he were named for that same man whose notes I had been admiring.

He shrugged in that economical way of his, as if saving words for a worthier hour, and said: ``I am named for my father, and he for his, and he for his; all of us scholars, or almost, and diligent at least when we were not frightened. This Aqyl, the first, is my ancestor---so they say in our house---though it is better to prove a line by its books than by its boasts.''

I laughed (quietly) at the manner of it and resolved then to keep my margins honest. I note also that the elder \textit{Aqyl} counsels the counting aloud of ordinary motions to keep time and temper. The practice steadies the hand when the rain takes the lamps and the shelves smell of wet leather. I will adopt it, if only to discover whether steadier breath makes steadier figures.

\medskip
\noindent\textit{Commonplaces for the Stalls (set down for my own correction):}
\begin{enumerate}\setlength\itemsep{0.25em}
  \item Count aloud when the mind runs ahead of the ink.
  \item Mark conjecture as conjecture; do not pass it as record.
  \item Where two sums disagree, copy both; cure the disagreement, then strike the error with a single line.
  \item Keep the hand even at the edges; the eye trusts a margin that does not wave.
  \item If a sentence can be made plainer without loss, do so.
\end{enumerate}

It is my intention, if Providence and the bursar permit, to search the older cupboards for further traces of the elder \textit{Aqyl}'s order: not the fame of his name, but the measure of his days.

\medskip
\begin{quote}
\textit{Present marginal hand (910 AR), M.\ Fenwood:} Asked \textit{Aqyl, son of Aqyl}, whether the line holds. He shrugged, smiled the same small smile, and said only: ``I am named for my father, and he for his.'' No ledger longer than that. It will do.
\end{quote}

\subsection*{Fragment L--2 (648 AR / 910 AR) --- The Salt-Pressed Leaves}
\label{frag:l2}
\phantomsection
\addcontentsline{toc}{subsection}{Fragment L-2 (648 AR / 910 AR) --- The Salt-Pressed Leaves}

\noindent\textit{Provenance note: two hands separated by centuries. The later hand (\textit{910 AR}) describes a walk through the University; the earlier (\textit{648 AR}) is a student entry by Lexicus of Thepyrgos.}

\medskip
\noindent\textbf{Present hand (910 AR), M.\ Fenwood}

Aqyl, son of Aqyl, led me by the long way, through the serpentine stacks where the floors tilt and the lamps make a river of light. We passed a door stenciled in plain administration: \textsc{condemned --- closed by order of the Wardens}. Aqyl produced a key that did not look like a key, and we went in.

The air had the taste of chalk. Cobwebs draped the ceiling like tired banners, and the tables wore a coat of dust that remembered every elbow. ``A private study,'' Aqyl said, sweeping a hand. ``\emph{Canr\'e} writings, mostly. Lexicus kept copies here. It is a forbidden place, though no one will name it so. The University holds many such places.''

I asked if we should be there. He shrugged. ``You asked for the edges. Edges are seldom in the catalog.'' He set a small lantern on a desk and added, quieter: ``Read heresies if you must, but do not let them lead you.''

A spine caught my eye---a ledger-brown volume with a torn label. The marginal hand on the fore-edge looked like Lexicus's. I cracked it open to a page where salt had left a pale ring on the paper, as if a damp band had once rested there.

\medskip
\noindent\textbf{Earlier hand (648 AR), Lexicus of Thepyrgos, Junior Reader}

Master Aqyl this day entrusted to me the study he keeps behind the north stalls, being tired (so he says) of carrying the key. He maintains that the first of his name bequeathed it to his line, yet would have it now employed by any who will keep a neat margin. I am resolved to be that man.

On a middle shelf I found a tract whose outer leaves show a curious stain, as though pressed with damp salt. The title is rubbed, but within there is set down a custom of the Cartwright before any game: he circles the board's perimeter with a light band of salt and speaks nothing until his first piece is moved. The writer calls this the \textit{Salt Stitch}. The note cites an earlier fragment, styled the \textit{Salt-Pressed Leaves} (whatever that means), and treats the habit as a warding: \emph{not a wall, but a price}.

I cannot yet say whether this is superstition or discipline. The Cartwright is otherwise a practical man. Yet I confess the figure of it pleases the mind. To bind the outermost ring and thus make the middle honest---is this not what we do in our books when we rule a margin?

\medskip
\noindent\textit{Present marginal hand (910 AR), M.\ Fenwood:} Same tract, same page. The ring-mark is here to this day. Aqyl watched me read and said nothing until I closed it. Then: ``Name nothing yet. Learn its weight first.''


\subsection*{Fragment L--3 (648 AR / 910 AR) --- Ash--Fenn, Tolls, and Wagered Dreams}
\label{frag:l3}
\phantomsection
\addcontentsline{toc}{subsection}{Fragment L-3 (648 AR / 910 AR) --- Ash-Fenn, Tolls, and Wagered Dreams}

\noindent\textit{Provenance: a leaf from a toll ledger (rubbed and smoke-tinged) copied by Lexicus (648 AR), with a later journal note by M.\ Fenwood (910 AR).}

\medskip
\noindent\textbf{Earlier hand (648 AR), Lexicus of Thepyrgos, Junior Reader}

In the north case I found a ledger of border dues, the hand plain as a bricklayer’s rule. One entry pricked my eye:

\begin{quote}
\small
``Ash--Fenn caravan: from the east. Twenty-three beasts, seven wagons. Paid in full at Wadi Gate; stamped at the ford. Next gate: ---'' (blank) ``Reason for pass-through: ---'' (blank).
\end{quote}

It is strange that a caravan so noted would vanish between gates, stranger that the clerk left two fields unfilled. I have marked three other pages where the same name appears, always ``from the east,'' always tidy in its sums, and then silence as though the road itself declined to witness.

I confess I do not yet comprehend Master Aqyl’s interest in such mundane books. He says the ordinary page is a faithful mirror if one looks long enough; I am young and wish for brighter glass.

Turning the leaf, a loose scrap fell from the binding---a toll-keeper’s own hand, crowded and eager, likely written after hours:

\begin{quote}
\small
``They were cheerful, the Ash--Fenn lot, and asked if I would sit a friendly game of Kanray while the beasts drank. We wagered dreams, which I thought a country jest. I am not a superstitious man. I lost, and bid them on their way. Curious thing is, I dreamt poorly after, a month’s turn of it, and woke as if I had borrowed someone else’s night.’’
\end{quote}

If this be a town humor, it is an odd one; yet the phrase ``wagering dreams'' appears again in a margin later in the book, in another hand. Perhaps it is a manner of speaking for small stakes and no coin. I will ask Master Aqyl whether the phrase has a history, or whether the clerk merely wished to write himself interesting.

The ledger’s ring-stamps are neat; the gaps are neater. I begin to suspect that the absence of a mark can carry as much account as the presence.

\medskip
\noindent\textbf{Present hand (910 AR), M.\ Fenwood}

I have gone back to the little study three days running. It is an alluring room, and I tell myself it is the quiet I am after. On the day I kept away (work, and a sore head), I slept oddly: too many doors, not enough rooms. It is nothing, I am sure, but I write it down because the toll-keeper wrote his down, and because keeping a ledger of such things has begun to feel like part of the work.

Aqyl has not asked where I am reading; I have not said. He warned me to read heresies without letting them read me. I do not yet know how to tell the difference, except to stop when the ink begins to feel warm in the hand.

The Ash--Fenn name recurs. Always ``from the east.'' Always accounted, then unaccounted for. If there is a trick in the books, it is a careful trick. If there is none, then it is only a clerk’s habit and I am lending it more gravity than it earns. Either way, tomorrow I will copy the three blanks and see whether their emptiness makes a pattern.

\subsection*{Fragment L--4 (648 AR / 910 AR) --- The Burning at Ash--Fenn}
\label{frag:l4}
\phantomsection
\addcontentsline{toc}{subsection}{Fragment L-4 (648 AR / 910 AR) --- The Burning at Ash-Fenn}

\noindent\textit{Provenance: a trembling report in an unknown hand, copied by Lexicus (648 AR); later marginal notes by M.\ Fenwood (910 AR). The original leaf is smoke-soft and smells faintly of brine.}

\medskip
\noindent\textbf{Earlier hand (648 AR), Lexicus of Thepyrgos, Junior Reader}

In a pasteboard folder marked only with a tally, I found a narrow leaf, the ink running as if laid under breath. It purports to be an eye-witness account of the \textit{Burning at Ash--Fenn}, written by a scholar whose name is withheld. I copy the opening as it stands, preserving its ornaments and its hesitations:

\begin{quote}\small
``We came upon the caravan after the bell. The charge was heresy against the Light. I will not set down the tale of it, for the deeds done and the words spoken by the dying are a weight I will not lend to the world. I write only that ash fell like a fine rain, and that the road kept no prints though many passed. 

As I turned to go, I heard voices speaking in a tongue I do not know, though I know many. They were not commands, only \emph{tellings}, as if a steady hand turned a ledger and asked me to read aloud. They said, 'Bring it to your mentor, Aqyl.' They said, 'He will weigh it.' I took fear then and fled for Thepyrgos with this account, meaning to place it in his hands and wash mine.''
\end{quote}

The next lines were added, I think, in the same trembling hand the day after:

\begin{quote}\small
``I found \textit{Aqyl} much aged and asked leave to lay the leaf before him. He called for his son---also \textit{Aqyl}---to sit and witness. The elder's eye watered but did not wander. He said, 'Read what you can, hide what you must.' The younger set a lamp between us and did not speak.''
\end{quote}

I do not know what to believe of this, only that I felt my own breath shorten as I copied it. The hall outside the little study was very quiet, yet I fancied I heard, beneath the quiet, a soft speech like cloth moved in the next room. I stopped my hand twice and began again.

\medskip
\noindent\textit{Note to self:} If this be \emph{heresy}, it cloaks itself in accounts and receipts. If it be only panic on a wet night, then why does the ink fall so heavy where the writer says \emph{wash mine}?

\medskip
\noindent\textbf{Present hand (910 AR), M.\ Fenwood}

I am not proud to write that this piece spooked me. Two nights now with poor sleep. Nothing elaborate: only a sense of whispering at the edge of hearing, and (last night) a slow drum I could not place, as if someone far off kept time for a procession that never arrived.

I went back to the study this morning and read the lines about the elder and the younger sitting together by the lamp. There is a steadiness in that picture that helps. Aqyl says a ledger is kept one entry at a time. I will keep mine that way and not be in a hurry to make a story of it.

\medskip
\noindent\textit{Present marginal hand, small:} \emph{If the road kept no prints, look to the fords. If the ash fell like rain, look to the wind.}

\subsection*{Fragment L--5 (648 AR / 910 AR) --- On Ionius and the Witness at the Ford}
\label{frag:l5}
\phantomsection
\addcontentsline{toc}{subsection}{Fragment L-5 (648 AR / 910 AR) --- On Ionius and the Witness at the Ford}

\noindent\textit{Provenance: a translated note by Lexicus (648 AR) from an older Thepyrgosi account; a later journal entry by M.\ Fenwood (910 AR).}

\medskip
\noindent\textbf{Earlier hand (648 AR), Lexicus of Thepyrgos, Junior Reader}

I have this day translated a fragment from our city’s older tongue, being the recollection of a youth who attached himself to the \textit{Cartwright} on account of his teachings in ordered play (styled in the margin as \textit{Canr\'e}, elsewhere \textit{Kanray}). At the first the lessons are entirely proper: on patience, on the advantage of plain records, on speaking only what one can justify by the state of the board. 

Yet as the company about the Cartwright grew, the account darkens. The youth writes that they began to keep fast-day vigils, to mark the rim of the board with damp salt, and to speak as little as possible between the first and the fifth move, as if silence were itself a counting. He calls these observances \textit{rites}, and protests that they were harmless while they were few.

Then comes a passage I can set down only in the driest manner, omitting much for decency. The youth names a man \textit{Ionius} and says he was \emph{given} at a ford beyond the east gate, under lanterns and in the rain. It is described not as punishment, but as an \emph{analogy enacted}: \textit{Ionius the witness at the ford}. Those who stood by were instructed to watch the water take him and to say nothing until dawn. The writer claims that, after, the Cartwright said the crossings of their lives would be \emph{priced more honestly} henceforth.

I do not pretend to judge whether the youth’s pen made a theatre of a lesser cruelty; but the recurrence of salt, silence, and the ford troubles me. I am, to my shame, \emph{intrigued} by the argument that an emblem, if lived, might impose measure upon those who otherwise prefer cleverness to clarity. I record the temptation here so that I may rebuke it later.

\medskip
\noindent\textit{Translator’s note (L.):} The idiom \textit{witness at the ford} appears again in two unrelated leaves; in both cases it bears the sense of a boundary made costly so that crossings are not squandered.

\medskip
\noindent\textbf{Present hand (910 AR), M.\ Fenwood}

This is where I ought to have stopped for the week. I am disturbed by Lexicus’s cool pen in the face of what he copies, and more disturbed by the admission that he was \emph{intrigued}. I was, for a moment, the same. Then I shut the book.

I am writing this to fix a rule for myself: I will not go back to that little room again. The shelves, the cobwebs, the salt rings, the way the margins seem to hum when the lantern is low—none of it is good for sleep. If there is anything worth recovering from those leaves, it can be done in daylight, with witnesses, and with less romance than I have lately allowed. 

I will take tomorrow’s notes in the public reading stalls and leave the key where it lies.

\subsection*{Fragment L--6 (648 AR / 910 AR) --- On Holy Geometry and a Sleepless Study}
\label{frag:l6}
\phantomsection
\addcontentsline{toc}{subsection}{Fragment L-6 (648 AR / 910 AR) --- On Holy Geometry and a Sleepless Study}

\noindent\textit{Provenance: a nocturnal journal leaf by M.\ Fenwood (910 AR), paired with a later entry by Lexicus (648 AR) concerning a mystic fragment styled ``Holy Geometry.''}

\medskip
\noindent\textbf{Present hand (910 AR), M.\ Fenwood}

I meant to keep away.

Instead I woke in the little study with my cheek on a folio, the lantern low and the dust making constellations of my sleeves. I could not say how I came there. I sat up, certain I should be unnerved, and yet what I felt was the opposite: that I was exactly where I ought to be, doing the thing my hands had learned without me.

A page lay open to a margin in Lexicus’s quick pencil. The note was nothing special—only a question mark beside a figure and a line: ``Name nothing yet.'' I closed the book and listened. The room kept its own breath.

\medskip
\noindent\textbf{Earlier hand (648 AR), Lexicus of Thepyrgos, Junior Reader}

Some weeks after my last, I have found a \emph{very} small tract stitched into the back of a sermon-book, the hand uneasy, the vellum thin. It names itself a \textit{fragment of Holy Geometry}. Much of it is crabbed and dark, but one passage shines through the cloud of its language: that there are those who hold \textit{Canr\'e} to be more than pastime, that its figures are \emph{forms of address} and that a board, rightly attended, is a \emph{prayer laid flat}.

I carried the slip at once to \textit{Aqyl, son of Aqyl}. He read three lines, then returned it to my palm and said, with more kindness than rebuke: ``Return it where you found it. Fragments torn from their bindings make liars of both sides.'' He looked at me a long while before adding: ``Are you sleeping, Lexicus? You have the look of a man who counts when he should be resting.''

I said I was well (which was not the truth entire) and promised to replace the fragment. I have done so, but the sentence about a prayer laid flat will not leave me.

\medskip
\noindent\textbf{Present hand (910 AR), M.\ Fenwood}

This room has been too easy to find of late. I am spending more hours here than at my other work and have begun to fall behind in the ordinary reading. Yesterday I met \textit{Aqyl, son of Aqyl} at the stairs. He took in my face and asked, very simply, ``Are you sleeping?'' I said I was (which was not the truth entire). He told me to eat something warm and to read in the public stalls for a night or two, where the lamps are bright and the bells keep time.

I am writing this in the morning to make a mark I can see from elsewhere. If I do not come back for a day, and the dreams return, I will count that as weather and not as instruction.

\subsection*{Fragment L--7 (648 AR / 910 AR) --- Fairy Stones and the Unnamed Ninth}
\label{frag:l7}
\phantomsection
\addcontentsline{toc}{subsection}{Fragment L-7 (648 AR / 910 AR) --- Fairy Stones and the Unnamed Ninth}

\noindent\textit{Provenance: a translated Theonan folio copied by Lexicus (648 AR); a later journal leaf by M.\ Fenwood (910 AR). The original island script is angular and spare; several lines are smoke-nibbled at the edges.}

\medskip
\noindent\textbf{Earlier hand (648 AR), Lexicus of Thepyrgos, Junior Reader}

This afternoon I rendered into our speech a small account from \textit{Theona}, the western isle. The writer calls his subject a game, but refuses its proper title. He writes:

\begin{quote}\small
``We will not speak the true name among strangers; in the market we say only \emph{Fairy Stones}. If you would learn it, you must swear to leave one seat empty and one turn uncounted.'' 
\end{quote}

The same folio collects, with little ceremony, those island notions our city delights to mock: giants that eat the flesh of the drowned; doorways that will not open if you name them; the unholiness of the number \textit{nine}. The author notes that fishers count their nets in eights and set aside the ninth rope as \emph{ward}; that in winter feasts the ninth cup is poured and left; that tollmen on the south road strike eight neat notches and then make a faint scratch for the ninth and look away.

It is a thin piece and not well argued, yet it holds together like the ribs of a small boat. The refusal to name, the leaving empty, the fear of nine---these seem to me related. If there is a custom behind it, it is not new.

\medskip
\noindent\textit{Lexicus’s margin:} Our city has its own little reverences (who does not leave the first line clear on a new page?), but we are fond of pretending they are accidents of neatness. Theonans do not pretend. I wonder whether \emph{Fairy Stones} masks a better name, or merely a truer one.

\medskip
\noindent\textbf{Present hand (910 AR), M.\ Fenwood}

Walking back from the study I found myself \emph{humming}. I do not know the tune. It keeps to itself, low and patient, as if it wanted to be a rope rather than a song. When I sat down, the words came without asking. I set them here so I can see what my hand did:

\begin{verse}
Eight steps to water, one left behind;\\
eight stones in a circle, one will not bind.\\
Say nothing at thresholds; name nothing at doors.\\
If you must cross, count softly:\\
one for the keeper, one for the ford,\\
one for the watcher who will not be stored.\\
Leave a cup poured; leave a chair bare.\\
What you do not call by its name\\
cannot answer your prayer.
\end{verse}

I do not recall ever hearing this. Perhaps it is only a trick of tiredness, the mind laying boards across a ditch. Still, the number keeps returning. I am writing it down here in the bright part of the day, where I can see it and decide later whether it is only weather.

\subsection*{Fragment L--8 (648 AR / 910 AR) --- A Dream of Holy Geometry}
\label{frag:l8}
\phantomsection
\addcontentsline{toc}{subsection}{Fragment L-8 (648 AR / 910 AR) --- A Dream of Holy Geometry}

\noindent\textit{Provenance: a night-note by Lexicus (648 AR) concerning a vivid dream; a later journal entry by M.\ Fenwood (910 AR). The Lexican leaf is blotched as if written upon waking.}

\medskip
\noindent\textbf{Earlier hand (648 AR), Lexicus of Thepyrgos, Junior Reader}

I woke before the bell with my hand already moving. The dream would not keep still unless I pinned it with ink.

I saw a plane laid out with lines that were not lines but \emph{bindings}, faint as breath on glass. The figure arranged itself by preference rather than command: eight regions that received the eye and a ninth that refused it, as if the page were shy of being complete. Around the outermost I felt a coolness, like damp salt drying to a ring.

There was a rhythm underneath (do not laugh at this), not music but the sense that a step awaited a step, and that speech would be in the way of it. In the middle space—if I may call it middle—I tried to set a small mark to test whether the figure were mine to disturb. The mark slid of its own accord to a neighboring place and would not abide where I had first wished it. I was not afraid, exactly. I was \emph{glad} to be contradicted by something that kept its own account, and then, right after, \emph{ashamed} to have felt glad.

I drew three little copies when I sat up. The first two are nothing but nets. The third preserves a hint of the ninth—an \emph{unwritable} hollow that makes the eights honest. I am excited by this and concerned in equal measure. If I am not careful I will begin to prefer the neatness of the emblem to the disorder of persons. This is a warning to myself to eat, to speak to someone, and to read plainer books for a day.

\medskip
\noindent\textit{Lexicus’s margin, later the same morning:} The phrase \emph{holy geometry} comes too easily to the tongue when one has slept poorly.

\medskip
\noindent\textbf{Present hand (910 AR), M.\ Fenwood}

Last night I dreamt of shapes that would not settle. Nothing dramatic—only angles that arrived where I meant curves, and a gap my eye kept walking around as if there were a low fence there. I am calling it \emph{the power of suggestion}. If you read enough about circles, you dream circles. If you spend too long copying talk of an unwelcome ninth, your mind obliges you with a hollow.

Still, I set down what I remember so I can file it with the weather rather than with omens: a sense of counting without numbers; a coolness at the edge of things; the feeling that silence itself is a kind of measure.

Today I will read in the bright stalls, keep to ordinary work, and eat something warm. If the shapes follow me into daylight, I will take that as proof that it is not the room but the hour—and I will sleep earlier.

\subsection*{Fragment L--9 (910 AR) --- A Package from Conlin: On the Twin--Oases}
\label{frag:l9}
\phantomsection
\addcontentsline{toc}{subsection}{Fragment L-9 (910 AR) --- A Package from Conlin: On the Twin-Oases}

\noindent\textit{Provenance: a bundle posted from C.\ Fenwood to M.\ Fenwood, containing family letters and working notes on \emph{Canr\'e}. One leaf bears a penciled header, ``Twin--Oases,'' struck through in a later hand.}

\medskip
\noindent\textbf{Present hand (910 AR), M.\ Fenwood}

This afternoon a parcel arrived from my brother \textit{Conlin}: old correspondence, a sheaf of copies, and two brittle offcuts that look like they were shaved from a larger draft of the \emph{Concordance}. He writes:

\begin{quote}\small
``Found these in the black trunk from the east room. Mostly letters. Two leaves look like Papa’s working pages. One is headed \emph{Twin--Oases}. The header’s scored out, but the notes underneath are tidy. If they’re yours by right, keep them. If not, return them to the trunk and pretend I never sent this.''
\end{quote}

The \emph{Twin--Oases} leaf is not an opening so much as a \emph{shape}: a way of beginning that treats two opposite edges as wells, keeps the middle shallow for a time, and asks the hands to move as if water were dear. In the margin, in my father’s small pencil:

\begin{quote}\small
``This feels like a \emph{rite}, not an opening. Disturbing. It teaches the crossing rather than the play.''
\end{quote}

I do not quite understand his unease. The figure reads as sensible to me: two “wets” to balance thirst, a hush at the start to price the first steps honestly, a circled rim that is not a wall. If a beginning can teach restraint, is that not a lesson worth keeping?

It unsettles me that it does not unsettle me.

Conlin’s note smells faintly of the cedar box; Papa’s pencil is faint enough that I had to angle the page to catch it. The header really is scored out with care, not rage. Someone decided this did not belong in a book of instruction.

I will ask \textit{Aqyl, son of Aqyl} whether he has seen this heading before, and why it was quietly moved to the drawer. For now I am copying the two leaves as they are and placing the originals back in the parcel. If Conlin asks, I will tell him they are safer in his cedar than in my desk.

\subsection*{Fragment L--10 (648 AR / 910 AR) --- On the House of Wells and the Witness}
\label{frag:l10}
\phantomsection
\addcontentsline{toc}{subsection}{Fragment L-10 (648 AR / 910 AR) --- On the House of Wells and the Witness}

\noindent\textit{Provenance: a stitched quire of uneven leaves in Lexicus’s hand (648 AR), ink crowded and overwarm at the margins; followed by a brief journal entry by M.\ Fenwood (910 AR). The quire’s thread is brittle and smells faintly of oil and salt.}

\medskip
\noindent\textbf{Earlier hand (648 AR), Lexicus of Thepyrgos, Junior Reader}

I will set down, at length, the structure as I now see it, that I may not lose the thread when the lamps gutter.

First: the \textit{House of Wells}. Their ledgers are proud of being plain, yet the plainness is a garment. I find small clippings of value (\emph{copper shaved thin}) recurring on days when ring-stamps are heavy; I find toll slips entered twice, once as road-duty and again as \emph{benefaction}. The difference is small enough to pass a sleepy audit, large enough to carry a habit along for years.

Second: the circle that names itself \textit{the Witness}. They do not write minutes; they write margins. Where the Cartwright teaches a careful beginning and a price for crossing, these copyists preach a crossing that \emph{watches back}. See the phrases recurring: \emph{ford held}; \emph{silence is counted}; \emph{salt binds the rim}. See also the strange practice of leaving one line unmarked and calling it \emph{kept for the ninth}.

Third: the junction of the two. In the Wells books, I mark chits stamped with triangles where coins should be; in the Witness margins, I mark triangles where tallies should be. The same hand? Two hands that learned from one teacher? I cannot yet prove it, but the drift is toward a cult of accountancy: that the world may be kept honest by making its thresholds costly. If it were only a philosophy, I would applaud. If it is a church, I am afraid.

There are voices in the next room.

They are not speaking to me exactly; they are \emph{answering} a question I have not asked, or have asked too softly. I set down an argument to quiet them:

\begin{enumerate}\setlength\itemsep{0.25em}
  \item If a rule is just, it will stand in daylight.
  \item If a rite requires secrecy, it is either childish theatre or theft.
  \item The Cartwright wrote to teach play, not to purchase souls.
  \item Therefore the Witness, if born of him, is a bastard child.
\end{enumerate}

The voices are not convinced. I could go into the next room and demand names, but that would break the frame of the work. I will not break the frame. The page is my room; the margin is my door; the door will stay shut until the figure shows itself without my calling.

\medskip
\noindent\textit{Lexicus’s margin, later:} If this be madness, let it at least be ruled madness. I will keep my sums straight even as my breath runs.

\medskip
\noindent\textbf{Present hand (910 AR), M.\ Fenwood}

I asked \textit{Aqyl, son of Aqyl} this morning, outright, whether there was ever a \emph{Cult of the Witness}. He frowned in a way I have not seen before and said only:

\begin{quote}\small
``There are places in this University that are not secrets, and yet they are best left alone. Stay out of that room. Read where the bells can scold you. If a thing is worth keeping, it does not need you to whisper to it.''
\end{quote}

I said I understood. I am not sure I do. I have written the warning here so that I must step over it if I ignore it.

\subsection*{Fragment L--11 (912 AR\textsuperscript{?} / 910 AR) --- The Veil of Names and the Closed Door}
\label{frag:l11}
\phantomsection
\addcontentsline{toc}{subsection}{Fragment L-11 (912 AR? / 910 AR) --- The Veil of Names and the Closed Door}

\noindent\textit{Provenance: an elated but unsteady note in the hand of Lexicus, dated ``912 AR'' in a later pencil (catalogers dispute the mark); paired with a present-day leaf by M.\ Fenwood (910 AR).}

\medskip
\noindent\textbf{Earlier hand (650 AR), Lexicus of Thepyrgos, Junior Reader}

I have it. I \emph{have} it, or the edge of it: the \textit{Veil of Names} is not a cipher of letters but a \emph{conduct}. One does not \emph{name} in certain rooms; one leaves the first line empty; one signs with a figure that is not one’s own. The grace of it is to keep the breath from preening. The danger is that a mask, worn for clarity, will begin to think for the face beneath.

For three nights my steps in sleep have brought me to a door I do not know by day. I will describe it so that I may not be tricked by my own haste: oak, swollen a little with damp; three iron straps across, three down, making nine intersections that are each stopped with a small rosette; the head of each rosette is pin-pricked around so that, taken together, they suggest a ring of salt-flowers. There is no handle. The keyhole is long, like an almond, with a tiny chip lifted from its lower lip. The sill shows a pale crust as if something had dried there and been swept away poorly.

Behind it, chanting. Not commands—\emph{keepings}. The syllables fold over one another like cloth; I know none of the words except the word I am \emph{not} to say.

I woke with my hand on the wall beside my bed, palm flat as if feeling for the grain of the oak. I am giddy with the nearness of it and sick with the same. If I open it (there must be a way, for why else the keyhole?), I will break the work. The \textit{Veil} says: do not name until the crossing is priced. So I will not name; I will keep the first line empty; I will count aloud over ordinary things and write what I can prove.

\medskip
\noindent\textit{Lexicus’s margin, cramped:} The door is a sentence. To open it is to finish the line before it earns its period.

\medskip
\noindent\textbf{Present hand (910 AR), M.\ Fenwood}

A voice in last night’s dream said, very plainly, ``Not yet. Do not show what you have. The time is not right.'' Then, as if offering a kindness: ``Do not trust Aqyl, son of Aqyl.''

Writing it now, in daylight, I dislike the taste of those sentences. They are not like my thoughts, nor like Aqyl’s speech. I am putting them here so that they must survive the morning to trouble me; that is a higher bar than a dream deserves.

If I were to be sensible, I would call this \emph{suggestion} and \emph{fatigue}. Read enough leaves in one voice and you begin to think in it; sleep too little and your mind will supply a chorus to save you the trouble of deciding. Still, the room keeps pulling at me, and the hour keeps slipping. I have not told Aqyl about the parcel from Conlin or the two cut leaves. I do not like that I am keeping two ledgers.

I will eat something warm, sit in the bright stalls, and say nothing of this page for a day. If the voice returns, it can make its case again—and I will hear it against the bells.

\subsection*{Fragment L--12 (650 AR / 910 AR) --- The Teacher, Copper \& Salt, and the Door}
\label{frag:l12}
\phantomsection
\addcontentsline{toc}{subsection}{Fragment L-12 (648 AR / 910 AR) --- The Teacher, Copper \& Salt, and the Door}

\noindent\textit{Provenance: two entries in Lexicus’s hand (first and “weeks later,” 648 AR), concerning a tutor of the Cartwright’s discipline; followed by a present-day leaf by M.\ Fenwood (910 AR).}

\medskip
\noindent\textbf{Earlier hand (650 AR), Lexicus of Thepyrgos, Junior Reader — first meeting}

I sought out a teacher reputed to keep the Cartwright’s strategy in the old, spare way. We sat at a plain table. Before a word, he drew a narrow ring of damp salt about the board’s edge and set his hands in his lap. When he finally spoke, it was in \emph{code}, as if discussing tariffs:

\begin{quote}\small
“Some crossings are \emph{priced}. Some vows are better \emph{kept unnamed}. A watcher keeps his ledger without boasting.”
\end{quote}

I answered that I knew of the vow he would not name, and (here I confess my rashness) that I had dreamt of a door—oak, nine rosettes, no handle—behind which men chant not commands but keepings. His eyes changed, not with anger but with a kind of inventory.

“You are before your hour,” he said at last. “When you can beat me clean at \textit{Canr\'e}, you will be ready to be shown what cannot be \emph{said}.”

We began to play.

\medskip
\noindent\textbf{Earlier hand (650 AR), Lexicus — weeks later}

After many sessions, and more losses than I care to number, I kept a steady book and at last defeated him—no cleverness, only the calm of counted breath. He did not sulk. He beckoned me close and spoke in a tongue I have not studied but \emph{understood} as if my thought were already arranged to receive it.

He called it the \textit{Rite of Copper and Salt}. A fragment of his utterance (set here without its cadence):

\begin{quote}\small
“\textit{Copper before threshold, salt upon rim; price what is entered, bind what is thin. Count what is taken, count what you spare; open what answers, and leave the rest bare.}”
\end{quote}

It is only \emph{logical}. Of course it is. Copper to mark what is dear; salt to make the edge honest; silence to value speech. He told me to come at dusk on the third day. “Bring no one,” he said. “The door you know will be open for those who have earned its measure.”

I am writing with a fast hand. The page feels too narrow for the hour.

\medskip
\noindent\textit{Lexicus’s margin, later that night:} The figure fits. I am not afraid. I am \emph{ready}. (I am also not sleeping.)

\medskip
\noindent\textbf{Present hand (910 AR), M.\ Fenwood}

I found the hall by accident, or so I tell myself. The corridor is the sort that collects notices and forgets them: flaking paint, old stencils, a buckled runner. But the door—\emph{the} door—was not like the corridor. Its oak looked newly rubbed; the iron rosettes were free of dust. On the sill, a fine pale crust had been brushed aside recently, not well.

There was a shallow scratch on the strike, fresh enough to shine. I put my palm to the wood and felt it cool, not cold. For a breath I thought I heard a low line of chanting, the kind that sits under the world like a seam. Then the lamps stuttered and the sound (if it was a sound) went thin as thread.

I did not try the keyhole. I stood and counted to eight and walked away before I could make a ninth step.

\subsection*{Fragment L--13 (650 AR / 910 AR) --- On Nine and the Remaining}
\label{frag:l13}
\phantomsection
\addcontentsline{toc}{subsection}{Fragment L-13 (650 AR / 910 AR) --- On Nine and the Remaining}

\noindent\textit{Provenance: a fervent note in Lexicus’s hand (650 AR), edges rubbed; paired with a present-day journal entry by M.\ Fenwood (910 AR).}

\medskip
\noindent\textbf{Earlier hand (650 AR), Lexicus of Thepyrgos, Junior Reader}

Nine, nine—they curse it so in the eastern isle, and will not number it aloud. My new companions are much the same. But in \textit{Dhahara} the same count is \emph{holy}. Their doctors say there are \emph{nine messengers}, pillars set to keep the world square; others, more careful, speak of \emph{the remaining}. They do not say \emph{from what} they remain. The courtesy is deliberate. I listen; I learn; and the more I learn the straighter the figure stands.

It is not a cipher of letters. It is a \emph{use}. Leave the first line empty. Count to eight and bind the edge. Refuse the ninth until the crossing is priced. If you do this, the room answers you without your naming it. I can see it now. \emph{I can see it}. The same hush that frightens the Theonan comforters is the hush the Dhaharan expositors bless.

\medskip
\noindent\textit{Lexicus’s margin, compressed:} If nine is a wound to some and a seal to others, then the work is to learn \emph{which} it is before speaking. (Do not be pleased with yourself for seeing the outline. Pride ruins sums.)

\medskip
\noindent\textbf{Present hand (910 AR), M.\ Fenwood}

Third night in a row with the corridor. This time the door was \emph{open}. No drama—just open, as if that were the ordinary state and I had been wrong about it before. The room beyond was dark in the way an unlit street is dark: not empty, only unhelpful.

I remember thinking, mid-dream, that I should be unnerved. I wasn’t. I stood at the threshold and felt a cool draft from inside, like air that has traveled along stone. I did not step through. I woke with my hand out, as if to touch the jamb.

I am calling this a \emph{recurring dream} and blaming it on pattern-seeking. Read enough about doors and you will dream one open eventually. Still, writing it down helps. If tomorrow’s version invites me in, I want the daylight version of me on record as someone who paused.

\subsection*{Fragment L--14 (653 AR / 910 AR) --- The Ash--Fenn Rite and the Candles}
\label{frag:l14}
\phantomsection
\addcontentsline{toc}{subsection}{Fragment L-14 (653 AR / 910 AR) --- The Ash-Fenn Rite and the Candles}

\noindent\textit{Provenance: an exultant, overfull entry in Lexicus’s hand (653 AR) recording his formal induction; paired with a present-day leaf by M.\ Fenwood (910 AR). The older page is salted at one edge; the newer smells faintly of lamp oil.}

\medskip
\noindent\textbf{Earlier hand (653 AR), Lexicus of Thepyrgos, Junior Reader}

Tonight they taught me the \textit{Ash--Fenn Rite}. I am bid not to name its steps in the open book; I am permitted to set down that it is a keeping, a pricing, and a remembering. They placed copper in my palm and circled my wrists with a ribbon touched to salt and water. The words were not the teacher’s words alone; the room made them easy to hear, as if I had been rehearsing them unawares.

I am—let the sentence stand—I am \emph{ecstatic}. Not with noise, but with a swift quiet that runs under the ribs. I am to learn and record what I can prove; I am to keep the first line empty and the last line honest. I am to be no one’s cleverness. I am to be a witness.

They say the circle grew from the Cartwright’s discipline and took its charge from the burning east of the gates. They say the Rite binds a wound without boasting of the scar. I believe them. I \emph{see} it: thresholds priced so crossings are not squandered; silence given its measure; the edge made true so the middle may be merciful.

I will keep two ledgers henceforth: one for daylight and one for the room. They do not contradict; they converse. If there is danger in this, it is that the night-book will begin to think for the day. I will not allow it. (I write this like a charm; perhaps it \emph{is} one.)

\medskip
\noindent\textit{Lexicus’s margin, crowded:} The Ash, the Fenn; the salt, the copper. Penance and price. I am steady. I am steady.

\medskip
\noindent\textbf{Present hand (910 AR), M.\ Fenwood}

I walked to the door again. I told myself I was only confirming the corridor. The oak felt cool under my palm, the kind of cool that lives a little deeper than the surface. I pressed my ear to the seam to be clever and earned a headache that bloomed behind the eye, not sharp, just \emph{insistent}. I stepped back and laughed at myself in the empty hall, which made the place feel less empty.

The next night I dreamt the room beyond. There was a table and candles set around it—not a crowd, not a feast, just a shape made of light. I counted them before I could think not to. I woke with the number sitting in my mouth like a word I was not ready to say.

This is a record, not a vow. Tomorrow I will read where the bells can scold me and let the door be wood again.

\subsection*{Fragment L--15 (654 AR / 910 AR) --- The Ninth and the Open Room}
\label{frag:l15}
\phantomsection
\addcontentsline{toc}{subsection}{Fragment L-15 (654 AR / 910 AR) --- The Ninth and the Open Room}

\noindent\textit{Provenance: a scorched leaf in Lexicus’s hand (654 AR), edges singed and brittle; paired with a present-day entry by M.\ Fenwood (910 AR). The older ink is hurried and overbold.}

\medskip
\noindent\textbf{Earlier hand (654 AR), Lexicus of Thepyrgos, Junior Reader}

They have taught me \emph{The Ninth}. I am commanded to speak it only in keeping, and I have obeyed. I have burned the day-book. I do not repent it. A book that cannot keep a vow is worse than no book at all.

I will set the shape in \emph{code}, for code is a fence that lets the field breathe:

\begin{quote}\small
\textit{Leave the first line empty. Bind the rim with what bites the tongue. Count crossings as prices, not as boasts. When the word that is not said is ready to be said, speak it once and only once. Then, for one breath, unmake the binding you have made. Do not spend that breath on cleverness. Spend it on measure. Close the book before the smoke names you.}
\end{quote}

It is clear. It is \emph{clear}. For a breath one may set aside a keeping—not to cheat, but to seal. The wound is named, the rim is cracked and made true again. I know the word. I will not write it. The page would not carry it without bending.

The room received me. The door did not resist. The watchers keep no faces, only ledgers. I am not afraid. The ash that clings to my sleeve is only ash. The work goes on where the bells cannot scold it.

\medskip
\noindent\textit{Lexicus’s margin, scorched:} The book that is not burned still burns the hand if it is kept open too long.

\medskip
\noindent\textbf{Present hand (910 AR), M.\ Fenwood}

The door was open.

I did not need a key today. The corridor that always looks tired looked merely itself; the oak looked newly rubbed, like before. \textit{Aqyl, son of Aqyl} stood within, beside a table where a \textit{Canr\'e} board was already set. The room was clean, not the dust-chamber I expected. Candles had been used and put out with care; the scent of them still hung in the wood.

He said my name in the ordinary way and then: “Come in.”

I am writing this much in the bright stalls before I go any further with my own thoughts. The bells are loud here. My hands are steady enough to copy what I saw and to stop there, for now.

\subsection*{Fragment L--16 (910 AR) --- Lerris Enrolls; A Quiet Vow}
\label{frag:l16}
\phantomsection
\addcontentsline{toc}{subsection}{Fragment L-16 (910 AR) --- Lerris Enrolls; A Quiet Vow}

\noindent\textit{Provenance: a brief present-day note by M.\ Fenwood; the paper is new, the hand steady.}

\medskip
\noindent\textbf{Present hand (910 AR), M.\ Fenwood}

My brother \textit{Lerris} arrived with a trunk and a grin and an enrollment chit that still smelled of fresh wax. We walked the outer courts like we used to walk the lanes at home, naming buildings, counting bells. He is taller than I remember and kinder than I deserve. He asked where to begin.

I said the bright stalls, the public rooms, the catalog, the ordinary work. I said the words a good brother says. Then I thought of the little study and the cool of the oak and the way some pages answer if you are patient. I thought of the door.

I am excited to teach him what I have learned. I am excited to teach him the \textit{Truth} about \textit{Kon\'reh}—why the edge matters, why the silence is counted, why the middle is a place one \emph{passes} rather than a place one \emph{keeps}. I hear myself and I do not flinch.

\medskip
Tomorrow I will show him the library in daylight and the stalls where the bells can scold us. If the hour asks for more, I will know it when the room answers. For now I will write down what I can prove and leave the first line empty.

\medskip
\noindent\textit{Small hand at the foot of the page:} The Ninth demands it\ldots{}

\section*{Part II — Rites \& Scenarios}
\label{part:rites}
\phantomsection
\addcontentsline{toc}{section}{Part II — Rites \& Scenarios}

\begin{quote}\small
``Call a rule a law for the hand; call a rite a lens for the eye.  
Turn the lens and the same moves speak differently. Remove it, and nothing is broken.''\\
\hfill — \textit{Lexicus of Thepyrgos}, margin note
\end{quote}

\noindent\textbf{What follows.} This part presents seven self-contained overlays (\emph{Rites}) you may toggle \textsc{ON} or \textsc{OFF} per session. Each one adjusts \textit{what you notice}—pace, pressure, information—without changing the tournament core. Leave any Rite \textsc{OFF} and play proceeds exactly as you already know it.

\medskip
\noindent\textbf{Anatomy of a scenario.} Every scenario includes:
\begin{itemize}\setlength\itemsep{0.3em}
  \item \textbf{Hook} — the in-world cue that frames the table mood.
  \item \textbf{Switch \& Scope} — explicit \textsc{ON/OFF}; never touches core rules.
  \item \textbf{Setup \& Components} — simple markers (ribbon, beads, chits) you can substitute from household items.
  \item \textbf{Rite (exact)} — concise, rules-precise text.
  \item \textbf{Clarifications} — edge cases, if any.
  \item \textbf{Clue beat} — an optional thread in the dossier’s larger puzzle.
  \item \textbf{Why it’s safe} — a brief reassurance of rules integrity.
  \item \textbf{If \textsc{OFF}} — a reminder that play is unchanged.
\end{itemize}

\medskip
\noindent\textbf{Reading the cues.} When a scenario mentions documents or fragments, they enrich the scene but never confer hidden advantages. The choice to use a Rite is aesthetic and procedural, not compulsory.

\medskip
\noindent\textbf{Suggested table aids.} A slim ribbon (for the perimeter), five beads or coins (for public counts), and a few copper-colored tokens (for optional actions). Printable templates appear in the appendices.

\medskip
\noindent\textit{Begin with \S\ref{scen:salt-stitch} \textemdash{} a simple perimeter ward that makes the edge feel like a threshold without moving a single rule.}

\subsection*{Scenario 1 — Salt Stitch}
\label{scen:salt-stitch}
\phantomsection
\addcontentsline{toc}{subsection}{Scenario 1 — Salt Stitch}

\noindent\textbf{Hook.} Markus notes the caravan custom of dusting thresholds with salt; the board’s rim echoes the ward.

\medskip
\noindent\textbf{Switch.} \textsc{ON / OFF} (default \textsc{OFF}) \hfill \textbf{Scope.} Scenario-scoped; never alters tournament core.

\medskip
\noindent\textbf{Components.} A thin ribbon, string, or corner markers sufficient to indicate the board’s outermost ring (\emph{salt-band}).

\medskip
\noindent\textbf{Setup.} Before normal setup, lay a visible \emph{salt-band} tracing the perimeter ring.

\medskip
\noindent\textbf{Rite (exact).} While this Rite is \textsc{ON}, \textbf{Green} pieces may not \emph{enter}, \emph{cross}, or \emph{end} on any perimeter square. All other pieces ignore the band. No other rules change.

\medskip
\noindent\textbf{Clarifications.}
\begin{itemize}\setlength\itemsep{0.25em}
  \item If any effect would place a Green onto a perimeter square (including specials or setup variants), that placement is illegal; choose another legal option.
  \item Adjacency across the band is allowed; only occupancy/crossing of perimeter squares by Greens is prohibited.
  \item Captures by or against Greens proceed normally, provided the Green’s path does not require stepping onto a perimeter square.
\end{itemize}

\medskip
\noindent\textbf{Example (two turns).}  
A Green approaches the rim to cut a lane; the opponent pressures along the edge. The Green must weave one file inward (may not step onto, or hop through, the perimeter), delaying the escape by a tempo without changing any costs or capture rules.

\medskip
\noindent\textbf{Clue beat.} A city map leaf shows pinholes; rotate \(\,17^\circ\) and sanctums align with gatehouses.

\medskip
\noindent\textbf{Why it’s safe.} Only restricts where \emph{Greens} may legally occupy or traverse; does not alter move generation, costs, rooting, specials, or scoring.

\medskip
\noindent\textbf{If \textsc{OFF}:} Play is unchanged.

\medskip
\noindent\textit{Diagram cue (optional).} Perimeter traced; corner exemplars marked with \(\times\) to illustrate barred \emph{end} squares for Greens (caption: all perimeter squares are barred to Greens).

\subsection*{Scenario 2 — Witness at the Ford}
\label{scen:witness-ford}
\phantomsection
\addcontentsline{toc}{subsection}{Scenario 2 — Witness at the Ford}

\noindent\textbf{Hook.} “Stay more than thrice at the ford and the Witness takes you.” The center is a crossing, not a seat.

\medskip
\noindent\textbf{Switch.} \textsc{ON / OFF} (default \textsc{OFF}) \hfill \textbf{Scope.} Scenario-scoped; never alters tournament core.

\medskip
\noindent\textbf{Components.} None beyond the standard set (optional: a small die or bead per Blue to track counts).

\medskip
\noindent\textbf{Setup.} None.

\medskip
\noindent\textbf{Rite (exact).} While this Rite is \textsc{ON}, each \textbf{Blue} tracks how many of its controller’s \emph{consecutive turns} it has \emph{ended} on any square of the \textbf{Central Four} (the 2\(\times\)2 center). If a Blue would end a \textbf{4\textsuperscript{th} consecutive turn} in the Central Four, that Blue is \textbf{forfeit} (captured) immediately at end of turn. \emph{Leaving} the Central Four at any point resets that Blue’s count to 0.

\medskip
\noindent\textbf{Clarifications.}
\begin{itemize}\setlength\itemsep{0.25em}
  \item “Consecutive turns” means turns of that Blue’s \emph{controller}. If a Blue remains in the Central Four across an opponent’s turn, the count is unaffected until its controller’s next turn ends.
  \item A Blue that is \emph{Rooted} within the Central Four still accumulates counts (it is “staying”).
  \item If a Blue leaves the Central Four (by moving, being displaced, or captured) its count resets to 0. If it later re-enters, start again from 1 the next time it ends its controller’s turn there.
  \item Track each Blue separately. If a Blue is captured before reaching four, remove its counter (if any).
  \item If a pass or skip occurs and the Blue remains in the Central Four, that still counts as having “ended the turn” there.
\end{itemize}

\medskip
\noindent\textbf{Example (tempo at the ford).}  
On Turn A1, Blue\(_\alpha\) ends in the Central Four (count 1). On B1 nothing changes for Blue\(_\alpha\). On A2 it still ends there (count 2); on A3 (count 3). If on A4 it would again end in the Central Four, it is forfeited at end of A4. If instead it steps out on A3 and returns on A4, the count resets—A4 becomes count 1.

\medskip
\noindent\textbf{Clue beat.} A toll ledger shows three sequential stamps; the fourth is crossed in ash, matching the “three then price” cadence seen in Candle Count.

\medskip
\noindent\textbf{Why it’s safe.} Adds a local stay-timer only for \emph{Blues} in the Central Four; no movement rules, costs, specials, or captures elsewhere are changed.

\medskip
\noindent\textbf{If \textsc{OFF}:} Play is unchanged.

\medskip
\noindent\textit{Diagram cue (optional).} Shade the 2\(\times\)2 center; place a small numeral bead beside any Blue that ends there to show its current count (1–3).

\subsection*{Scenario 3 — Veil of Names}
\label{scen:veil-of-names}
\phantomsection
\addcontentsline{toc}{subsection}{Scenario 3 — Veil of Names}

\noindent\textbf{Hook.} The cult forbids doctrine-naming: “Names bind exits.” Play as if identity were a mask you choose not to lift.

\medskip
\noindent\textbf{Switch.} \textsc{ON / OFF} (default \textsc{OFF}) \hfill \textbf{Scope.} Scenario-scoped; never alters tournament core.

\medskip
\noindent\textbf{Components.} Optional: two blank cards or slips for private notes (\emph{face-down school cards}).

\medskip
\noindent\textbf{Setup.} If desired, each player writes a single word or icon on a slip (their “school” or guiding maxim) and places it face-down near their board edge. Do not reveal it during play.

\medskip
\noindent\textbf{Rite (exact).} While this Rite is \textsc{ON}, \textbf{players do not declare schools or doctrines publicly}. Any thematic school choice is kept private and has \textbf{no mechanical effect}. All rules, moves, costs, specials, scoring, and timers proceed exactly as normal.

\medskip
\noindent\textbf{Clarifications.}
\begin{itemize}\setlength\itemsep{0.25em}
  \item This Rite introduces \emph{secrecy of theme only}. It does not grant, suppress, or modify any ability.
  \item If a format or local variant requires a public pre-declaration, \emph{ignore this Rite} for that session.
  \item Table talk and reads are allowed as normal; players may infer style but must not demand disclosure.
  \item At the end of the game, players may reveal their face-down slips for color; revealing is optional and confers nothing.
\end{itemize}

\medskip
\noindent\textbf{Example (table texture).}  
Two players sit without naming doctrine. One leans into cautious exits; the other pressures lanes early. Each is reading the other’s \emph{play}, not their banner—tension without rule-changes.

\medskip
\noindent\textbf{Clue beat.} Two training slates in the dossier mark evaluations that match no known doctrine—scholars nickname it the \emph{Gray Ledger}, which “prices exits rather than blocking them.”

\medskip
\noindent\textbf{Why it’s safe.} Pure information fog: no move generation, costs, captures, rooting, specials, or scoring are changed.

\medskip
\noindent\textbf{If \textsc{OFF}:} Play is unchanged.

\medskip
\noindent\textit{Diagram cue (optional).} Small icon of a face-down card with a crossed speech bubble; caption: “Theme kept private; rules unchanged.”

\subsection*{Scenario 4 — Candle Count}
\label{scen:candle-count}
\phantomsection
\addcontentsline{toc}{subsection}{Scenario 4 — Candle Count}

\noindent\textbf{Hook.} A chant sheet with five circles; bead marks that read like a ledger. Make tempo visible without changing a single rule.

\medskip
\noindent\textbf{Switch.} \textsc{ON / OFF} (default \textsc{OFF}) \hfill \textbf{Scope.} Scenario-scoped; never alters tournament core.

\medskip
\noindent\textbf{Components.} A five-step track and \textbf{5 beads} (coins, pebbles, or counters). Optional: a divider to separate \emph{unlit} and \emph{lit} sides.

\medskip
\noindent\textbf{Setup.} Place the five beads on the \emph{unlit} side of the track within view of both players.

\medskip
\noindent\textbf{Rite (exact).} While this Rite is \textsc{ON}, slide \textbf{one bead} from \emph{unlit} to \emph{lit} each time any of the following occurs:
\begin{enumerate}\setlength\itemsep{0.2em}
  \item A piece \textbf{ends} its turn in the \textbf{Cross}.
  \item A \textbf{Blue} becomes \textbf{Rooted}.
  \item Any \textbf{special} is used.
\end{enumerate}
When all five beads are lit, announce “\textit{chant complete},” then \textbf{reset} all five beads to the unlit side. No other rules change.

\medskip
\noindent\textbf{Clarifications.}
\begin{itemize}\setlength\itemsep{0.25em}
  \item If multiple triggers occur during a single turn (e.g., a special is used and a piece ends in the Cross), slide one bead per trigger, up to the remaining unlit beads.
  \item If the fifth bead lights mid-turn, announce completion immediately, reset the track, and continue play; additional triggers that same turn begin filling the new cycle.
  \item The track is public information; either player may slide beads when a trigger occurs (good manners: announce aloud).
  \item This Rite does \emph{not} add time, costs, or constraints; it only records notable moments.
\end{itemize}

\medskip
\noindent\textbf{Example (one cycle).}  
Turn A: a special is used (\(+1\)). Turn B: no triggers. Turn A: a Blue is Rooted (\(+1\)). Turn B: a piece ends in the Cross (\(+1\)). Turn A: special (\(+1\)). Turn B: Cross end again (\(+1\)) \(\Rightarrow\) fifth bead lights, chant complete; reset. Play continues unchanged.

\medskip
\noindent\textbf{Clue beat.} Each \emph{chant complete} marks a base-5 digit on the dossier’s brass strip; across scenarios, digits map to letters that assemble the final \emph{Wound Name} (see Part III).

\medskip
\noindent\textbf{Why it’s safe.} Pure information surfacing: no movement, capture, rooting, specials, scoring, or timers are altered.

\medskip
\noindent\textbf{If \textsc{OFF}:} Play is unchanged.

\medskip
\noindent\textit{Diagram cue (optional).} A five-circle track with three lit markers \((\bullet\bullet\bullet\circ\circ)\) and icons beside the triggers: \([{\small ✚}\ \text{Cross}]\), \([{\small \raisebox{0.15em}{\(\,\bot\,\)}}\ \text{Rooted}]\), \([{\small S}\ \text{Special}]\).

\subsection*{Scenario 5 — Copper \& Salt}
\label{scen:copper-and-salt}
\phantomsection
\addcontentsline{toc}{subsection}{Scenario 5 — Copper \& Salt}

\noindent\textbf{Hook.} House-of-Wells ledgers mirror temple tithes; little chits move where coins should be.

\medskip
\noindent\textbf{Switch.} \textsc{ON / OFF} (default \textsc{OFF}) \hfill \textbf{Scope.} Scenario-scoped; never alters tournament core.

\medskip
\noindent\textbf{Components.} \textbf{3 Copper} tokens per player; \emph{Patrol} marker (a single coin or chit); optional \emph{Cleared} marker (for Salt-band gaps).

\medskip
\noindent\textbf{Setup.} Give each player a personal bank with a \textbf{capacity of 3} Copper. Each player \textbf{starts with 0} Copper (unless another Rite says otherwise; e.g., \S\ref{scen:ash-fenn-rite} \emph{Oath of Copper} starts with \(+2\)). Place the Patrol and Cleared markers within reach.

\medskip
\noindent\textbf{Rite (exact).} While this Rite is \textsc{ON}:
\begin{enumerate}\setlength\itemsep{0.2em}
  \item \textbf{Gain Copper.} When you \textbf{Seed} or establish a legal \textbf{banner lane}, gain \(\,+1\) Copper (to a maximum of 3). Excess is lost.
  \item \textbf{Spend Copper (once, before your turn begins).} You may spend \(\,1\) Copper to do \emph{one} of the following:
  \begin{enumerate}\setlength\itemsep{0.2em}
    \item \textbf{Purify (Salt).} If \emph{Salt Stitch} (\S\ref{scen:salt-stitch}) is \textsc{ON}, choose any \textbf{perimeter} square and mark it \emph{Cleared}. For the remainder of the game, Greens treat that square as normal (they may \emph{enter/cross/end} there despite the salt-band). 
    \item \textbf{Peek (Leaf).} Draw/peek the next \emph{leaf} or clue packet for this scenario’s dossier thread (if in use), then return/resolve as instructed. (If no clue packet is being used, ignore this option.)
    \item \textbf{Patrol (Perimeter).} Place a \emph{Patrol} marker on a \textbf{perimeter} square. \textit{Until the start of your next turn}, your opponent may \textbf{not end} a move on that marked square. (Moving \emph{through} is allowed.) Remove the marker at the start of your next turn.
  \end{enumerate}
\end{enumerate}

\medskip
\noindent\textbf{Clarifications.}
\begin{itemize}\setlength\itemsep{0.25em}
  \item Copper tokens have no value toward victory; they are a light economy used only for the three options above.
  \item You may gain at most \(+2\) Copper on a single turn if you both \emph{Seed} and complete a \emph{banner lane}, respecting the cap of 3.
  \item \textit{Purify} is unavailable if \emph{Salt Stitch} is \textsc{OFF}. Each \emph{Cleared} square is permanent for the rest of the game.
  \item \textit{Patrol} denies only \emph{ending} a move on that square and only to your opponent; it expires at the start of \emph{your} next turn.
  \item Spending occurs \emph{before} your turn begins; you may not chain multiple spends in the same turn (choose one option per turn at most).
\end{itemize}

\medskip
\noindent\textbf{Example (one cycle).}  
You \emph{Seed} this turn (\(+1\) Copper; total now 1). Before your next turn, you spend that 1 Copper on \emph{Purify}, marking a rim square \emph{Cleared}. On the following turn you establish a \emph{banner lane} (\(+1\); total 1) and spend it \emph{before} your next turn on \emph{Patrol}, denying an enemy end on a critical corner until your turn returns.

\medskip
\noindent\textbf{Clue beat.} A tithe ledger phrases the \emph{Reforge Compact} like a hostage-exchange clause; the \emph{Peek} option reveals a marginal annotation pointing toward the finale.

\medskip
\noindent\textbf{Why it’s safe.} Off-board tokens and markers affect only \emph{information and tempo nudges}; no movement, capture, costs, rooting, specials, scoring, or timers are altered.

\medskip
\noindent\textbf{If \textsc{OFF}:} Play is unchanged.

\medskip
\noindent\textit{Diagram cue (optional).} Three coin icons near each player; a \emph{Cleared} dot on one rim square; a small shield icon (\(\small\shield\)) on a patrolled rim square (captioned “opponent may not \emph{end} here until your next turn”).

\subsection*{Scenario 6 — Ash–Fenn Rite}
\label{scen:ash-fenn-rite}
\phantomsection
\addcontentsline{toc}{subsection}{Scenario 6 — Ash–Fenn Rite}

\noindent\textbf{Hook.} The vault burned; players bind themselves to an opening penance—trading early ease for later relief.

\medskip
\noindent\textbf{Switch.} \textsc{ON / OFF} (default \textsc{OFF}) \hfill \textbf{Scope.} Scenario-scoped; never alters tournament core.

\medskip
\noindent\textbf{Components.} Optional: three \emph{Oath} cards per player (\textit{Water}, \textit{Salt}, \textit{Copper}); one small marker labeled \emph{Bound}.

\medskip
\noindent\textbf{Setup.} Each player \textbf{chooses exactly one} Oath secretly, then reveals simultaneously. Both players may choose the same Oath.

\medskip
\noindent\textbf{Rite (exact).} While this Rite is \textsc{ON}, each player gains the text of their chosen Oath:

\begin{description}\setlength\itemsep{0.35em}
  \item[\textit{Oath of Water.}] The \textbf{first time this game you would move a Blue}, you instead \textbf{forgo that move} (treat it as a voluntary pass for that Blue only; no other effects). \emph{Once per game}, you may have one of your Blues \textbf{step from a sanctum into the Cross even if it is Rooted}. This step must otherwise be legal (ignores only the Rooted restriction).

  \item[\textit{Oath of Salt.}] If \emph{Salt Stitch} (\S\ref{scen:salt-stitch}) is \textsc{ON}, you begin under the salt band as normal, but \emph{once per game} you may \textbf{ignore the salt-band for a single Blue move} (that Blue may \emph{enter/cross/end} on perimeter squares for that move only). \emph{If Salt Stitch is \textsc{OFF}}, this Oath has no effect (choose another Oath during setup if desired).

  \item[\textit{Oath of Copper.}] Start the game with \textbf{+2 Copper} (see \S\ref{scen:copper-and-salt}; your bank cap of 3 still applies). The \textbf{first time you Seed} this game, place a \emph{Bound} marker on that Seed’s target and \textbf{delay all effects of that Seed} (including any rooting or derived effects) \textbf{until the end of your \emph{next} turn}. Remove the \emph{Bound} marker when it resolves. If the target becomes illegal before resolution, the Seed \emph{fizzles} (remove it with no effect).
\end{description}

\medskip
\noindent\textbf{Clarifications.}
\begin{itemize}\setlength\itemsep{0.25em}
  \item \textit{Oath of Water}: “first time you would move a Blue” refers to the first Blue move you attempt in the game, regardless of turn number; you cannot “skip” triggering by selecting a different piece—trigger occurs at the first legal Blue move you choose.
  \item \textit{Oath of Salt}: the one-move waiver applies to a \emph{single Blue move} only and does not persist; other Blues and later moves remain bound by the salt-band.
  \item \textit{Oath of Copper}: you cannot exceed a personal Copper capacity of 3; gaining beyond the cap is lost. The delayed Seed has \emph{no interim effects} until it resolves.
  \item You may never benefit from more than one Oath; the unchosen Oaths confer nothing.
\end{itemize}

\medskip
\noindent\textbf{Example (opening stitches).}  
Player A takes \textit{Water}: on their first attempted Blue move they forgo it, later using the once-per-game sanctum\(\to\)Cross step to break a midgame bind.  
Player B takes \textit{Copper}: starts at 2 Copper (\( \to \) cap 3 on early gains) and declares a Seed; it sits \emph{Bound} until the end of their next turn, then roots normally—buying tempo now, paying certainty later.

\medskip
\noindent\textbf{Clue beat.} A scraped page from the \emph{Concordance}—the redacted \emph{Twin–Oases}—links penance to crossing; a marginal hand hints at a final word reserved for the closing scenario.

\medskip
\noindent\textbf{Why it’s safe.} One-time commitments and permissions that \emph{gate timing only}; core movement, captures, costs, specials, scoring, and timers remain pristine.

\medskip
\noindent\textbf{If \textsc{OFF}:} Play is unchanged.

\medskip
\noindent\textit{Diagram cue (optional).} Three small oath icons at player edge (\(\lozenge\) Water, \(\triangle\) Salt, \(\circ\) Copper); a \emph{Bound} marker on a pending Seed; a single arrow from sanctum to Cross labeled “once”.

\subsection*{Scenario 7 — The Ninth}
\label{scen:the-ninth}
\phantomsection
\addcontentsline{toc}{subsection}{Scenario 7 — The Ninth}

\noindent\textbf{Hook.} “Count eight before breath; the ninth is a wound.” The dossier insists there is a word you do not say until the crossing is priced.

\medskip
\noindent\textbf{Switch.} \textsc{ON / OFF} (default \textsc{OFF}) \hfill \textbf{Scope.} Scenario-scoped; never alters tournament core.

\medskip
\noindent\textbf{Components.} \emph{Wound Name} card (assembled from prior letters) or a sealed Name from the appendix; optional \emph{Crack} marker for one perimeter arc.

\medskip
\noindent\textbf{Setup.}
\begin{enumerate}\setlength\itemsep{0.2em}
  \item \textbf{Campaign mode.} Using letters decoded across earlier scenarios (e.g., \S\ref{scen:candle-count}), assemble the \emph{Wound Name}. Write it on a slip and place it \textbf{face-down} near the board.
  \item \textbf{One–shot mode.} Draw a sealed \emph{Wound Name} from the appendix and place it \textbf{face-down}.
  \item \textbf{Optional bite.} Mark a single perimeter \emph{arc} as \emph{cracked}. Pieces may \emph{pass through} that arc as normal but may \textbf{not end} on its two corner squares \textit{until} the Wound is named.
\end{enumerate}

\medskip
\noindent\textbf{Rite (exact).} While this Rite is \textsc{ON}:
\begin{enumerate}\setlength\itemsep{0.2em}
  \item On your turn, \textbf{after} completing a legal move, you may \textbf{speak} a single candidate Wound Name aloud (at most once per turn).
  \item If the spoken Name \textbf{matches} the face-down Wound Name, immediately gain a \textbf{one–turn Rite–Break}: \emph{choose any one active Rite} and \textbf{suspend its rules for you only} for the remainder of your current turn. Then reveal the face-down Name and read the final leaf.
  \item If the spoken Name \textbf{does not match}, nothing happens; you may try again on a later turn.
\end{enumerate}

\medskip
\noindent\textbf{Clarifications.}
\begin{itemize}\setlength\itemsep{0.25em}
  \item \textit{Rite–Break} affects \emph{only} Rites that are currently \textsc{ON} (e.g., ignore the salt-band once from \S\ref{scen:salt-stitch}; nullify the center stay-timer from \S\ref{scen:witness-ford}; take an immediate Peek from \S\ref{scen:copper-and-salt} without spending Copper). It never alters core movement, capture, costs, or timers.
  \item If \emph{no} other Rite is \textsc{ON}, naming the Wound has \textbf{no mechanical effect}; reveal the Name and proceed (narrative only).
  \item \textit{Optional bite} (\emph{cracked arc}) is a cosmetic constraint you may omit for a purer board. If used, the restriction on ending on the two corner squares is lifted as soon as the Wound is named.
  \item Verification: if dispute arises, consult the sealed appendix solution for the Name chosen at setup.
\end{itemize}

\medskip
\noindent\textbf{Example (resolution).}  
Midgame with \emph{Salt Stitch} and \emph{Witness at the Ford} both \textsc{ON}, a player completes a legal move, speaks the correct Name, then chooses to suspend \emph{Witness at the Ford} for this turn only—allowing a fourth consecutive end in the center to set up a capture. The Name is revealed; play continues under normal Rite effects from the next turn onward.

\medskip
\noindent\textbf{Clue beat.} The final letters from \S\ref{scen:candle-count} completions plus two marginal initials hidden in the dossier produce the Name; the face-down slip confirms it.

\medskip
\noindent\textbf{Why it’s safe.} The transient power touches \emph{Rites only}; core legality, win conditions, and scoring remain pristine. The \emph{cracked arc} is optional and easily omitted.

\medskip
\noindent\textbf{If \textsc{OFF}:} Play is unchanged.

\medskip
\noindent\textit{Diagram cue (optional).} A small face-down card icon labeled “Wound Name”; a dashed highlight over one perimeter arc (\emph{cracked}) and a tiny toggle symbol to illustrate a one–turn suspension of a Rite.

\subsection*{Epilogue — A Dhaharan Leaf: On Honoring the Ninth}
\label{epilogue:dhahara-ninth}
\phantomsection
\addcontentsline{toc}{subsection}{Epilogue — A Dhaharan Leaf: On Honoring the Ninth}

\noindent\textit{Provenance: a short Dhaharan folio, margin-stamped for a winter observance; translated with light normalization. The original script is calm and spacious, with long breaths between clauses.}

\medskip
\begin{quote}\small
We are not as the Ykrul counters nor the Kuvani whisperers who fear the last number and speak around it. We do not goad it nor bargain with it. We make room.

At dusk we sweep the threshold and lay a narrow ring of salt where feet remember to step. We wash the hands in water that has touched a word. We set three lanterns, then three, then three; we leave one unlit and do not explain ourselves.

We kindle incense until the air remembers a road. We say the small verses that do not ask. We count the breaths as if we were listening for a guest.

We drink the mild wine and we sit with the stones and the board until the lamps lean. We do not wager coin; we are practicing keeping. We speak little. When laughter comes we do not chase it away.

In the deep hour we pour a ninth cup and leave it where morning will find it. The ring of salt is not a wall; it is a way to notice the edge.

When the light returns we unmake what we made. We scatter a pinch of salt to the four roads. We lift the unlit lantern and let the day choose it. We do not declare that we have learned; we look again, and something that would not come forward before stands a little nearer, like a page held at a good angle.

If there is a ninth, it is not a captain nor a judge. It is the remaining: the part of the count that waits without complaint, the rest between measures, the seat one leaves empty so the room can breathe. Honor it, and let play teach the rest.
\end{quote}

\medskip
\noindent\textit{Translator’s note.} The Dhaharan term rendered here as “the remaining” admits readings as “what abides” or “what stays available.” The leaf does not command belief; it suggests a practice. The board is present as craft, not dogma.



\section*{Part III — Puzzles \& Scholar’s Path}
\label{part:puzzles}
\phantomsection
\addcontentsline{toc}{section}{Part III — Puzzles \& Scholar’s Path}

\begin{quote}\small
“Do not be clever first. Be clear first. Then, if the page still asks, be patient.”\\
\hfill — \textit{Aqyl}, to a junior reader
\end{quote}

\noindent\textbf{What this part is.} A set of diegetic puzzles woven from the dossier’s artifacts—maps, stamps, bead-tracks, marginalia. Each puzzle is self-contained and \emph{optional}. Solving changes what you notice, not what you are allowed to do.

\medskip
\noindent\textbf{What it is not.} There are no secret rules hidden here. Tournament play remains untouched whether you solve everything or nothing.

\medskip
\noindent\textbf{Fairness \& construction.}
\begin{itemize}\setlength\itemsep{0.3em}
  \item \emph{In-book solvable.} Every puzzle can be solved using materials in this volume (no outside lore required).
  \item \emph{Cross-checkable.} Each has an unambiguous solution in the sealed appendix.
  \item \emph{Diegetic cues.} All transformations (rotations, counts, overlays) are motivated by in-world practices—toll stamps, chant cadence, survey drift, etc.
  \item \emph{Accessibility.} No puzzle relies solely on color; shapes, symbols, or counts provide redundant signals.
\end{itemize}

\medskip
\noindent\textbf{How to use hints.} A four-rung ladder accompanies each puzzle:
\begin{enumerate}\setlength\itemsep{0.25em}
  \item \textbf{Nudge 1} — a gentle reframing (no new facts).
  \item \textbf{Nudge 2} — points at the relevant artifact features.
  \item \textbf{Nudge 3} — states the key operation (e.g., “rotate 17°”).
  \item \textbf{Reveal} — the full solution (in the sealed appendix).
\end{enumerate}
Hint codes appear as \textsc{[H1] [H2] [H3] [R]} with page references.

\medskip
\noindent\textbf{Scholar’s Path.} A suggested route threads the puzzles in a rising curve of difficulty, pairing each with a scenario where its discovery \emph{feels} best at the table. You can follow the path front-to-back, pick selectively, or read everything as lore.

\medskip
\noindent\textbf{Encodings you will meet.}
\begin{itemize}\setlength\itemsep{0.3em}
  \item \textbf{Thepyrgosi Cadence} — public events (Cross ends, Rooted Blues, specials) produce a five-bead rhythm; cycles map to base–5 digits (\S\ref{scen:candle-count}). The digit→letter table lives in the appendix.
  \item \textbf{Map Overlays} — city grids pin-holed to rotate to \(\,17^\circ\) (sanctums↔gates) and occasionally reflected against toll-river axes.
  \item \textbf{Ledger Substitutions} — tally marks, seal rosettes, and triangle stamps encode positions via simple substitution; every symbol is introduced before it’s required.
  \item \textbf{Veils \& Blanks} — deliberate omissions (first-line left empty, unnamed doctrines) are part of the signal; treat silence as a count.
\end{itemize}

\medskip
\noindent\textbf{Solo \& group use.}
\begin{itemize}\setlength\itemsep{0.3em}
  \item \emph{Solo dossier.} Work puzzles between scenarios; check \textsc{[H1]} only after a timed attempt (10–15 minutes), then proceed rung by rung as needed.
  \item \emph{Table assist.} Use only the \em
 
 
 \subsection*{Puzzle 1 — The Candle Ledger (Primer in Quinary)}
\label{pz:candle-ledger}
\phantomsection
\addcontentsline{toc}{subsection}{Puzzle 1 — The Candle Ledger (Primer in Quinary)}

\noindent\textbf{What you need.} The five–bead track from \S\ref{scen:candle-count} and this page.

\medskip
\noindent\textbf{Premise.} In \emph{Candle Count}, each \emph{chant complete} consists of \textbf{five triggers}. For this primer, \textbf{encode} each trigger as a single base–5 digit using the etched weights on the brass strip:

\begin{center}
\begin{tabular}{rl}
\([{\small ✚}\ \text{Cross end}]\) & \(\rightarrow\) digit \(\mathbf{2}\) \\
\([{\small \raisebox{0.15em}{\(\,\bot\,\)}}\ \text{Blue becomes Rooted}]\) & \(\rightarrow\) digit \(\mathbf{1}\) \\
\([{\small S}\ \text{Special used}]\) & \(\rightarrow\) digit \(\mathbf{3}\) \\
\end{tabular}
\end{center}

Write the \textbf{five digits in order} as the events occur during a chant. That 5-digit base–5 “\emph{chant code}” later maps to a letter via the table in the \emph{Sealed Appendix}.

\medskip
\noindent\textbf{Goal.} Convert the three logged chants below into their \textbf{5-digit base–5 codes}. Leave the letter blanks empty for now; you will translate them when you reach the appendix.

\medskip
\noindent\textbf{Logged chants (from a clerk’s slate).}
\begin{enumerate}\setlength\itemsep{0.25em}
  \item \([{\small ✚}]\;[{\small \(\,\bot\,\)}]\;[{\small S}]\;[{\small ✚}]\;[{\small ✚}]\) \hfill code: \(\_\ \_\ \_\ \_\ \_\) \hfill letter: \(\_\)
  \item \([{\small S}]\;[{\small ✚}]\;[{\small \(\,\bot\,\)}]\;[{\small S}]\;[{\small \(\,\bot\,\)}]\) \hfill code: \(\_\ \_\ \_\ \_\ \_\) \hfill letter: \(\_\)
  \item \([{\small ✚}]\;[{\small ✚}]\;[{\small S}]\;[{\small \(\,\bot\,\)}]\;[{\small S}]\) \hfill code: \(\_\ \_\ \_\ \_\ \_\) \hfill letter: \(\_\)
\end{enumerate}

\medskip
\noindent\textbf{Worked micro–example (not part of the three).}  
Sequence: \([{\small ✚}]\;[{\small S}]\;[{\small \(\,\bot\,\)}]\;[{\small ✚}]\;[{\small S}]\)  
Digits: \(2\ 3\ 1\ 2\ 3\) \(\Rightarrow\) chant code \(23123_{(5)}\) \(\Rightarrow\) \emph{translate using the appendix’s chant\(\to\)letter table.}

\medskip
\noindent\textbf{Logging grid.}
\begin{center}
\begin{tabular}{c|c|c}
\textbf{Line} & \textbf{5-digit chant code (base–5)} & \textbf{Letter (appendix)} \\
\hline
A & \hspace{3.5cm} & \hspace{1.2cm} \\
B & \hspace{3.5cm} & \hspace{1.2cm} \\
C & \hspace{3.5cm} & \hspace{1.2cm} \\
\end{tabular}
\end{center}

\medskip
\noindent\textbf{Fairness.} Every symbol used above appears earlier in Part II. No external knowledge is required. The digit\(\to\)letter mapping is published in the \emph{Sealed Appendix} to preserve discovery order.

\medskip
\noindent\textbf{Hint ladder.} (Reveal codes only as needed.)
\begin{itemize}\setlength\itemsep{0.25em}
  \item \textsc{[H1]} Count \emph{in order}. Each chant is exactly five events; write five digits left-to-right as they occur.
  \item \textsc{[H2]} Use the brass–strip weights: Cross = 2, Rooted = 1, Special = 3. No other values appear in this primer.
  \item \textsc{[H3]} Keep the base: treat the five digits as a base–5 code, not five separate numbers in base–10. The translation to letters lives in Appendix A.
  \item \textsc{[R]} Compare your three codes to the chant\(\to\)letter table in the \emph{Sealed Appendix}. The three letters form a word you would expect to meet near water.
\end{itemize}


  
\end{document}


