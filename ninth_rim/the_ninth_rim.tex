\documentclass[11pt]{article}

% --- Core packages & order ---
% math text & symbols
\usepackage{amsmath,amssymb}
% define \texorpdfstring (load late is safest)
\usepackage[hidelinks]{hyperref}
\usepackage[T1]{fontenc}
\usepackage{lmodern}
\usepackage[margin=1in]{geometry}
\usepackage{microtype}
\usepackage{xcolor}
\usepackage{graphicx}          % for \resizebox
\graphicspath{{figures/}{../figures/}} % covers root and build-as-CWD cases
\usepackage{booktabs}
\usepackage{tabularx,makecell,array}
\usepackage{multicol}
\usepackage{enumitem}
\usepackage{parskip}
\usepackage{ellipsis} % improves spacing around \dots/\ldots automatically
\usepackage{tabularx}
\usepackage{tikz}              % load TikZ first...
\usetikzlibrary{arrows.meta,calc,positioning}  % ...then libraries
\usepackage[colorlinks]{hyperref}   % load before cleveref
\usepackage[nameinlink,capitalize]{cleveref}
\crefname{theorem}{Theorem}{Theorems}
\crefname{lemma}{Lemma}{Lemmas}
\crefname{proposition}{Proposition}{Propositions}
\crefname{corollary}{Corollary}{Corollaries}
\crefname{definition}{Definition}{Definitions}
\crefname{example}{Example}{Examples}
\usepackage{subcaption}
\usepackage{needspace}
\usepackage{float}
\usepackage{placeins}
% Core math & theorem setup (in preamble)
\usepackage{amsmath,amssymb,amsthm}

\numberwithin{equation}{section} % optional

\theoremstyle{plain} % italic body, bold heading
\newtheorem{theorem}{Theorem}[section]
\newtheorem{lemma}[theorem]{Lemma}
\newtheorem{proposition}[theorem]{Proposition}
\newtheorem{corollary}[theorem]{Corollary}

\theoremstyle{definition} % upright body
\newtheorem{definition}[theorem]{Definition}
\newtheorem{example}[theorem]{Example}
\newtheorem{assumption}[theorem]{Assumption}

\theoremstyle{remark} % upright, small heading
\newtheorem*{remark}{Remark}
\newtheorem*{note}{Note}

% Optional: a black square QED
\renewcommand{\qedsymbol}{$\blacksquare$}
% --- Color palette ---
\definecolor{royal}{RGB}{12,64,145}
\definecolor{sanctum}{RGB}{0,120,80}
\definecolor{ink}{RGB}{30,30,30}
\definecolor{ccol}{RGB}{55,110,180}
\definecolor{scol}{RGB}{120,60,160}
\definecolor{rocol}{RGB}{220,130,40}
\definecolor{rccol}{RGB}{190,40,50}
\definecolor{capcol}{RGB}{60,140,80}

% --- tcolorbox (load once, after colors) ---
\usepackage[most]{tcolorbox}

% --- Small UI helpers (now safe because TikZ is loaded) ---
\newcommand{\CrownIcon}{%
  \tikz[baseline=-0.5ex,scale=0.14]{
    \fill[orange!70!yellow] (0,0) -- (0.65,0.9) -- (1.3,0) -- (1.95,0.9) -- (2.6,0) -- cycle;
    \fill[orange!80!brown]  (0,-0.36) rectangle (2.6,0);
  }%
}

% --- Optional badge & variant box (uses tcolorbox) ---
% Pill that pops on a dark title bar
\newtcbox{\optbadgeOnDark}{on line, arc=4pt, boxrule=0pt,
  colback=orange!70!yellow, colframe=white,
  left=4pt, right=4pt, top=1pt, bottom=1pt, boxsep=1.5pt,
  tcbox raise base, fontupper=\scriptsize\bfseries\color{black}}
  

\newtcolorbox{rulevariant}[1][]{%
  enhanced, breakable,
  colback=royal!3, colframe=royal!70!black, boxrule=0.8pt,
  rounded corners, left=7pt, right=7pt, top=7pt, bottom=7pt,
  coltitle=white,
  attach boxed title to top left={yshift=-2mm, xshift=3mm},
  boxed title style={colback=royal!85!black, colframe=royal!85!black},
  fonttitle=\bfseries,
  title={\CrownIcon\;Crown Buyback\;\optbadgeOnDark{OPTIONAL}},
  #1}

% --- Table/layout tweaks ---
\setlength{\columnsep}{0.75em}
\setcounter{tocdepth}{2}
\setcounter{secnumdepth}{2}
\newcolumntype{Y}{>{\raggedright\arraybackslash}X}
\setlength{\tabcolsep}{6pt}
\renewcommand{\arraystretch}{1.15}
\setlength{\parindent}{0pt}

% --- Status badges (TEXT mode; do not put them in \( ... \)) ---
\newcommand{\CC}[1]{\textcolor{blue!60!black}{\scriptsize\ttfamily[CF:#1]}}
\newcommand{\SC}[1]{\textcolor{red!60!black}{\scriptsize\ttfamily[S:#1]}}
\newcommand{\RoC}{\textcolor{teal!60!black}{\scriptsize\ttfamily[Rooted]}}
\newcommand{\RC}{\textcolor{purple!70!black}{\scriptsize\ttfamily[RC]}}
\newcommand{\CapC}[1]{\textcolor{green!40!black}{\scriptsize\ttfamily[G:#1]}}

% --- Thought helper ---
\newcommand{\think}[1]{\emph{\footnotesize #1}}

% =========================
% Board config + TikZ styles
% =========================
\newcommand{\BoardN}{8}               % board size (NxN)
\newcommand{\SanctumA}{1/4}           % left side-apex Sanctum
\newcommand{\SanctumB}{8/5}           % right side-apex Sanctum

\tikzset{
  sq/.style={draw, line width=0.3pt},
  cross/.style={fill=blue!22},      % center Cross (2x2)
  apexA/.style={fill=green!18},     % top apex square
  apexB/.style={fill=green!26},     % bottom apex square
  sanctumA/.style={fill=red!28},    % left Sanctum
  sanctumB/.style={fill=red!34},    % right Sanctum
  zoc/.style={fill=black!18},
  moveArrow/.style={-Latex, line width=0.8pt},
  piece/.style={circle, draw, fill=white, line width=0.8pt, minimum size=7.5pt, inner sep=0pt}
}

% Grid + shading (top-left origin: physical y = BoardN - y)
\newcommand{\DrawGrid}{%
  \foreach \x in {1,...,\BoardN}{%
    \foreach \y in {1,...,\BoardN}{%
      \pgfmathtruncatemacro{\yphys}{\BoardN-\y}%
      \pgfmathtruncatemacro{\shade}{mod(\x+\y,2)==0 ? 3 : 0}%
      \fill[black!\shade] ({\x-1},{\yphys}) rectangle ++(1,1);
      \draw[sq] ({\x-1},{\yphys}) rectangle ++(1,1);
    }%
  }%
}
\newcommand{\ShadeSquares}[2]{% #1=style, #2="x/y,x/y,..."
  \foreach \x/\y in {#2}{%
    \pgfmathtruncatemacro{\yphys}{\BoardN-\y}%
    \path[#1] ({\x-1},{\yphys}) rectangle ++(1,1);
  }%
}
\newcommand{\ShadeCross}[1]{%
  \pgfmathtruncatemacro{\L}{\BoardN/2}%
  \pgfmathtruncatemacro{\H}{\L+1}%
  \foreach \x/\y in {\L/\L,\L/\H,\H/\L,\H/\H}{%
    \pgfmathtruncatemacro{\yphys}{\BoardN-\y}%
    \path[#1] ({\x-1},{\yphys}) rectangle ++(1,1);
  }%
}
\newcommand{\RedrawGridLines}{%
  \foreach \x in {1,...,\BoardN}{%
    \foreach \y in {1,...,\BoardN}{%
      \pgfmathtruncatemacro{\yy}{\BoardN-\y}%
      \draw[sq] ({\x-1},{\yy}) rectangle ++(1,1);
    }%
  }%
}

% --- Preamble (once) ---
\usepackage{tikz}
\newcommand{\CrackedRimIcon}{%
  \tikz[baseline=-0.6ex,line width=0.5pt]{
    \draw (0,0) circle (1.6ex);
    \draw (0,0) circle (0.9ex);
    \draw[line cap=round] (1.25ex,0.72ex) -- (0.35ex,0.08ex) -- (-0.55ex,-0.45ex);
  }%
}

% Cardinal labels used in your minis (adjust as you prefer)
\newcommand{\LabelN}{OR}
\newcommand{\LabelE}{OL}
\newcommand{\LabelS}{HR}
\newcommand{\LabelW}{HL}

% Piece placers (safe now that TikZ is loaded)
\newcommand{\PlaceA}[3]{\pgfmathtruncatemacro{\yphys}{\BoardN-#3}\node[piece,fill=white,text=black] at ({#2-0.5},{\yphys+0.5}) {\scriptsize\bfseries #1};}
\newcommand{\PlaceB}[3]{\pgfmathtruncatemacro{\yphys}{\BoardN-#3}\node[piece,fill=black,text=white] at ({#2-0.5},{\yphys+0.5}) {\scriptsize\bfseries #1};}

% ==== Directional move notation (robust) ====
% Usage: \On[OL]{2}, \On[OR]{3}, \Hm[HL]{1}, \Hm[HR]{2}
\makeatletter
\newcommand{\KR@OnPretty}[1]{%
  \def\tmp{#1}\def\OL{OL}\def\OR{OR}%
  \if\relax\detokenize{#1}\relax\else
    \ifx\tmp\OL L\else\ifx\tmp\OR R\else #1\fi\fi
  \fi
}
\newcommand{\KR@HmPretty}[1]{%
  \def\tmp{#1}\def\HL{HL}\def\HR{HR}%
  \if\relax\detokenize{#1}\relax\else
    \ifx\tmp\HL L\else\ifx\tmp\HR R\else #1\fi\fi
  \fi
}
\newcommand{\KR@MoveCore}[3]{%
  \mbox{\textsc{#1}\if\relax\detokenize{#2}\relax\else$_{\mathrm{#2}}$\fi\,\textbf{#3}}%
}
% Force (re)define even if a 1-arg legacy \On/\Hm exists
\DeclareRobustCommand{\On}[2][]{\KR@MoveCore{on}{\KR@OnPretty{#1}}{#2}}
\DeclareRobustCommand{\Hm}[2][]{\KR@MoveCore{hm}{\KR@HmPretty{#1}}{#2}}
\makeatother

% === Faction blurb block (name + one-line desc + optional tagline) ===
% Usage:
%   \factionblurb{Ykrul (Kon'reh)}{Control-first pragmatists...}{“Count exits, not victims.” — Kargath}
%   \factionblurb{Ecktorian}{Engineers of symmetry...}{}   % <- no tagline line
\newcommand{\factionblurb}[3]{%
  \par\medskip
  \Needspace{3\baselineskip}% keep label+desc+tagline together when possible
  \noindent\textbf{#1. }#2\par
  \smallskip
  \if\relax\detokenize{#3}\relax\else\emph{#3}\par\fi
}

% ==== Simple Coach's Notes (no tables) ====
\newtcolorbox{coachbox}[1]{%
  enhanced, breakable,
  colback=royal!3, colframe=royal!70!black, boxrule=0.8pt,
  rounded corners, left=7pt, right=7pt, top=7pt, bottom=7pt,
  coltitle=white,
  attach boxed title to top left={yshift=-2mm, xshift=3mm},
  boxed title style={colback=royal!85!black, colframe=royal!85!black},
  fonttitle=\bfseries,
  title={#1}, before skip=6pt, after skip=6pt, width=\linewidth,
}

% Coach’s Notes environment: uses description list for Cue → Note
\newenvironment{coachnotes}[1]{%
  \begin{coachbox}{#1}%
    \footnotesize
    \begin{description}[leftmargin=2.8cm,labelsep=0.6em,font=\scshape,
                        itemsep=0.25em,parsep=0pt,topsep=0.2em]
}{%
    \end{description}%
  \end{coachbox}%
}

% Convenience macro for rows
\newcommand{\CNRow}[2]{\item[#1] #2}

% --- helpers (safe: only define if missing) ---
\providecommand{\playdesc}[1]{\par\smallskip\noindent\small\textbf{Play.} #1\par}
\newcommand{\flavline}[2]{\noindent\textbf{#1.} \textit{#2}\par}

% --- Engine-log friendly wrappers (reuse \On and \Hm) ---
% LANE: pass L or R (OL/OR/HL/HR still OK via your pretty-mapper)
\makeatletter

% convenience tags
\newcommand{\CFIn}[1]{\CC{in #1/3}}
\newcommand{\CFOut}{\CC{out}}

% allow \RC both bare and with a count: \RC  or  \RC[3/5]
\renewcommand{\RC}[1][]{%
  \textcolor{purple!70!black}{\scriptsize\ttfamily[RC%
  \if\relax\detokenize{#1}\relax\else~#1\fi]}}

% verb dispatchers -> your existing \On / \Hm
\expandafter\def\csname mvverb@on\endcsname#1#2{\On[#1]{#2}}
\expandafter\def\csname mvverb@hm\endcsname#1#2{\Hm[#1]{#2}}

% specials (no lane/step)
\expandafter\def\csname mvverb@H\endcsname#1#2{{\SC{H}}}      % Hop-capture tag
\expandafter\def\csname mvverb@D\endcsname#1#2{{\SC{D}}}      % Displacement tag
\expandafter\def\csname mvverb@seed\endcsname#1#2{{\textsc{seed}}}

% directional move: \mv[Side]{Piece}{verb}{Lane}{Steps}{tail}
\DeclareRobustCommand{\mv}[6][]{%
  \if\relax\detokenize{#1}\relax\else\textbf{#1:}\ \fi
  \textbf{#2}\ %
  \csname mvverb@#3\endcsname{#4}{#5}%
  \if\relax\detokenize{#6}\relax\else\ #6\fi
}

% special-only (no lane/steps): \mvs[Side]{Piece}{Special}{tail}
\DeclareRobustCommand{\mvs}[4][]{%
  \if\relax\detokenize{#1}\relax\else\textbf{#1:}\ \fi
  \textbf{#2}\ %
  \csname mvverb@#3\endcsname{}{ }%
  \if\relax\detokenize{#4}\relax\else\ #4\fi
}
\makeatother
\newcolumntype{L}{>{\raggedright\arraybackslash\hspace{0pt}}X}


\definecolor{muted}{HTML}{555555}

% A compact, reusable table and row macro:
\newenvironment{RosettaTable}[1]{%
  \subsection*{#1}%
  \addcontentsline{toc}{subsection}{#1}%
  \noindent\begin{tabularx}{\linewidth}{@{}p{0.25\linewidth}X@{}}%
  \toprule
  \textbf{Term (lore-facing)} & \textbf{At the table (exact meaning / mechanic)} \\
  \midrule
}{%
  \bottomrule
  \end{tabularx}
}
\newcommand{\rosrow}[2]{\textbf{#1} & #2 \\[0.40em]}

\begin{document}\color{ink}

%==============================
% Title Page
%==============================
\begin{titlepage}
  \thispagestyle{empty}
  \begingroup
  \centering
  \vspace{1.5cm}

  {\color{red}\fontsize{36}{40}\selectfont\bfseries KON'REH}\par
  \vspace{6pt}
  {\Large\bfseries The Ninth Rim \par
  {\large Scenarios & Dossier}\par

  \vspace{14pt}
  \rule{0.62\linewidth}{0.6pt}\par
  \vspace{6pt}
  {\normalsize \textit{Another Expansion to a Game of Apex, Sanctum, and Reforge}}\par
  \vspace{6pt}
  \rule{0.62\linewidth}{0.6pt}\par

  \vspace{18pt}
  {\Large \textit{by} Nicholas A.\ Gasper}\par
  {\normalsize Setting \& Lore by Nicholas A.\ Gasper}\par

  % Optional crest/emblem (uncomment and provide asset)
  % \vspace{12pt}
% Crest (PNG). Compiles even if the file is missing.
\vspace{10pt}
% Crest (PNG) with fallbacks in fig/
\includegraphics[
  width=.28\linewidth,
  keepaspectratio,
  trim={8pt 10pt 8pt 6pt}, % L B R T
  clip
]{figures/ninth-crest}

  \vfill

  % Optional epigraph — comment out if you prefer a blank footer
% --- Use anywhere ---
{\small\itshape
  \CrackedRimIcon\quad “Count eight. Leave the ninth to open.” \hfill — winter leaf from Dhahara
}\par

  \vspace{8pt}
  {\footnotesize
    University ordance Edition \textbullet\ \today
    % \quad|\quad Draft v0.9
  }\par

  \vspace{1cm}
  \endgroup
\end{titlepage}

\pagenumbering{roman}
\tableofcontents
\clearpage

%==============================
% Copyright / Legal Page
%==============================
\clearpage
\thispagestyle{empty}
\vspace{2cm}

{\small
\noindent \textbf{KON'REH} and associated setting terms including but not limited to:
\textit{Canray}, \textit{K’thra}, \textit{Kanry}, \textit{Twin Apex Seed}, \textit{Reforge},
the names of cultures (e.g., Ykrul, Ecktorian, Vhasian, Viterran, Aeler, Vilikari, Thepyrgosi (Thepyric), Ubral, Silkstrand),
proper nouns, places, characters, flavor quotes, worldbuilding lore, diagrams, iconography,
and the specific textual expression of rules, examples, and notation in this book are
© \the\year\ \textit{Nicholas A. Gasper}. All rights reserved.\\[8pt]

\noindent \textbf{Mechanics Disclaimer.}
The underlying game mechanics, procedures of play, and functional systems described herein are
not claimed as proprietary subject matter. No copyright is asserted in the \emph{ideas} of movement rates,
zones of control, countdowns, or other rules mechanics \emph{as mechanics}; copyright subsists in the
\emph{expression} of those ideas in this book (text, arrangement, examples, graphics, naming, and lore).\\[8pt]

\noindent \textbf{Trademarks.}
KON'REH and other marks herein may be trademarks or registered trademarks of their respective owners.
Use of the marks does not imply endorsement.\\[8pt]

\noindent \textbf{Fan Content Policy (Non-Commercial).}
You may reference these rules in reviews, tutorials, and fan aids, and you may create non-commercial
scenarios and player aids that include brief excerpts, provided you (i) credit
\textit{“Kon’reh © \the\year\ Nicholas A. Gasper”}, (ii) do not reproduce large portions of this book verbatim,
and (iii) do not imply official status. For commercial use, please contact the publisher.\\[8pt]

\noindent \textbf{All Rights Reserved.}
Except as permitted above or by applicable law, no portion of this publication may be reproduced,
stored in a retrieval system, or transmitted in any form or by any means without prior written permission
of the publisher.\\[8pt]

\noindent \textbf{Credits.}
Design \& Development: Nicholas A. Gasper \\
Editing: \textit{PLACEHOLDER} \\
Playtesting: \textit{PLACEHOLDER} \\[8pt]

\noindent \textbf{Publisher.}
PLACEHOLDER \\
\textit{ISBN:} (TBD)
}

\clearpage
\section*{Preface — A Letter to Alayse}
\label{sec:preface-markus-to-alayse}
\phantomsection
\addcontentsline{toc}{section}{Preface — A Letter to Alayse}

\noindent\textit{Alayse, my heart,}

The road down to \textit{Thepyrgos} is a long, slanting breath across wet country. I came by the river stages first, then by a cart whose canvas remembered every rain it had ever known. Fields wear the color of old copper; the ditches keep their small, stubborn mirrors; along the last rise the wind busied itself with barley while the bells spoke over one another, each to its own hour. It is a city that greets you with \emph{sound} before it shows its face.

At our gate my father pressed his travel copy of the \emph{Concordance} into my hands and said only, “Our \emph{fourth legacy} is yours now.” He smiled—as at a family tradition since the Imperial era days when our name was Valvano. We parted on that smile. It has stood between me and poor courage ever since, which is to say it has helped.

\medskip
The University keeps two faces. By day: polished and well–kept—the bright stalls with their brass lamps and measured shadows, the refectory with steam and bread, courts paved as smooth as thought. And then the other: glimpsed when a porter’s key turns or a lamp gutters—low vaults, patient stairwells, rooms that have learned to keep their breath, shelves that seem to turn themselves if you leave and come back too quickly. Not menacing, industrious; but the stone here keeps \emph{accounts}.

I have met my tutor. He is called \textit{Aqyl, son of Aqyl}.

You will laugh to know I faltered at the name. I have read \emph{Aqyl’s} lectures to tatters—the elder, the measured hand that teaches one to count aloud and copy plainer later—and here stands a man wearing the same name as easily as a linen shirt. He has a Thepyrgosi cast to him, though the eyes are almond and untired; he is thin and wiry, the sort of frame the wind respects, with salt–and–pepper curls that refuse discipline and an age that will not hold still under a guess. I asked—too bluntly—whether he is \emph{of} the elder. He shrugged a little: “I am named for my father, and he for his,” and let the matter lie where it fell, like a coin on a ledger line that refuses to roll. He is patient beyond reason and exact without cruelty; he taps a margin once and the fault knows what it is.

He says a clear gloss is a moral act, and that cleverness comes after honesty if it comes at all.

\medskip
My first days have been proper ink: inventories, tallies, the straight work of hands that want to be useful. Yet there are corners where the edges begin to speak to the middle. In the stalls the rain comes down like soft chain and the pages answer; a clerk’s circle in an old ledger refuses to be mere ornament; a map set beside a list of tolls begins, inexplicably, to \emph{agree} with it. Small things—nothing for a magistrate—yet I write them to you because you know the difference between superstition and a hunch you can test in daylight.

Do not be anxious for me. The bright rooms are many, and I keep to them. I eat something warm when the bells tell me to, and when the ink runs fast I put it away and count aloud, as my father taught. If I come upon any page that feels \emph{hot}, I lay a sheet over it and copy later with a cooler hand. “You are not required to be clever,” \textit{Aqyl} tells me. “You are required to be honest.” I am doing my best at the latter and waiting on the former.

\medskip
When you come (and you must), we will walk the outer courts as we used to walk the lanes, naming buildings, counting bells. The lamps mind themselves here; the hours keep their own borders. I think you will like how the light stands up straight in the winter wind and how the old stones approve of ordinary work.

Until then, know that I am safe, that I am listening, and that I write your name at the head of my day before I balance any column.

\medskip
\noindent\textit{Your faithful,}\\
M.

\begin{flushright}
\textit{Thepyrgos, Amber Reckoning: Leafturn, 876 AR}
\end{flushright}

\medskip
\noindent\textit{Dedication.} For His Grace Duke Braedon Fenwood III, who taught me to count aloud and to write what I can prove; for \textit{Aqyl} (the elder), who showed me that a clear gloss is a kind of courage; and for the players, who will decide which of these rumors hold.

\pagenumbering{arabic}
\clearpage

\section{Introduction}
\label{sec:introduction}
\phantomsection

\subsection{Purpose and Scope}
\textit{Kon'reh: The Ninth Rim} is a dossier–plus–scenarios volume, meant to be read as narrative and used at the table without changing tournament play. It presents historical fragments, ledgers, and field notes in facsimile, followed by seven optional, modular overlays (\emph{Rites}) that can be toggled per session.

\subsection{How to Use This Book}
Read \emph{Part I} for context and atmosphere; use \emph{Part II} to run one-shots or to link all seven scenarios into a short campaign. Each Rite is self-contained, clearly labeled, and defaults to \textsc{off}. Turning a Rite \textsc{on} changes the \emph{feel} of play (information, pacing, geography) but never the core rules.

\subsection{Title Gloss}
“The \emph{Ninth}” names the final Rite; the \emph{Rim} is the perimeter band the Ninth briefly permits you to cross.

\subsection{Structure of the Volume}
\begin{itemize}
  \item \textbf{Part I — A Most Curious Game (Lore in Facsimile):} curated documents, marginalia, and maps that frame the investigation.
  \item \textbf{Part II — Rites \& Scenarios:} seven independent modules (\emph{Salt Stitch}, \emph{Witness at the Ford}, \emph{Veil of Names}, \emph{Candle Count}, \emph{Copper \& Salt}, \emph{Ash–Fenn Rite}, \emph{The Ninth}) with setup, procedure, examples, and a “Why It’s Safe” note.
  \item \textbf{Part III — Puzzles \& Scholar’s Path:} hint ladders and a sealed appendix for readers who wish to decode embedded ciphers.
  \item \textbf{Part IV — The Sacred Geometry (Theory Sidebars):} brief, non-binding notes on rings, survey drift, cadence~$\to$~digits, and related curiosities.
  \item \textbf{Part V — Appendices \& Sealed Solutions:} compatibility matrix, printable props, icon legend, encoded tables, solutions, and essays.
\end{itemize}

\subsection{Editorial Notes and Conventions}
Voice labels (e.g., \emph{Markus}, \emph{Aqyl}, \emph{Clerk}) appear in margins to indicate provenance. Dates follow city-record style. Facsimiles are lightly normalized for legibility; conjectural restorations are bracketed. Diagram callouts and iconography are standardized for quick reference.

\subsection{Continuity}
This volume preserves the game’s tournament core. Scenario language is intentionally conservative and has been reviewed for rules safety. Historical attributions reflect best-available sources; where disputes exist, they are marked as such rather than harmonized.

\medskip
\noindent\textit{Read what you like, use what you will, and leave the rest in the archives.}

\clearpage
\addcontentsline{toc}{section}{Part I: A Most Curious Game }

\section{A Most Curious Game}
\label{sec:curious-game}
\phantomsection

\subsection{Fragment L--1 (648 AR) — On the Lineage of Names and the Measure of Days}
\label{frag:l1}
\phantomsection

\noindent\textit{Copied from a student booklet in a small slanted hand; iron–gall ink, edges smoke–kissed. A later owner has ruled a faint line for marginal glosses.}

\medskip
\noindent\textbf{Lexicus of Thepyrgos, Junior Reader}\\
\textit{Amber Reckoning: Planting, Tenday II, Day 4 (648 AR)}

Herewith I set down, so that I may not forget, certain observations taken in the reading–stalls and counting–rooms. This morning I was shown a parcel of leaves attributed to \textit{Aqyl of Thepyrgos}, surnamed in the index \textit{the elder}. They treat of cadence, of clarity, and of the ethics proper to a gloss. Their sentences are so plain and right–fitted to their task that I confess I blushed to see my own drafts beside them.

Being unwilling to labor under a confusion of persons, I asked my mentor, \textit{Master Aqyl}, whether he bore his name from that same hand whose notes I had been admiring. He answered with economy, not unkind: “I am named for my father, and he for his, and he for his; scholars mostly, when we were not frightened. This first Aqyl is of our house—so they say—though it is better to prove a line by its books than by its boasts.” The manner of his shrug conserved words while granting permission to proceed.

From the elder’s counsel I adopt at once the habit of counting aloud when the mind runs ahead; it steadies both temper and column when the rain takes the lamps and the bindings smell of wet leather. I will report what profit it yields.

\medskip
\noindent\textit{Commonplaces for the stalls (set down for my correction):}
\begin{enumerate}\setlength\itemsep{0.25em}
  \item Count aloud when the mind runs ahead of the ink; let breath govern hand.
  \item Mark conjecture as conjecture; do not pass it as record, even briefly.
  \item Where two sums disagree, copy both; resolve the quarrel; strike the error once and legibly.
  \item Keep the hand even at the edges; the eye trusts a margin that does not wave.
  \item If a sentence can be made plainer without loss, make it so and be grateful.
\end{enumerate}

If Providence and the bursar permit, I shall look into the older cupboards for further traces of the elder’s order: not the noise of his name, but the measure of his days.

\medskip
\begin{quote}\small
\textit{Present marginal hand (876 AR), M.\ Fenwood:} Asked \textit{Aqyl, son of Aqyl}, the same question this week. Received the same shrug and the same coin of a smile. Either the lineage is faithful—or the answer is what passes down with the name. In either case, tidy.
\end{quote}

\subsection{Fragment L--2 (648 AR / 876 AR) — The Salt–Pressed Leaves}
\label{frag:l2}
\phantomsection

\noindent\textit{Provenance.} Two hands, bound together by later custodians: the \emph{later} hand (876 AR) is a journal leaf by M.\ Fenwood describing a walk through the University; the \emph{earlier} hand (648 AR) is a student entry by Lexicus of Thepyrgos. Both were copied into the present volume from their respective sources; the salt–ring stain appears on the Lexicus tract and is noted by the later hand.

\medskip
\noindent\textbf{Present hand (876 AR), M.\ Fenwood}

\noindent\textit{Amber Reckoning: Leafturn, Tenday II, Day 3}

Aqyl, son of Aqyl, led me by the long way, down the serpentine stacks where the floor lists like a tired ship and the lamps run together into a narrow river of honey–light. His face was all edges and patience, the sort of calm that has outlived weather. We passed a door stenciled in the University’s square hand: \textsc{condemned \textemdash{} closed by order of the wardens}. Aqyl produced a key that was not quite a key—a sliver of dark metal, bone–thin, that seemed to remember other locks—and the notice became only paper.

Inside: chalk on the tongue. The air held the dry asperity of old slate. Cobwebs draped the ceiling like fatigued banners, and the tables wore a fur of dust that remembered every elbow that had ever leaned there. “A private study,” Aqyl said, as if declaring a weather report. “Copies and companions to the common shelves. Lexicus left his own here. It is a forbidden place, though no one will name it so. The University has many such unnames.”

I asked whether we should be there. He shrugged the way cliffs do. “You asked for the edges. Edges do not keep catalog hours.” He set a travel lantern upon a desk; its little flame took hold of the room with a steady, workman’s glow. Softer, almost to himself: “Read heresies if you must. Do not let them lead you.”

A spine caught at me—a ledger–brown volume, the label torn into a crescent. The fore–edge wore a hurried marginal hand that could have been Lexicus after a night without sleep. I opened where the book wanted. There, in the middle margin, a pale ring the color of bone had bled into the paper, as if a damp band had rested there and then been lifted: a halo pressed flat. Salt, I thought, and did not know why. The ring held a faint scatter of grit, and when I breathed upon it, the motes rose like ash and settled again in the shape of a circle, exact as a coin.

Something—only the air, surely—moved past my shoulder with the smallest suggestion of feathers. I turned (of course I turned) and saw only two chairs and their twin long shadows, meeting at the floor like folded wings. The lantern burned on, indifferent. Aqyl had wandered to a far shelf and was reading spines with that almond–lidded, not–young, not–old look of his, as if time had to ask to count him.

“Edges,” he said without looking, “will speak like thresholds if you let them. Keep your sums. Keep your breath.” Then, quieter: “If something answers, mark what it \emph{took} to answer.”

The page under my hand bore a penciled note in a small, precise script: \textit{ford held}. Beside it, three tiny triangles where tallies should have been. The dust along the table had drifted into a narrow crescent, and I could not tell whether my sleeve had done that, or some older, larger shape remembering how to fold itself.

I copied the note, and the ring, and the triangles, and left a line unfilled at the top of the page, as Father taught me, for whatever will not be named yet.

\medskip

\noindent\textbf{Earlier hand (648 AR), Lexicus of Thepyrgos, Junior Reader}

\noindent\textit{Amber Reckoning: Planting, Tenday II, Day 6 (648 AR)}

Master Aqyl this day entrusted to me the study he keeps behind the north stalls, being tired (so he says) of carrying the key. He maintains that the first of his name bequeathed it to his house, yet would have it now employed by any who will keep a neat margin. I am resolved to be that man.

On a middle shelf I found a tract whose outer leaves show a curious stain, as though pressed with damp salt. The title is rubbed, but within there is set down a custom of the Cartwright before any game: he traces a light band of salt about the board’s perimeter and speaks nothing until his first piece is moved. The writer names this the \textit{Salt Stitch}, and cites an earlier fragment styled the \textit{Salt–Pressed Leaves} (whatever that may mean). The habit is treated as a warding: \emph{not a wall, but a price}.

I cannot yet say whether this is superstition or discipline. The Cartwright is elsewhere a practical man. Yet I confess the figure pleases the mind. To bind the outermost ring and thus make the middle honest—do we not do as much in our books when we rule a margin?

\medskip
\noindent\textit{Present marginal hand (876 AR), M.\ Fenwood:} Same tract, same page; the ring–mark survives. I asked \textit{Aqyl, son of Aqyl} what to call it. He watched me read and said nothing until I closed the book. Then: “Name nothing yet. Learn its weight first.” (It is a good line; I am trying to like it less because he knows it.)

\subsection{Fragment L--3 (648 AR / 876 AR) — Ash–Fenn, Tolls, and Wagered Dreams}
\label{frag:l3}
\phantomsection

\noindent\textit{Provenance.} A toll–ledger leaf (rubbed, smoke–tinged) copied verbatim with notes by \textit{Lexicus of Thepyrgos} (648 AR); two \textit{later} journal leaves by M.\ Fenwood (876 AR) were bound after it by a University custodian.

\medskip
\noindent\textbf{Earlier hand (648 AR), Lexicus of Thepyrgos, Junior Reader}

In the north case I found a ledger of border dues, the hand plain as a bricklayer’s rule. One entry pricked my eye:

\begin{quote}\small
“Ash–Fenn caravan: from the east. Twenty–three beasts, seven wagons. Paid in full at Wadi Gate; stamped at the ford. Next gate: —” \textit{(blank)} “Reason for pass–through: —” \textit{(blank)}.
\end{quote}

Strange that a caravan so noted should vanish between gates; stranger that the clerk left two fields unfilled. I have marked three other leaves where the same name appears, always “from the east,” always tidy in its sums, and then silence—as though the road itself declined to witness.

I do not yet comprehend Master Aqyl’s interest in such mundane books. He maintains that the ordinary page is a faithful mirror if one looks long enough; I am young and wish for brighter glass.

Turning the leaf, a loose scrap fell from the binding—a toll–keeper’s own hand, crowded and eager, likely written after hours:

\begin{quote}\small
“They were cheerful, the Ash–Fenn lot, and asked if I would sit a friendly game of Canray while the beasts drank. We wagered dreams, which I thought a country jest. I am not a superstitious man. I lost and bid them on their way. Curious thing is, I dreamt poorly after—a month’s turn of it—and woke as if I had borrowed someone else’s night.”
\end{quote}

If this be a town humor, it is an odd one; yet the phrase “wagering dreams” reappears in a margin later in the book, in another hand. Perhaps it is a manner of speaking for small stakes without coin. I will ask Master Aqyl whether the phrase has a history, or whether the clerk merely wished to write himself interesting.

The ring–stamps are neat; the gaps are neater. I begin to suspect that the absence of a mark can carry as much account as the presence.

\medskip

\noindent\textbf{Present hand (876 AR), M.\ Fenwood}

I took the sliver of metal that is not quite a key and went to the door with the Wardens’ stencil. The notice was still there; paper does not mind being outranked. I chose the lock. The ward turned with a soft, professional disappointment, and the oak came free on a breath of chalk and old rain.

I crossed the threshold. No one asked me to. I wrote my name, small, on the blotting pad by the lantern to make the act answerable later. Then I shut the door behind me and let the room keep its own weather.

\medskip
\noindent\textbf{Present hand (876 AR), M.\ Fenwood}

I have gone back to the little study three days running. It is an alluring room—\emph{not} for comfort, which it refuses, but for the hush that gathers there like felt. I tell myself it is the quiet I am after. On the day I kept away (work, and a sore head), I slept oddly: too many doors, not enough rooms; a corridor that turned of its own accord and returned me to a threshold I had not crossed. It is nothing, I am sure. Still, I ledger it, as the toll–keeper ledgered his: not for drama, but to see if the numbers begin to lean.

Aqyl has not asked where I am reading; I have not told him. He warned me to read heresies without letting them read \emph{me}. I do not yet know the test, except this: stop when the ink begins to feel warm in the hand, as if the page were remembering a voice. (Last night it did; I set the book down and waited until the warmth went out of it like breath on a pane.)

The ash keeps turning up. Not a fall you can see, but a taste—chalk, a little iron—and a grit that will not be brushed. \textit{Ash–Fenn} recurs. Always “from the east.” Always counted, then—after some neat total—\emph{uncounted}. Accounts that conclude and then tilt, as if something leaned upon the last line and the clerk, to be prudent, looked away. If there is a trick in the books, it is a careful trick. If there is none, then it is only a habit of a careful man and I am lending it more weight than it earns. Either way, absences are a kind of ink.

There is also this, which I do not think I will tell Aqyl: when I read too long the lamp begins to sound like a drum in a room down the hall, and in the corner of my eye a shadow learns to be tall. I look and there is nothing—only the two chairs and their crossed shadows, meeting like folded wings. Perhaps my head is not yet right. Perhaps the University is very old and remembers how to be solemn.

Tomorrow I will copy the three blanks: the clerk’s three unprinted cells that fall where the sums should touch. I will copy them exactly, in their silence, and I will set them in a row to see if their emptiness makes a pattern—or if, turned, they prefer to be a circle.

\subsection{Fragment L--4 (648 AR / 876 AR) — The Burning at Ash–Fenn}
\label{frag:l4}
\phantomsection

\noindent\textit{Provenance.} An unsigned, trembling report (original leaf smoke–soft, brine–scented), copied in full by \textit{Lexicus of Thepyrgos} (648 AR); later marginal notes by M.\ Fenwood (876 AR).

\medskip
\noindent\textbf{Earlier hand (648 AR), Lexicus of Thepyrgos, Junior Reader}

In a pasteboard folder marked only by a tally, I found a narrow leaf, the ink run as if laid under breath. It purports to be an eye–witness account of the \textit{Burning at Ash–Fenn}, the writer’s name withheld. I copy the opening as it stands, preserving both ornament and hesitation:

\begin{quote}\small
“We came upon the caravan after the bell. The charge was heresy against the Light. I will not set down the tale of it, for the deeds done and the words spoken by the dying are a weight I will not lend to the world. I write only that ash fell like a fine rain, and that the road kept no prints though many passed.

As I turned to go, I heard voices speaking in a tongue I do not know, though I know many. They were not commands, only \emph{tellings}, as if a steady hand turned a ledger and asked me to read aloud. They said, ‘Bring it to your mentor, Aqyl.’ They said, ‘He will weigh it.’ I took fear then and fled for Thepyrgos with this account, meaning to place it in his hands and wash mine.”
\end{quote}

The next lines (ink paler; hand appears the same) read as if added on the following day:

\begin{quote}\small
“I found \textit{Aqyl} much aged and asked leave to lay the leaf before him. He called for his son—also \textit{Aqyl}—to sit and witness. The elder’s eye watered but did not wander. He said, ‘Read what you can, hide what you must.’ The younger set a lamp between us and did not speak.”
\end{quote}

I suspend judgment, noting only that my breath shortened while copying. The corridor outside the study was quiet; yet beneath the quiet there seemed a soft speech, like cloth moved in the next room. Twice I set down my pen and began again.

\medskip
\noindent\textit{Note to self.} If this be \emph{heresy}, it wears the dress of accounts and receipts. If it be only panic on a wet night, why does the ink fall heaviest where the writer says \emph{wash mine}? \emph{Cf.} toll leaves, Ash–Fenn entries; \emph{viz.} gaps that carry meaning.

\medskip
\noindent\textbf{Present hand (876 AR), M.\ Fenwood}

I am not proud to write that this piece spooked me. Two nights now with poor sleep. Nothing elaborate: only the thin sort of whispering that lives at the edge of hearing, as if a page were being turned in the next room and the reader very careful not to be seen. And (last night) a slow drum I could not place—one beat, then a long breath, then another—like a procession keeping time for a road that never delivers its pilgrims.

This morning I went back to the study and read the lines about the elder and the younger sitting together by the lamp. The picture steadies me: two faces sharing one circle of light, one hand turning the leaf, the other learning how to wait. If there is a lesson in that, it is not cleverness but patience.

Aqyl says a ledger is kept one entry at a time—numbers first, sums when they are due. I will keep mine that way and not make a story before it permits one. Today’s entry: the whispering is softer by daylight; the drum is not heard at all—\emph{save} that when I lay awake and counted my breath, last night’s beat matched it, patiently. I held my breath to shame my nerves and the beat withheld itself, just as patiently, until I breathed again.

The chairs kept their shadows, crossing like folded wings at the floor. There was also a gleam in the corner I had not noticed—a little crescent, no more than a sliver of damp where the dust refuses to settle. Perhaps only the lamp’s trick. Perhaps a ring that remembers a vessel. I touched the page and it felt cool, as if something had been lifted from it in the night and the paper had not yet warmed to the room.

\medskip
\noindent\textit{Present marginal hand, small.} \emph{If the road kept no prints, look to the fords. If the ash fell like rain, look to the wind.}
\subsection{Fragment L--5 (648 AR / 876 AR) — On Ionius and the Witness at the Ford}
\label{frag:l5}
\phantomsection

\noindent\textit{Provenance.} A translated note by \textit{Lexicus of Thepyrgos} (648 AR) from an older Thepyrgosi account; later, a journal leaf by M.\ Fenwood (876 AR). The source leaf shows rubrication on the name \emph{Ionius} and a faint waterline along the lower margin.

\medskip
\noindent\textbf{Earlier hand (648 AR), Lexicus of Thepyrgos, Junior Reader}

This day I translated a fragment from our city’s older tongue, being the recollection of a youth who attached himself to the \textit{Cartwright} on account of his instruction in ordered play (styled in the margin \textit{Canr\'e}). At the first the lessons are entirely proper: patience; the advantage of plain records; speaking only what one can justify by the state of the board.

As the company about the Cartwright increased, the account darkens. The youth writes of fast–day vigils; of circling the board’s rim with damp salt; of speaking as little as possible between first and fifth moves, as though silence itself were a counting. He calls these observances \textit{rites} and protests they were harmless while few.

Then follows a passage I set down in the driest manner and omit for decency’s sake all but its figure. The youth names a man \emph{Ionius} (the name is rubricated twice in the source) and says he was \emph{given} at a ford beyond the east gate, under lanterns and in the rain. This is described not as punishment but as \emph{analogy enacted}: \textit{Ionius, the witness at the ford}. Those present were instructed to watch the water take him and to keep silence until dawn. After, the Cartwright declared their crossings would henceforth be \emph{priced more honestly}.

Whether the youth’s pen makes theatre of a lesser cruelty I cannot say. Yet the recurrence of salt, silence, and ford troubles me. I am, to my shame, \emph{intrigued} by the claim that an emblem, if lived, might impose measure upon those who prefer cleverness to clarity. I record the temptation here, so that I may rebuke it later.

\medskip
\noindent\textit{Translator’s note (L.).} The idiom \textit{witness at the ford} appears again on two unrelated leaves with the sense of a boundary made costly, that crossings be not squandered. The scribe’s double–red on \emph{Ionius} is unusual in civic accounts.

\medskip
\noindent\textbf{Present hand (876 AR), M.\ Fenwood}

This is where I ought to have stopped for the week. I am unsettled by Lexicus’s cool pen in the face of what he copies, and more unsettled by his plain confession that he was \emph{intrigued}. I was, for a moment, the same. Then I shut the book with two hands, as if it were a window and the night were at it.

I set this down to bind myself: a rule, not a mood. I will not go back to that little room again. The shelves with their patient dust, the cobweb banners, the salt rings that print themselves anew if you breathe too near, the way the margins begin to hum when the lantern bends low—none of it is good for sleep or sums. There is a warmth that belongs to bread and hands; there is another that belongs to the inside of a bell. The leaves have the latter.

If anything worth keeping lies in those copies, I will fetch it in daylight, with witnesses, and with less romance than I have lately allowed. The University is large enough to house both scholarship and appetite; I will give mine back its leash.

Tomorrow I will take my notes in the public stalls and leave the key where it lies. I have brushed the dust from my sleeves and yet an ash–smell clings, like metal after thunder. Perhaps only the lamp. Perhaps the room remembering me. I choose to be dull and careful. Let whatever wants answering knock on a door with a porter.

\medskip
\noindent\textit{Present marginal hand (small).} The name \emph{Ionius} rang an old shelf–bell I cannot place. Perhaps only the house remembering its own books.

\subsection{Fragment L--6 (648 AR / 876 AR) — On Sacred Geometry and a Sleepless Study}
\label{frag:l6}
\phantomsection

\noindent\textit{Provenance.} A nocturnal journal leaf by M.\ Fenwood (876 AR), bound with an earlier notebook entry by \textit{Lexicus of Thepyrgos} (648 AR) concerning a stitched slip titled “Sacred Geometry.” The later leaf shows graphite pricking in groups of three and eight; the earlier carries stitch–holes at its head.

\medskip
\noindent\textbf{Present hand (876 AR), M.\ Fenwood}\\
\noindent\textit{Amber Reckoning: Leafturn, late, 876 AR}

I meant to keep away.

Instead I woke in the little study with my cheek on a folio, the lantern guttering low and the dust making constellations of my sleeves. The wick had tunneled; the oil was nearly spent. I could not have said how I came there. I sat up, prepared to be unnerved, and yet what rose in me was its opposite: the surety that I was precisely where I ought to be, doing the thing my hands had learned without waiting for my consent.

The folio bore a faint ring again—a pale halo on the paper, as if a vessel had stood there and cooled the room around it. Grains like salt pricked my fingertips. On the blotting sheet beside it my own pencil had been at work: a row of small marks in threes and eights, a space left carefully blank, then three more. I do not remember writing them. The marks looked patient, as if they had been counting something until I arrived to finish it.

There was a second light that did not belong to the lantern, a cold awareness behind my shoulder that made my shadow double for a breath and then agree with itself again. In that same moment the air lifted along my neck with the suggestion—not more than that—of feathers in passage. The rafters kept their cobweb banners; nothing moved that I could name.

The page lay open to a margin in Lexicus’s quick pencil. Nothing dramatic—only a question mark set beside a figure and the line, \emph{“Name nothing yet.”} The words felt correctly placed, like a hand flattening a wrinkle. I closed the book and listened. The room kept its own breath: a long hush, a soft return. Somewhere, far inside the walls, a slow drum counted to eight and declined to count the next.

\medskip
\noindent\textbf{Earlier hand (648 AR), Lexicus of Thepyrgos, Junior Reader}\\
\noindent\textit{Amber Reckoning: Planting, after fast–day, 648 AR}

Some weeks after my last, I discovered a \emph{very} small tract stitched into the back of a sermon–book—the hand uneasy, the vellum thin, the title in a timid red: \textit{a fragment of Sacred Geometry}. Much of it is crabbed and obscure; yet one sentence pierces: that certain men hold \textit{Canr\'e} to be more than pastime; that its figures are \emph{forms of address}; that a board, rightly attended, is a \emph{prayer laid flat}.

I carried the slip at once to \textit{Aqyl, son of Aqyl}. He read three lines, returned it to my palm, and said—kindly rather than sharp: “Return it where you found it. Fragments torn from their bindings make liars of both sides.” He regarded me a moment longer than comfort requires. “Are you sleeping, Lexicus? You have the look of a man who counts when he should be at rest.”

I answered that I was well (not the truth entire) and promised to replace the fragment. This I have done. The sentence about a prayer laid flat does not release me. I set it here under witness so that I may put it away later, \emph{cf.} fast–day cautions; \emph{viz.} the elder’s maxim on plain work before bright theories.

\medskip
\noindent\textbf{Present hand (876 AR), M.\ Fenwood}\\
\noindent\textit{Amber Reckoning: Leafturn, following morning, 876 AR}

This room has been too easy to find of late. The corridors that used to wind now come right, as if the stone had learned my steps and meant to be obliging. I am spending more hours here than at my other work and have begun to fall behind in the ordinary reading; the tidy stacks look at me with the patience of scolded cats.

Yesterday I met \textit{Aqyl, son of Aqyl} on the stair. He paused one tread above, the light cutting his cheek into two ages at once, and asked only: “Are you sleeping?” I said I was (which was not the truth entire). He considered me a heartbeat, almond eyes unkind to excuses, and said, “Eat something warm. Read in the public stalls for a night or two, where the lamps are bright and the bells keep time.” Then he went on, weighing spines in his head the way other men weigh bread; his thumb touched the rail eight times and declined a ninth without seeming to notice.

I write this in the morning to leave a mark I can see from elsewhere. If I do not come back for a day and the dreams return, I will count that as weather and not as instruction. Let the whispering be wind in the flues; let the slow drum belong to the clockworks; let the cold at the nape be only the hall breathing. If it is more, it has yet to learn my name; I will not teach it.

There is a faint grit on the page even now, though I wiped the desk last night—salt, or ash, or only old paper refusing to be new. I will keep to the bright lamps until this settles. If it does not settle, I will bring witnesses, and we will see whether the room keeps its breath when it is made to share.

\subsection{Fragment L--7 (648 AR / 876 AR) — Fairy Stones and the Unnamed Ninth}
\label{frag:l7}
\phantomsection

\noindent\textit{Provenance: a Theonan folio rendered into our speech by Lexicus (648 AR); later, a journal leaf by M.\ Fenwood (876 AR). The island script is angular and spare; several lines are smoke–nibbled at the edges.}

\medskip
\noindent\textbf{Earlier hand (648 AR), Lexicus of Thepyrgos, Junior Reader}\\
\noindent\textit{Amber Reckoning: Planting, Tenday III, Day 2 (648 AR)}

This afternoon I translated a small account from \textit{Theona}, the eastern isle. The writer calls his subject a game, but refuses its proper title. He writes:

\begin{quote}\small
“We will not speak the true name among strangers; in the market we say only \emph{Fairy Stones}. If you would learn it, you must swear to leave one seat empty and one turn uncounted.”
\end{quote}

The same folio assembles, with little ceremony, several island notions our city delights to mock: giants that eat the flesh of the drowned; doorways that will not open if you name them; the unholiness of the number \textit{nine}. The author notes that fishers count their nets in eights and set aside the ninth rope as \emph{ward}; that at winter feasts the ninth cup is poured and left; that tollmen on the south road strike eight neat notches, then make a faint scratch for the ninth and look away.

It is a thin piece and not well argued, yet it holds together like the ribs of a small boat. Refusal to name; leaving empty; fear of nine—these seem of a kind. If there is a custom beneath it, it is not new.

\medskip
\noindent\textit{Lexicus’s margin.} Our city keeps its own small reverences (who does not leave the first line clear on a new page?) and pretends they are accidents of neatness. Theonans do not pretend. Whether \emph{Fairy Stones} masks a better name or merely a truer one, I cannot yet say.

\medskip
\noindent\textbf{Present hand (876 AR), M.\ Fenwood}

Walking back from the study I found myself \emph{humming}. I do not know the tune. It keeps to itself, low and patient, as if it wished to be a rope rather than a song. When I sat down, the words came without asking. I set them here so I can see what my hand did:

\begin{verse}
Eight steps to water, one left behind;\\
eight stones in a circle, one will not bind.\\
Say nothing at thresholds; name nothing at doors.\\
If you must cross, count softly:\\
one for the keeper, one for the ford,\\
one for the watcher who will not be stored.\\
Leave a cup poured; leave a chair bare.\\
What you do not call by its name\\
cannot answer your prayer.
\end{verse}

I do not recall ever hearing this. Perhaps it is only a trick of tiredness, the mind laying boards across a ditch. Still, the number keeps returning. I write it in the bright part of the day, where I can look at it and decide later whether it is only weather.

Later, on the west stair, I met \textit{Aqyl, son of Aqyl}. He was coming up softly, thumb and forefinger tapping the rail: eight light touches and then a pause that did not quite become a ninth. He said nothing at first, but as we passed I heard him \emph{answer} my tune under his breath—no more than a thread of it, a low second that seemed to know where the line would turn before I did. When he noticed my listening he smiled the way one does at a child’s rhyme and said, “Old border thing. Market noise. Eat something warm.” His almond eyes had a weight I could not count. He touched the newel once, as if to number it too, and went on.

If the mind makes its own songs, let it make this one. The lamps are bright; the bells keep time; and though the words sit oddly on the page, they sit well in the mouth. Oh well—\emph{it is a pretty ballad}.

\subsection{Fragment L--8 (648 AR / 876 AR) — A Dream of Sacred Geometry}
\label{frag:l8}
\phantomsection

\noindent\textit{Provenance: a night–note by Lexicus (648 AR) concerning a vivid dream; a later journal leaf by M.\ Fenwood (876 AR). The Lexican page is blotched as if written upon waking.}

\medskip
\noindent\textbf{Earlier hand (648 AR), Lexicus of Thepyrgos, Junior Reader}

I woke before the bell with my hand already moving. The dream would not keep still unless I pinned it with ink.

I saw a plane laid out with lines that were not lines but \emph{bindings}, faint as breath on glass. The figure arranged itself by preference rather than command: eight regions that received the eye and a ninth that refused it, as if the page were shy of being complete. Around the outermost I felt a coolness, like damp salt drying to a ring.

There was a rhythm underneath (do not laugh at this), not music but the sense that a step awaited a step, and that speech would be in the way of it. In the middle space—if I may call it middle—I tried to set a small mark to test whether the figure were mine to disturb. The mark slid of its own accord to a neighboring place and would not abide where I had first wished it. I was not afraid, exactly. I was \emph{glad} to be contradicted by something that kept its own account, and then, right after, \emph{ashamed} to have felt glad.

I drew three little copies when I sat up. The first two are nothing but nets. The third preserves a hint of the ninth—an \emph{unwritable} hollow that makes the eights honest. I am excited by this and concerned in equal measure. If I am not careful I will begin to prefer the neatness of the emblem to the disorder of persons. This is a warning to myself to eat, to speak to someone, and to read plainer books for a day.

\medskip
\noindent\textit{Lexicus’s margin, later the same morning:} The phrase \emph{Sacred geometry} comes too easily to the tongue when one has slept poorly.

\medskip
\noindent\textbf{Present hand (876 AR), M.\ Fenwood}

Last night I dreamt of shapes that would not settle. Nothing theatrical—only angles arriving where I meant curves, and a narrow gap my eye kept walking around as if a low fence stood there, polite and immovable. I am calling it \emph{the power of suggestion}. If you read long enough about circles, you dream circles. If you spend too long copying talk of an unwelcome ninth, the mind obliges with a hollow.

Still, I ledger what I remember, so it can live with weather and not with omens: a sense of \emph{counting without numbers}; a coolness that laid itself along the edge of things like a ribbon; the feeling that silence itself had learned to be a kind of measure. Once, I thought I heard a bell mark the hour—eight strokes, then a patient pause that did not consent to be the next.

Today I will keep to the \emph{bright stalls}, do the ordinary work, and eat something warm. If the shapes follow me into daylight, I will call that proof it is not the room but the hour, and I will sleep earlier. My thumb and forefinger have begun to tap railings without asking—eight light touches and then a breath. A habit, surely. If it becomes anything else, I will write it down and call it weather again.

\subsection{Fragment L--9 (876 AR) — A Package from Constano: On the Twin--Oases}
\label{frag:l9}
\phantomsection

\noindent\textit{Provenance: a bundle posted from C.\ Fenwood to M.\ Fenwood, containing family letters and working notes on \emph{Canr\'e}. One leaf bears a penciled header, “Twin--Oases,” struck through in a later hand.}

\medskip
\noindent\textbf{Present hand (876 AR), M.\ Fenwood}

This afternoon a parcel arrived from my brother \textit{Constano}: old correspondence, a sheaf of copies, and two brittle offcuts shaved, by the look, from a larger draft of the \emph{Concordance}. He writes:

\begin{quote}\small
“Found these in the black trunk from the east room. Mostly letters. Two leaves look like Papa’s working pages. One is headed \emph{Twin--Oases}. The header’s scored out, but the notes underneath are tidy. If they’re yours by right, keep them. If not, return them to the trunk and pretend I never sent this.”
\end{quote}

The \emph{Twin--Oases} leaf is not an opening so much as a \emph{shape}: two opposite edges taking the part of wells, the middle kept shallow for a time, the hand taught to move as if water were dear. It is a way of beginning that prices thirst. In the margin, in my father’s small pencil:

\begin{quote}\small
“This feels like a \emph{rite}, not an opening. Disturbing. It teaches the crossing rather than the play.”
\end{quote}

I do not quite understand his unease. The figure reads as sensible to me: two “wets” to balance want, a hush at the start to make the first steps honest, a rim circled not as a wall but as a reminder to step carefully. If a beginning can teach restraint, is that not a lesson worth keeping?

It unsettles me that it does not unsettle me.

Constano’s note smells of the cedar box; Papa’s pencil is faint enough that I must angle the page to catch it. The header truly is scored out with care, not temper—three light strokes, even, as though someone were hiding a candle rather than putting out a fire. In the scored place, if you run a finger very gently, the paper dips like a shallow cup. On the verso someone has begun a count, neat as rails: \emph{one two three four five six seven eight}—then a courteous space—and again \emph{one}. It looks like practice; it feels like permission.

I will ask \textit{Aqyl, son of Aqyl} whether he has seen this heading and why it was moved quietly to a drawer. (On the south landing today I thought I heard him hum a low second to a tune I could not quite place; his thumb tapped the rail—eight, and then a polite pause.) For now I am copying both leaves exactly as they are and returning the originals to the parcel. If Constano asks, I will tell him they are safer in his cedar than in my desk.

There is a dry taste at the back of the mouth as I write this, like chalk that remembers rain. When I set the copy aside, a pale damp ring ghosted the blotting paper and then faded. I tell myself it is nothing but the press of my cup. I will keep saying so until it is true.

\subsection{Fragment L--10 (648 AR / 876 AR) — On the House of Wells and the Witness}
\label{frag:l10}
\phantomsection

\noindent\textit{Provenance: a stitched quire of uneven leaves in Lexicus’s hand (648 AR), ink crowded and overwarm at the margins; followed by a brief journal entry by M.\ Fenwood (876 AR). The quire’s thread is brittle and smells faintly of oil and salt.}

\medskip
\noindent\textbf{Earlier hand (648 AR), Lexicus of Thepyrgos, Junior Reader}

I will set down, at length, the structure as I now see it, that I may not lose the thread when the lamps gutter.

First: the \textit{House of Wells}. Their ledgers are proud of being plain, yet the plainness is a garment. I find small clippings of value (\emph{copper shaved thin}) recurring on days when ring–stamps are heavy; I find toll slips entered twice, once as road–duty and again as \emph{benefaction}. The difference is small enough to pass a sleepy audit, large enough to carry a habit along for years.

Second: the circle that names itself \textit{the Witness}. They do not write minutes; they write margins. Where the Cartwright teaches a careful beginning and a price for crossing, these copyists preach a crossing that \emph{watches back}. See the phrases recurring: \emph{ford held}; \emph{silence is counted}; \emph{salt binds the rim}. See also the practice of leaving one line unmarked and calling it \emph{kept for the missing place}.

Third: the junction of the two. In the Wells books, I mark chits stamped with triangles where coins should be; in the Witness margins, I mark triangles where tallies should be. The same hand? Two hands that learned from one teacher? I cannot yet prove it, but the drift is toward a cult of accountancy: that the world may be kept honest by making its thresholds costly. If it were only a philosophy, I would applaud. If it is a church, I am afraid.

There are voices in the next room.

They are not speaking to me exactly; they are \emph{answering} a question I have not asked, or have asked too softly. I set down an argument to quiet them:

\begin{enumerate}\setlength\itemsep{0.25em}
  \item If a rule is just, it will stand in daylight.
  \item If a rite requires secrecy, it is either childish theatre or theft.
  \item The Cartwright wrote to teach play, not to purchase souls.
  \item Therefore the Witness, if born of him, is a bastard child.
\end{enumerate}

The voices are not convinced. I could go into the next room and demand names, but that would break the frame of the work. I will not break the frame. The page is my room; the margin is my door; the door will stay shut until the figure shows itself without my calling.

\medskip
\noindent\textit{Lexicus’s margin, later:} If this be madness, let it at least be ruled madness. I will keep my sums straight even as my breath runs.

\medskip
\noindent\textbf{Present hand (876 AR), M.\ Fenwood}

I asked \textit{Aqyl, son of Aqyl} this morning, outright, whether there was ever a \emph{Cult of the Witness}. A new frown gathered—small, exact, like a net drawn tight. He said only:

\begin{quote}\small
“There are places in this University that are not secrets, and yet they are best left alone. Stay out of that room. Read where the bells can scold you. If a thing is worth keeping, it does not need you to whisper to it.”
\end{quote}

I said I understood. I am not sure I do. I have written the warning here so that I must step over it if I ignore it.

When we parted at the stair, his thumb tapped the rail—eight light touches and then a pause that refused to become the next. He did not seem to notice. As he turned away he hummed two notes that found my earlier tune and held it under breath, like a thread tucked into a seam. I do not think he lied. I do not think he told me everything.

\subsection{Fragment L--11 (876 AR)  — The Veil of Names and the Closed Door}
\label{frag:l11}
\phantomsection

\noindent\textit{Provenance: an elated but unsteady note in the hand of Lexicus, dated “\textemdash12 AR” in a later pencil (catalogers dispute the obscured digit); paired with a present-day leaf by M.\ Fenwood (876 AR).}

\medskip
\noindent\textbf{Earlier hand (650 AR), Lexicus of Thepyrgos, Junior Reader}

I have it. I \emph{have} it, or the edge of it: the \textit{Veil of Names} is not a cipher of letters but a \emph{conduct}. One does not \emph{name} in certain rooms; one leaves the first line empty; one signs with a figure that is not one’s own. The grace of it is to keep the breath from preening. The danger is that a mask, worn for clarity, will begin to think for the face beneath.

For three nights my steps in sleep have brought me to a door I do not know by day. I describe it so that haste will not trick me: oak, swollen a little with damp; three iron straps across, three down, their \emph{crossings} each stopped with a small rosette; the heads of those rosettes are pin–pricked around so that, taken together, they suggest a ring of salt–flowers. There is no handle. The keyhole is long, almond–shaped, with a tiny chip lifted from its lower lip. The sill shows a pale crust as if something had dried there and been poorly swept.

Behind it, chanting. Not commands—\emph{keepings}. The syllables fold over one another like cloth; I know none of the words except the word I am \emph{not} to say.

I woke with my hand on the wall beside my bed, palm flat as if feeling for the grain of the oak. I am giddy with the nearness of it and sick with the same. If I open it (there must be a way, for why else the keyhole?), I will break the work. The \textit{Veil} says: do not name until the crossing is priced. So I will not name; I will keep the first line empty; I will count aloud over ordinary things and write what I can prove.

\medskip
\noindent\textit{Lexicus’s margin, cramped:} The door is a sentence. To open it is to finish the line before it earns its period.

\medskip
\noindent\textbf{Present hand (876 AR), M.\ Fenwood}

A voice in last night’s dream said, very plainly, “Not yet. Do not show what you have. The time is not right.” Then, as if offering a kindness: “Do not trust Aqyl, son of Aqyl.”

Writing it now, in daylight, I dislike the taste of those sentences. They are not like my thoughts, nor like Aqyl’s speech. They sit too cleanly, as if cut from a slate and pressed under my door. I put them here so they must survive the morning to trouble me; that is a higher bar than a dream deserves.

If I were sensible I would call this \emph{suggestion} and \emph{fatigue}. Read enough leaves in one voice and you begin to think in it; sleep too little and the mind supplies a chorus to save you the labor of deciding. Still, the little room keeps tugging at my sleeve, and the hour keeps slipping its buttons. I have not told Aqyl about the parcel from Constano or the two cut leaves. I do not like that I am keeping two ledgers—one for the day, one for the door.

I will eat something warm, sit in the bright stalls, and speak of none of this page for a day. If the voice returns, it can make its case again—and I will hear it \emph{against the bells}. If I am lucky, the bells will win. If I am not, then I will at least have counted how many times the hour refuses to step beyond eight.

As I closed the notebook a grit came off on my thumb—salt, or ash, or only old paper remembering weather. In the corridor the porter’s clock struck eight, then paused, obedient as a dog at a threshold. On the stair below, someone hummed two low notes and let the third go missing.

\subsection{Fragment L--12 (650 AR / 876 AR) — The Teacher, Copper \& Salt, and the Door}
\label{frag:l12}
\phantomsection

\noindent\textit{Provenance: two entries in Lexicus’s hand (first and “weeks later,” 650 AR) concerning a tutor of the Cartwright’s discipline; followed by a present–day leaf by M.\ Fenwood (876 AR).}

\medskip
\noindent\textbf{Earlier hand (650 AR), Lexicus of Thepyrgos, Junior Reader — first meeting}

I sought out a teacher reputed to keep the Cartwright’s strategy in the old, spare way. We sat at a plain table. Before a word, he drew a narrow ring of damp salt about the board’s edge and set his hands in his lap. When he finally spoke, it was in \emph{code}, as though discussing tariffs:

\begin{quote}\small
“Some crossings are \emph{priced}. Some vows are better \emph{kept unnamed}. A watcher keeps his ledger without boasting.”
\end{quote}

I answered that I knew of the vow he would not name, and (here I confess my rashness) that I had dreamt of a door—oak, rosettes set at each crossing, no handle—behind which men chant not commands but keepings. His eyes changed, not with anger but with a kind of inventory.

“You are before your hour,” he said at last. “When you can beat me clean at \textit{Canr\'e}, you will be ready to be shown what cannot be \emph{said}.”

We began to play.

\medskip
\noindent\textbf{Earlier hand (650 AR), Lexicus — weeks later}

After many sessions, and more losses than I care to number, I kept a steady book and at last defeated him—no cleverness, only the calm of counted breath. He did not sulk. He beckoned me close and spoke in a tongue I have not studied but \emph{understood}, as if my thought were already arranged to receive it.

He called it the \textit{Rite of Copper and Salt}. A fragment of his utterance (set here without its cadence):

\begin{quote}\small
\textit{Copper before threshold, salt upon rim;\\
price what is entered, bind what is thin.\\
Count what is taken, count what you spare;\\
open what answers, and leave the rest bare.}
\end{quote}

It is only \emph{logical}. Of course it is. Copper to mark what is dear; salt to make the edge honest; silence to value speech. He told me to come at dusk on the third day. “Bring no one,” he said. “The door you know will be open for those who have earned its measure.”

I am writing with a fast hand. The page feels too narrow for the hour.

\medskip
\noindent\textit{Lexicus’s margin, later that night:} The figure fits. I am not afraid. I am \emph{ready}. (I am also not sleeping.)

\medskip
\noindent\textbf{Present hand (876 AR), M.\ Fenwood}

I found the hall by accident, or so I tell myself. The corridor is the sort that collects notices and forgets them: flaking paint, old stencils, a buckled runner that remembers more feet than names. But the door—\emph{the} door—was not like its corridor. The oak looked newly rubbed; the iron rosettes were clean as coins. On the sill, a fine pale crust had been pushed aside lately, not well, leaving a half–moon ridge like surf that never learned to retreat.

There was a shallow scratch on the strike, bright as a breath. I set my palm to the wood and felt it cool, not cold—the cool of a vessel that has been emptied but not yet put away. For a breath (mine, I hope) I heard a low line of chanting, the sort that lives under the world like a seam and hums whether you deserve it or not. Then the lamps along the hall took a stutter together, and the sound—if there was a sound—drew thin as thread and tucked itself into the grain.

The air smelled of iron and damp feathers. A draft rose from under the threshold and laid itself along the bones of my hand as if counting them. In the lacquer of the nearest rosette I saw a small ring of light where no lamp stood; it widened, narrowed, and waited. I did not try the keyhole. I did not knock. I stood and counted to eight and left before my right foot could choose the next.

On the landing below I found my breath again, and with it the tune. I was humming without meaning to. Someone on the stair above—light step, patient—took a low second to it and let the third go missing. When I looked up there was no one there. Only the lamps, now steady, and a long shadow that decided to be two and then agreed to be one.

I have written this quickly and without sum. I am going to the bright stalls to eat something warm. I will leave this page open on the desk and see whether the dust settles evenly. If it does not, I will call that weather. If it settles in a circle, I will close the book with both hands and sleep with the bells.

\subsection{Fragment L--13 (650 AR / 876 AR) — On the Missing Count and the Remaining}
\label{frag:l13}
\phantomsection

\noindent\textit{Provenance: a fervent note in Lexicus’s hand (650 AR), edges rubbed; paired with a present–day journal leaf by M.\ Fenwood (876 AR). The older leaf shows pressure–marks where the pen bit; the newer smells faintly of lamp oil.}

\medskip
\noindent\textbf{Earlier hand (650 AR), Lexicus of Thepyrgos, Junior Reader}

The step beyond \emph{eight}—they curse it so upon the eastern isle and will not number it aloud. My \emph{new companions} are much the same. But in \textit{Dhahara} that same count is \emph{Sacred}. Their doctors speak of \emph{messengers}, pillars that keep the world square; others, more careful, speak of \emph{the Remaining}. They do not say \emph{from what} they remain. The courtesy is deliberate; it is a price paid in silence. I listen; I learn; and as I learn the figure stands straighter in the mind.

It is not a cipher of letters. It is a \emph{use}. Leave the first line empty. Count to \emph{eight} and bind the rim. Refuse the next until the crossing is priced. Do this, and the room \emph{answers} without your naming it. I can see it now. \emph{I can see it}. The same hush the Theonan comforters fear is the hush the Dhaharan expositors bless.

The habit enters the hand before it enters the reason. This is a confession; this is also a method. I will keep both ledgers: one that says \emph{plainly} what is done; one that leaves space for what answers back. They must not trade places. (Write this again when the lamps waver.)

\medskip
\noindent\textit{Lexicus’s margin, compressed:} If the missing count is a wound to some and a seal to others, the work is to learn \emph{which} it is before speaking. Pride ruins sums; hunger ruins silence.

\medskip
\noindent\textbf{Present hand (876 AR), M.\ Fenwood}

Third night in a row with the corridor. This time the door was \emph{open}. Without ceremony—simply open, as if openness were its natural state and my memory the error. The room beyond was dark the way a street is dark when the lamps are not yet lit: not empty, only unwilling to be of use.

Mid–dream I thought I \emph{ought} to be unnerved. I was not. I stood at the threshold and felt a cool draft from within, the kind of air that has traveled long along stone. It smelt faintly of iron and old rain. I did not step through. I woke with my hand lifted, fingers spaced as if to learn the grain of the jamb.

I am calling this a \emph{recurring dream} and blaming it on pattern–seeking. Read enough about doors and one will open to oblige you. Still, ledgering helps. I counted \emph{eight} heartbeats at the sill before waking; the next went missing of its own accord. Somewhere—this is foolish, but I write it—the faintest hum held a low second and declined the third.

If tomorrow’s version invites me in, let this daylight hand stand here like a porter on the page, and make me pause.

\bigskip

\subsection{Fragment L--14 (653 AR / 876 AR) — The Ash–Fenn Rite and the Candles}
\label{frag:l14}
\phantomsection

\noindent\textit{Provenance: an exultant, overfull entry in Lexicus’s hand (653 AR) recording his formal induction; paired with a present–day leaf by M.\ Fenwood (876 AR). The older page is salted at one edge; the newer leaf is clean but for a pale ring on the blotter.}

\medskip
\noindent\textbf{Earlier hand (653 AR), Lexicus of Thepyrgos, Junior Reader}

Tonight they taught me the \textit{Ash–Fenn Rite}. I am bid not to name its steps in the open book; I am permitted to set down that it is a keeping, a pricing, a remembering. Copper placed in my palm; a ribbon circled my wrists, touched to salt and to water. The words were not the teacher’s alone—the room made them easy, as if I had been rehearsing them unawares.

I am—let the sentence stand—I am \emph{ecstatic}. Not with noise, but with the swift quiet that runs under the ribs. I am to learn and record what I can prove; to keep the first line empty and the last line honest; to be no one’s cleverness; to be a \emph{witness}.

They say the circle grew from the Cartwright’s discipline and took its charge from the burning east of the gates. They say the Rite binds a wound without boasting of the scar. I believe them. I \emph{see} it: thresholds priced so crossings are not squandered; silence given its measure; the rim made true so the middle may be merciful.

Henceforth I will keep two ledgers: one for daylight and one for the room. They do not contradict; they converse. If there is danger, it is that the night–book will begin to think for the day. I will not allow it. (Write this like a charm; perhaps it \emph{is} one.)

\medskip
\noindent\textit{Lexicus’s margin, crowded:} The Ash, the Fenn; the salt, the copper. Penance and price. I am steady. I am steady. I am \emph{steady}.

\medskip
\noindent\textbf{Present hand (876 AR), M.\ Fenwood}

I walked to the door again. I told myself I was only confirming the corridor. The oak was cool under my palm—the sort of cool that lives a little deeper than the surface, as if the wood kept a private weather. I set my ear to the seam to be clever and bought myself a headache that bloomed behind the right eye, not sharp, only \emph{insistent}, like a finger laid there and asked to stay. I stepped back and laughed at myself in the empty hall, which made the place feel less empty. The laugh came back late, from somewhere farther in than the walls should allow.

That night I dreamt the room beyond. There was a table and candles set about it—not a crowd, not a feast, only a \emph{shape} made of light. I counted them before I could think not to: \emph{eight} small flames keeping company, and another that preferred to be a cup of oil. I woke with the number sitting in my mouth like a word I am not yet permitted to say.

This is a record, not a vow. Tomorrow I will read where the bells can scold me and let the door be wood again. I will eat something warm and keep to the bright stalls. If the room wants more than that, it can knock where a porter listens.

\subsection{Fragment L--15 (654 AR / 876 AR) — The Ninth and the Open Room}
\label{frag:l15}
\phantomsection

\noindent\textit{Provenance: a scorched leaf in Lexicus’s hand (654 AR), edges singed and brittle; paired with a present–day entry by M.\ Fenwood (876 AR). The older ink is hurried and overbold; the newer bears a pale salt grit in the gutter.}

\medskip
\noindent\textbf{Earlier hand (654 AR), Lexicus of Thepyrgos, Junior Reader}

They have taught me \emph{The Ninth}. I am commanded to speak it only in keeping, and I have obeyed. I have burned the day–book. I do not repent it. A book that cannot keep a vow is worse than no book at all.

I will set the shape in \emph{code}, for code is a fence that lets the field breathe:

\begin{quote}\small
\textit{Leave the first line empty. Bind the rim with what bites the tongue.\\
Count crossings as prices, not as boasts.\\
When the word that is not said is ready to be said, speak it once and only once.\\
Then, for one breath, unmake the binding you have made.\\
Do not spend that breath on cleverness; spend it on measure.\\
Close the book before the smoke names you.}
\end{quote}

It is clear. It is \emph{clear}. For one breath a keeping may be set aside—\emph{not} to cheat, but to seal. The wound is named; the rim is cracked and made true again. I know the word. I will not write it. The page would not carry it without bending.

The room received me. The door did not resist. The watchers keep no faces, only ledgers. I am not afraid. Ash on my sleeve is only ash. The work continues where the bells do not scold.

\medskip
\noindent\textit{Lexicus’s margin, scorched:} A book left unburned may burn the hand if kept open too long.

\medskip
\noindent\textbf{Present hand (876 AR), M.\ Fenwood}

In the bright stalls I opened yesterday’s notebook to the page where I had copied “ford held.” The pencil’s faint groove remained, yet the words themselves were gone—as if the paper had remembered being blank. Under a slant of light the graphite dust shows, but no phrase returns to be swept back into being.

I rubbed the margin; my thumb came away gritty. Salt, or ash. I will copy the words again in a heavier hand and see if they keep.

\medskip
\noindent\textbf{Present hand (876 AR), M.\ Fenwood}

The door was open.

I needed no key. The corridor that always looks tired looked merely itself; the oak looked newly rubbed, as before. \textit{Aqyl, son of Aqyl} stood within, beside a table where a \emph{Canr\'e} board was already set. The room was clean—no dust bloom, no cobweb banners—only the faint crust of pale grit brushed to the skirting, and the scent of candles put out with care warmed into the wood.

He said my name in the ordinary way, and then: “Come in.”

\medskip
\noindent\textit{Later (same day), copied with a cool hand.}

Eight candles were lit; a ninth vessel stood unlit—a small cup of oil with a clean wick asleep in it. No flourish. On the floor, only the rumor of a ring, as if a damp cord had once been set down and lifted again before it could confess. A shallow bowl of water stood at the table’s end. He did not explain it.

We sat. He placed my hand upon the board as if teaching a child to write. “Count,” he said, very quietly. “Not the numbers you know. The ones the room knows.” He hummed two notes under his breath; my own hum—uninvited—found a third that did not wish to arrive. We began.

The pieces—let me call them that—moved as furniture moves in a house long lived in: without thinking, habit to habit. The middle learned to be a shallow river; the edges remembered weight. Each time I paused, the air along my neck lifted with the suggestion of feathers, and the eight flames leaned as if a draft went past them on purpose. When I strayed toward the easy line, Aqyl’s thumb tapped the table—eight light touches and then a pause that refused to become the ninth. He did not look at me when he did it. He looked at the unlit cup.

“Name nothing yet,” he said once, with the same voice he uses to correct a sum. “When it is time, you will not have to reach for the word; it will arrive already shaped to your mouth.”

I do not think we played long, and yet I came away with the sense of a road walked and priced: crossings taught, not taken; chairs left waiting; cups poured and refused. Once—only once—the rules I had brought into the room moved aside as a porter lifts a rope. I set a piece where I knew I should not, and the air accepted it without rebuke, like water learning a new edge for a breath. Then the moment closed as a pupil closes upon light, and the ordinary measure resumed.

When we rose, the eight wicks sat smaller, blue at their bases, and the oil in the ninth cup held a patient chill—like a word that will not yet consent to be spoken. Aqyl pinched each flame out with steadiness and did not look at me until the last thin smoke climbed and made its small rope toward the rafters.

“Eat something warm,” he said, as always, and—lightly—“Read where the bells can scold you.” Then, softer, not quite a question: “You will come back.”

I am in the bright stalls now. The bells are loud. The page holds still. My hand is steady. I have copied what I saw and will stop there, for now.

If there is more to tell, it will knock. If it does not knock, I will hum on the way home and let the lamps do their work. The tune sits easily in the mouth. \emph{After all, it is a pretty ballad.}

\subsection{Fragment L--16 (876 AR) — Lerris Enrolls; A Quiet Vow}
\label{frag:l16}
\phantomsection

\noindent\textit{Provenance: a present–day note by M.\ Fenwood; paper new, hand steady. A sliver of dark metal (the not–key) is pasted to the margin with sealing–wax residue; beneath it, a Wardens’ receipt is pinned through.}

\medskip
\noindent\textbf{Present hand (876 AR), M.\ Fenwood}

My brother \textit{Lerris} arrived with a trunk and a grin and an enrollment chit that still smelled of warm wax. We walked the outer courts as we used to walk the lanes at home, naming buildings, counting bells. He is taller than I remember and kinder than I deserve. He asked where to begin.

I said: the bright stalls, the public rooms, the catalog, the ordinary work. We found him a bed in the north dormitory, ate something warm (stew, brown bread), and watched the porter lock the gate. The lamps minded their order. The hour struck and minded its business. We counted to eight and held.

I thought of the little study, and the cool of the oak, and the way some pages answer if you are patient. I thought of the door.

\medskip
\noindent\textit{Mark what it took to answer.} Warm food. Bells within hearing. A second set of footsteps beside mine—kept for the count. A page to write on.

I have wrapped the sliver–that–is–not–a–key in plain paper and sealed it. At first bell I went to the Wardens’ desk and handed it across. The clerk weighed the metal like bread, wrote a number, circled it once and then again, and kept the piece. He sealed the \emph{wrapper} to a narrow finding card “for the file,” stamped the same number, and returned it with a stub. I have pasted the stub here so the act does not depend upon my memory.

I will write \textit{Alayse} tonight—first line empty, sealed with salt as is proper—and tell her the day’s true sum. I will read in the bright rooms, under lamps that scold, and I will not enter the condemned door: not alone, not without a porter’s book signed, not without naming what it takes to answer.

If the room wants more, it can knock where a porter listens and say its price in daylight. Until then, \textit{we} will copy what we can prove and leave the first line empty.

\medskip
\noindent\textit{Small hand at the foot of the page, in a finer ink:} The Ninth demands it\ldots{}

\subsection*{Coda — Letter from Lerris to Conlin (876 AR)}
\addcontentsline{toc}{subsection}{Coda — Letter from Lerris to Conlin (876 AR)}

\noindent\textbf{From:} Lerris Fenwood, North Dormitory, Thepyrgos \\
\textbf{To:} Conlin Fenwood, House Fenwood \\
\textbf{Seal:} common wax; first line left empty

\medskip
\noindent Brother,

I am arrived. The porter took your signed chit and found it in order. Markus met me in the outer court. He is thinner than I remember and very steady. We walked the squares as at home, naming buildings, keeping the bells. When they came to eight he let the hush run and then we went on.

He showed me the bright stalls and the public rooms, set me in the north dormitory, and saw me fed (stew, brown bread). In the catalog hall a thin gentleman in a linen coat greeted us—Aqyl, he wrote, and then, after a breath, \emph{s.\ Aqyl}. He asked after the Valvano legacy and said our house had kept good ledgers. He trimmed a finding card for me and said, mild as weather, that a clean first line keeps the lamps honest. He was warm and in a hurry and was gone before I could thank him twice.

The bright rooms are easy to work in. If I step wrong-footed, one of the lamps gives a small scold and then settles. There is a drum somewhere that keeps even time with the bells when you stand by the catalog table. It makes the counting easy.

This evening, when I shook out my coat before folding it, a narrow card in a paper sleeve slid from the inside pocket. It is bone-colored, sealed along one edge, and carries a long number in a precise hand. I thought perhaps the catalog man tucked it there with the card. As I turned it, there was a brief whisper at my shoulder—one soft syllable, maybe only the porter on the stair—and I shook it off.

I am resolved to return it to the desk at first bell. I must confess I was curious about its contents, and to my shame I opened it; there is within a thin, bright sliver, colder than paper, that—

\medskip
\noindent Forgive me. I must conclude this letter and keep the hours.

\medskip
\noindent Your brother, \\
Lerris

\clearpage

\addcontentsline{toc}{section}{Part II — Rites \& Scenarios}

\section{Rites \& Scenarios}
\label{part:rites}
\phantomsection

\begin{quote}\small
``Call a rule a law for the hand; call a rite a lens for the eye.  
Turn the lens and the same moves speak differently. Remove it, and nothing is broken.''\\
\hfill — \textit{Lexicus of Thepyrgos}, margin note
\end{quote}

\noindent\textbf{What follows.} This part presents seven self-contained overlays (\emph{Rites}) you may toggle \textsc{ON} or \textsc{OFF} per session. Each one adjusts \textit{what you notice}—pace, pressure, information—without changing the tournament core. Leave any Rite \textsc{OFF} and play proceeds exactly as you already know it.

\medskip
\noindent\textbf{Anatomy of a scenario.} Every scenario includes:
\begin{itemize}\setlength\itemsep{0.3em}
  \item \textbf{Hook} — the in-world cue that frames the table mood.
  \item \textbf{Switch \& Scope} — explicit \textsc{ON/OFF}; never touches core rules.
  \item \textbf{Setup \& Components} — simple markers (ribbon, beads, chits) you can substitute from household items.
  \item \textbf{Rite (exact)} — concise, rules-precise text.
  \item \textbf{Clarifications} — edge cases, if any.
  \item \textbf{Clue beat} — an optional thread in the dossier’s larger puzzle.
  \item \textbf{Why it’s safe} — a brief reassurance of rules integrity.
  \item \textbf{If \textsc{OFF}} — a reminder that play is unchanged.
\end{itemize}

\medskip
\noindent\textbf{Reading the cues.} When a scenario mentions documents or fragments, they enrich the scene but never confer hidden advantages. The choice to use a Rite is aesthetic and procedural, not compulsory.

\medskip
\noindent\textbf{Suggested table aids.} A slim ribbon (for the perimeter), five beads or coins (for public counts), and a few copper-colored tokens (for optional actions). Printable templates appear in the appendices.

\medskip
\noindent\textit{Begin with \S\ref{scen:salt-stitch} \textemdash{} a simple perimeter ward that makes the edge feel like a threshold without moving a single rule.}

\subsection{Scenario 1 — Salt Stitch}
\label{scen:salt-stitch}
\phantomsection

\noindent\textbf{Hook.} Markus notes the caravan custom of dusting thresholds with salt; the board’s rim echoes the ward.

\medskip
\noindent\textbf{Switch.} \textsc{ON / OFF} (default \textsc{OFF}) \hfill \textbf{Scope.} Scenario-scoped; never alters tournament core.

\medskip
\noindent\textbf{Components.} A thin ribbon, string, or corner markers sufficient to indicate the board’s outermost ring (\emph{salt-band}).

\medskip
\noindent\textbf{Setup.} Before normal setup, lay a visible \emph{salt-band} tracing the perimeter ring.

\medskip
\noindent\textbf{Rite (exact).} While this Rite is \textsc{ON}, \textbf{Green} pieces may not \emph{enter}, \emph{cross}, or \emph{end} on any perimeter square. All other pieces ignore the band. No other rules change.

\medskip
\noindent\textbf{Clarifications.}
\begin{itemize}\setlength\itemsep{0.25em}
  \item If any effect would place a Green onto a perimeter square (including specials or setup variants), that placement is illegal; choose another legal option.
  \item Adjacency across the band is allowed; only occupancy/crossing of perimeter squares by Greens is prohibited.
  \item Captures by or against Greens proceed normally, provided the Green’s path does not require stepping onto a perimeter square.
\end{itemize}

\medskip
\noindent\textbf{Example (two turns).}  
A Green approaches the rim to cut a lane; the opponent pressures along the edge. The Green must weave one file inward (may not step onto, or hop through, the perimeter), delaying the escape by a tempo without changing any costs or capture rules.

\medskip
\noindent\textbf{Clue beat.} A city map leaf shows pinholes; rotate \(\,17^\circ\) and sanctums align with gatehouses.

\medskip
\noindent\textbf{Why it’s safe.} Only restricts where \emph{Greens} may legally occupy or traverse; does not alter move generation, costs, rooting, specials, or scoring.

\medskip
\noindent\textbf{If \textsc{OFF}:} Play is unchanged.

\medskip
\noindent\textit{Diagram cue (optional).} Perimeter traced; corner exemplars marked with \(\times\) to illustrate barred \emph{end} squares for Greens (caption: all perimeter squares are barred to Greens).

\subsection{Scenario 2 — Witness at the Ford}
\label{scen:witness-ford}
\phantomsection

\noindent\textbf{Hook.} “Stay more than thrice at the ford and the Witness takes you.” The center is a crossing, not a seat.

\medskip
\noindent\textbf{Switch.} \textsc{ON / OFF} (default \textsc{OFF}) \hfill \textbf{Scope.} Scenario-scoped; never alters tournament core.

\medskip
\noindent\textbf{Components.} None beyond the standard set (optional: a small die or bead per Blue to track counts).

\medskip
\noindent\textbf{Setup.} None.

\medskip
\noindent\textbf{Rite (exact).} While this Rite is \textsc{ON}, each \textbf{Blue} tracks how many of its controller’s \emph{consecutive turns} it has \emph{ended} on any square of the \textbf{Central Four} (the 2\(\times\)2 center). If a Blue would end a \textbf{4\textsuperscript{th} consecutive turn} in the Central Four, that Blue is \textbf{forfeit} (captured) immediately at end of turn. \emph{Leaving} the Central Four at any point resets that Blue’s count to 0.

\medskip
\noindent\textbf{Clarifications.}
\begin{itemize}\setlength\itemsep{0.25em}
  \item “Consecutive turns” means turns of that Blue’s \emph{controller}. If a Blue remains in the Central Four across an opponent’s turn, the count is unaffected until its controller’s next turn ends.
  \item A Blue that is \emph{Rooted} within the Central Four still accumulates counts (it is “staying”).
  \item If a Blue leaves the Central Four (by moving, being displaced, or captured) its count resets to 0. If it later re-enters, start again from 1 the next time it ends its controller’s turn there.
  \item Track each Blue separately. If a Blue is captured before reaching four, remove its counter (if any).
  \item If a pass or skip occurs and the Blue remains in the Central Four, that still counts as having “ended the turn” there.
\end{itemize}

\medskip
\noindent\textbf{Example (tempo at the ford).}  
On Turn A1, Blue\(_\alpha\) ends in the Central Four (count 1). On B1 nothing changes for Blue\(_\alpha\). On A2 it still ends there (count 2); on A3 (count 3). If on A4 it would again end in the Central Four, it is forfeited at end of A4. If instead it steps out on A3 and returns on A4, the count resets—A4 becomes count 1.

\medskip
\noindent\textbf{Clue beat.} A toll ledger shows three sequential stamps; the fourth is crossed in ash, matching the “three then price” cadence seen in Candle Count.

\medskip
\noindent\textbf{Why it’s safe.} Adds a local stay-timer only for \emph{Blues} in the Central Four; no movement rules, costs, specials, or captures elsewhere are changed.

\medskip
\noindent\textbf{If \textsc{OFF}:} Play is unchanged.

\medskip
\noindent\textit{Diagram cue (optional).} Shade the 2\(\times\)2 center; place a small numeral bead beside any Blue that ends there to show its current count (1–3).

\subsection{Scenario 3 — Veil of Names}
\label{scen:veil-of-names}
\phantomsection

\noindent\textbf{Hook.} The cult forbids doctrine-naming: “Names bind exits.” Play as if identity were a mask you choose not to lift.

\medskip
\noindent\textbf{Switch.} \textsc{ON / OFF} (default \textsc{OFF}) \hfill \textbf{Scope.} Scenario-scoped; never alters tournament core.

\medskip
\noindent\textbf{Components.} Optional: two blank cards or slips for private notes (\emph{face-down school cards}).

\medskip
\noindent\textbf{Setup.} If desired, each player writes a single word or icon on a slip (their “school” or guiding maxim) and places it face-down near their board edge. Do not reveal it during play.

\medskip
\noindent\textbf{Rite (exact).} While this Rite is \textsc{ON}, \textbf{players do not declare schools or doctrines publicly}. Any thematic school choice is kept private and has \textbf{no mechanical effect}. All rules, moves, costs, specials, scoring, and timers proceed exactly as normal.

\medskip
\noindent\textbf{Clarifications.}
\begin{itemize}\setlength\itemsep{0.25em}
  \item This Rite introduces \emph{secrecy of theme only}. It does not grant, suppress, or modify any ability.
  \item If a format or local variant requires a public pre-declaration, \emph{ignore this Rite} for that session.
  \item Table talk and reads are allowed as normal; players may infer style but must not demand disclosure.
  \item At the end of the game, players may reveal their face-down slips for color; revealing is optional and confers nothing.
\end{itemize}

\medskip
\noindent\textbf{Example (table texture).}  
Two players sit without naming doctrine. One leans into cautious exits; the other pressures lanes early. Each is reading the other’s \emph{play}, not their banner—tension without rule-changes.

\medskip
\noindent\textbf{Clue beat.} Two training slates in the dossier mark evaluations that match no known doctrine—scholars nickname it the \emph{Gray Ledger}, which “prices exits rather than blocking them.”

\medskip
\noindent\textbf{Why it’s safe.} Pure information fog: no move generation, costs, captures, rooting, specials, or scoring are changed.

\medskip
\noindent\textbf{If \textsc{OFF}:} Play is unchanged.

\medskip
\noindent\textit{Diagram cue (optional).} Small icon of a face-down card with a crossed speech bubble; caption: “Theme kept private; rules unchanged.”

\subsection{Scenario 4 — Candle Count}
\label{scen:candle-count}
\phantomsection

\noindent\textbf{Hook.} A chant sheet with five circles; bead marks that read like a ledger. Make tempo visible without changing a single rule.

\medskip
\noindent\textbf{Switch.} \textsc{ON / OFF} (default \textsc{OFF}) \hfill \textbf{Scope.} Scenario-scoped; never alters tournament core.

\medskip
\noindent\textbf{Components.} A five-step track and \textbf{5 beads} (coins, pebbles, or counters). Optional: a divider to separate \emph{unlit} and \emph{lit} sides.

\medskip
\noindent\textbf{Setup.} Place the five beads on the \emph{unlit} side of the track within view of both players.

\medskip
\noindent\textbf{Rite (exact).} While this Rite is \textsc{ON}, slide \textbf{one bead} from \emph{unlit} to \emph{lit} each time any of the following occurs:
\begin{enumerate}\setlength\itemsep{0.2em}
  \item A piece \textbf{ends} its turn in the \textbf{Cross}.
  \item A \textbf{Blue} becomes \textbf{Rooted}.
  \item Any \textbf{special} is used.
\end{enumerate}
When all five beads are lit, announce “\textit{chant complete},” then \textbf{reset} all five beads to the unlit side. No other rules change.

\medskip
\noindent\textbf{Clarifications.}
\begin{itemize}\setlength\itemsep{0.25em}
  \item If multiple triggers occur during a single turn (e.g., a special is used and a piece ends in the Cross), slide one bead per trigger, up to the remaining unlit beads.
  \item If the fifth bead lights mid-turn, announce completion immediately, reset the track, and continue play; additional triggers that same turn begin filling the new cycle.
  \item The track is public information; either player may slide beads when a trigger occurs (good manners: announce aloud).
  \item This Rite does \emph{not} add time, costs, or constraints; it only records notable moments.
\end{itemize}

\medskip
\noindent\textbf{Example (one cycle).}  
Turn A: a special is used (\(+1\)). Turn B: no triggers. Turn A: a Blue is Rooted (\(+1\)). Turn B: a piece ends in the Cross (\(+1\)). Turn A: special (\(+1\)). Turn B: Cross end again (\(+1\)) \(\Rightarrow\) fifth bead lights, chant complete; reset. Play continues unchanged.

\medskip
\noindent\textbf{Clue beat.} Each \emph{chant complete} marks a base-5 digit on the dossier’s brass strip; across scenarios, digits map to letters that assemble the final \emph{Wound Name} (see Part III).

\medskip
\noindent\textbf{Why it’s safe.} Pure information surfacing: no movement, capture, rooting, specials, scoring, or timers are altered.

\medskip
\noindent\textbf{If \textsc{OFF}:} Play is unchanged.

\medskip
\noindent\textit{Diagram cue (optional).} A five-circle track with three lit markers \((\bullet\bullet\bullet\circ\circ)\) and icons beside the triggers: \([{\small ✚}\ \text{Cross}]\), \([{\small \raisebox{0.15em}{\(\,\bot\,\)}}\ \text{Rooted}]\), \([{\small S}\ \text{Special}]\).

\subsection{Scenario 5 — Copper \& Salt}
\label{scen:copper-and-salt}
\phantomsection

\noindent\textbf{Hook.} House of Wells ledgers mirror temple tithes; little chits move where coins should be.

\medskip
\noindent\textbf{Switch.} \textsc{ON / OFF} (default \textsc{OFF}) \hfill \textbf{Scope.} Scenario-scoped; never alters tournament core.

\medskip
\noindent\textbf{Components.} \textbf{3 Copper} tokens per player; \emph{Patrol} marker (a single coin or chit); optional \emph{Cleared} marker (for Salt-band gaps).

\medskip
\noindent\textbf{Setup.} Give each player a personal bank with a \textbf{capacity of 3} Copper. Each player \textbf{starts with 0} Copper (unless another Rite says otherwise; e.g., \S\ref{scen:ash-fenn-rite} \emph{Oath of Copper} starts with \(+2\)). Place the Patrol and Cleared markers within reach.

\medskip
\noindent\textbf{Rite (exact).} While this Rite is \textsc{ON}:
\begin{enumerate}\setlength\itemsep{0.2em}
  \item \textbf{Gain Copper.} When you \textbf{Seed} or establish a legal \textbf{banner lane}, gain \(\,+1\) Copper (to a maximum of 3). Excess is lost.
  \item \textbf{Spend Copper (once, before your turn begins).} You may spend \(\,1\) Copper to do \emph{one} of the following:
  \begin{enumerate}\setlength\itemsep{0.2em}
    \item \textbf{Purify (Salt).} If \emph{Salt Stitch} (\S\ref{scen:salt-stitch}) is \textsc{ON}, choose any \textbf{perimeter} square and mark it \emph{Cleared}. For the remainder of the game, Greens treat that square as normal (they may \emph{enter/cross/end} there despite the salt-band). 
    \item \textbf{Peek (Leaf).} Draw/peek the next \emph{leaf} or clue packet for this scenario’s dossier thread (if in use), then return/resolve as instructed. (If no clue packet is being used, ignore this option.)
    \item \textbf{Patrol (Perimeter).} Place a \emph{Patrol} marker on a \textbf{perimeter} square. \textit{Until the start of your next turn}, your opponent may \textbf{not end} a move on that marked square. (Moving \emph{through} is allowed.) Remove the marker at the start of your next turn.
  \end{enumerate}
\end{enumerate}

\medskip
\noindent\textbf{Clarifications.}
\begin{itemize}\setlength\itemsep{0.25em}
  \item Copper tokens have no value toward victory; they are a light economy used only for the three options above.
  \item You may gain at most \(+2\) Copper on a single turn if you both \emph{Seed} and complete a \emph{banner lane}, respecting the cap of 3.
  \item \textit{Purify} is unavailable if \emph{Salt Stitch} is \textsc{OFF}. Each \emph{Cleared} square is permanent for the rest of the game.
  \item \textit{Patrol} denies only \emph{ending} a move on that square and only to your opponent; it expires at the start of \emph{your} next turn.
  \item Spending occurs \emph{before} your turn begins; you may not chain multiple spends in the same turn (choose one option per turn at most).
\end{itemize}

\medskip
\noindent\textbf{Example (one cycle).}  
You \emph{Seed} this turn (\(+1\) Copper; total now 1). Before your next turn, you spend that 1 Copper on \emph{Purify}, marking a rim square \emph{Cleared}. On the following turn you establish a \emph{banner lane} (\(+1\); total 1) and spend it \emph{before} your next turn on \emph{Patrol}, denying an enemy end on a critical corner until your turn returns.

\medskip
\noindent\textbf{Clue beat.} A tithe ledger phrases the \emph{Reforge Compact} like a hostage-exchange clause; the \emph{Peek} option reveals a marginal annotation pointing toward the finale.

\medskip
\noindent\textbf{Why it’s safe.} Off-board tokens and markers affect only \emph{information and tempo nudges}; no movement, capture, costs, rooting, specials, scoring, or timers are altered.

\medskip
\noindent\textbf{If \textsc{OFF}:} Play is unchanged.

\medskip
\noindent\textit{Diagram cue (optional).} Three coin icons near each player; a \emph{Cleared} dot on one rim square; a small shield icon (\(\small\shield\)) on a patrolled rim square (captioned “opponent may not \emph{end} here until your next turn”).

\subsection{Scenario 6 — Ash–Fenn Rite}
\label{scen:ash-fenn-rite}
\phantomsection

\noindent\textbf{Hook.} The vault burned; players bind themselves to an opening penance—trading early ease for later relief.

\medskip
\noindent\textbf{Switch.} \textsc{ON / OFF} (default \textsc{OFF}) \hfill \textbf{Scope.} Scenario-scoped; never alters tournament core.

\medskip
\noindent\textbf{Components.} Optional: three \emph{Oath} cards per player (\textit{Water}, \textit{Salt}, \textit{Copper}); one small marker labeled \emph{Bound}.

\medskip
\noindent\textbf{Setup.} Each player \textbf{chooses exactly one} Oath secretly, then reveals simultaneously. Both players may choose the same Oath.

\medskip
\noindent\textbf{Rite (exact).} While this Rite is \textsc{ON}, each player gains the text of their chosen Oath:

\begin{description}\setlength\itemsep{0.35em}
  \item[\textit{Oath of Water.}] The \textbf{first time this game you would move a Blue}, you instead \textbf{forgo that move} (treat it as a voluntary pass for that Blue only; no other effects). \emph{Once per game}, you may have one of your Blues \textbf{step from a sanctum into the Cross even if it is Rooted}. This step must otherwise be legal (ignores only the Rooted restriction).

  \item[\textit{Oath of Salt.}] If \emph{Salt Stitch} (\S\ref{scen:salt-stitch}) is \textsc{ON}, you begin under the salt band as normal, but \emph{once per game} you may \textbf{ignore the salt-band for a single Blue move} (that Blue may \emph{enter/cross/end} on perimeter squares for that move only). \emph{If Salt Stitch is \textsc{OFF}}, this Oath has no effect (choose another Oath during setup if desired).

  \item[\textit{Oath of Copper.}] Start the game with \textbf{+2 Copper} (see \S\ref{scen:copper-and-salt}; your bank cap of 3 still applies). The \textbf{first time you Seed} this game, place a \emph{Bound} marker on that Seed’s target and \textbf{delay all effects of that Seed} (including any rooting or derived effects) \textbf{until the end of your \emph{next} turn}. Remove the \emph{Bound} marker when it resolves. If the target becomes illegal before resolution, the Seed \emph{fizzles} (remove it with no effect).
\end{description}

\medskip
\noindent\textbf{Clarifications.}
\begin{itemize}\setlength\itemsep{0.25em}
  \item \textit{Oath of Water}: “first time you would move a Blue” refers to the first Blue move you attempt in the game, regardless of turn number; you cannot “skip” triggering by selecting a different piece—trigger occurs at the first legal Blue move you choose.
  \item \textit{Oath of Salt}: the one-move waiver applies to a \emph{single Blue move} only and does not persist; other Blues and later moves remain bound by the salt-band.
  \item \textit{Oath of Copper}: you cannot exceed a personal Copper capacity of 3; gaining beyond the cap is lost. The delayed Seed has \emph{no interim effects} until it resolves.
  \item You may never benefit from more than one Oath; the unchosen Oaths confer nothing.
\end{itemize}

\medskip
\noindent\textbf{Example (opening stitches).}  
Player A takes \textit{Water}: on their first attempted Blue move they forgo it, later using the once-per-game sanctum\(\to\)Cross step to break a midgame bind.  
Player B takes \textit{Copper}: starts at 2 Copper (\( \to \) cap 3 on early gains) and declares a Seed; it sits \emph{Bound} until the end of their next turn, then roots normally—buying tempo now, paying certainty later.

\medskip
\noindent\textbf{Clue beat.} A scraped page from the \emph{Concordance}—the redacted \emph{Twin–Oases}—links penance to crossing; a marginal hand hints at a final word reserved for the closing scenario.

\medskip
\noindent\textbf{Why it’s safe.} One-time commitments and permissions that \emph{gate timing only}; core movement, captures, costs, specials, scoring, and timers remain pristine.

\medskip
\noindent\textbf{If \textsc{OFF}:} Play is unchanged.

\medskip
\noindent\textit{Diagram cue (optional).} Three small oath icons at player edge (\(\lozenge\) Water, \(\triangle\) Salt, \(\circ\) Copper); a \emph{Bound} marker on a pending Seed; a single arrow from sanctum to Cross labeled “once”.

\subsection{Scenario 7 — The Ninth}
\label{scen:the-ninth}
\phantomsection

\noindent\textbf{Hook.} “Count eight before breath; the ninth is a wound.” The dossier insists there is a word you do not say until the crossing is priced.

\medskip
\noindent\textbf{Switch.} \textsc{ON / OFF} (default \textsc{OFF}) \hfill \textbf{Scope.} Scenario-scoped; never alters tournament core.

\medskip
\noindent\textbf{Components.} \emph{Wound Name} card (assembled from prior letters) or a sealed Name from the appendix; optional \emph{Crack} marker for one perimeter arc.

\medskip
\noindent\textbf{Setup.}
\begin{enumerate}\setlength\itemsep{0.2em}
  \item \textbf{Campaign mode.} Using letters decoded across earlier scenarios (e.g., \S\ref{scen:candle-count}), assemble the \emph{Wound Name}. Write it on a slip and place it \textbf{face-down} near the board.
  \item \textbf{One–shot mode.} Draw a sealed \emph{Wound Name} from the appendix and place it \textbf{face-down}.
  \item \textbf{Optional bite.} Mark a single perimeter \emph{arc} as \emph{cracked}. Pieces may \emph{pass through} that arc as normal but may \textbf{not end} on its two corner squares \textit{until} the Wound is named.
\end{enumerate}

\medskip
\noindent\textbf{Rite (exact).} While this Rite is \textsc{ON}:
\begin{enumerate}\setlength\itemsep{0.2em}
  \item On your turn, \textbf{after} completing a legal move, you may \textbf{speak} a single candidate Wound Name aloud (at most once per turn).
  \item If the spoken Name \textbf{matches} the face-down Wound Name, immediately gain a \textbf{one–turn Rite–Break}: \emph{choose any one active Rite} and \textbf{suspend its rules for you only} for the remainder of your current turn. Then reveal the face-down Name and read the final leaf.
  \item If the spoken Name \textbf{does not match}, nothing happens; you may try again on a later turn.
\end{enumerate}

\medskip
\noindent\textbf{Clarifications.}
\begin{itemize}\setlength\itemsep{0.25em}
  \item \textit{Rite–Break} affects \emph{only} Rites that are currently \textsc{ON} (e.g., ignore the salt-band once from \S\ref{scen:salt-stitch}; nullify the center stay-timer from \S\ref{scen:witness-ford}; take an immediate Peek from \S\ref{scen:copper-and-salt} without spending Copper). It never alters core movement, capture, costs, or timers.
  \item If \emph{no} other Rite is \textsc{ON}, naming the Wound has \textbf{no mechanical effect}; reveal the Name and proceed (narrative only).
  \item \textit{Optional bite} (\emph{cracked arc}) is a cosmetic constraint you may omit for a purer board. If used, the restriction on ending on the two corner squares is lifted as soon as the Wound is named.
  \item Verification: if dispute arises, consult the sealed appendix solution for the Name chosen at setup.
\end{itemize}

\medskip
\noindent\textbf{Example (resolution).}  
Midgame with \emph{Salt Stitch} and \emph{Witness at the Ford} both \textsc{ON}, a player completes a legal move, speaks the correct Name, then chooses to suspend \emph{Witness at the Ford} for this turn only—allowing a fourth consecutive end in the center to set up a capture. The Name is revealed; play continues under normal Rite effects from the next turn onward.

\medskip
\noindent\textbf{Clue beat.} The final letters from \S\ref{scen:candle-count} completions plus two marginal initials hidden in the dossier produce the Name; the face-down slip confirms it.

\medskip
\noindent\textbf{Why it’s safe.} The transient power touches \emph{Rites only}; core legality, win conditions, and scoring remain pristine. The \emph{cracked arc} is optional and easily omitted.

\medskip
\noindent\textbf{If \textsc{OFF}:} Play is unchanged.

\medskip
\noindent\textit{Diagram cue (optional).} A small face-down card icon labeled “Wound Name”; a dashed highlight over one perimeter arc (\emph{cracked}) and a tiny toggle symbol to illustrate a one–turn suspension of a Rite.

\clearpage


\addcontentsline{toc}{section}{Part III — Puzzles \& Scholar’s Path}

\section{Puzzles \& Scholar’s Path}
\label{part:puzzles}
\phantomsection

\begin{quote}\small
“Do not be clever first. Be clear first. Then, if the page still asks, be patient.”\\
\hfill — \textit{Aqyl}, to a junior reader
\end{quote}

\noindent\textbf{What this part is.} A set of diegetic puzzles woven from the dossier’s artifacts—maps, stamps, bead-tracks, marginalia. Each puzzle is self-contained and \emph{optional}. Solving changes what you notice, not what you are allowed to do.

\medskip
\noindent\textbf{What it is not.} There are no secret rules hidden here. Tournament play remains untouched whether you solve everything or nothing.

\medskip
\noindent\textbf{Fairness \& construction.}
\begin{itemize}\setlength\itemsep{0.3em}
  \item \emph{In-book solvable.} Every puzzle can be solved using materials in this volume (no outside lore required).
  \item \emph{Cross-checkable.} Each has an unambiguous solution in the sealed appendix.
  \item \emph{Diegetic cues.} All transformations (rotations, counts, overlays) are motivated by in-world practices—toll stamps, chant cadence, survey drift, etc.
  \item \emph{Accessibility.} No puzzle relies solely on color; shapes, symbols, or counts provide redundant signals.
\end{itemize}

\medskip
\noindent\textbf{How to use hints.} A four-rung ladder accompanies each puzzle:
\begin{enumerate}\setlength\itemsep{0.25em}
  \item \textbf{Nudge 1} — a gentle reframing (no new facts).
  \item \textbf{Nudge 2} — points at the relevant artifact features.
  \item \textbf{Nudge 3} — states the key operation (e.g., “rotate 17°”).
  \item \textbf{Reveal} — the full solution (in the sealed appendix).
\end{enumerate}
Hint codes appear as \textsc{[H1] [H2] [H3] [R]} with page references.

\medskip
\noindent\textbf{Scholar’s Path.} A suggested route threads the puzzles in a rising curve of difficulty, pairing each with a scenario where its discovery \emph{feels} best at the table. You can follow the path front-to-back, pick selectively, or read everything as lore.

\medskip
\noindent\textbf{Encodings you will meet.}
\begin{itemize}\setlength\itemsep{0.3em}
  \item \textbf{Thepyrgosi Cadence} — public events (Cross ends, Rooted Blues, specials) produce a five-bead rhythm; cycles map to base–5 digits (\S\ref{scen:candle-count}). The digit→letter table lives in the appendix.
  \item \textbf{Map Overlays} — city grids pin-holed to rotate to \(\,17^\circ\) (sanctums↔gates) and occasionally reflected against toll-river axes.
  \item \textbf{Ledger Substitutions} — tally marks, seal rosettes, and triangle stamps encode positions via simple substitution; every symbol is introduced before it’s required.
  \item \textbf{Veils \& Blanks} — deliberate omissions (first-line left empty, unnamed doctrines) are part of the signal; treat silence as a count.
\end{itemize}

\medskip
\noindent\textbf{Solo \& group use.}
\begin{itemize}\setlength{\itemsep}{0.3em}
  \item \emph{Solo dossier.} Work puzzles between scenarios; check \textsc{[H1]} only after a timed attempt (10–15 minutes), then proceed rung by rung as needed.
  \item \emph{Table assist.} Use only the \emph{table-facing rules line} for each Rite (bold header + one-line mechanic); ignore the dossier prose while clocks run. Announce toggles pre-game and use minimal aids (beads/ward markers).
\end{itemize}
 
\subsection{Puzzle 1 — The Candle Ledger (Primer in Quinary)}
\label{pz:candle-ledger}
\phantomsection

\noindent\textbf{What you need.} The five–bead track from \S\ref{scen:candle-count} and this page.

\medskip
\noindent\textbf{Premise.} In \emph{Candle Count}, each \emph{chant complete} consists of \textbf{five triggers}. For this primer, \textbf{encode} each trigger as a single base–5 digit using the etched weights on the brass strip:

\begin{center}
\begin{tabular}{rl}
\([{\small ✚}\ \text{Cross end}]\) & \(\rightarrow\) digit \(\mathbf{2}\) \\
\([{\small \raisebox{0.15em}{\(\,\bot\,\)}}\ \text{Blue becomes Rooted}]\) & \(\rightarrow\) digit \(\mathbf{1}\) \\
\([{\small S}\ \text{Special used}]\) & \(\rightarrow\) digit \(\mathbf{3}\) \\
\end{tabular}
\end{center}

Write the \textbf{five digits in order} as the events occur during a chant. That 5-digit base–5 “\emph{chant code}” later maps to a letter via the table in the \emph{Sealed Appendix}.

\medskip
\noindent\textbf{Goal.} Convert the three logged chants below into their \textbf{5-digit base–5 codes}. Leave the letter blanks empty for now; you will translate them when you reach the appendix.

\medskip
\noindent\textbf{Logged chants (from a clerk’s slate).}
\begin{enumerate}\setlength\itemsep{0.25em}
  \item \([{\small ✚}]\;[{\small \(\,\bot\,\)}]\;[{\small S}]\;[{\small ✚}]\;[{\small ✚}]\) \hfill code: \(\_\ \_\ \_\ \_\ \_\) \hfill letter: \(\_\)
  \item \([{\small S}]\;[{\small ✚}]\;[{\small \(\,\bot\,\)}]\;[{\small S}]\;[{\small \(\,\bot\,\)}]\) \hfill code: \(\_\ \_\ \_\ \_\ \_\) \hfill letter: \(\_\)
  \item \([{\small ✚}]\;[{\small ✚}]\;[{\small S}]\;[{\small \(\,\bot\,\)}]\;[{\small S}]\) \hfill code: \(\_\ \_\ \_\ \_\ \_\) \hfill letter: \(\_\)
\end{enumerate}

\medskip
\noindent\textbf{Worked micro–example (not part of the three).}  
Sequence: \([{\small ✚}]\;[{\small S}]\;[{\small \(\,\bot\,\)}]\;[{\small ✚}]\;[{\small S}]\)  
Digits: \(2\ 3\ 1\ 2\ 3\) \(\Rightarrow\) chant code \(23123_{(5)}\) \(\Rightarrow\) \emph{translate using the appendix’s chant\(\to\)letter table.}

\medskip
\noindent\textbf{Logging grid.}
\begin{center}
\begin{tabular}{c|c|c}
\textbf{Line} & \textbf{5-digit chant code (base–5)} & \textbf{Letter (appendix)} \\
\hline
A & \hspace{3.5cm} & \hspace{1.2cm} \\
B & \hspace{3.5cm} & \hspace{1.2cm} \\
C & \hspace{3.5cm} & \hspace{1.2cm} \\
\end{tabular}
\end{center}

\medskip
\noindent\textbf{Fairness.} Every symbol used above appears earlier in Part II. No external knowledge is required. The digit\(\to\)letter mapping is published in the \emph{Sealed Appendix} to preserve discovery order.

\medskip
\noindent\textbf{Hint ladder.} (Reveal codes only as needed.)
\begin{itemize}\setlength\itemsep{0.25em}
  \item \textsc{[H1]} Count \emph{in order}. Each chant is exactly five events; write five digits left-to-right as they occur.
  \item \textsc{[H2]} Use the brass–strip weights: Cross = 2, Rooted = 1, Special = 3. No other values appear in this primer.
  \item \textsc{[H3]} Keep the base: treat the five digits as a base–5 code, not five separate numbers in base–10. The translation to letters lives in Appendix A.
  \item \textsc{[R]} Compare your three codes to the chant\(\to\)letter table in the \emph{Sealed Appendix}. The three letters form a word you would expect to meet near water.
\end{itemize}

\subsection{Puzzle 2 — The Map at Seventeen (Gate Overlay)}
\label{pz:map-at-seventeen}
\phantomsection

\noindent\textbf{What you need.} The city map leaf with \emph{pinholes}, a board/grid, and this page.

\medskip
\noindent\textbf{Premise.} The dossier claims the city plan \emph{overlays} the board when rotated \(\mathbf{17^\circ}\) (sanctums align with gatehouses). Five small \emph{seal marks} on the map’s rim are intended to land on five specific perimeter squares after the rotation. Each seal mark encodes a digit (base–5); read the five digits in order (clockwise) to recover a single letter.

\medskip
\noindent\textbf{Legend (ledger substitutions for this page).}
\begin{center}
\begin{tabular}{c|c}
\textbf{Seal symbol} & \textbf{Digit (base–5)} \\
\hline
\(\triangle\) (triangle stamp) & \(\mathbf{0}\) \\
\(\large\ast\) (rosette seal)\(^\dagger\) & \(\mathbf{1}\) \\
\(\Vert\) (double–bar tally) & \(\mathbf{2}\) \\
\(\bullet\) (ink dot) & \(\mathbf{3}\) \\
\(\square\) (blank/box placeholder) & \(\mathbf{4}\) \\
\end{tabular}
\end{center}
\noindent\(^\dagger\) Any small flowered stamp counts as “rosette.”

\medskip
\noindent\textbf{Goal.} Produce a \textbf{5–digit base–5 code} by locating the five seals in \emph{clockwise order} (starting from the \textbf{northernmost}), then convert that 5–digit code to a \textbf{single letter} using the table in the \emph{Sealed Appendix}.

\medskip
\noindent\textbf{Procedure.}
\begin{enumerate}\setlength\itemsep{0.2em}
  \item Place the map leaf over the board and rotate it \(\mathbf{17^\circ}\) \textbf{clockwise}. Pinholes should align so that four gatehouses sit on sanctums.
  \item Find the \textbf{five} tiny seal marks printed near the map rim (they’ll now sit over five perimeter squares).
  \item Starting from the \textbf{northernmost} of the five, read them in \textbf{clockwise} order, converting each to a digit using the Legend.
  \item Write the resulting \textbf{5–digit base–5} code. Translate to a \textbf{letter} via the appendix’s quinary chant/code \(\rightarrow\) letter table.
\end{enumerate}

\medskip
\noindent\textbf{Worked micro–example (illustrative only).}  
Suppose your five seals (clockwise from north) are: \(\triangle,\ \bullet,\ \Vert,\ \large\ast,\ \square\)  
Digits: \(0\ 3\ 2\ 1\ 4\) \(\Rightarrow\) code \(03214_{(5)}\) \(\Rightarrow\) look up \(03214\) in the appendix to obtain a letter.

\medskip
\noindent\textbf{Logging box.}
\begin{center}
\begin{tabular}{c|c|c}
\textbf{Order (N\(\rightarrow\)CW)} & \textbf{5–digit code (base–5)} & \textbf{Letter (appendix)} \\
\hline
\(\#1\) \(\#2\) \(\#3\) \(\#4\) \(\#5\) & \hspace{3.5cm} & \hspace{1.2cm} \\
\end{tabular}
\end{center}

\medskip
\noindent\textbf{Fairness.} All five symbols appear elsewhere in the dossier; the \(17^\circ\) rotation is explicitly clued in Part II; no outside maps or lore needed.

\medskip
\noindent\textbf{Hint ladder.}
\begin{itemize}\setlength\itemsep{0.25em}
  \item \textsc{[H1]} Align sanctums to gatehouses by rotating the map \(\,17^\circ\,\) clockwise; don’t worry about exact millimeters—line up the \emph{idea}.
  \item \textsc{[H2]} Start at the \emph{northernmost} seal and go clockwise; if two are tied for “north,” choose the one slightly east.
  \item \textsc{[H3]} Use the Legend on this page; do \emph{not} invent extra symbols. Treat the result as a single 5–digit base–5 number.
  \item \textsc{[R]} The code resolves to a letter commonly associated with thresholds in this volume.
\end{itemize}

\subsection{Puzzle 3 — The House of Wells Gutter (Triangle Ledger)}
\label{pz:triangle-ledger}
\phantomsection

\noindent\textbf{What you need.} The toll ledger leaf with the stamped \emph{left gutter} (triangles, rosettes, bars, dots, and blanks) and this page.

\medskip
\noindent\textbf{Premise.} House of Wells clerks mark a five-slot stamp rail in the \emph{left gutter} of certain pages—one tiny seal per slot. The sequence is a substitution in base–5. Read the five stamps \emph{top-to-bottom} to form a 5-digit code, then map that code to a letter (see Sealed Appendix).

\medskip
\noindent\textbf{Legend (reuse from \S\ref{pz:map-at-seventeen}).}
\begin{center}
\begin{tabular}{c|c}
\textbf{Seal symbol} & \textbf{Digit (base–5)} \\
\hline
\(\triangle\) (triangle stamp) & \(\mathbf{0}\) \\
\(\large\ast\) (rosette seal) & \(\mathbf{1}\) \\
\(\Vert\) (double–bar tally) & \(\mathbf{2}\) \\
\(\bullet\) (ink dot) & \(\mathbf{3}\) \\
\(\square\) (blank/box placeholder) & \(\mathbf{4}\) \\
\end{tabular}
\end{center}

\medskip
\noindent\textbf{Goal.} Extract \textbf{three letters} hidden in the ledger by decoding three five-stamp columns in the left gutter (labeled \textsc{A}, \textsc{B}, \textsc{C}):

\begin{enumerate}\setlength\itemsep{0.25em}
  \item For each column \textsc{A}/\textsc{B}/\textsc{C}, read the five tiny seals \emph{top-to-bottom}.
  \item Convert each seal to its digit using the Legend to get a \textbf{5-digit base–5 code}.
  \item Translate each 5-digit code to a \textbf{letter} via the table in the \emph{Sealed Appendix}.
\end{enumerate}

\medskip
\noindent\textbf{Procedure (what to ignore).} Ignore all amounts, dates, and handwritten totals. Only the \emph{narrow stamp rail} at the very left matters. If a slot shows an abraded mark, treat it as the symbol it most closely resembles (the appendix art provides a comparison row).

\medskip
\noindent\textbf{Worked micro–example (illustrative).}  
Suppose gutter \textsc{A} shows (top→bottom): \(\triangle,\ \Vert,\ \bullet,\ \large\ast,\ \square\).  
Digits: \(0\,2\,3\,1\,4\) \(\Rightarrow\) code \(02314_{(5)}\) \(\Rightarrow\) look up in the appendix to get a letter.

\medskip
\noindent\textbf{Logging grid.}
\begin{center}
\begin{tabular}{c|c|c}
\textbf{Gutter} & \textbf{5-digit code (base–5)} & \textbf{Letter (appendix)} \\
\hline
A & \hspace{3.5cm} & \hspace{1.2cm} \\
B & \hspace{3.5cm} & \hspace{1.2cm} \\
C & \hspace{3.5cm} & \hspace{1.2cm} \\
\end{tabular}
\end{center}

\medskip
\noindent\textbf{Fairness.} The five symbols were introduced earlier. No arithmetic beyond substitution is required. Smudged stamps are printed with enough detail to distinguish dot vs. rosette vs. blank.

\medskip
\noindent\textbf{Hint ladder.}
\begin{itemize}\setlength\itemsep{0.25em}
  \item \textsc{[H1]} Ignore the money. Read only the skinny column of \emph{stamps} at the far left.
  \item \textsc{[H2]} Each gutter is \emph{five} slots tall; top-to-bottom equals left-to-right in the code you’ll write.
  \item \textsc{[H3]} Use the same symbol\(\rightarrow\)digit mapping as the Seventeen Map (\S\ref{pz:map-at-seventeen}); treat the result as a single base–5 number.
  \item \textsc{[R]} The three letters, in order, spell something that lives along the edge of things.
\end{itemize}

\subsection{Puzzle 4 — Twin–Oases Mirror (River Fold)}
\label{pz:twin-oases}
\phantomsection

\noindent\textbf{What you need.} The scraped \emph{Concordance} leaf titled “Twin–Oases” (the one with two well marks at opposite edges and a faint river line) and this page.

\medskip
\noindent\textbf{Premise.} The erased “opening” is really an overlay: the leaf’s \emph{river axis} is a fold line. When you mirror the east half to the west (no actual folding required), five small \emph{spill-marks} along the rim land over specific perimeter squares. Each spill-mark contains \textbf{0–4 hatch strokes}; the count is a digit in base–5.

\medskip
\noindent\textbf{Goal.} Read the five mirrored spill-marks in \textbf{clockwise order} beginning at the \textbf{northernmost}, counting the hatch strokes in each (0–4) to form a \textbf{5-digit base–5 code}. Convert that code to a \textbf{single letter} via the table in the \emph{Sealed Appendix}.

\medskip
\noindent\textbf{Procedure.}
\begin{enumerate}\setlength\itemsep{0.2em}
  \item Identify the faint \textbf{river axis} printed across the Twin–Oases leaf (a ruled line with tiny ford dots). Treat it as a \emph{mirror line}.
  \item Mentally reflect the \textbf{east} half of the leaf onto the \textbf{west} (or lightly trace the reflection). After reflection, locate the \textbf{five} small \emph{spill-marks} that now sit over perimeter squares.
  \item For each spill-mark, \textbf{count the short hatch strokes} inside the droplet: \(\,0,1,2,3,4\,\mapsto\,\) the same digit in base–5.
  \item Starting from the \textbf{northernmost} mirrored mark, read them \textbf{clockwise} to get a \textbf{5-digit base–5} code; translate to a letter using the appendix.
\end{enumerate}

\medskip
\noindent\textbf{Legend (for this page).}
\begin{center}
\begin{tabular}{c|c}
\textbf{Spill-mark hatch count} & \textbf{Digit (base–5)} \\
\hline
\(\,\)no hatches & \(\mathbf{0}\) \\
\(/\) one hatch & \(\mathbf{1}\) \\
\(/\!\!/\) two hatches & \(\mathbf{2}\) \\
\(/\!\!/\!\!/\) three hatches & \(\mathbf{3}\) \\
\(/\!\!/\!\!/\!\!/\) four hatches & \(\mathbf{4}\) \\
\end{tabular}
\end{center}

\medskip
\noindent\textbf{Worked micro–example (illustrative only).}  
Mirrored spill-marks (N→clockwise) show: \(\text{no hatches},\ /\!\!/,\ /\!\!/\!\!/\!\!/,\ /,\ /\!\!/\).  
Digits: \(0\ 2\ 4\ 1\ 2\) \(\Rightarrow\) code \(02412_{(5)}\) \(\Rightarrow\) look up in the appendix to obtain a letter.

\medskip
\noindent\textbf{Logging box.}
\begin{center}
\begin{tabular}{c|c|c}
\textbf{Order (N→CW)} & \textbf{5–digit code (base–5)} & \textbf{Letter (appendix)} \\
\hline
\(\#1\) \(\#2\) \(\#3\) \(\#4\) \(\#5\) & \hspace{3.5cm} & \hspace{1.2cm} \\
\end{tabular}
\end{center}

\medskip
\noindent\textbf{Fairness.} The river axis is visibly printed on the Twin–Oases leaf; the five spill-marks are distinct droplets with short internal hatches. No color is required; counts are by shape alone.

\medskip
\noindent\textbf{Hint ladder.}
\begin{itemize}\setlength\itemsep{0.25em}
  \item \textsc{[H1]} The \emph{river axis} is your mirror line; reflect east onto west (not a rotation).
  \item \textsc{[H2]} Start at the northernmost spill-mark after reflection; proceed clockwise.
  \item \textsc{[H3]} Count the tiny short strokes inside each droplet: 0–4; treat the result as a single base–5 number.
  \item \textsc{[R]} Your letter is one repeatedly associated with water and thresholds in this volume.
\end{itemize}

\subsection{Puzzle 5 — The Veil Page (Redactions Count)}
\label{pz:veil-page}
\phantomsection

\noindent\textbf{What you need.} The deposition leaf labeled “\textit{Veil of Names}” (the one with black redaction bars in the text) and this page.

\medskip
\noindent\textbf{Premise.} In sworn statements where names are “veiled,” scribes leave the \emph{first rubric line empty} and then redact \textbf{0–4} names per line in the next five lines. The \textbf{count of bars on each of those five lines} (top to bottom) forms a \textbf{5–digit base–5 code}. That code maps to a single letter in the \emph{Sealed Appendix}.

\medskip
\noindent\textbf{Goal.} For each stanza on the leaf (\textsc{A} and \textsc{B}), \emph{ignore the deliberately blank opening line}, then read the next \textbf{five} lines top–to–bottom, counting black bars on each line (0–4). Write the resulting 5–digit base–5 code and convert it to a letter via the appendix table.

\medskip
\noindent\textbf{Procedure.}
\begin{enumerate}\setlength\itemsep{0.2em}
  \item Locate stanza \textsc{A}. It begins with a visibly \textbf{blank first line} (Veil rule: “leave the first line empty”). \emph{Do not count} that line.
  \item For the \textbf{next five lines}, count the number of black redaction bars on each line (each continuous bar counts as 1, even if long).
  \item Record those five counts as digits \(0\)–\(4\) to form your \textbf{5–digit base–5 code}.
  \item Repeat for stanza \textsc{B}.
  \item Translate each code to a \textbf{letter} using the table in the \emph{Sealed Appendix}.
\end{enumerate}

\medskip
\noindent\textbf{Clarifications.}
\begin{itemize}\setlength\itemsep{0.25em}
  \item Treat a broken bar that is clearly a scribal crack as \emph{one} bar.
  \item Ignore punctuation, marginal notes, or small “x” placeholders; \emph{only} solid black bars count.
  \item Each counted line is guaranteed to have between \(\,0\) and \(4\,\) bars by design—never more.
\end{itemize}

\medskip
\noindent\textbf{Worked micro–example (illustrative).}  
Suppose stanza \textsc{A} shows (after the blank line):  
Line 1: \(\blacksquare\) \quad Line 2: \(\blacksquare\ \blacksquare\) \quad Line 3: (no bars) \quad Line 4: \(\blacksquare\ \blacksquare\ \blacksquare\) \quad Line 5: \(\blacksquare\)  
Counts \(=\ 1,2,0,3,1 \Rightarrow\) code \(12031_{(5)}\) \(\Rightarrow\) look up in the appendix to obtain a letter.

\medskip
\noindent\textbf{Logging grid.}
\begin{center}
\begin{tabular}{c|c|c}
\textbf{Stanza} & \textbf{5–digit code (base–5)} & \textbf{Letter (appendix)} \\
\hline
A & \hspace{3.5cm} & \hspace{1.2cm} \\
B & \hspace{3.5cm} & \hspace{1.2cm} \\
\end{tabular}
\end{center}

\medskip
\noindent\textbf{Fairness.} The “first line empty” cue is explicit on the leaf; bars are high-contrast and countable; no color needed.

\medskip
\noindent\textbf{Hint ladder.}
\begin{itemize}\setlength\itemsep{0.25em}
  \item \textsc{[H1]} The blank opening line is part of the signal—skip it, then count the next five.
  \item \textsc{[H2]} Each continuous black rectangle counts as one, regardless of length.
  \item \textsc{[H3]} Your result is a \emph{single} 5–digit base–5 number per stanza; do not add or average.
  \item \textsc{[R]} The two letters together form a word often left unsaid in the dossier.
\end{itemize}

\clearpage

\addcontentsline{toc}{section}{Part IV — The Sacred Geometry (Theory Sidebars)}}

\section{The Sacred Geometry}
\label{part:geometry}
\phantomsection

\begin{quote}\small
“Keep the sums straight even when the breath runs. What you cannot prove, bracket; what you can, make plain.”\\
\hfill — \textit{Aqyl}, notation lecture (excerpt)
\end{quote}

\noindent\textbf{What this part is.} Short, non-binding essays on measurements, maps, and counting conventions that appear in the dossier. These sidebars are \emph{interpretive lenses}, not rules. Use them to make sense of artifacts or to add table texture; ignore them and tournament play remains pristine.

\medskip
\noindent\textbf{How to read these.}
\begin{itemize}\setlength\itemsep{0.3em}
  \item \textbf{Plain claims first.} Each sidebar begins with a one-sentence claim you can test.
  \item \textbf{Sources.} Citations point to fragments by page/folio; disputed points are labeled \emph{contested}.
  \item \textbf{Use at the table.} A small box suggests a harmless way to \emph{notice} the idea during casual play (never a rule change).
\end{itemize}

\medskip
\noindent\textbf{Contents (sidebars).}
\begin{description}\setlength\itemsep{0.35em}
  \item[The Sacred Geometry I: Irregular First-Era Rings.] Early ring spacing drift and why perimeter “wobble” shows up in field copies.
  \item[The Sacred Geometry II: Survey Drift \& Gate Alignment.] Why the map overlays at \(17^\circ\); tolerances for sanctum↔gate matching.
  \item[The Sacred Geometry III: Breath\(\rightarrow\)Quinary.] From public chant events to base-5 digits; a minimal table and worked line.
  \item[The Sacred Geometry IV: Fords \& Sancta — A Taxonomy.] When is a crossing a “ford” vs. a “stay”? Notes on language slippage.
  \item[The Sacred Geometry V: The Gray Ledger (Unpublished).] A scholarly reconstruction of a doctrine that “prices exits”; status: \emph{contested}.
  \item[The Sacred Geometry VI: Nine Across Traditions.] Theona’s avoidance vs. Dhahara’s “remaining”; cautions on over-mapping theology.
  \item[The Sacred Geometry VII: Ledger Marks \& Substitutions.] Triangle/rosette/bar/dot/blank as stable encodings across Houses.
  \item[The Sacred Geometry VIII: Disputed Fragment — Lexicus.] Provenance and authenticity checks on L-series leaves; Aqyl’s cadence test.
\end{description}

\medskip
\noindent\textit{Editorial note.} These pages are written conservatively. Where a claim touches doctrine, we present multiple readings and let the artifacts speak. Nothing in Part IV alters movement, costs, capture, rooting, specials, scoring, or timers.

\subsection{The Sacred Geometry I — Irregular First–Era Rings}
\label{geometry:irregular-rings}
\phantomsection

\noindent\textbf{Plain claim.} Early field copies show measurable “wobble” in the perimeter ring; this drift is copy–artifact, not doctrine.

\medskip
\noindent\textbf{Notes.} Hand–ruled boards and roadside tracings (cf. \S\ref{frag:l2}, \S\ref{scen:salt-stitch}) produce slight arc eccentricities—typically \(1\!-\!3\%\) of square width. These irregularities correlate with travel wear (salt, damp) rather than with any stated Rite. Lexicus’s “salt–pressed” leaves (\S\ref{frag:l2}) exaggerate the effect where the rim held moisture.

\medskip
\noindent\textbf{Use at the table (optional).} If you’ve printed the perimeter ribbon for \S\ref{scen:salt-stitch}, allow a gentle, cosmetic misalignment rather than forcing a perfect circle. \emph{No rules change.}

\medskip
\noindent\textbf{Sources.} L–series fragments \S\ref{frag:l2}, \S\ref{frag:l12}; map overlay \S\ref{pz:map-at-seventeen}. Status: \emph{uncontroversial}.

\subsection{The Sacred Geometry II — Survey Drift \& Gate Alignment}
\label{geometry:survey-drift}
\phantomsection

\noindent\textbf{Plain claim.} Rotating the Thepyrgos plan by \(\approx 17^\circ\) aligns four historical gatehouses with board \emph{sancta}; treat the angle as accumulated survey drift, not intent.

\medskip
\noindent\textbf{Notes.} The pin–holed city leaf (\S\ref{pz:map-at-seventeen}) yields a best–fit rotation of \(16.8^\circ\) within a \(\pm 2^\circ\) tolerance. The “\(17^\circ\)” cue recurs as marginal instruction (\S\ref{scen:salt-stitch}, \emph{Clue beat}). The alignment disappears under modern map–north assumptions and reappears if one adopts older river–north conventions. No metaphysical claim is required; this is a cartographic convenience that happens to surface pleasing correspondences.

\medskip
\noindent\textbf{Use at the table (optional).} When staging props, allow any rotation in \([15^\circ,19^\circ]\) to “read” visually. Do \emph{not} rotate the actual board or alter legal geometry. \emph{No rules change.}

\medskip
\noindent\textbf{Sources.} Map leaf \S\ref{pz:map-at-seventeen}; \textit{Salt Stitch} \S\ref{scen:salt-stitch}. \textbf{Status:} \emph{supported}.

\subsection{The Sacred Geometry III — Breath $\rightarrow$ Quinary}
\label{geometry:breath-quinary}
\phantomsection

\noindent\textbf{Plain claim.} Public events can be logged as a five–bead cadence that reads as base–5 digits.

\medskip
\noindent\textbf{Minimal mapping (as used in \S\ref{scen:candle-count}).}
\begin{center}
\begin{tabular}{rl}
Cross end & \(\rightarrow 2\) \\
Blue becomes Rooted & \(\rightarrow 1\) \\
Special used & \(\rightarrow 3\)
\end{tabular}
\end{center}
Write the \emph{five} digits of a chant in order to form a quinary code; translate via the sealed table (see \S\ref{pz:candle-ledger}). Other tables exist historically, but this volume fixes the above for consistency.

\medskip
\noindent\textbf{Use at the table (optional).} Announce triggers aloud (“Cross,” “Rooted,” “Special”) while sliding beads. \emph{No rules change.}

\medskip
\noindent\textbf{Sources.} Candle Count \S\ref{scen:candle-count}; Puzzle 1 \S\ref{pz:candle-ledger}. Status: \emph{canonical within this volume}.


\subsection{The Sacred Geometry IV — Fords \& Sancta: A Taxonomy}
\label{geometry:ford-vs-sanctum}
\phantomsection

\noindent\textbf{Plain claim.} A \emph{ford} is a place priced for crossing; a \emph{sanctum} is a place priced for \emph{not} crossing.

\medskip
\noindent\textbf{Notes.} The dossier uses “ford” for central, transitory pressure (\S\ref{frag:l5}, \S\ref{scen:witness-ford}) and “sanctum” for the fixed, often gate–aligned corners (\S\ref{pz:map-at-seventeen}). Heuristic: if a location’s value \emph{increases} the longer one stays, it behaves sanctum–like; if cost escalates when one lingers (e.g., Witness timer), it is ford–like.

\medskip
\noindent\textbf{Use at the table (optional).} Describe your threats using “cross” vs. “keep” language to signal mood. \emph{No rules change.}

\medskip
\noindent\textbf{Sources.} L–series \S\ref{frag:l4}, \S\ref{frag:l5}; Scenario 2 \S\ref{scen:witness-ford}. Status: \emph{interpretive}.

\subsection{The Sacred Geometry V — The Gray Ledger (Unpublished)}
\label{geometry:gray-ledger}
\phantomsection

\noindent\textbf{Plain claim.} A reconstructed doctrine that “prices exits” appears in training slates but nowhere formal; provenance is \emph{contested}.

\medskip
\noindent\textbf{Notes.} Two slates (\S\ref{scen:veil-of-names} \emph{Clue beat}) score lines by deferred exit rather than denial—a style gloss, not a rule. The name “Gray Ledger” is a modern label for these margins. No mechanical text survives beyond suggestive evaluation marks.

\medskip
\noindent\textbf{Use at the table (optional).} In casual play, try phrasing threats as \emph{costs} (“you can exit, but it prices your next”) to shift tempo reading. \emph{No rules change.}

\medskip
\noindent\textbf{Sources.} Veil of Names \S\ref{scen:veil-of-names}; L–series \S\ref{frag:l10}. Status: \emph{contested}.

\subsection{The Sacred Geometry VI — Nine Across Traditions}
\label{geometry:nine}
\phantomsection

\noindent\textbf{Plain claim.} “Nine” functions as \emph{avoidance} in Theona and as \emph{remaining} in Dhahara; neither maps cleanly to rules.

\medskip
\noindent\textbf{Notes.} Theonan avoidance—unnamed ninth cup/rope (\S\ref{frag:l7})—contrasts with Dhaharan practice (\S\ref{epilogue:dhahara-ninth}) where the ninth is the seat left open, nearer a contemplative “saint” than an angel. Lexicus’s later fervor (\S\ref{frag:l13}) should be read as personal heat, not instruction.

\medskip
\noindent\textbf{Use at the table (optional).} If you observe a “ninth,” let it be a \emph{pause} (a breath), not a mechanic. \emph{No rules change.}

\medskip
\noindent\textbf{Sources.} L–series \S\ref{frag:l7}, \S\ref{frag:l13}; Dhaharan leaf \S\ref{epilogue:dhahara-ninth}. Status: \emph{comparative}.

\subsection{The Sacred Geometry VII — Ledger Marks \& Substitutions}
\label{geometry:ledger-marks}
\phantomsection

\noindent\textbf{Plain claim.} Five symbols recur across houses as a stable quinary: \(\triangle\), rosette \(\large\ast\), \(\Vert\), \(\bullet\), \(\square\).

\medskip
\noindent\textbf{Digit mapping (used in this volume).}
\begin{center}
\begin{tabular}{c|c}
\textbf{Symbol} & \textbf{Digit (base–5)} \\
\hline
\(\triangle\) & \(0\) \\
\(\large\ast\) & \(1\) \\
\(\Vert\) & \(2\) \\
\(\bullet\) & \(3\) \\
\(\square\) & \(4\) \\
\end{tabular}
\end{center}
Variant house–hands exist but resolve to the same five classes. Use the legend consistently across puzzles (\S\ref{pz:map-at-seventeen}, \S\ref{pz:triangle-ledger}).

\medskip
\noindent\textbf{Use at the table (optional).} When annotating casual games, you may tag turns with a symbol instead of a word (e.g., \(\bullet\) for “special used”). \emph{No rules change.}

\medskip
\noindent\textbf{Sources.} Puzzles \S\ref{pz:map-at-seventeen}, \S\ref{pz:triangle-ledger}; ledger leaves \S\ref{frag:l3}. Status: \emph{standardized here}.

\subsection{The Sacred Geometry VIII — Disputed Fragment: Lexicus}
\label{geometry:lexicus-disputed}
\phantomsection

\noindent\textbf{Plain claim.} The L–series shows internal consistency, but late leaves (\S\ref{frag:l15}) are smoke–damaged and ideologically hot; provenance is mixed.

\medskip
\noindent\textbf{Notes.} Aqyl’s cadence test (“count aloud; copy plainer later”) supports early L–leaves (\S\ref{frag:l1}, \S\ref{frag:l2}). The ecstatic entries around the Ash–Fenn Rite and “The Ninth” (\S\ref{frag:l14}, \S\ref{frag:l15}) depart from that sobriety, and one day–book is explicitly burned. Hand–comparison suggests the same scribe, overwriting in haste.

\medskip
\noindent\textbf{Use at the table (optional).} Where two fragments disagree on meaning, prefer the cooler hand for framing mood. \emph{No rules change.}

\medskip
\noindent\textbf{Sources.} L–series \S\ref{frag:l1}, \S\ref{frag:l2}, \S\ref{frag:l14}, \S\ref{frag:l15}; Aqyl note \S\ref{part:puzzles}. Status: \emph{disputed, readable}.

\addcontentsline{toc}{section}{Part V — Appendices \& Sealed Solutions}

\begin{quote}\small
“Read what you can; hide what you must. The page keeps the order either way.”\\
\hfill — \textit{Lexicus}, note to self
\end{quote}

\appendix

% Replace/extend Appendix A’s table with iconography
\section{Compatibility Matrix (Rites, Icons \& Notes)}
\label{app:compat}
\phantomsection

\noindent\textit{All Rites default \textsc{OFF}. Turn any \textsc{ON} without touching tournament core. Icons reference the pictograms in \S\ref{app:legend} (Icon \& Encoding key).}

\medskip
\renewcommand{\arraystretch}{1.18}
\begin{tabular}{p{2.2cm} p{3.8cm} p{6.6cm} p{4.2cm}}
\toprule
\textbf{Icon} & \textbf{Rite} & \textbf{Pairs well with} & \textbf{Notes / Cautions}\\
\midrule
\textit{rim band} % (see Appendix C: dashed rim arc)
& \textbf{Salt Stitch} (\S\ref{scen:salt-stitch})
& Candle Count; Copper \& Salt; Ash–Fenn (Oath of Salt); The Ninth
& \emph{Purify} (Copper \& Salt) can open one rim square; Oath of Salt grants a one–move waiver; The Ninth’s Rite–Break can ignore the band for one turn.\\

\textit{center ford} % (2×2 center cue)
& \textbf{Witness at the Ford} (\S\ref{scen:witness-ford})
& Candle Count; The Ninth
& Stay–timer only for Blues in the Central Four; The Ninth can suspend it for one turn.\\

\textit{veil card} % (face–down slip)
& \textbf{Veil of Names} (\S\ref{scen:veil-of-names})
& Any
& Pure information fog; no interactions to manage. Optional face–down slips for flavor.\\

\textit{five beads} % (five–step track)
& \textbf{Candle Count} (\S\ref{scen:candle-count})
& All
& Surfaces cadence; feeds Part III quinary puzzles; no timing/legality changes.\\

\textit{coin + salt} % (copper token \& salt pile)
& \textbf{Copper \& Salt} (\S\ref{scen:copper-and-salt})
& Salt Stitch; Ash–Fenn (Oath of Copper); The Ninth
& \emph{Purify}/\emph{Patrol}/\emph{Peek} add table texture; cap 3 Copper; The Ninth can grant a one–turn free \emph{Peek} or treat a spend as suspended.\\

\textit{oath set} % (Water / Salt / Copper glyphs)
& \textbf{Ash–Fenn Rite} (\S\ref{scen:ash-fenn-rite})
& Salt Stitch; Copper \& Salt
& Choose exactly one Oath. \emph{Oath of Salt} only matters if Salt Stitch is \textsc{ON}; \emph{Oath of Copper} shines with Copper \& Salt \textsc{ON}.\\

\textit{cracked arc} % (dashed rim arc with crack)
& \textbf{The Ninth} (\S\ref{scen:the-ninth})
& Any (best with ≥1 other Rite)
& Naming the Wound grants a one–turn \emph{Rite–Break} (suspend one active Rite for you only). If no other Rite is \textsc{ON}, effect is narrative only. Optional cracked arc is cosmetic.\\
\bottomrule
\end{tabular}

\medskip
\noindent\textit{Icon key reminder (from \S\ref{app:legend}).} \textit{rim band} = dashed rim arc; \textit{center ford} = shaded 2×2 center; \textit{veil card} = face–down card; \textit{five beads} = five-step track; \textit{coin + salt} = copper token with salt mark; \textit{oath set} = three oath glyphs; \textit{cracked arc} = dashed rim arc with break.

\medskip
\noindent\textit{Designer whisper (small).} These pairings change what you notice, not what you’re allowed to do.

\clearpage
\section{Printables \& Table Props}
\label{app:printables}
\phantomsection

\noindent\textbf{Print at 100\% scale.} Cut along hairlines; light gray backs for duplex pages. Substitutions welcome.

\medskip
\begin{tabular}{p{5.2cm} p{5.6cm} p{3.7cm}}
\toprule
\textbf{Prop} & \textbf{Use} & \textbf{Ref}\\
\midrule
Perimeter ribbon (salt–band) & Trace board rim for \emph{Salt Stitch} & \S\ref{scen:salt-stitch}\\
Five–bead track & Public cadence for \emph{Candle Count} & \S\ref{scen:candle-count}\\
Copper tokens (x6) & Light economy for \emph{Copper \& Salt} & \S\ref{scen:copper-and-salt}\\
Patrol/Cleared markers & Rim denial / purified square & \S\ref{scen:copper-and-salt}\\
Oath cards (Water/Salt/Copper) & \emph{Ash–Fenn Rite} pregame stitches & \S\ref{scen:ash-fenn-rite}\\
Map leaf (pinholes) & 17° overlay puzzle & \S\ref{pz:map-at-seventeen}\\
Twin–Oases leaf & River–fold mirror puzzle & \S\ref{pz:twin-oases}\\
Veil deposition leaf & Redactions count puzzle & \S\ref{pz:veil-page}\\
Brass cipher strip & Base–5 reminder (icons→digits) & \S\ref{pz:candle-ledger}\\
\bottomrule
\end{tabular}
\medskip
\noindent\textit{Printer marks: crop marks 3\,mm outside trim, light–gray hairlines for cut/fold, duplex alignment dots (\(\bullet\)) at mid–short edge, registration bars on the long edge; print at 100\% scale (“Actual size”), no fit-to-page.}

\clearpage
\section{Icons \& Encodings (Quick Legend)}
\label{app:legend}
\phantomsection

\begin{tabular}{p{5.5cm} p{8.9cm}}
\toprule
\textbf{Icon / Mark} & \textbf{Meaning}\\
\midrule
\(\triangle,\ \large\ast,\ \Vert,\ \bullet,\ \square\) & Quinary symbols (0–4) for ledger/pin–seal puzzles (\S\ref{geometry:ledger-marks}).\\
\([{\small ✚}]\) & A turn \emph{ends} in the Cross (Candle trigger).\\
\([{\small \raisebox{0.15em}{\(\,\bot\,\)}}]\) & A Blue becomes \emph{Rooted} (Candle trigger).\\
\([\text{S}]\) & A \emph{special} is used (Candle trigger).\\
Dashed rim arc & Optional “cracked” arc for \emph{The Ninth}.\\
\bottomrule
\end{tabular}

\clearpage
\section{ Quinary Code \textrightarrow{} Letter Table (Sealed)}
\label{app:quinary-table}
\phantomsection

\noindent\textit{Seal this page. Open only when verifying solutions or running Scenario 7 one–shot.}

\vspace{0.5em}
\fbox{\parbox{0.96\linewidth}{\small
This page intentionally sealed. Contains the mapping from 5–digit base–5 chant/overlay codes to letters used to assemble the \emph{Wound Name}.}}

\subsection{Puzzle Solutions (Sealed)}
\label{app:solutions}
\phantomsection

\noindent\textit{Each solution is cross–referenced to its puzzle page and includes a diagram. Open only after attempting the hint ladder.}

\vspace{0.5em}
\fbox{\parbox{0.96\linewidth}{\small
This page intentionally sealed. Contains worked solutions for Puzzles \S\ref{pz:candle-ledger}–\S\ref{pz:veil-page} and any later additions.}}

\clearpage
\section{An Aelinnel’s Treatise on Core Mechanics \& Meta}
\label{geometry:aelinnel-core}
\phantomsection

\noindent\textit{Author: Lubick of the Lower Gauge (gnome). Affiliation: none (I like tea, not oaths). Style: fast. Accuracy: faster.\footnote{If a lemma displeases you, put it under a cup and see if the steam moves it. If it moves, keep it.}}

\medskip
\noindent\textbf{Thesis (plain).} The core game is a finite, impartial, perfect–information contest on a planar grid–graph with two piece classes—\textbf{Blue} (mobile, anchorable) and \textbf{Green} (shaping, evasive)—interacting through lanes, corners, and a 2\(\times\)2 \emph{Cross}. All meta patterns reduce to three quantities: \emph{cut thickness}, \emph{tempo parity}, and \emph{conversion efficiency}.

\subsubsection{0. Model \& Notation (no incense, just math)}
Let \(G=(V,E)\) be the board graph (orthogonal adjacency). Corners and near–corners are \emph{sancta} (cf. taxonomy \S\ref{geometry:ford-vs-sanctum}). The \emph{Cross} is the central \(2\times2\) block \(C\subset V\). A \emph{lane} is a simple path \(P\subseteq V\) from a player’s near region to any exit scoring condition (details suppressed; the math survives). 

Positions are tuples \(\mathsf{s}=(B,G,\sigma)\) with Blue set \(B\subseteq V\), Green set \(G\subseteq V\), and status \(\sigma\) (whose bits include which Blues are \emph{Rooted}, specials availability, etc.). A move is a function \(f:\mathsf{s}\mapsto \mathsf{s}'\) respecting core legality. A \emph{plan} is a ply sequence \(p=(f_1,\dots,f_k)\).

\paragraph{Influence fields.} Define a discrete potential for each color:
\[
\Phi_{\text{Blue}}(v)= \alpha\cdot \text{mobility}(v) + \beta\cdot \text{lane\_proximity}(v) - \gamma\cdot \text{fragility}(v),
\]
\[
\Phi_{\text{Green}}(v)= \hat{\alpha}\cdot \text{escape\_routes}(v) + \hat{\beta}\cdot \text{bend\_leverage}(v).
\]
Calibrate \((\alpha,\beta,\gamma)\) by post–game audits (see drills below).

\subsubsection{1. Cuts, Thickness, \& Escapes}
\label{ael-core:cuts}
A \emph{vertex cut} \(S\subset V\) separates your build from its exit. Its \emph{thickness} is \(|S|\), but \emph{practical thickness} is 
\[
\Theta(S)= |S| + \sum_{v\in S}\mathbf{1}_{\text{re-enterable in }\le 2\text{ plies}}.
\]
\begin{lemma}[Corner insurance]
Any cut that includes a true corner (sanctum) has \(\Theta\) at least one higher for your opponent if you hold a nearby pivot two steps deep. (Corners resist single–ply dissolves.)
\end{lemma}

\noindent\textbf{Meta cue.} Strong openings thicken \emph{one} cut and threaten to migrate the fight to a second; weak openings thin \emph{two} cuts and threaten neither.

\subsubsection{2. Tempo \& Parity (Octaves, not dogma)}
\label{ael-core:parity}
Many local races complete in \(8\) plies under orthodox shapes. Call this an \emph{octave loop}. The 9th ply is a \emph{naming ply}: whoever owns it chooses to cash, refuse, or transpose.

\begin{proposition}[Octave advantage]
If a line resolves in exactly \(8\) plies and you moved first in that locale, then unless the opponent altered graph structure (special, root), you can force the naming ply to be \emph{yours}. 
\end{proposition}

\noindent\textbf{Practical rule.} Track local parity per lane: write \(\oplus\) if you own the naming ply, \(\ominus\) if they do. Shift fights to regions where you show \(\oplus\), stall where you show \(\ominus\).

\subsubsection{3. Rooting as Constraint Programming}
\label{ael-core:root}
A Blue marked \emph{Rooted} pins a local constraint. Let \(R\subseteq B\) be rooted Blues. Define \(\mathsf{free\_capacity} = \sum_{b\in B\setminus R}\text{deg}(b)\). 

\begin{lemma}[Anchor tax]
Rooting an enemy Blue reduces their \(\mathsf{free\_capacity}\) by at least its local branching factor minus 1. If the rooted Blue sits in \(C\), the tax is larger by the number of forking lanes through \(C\).
\end{lemma}

\noindent\textbf{Meta cue.} Rooting is strongest where lane density (alternative exits) is high; weakest where you already thinned cuts (you’re taxing a starved graph).

\subsubsection{4. Specials as Local Rewrites}
\label{ael-core:specials}
Treat each special as a bounded rewrite on \(G\) or \(\sigma\). Good doctrine: \emph{Spend specials to break symmetry or to lock an \(\oplus\) locale into payoff.} Bad doctrine: \emph{Spend to look clever on \(\ominus\).} 

\begin{theorem}[Symmetry break maxim]
If a special increases \(\Theta(S)\) in one region while preserving it elsewhere, the unique best use is in the region where you already hold \(\oplus\). (Cash the lead, don’t rent it.)
\end{theorem}

\subsubsection{5. Evaluation Function (you may write this on your cuff)}
Let 
\[
\mathcal{E} = w_1 \cdot \text{Material} + w_2 \cdot \text{Mobility} + w_3 \cdot \text{CutThickness} + w_4 \cdot \text{ParitySum} + w_5 \cdot \text{Conversion},
\]
where
\[
\text{ParitySum}=\sum_{\text{locales }L} \begin{cases}+1 & \oplus(L)\\ -1 & \ominus(L)\end{cases}, \quad
\text{Conversion}=\sum_{\text{threats }T}\Pr[\text{T}\to\text{score in }\le 3].
\]
Normalize \(w_i\) by post–hoc regression on your own games (yes, really). If your \(\mathcal{E}\) says “+1” but your cuts say “hollow,” believe the cuts.

\subsubsection{6. Openings \& Meta Lines (current, sane)}
\begin{description}\setlength\itemsep{0.35em}
  \item[\textit{Cage \& Pivot.}] Thicken one rim cut, plant a midboard pivot that threatens both flanks on ply 6–8. Good into fast–lane rushers.
  \item[\textit{Mirror–Then–Shiver.}] Mirror for four plies, then break symmetry toward your \(\oplus\) locale. Punishes over–mirrors.
  \item[\textit{Ladder–to–Ladder.}] Build a rim ladder that stops one short of clinch; transpose the clinch to the far side when they commit. (Yes, that “one short” is deliberate—see \S\ref{ael-core:parity}.)
\end{description}

\subsubsection{7. Endgame Principles (no candles required)}
\begin{itemize}\setlength\itemsep{0.25em}
  \item \textbf{Two–cut theorem (practical).} In stable ends, whoever can thicken two disjoint cuts by +1 wins more than half of tight games, ceteris paribus.
  \item \textbf{Zugzwang test.} If every legal move in your locale reduces \(\Theta\) by at least 1, abandon the locale; build a fresh \(\oplus\) elsewhere.
  \item \textbf{Corner banking.} Bank tempo by keeping a near–corner unsealed until your cross–threat matures; cash late to avoid giving them a naming ply.
\end{itemize}

\subsubsection{8. Cross Pressure (without taboos)}
\label{ael-core:cross}
Holding the \emph{Cross} is good only if it shortens \emph{your} lanes or lengthens \emph{theirs}. Otherwise, it is a tax upon your own \(\mathsf{free\_capacity}\).

\begin{lemma}[Center discipline]
If occupying \(C\) does not change any shortest–path length by \(\ge 1\) in your favor, vacate within two plies or pivot to a fork. (Write this on your wrist.)
\end{lemma}

\subsubsection{9. A Note on “Remaining” (observational, not doctrinal)}
\label{ael-core:remaining}
Some players leave one step unspent in profitable lines and seem to win more. I observe this \emph{kept step} correlates with owning the naming ply in octave loops (\S\ref{ael-core:parity}). See also cross–tradition notes on nine (\S\ref{geometry:nine}). I am not in anyone’s circle; I simply counted.

\subsubsection{10. Tiny Theorems (carry in pocket)}
\begin{enumerate}\setlength\itemsep{0.25em}
  \item \textbf{Fork Value.} A pivot that threatens two exits with disjoint first steps is worth \(\ge +1\) \(\mathcal{E}\) even before conversion.
  \item \textbf{False Thick.} A cut that is thick only by pieces already overworked is thin in practice: subtract 1 from \(\Theta\).
  \item \textbf{Late Lane Law.} If a lane is equal on ply 4 and still equal on ply 6, the player with more \(\oplus\) locales elsewhere will own its naming ply by inertia. Shift now or donate later.
\end{enumerate}

\subsubsection{11. Drills (do these; your future self will applaud)}
\begin{itemize}\setlength\itemsep{0.25em}
  \item \textbf{Cut Audit.} Pause at plies 4, 8, 12; draw your two thickest cuts and mark their \(\Theta\). If neither improves in two checkpoints, your plan is a poem—not a plan.
  \item \textbf{Parity Log.} Keep a side card of locales with \(\oplus/\ominus\). Aim to maintain \(+\!2\) net by midgame.
  \item \textbf{Root Value Sprints.} Play ten positions where you can root now or improve a cut; record which yields higher conversion in \(\le 3\) plies.
\end{itemize}

\medskip
\noindent\textit{Closing (sip).} The board does not care about your stories; it cares about your cuts, your parity, and whether your threats convert on time. If someone whispers about thresholds and breath, smile, nod, and then check whether your shortest path just got shorter. If yes, take it. If not, have tea and thicken a cut.

\clearpage
\section{New Schools}
\label{app:schools}
\phantomsection

\subsubsection{Theona — “Fairy Stones”}
\label{school:theona}
\phantomsection
\addcontentsline{toc}{subsubsection}{Theona — “Fairy Stones”}

\noindent\textit{Lore preface.} On the eastern isle they leave one seat empty and one turn uncounted. Fishers speak their nets in eights and set aside a ninth rope as ward. They call the game only \emph{Fairy Stones} among strangers, and if pressed for doctrine they change the subject to weather, thresholds, and the duty of not naming a thing before it has earned its name. Their tables are low; the lanterns are small; the salt is put away when guests arrive. They will teach by showing you what they \emph{didn’t} do.

\medskip
\noindent\textbf{Concordance profile.} \emph{A restraint school that prices crossings by omission. Theona wins by offering “almost–exits,” escorting foes through honest lanes, and arriving one breath later with better measure.}

\medskip
\noindent\textbf{Core premise.} \textit{Leave the ninth empty.} In practice: favor sequences that end on the eighth count of a local race, compelling the opponent to “name” the crossing first. Theona converts opponent initiative into priced exits.

\medskip
\noindent\textbf{Table mood.} Quiet pressure; edges kept clean; center touched and released. Announce little; show much.

\medskip
\noindent\textbf{Heuristics (testable).}
\begin{enumerate}\setlength\itemsep{0.25em}
  \item \textbf{Keep one square unclaimed} in any local three–move plan; use it as a pivot rather than a goal.
  \item \textbf{Offer tidy lanes} that look free but tax tempo two moves later.
  \item \textbf{Touch the center, don’t sit in it.} Create work there; cash it on the rim.
  \item \textbf{Finish on “eight.”} If a chase resolves in eight plies, prefer to be the one who \emph{declines} the ninth.
\end{enumerate}

\medskip
\noindent\textbf{Preferred shapes.} Narrow diagonals, rim ladders with a deliberate gap, corner–adjacent sancta that stay \emph{potential} rather than property.

\medskip
\noindent\textbf{Openers (themes, not rules).}
\begin{description}\setlength\itemsep{0.35em}
  \item[\textit{Empty Chair.}] Develop toward two edges while leaving an obvious clinch unplayed; invite the opponent to close it and inherit your prepared counter–lane.
  \item[\textit{Unpoured Cup.}] Advance a file to the brink, then mirror on the far side; cross only after the mirror forces a double–duty defense.
\end{description}

\medskip
\noindent\textbf{Midgame patterns.}
\begin{itemize}\setlength\itemsep{0.25em}
  \item \textbf{Near–Ford Escort.} Guide an opposing advance across a ford you’ve priced; recoup on the flank they abandoned.
  \item \textbf{Eighth–Beat Switch.} Hold a resource until move eight of a local loop, then trade lanes while the opponent “names” the ninth.
\end{itemize}

\medskip
\noindent\textbf{Tactics.}
\begin{itemize}\setlength\itemsep{0.25em}
  \item \textbf{False Sanctum.} Present a corner as ripe; your next pushes reveal it as staging, not settlement.
  \item \textbf{Soft Denial.} Block endings, not paths: let moves pass but forbid \emph{staying} where it profits them most.
\end{itemize}

\medskip
\noindent\textbf{Counters (vs. Theona).}
\begin{itemize}\setlength\itemsep{0.25em}
  \item \textbf{Name first—cleanly.} Take the “ninth” with a line that doesn’t hand them a priced reply.
  \item \textbf{Refuse escorts.} If offered a tidy lane, ask what purchase it buys them on the opposite edge before walking it.
  \item \textbf{Break symmetry early.} Theona thrives on mirrored patience; asymmetry forces them to \emph{name}.
\end{itemize}

\medskip
\noindent\textbf{Drills.}
\begin{enumerate}\setlength\itemsep{0.25em}
  \item \textbf{Eight–Count Scrimmage.} Play blitz micro–games where you must \emph{not} be the player who completes the ninth tempo in any localized race.
  \item \textbf{Empty–Seat Review.} After a game, circle all positions where you could have left one “chair” open; score how often that would have improved your conversion.
\end{enumerate}

\medskip
\noindent\textbf{Use with Rites (optional, cosmetic).} \emph{Veil of Names} (\S\ref{scen:veil-of-names}) complements the style’s quiet; \emph{Candle Count} (\S\ref{scen:candle-count}) makes the “eighth–beat” decisions visible without changing a rule. Leave both \textsc{OFF} for tournament play.

\medskip
\begin{quote}\small
“Leave one cup poured and unclaimed; leave one chair waiting. The board will tell you which guest has arrived.”\\
\hfill — \textit{Theonan fishers’ saying}
\end{quote}

\subsubsection{The Witness — “The Remaining Compact”}
\label{school:witness}
\phantomsection

\noindent\textit{Lore preface.} They say the circle that calls itself \emph{the Witness} keeps three small vows and two large courtesies. The vows are salt, copper, and silence: \emph{bind the edge}, \emph{price the crossing}, \emph{name nothing until it earns the name}. The courtesies are to the eighth and the ninth: \emph{work to the eighth}, then \emph{leave one thing unclosed} so the room can answer. Their numbers are not spells; they are habits with a taste for thresholds.

\medskip
\noindent\textbf{Concordance profile.} \emph{A threshold school that taxes crossings and converts pressure into measured exits. The Witness wins by staging “priced invitations,” sealing on tempo rather than on spectacle.}

\medskip
\noindent\textbf{Core premise.} \textit{Price first, cross later.} Make each opponent advance carry a delayed cost; step through only when your ledger is favorable.

\medskip
\noindent\textbf{Table mood.} Clinical calm. The board feels bound at the rim, breathable in the middle, and suspicious of free lunches.

\medskip
\noindent\textbf{Sacred counts (as heuristics, not rules).}
\begin{description}\setlength\itemsep{0.35em}
  \item[\textbf{3} — Proof.] Before you take a crossing, demand three proofs: \emph{(i)} you gain a lane, \emph{(ii)} you deny an answer, \emph{(iii)} you keep a recourse.
  \item[\textbf{5} — Candles.] Evaluate local fights in five-beat phrases; prefer lines that “light” two of the public triggers while conceding at most one.
  \item[\textbf{8} — Work.] Resolve races by the eighth ply; if resolution slips to the ninth, make it \emph{their} hand that names it.
  \item[\textbf{9} — Remaining.] Leave one square, line, or capture \emph{deliberately unclaimed}. That absence is leverage, not loss.
\end{description}

\medskip
\noindent\textbf{Heuristics (testable).}
\begin{enumerate}\setlength\itemsep{0.25em}
  \item \textbf{Bind the rim first.} Secure a perimeter file that makes enemy edge-walks inefficient; cash in later.
  \item \textbf{Tax the ford.} Whenever the center is touched, ensure a follow-up that turns \emph{their} stay into a tempo leak.
  \item \textbf{Hold a sealed reply.} Enter a line only with a quiet counter in pocket—one that flips valuation without noise.
  \item \textbf{Prefer delayed value.} Choose sequences that look equal on move 4 but show surplus on move 6–8.
\end{enumerate}

\medskip
\noindent\textbf{Preferred shapes.} Quincunx scaffolds; triads that point at a rim crack; ladder lanes that stop one square short (the “kept step”).

\medskip
\noindent\textbf{Openers (themes, not rules).}
\begin{description}\setlength\itemsep{0.35em}
  \item[\textit{Copper Ledger.}] Offer a clean lane; spend tempo elsewhere so that accepting your “gift” walks into a ready denial.
  \item[\textit{Salt Ring.}] Develop two files inward, keeping the outermost squares unattractive; trade edges late when your center refusal has priced their options.
\end{description}

\medskip
\noindent\textbf{Midgame patterns.}
\begin{itemize}\setlength\itemsep{0.25em}
  \item \textbf{Crack \& Seal.} Invite a short-term breach you can close one ply later, turning their entry into anchorage for your exit.
  \item \textbf{Ford Titration.} Touch the center briefly to force an answer, then vacate and score on the flank they under-defended.
\end{itemize}

\medskip
\noindent\textbf{Tactics.}
\begin{itemize}\setlength\itemsep{0.25em}
  \item \textbf{Priced Invitation.} Present an apparently winning capture or lane whose aftermath loses a tempo or shape.
  \item \textbf{Silent Exchange.} Trade equal material for superior turns; prefer quiet gains over flashy nets.
\end{itemize}

\medskip
\noindent\textbf{Counters (vs. the Witness).}
\begin{itemize}\setlength\itemsep{0.25em}
  \item \textbf{Name early.} Commit to a clean, forcing line that denies them the “kept ninth.”
  \item \textbf{Over-pay nothing.} If a lane looks free, count two moves deeper; refuse if the ledger turns.
  \item \textbf{Pull to open water.} Spread the fight; they prefer priced corridors to broad fronts.
\end{itemize}

\medskip
\noindent\textbf{Drills.}
\begin{enumerate}\setlength\itemsep{0.25em}
  \item \textbf{Three-Proof Test.} For ten sample crossings, write the three proofs; only play those that pass all three.
  \item \textbf{Eighth-Beat Audit.} Review finished games; mark races that resolved on move 9+. Reconstruct how to compel resolution on 8.
\end{enumerate}

\medskip
\noindent\textbf{Use with Rites (optional, cosmetic).} \emph{Candle Count} (\S\ref{scen:candle-count}) spotlights your five-beat evaluations; \emph{Salt Stitch} (\S\ref{scen:salt-stitch}) sharpens rim-binding aesthetics. Leave both \textsc{OFF} for tournament integrity.

\medskip
\begin{quote}\small
“Bind the edge, price the water, keep one step unspent. The board remembers who pays and who witnesses.”\\
\hfill — \textit{from a House of Wells catechism, unattributed}
\end{quote}



\section*{Epilogue — A Dhaharan Leaf: On Honoring the Ninth}
\label{epilogue:dhahara-ninth}
\phantomsection
\addcontentsline{toc}{section}{Epilogue — A Dhaharan Leaf: On Honoring the Ninth}

\noindent\textit{Provenance: a short Dhaharan folio, margin–stamped for a winter observance; translated with light normalization. The original script is calm and spacious, with long breaths between clauses.}

\medskip
\begin{quote}\small
We are not as the Ykrul counters nor the Kuvani whisperers who fear the last place and speak around it. We do not goad it nor bargain with it. We make room.

At dusk we sweep the threshold and lay a narrow ring of salt where feet remember to step. We wash the hands in water that has touched a word. We set three lanterns, then three, then three, and leave one unlit. We do not explain ourselves.

We kindle incense until the air remembers a road. We say the small verses that do not ask. We count the breaths as if listening for a guest.

We drink the mild wine and sit with stones and board until the lamps lean. We do not wager coin; we practice keeping. We speak little. When laughter comes, we do not chase it away.

In the deep hour we pour a ninth cup and leave it where morning will find it. This is the seat the Messenger yielded when it stooped to be our measure of play; some call that yielding a fall, others a mercy. We call it \emph{the remaining}.

The ring of salt is not a wall; it is a way to notice the edge. The unlit flame is not refusal; it is a place kept open so breath may turn.

When light returns, we unmake what we made. We scatter a pinch of salt to the four roads. We lift the unlit lantern and let the day choose it. We do not declare that we have learned; we look again, and something that would not come forward before stands a little nearer, like a page held at a kinder angle.

If there is a ninth, it is not a captain and not a judge. It abides. It is the rest between measures, the seat left empty so the room can breathe. Honor it, and let play teach the rest.
\end{quote}

\medskip
\noindent\textit{Translator’s note.} The Dhaharan term rendered “the remaining” admits readings as “what abides” and “what stays available.” The folio hints (without disputation) that the “Messenger” left its place to lie \emph{flat as play}. The leaf commands nothing; it suggests a practice. The board appears as craft, not dogma.

\section*{Epilogue — A Letter from Alayse}
\addcontentsline{toc}{section}{Epilogue — A Letter from Alayse}

\noindent\textbf{From:} Alayse Fenwood, House Fenwood \\
\textbf{To:} Elyas Everblood \\
\textbf{Seal:} common wax

\medskip
\noindent Elyas,

I grow most concerned for Markus of late. I know he loves me and our daughter, yet his soft smile so seldom reaches his eyes. I jest that his years in Thepyrgos scorched his skin and his gaze. I know the truth of it; his Ecktorian blood is truer than his brothers’ and he darkens with age. He laughs and calls me a silly thing\ldots{} and yet.

And yet I find him often in the yard, staring. Our Ykrul champion says he prays to the eight winds, but Markus always faces south, even when the breeze comes down from the north. Perhaps he misses the warm kiss of the southern air. Sometimes he will stand until the eighth bell and only then come in, as if keeping a small appointment with the hush. Perhaps it is only that he worries after Lerris---still enrolled at the University these twelve, nearly thirteen, years. The stacks keep what they keep.

It troubles me---but he is my love; nought shall change that.

Alahana is well. She grows taller day by day and oft asks after her ``Summer Uncle.''

Be well on your travels, little brother. I think of you often.

\medskip
\noindent Alayse, 889~AR

\end{document}
