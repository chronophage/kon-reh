\documentclass[11pt]{article}

% --- Basic, readable preamble -----------------------
\usepackage[T1]{fontenc}
\usepackage[utf8]{inputenc}
\usepackage{microtype}
\usepackage[margin=1in]{geometry}
\usepackage{setspace}
\usepackage{parskip} % blank lines instead of indents
\usepackage{enumitem}
\usepackage{hyperref}
\hypersetup{hidelinks}

\title{The Lore, With Sleeves Rolled Up\\\large A short note for the curious reader of \emph{Kon'reh}}
\author{}
\date{}

\begin{document}
\maketitle
\onehalfspacing

\section*{What You Hold}
This book is not a map; it is a set of fingerprints. The pages that follow were copied, disputed, rescued from cupboards, and sometimes---let us be honest---arranged to look cleaner than they ever were. If the rules are the board's stone, the lore is the dust on your fingers when you lift a piece. You may wash it off and play pristinely. Or you may look at the dust and say: \emph{this is how hands remember}.

\section*{Three Ways Through}
There are three doors into \emph{Kon'reh}'s past.

\begin{enumerate}[leftmargin=*,label=\arabic*)]
  \item \textbf{The Concordance} is the \emph{messy room}: letters, sayings, contradictions. Nothing fits neatly, and that is the point. Truth arrives in splinters.
  \item \textbf{The Corpus} is the \emph{tidy hall}: an editor smooths the splinters into a staff for walking. Use it. Be grateful. Also notice what fell to the floor.
  \item \textbf{The Ninth Rim} is the \emph{back stair}: you return to the room with a lamp at midnight, convinced that one more shelf is hidden in the wall. Sometimes you are wrong. Sometimes you are right. The lamp matters either way.
\end{enumerate}

You can play the game without walking any of these halls. But if you choose to wander, you will discover that each door opens not just onto a history, but onto a way of reading---and, eventually, a way of playing.

\section*{Five Readers (You May Be All of Them)}
\begin{description}[leftmargin=!,labelwidth=2.8cm,style=nextline]
  \item[\textbf{The Archivist}] copies plainly, marks conjecture, keeps both sums when the numbers disagree.\\
  \textit{Motto:} ``Leave the record breathing.''\\
  \textit{Try this:} Before a move, list what you \emph{know} and what you only \emph{believe}.

  \item[\textbf{The Canonizer}] tidies the story so others can use it. A good teacher. A dangerous historian.\\
  \textit{Motto:} ``Make it useful---then ask what the usefulness erased.''\\
  \textit{Try this:} Explain your plan in two sentences. What did you omit?

  \item[\textbf{The Silenced Source}] speaks in other people's footnotes. Her ideas persist; her name thins.\\
  \textit{Motto:} ``Follow the practice to the hand that first wrote it.''\\
  \textit{Try this:} When a phrase on the table helps you, ask: \emph{who paid for this clarity?}

  \item[\textbf{The Commentator}] writes in margins. Counts breaths. Notices how fear edits memory.\\
  \textit{Motto:} ``Observe yourself observing.''\\
  \textit{Try this:} After each turn, record one sentence: \emph{what changed in me, not the board}.

  \item[\textbf{The Heretic}] treats closure as a puzzle-box. Opens it. Finds another box. Smiles.\\
  \textit{Motto:} ``If doctrine forbids a question, the question has already begun to answer.''\\
  \textit{Try this:} Name the line you refuse to see---then look at it for one full minute.
\end{description}

None of these is ``correct.'' They are stances, lenses, postures of attention. Today you may be one; tomorrow, several. The board is constant. You are not.

\section*{What the Lore Admits}
It admits that archives survive by luck and by power. That editors rescue and steal. That we tell stories to make hard things bearable and sharp things memorable. It admits that rules are clean and that living with rules is not. It admits that a perfect game can feel empty if no one speaks about what it meant to try and fail and try again.

Lore changes nothing that is legal. It changes what you \emph{notice}. And noticing is the oldest technology we have.

\section*{The Sacred Geometry (Two Kinds)}
There is the geometry under the pieces: lanes, seals, centers, times. Learn it as you would learn a city---first by getting lost, then by drawing your own small map.

There is also the geometry under the reader: habits of seeing, inherited metaphors, proverbs that press your hand toward one move and away from another. When we say ``sacred,'' we mean only this: a structure that shapes you even when you are not looking directly at it.

\section*{How to Read This, If You Wish}
\begin{enumerate}[leftmargin=*]
  \item \textbf{Begin anywhere.} The order matters less than your attention.
  \item \textbf{Underline your doubts.} Doubt is a highlighter.
  \item \textbf{Keep a ledger of phrases} that make you feel braver or quieter. Use them at the table like tools, not like spells.
  \item \textbf{Revisit your certainties.} If a page used to seem obvious and now feels complicated, congratulations: you have learned to read yourself.
  \item \textbf{Close the book often.} Play. Then come back with new dust on your fingers.
\end{enumerate}

\section*{A Small Confession}
You will find theft here, and curation, and love. You will meet voices that wanted credit and received utility instead. You will meet careful men who made a clean path and left muddy footprints in the margins. You will meet a reader who stayed too long in the stacks and a teacher who left the lamp burning for strangers.

You will also meet yourself---the version who prefers rules, the one who prefers stories, and the quiet third who understands that both are ways to hold the same stone.

\section*{Parting Note}
The meaning of \emph{Kon'reh} is not hidden in a cipher or sealed in an appendix. It is made, each night, by the person who reads and the person who plays. If you need no story, you lose nothing. If you need one, take this book as permission to build yours carefully and to suspect it kindly.

The board is fixed.\\
The move is yours.\\
What it meant---that is the game you carry home.

\end{document}
